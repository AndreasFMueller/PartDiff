%
% nichtlinear.tex
%
% (c) 2008 Prof Dr Andreas Mueller, Hochschule Rapperswil 
% $Id: c07-nichtlinear.tex,v 1.3 2008/10/31 08:04:16 afm Exp $
%
\chapter{Nichtlineare Partielle Differentialgleichungen\label{chapter-nichtlinear}}
\index{Differentialgleichungen!partielle!nichtlineare}
\lhead{Nichtlineare PDGL}
\rhead{}
Die Grundgleichungen der Elektrodynamik und der Elastizitätstheorie
sind linear. Die in den vorangegangenen Kapiteln dargestellten
Lösungensmethoden sind auf derartige Differentialgleichungen
zugeschnitten. Einige bedeutenden Probleme der Naturwissenschaften
führen dagegen auf nichtlineare Gleichungen, für die die
Methoden ungeeignet sind. In diesem Kapitel werden einige
solche Gleichungen vorgeführt und die dabei auftretenden
Schwierigkeiten illustriert.

\section{Bespiele nichtlinearer PDGL}
\rhead{Beispiele}
Wir stellen ein paar Beispiel nichtlinearer Differentialgleichungen
zusammen, die sich in Anwendungsproblemen stellen. Einzelne Beispiele
kamen bereits im ersten Kapitel zur Sprache.

\subsection{Eulersche Gleichung}
Die Eulersche Gleichung
\begin{equation}
\frac{\partial \vec v}{\partial t}+(\vec v\cdot\nabla)\vec v=-\frac1\varrho\operatorname{grad}p
\label{euler}
\end{equation}
drückt das Newtonsche Gesetz für eine Flüssigkeit der Dichte $\varrho$ aus,
welche sich mit Geschwindigkeit $\vec v$ bewegt.
Zusammen mit der Kontinuitätsgleichung
\[
\frac{\partial\varrho}{\partial t}+\operatorname{div}(\varrho \vec v)=0
\]
ergeben sich vier Gleichungen für die fünf unbekannten Funktionen
$\vec v$ (drei Komponenten) und $\varrho$ und $p$. Für eine vollständige
Lösung des Problems sind also noch weitere Gleichungen notwendig, auf die
wir jedoch nicht eingehen wollen.

Als Gleichung für $\vec v$ ist (\ref{euler}) nicht linear wegen des Termes
$(\vec v\cdot \nabla)\vec v$. Ausgeschrieben ist seine $i$-Komponente
\[
\biggl(\sum_{k=1}^3v_k\partial_k\biggr)v_i.
\]
In nur einer Dimension wird die Gleichung zu
\[
\partial_tv+v\partial_xv=-\frac1\varrho\operatorname{grad}p
\]
Ist $v$ eine Lösung der homogenen Gleichung, also
\[
\partial_t v+v\partial_x v=0,
\]
dann erfüllt $\lambda v$ die Differentialgleichung nicht mehr. Würde
$\lambda v$ die Differentialgleichung erfüllen, wurde folgen
\begin{align*}
&&\lambda(\partial_t v+\lambda v\partial_xv)&=0
\\
\Rightarrow
&&
\partial_t v+\lambda v\partial_xv&=0
\\
\Rightarrow
&&
(\lambda -1)v\partial_xv&=0
\end{align*}
Dies ist jedoch nur möglich, wenn $\partial_xv=0$, also für eine
konstante Strömung.

\subsection{Navier-Stokes Gleichung}
Die wohl berühmteste nichtlineare partielle Differentialgleichung ist
die Navier-Stokes Gleichung, welche ein strömendes zähes Medium beschreibt.
\[
\varrho\biggl(
\frac{\partial \vec v}{\partial t}+(\vec v\operatorname \nabla)\vec v
\biggr)
=
-\operatorname{grad}p
+\eta\Delta \vec v+\biggl(\zeta+\frac{\eta}3\biggr)\operatorname{grad}\operatorname{div}\vec v
\]
Auch diese Gleichung für $\vec v$ ist wegen des Ausdrucks $(\vec v\cdot\nabla)\vec v$
nicht linear.

\subsection{Gleichung von Burgers}
In einer Dimension  wird die Navier-Stokes Gleichung für eine inkompressibles
Medium zu
\[
\varrho(\partial_t v+v\partial_x v)-a\partial_x^2v=0
\]
($a$ ist ein Ausdruck in $\zeta$ und $\eta$).
Durch die Substitution
\[
u(t,x)=v(t,-x)
\]
erhält man daraus 
\begin{align*}
\partial_tu(t,x)&=\partial_tv(t,-x)\\
\partial_xu(t,x)&=-\partial_xv(t,-x)\\
\partial_x^2u(t,x)&=\partial_x^2v(t,-x)\\
\varrho
\partial_t v+v\partial_x v-\partial_x^2v
&=
\varrho(\partial_tu(t,x)-u\partial_x(t,x))-a\partial_x^2u
\end{align*}
Für $\varrho=1$ und $a=1$ wird die Gleichung zu
\[
\partial_tu=u\partial_xu+\partial_x^2u,
\]
in dieser Form heisst sie die Gleichung von Burgers,
und wird weiter unten im Abschnitt
\ref{burgers} genauer studiert.

Aus der Euler Gleichung lässt sich analog die Burgers Gleichung
für die ideale Flüssigkeit ableiten:
\[
\frac{\partial}{\partial t}u+\frac{\partial}{\partial x}\left(\frac{u^2}2\right)
=0
\]
Wir verwenden diese in Abschnitt \ref{burgersunstetig} um zu illustrieren,
wie nichtlineare Gleichungen spontan Unstetigkeiten entwickeln können.

\subsection{Korteweg-deVries Gleichung}
Die Wellenausbreitung in einem Kanal kann mit einer PDGL beschrieben werden,
welche äquivalent ist zu
\[
\partial_tu+6u\partial_xu+\partial_x^3u=0.
\]
Wegen des Terms $u\partial_xu$ ist auch diese Gleichung nicht linear.
Sie ist berühmt für die sogenannten Solitonen, Wellen, die über
lange Zeit ihre Form erhalten.

\section{Was funktioniert alles nicht mehr?}
\rhead{Besonderheiten nichtlinearer Gleichungen}
In den Lösungsverfahren linearer PDGL wurde die Linearität an verschiedenen
Stellen entscheidend benutzt:
\begin{enumerate}
\item
Das Überlagerungsprinzip ermöglicht, lokale Lösungen, die zum Beispiel
mit Hilfe eines Separationsansatzes gewonnen wurden, so zu kombinieren, dass
Anfangsbedingungen erfüllt werden können.
\item
Partikuläre Lösung und Lösung des homogenen Systems.
Das Überlagerungsprinzip ermöglicht die Lösung des inhomogenen Systems
in zwei Schritten. Einerseits wird eine beliebige Lösung der inhomogenen
Gleichung ermittelt, andererseits werden eventuell geforderte Anfangs-
oder Randbedingungen mit Hilfe der allgemeinen Lösungen des homogenen
Problems befriedigt.
\item
Konstruktion der Greenschen Funktion. In der Konstruktion der
Greenschen Funktion war verwendet worden, dass Singularitäts-Lösungen 
der PDGL mit harmonischen Funktionen kombiniert werden können, so 
dass sie auch die Randbedingungen erfüllen.
\end{enumerate}

\section{Gleichung von Burgers\label{burgers}}
\rhead{Burgers Gleichung}
Als Beispiel einer nichtlinearen Gleichung betrachten wir die Gleichung
von Burgers in der Form
\[
\partial_t u=\partial_x^2u+u\partial_xu
\]
und zeigen ein paar Möglichkeiten, wie solche Gleichungen
gelöst werden können.

\subsection{Koordinatentransformationen}
Koordinatentransformationen können helfen, die Eigenschaften der
Lösungen einer partiellen Differentialgleichung zu ergründen.

Sie $u(t,x)$ eine Lösung der Burgers Gleichung. Dann sind auch
zeitlich und örtlich verschobenen Kopien
\[
w(t,x)=u(t+C_4, x+C_3)
\]
der Funktion $u$ Lösungen.
Streckt man hingegen die $x$-Achse mit dem Faktor $C_1$, ersetzt
also $x$ durch $C_1x$, dann
muss man auch die $t$-Achse entsprechend korrigieren, also $t$
durch $C_1^2t$ ersetzen. Setzt man
\[
w(t,x)=u(C_1^2t,C_1x)
\]
in die Burgers Gleichung ergibt
\[
C_1^2\partial_t w(t,x)=C_1^2\partial_xw(t,x)+C_1w\partial_xw(t,x)
\]
was offenbar die Gleichung nicht löst. Erst die Funktion $C_1w(t,x)$
erfüllt die Gleichung, denn damit wird die Differentialgleichung zu
\[
C_1^3\partial_t w(t,x)=C_1^3\partial_xw(t,x)+C_1^3w\partial_xw(t,x)
\]
Aus einer zeitabhängigen Verschiebung von $u$ in der Form
\[
w(t,x)=u(t,x+C_2t)
\]
kann ebenfalls eine Lösung gewonnen werden:
\begin{align*}
\partial_t w(t,x)&=\partial_t u(t,x+C_2t)+C_2\partial_x u(t,x+C_2t)
\\
\partial_x w(t,x)&=\partial_x u(t,x+C_2t)
\\
\partial_x^2 w(t,x)&=\partial_x^2 u(t,x+C_2t)
\end{align*}
eingesetzt in die Differentialgleichung ergibt die Gleichung
\[
\partial_t u(t,x+C_2t)+C_2\partial u(t,x+C_2t)
=
u(t,x+C_2t)\partial_xu(t,x+C_2t)
+
\partial_x^2 u(t,x+C_2t)
\]
die jedoch wegen des zweiten Terms auf der linken Seite nicht erfüllt
sein kann. Addiert man aber zu $w$ noch die Konstante $C_2$, ergibt sich
aus dem nichtlinearen Term auf der rechten Seite zusätzlich
\[
C_2\partial_xu(t,x+C_2t),
\]
so dass also
\[
w(t,x)=u(t,x+C_2t)+C_2
\]
eine Lösung der Burgers Gleichung ist.

\subsection{Stationäre Lösung}
Hat die Burgers Gleichung stationäre Lösungen? Eine stationäre Lösung ist
eine Lösung, die nicht von der Zeit abhängt, also
$u(t,x)=u(x)$.
Eine stationäre Lösung muss die gewöhnliche
Differentialgleichung
\[
u''(x)+u(x)u'(x)=0
\]
erfüllen. Der zweite Term ist bis auf einen Faktor die Ableitung
des Quadrates $u(x)^2$. Durch die Substitution $u(x)=y(x/2)$ kann man
die Differentialgleichung in die Form
\begin{align*}
\frac14y''(x)+\frac12y(x)y'(x)&=0
\\
y''(x)+2y(x)y'(x)&=0&\Rightarrow&\qquad y''(x)+\frac{d}{dx}(y(x)^2)=0
\end{align*}
bringen.
Dies ist die Ableitung der Gleichung
\[
y'(x)+y^2(x)=B,
\]
die mit Separation gelöst werden kann:
\begin{align*}
\frac{dy}{dx}&=B-y^2\\
\int\frac{dy}{B-y^2}&=x+A
\end{align*}
Für $B>0$  findet man das Integral in Formelsammlungen als
\[
\frac1{\sqrt{B}}\operatorname{ar}\tanh \frac{y}{\sqrt{B}}=x+A
\]
Nach $y$ aufgelöst findet man also
\begin{align*}
y(x)&=\sqrt{B}\tanh(\sqrt{B}(x+A))
\end{align*}
Durch Einsetzen findet man jetzt auch die Lösung
\begin{align*}
u(x)&=
\sqrt{B}\tanh\left(\sqrt{B}\left(\frac{x}2+A\right)\right)
\end{align*}
Da die genauen Werte der Integrationskonstanten bedeutungslos sind, können
wir die Lösung auch als
\[
u(x)= 2A \tanh (Ax+B) 
\]
schreiben.

Mit Hilfe der Koordinatentransformation findet man jetzt weitere Lösungen
der Gleichung
\[
u(t,x)=
2A \tanh (A(x+\lambda t)+B) +\lambda.
\]

\subsection{Spezielle Lösungen}
Aus der physikalischen Motivation für die Gleichung lassen sich auch
einige spezielle Lösungen ableiten.
Nimmt die Geschwindigkeit linear zu, dann bleibt dies über die Zeit auch
so, aber die Steigung der Geschwindigkeitszunahme wird sich ändern.
Die Funktion
\[
u(t,x)=\frac{A-x}{B+t}
\]
sollte daher
eine Lösung sein. Tatsächlich findet man durch Einsetzen
\begin{align*}
\partial_t u(t,x)&=-\frac{A-x}{(B+t)^2}
\\
\partial_x u(t,x)&=-\frac{1}{B+t}
\\
\partial_x^2 u(t,x)&=0
\\
u(t,x)
\partial_xu(t,x)&=-\frac{A-x}{(B+t)^2}
\end{align*}
Da die zweiten Ableitungen nach $x$ verschwinden,
zeigen die erste und letzte Gleichung, dass die Burgers-Gleichung
erfüllt ist.

\section{Charakteristiken bei nichtlinearen PDGL erster Ordnung}
Im Kapitel~\ref{chapter-geometrie} war die Idee erfolgreich, die Lösung,
die man sich als Fläche im Raum mit den Koordinaten $(x,y,u)$
vorstellen konnte, aus Kurven zusammenzusetzen, die sich auf der
Fläche befinden. Das hat vor allem deshalb funktioniert, weil es
einfach war, einen Vektor zu finden, der tangential an die Fläche
war. Im wesentlichen war die quasilineare Differentialgleichung ein
Skalarprodukt mit der Flächennormalen als einem Faktor, so dass
der andere Faktor eine Tangente sein musste. Da alle möglichen
normalen in einer Ebene lagen, gab es nur eine einzige gemeinsme
Tangente, welche wir für die Differentialgleichung der
Charakteristiken verwenden konnten.

Im nichtlinearen Fall funktioniert dies nicht mehr so einfach.
Die möglichen Normalen liegen nicht mehr alle in einer Ebene,
weil die Beziehung zwischen den partiellen Ableitungen nicht
mehr linear ist. Es gibt also auch nicht mehr nur eine einzige
Tangente, die Wahl der geeigneten Tangente hängt jetzt auch
von den Werten der partiellen Ableitungen in einem Punkt ab.
Es genügt also nicht mehr nach einer Kurve
\[
s\mapsto
\begin{pmatrix}
x(s)\\
y(s)\\
u(s)
\end{pmatrix}
\]
zu finden, stattdessen müssen wir eine Kurve 
\[
s\mapsto
\begin{pmatrix}
x(s)\\
y(s)\\
u(s)\\
p(s)\\
q(s)
\end{pmatrix}.
\]
suchen, wobei $p$ und $q$ wieder für die partiellen Ableitungen
steht. Ziel ist jetzt also, ein System von Differentialgleichungen
für diese fünf Funktionen zu finden.

\subsection{Cauchy-Methode im nichtlinearen Fall}
Gegeben sei eine nichtlineare Differentialgleichung von zwei
Variablen $x$ und $y$, die wir in der Form
\[
F(x,y,u,p,q)=0
\]
schreiben, wobei $p$ und $q$ wieder für die partiellen Ableitungen
von $u$ nach $x$ bzw.~$y$ stehen.

Wir suchen Kurven auf der Lösungsfläche, wir verwenden $s$ als
Kurvenparameter.
Nehmen wir weiter an, wir hätten bereits eine Lösungsfunktion $u(x,y)$.
Weil es jetzt keinen
einfachen linearen Zusammenhang zwischen den ersten Ableitungen
mehr gibt, wie im quasilinearen Fall, müssen wir allgemeiner
auch $p(s)$ und $p(s)$ mitschleppen, wir suchen also fünf
Funktionen
\begin{equation}
x(s),\;
y(s),\; 
u(s),\; 
p(s),\; 
\text{und}\;
q(s),
\label{nichtlinear:kurve}
\end{equation}
so dass die Kurve $(x(s),y(s),z(s))$ auf der Lösungsfläche
verläuft und ausserdem die partiellen Ableitungen 
von $u(x,y)$ mit den Werten von $p$ und $q$ übereinstimmen.
Es muss also gelten
\begin{align*}
\frac{\partial u}{\partial x}(x(s), y(s))&=p(s)\\
\frac{\partial u}{\partial y}(x(s), y(s))&=q(s)
\end{align*}
Setzt man die Funktionen (\ref{nichtlinear:kurve}) in die Funktion
$F$ ein, ist die Gleichung immer erfüllt. Als muss auch die Ableitung
verschwinden:
\begin{align}
0
&=
\frac{d}{ds}F(x(s),y(s),u(s),p(s),q(s))=0
\notag
\\
&=
\frac{\partial F}{\partial x}x'(s)
+
\frac{\partial F}{\partial y}y'(s)
+
\frac{\partial F}{\partial u}u'(s)
+
\frac{\partial F}{\partial p}p'(s)
+
\frac{\partial F}{\partial q}q'(s)
\label{nichtlinear:totale}
\end{align}
Die Ableitung der Bedingung, dass die Kurve in der Fläche bleiben muss,
liefert eine weitere Gleichung für die unbekannten Funktionen:
\begin{align}
\frac{d}{ds}u(x(s),y(s))&=\frac{d}{ds}u(s)
\notag
\\
\frac{\partial u}{\partial x}x'(s)
+
\frac{\partial u}{\partial y}y'(s)
&=
u'(s)
\notag
\\
p(s)x'(s)+q(s)y'(s)
&=
u'(s).
\label{nichtlinear:flaeche}
\end{align}
%Schliesslich muss auch gelten, dass die gemischten zweiten Ableitungen
%nicht von der Reihenfolge der Ableitungen abhängen:
%\[
%\frac{\partial^2u}{\partial x\partial y}
%=
%\frac{\partial^2u}{\partial y\partial x}.
%\]
Insgesamt hat man jetzt also die Gleichungen
\begin{equation}
\begin{pmatrix}
\partial_x F+p\partial_uF& \partial_y F+q\partial_uF&0&\partial_pF&\partial_qF\\
p&q&-1&0&0
\end{pmatrix}
\begin{pmatrix}
x'(s)\\
y'(s)\\
u'(s)\\
p'(s)\\
q'(s)
\end{pmatrix}=0.
\label{nichtlinear:gleichungen}
\end{equation}
Sowenig wie die Wahl der Tangenten bei den quasilinearen partiellen
Differentialgleichungen eindeutig war, kann man erwarten dass im
allgemeinen nichtlinearen Fall ein Richtungsvektor eindeutig zu
bestimmen ist. Aber der Vektor 
\[
\begin{pmatrix}
\partial_pF\\
\partial_qF\\
p\partial_pF+q\partial_qF\\
-\partial_xF-p\partial_uF\\
-\partial_yF-q\partial_uF
\end{pmatrix}
\]
löst das Gleichungssystem (\ref{nichtlinear:gleichungen}).
Man kann also versuchen, die Kurve als Lösungskurve des
Differentialgleichungssystems
\begin{align*}
x'(s)
&=
\partial_pF
\\
y'(s)
&=
\partial_qF
\\
u'(s)
&=
p\partial_pF+q\partial_qF
\\
p'(s)
&=
-\partial_xF-p\partial_uF
\\
q'(s)
&=
-\partial_yF-q\partial_uF
\end{align*}
Diese Kurven heissen die Charakteristiken der nichtlinearen
Differentialgleichung.
Sie können auf die genau gleiche Art zur Lösung der Gleichung
verwendet werden wie im quasilinearen Fall.

\subsection{Beispiel: Eikonal-Gleichung}
Als Beispiel betrachten wir die Eikonal-Gleichung 
\[
\biggl(\frac{\partial u}{\partial x}\biggr)^2
+
\biggl(\frac{\partial u}{\partial y}\biggr)^2
=
n(x,y)^2.
\]
Sie spielt in der Optik eine wichtige Rolle, sie beschreibt die
Phase einer Welle, die sich in einem Medium mit Brechungsindex $n(x,y)$
ausbreitet. Wir wollen die Gleichung im Gebiet $\Omega=\{(x,y)\,|\,y>0\}$
lösen mit der Randbedingung $u(0,y)=0$. Weiter unten werden wir auch 
die Funktion $n(x,y)$ noch weiter einschränken.

\subsubsection{Differentialgleichungen der Charakteristiken}
Die Funktion $F$ ist offenbar
\[
F(x,y,u,p,q)=p^2+q^2-n(x,y)^2
\]
Die partiellen Ableitungen sind
\begin{align*}
\partial_xF&=-2n(x,y)\partial_x n(x,y)\\
\partial_yF&=-2n(x,y)\partial_y n(x,y)\\
\partial_uF&=0\\
\partial_pF&=2p\\
\partial_qF&=2q.
\end{align*}
Wir spezialisieren jetzt auf den Fall $n(x,y)=y$, dann wird das 
Differentialgleichungssystem für die Charakteristiken
\begin{align*}
x'&=\partial_pF=2p\\
y'&=\partial_qF=2q\\
u'&=p\partial_pF+q\partial_qF=2p^2+2q^2\\
p'&=-\partial_xF-p\partial_uF=0\\
q'&=-\partial_yF-q\partial_uF=2y
\end{align*}
Wir suchen eine Familie von Kurven, welche für $s=0$ jeweils
im Punkt $(0,y_0,0)$ beginnen.
Ausserdem muss auch $\partial_yu(0,y_0)=q(0,y_0)=0$ gelten.

\subsubsection{Lösung der Charakteristikengleichung}
Aus der Gleichung für $p'$ folgt, dass $p$ entlang einer Charakteristik
konstant ist, $p=p_0$. Aus der Gleichung für $x'$ folgt, dann
dass $x=p_0s+p_1$ ist. Weil aber $x(0)=0$ sein muss, folgt $p_1=0$.
Offenbar spielt es auch keine Rolle, welchen Wert wir für $p_0$
wählen, eine andere Wahl ändert nur die Geschwindigkeit, mit der
die Charakteristik durchlaufen wird. Wir setzen also $s=x$.

Die beiden Funktionen sind über die zweite und fünfte Gleichung
aneinander gekoppelt, leitet man die zweite einmal ab und setzt die
fünfte ein, erhält man die Gleichung
\[
y''=4y.
\]
Diese hat die bekannten Lösungen $\cosh 2s$ und $\sinh 2s$.
Mit geeigneten Konstanten $A$ und $B$ ist die allgemeine
Lösung also
\[
y(x)=A\cosh 2x + B\sinh 2x.
\]
Da $y(0)=y_0$ gelten muss, folgt $A=y_0$. Durch Ableiten finden wir
$y'(x)=2q(x)=2y_0\sinh 2x + 2B\cosh 2x$, also ist $q(x)=y_0\sinh 2x+B\cosh 2x$. 
Weil aber für $x=0$ gelten muss $q(0)=0$, ist $B=0$.
Somit haben wir die Funktionen $y$ und $q$ vollständig bestimmt:
\begin{align}
y(x)&=y_0\cosh 2x,
\label{nichtlinear:y}
\\
q(x)&=y_0\sinh 2x.\notag
\end{align}

Jetzt bleibt nur noch $u$ zu bestimmen, wozu wir die dritte Gleichung
verwenden:
\begin{align}
u'&=2p^2+2q^2=
2 + 2y_0^2\cosh^22x
\notag
\\
\Rightarrow\qquad
u&=\int
2 + 2y_0^2\cosh^22x
\,dx
\notag
\\
&=2x +2y_0^2\biggl(\frac18\sinh 4x+\frac12x\biggr)
\label{nichtlinear:l2}
\end{align}
Quadriert man (\ref{nichtlinear:y}), erhält man
\begin{align*}
y^2
&=
y_0^2\biggl(\frac12\cosh 4x+\frac12\biggr)
\\
y_0^2&=\frac{2y^2}{\cosh4x + 1}.
\end{align*}
Dies kann man in (\ref{nichtlinear:l2}) einsetzen, was die Lösung
liefert:
\begin{equation}
u(x,y)
=
2x+\frac{y^2(\sinh4x+4x)}{4(\cosh4x+1)}
\label{nichtlinear:loesung}
\end{equation}

\section{Linearisierung}
\rhead{Linearisierung}
In einigen Fällen sind Lösungen einer nichtlinearen PDGL gesucht, die 
nur wenig von einer bekannten Lösung abweichen. Beispielsweise ändert
ein stromlinienförmig gebautes Flugzeug bei hoher Geschwindigkeit die
Strömung nur vergleichsweise wenig. Man kann daher versuchen, aus der
nichtlinearen Gleichung eine lineare Gleichung für die Abweichung
von der bekannten Lösung abzuleiten. Dieses Verfahren wird oft auch
Störungstheorie genannt.

\subsection{Das allgemeine Vorgehen}
Der Einfachheit halber führen wir das Verfahren nur für Gleichungen
erster Ordnung in zwei Variablen durch. Eine solche PDGL kann mit Hilfe
einer Funktion $F(x,y,u,p,q)$ von fünf Variablen geschrieben werden als
\begin{equation}
F(x,y,u(x,y), \partial_xu(x,y),\partial_yu(x,y))=0.
\label{nichtlinear}
\end{equation}
Sei $u(x,y)$ eine Lösung der Gleichung (\ref{nichtlinear}). Wir suchen
jetzt weitere Lösungen der Gleichung, die sich jedoch nur wenig von
$u$ unterscheiden dürfen. Wir setzen diese Lösungen in der Form
\begin{equation}
u(x,y)+av(x,y)
\label{linearisierungansatz}
\end{equation}
an, wobei der Parameter $a$ dazu dienen soll, die den
zweiten Term beliebig klein machen zu können. Wir möchten eine Gleichung
für $v(x,y)$ aufstellen.

Wir setzen den Ansatz (\ref{linearisierungansatz}) in die Gleichung
ein, und erhalten
\[
F(x,y,u(x,y)+av(x,y),\partial_xu(x,y)+a\partial_xv(x,y),
\partial_yu(x,y)+a\partial_yv(x,y))=0
\]
Für $a=0$ ist die Gleichung erfüllt, wir suchen ein $v$ so dass die Gleichung
für kleine $a$ näherungsweise auch erfüllt ist. Dies erreichen wir,
indem wir nach $a$ ableiten:
\begin{align*}
0&=
\left.\frac{d}{da}
F(x,y,u(x,y)+av(x,y),\partial_xu(x,y)+a\partial_xv(x,y),
\partial_yu(x,y)+a\partial_yv(x,y))\right|_{a=0}
\\
&=F(x,y,u(x,y),\partial_xu(x,y),\partial_yu(x,y))
\\
&\qquad
+
\partial_uF(x,y,u(x,y),\partial_xu(x,y),\partial_yu(x,y))\cdot v(x,y)
\\
&\qquad
+
\partial_pF(x,y,u(x,y),\partial_xu(x,y),\partial_yu(x,y))\cdot \partial_xv(x,y)
\\
&\qquad
+
\partial_qF(x,y,u(x,y),\partial_xu(x,y),\partial_yu(x,y))\cdot \partial_yv(x,y)
\end{align*}
Der erste Term fällt weg, weil $u$ bereits eine Lösung ist,
es bleibt eine lineare PDGL für $v$:
\begin{align*}
0&=
\partial_uF(x,y,u(x,y),\partial_xu(x,y),\partial_yu(x,y))\cdot v(x,y)
\\
&\qquad
+
\partial_pF(x,y,u(x,y),\partial_xu(x,y),\partial_yu(x,y))\cdot \partial_xv(x,y)
\\
&\qquad
+
\partial_qF(x,y,u(x,y),\partial_xu(x,y),\partial_yu(x,y))\cdot \partial_yv(x,y)
\end{align*}
Etwas allgemeiner könnte auch noch die Funktion $F$ von $a$ abhängen,
also $F(a,x,y,u,p,q)$. In diesem Fall wird die Ableitung nach $a$ an
der Stelle $a=0$ zu
\begin{align*}
0&=
\left.\frac{d}{da}
F(x,y,u(x,y)+av(x,y),\partial_xu(x,y)+a\partial_xv(x,y),
\partial_yu(x,y)+a\partial_yv(x,y))\right|_{a=0}
\\
&=
F(0,x,y,u(x,y),\partial_xu(x,y),\partial_yu(x,y))
\\
&\qquad
+\partial_aF(0,x,y,u(x,y),\partial_xu(x,y),\partial_yu(x,y))
\\
&\qquad
+
\partial_uF(a,x,y,u(x,y),\partial_xu(x,y),\partial_yu(x,y))\cdot v(x,y)
\\
&\qquad
+
\partial_pF(a,x,y,u(x,y),\partial_xu(x,y),\partial_yu(x,y))\cdot \partial_xv(x,y)
\\
&\qquad
+
\partial_qF(a,x,y,u(x,y),\partial_xu(x,y),\partial_yu(x,y))\cdot \partial_yv(x,y)
\end{align*}
Die lineare PDGL ist in diesem Fall
\begin{align*}
0
&=
\partial_aF(0,x,y,u(x,y),\partial_xu(x,y),\partial_yu(x,y))
\\
&\qquad
+
\partial_uF(x,y,u(x,y),\partial_xu(x,y),\partial_yu(x,y))\cdot v(x,y)
\\
&\qquad
+
\partial_pF(x,y,u(x,y),\partial_xu(x,y),\partial_yu(x,y))\cdot \partial_xv(x,y)
\\
&\qquad
+
\partial_qF(x,y,u(x,y),\partial_xu(x,y),\partial_yu(x,y))\cdot \partial_yv(x,y)
\end{align*}
Jede Lösung der nichtlinearen Gleichung gibt also Anlass zu Lösungen
in ``unmittelbarer'' Nähe, welche aus der linearisierten Gleichung
gefunden werden können.

\subsection{Linearisierung von PDGL zweiter Ordnung}
Eine PDGL zweiter Ordnung ist gegeben durch eine Funktion von neun
Variablen
\[
F(x,y,u,p,q,r,s,t),
\]
in die man die Funktionswerte und die Ableitungen einsetzt:
\[
F(x,y,u(x,y), \partial_xu(x,y),\partial_yu(x,y),
\partial_x^2u(x,y),
\partial_x\partial_yu(x,y),
\partial_y^2u(x,y))
=0
\]
Um die linearisierte PDGL zu finden, geht man wieder von einer
Lösung $u(x,y)$ aus, und sucht eine ``Nachbarlösung'' in der
Form $u(x,y)+av(x,y)$. Diesen Ansatz setzt man in die 
Differentialgleichung ein und leitet an der Stelle $a=0$
nach $a$ ab. Wie im Falle der Gleichung erster Ordnung kann auch
hier der Parameter $a$ auch in $F$ vorkommen, wir führen gleich
von Anfang an diesen allgemeineren Fall  durch:
\begin{align*}
0&=
\frac{d}{da}
F(a,x,y,u(x,y), \partial_xu(x,y)+a\partial_xv(x,y),\partial_yu(x,y)+a\partial_yv(x,y),
\\
&\qquad
\partial_x^2u(x,y)+a\partial_x^2v(x,y),
\partial_x\partial_yu(x,y)+a\partial_x\partial_yv(x,y),
\partial_y^2u(x,y)+a\partial_y^2v(x,y))\bigg|_{a=0}
\\
&=
F(x,y,u(x,y), \partial_xu(x,y),\partial_yu(x,y),
\partial_x^2u(x,y),
\partial_x\partial_yu(x,y),
\partial_y^2u(x,y))
\\
&\qquad+
\partial_a
F(0,x,y,u(x,y), \partial_xu(x,y),\partial_yu(x,y),
\partial_x^2u(x,y),
\partial_x\partial_yu(x,y),
\partial_y^2u(x,y))
\\
&\qquad+
\partial_u
F(0,x,y,u(x,y), \partial_xu(x,y),\partial_yu(x,y),
\partial_x^2u(x,y),
\partial_x\partial_yu(x,y),
\partial_y^2u(x,y))\cdot v(x,y)
\\
&\qquad+
\partial_p
F(0,x,y,u(x,y), \partial_xu(x,y),\partial_yu(x,y),
\partial_x^2u(x,y),
\partial_x\partial_yu(x,y),
\partial_y^2u(x,y))\cdot \partial_xv(x,y)
\\
&\qquad+
\partial_q
F(0,x,y,u(x,y), \partial_xu(x,y),\partial_yu(x,y),
\partial_x^2u(x,y),
\partial_x\partial_yu(x,y),
\partial_y^2u(x,y))\cdot \partial_yv(x,y)
\\
&\qquad+
\partial_q
F(0,x,y,u(x,y), \partial_xu(x,y),\partial_yu(x,y),
\partial_x^2u(x,y),
\partial_x\partial_yu(x,y),
\partial_y^2u(x,y))\cdot \partial_x^2v(x,y)
\\
&\qquad+
\partial_r
F(0,x,y,u(x,y), \partial_xu(x,y),\partial_yu(x,y),
\partial_x^2u(x,y),
\partial_x\partial_yu(x,y),
\partial_y^2u(x,y))\cdot \partial_x\partial_yv(x,y)
\\
&\qquad+
\partial_s
F(0,x,y,u(x,y), \partial_xu(x,y),\partial_yu(x,y),
\partial_x^2u(x,y),
\partial_x\partial_yu(x,y),
\partial_y^2u(x,y))\cdot \partial_y^2v(x,y)
\end{align*}

\subsection{Linearisierung der Burgers Gleichung}
Wir wenden das Linearisierungsverfahren auf die Gleichung von Burgers an.
Wir schreiben sie zur leichteren Übertragbarkeit der Formeln
mit $x$ und $y$ als unabhängige Variablen anstellen von $x$ und $t$,
wir suche also Lösungen der Differentialgleichung
\[
\partial_yu-u\partial_xu-\partial_x^2u=0
\]
Die Funktion $F$ ist in diesem Fall
\[
F(x,y,u,p,q,r,s,t)=q-r-up.
\]
Die Linearisierungsformeln sagen, dass wir die Ableitungen von $F$ nach
den Variablen verwenden müssen als Koeffizienten der Ableitungen,
für die die Variablen stehen.  Dabei müssen wir in die partiellen Ableitungen
von $F$ jeweils die Lösung $u$ bzw.~ihre partiellen Ableitungen einsetzen.
Die Koeffizienten sind
\begin{align*}
\partial_uF&=-p=-\partial_xu(x,y)
\\
\partial_pF
&=-u=-u(x,y)
\\
\partial_qF
&=1
\\
\partial_rF
&=-1
\end{align*}
alle anderen partiellen Ableitungen von $F$ verschwinden. Die linearisierte
Gleichung lautet also
\begin{align*}
-\partial_xu(x,y)\cdot v(x,y)
-
u(x,y)\cdot\partial_xv(x,y)
+\partial_yv(x,y)
-\partial_x^2v(x,y)=0,
\end{align*}
eine parabolische PDGL. In den ursprünglichen Koordinaten und der
üblichen Reihenfolge der Ableitungen geschrieben
lautet sie
\[
-\partial_x^2v(t,x)
+\partial_tv(t,x)
+ u(t,x)\cdot\partial_xv(t,x)
+\partial_xu(t,x)\cdot v(t,x)
=0,
\]

\subsubsection{Konstante Geschwindigkeit}
Die konstante Funktion $u(t,x)=c$ ist eine Lösung der nichtlinearen
Gleichung. Die linearisierte Gleichung wird damit zu
\[
\partial_tv
-\partial_x^2v
-c\partial_xv=0.
\]
Leider ist diese Gleichung auch noch nicht direkt in einer Form,
in der wir sie lösen können. Schreiben wir die gesuchte Funktion
in der Form
\[
v(t,x)=w(t,x+ct)
\]
und bezeichnen die partiellen Ableitungen von $w$ nach der ersten
und zweiten Variablen mit $\partial_1w$ bzw.~$\partial_2w$, dann
erhalten wir zunächst die partiellen Ableitungen von $v$
in der Form
\begin{align*}
\partial_t v(t,x)&=\partial_1w(t,x+ct)+c\partial_2w(t,x+ct)
\\
\partial_x v(t,x)&=\partial_2w(t,x+ct)
\\
\partial_x^2v(t,x)&=\partial_2^2w(t,x+ct)
\end{align*}
In die PDGL eingesetzt erhalten wir
\begin{align*}
0&=
\partial_t v(t,x)
-\partial_x^2v(t,x)
-c\partial_x v(t,x)
\\
&=
\partial_1w(t,x+ct)+c\partial_2w(t,x+ct)
-\partial_2^2w(t,x+ct)
-c\partial_2w(t,x+ct)
\\
&=\partial_1w(t,x+ct)-\partial_2w(t,x+ct)
\end{align*}
Die Funktion $w$ ist also eine Lösung der Wärmeleitungsgleichung.

\subsubsection{Lösungen der linearisierten Gleichung}
Die Wärmeleitungsgleichung hat die Standardlösungen
\[
w(t,x)=\frac1{\sqrt{t}}e^{-\frac{(x-\xi)^2}{4t}},
\]
die man durch Nachrechnen unmittelbar bestätigen kann:
\begin{align*}
\partial_t w(t,x)
&=
\left(
-\frac1{2t^{\frac32}}
+\frac{(x-\xi)^2}{4t^{\frac52}}
\right)e^{-\frac{x^2}{4t}}
\\
\partial_x w(t,x)
&=
-\frac{(x-\xi)}{2t^{\frac32}}
e^{-\frac{x^2}{4t}}
\\
\partial_x^2w(t,x)
&=
\left(
\frac{(x-\xi)^2}{4t^{\frac52}}
-\frac{1}{2t^{\frac32}}
\right)e^{-\frac{x^2}{4t}}
\\
\partial_tw(t,x)-\partial_x^2w(t,x)
&=
\left(
-\frac1{2t^{\frac32}}
+\frac{(x-\xi)^2}{4t^{\frac52}}
-\frac{(x-\xi)^2}{4t^{\frac52}}
+\frac{1}{2t^{\frac32}}
\right)e^{-\frac{(x-\xi)^2}{4t}}=0
\end{align*}
Als Lösungen der linearisierten Gleichung kommen also Funktionen der
Form
\[
v(t,x)=\frac1{\sqrt{t}}e^{-\frac{(x-\xi+ct)^2}{4t}}
\]
oder Überlagerungen derselben in Frage.

\subsubsection{Stabilität der Strömung}
Die im vorangegangenen Abschnitt abgeleiteten Lösung der linearisierten
Gleichung lässt sich auch so interpretieren: eine kleine Störung zur Zeit
$t=0$ wird Anlass zu einem mit Geschwindigkeit $c$ nach links
laufenden gaussschen Wellenbuckel Anlass geben. Dieser Buckel wird mit
der Zeit immer flacher und ausgedehnter werden. Insbesondere verschwinden
kleine Störungen mit der Zeit wieder, die Strömung ist also stabil.

Tritt jedoch eine Störung auf, die so gross ist, dass die Linearisierung
nicht mehr anwendbar ist, dann werden neue Phänomene auftreten, insbesondere kann
die Strömung instabil sein. Störungen können sich aufschaukeln bis die
Lösung nicht mehr stetig ist (Geschwindigkeitssprünge, Schockwellen)
oder die Geschwindigkeit könnte unvorhersagbar zu schwanken beginnen
(Turbulenz).

Alternativ könnte man auch mit dem Maximumprinzip argumentieren: Extremwerte
sind in der Anfangsbedingung zu finden, eine Störung muss also mit der
Zeit immer schwächer werden.

\subsubsection{Ansteigende Geschwindigkeit}
Wir untersuchen diese Gleichung noch für den Fall der bereits 
früher untersuchten speziellen Lösung 
\[
u(t,x)=\frac{A-x}{B+t}
\]
explizit aufschreiben. Dazu benötigen wir die partielle Ableitung
nach $x$, die wir ebenfalls bereits früher berechnet haben:
\begin{align*}
\partial_x u(t,x)&=-\frac{1}{B+t}
\end{align*}
Die linearisierte Gleichung wird damit zu
\begin{align*}
-\partial_x^2v(t,x)
+\partial_tv(t,x)
+ \frac{A-x}{B+t}\partial_xv(t,x)
-\frac{1}{B+t} v(t,x)
&=0
\\
-(B+t)\partial_x^2v(t,x)
+(B+t)\partial_tv(t,x)
+ (A-x)\partial_xv(t,x)
- v(t,x)
&=0
\end{align*}
Leider lässt sich in diesem Fall nicht so offensichtlich eine Lösung finden.

\section{Entstehung von Singularitäten\label{burgersunstetig}}
\rhead{Singularitäten}
\begin{figure}
\begin{center}
\includegraphics[width=0.8\hsize]{../common/images/burgers-1}
\end{center}
\caption{Anfangsbedingung für die Burgers Gleichung einer idealen
Flüssigkeit\label{burgersanfang}}
\end{figure}
In diesem Abschnitt möchten wir die Burgers Gleichung für die ideale
Flüssigkeit
\[
\frac{\partial}{\partial t}u+\frac{\partial}{\partial x}\left(\frac{u^2}2\right)=0
\]
auf dem Interval $x\in[0,1]$ und für $t>0$
mit der Anfangsbedingung
\[
u(0,x)=\begin{cases}
0\qquad&\text{$x<\frac14$ oder $x>\frac34$}\\
x-\frac14\qquad&\frac14\le x\le \frac12\\
\frac34-x\qquad&\frac12\le x\le \frac34
\end{cases}
\]
lösen (siehe Abbildung \ref{burgersanfang}).

\begin{figure}
\begin{center}
\includegraphics[width=0.8\hsize]{../common/images/burgers-2}
\end{center}
\caption{Niveaulinien (Charakteristiken) der Lösung der Burgers-Gleichung für
die ideale Flüssigkeit\label{burgersniveau}}
\end{figure}
Die Differentialgleichung ist von erster Ordnung, in der Form
\[
\partial_tu+u\partial_xu=0
\]
haben wir die zugehörige geometrische Theorie in Abschnitt
\ref{pdgl1ordnung} besprochen. Dort wurde darauf hingewiesen,
dass der Vektor 
\[
\begin{pmatrix}
1\\u\\0
\end{pmatrix}
\]
immer an die Fläche $u=u(t,x)$ tangential ist. Die Lösungsfläche
entsteht dadurch, dass die Anfangsbedingung mit Hilfe dieses Vektors
verschoben wird. 
Der Punkt $(0,x,u(0,x))$ entwickelt sich nach dieser Methode zu
Punkten $(t,x+u(0,x)t, u(0,x))$ der Lösungsfläche. Die Kurven
gleichen Funktionswertes bilden also Geraden, deren Steigung im
$x$-$t$-Koordinatensystem $u(0,t)^{-1}$ ist. Die Niveaulinien der
Lösung sehen daher aus wie in der Abbildung \ref{burgersniveau}
dargestellt.
Daraus kann man jetzt auch die Lösungen ablesen. In der Abbildung
\ref{burgerssprung} kann man die Entwicklung der Spitze zu einem
Sprung an der Stelle $x=\frac34$ beobachten.

\begin{figure}
\begin{center}
\includegraphics[width=0.8\hsize]{../common/images/burgers-1}
\includegraphics[width=0.8\hsize]{../common/images/burgers-3}
\includegraphics[width=0.8\hsize]{../common/images/burgers-4}
\includegraphics[width=0.8\hsize]{../common/images/burgers-5}
\includegraphics[width=0.8\hsize]{../common/images/burgers-6}
\end{center}
\caption{Entwicklung eines Sprungs in der Lösung der Gleichung von Burgers\label{burgerssprung}}
\end{figure}

Man könnte argumentieren, dass die Anfangsbedingung ja bereits nicht differenzierbar
ist, und dass die Lösungen dies ebenfalls nicht sind. Man kann
jedoch eine beliebige glatte Funktion als Anfangsbedingung wählen, welche
innerhalb des Intervals $[\frac14,\frac12]$ monoton von $0$ auf $\frac14$ 
ansteigt, im Punkt $x=\frac12$ den maximalen Wert $\frac14$ annimmt,
und im Interval $[\frac12,\frac34]$ monoton auf $0$ abfällt.
Das Maximum wird sich entlang der Charakteristiken zum Punkt
$(1,\frac34,\frac14)$ entwicklen. Der Funktionswert $u(0,x)$ im Punkt $(0,x)$
für $x\in[\frac14,\frac12]$ ist derselbe wie im Punkt $(1,x+u(0,x))$.
Der Funktionsgraph $x\mapsto u(1,x)$ hat also die Parameterdarstellung
\[
[{\textstyle\frac14},{\textstyle\frac12}]\to\mathbb R\colon s\mapsto (s+u(0,s),u(0,s))
\]
Diese ist jedenfalls eine glatte Funktion. Andererseits komprimieren
die Charakteristiken das Interval $[\frac12,\frac34]$ auf
den Punkt $(1,\frac34)$ so dass bei $x=\frac34$ wieder ein Sprung entsteht.

