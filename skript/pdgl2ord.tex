\chapter{Partielle Differentialgleichungen zweiter Ordnung\label{chapter-2ordnung}}
\lhead{Lineare PDGL 2. Ordnung}
Lineare Differentialgleichungen zweiter Ordnung haben die Form
\begin{equation}
\sum_{i,j=1}^na_{ij}\partial_i\partial_j u+\sum_{i=1}^nb_i\partial_i u+cu=f.
\label{operator2ordnung}
\end{equation}
Alle Musterbeispiele im Kapitel \ref{chapter-beispiele} passen auf dieses 
Beispiel. Trotz der oberfl"achlichen Gemeinsamkeit zeigen die L"osungen
der Beispielgleichungen grunds"atzlich verschiedenes Verhalten.
\begin{itemize}
\item Die Wellengleichung hat L"osungen, die sich mit endlicher
Ausbreitungsgeschwindigkeit im Definitionsgebiet ausbreiten. Eine "Anderung
der Anfangsbedingung macht sich erst nach einer gewissen Zeit in
entfernten Teilen des Gebietes bemerkbar.
\item "Anderungen der Randbedingungen der W"armeleitungsgleichung wirken
sich sofort auf die Werte der L"osung zu sp"aterer Zeit aus und breiten
sich mit unendlicher Geschwindigkeit durch das Gebiet aus, sie haben jedoch
keine Auswirkungen auf die Vergangenheit.
\item "Andert man die Randbedingung der Poisson-Gleichung $\Delta \varphi=f$,
"andert sich die L"osung sofort und "uberall im Gebiet.
\end{itemize}
In diesem Kapitel sollen partielle Differentialgleichungen zweiter Ordnung
nach diesem Schema klassiert werden.

\section{Typen von Differentialgleichungen zweiter Ordnung}
\rhead{Klassifikation}
Zun"achst betrachten wir nur die zweiten Ableitungen in (\ref{operator2ordnung}).
F"ur eine zweimal stetig differenzierbare Funktion $u$ sind die zweiten
Ableitungen unabh"angig von der Reihenfolge, also
$\partial_i\partial_ju=\partial_j\partial_iu$.
Durch Zusammenfassen von Termen mit vertauschten Indizes $i,j$ kann 
erreichen, dass $a_{ij}=a_{ji}$ ist.
Die Matrix $(a_{ij})$ ist also symmetrisch.

\begin{definition}Ist $L$ ein linearer Differentialoperator zweiter Ordnung
\[
Lu=\sum_{i,j=1}^na_{ij}\partial_i\partial_ju,
\]
dann heisst die symmetrische Matrix
$(a_{ij})$ das Symbol des Operators.
\end{definition}

Eine Koordinatentransformation
\[
x_i=\sum_{j}t_{ij}x_j'
\]
transformiert die Ableitungen nach der Kettenregel
\[
\frac{\partial}{\partial x'_i}u(x)=\sum_{j=1}^nt_{ij}\frac{\partial}{\partial x_j}u(x).
\]
Wir w"ahlen f"ur die Transformation eine Drehung, weil dann die
inverse Matrix durch die transponierte Matrix gegeben ist, also
\begin{align*}
x'_j&=\sum_{i=1}^nx_it_{ij}.
\\
\sum_{i=1}^nt_{ij}\frac{\partial}{\partial x'_i}u&=\frac{\partial}{\partial x_j}u
\end{align*}
Damit kann man auch den Differentialoperator $L$ in den Ableitungen nach $x'_i$
schreiben:
\begin{align*}
Lu&=\sum_{i,j=1}^na_{ij}\frac{\partial^2}{\partial x_i\partial x_j}u
\\
&=\sum_{i,j,k,l=1}^n a_{ij}
t_{ki}\frac{\partial}{\partial x_k'}
t_{lj}\frac{\partial}{\partial x_l'}u
\\
&=\sum_{i,j,k,l=1}^n
t_{ki}a_{ij}t_{lj}\frac{\partial}{\partial x_k'}
\frac{\partial}{\partial x_l'}u
\end{align*}
Wer lesen daraus ab, dass das Symbol $A$ des Operators $L$ in den Koordinaten
$x_i$ durch die Koordinatentransformation auf die Koordinaten $x_j'$ 
mit der Matrix $T=(t_{ij})$ transformiert wird durch
\[
A'=TAT^*.
\]
In der linearen Algebra lernt man, dass man zu einer symmetrischen
Matrix $A$ immer eine Drehmatrix $T$ finden kann, so dass
$A'=TAT^*$ Diagonalform hat, wobei die Diagonalelemente die
Eigenwerte der Eigenvektoren sind.
\[
TAT^*
=
\begin{pmatrix}\lambda_1&\dots&0\\
\vdots&\ddots&\vdots\\
0&\dots&\lambda_n
\end{pmatrix}
\]
Man kann also durch geeignete Wahl des Koordinatensystems mindestens in
einem Punkt einen Differentialoperator zweiter Ordnung immer in die
Form
\[
\sum_{i=1}\lambda_i\frac{\partial^2}{\partial x_i^2}u
\]
bringen.

Durch eine Streckung der Koordinatenachse $x_i$ um $\sqrt{|\lambda_i|}$
kann man ausserdem erreichen, dass der Differentialoperator die
Form
\[
\sum_{i=1}\varepsilon_i\frac{\partial^2}{\partial x_i^2}u
\]
erh"alt, wobei die Zahlen $\varepsilon_i$ die Werte, $1$, $-1$ oder $0$
haben k"onnen.

Differentialoperatoren zweiter Ordnung werden also charakterisiert durch
die Vorzeichen der Eigenwerte des Symbols. Wir bezeichnen die Anzahl
der positiven Eigenwerte mit $P$, die Anzahl der negativen Eigenwerte
mit $N$, und die Anzahl der verschwindenden Eigenwerte mit $Z$.
Die Beispiele des Kapitels \ref{chapter-beispiele} k"onnen wie
folgt kategorisiert werden:
\begin{center}
\begin{tabular}{l|ccc}
Differentialoperator&P&N&Z
\\
\hline
Laplace&
$n$&$0$&$0$
\\
Wellengleichung&
$n-1$&$1$&$0$
\\
W"armeleitung&
$n-1$&$0$&$1$
\end{tabular}
\end{center}
Daraus leiten wir die folgende Klassifikation ab:

\begin{definition} Die Differentialgleichung 
(\ref{operator2ordnung})
mit den Kennzahlen $P$, $N$ und $Z$ heissen
\begin{center}
\begin{tabular}{lcl}
hyperbolisch&falls&$Z=0$ und $P=1$ oder $P=n-1$\\
parabolisch&falls&$Z>0$\\
elliptisch&falls&$Z=0$ und $P=n$ oder $P=0$\\
ultrahyperbolisch&falls&$Z=0$ und $1<P<n-1$
\end{tabular}
\end{center}
\end{definition}
Insbesondere sind die Gleichungen des Kapitels \ref{chapter-beispiele}
Beispiele f"ur diese Typen von partiellen Differentialgleichungen
wie folgt:
\begin{itemize}
\item {\bf elliptisch:} Potential
\item {\bf hyperbolisch:} Wellengleichung, linearisierte "Uberschallstr"omung
\item {\bf parabolisch:} W"armeleitung, Diffusion
\end{itemize}

\section{Kanonische Form}
\rhead{Kanonische Form}
F"ur lineare PDGL mit konstanten Koeffizienten, f"ur die also die Gr"ossen
$a_{ij}$ nicht von den Koordinaten abh"angen, kann man durch eine Drehung
des Koordinatensystems immer erreichen, dass die Koeffizentenmatrix diagonal
ist mit Diagonalelementen aus $\{0,\pm1\}$. Es stellt sich die Frage, ob
dies wohl auch bei partiellen Differentialgleichungen mit allgemeinen
Koeffizienten m"oglich ist.

In zwei Dimensionen l"asst sich ein geeignetes Koordinatensystem mit folgender
Idee konstruieren. 
In jedem Punkt $(x,y)$ gibt es zwei Vektoren $\vec e_+$ und
$\vec e_-$, welche Eigenvektoren zum gr"osseren bzw.~kleineren Eigenvektor
der Matrix $A$ sind. Solange die Eigenwerte verschieden sind, lassen sich
die Vektorfelder $\vec e_+$ und $\vec e_-$ auf stetige Weise konstruieren.
Die Vektoren stehen in jedem Punkt senkrecht aufeinander.

Nun kann man zu jedem Vektorfeld eine Schar von L"osungskurven
konstruieren. Jede Schar h"angt von einem Parameter ab, wir nennen die
beiden Parameter $\xi$ und $\eta$. Ein Punkt in der Ebene kann mit
den Koordinaten $\xi$ und $\eta$ beschrieben werden.
In diesen Koordinaten hat die Matrix $A$ Diagonalform.

In drei Dimensionen l"asst sich diese Idee nicht mehr durchf"uhren,
die Felder der Eigenvektoren m"ussen hier zus"atzliche Bedingungen
erf"ullen.

Da man aber in der N"ahe eines Punktes immer erreichen kann, dass $A$ diagonal
ist, ist dort die Differentialgleichung n"aherungsweise diagonal.
Wenn wir vor allem an den allgemeinen Eigenschaften einer L"osung interessiert
sind, gen"ugt es also
also f"ur alle Eigenschaften, die sich auf das Verhalten der L"osung im kleinen
beziehen, die Standarddifferentialgleichungen des folgenden Katalogs
zu studieren:
\begin{align*}
\Delta u&=0\\
\partial_tu&=k\Delta u\\
\partial_t^2u&=a^2\Delta u\\
\end{align*}
was wir in dieser Reihenfolge in den folgenden drei Kapiteln tun
werden.

