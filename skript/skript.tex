%
% skript.tex -- Skript zur Vorlesung partielle Differentialgleichungen
%               gehalten am MSE Master, Herbstsemester 2008
%
% (c) 2006-2011 Prof. Dr. Andreas Mueller, HSR
% $Id: skript.tex,v 1.3 2008/09/13 23:01:45 afm Exp $
%
\documentclass[a4paper,12pt]{book}
\usepackage{german}
\usepackage{times}
\usepackage{amsmath}
\usepackage{amssymb}
\usepackage{amsfonts}
\usepackage{amsthm}
\usepackage{amscd}
\usepackage{graphicx}
\usepackage{fancyhdr}
\usepackage{textcomp}
\usepackage{picins}
\usepackage{txfonts}
\usepackage[all]{xy}
\usepackage{paralist}
\usepackage{hyperref}
\makeindex
\begin{document}
\pagestyle{fancy}
\lhead{}
\rhead{}
\frontmatter
\newcommand\HRule{\noindent\rule{\linewidth}{1.5pt}}
\begin{titlepage}
\vspace*{\stretch{1}}
\HRule
\vspace*{10pt}
\begin{flushright}
{\Huge
Partielle Differentialgleichungen}
\end{flushright}
\begin{flushright}
{\Large Theoretischer Teil}
\end{flushright}
\HRule
\begin{flushright}
\vspace{30pt}
\LARGE
Andreas M"uller
\end{flushright}
\vspace*{\stretch{2}}
\begin{center}
Hochschule f"ur Technik, Rapperswil, 2008-2013
\end{center}
\end{titlepage}
\hypersetup{
    colorlinks=true,
    linktoc=all,
    linkcolor=blue
}
\tableofcontents
\newtheorem{satz}{Satz}[chapter]
\newtheorem{problem}[satz]{Problem}
\newtheorem{hilfssatz}[satz]{Hilfssatz}
\newtheorem{definition}[satz]{Definition}
\newtheorem{annahme}[satz]{Annahme}
\newtheorem{aufgabe}[satz]{Aufgabe}
\newenvironment{beispiel}[1][Beispiel]{%
\begin{proof}[#1]%
\renewcommand{\qedsymbol}{$\bigcirc$}
}{\end{proof}}
\mainmatter
\input a-einleitung.tex
\input beispiele.tex
\input klassifikation.tex
\input geometrie.tex
\input separation.tex
\input tsunami.tex
\input jacobi.tex
\input transformation.tex
\input pdgl2ord.tex
\input elliptisch.tex
\input parabolisch.tex
\input hyperbolisch.tex
%\input nichtlinear.tex
\input skript.ind
\end{document}
