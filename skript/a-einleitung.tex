%
% a-einleitung.tex
%
% (c) 2008 Prof Dr Andreas Mueller, Hochschule Rapperswil
%
\lhead{Einleitung}
\chapter*{Einleitung}
\index{Mechanik!klassische}
\index{Elektrizitatslehre@Elektrizit\ätslehre}
\index{Populationsdynamik}
Gewöhnliche Differentialgleichungen werden in der klassischen Mechanik,
in der Elektrizitätslehre, in der Populationsdynamik, und ganz 
allgemein überall dort erfolgreich eingesetzt, wo die
Änderungsrate einer skalaren Grösse von den aktuellen Werten dieser Grösse
abhängt.
\index{Massepunkt}
\index{Newtonsches Gesetz}
Ein Massepunkt bei der Ortskoordinate $x$ ändert seine Position
gemäss dem Newtonschen Gesetz
\[
ma=m\ddot x=F.
\]
Hängt $F$ von der aktuellen Position des Massepunktes ab, wie dies zum
Beispiel bei einem an einer Feder aufgehängten Gewicht der Fall ist,
liegt eine Differentialgleichung für die von der Zeit abhängige
Position $x(t)$ des Massepunktes vor:
\[
m\frac{d^2}{dt^2}x(t)=F(x(t)).
\]
In der klassischen Analysis wird gezeigt, dass solche Gleichungen bei
gegebenen Anfangsbedingungen eine eindeutig bestimmte Lösung haben.
Mit etwas zusätzlichem Aufwand werden auch Differentialgleichungen
für mehrere skalare Grössen behandelt, die alle vom gleichen Parameter
(typischerweise der Zeit) abhängen.

\index{Felder!elektrische}
\index{Temperaturverteilung}
\index{Druck}
\index{Gas}
\index{Stromungsgeschwindigkeit@Str\ömungsgeschwindigkeit}
\index{Flussigkeit@Fl\üssigkeit}
Zur Beschreibung elektrischer Felder, der Temperaturverteilung in
einem Fest\-körper, des Druckes in einem Gas oder der Strömungsgeschwindigkeit
in einer Flüssigkeit reicht eine einzelne oder einige wenige skalare
Grössen jedoch nicht. Vielmehr muss man in den genannten Beispielen
zu jedem Zeitpunkt und an jedem Punkt des Raumes die interessierende Grösse
bestimmen. Sie ist also nicht mehr nur eine Funktion der Zeit, sondern eine
Funktion von Ort und Zeit.
\index{Saite!schwingende}
Die Auslenkung einer schwingenden Saite
beispielsweise ist eine Funktion $f(x,t)$, sie beschreibt, wie stark
die Saite an der Koordinate $x$ zur Zeit $t$ von der Ruhelage ausgelenkt ist.
Insbesondere hat die Saite zu jeder Zeit $t_0$ eine andere Form, die durch
die Funktion von einer Variablen
\[
x\mapsto f(x,t_0)
\]
gegeben ist.

Man könnte vielleicht die Hoffnung hegen, dass die zeitliche Änderung
der Auslenkung an der Stelle $x$ nur von der Position $x$ abhängt.
Dies kann aber nicht sein, denn wenn die Saite in einer Umgebung
des Punktes $x$ sehr stark schwankt, ``ziehen'' die ``benachbarten'' Teile
der Saite ebenfalls an dem Punkt, vor allem dann, wenn die Saite gekrümmt
ist. Die zeitliche Änderung der Auslenkung, also die Ableitung von $f(x,t)$
nach der Zeit, hängt also nicht nur von $f(x,t)$ ab, sondern auch von
den Ableitungen nach $x$. Es entsteht eine Gleichung, die die verschiedenen
partiellen Ableitungen von $f$ miteinander verknüpft, zum Beispiel in
etwa folgender Form:
\begin{align*}
a^2\frac{\partial^2}{\partial x^2}f&= \frac{\partial^2}{\partial t^2}f
\end{align*}
Kann man diese sogenannte partielle Differentialgleichung lösen,
kann man die Form der Saite zu jedem
beliebigen zukünftigen Zeitpunkt voraussagen.
\index{Differentialgleichung!partielle}

\index{Differentialgleichung!gew\öhnliche}
Bei gewöhnlichen Differentialgleichungen hat man gelernt, dass die Gleichung
alleine die Lösung noch nicht eindeutig bestimmt, sondern dass dazu noch
weitere Angaben, sogenannte Anfangs- oder Randbedingungen nötig sind.
Die Theorie der partiellen Differentialgleichungen muss daher auf folgende
Fragen eine Antwort liefern:
\begin{enumerate}
\item Wie muss ein Problem, welches mit Hilfe partieller Differentialgleichungen
gelöst werden soll, überhaupt gestellt werden, damit die Lösung wohlbestimmt
ist?
\item Welche Eigenschaften haben die Lösungen? Wieviele Ableitungen haben
die Lösungen? Wie sieht die Lösung in der Nähe des Randes aus, bleiben
sie stetig, oder werden Sie unbegrenzt gross?
\item Wie berechnet man die Lösung einer partiellen Differentialgleichung?
\end{enumerate}
Die Vorlesung ``Partielle Differentialgleichungen'' im Master of Sciences in
Engineering versucht diese Fragen zu beantworten. Der Kurs ist in zwei Teile
geteilt. Im ersten Teil werden die genannten Fragen theoretisch untersucht.
Leider stellt sich heraus, dass man zwar klare Aussagen darüber machen kann,
welche Gleichungen lösbar sind, und wieviele Lösungen gegebenenfalls
existieren, dass man auch deren Eigenschaften sehr gut beschreiben kann,
aber leider auch nur in ganz wenigen Fällen geschlossene Formeln für die
Lösungen finden kann. Mit der mathematischen Gewissheit, dass Lösungen
existieren und sich ``anständig benehmen'' eröffnet sich aber auch die
im zweiten Teil behandelte Möglichkeit, die Gleichungen mit Hilfe eines
Computers numerisch zu lösen.

Dieses Skript enthält die im ersten Teil der Vorlesung behandelten 
Themen. Es gliedert sich in folgende Kapitel:
\begin{enumerate}
\item Beispiele von partiellen Differentialgleichungen. Die wichtigsten
Differentialgleichungen aus Mechanik, Wärmelehre und Elektrizitätslehre
werden vorgestellt. Sie dienen in späteren Kapiteln als Beispiele
für theoretische Überlegungen wie auch für Lösungsmethoden.
\item Lösung mit Separation der Variablen.
\index{Separation der Variablen}
\item Lösung mit Hilfe von Integraltransformationen
\index{Integraltransformation}
\item Elliptische partielle Differentialgleichungen
\index{Differentialgleichung!partielle!elliptische}
\item Parabolische partielle Differentialgleichungen
\index{Differentialgleichung!partielle!parabolische}
\item Hyperbolische partielle Differentialgleichungen
\index{Differentialgleichung!partielle!hyperbolische}
\item Nichtlineare Phänomene am Beispiel der Gleichung von Burgers.
\index{Differentialgleichung!partielle!nichtlineare}
\end{enumerate}


