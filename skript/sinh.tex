%
% sinh.tex -- warum sinh und cosh für gewöhnliche DGl so nützlich sind
%
% (c) 2015 Prof Dr Andreas Müller, Hochschule Rapperswil
%
\documentclass[a4paper,12pt]{article}
\usepackage{etex}
\usepackage{geometry}
\geometry{papersize={200mm,280mm},total={160mm,240mm},top=21mm}
\usepackage[ngerman]{babel}
\usepackage{times}
\usepackage{amsmath}
\usepackage{amssymb}
\usepackage{amsfonts}
\usepackage{amsthm}
\usepackage{txfonts}
\usepackage{verbatim}
\usepackage{array}
\usepackage{graphicx}
\begin{document}
\title{Hyperbolische Funktionen und Differentialgleichungen}
\author{Andreas M\üller}
\date{}
\maketitle
\begin{abstract}
Das Standardverfahren für die Lösung linearer Differentialgleichungen
mit konstanter Koeffizienten liefert typischerweise Schwingungslösungen
oder exponentiell abfallende oder anwachsende Lösungen. Die Koeffizienten
der allgemeinen Lösungen müssen dann mit Hilfe der Anfangswerte bestimmt
werden. Für die Schwingunsglösungen ist das meist sehr viel einfacher
als für die exponentiellen Lösungen. Hier wird gezeigt, wie man mit
Hilfe der hyperbolischen Funktionen die Lösungen ebenso einfach ausdrücken
kann.
\end{abstract}
\section{Differentialgleichungen zweiter Ordnung}
Das Standardverfahren für die Lösung einer gewöhnlichen linearen
Differentialgleichung mit konstanten Koeffizienten
\[
a_2y''+ a_1y'+a_0y=0
\]
schreibt vor, dass man erst die Nullstellen $\lambda_{1,2}$
des charakteristischen Polynoms
\[
p(\lambda)=a_2\lambda^2+a_1\lambda+a_0
\]
finden muss.
Die allgemeine Lösung der Differentialgleichung ist dann
eine Linearkombination
\[
y(x)=
A_1 e^{\lambda_1t}
+
A_2 e^{\lambda_2t},
\]
die Konstanten $A_1$ und $A_2$ sind aus den Anfangswerten zu bestimmen.
Sinde $y_0$ und $v_0$ der Anfangswert und die Anfangsableitung, dann
findet man das lineare Gleichungssystem
\newcolumntype{\linsysR}{>{$}r<{$}}
\newcolumntype{\linsysL}{>{$}l<{$}}
\newcolumntype{\linsysC}{>{$}c<{$}}
\newenvironment{linsys}[1]{%
\begin{tabular}{*{#1}{\linsysR@{\;}\linsysC}@{\;}\linsysR}}%
{\end{tabular}}
\[
\begin{linsys}{3}
         A_1&+&         A_2&=&y_0\phantom{.}\\
\lambda_1A_1&+&\lambda_2A_2&=&v_0.
\end{linsys}
\]
Besonders einfach wird die Bestimmung jedoch für die Differentialgleichung
\[
y''+k^2 y=0.
\]
Dann sind die $\lambda_{1,2}=\pm\sqrt{-k^2}$ imaginär, und man kann statt
der Exponentiallösungen auch den Ansatz
\[
y(x)=A\cos kx+B\sin kx
\]
verwenden.
Da der Wert von $\sin kx$ bei $x=0$ verschwindet, und die Ableitung von
$\cos kx$ ebenfalls, ist die Bestimmung der Konstanten viel einfacher:
\[
A=y_0
\qquad\text{und}\qquad
B=\frac1kv_0,
\]
oder
\begin{equation}
y(x)=y_0\cos kx + \frac{v_0}{k}\sin kx
\label{hyp:loesung}
\end{equation}

Für die analoge Differentialgleichung $y''-k^2y=0$ geht dies nicht.
Die Nullstellen des charakteristischen Polynoms sind hier 
$\lambda_{1,2}=\pm k$, und es führt nichts an dem linearen Gleichungssystem
\[
\begin{linsys}{3}
 A_1&+& A_2&=&y_0\\
kA_1&-&kA_2&=&v_0
\end{linsys}
\]
vorbei.
Die Lösung kann allerdings zum Beispiel mit dem Determinantenverfahren
ziemlich direkt gefunden werden:
\begin{align*}
A_1
&=
\frac{\left|\begin{matrix}y_0&1\\v_0&-k\end{matrix}\right|}{\left|\begin{matrix}1&1\\k&-k\end{matrix}\right|}
=
\frac{-ky_0+v_0}{-2k},
&
A_2
&=
\frac{\left|\begin{matrix}1&y_0\\k&v_0\end{matrix}\right|}{\left|\begin{matrix}1&1\\k&-k\end{matrix}\right|}
=\frac{v_0-ky_0}{-2k}.
\end{align*}
Damit kann man jetzt die Lösung auch in diesem Fall hinschreiben:
\begin{align}
y(x)
&=
A_1e^{kx}+A_2e^{-kx}
=
\frac12\biggl(
\frac{ky_0-v_0}k e^{kx}
+
\frac{-v_0+ky_0}k e^{-kx}
\biggr)
\notag
\\
&=
y_0\frac{e^{kx}+e^{-kx}}2
+\frac{v_0}{k}\frac{e^{kx}-e^{-kx}}2.
\label{hyp:hyperbelfunktionen}
\end{align}
Die Erfüllung der Anfangsbedingung könnte also auch in diesem Falle
sehr einfach sein, wenn man nicht die Funktionen $e^{\pm kx}$ verwenden
würde, sondern deren Linearkombinationen wie in (\ref{hyp:hyperbelfunktionen}).

\section{Hyperbolische Funktionen}
\begin{figure}
\centering
\includegraphics{../common/images/hf-1.pdf}
\caption{Graphen der Funktionen $x\mapsto\cosh x$ (blau)
und $x\mapsto\sinh x$ (rot)
\label{hyp:graphen}}
\end{figure}
Die hyperbolischen Funktionen sind durch
\[
\sinh x =\frac{e^x-e^{-x}}2
\qquad\text{und}\qquad
\cosh x = \frac{e^x+e^{-x}}2
\]
definiert.
Abbildung~\ref{hyp:graphen} zeigt die Graphen der beiden Funktionen.
Auf den ersten Blick haben diese Definitionen nichts mit den bekannten
trigonometrischen Funktionen zu tun, die Namen sinus hyperbolicus für
$\sinh$ und cosinus hyperbolicus für $\cosh$ scheinen ungerechtfertigt.

\subsection{Komplexe Definition der trigonometrischen Funktionen}
Aus der Euler-Formel
\[
e^{it}=\cos t+i\sin t
\]
lässt sich auch eine Definition der trigonometrischen Funktionen
ableiten. Dazu wendet man die Euler-Formel auf $t$ und $-t$ an:
\begin{align*}
\cos t+i\sin t&=e^{it}\\
\cos t-i\sin t&=e^{-it}
\end{align*}
Dieses Gleichungssystem kann man mit der Additionsmethode nach den
Funktionen $\cos t$ und $\sin t$ auflösen. Man findet:
\begin{align*}
\cos t
&=
\frac{e^{it}+e^{-it}}2
&
\sin t
&=
\frac{e^{it}-e^{-it}}{2i}
\end{align*}
Die trigonometrischen Funktionen können also auf eine Art definiert werden,
die der Definition der hyperbolischen Funktionen völlig analog ist.
Der einzige Unterschied ist ein Faktor $i$ hie und da.
Wir erwarten daher, dass die hyperbolischen Funktionen Eigenschaften
haben, die den Eigenschaften der trigonometrischen Funktionen völlig
analog sind.
Mehr als ein verändertes Vorzeichen hie und da erwarten wir nicht.


\subsection{Geometrie}
\begin{figure}
\centering
\includegraphics{../common/images/hf-2.pdf}
\caption{Die Kurve mit der Parameterdarstellung
$t\mapsto (\cosh t, \sinh t)$ ist eine ``Einheits-Hyperbel'' mit 
Asymptoten $y=\pm x$.
\label{anhang:hyperbel}}
\end{figure}
Die trigonometrischen Funktionen entstanden aus dem Bedürfnis, 
rechtwinklige Dreiecke berechnen zu können.
Abstrahiert man diese bis auf Ähnlichkeit, geht es nur noch um rechtwinklige
Dreiecke, die mindestens eine Seite der Länge $1$ haben.
Diese kann man auf die gewohnte Art am Einheitskreis illustrieren.
Die Beziehung 
\[
\sin^2t+\cos^2t=1
\]
für die trigonometrischen Funktionen erlauben dann, den Einheitskreis als
\[
t\mapsto (\cos t,\sin t)
\]
zu parametrisieren.

Die hyperbolischen Funktionen parametrisieren in der Form
\[
t\mapsto (\cosh t, \sinh t)
\]
natürlich auch eine Kurve in der Ebene.
Um herauszufinden, welcher Kurve das sein könnte, berechnen wir die 
Quadrate der hyperbolischen Funktionen:
\begin{align}
\sinh^2 t&=\frac14(e^{2t}-2+e^{-2t}),
\\
\cosh^2 t&=\frac14(e^{2t}+2+e^{-2t}).
\label{hyp:definition}
\end{align}
Sie stimmen bis auf den mittleren Term überein.
Die Differenz ist 
\[
\cosh^2t - \sinh^2t=1,
\]
die hyperbolischen Funktionen $x=\cosh t$ und $y=\sinh t$
beschreiben also eine Kurve mit der Gleichung
\[
x^2-y^2=1.
\]
Dies ist eine Hyperbel mit Asymptoten $y=\pm x$
(Abbildung~\ref{anhang:hyperbel}).
Damit ist die Bezeichnung als hyperbolische Funktionen gerechtfertigt.

\subsection{Additionstheoreme}
Gibt es auch ein Additionstheorem für die hyperbolischen Funktionen?
Wenn die Analogie zu den trigonometrischen Funktionen durchführbar ist,
dann müssten die Additionstheoreme ungefähr die Form
\begin{align*}
\sinh(a+b)&=\sinh a\cosh b + \cosh a\sinh b,\\
\cosh(a+b)&=\cosh a\cosh b - \sinh a\sinh b
\end{align*}
haben, mit einem abweichenden Vorzeichen hie und da.
Um dies nachzuprüfen, verwenden wir die Definition und rechnen
\begin{align*}
\sinh a\cosh b + \cosh b\sinh b
&=
\frac14(e^a-e^{-a})(e^b+e^{-b})
+
\frac14(e^a+e^{-a})(e^b-e^{-b})
\\
&=\frac14(e^{a+b}+e^{a-b}-e^{-a+b}-e^{-a-b} + e^{a+b}-e^{a-b}+e^{-a+b}-e^{-a-b})
\\
&=
\frac14(2e^{a+b}-2e^{-a-b})
=
\frac12(e^{a+b}-e^{-a-b})=\sinh(a+b),
\\
\cosh a\cosh b-\sinh a\sinh b
&=
\frac14(e^a+e^{-a})(e^b+e^{-b})
-
\frac14(e^a-e^{-a})(e^b-e^{-b})
\\
&=
\frac14(e^{a+b}+e^{a-b}+e^{-a+b}+e^{-a-b})
-
\frac14(e^{a+b}-e^{a-b}-e^{-a+b}+e^{-a-b})
\\
&=
\frac14(2e^{a-b}+2e^{-a+b})
=
\frac12(e^{a-b}+e^{-a+b})=\cosh(a-b)
\end{align*}
aus.
Die Vermutung betreffend der korrekten Form der Additionstheoreme
war also richtig im Falle von $\sinh(a+b)$, für $\cosh(a+b)$ muss
allerdings ein Vorzeichen gewechselt werden.
Die korrekte Form der Additionstheoreme ist
\begin{align*}
\sinh(a\pm b)&=\sinh a\cosh b \pm \cosh a\sinh b,\\
\cosh(a\pm b)&=\cosh a\cosh b \pm \sinh a\sinh b.
\end{align*}

\subsection{Ableitungen}
In der Analysis lernt man, die Ableitungen der trigonometrischen Funktionen
aus den Additionstheoremen abzuleiten.
Da die Additionstheoreme der hyperbolischen Funktionen mit den 
Additionstheoremen bis auf ein Vorzeichen übereinstimmen, sollten
auch die Ableitungsregeln bis auf ein Vorzeichen mit den Ableitungsregeln
der trigonometrischen Funktionen übereinstimmen. 
Natürlich können wir aus der Definition (\ref{hyp:definition}) die
Ableitungen auch direkt berechnen:
\begin{align*}
\frac{d}{dx}\cosh x
&=
\frac12\frac{d}{dx}(e^x+e^{-x})
=
\frac12(e^x-e^{-x})=\sinh x
\\
\frac{d}{dx}\sinh x
&=
\frac12\frac{d}{dx}(e^x-e^{-x})
=
\frac12(e^x+e^{-x})=\cosh x
\end{align*}
Die Ableitungen sind also sogar noch ein bisschen einfacher, da sie sich
schon ab der zweiten Ableitung wiederholen.
Die zweiten Ableitungen sind bereits wieder die ursprünglichen Funktionen
\begin{align*}
\sinh''x&=\sinh x
&
\cosh''x&=\cosh x.
\end{align*}

\subsection{Werte für Argument 0}
Die Werte der trigonometrischen Funktionen und ihrer ersten Ableitungen
im Nullpunkt und die entsprechenden Werte für die hyperbolischen
Funktionen sind ebenfalls völlig analog:
\begin{center}
\begin{tabular}{|l|>{$}c<{$}>{$}c<{$}|>{$}c<{$}>{$}c<{$}|}
\hline
Wert für $x=0$ von&\sin x&\cos x&\sinh x&\cosh x\\
\hline
Funktion           &  0   &  1   &   0   &   1   \\
Ableitung          &  1   &  0   &   1   &   0   \\
\hline
\end{tabular}
\end{center}

\subsection{Lösung von Differentialgleichungen}
Die Werte für Argument $0$ sind genau die Eigenschaften,
welche die Erfüllung der Anfangsbedingung
einer Differentialgleichung so einfach gemacht haben.
Die Differentialgleichung 
\[
y''-k^2y=0
\]
mit Anfangsbedingungen
\[
y(0)=y_0\qquad\text{und}\qquad y'(0)=v_0
\]
hat die Lösung
\[
y(x)=y_0\cosh kx +\frac{v_0}{k}\sinh kx,
\]
in völliger Analogie zu (\ref{hyp:loesung}).

\end{document}
