%
% parabolisch.tex
%
% (c) 2008 Prof Dr Andreas Mueller, Hochschule Rapperswil
% $Id: c05-parabolisch.tex,v 1.3 2008/10/31 08:04:16 afm Exp $
%
\chapter{Parabolische Differentialgleichungen\label{chapter-parabolisch}}
\index{Differentialgleichung!partielle!parabolische}
\lhead{Parabolische PDGL}
\rhead{}
Im vorangegangen Kapitel waren wir in der Lage, mit Hilfe der Greenschen
Funktion eine L"osung eines elliptischen Randwertproblems zu finden.
Dieser L"osungsweg hing vom Maxmimumprinzip ab.
Es garantierte die Eindeutigkeit einer
L"osung und machte damit die Konstruktion der Greenschen Funktion
erst m"oglich.

\index{parabolische partielle Differentialgleichung}
F"ur parabolische Differentialgleichung ist ein analoges Vorgehen
m"oglich, wobei aber immer der Zeitkoordinate (oder der Richtung
des Eigenvektors mit Eigenwert $0$ der Koeffizientenmatrix)
eine besondere Bedeutung zukommt.
Dies wird in den ersten drei
Abschnitten ohne Beweise  oder tiefer gehende Analyse zusammengefasst.

Im Unterschied zu den elliptischen
Gleichungen kann man die W"armeleitunsgleichung aber auch
\index{Warmeleitungsgleichung@W\"armeleitungsgleichung}
als eine ``Zeitentwicklungsgleichung'' verstehen, so wie die
gew"ohnliche Differentialgleichung $\dot x=f(x,t)$
f"ur den Zustandsvektor $x$ die Zeitentwicklung in einem endlichdimensionalen
System beschreibt. Der Definitionsbereich der W"armeleitungsgleichung
ist ja typischerweise von der Form $\Omega\times[0,\infty[$, wobei
$\Omega$ ein Raumgebiet ist.
Die L"osung der W"armeleitungsgleichung ist eine zeitabh"angige Funktion
auf $\Omega$, was zum Beispiel durch einen Separationsansatz
ausgedr"uckt werden kann. Es stellt sich dann heraus, dass die L"osungen
des elliptischen Problems auf $\Omega$ dazu verwendet werden k"onnen,
die W"armeleitungsgleichung zu l"osen.

\section{Problemstellung}
\rhead{Problemstellung}
Sei $\Omega$ ein Gebiet in $\mathbb R^n$. Der Operator 
\[
\partial_t-\kappa\Delta
\]
ist ein parabolischer Operator auf $\mathbb R_+\times \Omega$.
Eine L"osung des W"armeleitungsproblems ist eine Funktion
$u\colon\mathbb R_+\times\Omega\to\mathbb R,$
welche folgende Bedingungen erf"ullt
\begin{align*}
\partial_tu(t,x)-\kappa\Delta u(t,x)&=f(t,x)&&(t,x)\in\mathbb R_+\times\Omega
\\
u(0,x)&=u_0(x)&&x\in\Omega
\\
\alpha u+\beta\frac{\partial u}{\partial n}&=g(t, x)&&x\in\partial\Omega, t>0
\end{align*}
Falls das Gebiet $\Omega$ unbeschr"ankt ist, zum Beispiel $\Omega=\mathbb R^n$,
wird es notwendig sein, zus"atzliche Bedingungen f"ur das Verhalten
der L"osung f"ur $t\to\infty$ hinzuzuf"ugen, damit die L"osungen weiterhin
wohlbestimmt sind.

\section{Maximum-Minimum-Prinzip}
\index{Maximumprinzip}
\rhead{Maximum-Prinzip}
In der Theorie der elliptischen PDGL war die Mittelwerteigenschaft
der Ausgangspunkt f"ur den Beweis des Maximumprinzips. Obwohl
diese Eigenschaft bei den 
L"osungen der W"armeleitungsgleichung nicht mehr erf"ullt ist,
gilt f"ur sie ein Maximum-Minimum-Prinzip.
\begin{satz}[Maximum-Minimum-Prinzip f"ur die W"armeleitungsgleichung]
Sei $\Omega$ ein beschr"anktes Gebiet und
$u$ eine Funktion auf $[0,T]\times\Omega$.
Weiter sei
\begin{align*}
M_{\partial \Omega}&:=\max\{u(t,x)\,|\,x\in\partial\Omega, t\in[0,T]\}
&
m_{\partial \Omega}&:=\min\{u(t,x)\,|\,x\in\partial\Omega, t\in[0,T]\}
\\
M_0&:=
\max\{u(0,x)\,|\,x\in\Omega\}
&
m_0&:=
\min\{u(0,x)\,|\,x\in\Omega\}
\\
M&:=\max\{M_0,M_{\partial\Omega}\}
&
m&:=\min\{M_0,M_{\partial\Omega}\}
\end{align*}
Falls $u$ die homogene
W"armeleitungsgleichung erf"ullt, also $\partial_tu-\kappa\Delta u=0$,
gilt
\[
m\le u(t,x)\le M\quad\forall(t,x)\in[0,T]\times\bar\Omega.
\]
\end{satz}
Wie bei den elliptischen partiellen Differentialgleichungen folgt, dass
die das homogene Anfangs- und Randwert-Problem der W"armeleitungsgleichung
nur eine L"osung hat.

\section{Partikul"are L"osung}
\index{partikul\"are L\"osung}
\rhead{Partikul"are L"osung}
Um das anfangs des Kapitels gestellt Problem zu l"osen, kann man jetzt
gleich vorgehen wie im Fall elliptischer Randwertprobleme. Zun"achst
braucht man singul"are L"osungen, also L"osungen, welche die
W"armeleitungsgleichung mit einer $\delta$-Funktion als rechte Seite
l"osen.
Man kann durch etwas m"uhsame Rechnung nachpr"ufen, dass f"ur $0<\tau<t$
der Ausdruck
\[
\frac1{(4\pi\kappa(t-\tau))^{\frac{n}2}}
\exp\biggl(-\frac{|x-\xi|^2}{4\kappa(t-\tau)}\biggr)
\]
die W"armeleitungsgleichung erf"ullt. Daraus l"asst sich
dann die Funktion
\begin{equation}
K(t-\tau, x-\xi)
=
\vartheta(t-\tau)
\frac1{(4\pi\kappa(t-\tau))^{\frac{n}2}}
\exp\biggl(-\frac{|x-\xi|^2}{4\kappa(t-\tau)}\biggr)
\label{parabolischsingulaer}
\end{equation}
konstruieren, welche die gew"unschten Eigenschaften hat:

\begin{satz}
Die Funktion $K$ erf"ullt die W"armeleitungsgleichung
\begin{align*}
\partial_tK-\kappa\Delta_xK&=0&&(\tau, \xi)\ne(t,x)
\\
\lim_{t\to\tau^+}K(t-\tau, x-\xi)&=\delta(x-\xi).
\end{align*}
Bez"uglich der Koordinaten $(\tau,\xi)$ gilt
\begin{align*}
-\partial_{\tau} K-\kappa\Delta_{\xi}K&=0\qquad(\tau,\xi)\ne(t,x)
\\
\lim_{\tau\to t^-}K(t-\tau, x-\xi)&=\delta(x-\xi)
\end{align*}
\end{satz}

Eine partikul"are L"osung des Problems kann wieder mit Hilfe eines Integrals
gefunden werden:
\begin{align*}
u(t,x)
&=
\int_\Omega\int_0^\infty
K(t-\tau,x-\xi)f(\tau,\xi)
\,d\tau\,d\xi
\\
&=
\int_\Omega\int_0^\infty
\vartheta(t-\tau)\frac1{(4\pi\kappa(t-\tau))^{\frac{n}2}}
\exp\biggl(-\frac{|x-\xi|^2}{4\kappa(t-\tau)}\biggr)
f(\tau,\xi)
\,d\tau\,d\xi
\\
&=
\int_\Omega\int_0^t
\frac1{(4\pi\kappa(t-\tau))^{\frac{n}2}}
\exp\biggl(-\frac{|x-\xi|^2}{4\kappa(t-\tau)}\biggr)
f(\tau,\xi)
\,d\tau\,d\xi
\end{align*}
Setzt man dies in die W"armeleitungsgleichung ein, ergibt sich
\begin{align*}
(\partial_t-\kappa\Delta)u
&=
\int_\Omega\int_0^t
(\partial_t-\kappa\Delta)\biggl[
\frac1{(4\pi\kappa(t-\tau))^{\frac{n}2}}
\exp\biggl(-\frac{|x-\xi|^2}{4\kappa(t-\tau)}\biggr)\biggr]
f(\tau,\xi)
\,d\tau\,d\xi
\\
&\quad+
\lim_{\tau\to t-}
\int_\Omega
\frac1{(4\pi\kappa(t-\tau))^{\frac{n}2}}
\exp\biggl(-\frac{|x-\xi|^2}{4\kappa(t-\tau)}\biggr)
f(t,\xi)
\,d\xi
\end{align*}
Das erste Integral verschwindet, weil der Ausdruck in der eckigen Klammer
f"ur alle $0<\tau<t$ die W"armeleitungsgleichung erf"ullt, also von
$\partial_t-\kappa\Delta$ zu $0$ gemacht wird.
Der Teil des Integranden vor $f(t,\xi)$ im zweiten Integral ist eine
Approximation der $\delta$-Funktion an der Stelle $x$, also ist der Grenzwert
$f(t,x)$. Insgesamt ist also
\[
(\partial_t-\kappa\Delta)u=f,
\]
wie behauptet.

\section{Greensche Funktion}
\index{Greensche Funktion}
\rhead{Greensche Funktion}
Wir wollen jetzt die singul"aren L"osungen $K$ zu einer L"osung des Problems
mit homogenen Randbedingungen erweitern.
Wie beim Laplace-Problem suchen wir jetzt eine Funktion $J(t,x,\tau,\xi)$,
welche wie $K$ die Differentialgleichungen erf"ullt, also
\begin{align*}
\partial_t J-\kappa\Delta_xJ&=0
&
-\partial_\tau J-\kappa\Delta_\xi J&=0
&t>\tau
\\
J&=0&J&=0&t<\tau
\end{align*}
und ausserdem die Randbedingung 
\begin{align*}
J(t,x,\tau,\xi)&=-K(t,x,\tau,\xi)&&x\in\partial\Omega
\end{align*}
Falls dieses Problem immer eine L"osung hat, nennen wir die
Summe
\[
G(t,x,\tau,\xi)=K(t,x,\tau,\xi)+J(t,x,\tau,\xi)
\]
wieder die Greensche Funktion f"ur das W"armeleitungsproblem
mit Dirichlet-Rand\-bedingungen. Und ebenfalls analog wie bei elliptischen
partiellen Differentialgleichungen kann man eine L"osungsformel
angeben
\begin{align}
u(t,x)&=
\int_0^t\int_{\Omega}G(t,x,\tau,\xi)f(\tau,\xi)\,d\xi\,d\tau
\notag
\\
&\quad+\int_{\Omega}G(t,x,0,\xi)u_0(\xi)\,d\xi
\notag
\\
&\quad +\kappa\int_0^t\int_{\partial \Omega}
G(t,x,\tau,\xi)\operatorname{grad}_\xi u
-u\operatorname{grad}_\xi G(t,x,\tau,\xi)\cdot dn\,d\tau
\label{green-parabolisch}
\end{align}

Selbstverst"andlich kann die Theorie wie bei elliptischen Gleichungen
auch f"ur Neumann-Randbedingungen verfeinert werden.

\section{Kausalit"at}
\index{Kausalit\"at}
\rhead{Kausalit"at}
Die singul"are L"osung $K$ zeigt, dass eine "Anderung von $f$ an der Stelle
$(t_0,x_0)$ sich genau auf alle Funktionswerte der L"osung $u(t,x)$ f"ur
$t>t_0$ auswirkt. Eine St"orung zur Zeit $t_0$ breitet sich also sofort
"uber das ganze Gebiet $\Omega$ aus.

\section{Eigenfunktionen und L"osungen der W"armeleitungsgleichung}
\index{Eigenfunktion}
\rhead{Eigenfunktionen}
Im vorangegangenen Kapitel haben wir das Problem
\begin{align*}
\Delta u&=f&&\text{in $\Omega$}\\
u&=g&&\text{auf $\partial\Omega$}
\end{align*}
studiert und dabei festgestellt, dass es "ahnlich wie bei Matrixgleichungen
$Ax=b$ eine Art Inverse gibt, welche durch die Greensche Funktion
bereitgestellt wird. Sogar formal war die L"osung analog einer Matrixgleichung,
die Rolle der Matrix mit zwei Indizes "ubernahm eine Funktion mit zwei
Variablen, statt einer Summe war ein Integral zu bilden.

F"ur eine parabolische PDGL tritt die Zeitentwicklung hinzu. In
ein Matrixproblem "ubersetzt geht es darum, einen zeitlich
ver"anderlichen Vektor $x(t)$ zu finden, dessen Ableitung mit einer
Matrixgleichung $\dot x=Ax$ berechnet werden kann. Wir versuchen daher,
die L"osung dieser Art von Matrix-Differentialgleichung auf das
W"armeleitungsproblem anzuwenden.

\subsection{Systeme gew"ohnlicher DGL erster Ordnung}
Das einer homogenen parabolischen linearen PDGL "aquivalente Problem
ist ein homogenes System
von linearen gew"ohnlichen Differentialgleichungen f"ur eine vektorwertige Funktion
$t\mapsto x(t)\in\mathbb R^n$.
Ein solches System ist gegeben durch eine Matrix $A$ und die Gleichung
\[
\frac{d}{dt}x(t)=Ax(t).
\]
\index{Diagonalform}
H"atte die Matrix $A$ Diagonalform, k"onnte die allgemeine L"osung 
sofort angegeben werden:
\[
A=\begin{pmatrix}
\lambda_1&\dots&0\\
\vdots&\ddots&\vdots\\
0&\dots&\lambda_n
\end{pmatrix}
\qquad\Rightarrow\qquad
x(t)=\begin{pmatrix}
x_1(0)e^{\lambda_1t}
\\
\vdots
\\
x_n(0)e^{\lambda_nt}
\end{pmatrix}.
\]
Unter gewissen Voraussetzungen an die Matrix $A$ kann man $n$
orthonormierte Eigenvektoren
\index{Eigenvektoren!orthonormiert}
$e_1,\dots,e_n$ finden. Nimmt man einen solchen Vektor als
Anfangsbedingung, ist die L"osung $e_ie^{\lambda_it}$. Schreibt man die
Anfangsbedingung in den Eigenvektoren, z.~B.
\[
x(0)=\sum_{i=1}^n(e_i \cdot x(0))e_i,
\]
kann man die L"osung der Differentialgleichung sofort hinschreiben:
\begin{equation}
x(t)=\sum_{i=1}^n
e^{\lambda_i t}
(e_i\cdot x(0))e_i
.
\label{development}
\end{equation}
Tats"achlich ergibt Einsetzen in die beiden Seiten der Differentialgleichung
\begin{align*}
\frac{d}{dt}x(t)&=\sum_{i=1}^n\lambda_ie^{\lambda_i t}(e_i\cdot x(0))e_i
\\
Ax(t)
&=\sum_{i=1}e^{\lambda_it}(e_i\cdot x(0))Ae_i
=\sum_{i=1}e^{\lambda_it}(e_i\cdot x(0))\lambda_i e_i
\end{align*}
Damit haben wir die L"osungen des homogenen Differentialgleichungssystems
gefunden.

\subsection{Variation der Konstanten}
\index{Variation der Konstanten}
Das Verfahren der Variation der Konstanten erlaubt, eine inhomogene
Differentialgleichung erster Ordnung zu l"osen, also
\[
\frac{d}{dt}x(t)-Ax(t)=f(t),
\]
wobei $f$ eine vektorwertige Funktion ist.
Die allgemeine L"osung der homogenen Gleichung war
\[
\sum_{i=0}^nc_ie^{\lambda_i t}e_i,
\]
wobei $c_i=(e_i\cdot x(0))$ war.
Das Verfahren der Variation der Konstanten schreibt vor, die Konstanten
$c_i$ durch Funktionen $c_i(t)$ zu ersetzen,
\[
x(t)=\sum_{i=1}^nc_i(t)e^{\lambda_it}e_i,
\]
und in die Differentialgleichung
einzusetzen, also
\begin{align*}
\sum_{i=1}^n\dot c_i(t)e^{\lambda_it}e_i
+
\sum_{i=1}^n\lambda_i c_i(t)e^{\lambda_it}e_i
-\sum_{i=1}^n\lambda_i c_i(t)e^{\lambda_it}e_i
&=
f(t)
\\
\sum_{i=1}^n\dot c_i(t)e^{\lambda_it}e_i
&=
f(t)
\end{align*}
Bildet man das Skalarprodukt mit $e_i$, ergibt sich 
\[
\dot c_i(t)e^{\lambda_i t}=(e_i\cdot f(t)).
\]
Wir k"urzen die rechte Seite mit $f_i(t)=e_i\cdot f(t)$ ab.
Dann ist
\[
c_i(t)=c_i(0)+\int_0^te^{-\lambda_i \tau}f_i(\tau)\,d\tau
\]
und f"ur die L"osung
\begin{align*}
x(t)&=
\sum_{i=1}^n
(e_i\cdot x(0))e_i+
\sum_{i=1}^ne^{\lambda_i t}\int_0^te^{-\lambda_i \tau}(e_i\cdot f(\tau))e_i\,d\tau
\\
&=
x(0)
+
\int_0^t
\biggl(
\sum_{i=1}^n
e^{\lambda_i(t- \tau)}(e_i\cdot f(\tau))\biggr)e_i\,d\tau
\end{align*}
Man kann dies durch Einsetzen nachpr"ufen:
\begin{align*}
\frac{d}{dt}x(t)
&=
\sum_{i=1}^n\left.e^{-\lambda_i (t-\tau)}(e_i\cdot f(\tau))e_i \right|_{\tau=t}
\\
&=
\sum_{i=1}^n(e_i\cdot f(t))e_i=f(t)
\end{align*}

\subsection{Eigenwerte und Eigenvektoren}
\index{Eigenwert}
\index{Eigenvektor}
Ist das Gebiet $\Omega$ beschr"ankt mit nicht zu ``kompliziertem'' Rand,
kann man f"ur zeigen, dass eine Folge $u_i(x)$
von Eigenfunktionen zu jedem Eigenwert $\lambda_i$ existiert,
\index{Eigenfunktion}
die das Dirichlet-Problem mit homogenen
Randbedingungen l"ost, also
\[
\Delta u_i=\lambda_iu_i,\qquad u_{i|\partial\Omega} = 0.
\]
Ausserdem kann man diese L"osungen so skalieren, dass sie ``Betrag'' 1 haben:
\[
\int_{\Omega}|u_i(\xi)|^2\,d\xi=1
\]
und ausserdem orthogonal aufeinander stehen:
\[
\int_{\Omega}u_i(\xi)u_j(\xi)\,d\xi=0\qquad\forall i\ne j.
\]
\index{orthogonale Funktionen}

Verwendet man $u_i$ als ein Faktor in einem Separationsansatz f"ur
die parabolische Gleichung
\[
\partial_tu=\kappa\Delta u,
\]
erh"alt man f"ur die Gleichung
\begin{align*}
\partial_t (T(t)u_i(t))-\kappa\Delta(T(t)u_i(t))&=0
\\
T'(t)u_i(t)-\kappa T(t)\Delta u_i(t)&=0
\\
T'(t)u_i(t)-\kappa T(t)\lambda_i u_i(t)&=0
\\
\frac{T'(t)}{T(t)}&=\kappa\lambda_i
\\
\Rightarrow\qquad T(t)=Ce^{\kappa\lambda_it}
\end{align*}
F"ur solche Anfangsbedingungen kann also die L"osung sofort angegeben
werden.
\subsection{Die inhomogene Gleichung}
Das Verfahren der Variation der Konstanten kann auch f"ur partielle
Differentialgleichungen durchgef"uhrt werden. Wir setzen die L"osung
an als
\[
u(t,x)=\sum_{i=0}^\infty c_i(t) e^{\kappa\lambda_i t}u_i(x)
\]
Einsetzen in die Differentialgleichung ergibt
\begin{align*}
\sum_{i=0}^\infty \dot c_i(t)e^{\kappa\lambda_it}u_i(x)
+\kappa\sum_{i=0}^\infty c_i(t)\kappa\lambda_i e^{\lambda_it}u_i(x)
-\kappa\sum_{i=0}^\infty c_i(t)e^{\kappa\lambda_it}\lambda_iu_i(x)
&=f(x)
\\
\sum_{i=0}^\infty \dot c_i(t)e^{\kappa\lambda_it}u_i(x)
&=f(t,x)
\end{align*}
Skalarprodukt mit $u_i$ ergibt
\begin{align*}
\dot c_i(t)&= e^{-\kappa\lambda_it}\int_{\Omega}u_i(\xi)f(t,\xi)\,d\xi
\\
c_i(t)&=c_i(0)+\int_0^te^{-\kappa\lambda_i\tau}\int_{\Omega}u_i(\xi)f(\tau,\xi)\,d\xi\,d\tau
\end{align*}
und f"ur die partikul"are L"osung
\begin{align*}
u(t,x)&=
\sum_{i=0}^\infty
u_i(x)
\int_0^t
e^{\kappa\lambda_i(t-\tau)}\int_{\Omega}u_i(\xi)f(\tau,\xi)\,d\xi\,d\tau
\end{align*}
Die Greensche Funktion f"ur das W"armeleitungsproblem mit
Dirichlet-Randbedinungen kann also auch als
\[
G(t,x,\tau,\xi)
=
\sum_{i=0}^\infty
e^{\kappa\lambda_i (t-\tau)}
u_i(x)
u_i(\xi)
\]
geschrieben werden. Die L"osung des parabolischen Problems ist damit
auf die L"osung des elliptischen Eigenwertproblems reduziert worden.


