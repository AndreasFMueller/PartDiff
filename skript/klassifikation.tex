%
% klassifikation.tex
% 
%
% (c) 2008 Prof Dr Andreas Mueller
%
\chapter{Begriffe und Klassifikation}
\lhead{Begriffe und Klassifikation}
Partielle Differentialgleichungen sind Gleichungen f"ur eine
unbekannte Funktion mehrere Variablen, die Werte der Funktion
mit ihren partiellen Ableitungen verkn"upft.
In diesem Kapitel schreiben wir f"ur die gesuchte Funktion immer
$u(x_1,\dots,x_n)$, die Variablen sind $x_1,\dots,x_n$.
Wir schreiben daf"ur auch
\[
u\colon \mathbb R^n\to\mathbb R:(x_1,\dots,x_n)\mapsto u(x_1,\dots,x_n).
\]
Zu kl"aren ist folgendes:
\begin{compactenum}
\item Was genau ist eine partielle Differentialgleichung?
\item Wo ist sie definiert?
\item Wie sind Anfangs- und Randbedingungen zu formulieren?
\item Welche Arten von partiellen Differentialgleichungen gibt es?
\end{compactenum}
Wir schreiben oft auch kompakter $x=(x_1,\dots,x_n)$ f"ur die Punkte
aus $\mathbb R^n$.

\section{Differentialgleichungen\label{klassifikation:differentialgleichungen}}
Eine partielle Differentialgleichung verkn"upft die Funktionswerte
$u(x_1,\dots,x_n)$ mit den Werten der partiellen Ableitungen
\begin{equation}
\frac{\partial u}{\partial x_1},
\frac{\partial u}{\partial x_2},
\dots,
\frac{\partial u}{\partial x_n},
\frac{\partial^2 u}{\partial x_1^2},
\frac{\partial^2 u}{\partial x_1\partial x_2},\dots,
\frac{\partial^2 u}{\partial x_1\partial x_n},\dots,
\frac{\partial^2 u}{\partial x_n^2},
\frac{\partial^3 u}{\partial x_1^3},\dots,
\frac{\partial^3 u}{\partial x_{i_1}\partial x_{i_2}\partial x_{i_3}},\dots
\label{ableitungen}
\end{equation}
W"ahrend es bei einer Funktion von einer Variablen nur eine Ableitung
jeder Ordnung gibt, hat eine Funktion von $n$ Variablen $n^k$ Ableitungen
der Ordnung $k$.

Eine Differntialgleichung liegt vor, wenn bekannt ist, wie verschiedenen
Ableitungen miteinander verkn"upft werden m"ussen.
Die Wellengleichung hatte zum Beispiel die Form
\begin{equation}
\frac{\partial^2 u}{\partial x_1^2}
-
c^2\frac{\partial^2 u}{\partial x_2^2},
\label{wellengleichung-tform}
\end{equation}
es werden also nur die beiden zweiten Ableitungen
\[
\frac{\partial^2 u}{\partial x_1^2}
\qquad
\text{und}
\qquad
\frac{\partial^2 u}{\partial x_2^2}
\]
miteinander vernk"upft, die anderen Ableitungen der Liste
(\ref{ableitungen}) kommen gar nicht vor.
Schreiben wir
\[
F(t_{11}, t_{22}) = t_{11} -c^2t_{22},
\]
dann bekommt die Wellengleichung (\ref{wellengleichung-tform})
die Form
\[
F\biggl(
\frac{\partial^2 u}{\partial x_1^2},
\frac{\partial^2 u}{\partial x_2^2}
\biggr)=0.
\]
Die Funktion $F$ k"onnte allerdings noch beliebig viel komplizierter
ausfallen.
Sie k"onnte zum Beispiel auch noch von den Variablen $x_1,\dots,x_n$,
den Funktionswerten $u(x_1,\dots,x_n)$ und von weiteren Ableitungen
abh"angen.
Und die Abh"angigkeit von den Ableitungen k"onnte komplizierter sein,
nicht nur linear wie in diesem Beispiel.

Allgemein ist eine partielle Differentialgleichung gegeben
durch eine Funktion
\[
F(x_1,\dots,x_n,u,\dots\text{Variablen f"ur partielle Ableitungen von $u$}\dots).
\]
Die Differentialgleichung
entsteht dadurch, dass man die unbekannte Funktion
und ihre partiellen Ableitungen in die Funktion $F$ einsetzt und
gleich $0$ setzt:
\[
F\biggl(x_1,\dots,x_n,u(x_1,\dots,x_n),\dots,
\frac{\partial^k u}{\partial x_{i_1}\partial x_{i_2}\dots \partial x_{i_k}},\dots\biggr)=0.
\]
Oft verwendete Bezeichnungen f"ur die Variablen f"ur die Ableitungen sind
$p_i$ f"ur die ersten Ableitungen und $t_{ij}$ f"ur die zweiten Ableitungen.


\subsection{Ordnung\label{klassifikation:ordnung}}
\index{Ordnung}
\index{Differntialgleichung!gew\"ohnliche}
Wie bei gew"ohnlichen Differentialgleichungen ist die Ordnung
die h"ochste Ableitung der unbekannten Funktion, die in der
Differentialgleichung vorkommt. Wir wollen dies etwas ausf"uhrlicher
diskutieren f"ur partielle Differentialgleichungen f"ur eine Funktion
$u$ in den Variablen $x_1,\dots,x_n$.

\subsubsection{Partielle Differentialgleichung erster Ordnung}
In einer PDGL erster Ordnung kommen nur erste Ableitungen der
unbekannten Funktion vor.
Sie l"asst sich also in der Form 
\[
F\biggl(x_1,\dots,x_n, u, \frac{\partial u}{\partial x_1},\dots,\frac{\partial u}{\partial x_n}\biggr)=0
\]
schreiben. Eine PDGL erster Ordnung ist also gleichbedeutend mit einer
Funktion 
$F(x_1,\dots,x_n,u,p_1,\dots,p_n),$
durch die Substitution
\[
p_i\to \frac{\partial u}{\partial x_i}
\]
wird daraus die Differentialgleichung.

Bei PDGL erster Ordnung in zwei Variablen schreibt man auch
$F(x,y,u,p,q)$, ersetzt also
\[
p\to\frac{\partial u}{\partial x},
\qquad
q\to\frac{\partial u}{\partial y}.
\]

\subsubsection{Partielle Differentialgleichung zweiter Ordnung}
Wie das erste Kapitel gezeigt hat, sind die PDGL zweiter Ordnung bei weitem
die wichtigsten. Eine solche PDGL enth"alt neben den ersten Ableitungen
auch alle zweiten Ableitungen. Da die L"osungsfunktion stetig differenzierbar
sein wird, h"angen die gemischten Ableitungen nicht von der Reihenfolge
ab, also
\[
\frac{\partial^2 u}{\partial x_i\partial x_j}
=
\frac{\partial^2 u}{\partial x_j\partial x_i}
\quad\forall i,j
\]
Die Differentialgleichung hat die Form
\[
F\biggl(x_1,\dots,x_n,u,
\frac{\partial u}{\partial x_1},\dots,\frac{\partial u}{\partial x_n},
\frac{\partial^2 u}{\partial x_1^2},\dots,\frac{\partial^2 u}{\partial x_n^2}\biggr)
\]
mit einer Funktion
\[
F(x_1,\dots,x_n,u,p_1,\dots,p_n,t_{11},t_{12},\dots,t_{n,n-1},t_{nn})
\]
in den Variablen $x_i$, $u$, $p_i$ und $t_{ij}$. Mit der Substitution
\[
p_i\to \frac{\partial u}{\partial x_i}
,\quad
t_{ij}\to \frac{\partial^2 u}{\partial x_i\partial x_j}
\]
wird daraus die Differentialgleichungen.

Die im ersten Kapitel diskutierten Differentialgleichungen entsprechen den folgenden 
Funktionen
\begin{align*}
F(t_{11},\dots,t_{nn})&=t_{11}-a^2(t_{22}+\dots+t_{nn})&&\text{Wellengleichung}
\\
F(p_1,t_{22},\dots,t_{nn})&=p_1-a^2(t_{22}+\dots+t_{nn})&&\text{W"armeleitung}
\\
F(t_{11},\dots,t_{nn})&=t_{11}+\dots+t_{nn}&&\text{Poisson-Problem}
\end{align*}

\subsection{*Multiindizes und h"ohere partielle Ableitungen\label{klassifikation:multiindizes}}
Im vorangegangenen Abschnitt wurden gemischte partielle Ableitungen
nach verschiedenen unabh"angigen Variablen ben"otigt. Die Notation
\[
\frac{\partial^k u}{\partial x_{i_1}\partial x_{i_2}\dots\partial x_{i_k}}
\]
wirkt etwas schwerf"allig. In dem Formelzeichen steckt ja eigentlich
nur die Information, dass nach den Variablen abgeleitet werden
soll, die die Nummern $i_1,\dots,i_k$ haben, und dabei kommt es
nicht einmal auf deren Reihenfolge an, nur die Anzahl Ableitungen
f"ur jede Variable ist wesentlich.
\begin{definition}
Wir nennen ${\bf k}=(k_1,\dots,k_n)$ einen Multiindex der L"ange $n$.
Der Grad des Multiindex ist $|{\bf k}|=k_1+\dots+k_n$.
\end{definition}
Mit Multiindizes kann man Terme, die in der mehrdimensionalen
Analysis oft vorkommen, deutlich kompakter schreiben:
\begin{align*}
x^{\mathbf k}&=x_1^{k_1}x_2^{k_2}\dots x_n^{k_n}\\
\partial_{\mathbf k}u
&=\frac{\partial^{k_1}}{\partial x_1^{k_1}}\dots
\frac{\partial^{k_n}}{\partial x_n^{k_n}}u
=\frac{\partial^{|{\mathbf k}|}}{\partial x_1^{k_1}\dots\partial x_n^{k_n}}u
=D^{\mathbf k}u=D_1^{k_1}D_2^{k_2}\dots D_n^{k_n}u
\end{align*}
Nat"urlich kann man Multiindizes auch dazu verwenden, andere 
Objekte zu indizieren. Zum Beispiel kann man eine mehrfache Potenzreihe
jetzt so schreiben:
\[
f(x_1,\dots,x_n)=\sum_{\mathbf k}a_{\mathbf k}x^{\mathbf k},
\]
wobei die $a_{\mathbf k}\in\mathbb R$ die Koeffizienten der Potenzreihe sind.

Eine partielle Differentialgleichung $m$-ter Ordnung mit $n$ unabh"angigen 
Variable ist jetzt eine Funktion
\begin{align*}
F&\colon \mathbb R^n\times \mathbb R^{\{{\mathbf k}|\,|{\mathbf k}|\le m\}} \to \mathbb R
\\
&\colon(x_1,\dots,x_n,u, \dots)\mapsto F(x_1,\dots,x_n,u,\dots)
\end{align*}
Die Funktion hat ausser den Argumenten $x_1,\dots,x_n$ Argumente,
die mit den Multiindizes angeschrieben sind. Der kleinste Multiindex
ist $(0,\dots,0)$, er steht f"ur die Funktion $u$. In das Argument mit
dem Multiindex ${\mathbf k}$ muss die Ableitung von $u$ nach diesem
Multiindex eingesetzt werden. Die ersten Multiindizes sind
\[
(1,0,\dots,0), (0,1,\dots,0),\dots, (0,\dots, 0,1),
\]
die zugeh"origen Ableitungen sind die
ersten partiellen Ableitungen:
\[
F(x_1,\dots,x_n,u,\partial_1 u,\dots,\partial_2 u,\dots)=0.
\]

\subsection{Umwandlung in ein System niedrigerer Ordnung\label{klassifikation:umwandlung}}
Wie bei gew"ohnlichen Differentialgleichungen kann man auch eine
partielle Differentialgleichung h"oherer Ordnung in ein System
von partiellen Differentialgleichungen niedrigerer Ordnung
umwandeln.

Bei einer gew"ohnlichen Differentialgleichung 
\[
F(x,y,y',y'',\dots,y^{(n)})=0
\]
erreicht man dies, indem man statt nur einer Funktion $y$ einen
Vektor von Funktionen $y_0,\dots,y_{n-1}$ sucht, mit den Gleichungen
\begin{align*}
\frac{dy_0}{dx}&=y_1\\
\frac{dy_1}{dx}&=y_2\\
&\vdots\\
\frac{dy_{n-2}}{dx}&=y_{n-1}\\
F\biggl(x,y_0,y_1,y_2,\dots,\frac{dy_{n-1}}{dx}\biggr)&=0
\end{align*}
Setzt man in diesen Gleichungen $y_0=y$, findet man nacheinander
$y_k=y^{(k)}$, und aus der letzten Gleichung wieder
$F(x,y,y',y'',\dots,y^{(n)})=0$. Aus der Differentialgleichung $n$-ter
Ordnung ist eine System von $n$ Differentialgleichungen erster Ordnung
geworden.

Sei jetzt eine partielle Differentialgleichung zweiter Ordnung mit
zwei unabh"angigen Variablen $x$ und $y$ gegeben, also eine 
Gleichung der Form
\[
F\biggl(x,y,u,\frac{\partial u}{\partial x},\frac{\partial u}{\partial y},
\frac{\partial^2 u}{\partial x^2},\frac{\partial^2 u}{\partial x\partial y},
\frac{\partial^2u}{\partial y^2}\biggr)=0.
\]
Nach dem Muster der gew"ohnlichen Differentialgleichung brauchen wir
jetzt neue Funktionen, die f"ur die Ableitungen stehen, also zum
Beispiel $p(x,y)$ und $q(x,y)$. Wir m"ussen ausdr"ucken, dass diese
Funktionen Ableitungen von $u$ sind, das liefert uns zwei neue
partielle Differentialgleichungen
\[
p=\frac{\partial u}{\partial x},\qquad q=\frac{\partial u}{\partial y}.
\]
Auch die zweiten Ableitungen k"onnen wir damit ausdr"ucken:
\begin{align*}
\frac{\partial^2 u}{\partial x^2}&=\frac{\partial p}{\partial x}\\
\frac{\partial^2 u}{\partial x\partial y}&=\frac{\partial p}{\partial y}=\frac{\partial q}{\partial x}\\
\frac{\partial^2 u}{\partial y^2}&=\frac{\partial q}{\partial y}
\end{align*}
Die urspr"ungliche Differentialgleichung allein reicht jetzt
nat"urlich nicht mehr, wir haben jetzt ein Differentialgleichungssystem
erster Ordnung
\begin{align*}
F\biggl(x,y,u,p,q,\frac{\partial p}{\partial x},\frac{\partial p}{\partial y},\frac{\partial q}{\partial y}\biggr)&=0\\
p&=\frac{\partial u}{\partial x}\\
q&=\frac{\partial u}{\partial y}\\
\frac{\partial p}{\partial y}&=\frac{\partial q}{\partial x}
\end{align*}
Wir haben also ein partielle Differentialgleichung zweiter Ordnung auf
ein System von partiellen Differentialgleichungen erster Ordnung f"ur
drei unbekannte Funktionen reduziert.

\subsection{*Umwandlung einer Gleichung beliebiger Ordnung\label{klassifikation:beliebigeordnung}}
Sei jetzt eine partielle Differentialgleichung $m$-ter Ordnung
gegeben. Sie hat die Gestalt
\[
F\bigl(
x_1,\dots,x_n, u, \frac{\partial u}{\partial x_1},\dots,\frac{\partial u}{\partial x_n},\frac{\partial^2u}{\partial x_1^2},\frac{\partial^2 u}{\partial x_1\partial x_2},\dots,\frac{\partial^2u}{\partial x_n^2},\dots
\bigr)=0
\]
wobei als Argumente Ableitungen mit Multiindizes ${\mathbf k}$ mit
$|{\mathbf k}|\le m$ vorkommen. Daraus kann man jetzt ein System von
partiellen Differentialgleichungen machen, indem man f"ur jeden
Multiindex $|{\mathbf k}|$ mit $|{\mathbf k}|\le m$ eine zus"atzliche
Funktion $p_{\mathbf k}$ einf"uhren. In die Funktion k"onnen wir
dann statt der Ableitungen von $u$ die Funktionen $p_{\mathbf k}$
einsetzen.
\[
F\bigl(
x_1,\dots,x_n, u, p_1,\dots,p_n,p_{(2,0,\dots)},p_{(1,1,\dots)},
\dots,p_{(0,\dots,0,2)},\dots, \frac{\partial p_{\mathbf k}}{\partial x_i})
\bigr)=0,
\]
wobei nur Ableitungen erster Ordnung vorkommen, also nur Ableitungsterme
mit $|{\mathbf k}|\le m$.
Dazu kommt aber eine ganze Menge von neuen Gleichungen,
welche sagen, dass die die $p_{\mathbf k}$ eigentlich Ableitungen sind:
\[
\frac{\partial}{\partial x_i}p_{\mathbf k}=p_{{\mathbf k} + (0,\dots,1,\dots,0)}\quad\forall i\;\forall{\mathbf k}(|{\mathbf k}|<m),
\]
wobei die $1$ an der $i$-ten Stelle steht. Dazu kommen Gleichungen
die besagen, dass die gemischten Ableitungen vertauscht werden k"onnen.
Sei ${\mathbf k}$ ein Multiindex, und sei ${\mathbf k}'$ ein Multiindex,
aus dem ${\mathbf k}$ entsteht, wenn man an der Stelle $i$ eins addiert.
Ebenso sei ${\mathbf k}''$ ein Multiindex, aus dem ${\mathbf k}$ entsteht,
wenn man an der Stellen $j$ eins addiert. F"ur jede solche Konstellation
erh"alt man eine zus"atzliche Gleichung:
\[
\frac{\partial}{\partial x_i}p_{{\mathbf k}'}=\frac{\partial}{\partial x_j}p_{{\mathbf k}''},
\]
denn beide Terme stehen ja eigentlich f"ur $\partial_{\mathbf k}u$.
 
\section{Gebiete\label{klassifikation:gebiete}}
Bei gew"ohnlichen Differentialgleichungen ist eine
Funktion von nur einer Variablen gesucht, das Definitionsgebiet ist
daher immer ein Interval ist.
Eine L"osung ist dann festgelegt durch die Differentialgleichung
und bestimmte Werte an einem (Anfangswerte) oder beiden (Randwertproblem)
Enden des Definitionsgebietes.

Bei partiellen Differentialgleichungen kommen fast
beliebige Teilmengen des $\mathbb R^n$ als Definitionsbereich
in Frage. Und die einfache Unterscheidung zwischen Randwertproblem
und Anfangswertproblem wird ebenfalls verwischt, es ist nur klar,
dass wahrscheinlich auf Teilen des Randes des Definitionsgebietes
Werte der Funktion und eventuell gewisser Ableitungen
vorgegeben werden m"ussen, um die L"osungen der Differentialgleichung
eindeutig festzulegen.

Die partiellen Ableitungen einer Funktion $u$
in einem Punkt $(x_1,\dots,x_n)$ 
sind nur dann sinnvoll definiert, wenn die Funktion in einer 
kleinen Umgebung von $(x_1,\dots x_n)$ ausgewertet werden kann.
Es muss ja f"ur die partielle Ableitung nach $x_i$ der Grenzwert
\[
\frac{\partial u}{\partial x_i}=
\lim_{\Delta x_i\to 0}\frac{u(x_1,\dots,x_i+\Delta x_i, \dots ,x_n)-u(x_1,\dots,x_n)}{\Delta x_i}
\]
bestimmt werden k"onnen.

\begin{definition}
Ein Gebiet $\Omega\subset \mathbb R^n$ ist eine offene Teilmenge
von $\mathbb R^n$. Mit jedem Punkt $(x_1,\dots,x_n)\in\Omega$ ist
auch ein kleiner Ball
\[
B(x, r)=\{x'\in\mathbb R^n\,|\,|x-x'|<r\}.
\]
in $\Omega$ enthalten, wenn nur $r$ klein genug aber $r>0$ gew"ahlt wird:
$B(x,r)\subset\Omega$.
\end{definition}

\begin{beispiel}
Die Menge 
\[
\Omega_1=\{ (x,y)\in\mathbb R^2\,|\, 0 < x < 1, 0<y<1\}
\]
ist ein Gebiet. Ein Punkt $(x,y)\in\Omega_1$ kann nicht auf dem
Rand liegen. Ist $r$ der kleinste der Werte $x$, $y$, $1-x$ und $1-y$,
dann ist auch die Kreisscheibe $B(x,r)\subset\Omega_1$.
\end{beispiel}

\begin{beispiel}
Die Menge 
\[
\Omega_2 = \{ (x,y)\in\mathbb R^2\,|\, x\ge 0\}.
\]
ist {\it kein} Gebiet. $\Omega_2$ ist die rechte Halbebene in $\mathbb R^2$.
$\Omega_2$ enth"alt auch die Punkte $(0,y)$. Jeder Ball um einen
Punkt der Form $(0,y)$ enth"alt auch Punkte mit negativer
$x$-Koordinate, also Punkte, die nicht zu $\Omega_2$ geh"oren.
Somit kann ein Ball um $(0,y)$ nicht in $\Omega_2$ enthalten sein.
\end{beispiel}

Der Unterschied zwischen den beiden Mengen $\Omega_1$ und $\Omega_2$
ist offenbar, dass $\Omega_2$ offenbar auch Teile des Randes
enth"alt. Der Rand einer Menge besteht aus den Punkten, in deren
Umgebung sich immer Punkte der Menge wie auch Punkte ausserhalb
der Menge finden. Der Rand der Menge $\Omega_1$ ist leer, der Rand
der Menge $\Omega_2$ ist die $y$-Achse. Gebiete sind also Mengen,
die ihren Rand nicht enthalten. F"ur den Rand einer Menge $M$
schreibt man auch $\partial M$.

\begin{beispiel}
\[
\partial \Omega_1=\emptyset,\qquad\partial \Omega_2=\{(0,y)\,|y\in\mathbb R\}.
\]
\end{beispiel}

\begin{definition}
Ist $M\subset\mathbb R^n$ eine beliebige Teilmenge von $\mathbb R^n$,
dann bezeichnet man mit $\mathring M$ das Innere, also die Punkte,
f"ur die auch noch eine Umgebung in $M$ enthalten ist. Mit
$\bar M=M\cup\partial M$ bezeichnet.
\end{definition}

\section{Anfangs- und Randbedingungen\label{klassifikation:randbedingungen}}
\rhead{Anfangs- und Randbedingungen}
Wie bei gew"ohnlichen Differentialgleichungen reicht auch bei
partiellen Differentialgleichungen die Gleichung alleine nicht
zur Bestimmung der L"osung.
\subsection{Anfangs- und Randwerte bei gew"ohnlichen Differentialgleichungen\label{klassifkation:anfangswerte-ode}}
Bei gew"ohnlichen Differentialgleichungen
m"ussen zus"atzlich Anfangs- oder Randwerte spezifiziert werden.
Um die L"osung der Differentialgleichung
\[
y''+p(x)y'+q(x)y=0
\]
auf dem Intervall $[0,1]$
festzulegen, braucht man insgesamt zwei Bedingungen, zum Beispiel
\begin{itemize}
\item die Werte von $y(0)$ und $y'(0)$
\item die Werte von $y(0)$ und $y(1)$
\item Zwei lineare Gleichungen der Form
\begin{align*}
\alpha_0y(0)+\beta_0y'(0)&=\gamma_0\\
\alpha_1y(1)+\beta_1y'(1)&=\gamma_1.
\end{align*}
%wobei die Koeffizientenmatrix auf der linken Seite regul"ar sein muss.
\end{itemize}

\begin{beispiel}
Selbst in dem einfachen Beispiel einer linearen Differentialgleichunge
zweiter Ordnung ist nicht jede beliebige Kombination von Randwertvorgaben
m"oglich. Die Gleichung
\[
y''=0
\]
bedeutet zum Beispiel, dass die Funktion $y$ keine Kr"ummung hat, sie
ist also linear, $y=Ax+B$. Insbesondere muss die Steigung am rechten
Rand des Intervals gleich gross sein wie am linken Rand. Versucht man
also die Randwerte der Ableitungen als
\[
y'(0)=0\qquad y'(1)=1,
\]
erh"alt man ein unl"osbares Problem, die Randwerte sind widerspr"uchlich.
\end{beispiel}

\subsection{Randwerte f"ur partielle Differentialgleichungen\label{klassifikation:randwerte-pde}}
Bei einer partiellen Differentialgleichung ist die Spezifikation
der Randwerte ungleich komplizierter.
\subsubsection{Randwerte f"ur die Saite}
Um ein Gef"uhl zu entwickeln, betrachten wir die n"otigen
Randwertvorgaben f"ur die schwingende Saite.
\begin{figure}
\begin{center}
\includegraphics{images/randwerte-1.pdf}
\end{center}
\caption{Randwerte f"ur die Wellengleichung der schwingenden Saite.
Das Gebiet ist hellrot eingezeichnet, sein Rand tiefrot, mit den 
notwendigen Randwerten entlang des Randes\label{klassifikation:randwertesaite}}
\end{figure}
In Abbildung \ref{klassifikation:randwertesaite} ist das Gebiet f"ur
die Saitengleichung hellrot gezeichnet. Randwerte k"onnen auf dem
linken und/oder rechten Rand vorgegeben werden und/oder auf dem
Intervall $[0,l]$ der $x$-Achse.

Aus der Theorie der gew"ohnlichen Differentialgleichungen wissen wir,
dass f"ur eine Differentialgleichung zweiter Ordnung zwei Randwertvorgaben
gemacht werden m"ussen.
Man kann die Funktionswerte an jedem Ende des
Intervals angeben, oder man kann an einem Ende den Funktionswert
sowie die erste Ableitung angeben.

Genau dies wird bei der Saitengleichung gemacht. Vernachl"assigt man
die $t$-Abh"angigkeit, sieht man nur noch eine Differentialgleichung
zweiter Ordnung in $x$ und die Gleichungen 
\[
u(0,t)=0\qquad u(l,t)=0
\]
geben die Randwerte f"ur diese Differentialgleichung vor.
Schaut man dagegen nur die $t$-Abh"angigkeit an,
sieht man eine Differentialgleichung
in der Variablen $t$ auf der positiven reellen Achse, es bleibt also
nichts anderes "ubrig, als Anfangswert und Anfangsableitung (nach $t$)
vorzugeben.

Im Prinzip gibt es mehrere erste Ableitungen, die vorzugeben versuchen 
k"onnten. Wenn aber die Randwerte 
$ u(x,0)=f(x)$
bekannt sind, dann sind auch die partiellen Ableitungen nach $x$ bereits
bekannt:
\[
\frac{\partial u}{\partial x}u(x,0)=f'(x).
\]
Man kann also h"ochstens noch die Ableitung senkrecht dazu,
in $t$-Richtung vorgeben.

Es stellt sich heraus, dass mit diesen Vorgaben das Problem eindeutig
l"osbar ist.
Leider ist es im Allgemeinen deutlich schwieriger, welche Randwertvorgaben
n"otig sind, um die L"osung eindeutig zu bestimmen, oder auch nur
widerspruchsfreie Randwertvorgaben zu machen.

\subsubsection{Allgemeine Diskussion}
Zun"achst ist nicht einmal klar, welcher Teil des Randes
"uberhaupt herangezogen werden muss. Im Prinzip k"onnen Randwerte
auf jeder beliebigen Teilmenge des Randes spezifiziert werden.

Ausserdem k"onnen nicht nur Werte der unbekannten Funktion, sondern auch
partielle Ableitungen derselben vorgegben werden. Allerdings sind 
diese nicht beliebig. Betrachten Wir dazu die rechte Halbebene
\[
\Omega=\{(x,y)\in\mathbb R^2\,|\,x>0\}.
\]
Sei $u(x,y)$ eine Funktion, die in $\bar\Omega$ definiert ist.
Sind die Werte von $u$ entlang des Randes $\partial\Omega$ vorgegeben,
dann sind offenbar auch alle Ableitungen in Richtungen tangential
an den Rand vorgeben:
\[
\text{
$u(0,y)$ bekannt
}\qquad\Rightarrow\qquad
\text{$\frac{\partial u}{\partial y}$ bekannt.}
\]
Man kann also sinnvollerweise nur noch die Ableitungen
in Richtungen senkrecht auf den Rand vorgeben.

Betrachten wir jetzt den allgemeine Fall: sei die Funktion $u$
in einem Punkt $x=(x_1,\dots,x_k)\in\mathbb R^k$ des Randes
eines Gebietes $\Omega\subset\mathbb R^k$ definiert.
Wir nehmen an, dass der Rand in der Umgebung des Punktes $x$
eine wohldefinierte Tangentialebene mit der Normalen $\vec n$ hat.
Da die Werte von $u$ entlang des Randes bekannt sind, sind 
auch die Richtungsableitungen von $u$ in Richtungen $\vec v$
senkrecht auf $\vec n$ bekannt, man berechnet sie mit dem
Gradienten:
\[
D_{\vec v}u(x_1,\dots,x_k)=\vec v\cdot\operatorname{grad}u.
\]
Nur die Richtungsableitung in Richtung der Normalen 
ist noch nicht festgelegt.
Wir schreiben f"ur die Ableitung in Richtung $\vec n$
\[
\frac{\partial u}{\partial\vec n}=\frac{\partial u}{\partial n}
=\vec n\cdot\operatorname{grad}u.
\]

\begin{beispiel}
Sei $\Omega=\{(x,y)\in\mathbb R^2\,|\,x^2+y^2<1\}$ die Einheitskreisscheibe.
Man findet die Normalableitung der Funktion $u(x,y)=x^3y$ in jedem
Punkt des Einheitskreises.

Die nach aussen gereichtete Normale im Punkt $(x,y)$ ist
\[
\vec n=\begin{pmatrix}x\\y\end{pmatrix}.
\]
Die Normalableitung einer Funktion $u(x,y)$ ist dann
\[
\frac{\partial u}{\partial n}=
x\frac{\partial u}{\partial x}
+
y\frac{\partial u}{\partial y}.
\]
Oder f"ur die spezielle Funktion $u(x,y)=x^3y$:
\[
\frac{\partial u}{\partial n}
=
x\cdot 3x^2y+y\cdot x^3=
4x^3y.
\]
\end{beispiel}

\begin{beispiel}
Man bestimme die Normalableitung der Funktion $x^2-y^2$ auf dem Rand
des Gebietes $\Omega=\{(x,y)\in\mathbb R^2\,|\,y+x>0\}$.

Das Gebiet wird durch die Gerade $y=-x$ berandet, welche als Normale
den Vektor 
\[
\vec n=\begin{pmatrix}1\\1\end{pmatrix}
\]
hat. Die Richtungsableitung von $u$ im Punkt $(x,y)$ in Richtung $\vec n$
ist
\[
D_{\vec n}u(x,y)=\vec n\cdot\operatorname{grad}u
=\begin{pmatrix}1\\1\end{pmatrix}\cdot\begin{pmatrix}2x\\-2y\end{pmatrix}
=2x-2y
\]
In einem Punkt $(x,-x)$ des Randes ist die Normalableitung also
\[
\frac{\partial u}{\partial n}
=
2x+2x=4x.
\]
\end{beispiel}

\subsection{Spezielle Randwertvorgaben\label{klassifikation:randwerte-speziell}}
\subsubsection{Das Cauchy-Problem\label{klassifikation:cauchy-problem}}
Der Rand dieses Gebietes
besteht nicht mehr nur aus einzelnen Punkten wie im eindimensionalen
Fall, sondern aus Kurven oder Fl"achen. Um zu verstehen, wie die
Randbedingungen entlang dieser Kurven oder Fl"achen formuliert
werden k"onnen, betrachten
wir das sogenannte Cauchy-Problem f"ur eine Funktion $u(x,y)$ von
zwei Variablen. Wir suchen eine L"osung, welche wir durch Angabe
von Randwerten auf f"ur $x=0$ ($y$-Achse) festlegen m"ochten.

Nat"urlich l"asst sich das Cauchy-Problem auch f"ur PDGL mit mehr
als zwei Variablen formulieren. In diesem Fall wird eine Funktion
$u(x_1,\dots,x_n)$ gesucht, welche f"ur $x_1=0$ vorgegebene Werte
\[
u(0,x_2,\dots,x_n)=g(x_2,\dots,x_n)
\]
annimmt.
In diesem Fall sind die Werte von $u$ auf einer Koordinaten(hyper)ebene
vorgegeben worden.

Das allgmeine Cauchy-Problem in zwei Dimensionen legt die Werte der
unbekannten Funktion auf einer
Kurve im Definitionsgebiet fest und fragt danach, ob eine L"osung existiert, und
ob sie eindeutig bestimmt ist.
Das allgemeine Cauchy-Problem in beleibig vielen Dimensionen legt die
Werte der unbekannten Funktion auf einer Hyperfl"ache im Definitionsgebiet
fest und frag danach, ob eine L"osung existeirt,
und ob sie eindeutigt bestimmt ist.

\subsubsection{Partielle Differentialgleichungen erster Ordnung}
Eine gew"ohnliche Differentialgleichungen erster Ordnung ben"otigt
nur einen einzigen Anfangswert.
In Analogie dazu werden in unserem Beispielproblem 
die Werte von $u$ entlang der $y$-Achse erforderlich sein.
Wir nehmen daher an, dass die Bedingung
\[
u(0,y)=g(y)
\]
die L"osung $u$ festlegt.

Wir nehmen an, dass die Funktion $g$ stetig differenziertbar ist.
Dann ist
\[
\frac{\partial u}{\partial y}(0, y)=g'(y),
\]
die partielle Ableitung nach $y$ von $u$ ist also entlang der
Geraden $x=0$ auch schon festgelegt. 

Eine Differentialgleichung erster Ordnung legt eine Beziehung
der Form
\[
F(x,y,u(x,y), \partial_x u, \partial_y u)=0
\]
zwischen der Funktion und ihren partiellen Ableitungen fest.
Zwar gilt dies im Allgemeinen nur im inneren des Definitionsbereiches,
doch
unter g"unstigen Voraussetzungen wird die Funktion $F(x,y,u,p,q)$
nach $p$ aufl"osbar sein, und zwar sogar auf dem Rand, also f"ur
$x\to 0$. Gibt es eine L"osung $u(x,y)$ des Cauchy-Problems, dann
ist 
\begin{align*}
0&=
F(x,y,u(x,y), \partial_x u(x,y), \partial_y u(x,y)
\\
\Rightarrow
0&=
\lim_{x\to 0}
F(x,y,u(x,y), \partial_x u(x,y), \partial_y u(x,y)\\
&=F(0,y,u(x,y),\partial_x u(0,y), \partial_y u(0,y))
\\
&=
F(0,y,g(y),\partial_x u(0,y), g'(y))
\end{align*}
Da wir von der Funktion $F$ angenommen haben, dass sie sich
nach $p$ aufl"osen l"asst, ist also $\partial_x u(0,y)$ durch
die Anfangsbedingungen und die Gleichung ebenfalls festgelegt.

Das Cauchy-Problem f"ur eine Differentialgleichung erster Ordnung
muss als nur Anfangswerte entlang einer Geraden vorgeben, Ableitungen
sind nicht n"otig, da sie durch die Funktionswerte und die Gleichung
bereits festgelegt sind.

Statt der Werte entlang der $y$-Achse k"onnten wir auch nur den
Wert in einem Punkt sowie die Ableitungen
in diese Richtung vorgeben, zum Beispiel in der Form
\begin{align*}
u(0,0)&=u_0\\
\frac{\partial u}{\partial y}(0,y)&=h(y).
\end{align*}
Dies liefert jedoch nichts neues, denn die Funktion $g(y)=u(0,y)$
wird durch die gew"ohnliche Differentialgleichung
\[
g'(y)=\frac{\partial u}{\partial y}(0,y)=h(y)
\]
f"ur die unbekannte Funktion $g(y)$
mit der Anfangsbedinung $g(0)=u_0$ bestimmt.
Indem man diese Gleichung l"ost, f"uhrt man Anfangsbedingungen
mit Ableitungen entlang der $y$-Achse auf gew"ohnliche
Anfangswerte zur"uck.

\subsubsection{Partielle Differentialgleichungen zweiter Ordnung}
F"ur eine gew"ohnliche Differentialgleichung zweiter Ordnung m"ussen
zus"atzliche Anfangsbedingungen spezifiziert werden, typischerweise
die erste Ableitung.

F"ur eine partielle Differentialgleichung zweiter
Ordnung wird man entsprechend zu den Anfangswerten $g(y)$ die Werte
der ersten Ableitungen hinzunehmen wollen.
Wie wir im vorangegangenen Abschnitt gesehen haben, sind die partiellen
Ableitungen in $y$-Richtung durch die Anfangsbedingung $g(y)$ bereits
festgelegt. Nur die erste Ableitung in $x$-Richtung kann noch
vorgegeben werden. Die Anfangsbedingung kann also so formuliert
werden:
\begin{align}
&\text{Dirichlet-Randbedingung:}&
u(0,y)&=g(y)&
\label{klassifikation:dirichlet-randbedingung}
\\
&\text{Neumann-Randbedingung:}&
\frac{\partial u}{\partial x}(0,y)&=h(y)
\label{klassifikation:neumann-randbedingung}
\end{align}

\subsubsection{Beliebige Anfangskurve}
Im allgemeinen Fall sind die Randwerte nicht auf einer Koordinatenachse
vorgegeben. Es gibt sogar F"alle, wo dies gar nicht m"oglich ist,
ein Beispiel wird in \ref{unloesbar} gezeigt.
Die Festlegung von Werten entlang der der $y$-Achse entspricht
der Forderung, dass die L"osungsfl"ache durche eine bestimmte Kurve
verlaufen muss, die als Parameterdarstellung $y\mapsto (0,y,g(y))$
hat. Die Verallgemeinerung das allgemeine Cauchy-Problem:

\begin{problem}[Cauchy-Problem] Sei $\gamma$ eine Kurve im Raum und
$F(x,y,u,\partial_xu,\partial_yu)=0$ eine PDGL. Eine Funktion $u(x,y)$
heisst eine L"osung des Cauchy-Problems mit Anfangskurve $\gamma$, wenn
$\gamma$ im Graphen von $u$ enthalten ist.
\end{problem}

Durch die Wahl eines geeigneten Koordinatensystems kann aber immer
erreicht werden, dass in der Umgebung eines einzelnen Randpunktes
dieses Situation vorliegt.
Die Richtung der $x$-Achse ist dann die Normale auf die Randkurve oder
Randfl"ache.
Die Ableitung entlang der $x$-Achse ist daher eigentlich eine Ableitung
in Normalenrichtung, wir schreiben daf"ur
\[
\frac{\partial u}{\partial n}
\]
und nennen sie die Normalableitung.
Bei der eben beschriebenen Wahl des Koordinatensystems kann man
die Normalableitung mit Hilfe der Formel
\[
\frac{\partial u}{\partial n}
=\frac{\partial u}{\partial x}
\]
berechnen.

Ist $n$ der Normalenvektor auf die Kurve (oder Fl"ache f"ur $n=3$), auf
der die Randwerte vorgegeben sind, dann kann die Normalableitung mit 
Hilfe der Richtungsableitung berechnet werden, es ist
\[
\frac{\partial u}{\partial n}=D_nu = n\cdot \operatorname{grad} u.
\]

\subsection{Allgemeines Randwertproblem\label{klassifikation:allgemeines-randwertproblem}}
Bei partiellen Differentialgleichungen hat der Definitionsbereich einen
weit gr"osseren Einfluss auf die L"osung als bei gew"ohnlichen
Differentialgleichungen. Das in den voranstehenden Abschnitten
diskutierte Cauchy-Problem erlaubt zun"achst zu beurteilen, welche
Art von Randbedingungen sinnvoll ist. 
So haben Wir gesehen, dass Randwerte in folgenden Formen vorgegeben
werden k"onnen: 
\begin{itemize}
\item Als Werte entlang des Randes des Definitionsgebietes der
Differentialgleichung, also in der Form
\[
u(x)=g(x)\quad \forall x\in\partial G\subset \mathbb R^n.
\]
Diese Randbedingungen heissen Dirichlet-Randbedingungen.
\item Als Werte der Normalableitung auf dem
Rand, geschrieben
\[
\frac{\partial u}{\partial n}(x)=h(x)\quad\forall x\in\partial G\subset \mathbb R^n.
\]
Dieser Randbedingungen heissen Neumann-Randbedingungen.
\item Als Kombination von Funktionswerten und Ableitungen: es ist
m"oglich, auf verschiedenen Teilen des Randes Randwerte oder Normalableitungen
vorzugeben, es ist aber auch m"oglich, eine Linearkombination
von Randwerten und Normalableitungen vorzugeben.
\end{itemize}
Dies gen"ugt jedoch noch nicht, zu entscheiden, ob durch diese
Bedingungen die L"osungen eindeutig festgelegt sind.
Dazu ist ein vertieftere Theorie n"otig. W"ahrend bei den gew"ohnlichen
Differentialgieichungen eine relativ einfache Methode (Picard-Iteration)
unter relativ milden Voraussetzungen zu beweisen  erlaubt, dass
gew"ohnliche Differentialgleichungen zu gegebenen Anfangswerten eine
eindeutig bestimmte L"osung haben, wenigstens f"ur eine gewisses Zeitinterval,
ist dieser Beweis f"ur partielle Differentialgleichungen ungleich schwieriger.
F"ur gewisse Typen von partiellen Differentialgleichungen ist dies jedoch
m"oglich, einige davon werden in den kommenden Kapiteln behandelt.

\section{L"osung einer partiellen Differentialgleichung\label{klassifikation:loesung}}
Zur L"osung einer partiellen Differentialgleichung m"ussen also
drei Dinge vorgeben sein
\begin{enumerate}
\item Eine Differentialgleichung
\item Ein Gebiet $\Omega$
\item Randwertvorgaben auf einem Teil des Randes $\partial\Omega$.
\end{enumerate}
\begin{definition}
Eine L"osung dieses Problems ist eine Funktion $u$, welche
auf $\bar\Omega$ (nicht nur auf $\Omega$!) definiert ist,
im Gebiet $\Omega$ so oft differenzierbar ist wie n"otig, damit
sie sinnvoll in die Differentialgleichung eingesetzt werden kann,
in $\Omega$ die Differentialgleichung erf"ullt,
und ausserdem die Randwertbedingungen erf"ullt.
\end{definition}

Das Problem heisst ``gut gestellt'', wenn die gew"ahlten Vorgaben
die L"osung eindeutig bestimmen.
Es ist die Aufgabe der Theorie zu kl"aren, ob ein Problem
gut gestellt ist, denn erst dann kann man von einem numerischen
Verfahren eine sinnvolle Antwort erwarten.

\section{Spezielle Typen von partiellen Differentialgleichungen\label{klassifikation:spezielletypen}}
\rhead{Spezielle PDGL}
F"ur einige spezielle Typen von Differentialgleichungen werden wir
die Frage nach Existenz und Eindeutigkeit einer L"osung beantworten
k"onnen, und manchmal sogar eine explizite L"osung angeben k"onnen.
Dazu geh"oren die linearen und die quasilinearen partiellen
Differentialgleichungen, die in diesem Abschnitt beschrieben werden.

\subsection{Lineare partielle Differentialgleichungen\label{klassifikation:linear}}
Die bisher vorgestellten Differentialgleichungen sind also alle lineare
Ausdr"ucke in den ersten und zweiten Ableitungen, man k"onnte sie in der
Form
\begin{align*}
\sum_{i,j=1}^n a_{ij}(x)\frac{\partial}{\partial x_i} \frac{\partial}{\partial x_j}\psi(x)
+\sum_{i=1}^na_i(x)\frac{\partial}{\partial x_i}\psi(x)&=f(x)
\\
\sum_{i,j=1}^n a_{ij}(x)\partial_i \partial_j\psi(x)
+\sum_{i=1}^na_i(x)\partial_i\psi(x)&=f(x)
\end{align*}
Die Gleichung heisst homogen, wenn $f=0$ ist.

Damit diese Probleme "uberhaupt eine L"osung haben, m"ussen noch 
Randbedingungen hinzugef"ugt werden.
Auch diese lassen sich in der Form linearer Gleichungen zwischen den 
den Funktionswerten und den Normalableitungen von $\psi$ auf dem Rand des
Gebietes gegeben:
\begin{align*}
a\psi(x)+
b\frac{\partial}{\partial n}\psi(x)
&=g(x)\quad\forall x\in\gamma\\
a\psi(x)+b\partial_n\psi(x)&=g(x)\quad\forall x\in\gamma
\end{align*}
Die Randbedingungen heissen homogen, wenn $g=0$ ist.

Sind $\psi_1$ und $\psi_2$ zwei L"osungen der homogenen Gleichungen und der
homogenen Randbedingungen, dann ist auch $\alpha_1\psi_1+\alpha_2\psi_2$
L"osungen der homogenen Gleichung under homogenen Randbedingungen:
\begin{align*}
&\sum_{i,j=1}^n a_{ij}(x)\partial_i \partial_j
(\alpha_1\psi_1(x)+\alpha_2\psi_2(x))
+\sum_{i=1}^na_i(x)\partial_i
(\alpha_1\psi_1(x)+\alpha_2\psi_2(x))
\\
\alpha_1
&\sum_{i,j=1}^n a_{ij}(x)\partial_i \partial_j
\psi_1(x)
+
\alpha_1
\sum_{i=1}^na_i(x)\partial_i
\psi_1(x)
\\
+
\alpha_2
&\sum_{i,j=1}^n a_{ij}(x)\partial_i \partial_j
\psi_2(x)
+
\alpha_2
\sum_{i=1}^na_i(x)\partial_i
\psi_2(x))
=0
\end{align*}
oder f"ur die Randbedingung
\begin{align*}
a(\alpha_1\psi_1(x) +\alpha_2\psi_2(x))
\;+&b\partial_n
(\alpha_1\psi_1(x) +\alpha_2\psi_2(x))\\
=\alpha_1(a\psi_1(x)
\;+&b\partial_n
\psi_1(x))\\
+\;\alpha_2(a\psi_2(x)
\;+&b\partial_n
\psi_2(x))
=0\quad\forall x\in\gamma
\end{align*}
Die L"osungen einer homogenen partiellen Differentialgleichung
bilden also einen Vektorraum. Insbesondere lassen sich beliebige
L"osungen der inhomogenen Differentialgleichung dadurch finden, dass
man eine partikul"are L"osung $\psi_p$ der homogenen Differentialgleichung
findet, und dazu eine beliebige L"osung der homogenen Differentialgleichung
$\psi_h$ hinzuaddiert.

\subsection{Quasilineare partielle Differentialgleichungen erster Ordnung\label{klassifikation:quasilinear}}
Einen interessanten Spezialfall bilden die quasilinearen PDGL erster
Ordnung. Sie sind nicht unbedingt linear, aber die partiellen
Ableitungen erster Ordnung kommen nur linear vor. Die Funktion
$
F(x,y,u,p,q)
$
ist also linear in $p$ und $q$. Die Variablen $x$ und $y$ sowie die
gesuchte Funktion k"onnen daher nur in den Koeffizienten der
Variablen $p$ und $q$ vorkommen, $F$ muss von der Form
\[
F(x,y,u,p,q)=a(x,y,u)p+b(x,y,u)q+c(x,y,u)
\]
sein. Eine quasilineare Differentialgleichung in zwei Variablen
hat also die Form
\[
a(x,y,u(x,y))\frac{\partial u}{\partial x}+b(x,y,u(x,y))\frac{\partial u}{\partial y}
=c(x,y,u(x,y)).
\]
F"ur mehr Variablen $x_1,\dots,x_n$ gilt analog, dass eine quasilineare
Differentialgleichung die Form
\[
a_1(x_1,\dots,x_n,u)\frac{\partial u}{\partial x_1}
+
a_2(x_1,\dots,x_n,u)\frac{\partial u}{\partial x_2}
+\dots
+
a_n(x_1,\dots,x_n,u)\frac{\partial u}{\partial x_n}
=c(x_1,\dots,x_n,u)
\]
hat.

Nat"urlich lassen sich auch quasilineare partielle Differentialgleichungen
h"oherer Ordnung definieren, in einer solchen Differentialgleichung
kommen zwar h"oheren partielle Ableitungen vor, aber immer nur linear,
man kann sie also immer in der Form schreiben:
\[
\sum_{\bf k} a_{\bf k}(x,y,u)\partial_{\bf k} u(x,y) = 0,
\]
worin die Funktionen $a_{\bf k}(x,y,u)$ nur f"ur endlich viele
Multiindizes ${\bf k}$ von $0$ verschieden sind.

\subsection{Nichtlineare Gleichungen\label{klassifikation:nichtlinear}}
Die bisher vorgestellten Beispiele von partiellen Differentialgleichungen
sind alle linear. Viele Gleichungen der Physik sind jedoch
nicht linear.  Ber"uhmtestes Beispiel sind die Gleichungen, die
die Str"omung eines Gases beschreiben. Bereits Leonhard Euler hat 
f"ur die Str"omung eines idealen Gases ein System von partiellen
\index{Dichte}
\index{Druck}
Differentialgleichungen gefunden f"ur die Dichte $\varrho$, den Druck
$p$ und die Str"omungsgeschwindigkeit $\vec v$, alle drei sind Funktionen
\index{Stromungsgeschwindigkeit@Str\"omungsgeschwindigkeit}
von allen drei Raumkoordinaten und der Zeit. Die wichtigste davon
ist die Eulersche Gleichung:
\index{Eulersche Gleichung}
\begin{align*}
\frac{\partial \vec v}{\partial t}
+(\vec v\cdot \nabla)\vec v
=-\frac1{\varrho}\operatorname{grad}p
\\
\frac{\partial v_i}{\partial t}
+\sum_{j=1}^3v_j\frac{\partial v_i}{\partial x_j}
=
-\frac1{\varrho}\frac{\partial p}{\partial x_i}
\end{align*}
Diese Gleichung kann nicht linear sein, weil im zweiten Term
Produkte von $v_j$ mit Ableitungen von $v_i$ vorkommen.
Ber"ucksichtigt man auch noch die Z"ahigkeit, wird die Gleichung
noch komplizierter:
\[
\varrho\left(
\frac{\partial\vec v}{\partial t}
+
(\vec v\cdot\nabla)\vec v
\right)
=
-\operatorname{grad}p+\eta\Delta \vec v+\left(\zeta+\frac{\eta}3\right)
\operatorname{grad}\operatorname{div}\vec v
\]

In einer Dimension bleibt von dieser Gleichung nur noch eine Komponente
$u(t,x)$ "ubrig, f"ur die eine Gleichung der ungef"ahren Form
\[
\frac{\partial u}{\partial t}+u\frac{\partial u}{\partial x}
=\eta\frac{\partial^2u}{\partial x^2}
\]
gelten muss (wir haben den Druckgradienten vernachl"assigt und
$\varrho = 1$ angenommen). Diese Gleichung wurde von
Johannes Martinus Burgers ausgiebig studiert, und heisst
daher Gleichung von Burgers.
Eine L"osung des Anfangswertproblems der Gleichung von Burgers kann 
in expliziter Form gefunden werden, wir beschreiben diese in Kapitel
\ref{chapter-nichtlinear}.


\section{Anhang: Zusammenstellung der verwendeten Notation}
\rhead{Notation}
Die traditionelle Schreibweise der partiellen Ableitungen ist nicht sehr platzsparend,
daher verwenden wir im folgenden auch die gleichbedeutenden Schreibweisen
\begin{align*}
\frac{\partial f(x,y)}{\partial x}
&=\frac{\partial}{\partial x}f(x,y)
=\partial_x f(x,y)\\
\frac{\partial f}{\partial x_i}(x,y)
&=\partial_{x_i}f(x,y)=\partial_if(x,y)
\end{align*}
Ausserdem schreiben wir die die Ableitungen auch in Operatorschreibweise:
\begin{align*}
\frac{\partial}{\partial x}
&=
\partial_x\\
\frac{\partial}{\partial x_i}
&=
\partial_{x_i}
=\partial_i
\end{align*}
In dieser Notation lassen sich die bekannten Operatoren wie folgt schreiben:
\begin{align*}
\Delta &=\partial_x^2+\partial_y^2+\dots=\sum_{i=0}^n\partial_i^2\\
\operatorname{grad}&=\nabla=\begin{pmatrix}\partial_1\\\vdots\\\partial_n\end{pmatrix}
\end{align*}
Die Normalableitung, die in Randbedingungen f"ur partielle Differentialgleichungen
eine besondere Rolle spielt, wird 
\[
\frac{\partial u}{\partial n}=\partial_nu
\]
geschrieben.

\section{Zusammenfassung: Das Wichtigste in K"urze}
\begin{enumerate}
\item Eine Differentialgleichung ist eine Funktion
der Koordinaten, der gesuchten Funktion und ihrere partiellen Ableitungen
(\ref{klassifikation:differentialgleichungen}).
\item Die Ordnung ist die h"ochste in der Differentialgleichung vorkommende
Ableitung (\ref{klassifikation:ordnung}).
\item In zwei Dimensionen kann eine Differentialgleichung erster Ordnung
immer durch eine Funktion $F(x,y,u,p,q)$ beschrieben werden. Die
Differentialgleichung entsteht durch ersetzen von $p$ und $q$ durch
die partiellen Ableitungen von $u$:
\[
F\biggl(
x,y, u,
\frac{\partial u}{\partial x},
\frac{\partial u}{\partial y}
\biggr)=0.
\]
\item Eine partielle Differentialgleichung kann immer auf ein System
von partiellen Differentialgleichungen erster Ordnung reduziert werden
(\ref{klassifikation:umwandlung}).
\item Bei lineare Differentialgleichungen ist die Funktion $F$ linear
in der gesuchten Funktion und den partiellen Ableitungen
(\label{klassifikation:linear}). In zwei
Dimensionen ist $F(x,y,u,p,q)$ linear in $u$, $p$ und $q$.
\item Eine quasilineare Differentialgleichung ist linear in den
partiellen Ableitungen der gesuchten Funktion, aber nicht unbedingt
in der Funktion selbst (\ref{klassifikation:quasilinear}).
\item Das Gebiet $\Omega$, auf dem die Differentialgleichung definiert
ist, hat einen entscheidenden Einfluss auf die L"osungen (\ref{klassifikation:gebiete}).
\item Die L"osung ist erst eindeutig bestimmt, wenn geeignete Randbedingungen
vorgegeben sind (\ref{klassifikation:randbedingungen}). 
\item Bei partiellen Differentialgleichungen erster Ordnung sind typischerweise
Randwert f"ur einen Teil des Randes vorzugeben.
\item Bei partiellen Differentialgleichungen zweiter Ordnung sind
Randwert (Dirichlet-Randbedingungen,
(\ref{klassifikation:dirichlet-randbedingung}))
und/oder die Normalableitung
(Neumann-Randbedingung (\ref{klassifikation:neumann-randbedingung})
vorzugeben.
\item Eine L"osung einer partiellen Differentialgleichung ist eine
Funktion, die auf dem abgeschlossenen Gebiet definiert ist, im Inneren
des Gebietes die Differentialgleichung und auf dem Rand
die vorgegebenen Randwert erf"ullt (\ref{klassifikation:loesung}).

\end{enumerate}
