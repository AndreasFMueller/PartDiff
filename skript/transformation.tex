%
% transformation.tex
%
% (c) 2008 Prof Dr Andreas Mueller, Hochschule Rapperswil
% $Id: c03-transformation.tex,v 1.4 2008/10/31 08:04:16 afm Exp $
%
\chapter{Transformation}
\lhead{Transformation}
Bei der L"osung gew"ohnlicher linearer Differentialgleichungen lieferte die
Laplacetransformation eine hervorragende Methode, mit der sehr
viele Anfangswertprobleme gel"ost werden konnten.
In diesem Kapitel wird gezeigt, wie diese Methode wie auch
Fouriertransformation oder Fourierreihen 
f"ur lineare partielle Differentialgleichungen nutzbar gemacht
werden k"onnen.

\section{Einf"uhrungsbeispiel}
\subsection{Wellengleichung f"ur die schwingende Saite}
\rhead{Schwingende Saite}
Zur Einf"uhrung betrachten wir erneut das Beispiel der
schwingenden Saite mit der Gleichung
\[
\partial_t^2u=\partial_x^2u.
\]
Durch einen Separationsansatz hatten wir dieses Problem
darauf zur"uckgef"uhrt, Gleichungen $X''=-k^2X$ und $T''=-k^2T$
zu l"osen.
Dabei haben wir die harmonischen Funktionen wieder gefunden,
mit Hilfe der Fouriertheorie ist sodann die Bestimmung einer L"osung
gelungen.

Andererseits k"onnten wir auch mit der Fouriertheorie beginnen und
verwenden, dass sich eine Funktion auf dem Interval $[0,\pi]$
durch Spiegelung zu einer periodischen Funktion auf $[-\pi,0]$
erweitern l"asst. Indem wir zu jedem Zeitpunkt $t$ eine Fourieranalyse
durchf"uhren, l"asst sich die Funktion $x\mapsto u(t,x)$ als
Fourierreihe 
\[
u(t,x)=\frac{a_0(t)}2+\sum_{k=1}^\infty a_k(t)\cos kx+b_k(t)\sin kx
\]
mit zeitabh"angigen Koeffizienten schreiben.
Setzen wir diesen Ansatz in die Differentialgleichung ein, ergibt sich
\begin{align*}
\partial_t^2u(t,x)&=\frac{a_0''(t)}2
+\sum_{k=1}^\infty a_k''(t)\cos kx+b_k''(t)\sin kx\\
\partial_x^2u(t,x)&=
-\sum_{k=1}^\infty a_k(t)k^2\cos kx+b_k(t)k^2\sin kx
\end{align*}
Wenn diese Funktionen gleich sein sollen, muss
\[
\frac{a_0''(t)}2
+\sum_{k=1}^\infty (a_k''(t)+k^2a_k(t))\cos kx+(b_k''(t)+k^2b_k(t))\sin kx=0
\]
gelten. Nach der Fourier-Theorie ist dies nur m"oglich, wenn die
Fourier-Koeffizienten alle verschwinden:
\begin{align*}
a_0''(t)&=0\\
a_k''(t)&=-k^2a_k(t)\\
b_k''(t)&=-k^2b_k(t)
\end{align*}
mit $k>0$.
Wir haben also durch "Ubergang zu Fourierkoeffizienten ein System
von gew"ohnlichen Differentialgleichungen f"ur die Fourier-Koeffizienten
gefunden. Die L"osungen sind wohl bekannt:
\begin{align*}
a_0(t)&=m_0t+c_0\\
a_k(t)&=A^a_k\cos kt+B^a_k\sin kt\\
b_k(t)&=A^b_k\cos kt+B^b_k\sin kt\\
\end{align*}
Dies entspricht den L"osungen, die wir bereits im vorangegangenen
Kapitel gefunden haben.

\subsection{Anfangsbedingungen}
Die Differentialgleichungen f"ur die Koeffizienten $a_k(t)$ und $b_k(t)$
k"onnen erst dann vollst"andig gel"ost werden, wenn Anfangs oder Randbedingungen
gegeben sind. Sie also zus"atzlich zur Wellengleichung die Anfangsbedingung
\begin{align*}
u(0,x)&=f(x)\\
\frac{\partial u}{\partial t}&=g(x)
\end{align*}
gegeben. auch die Funktionen $f$ und $g$ k"onnen duch eine Fourierreihe
dargestellt werden, werden wir schreiben
\begin{align*}
f(x)&=\frac{a_0^f}2+\sum_{k=1}^\infty a_k^f\cos kx+b_k^f\sin kx\\
g(x)&=\frac{a_0^g}2+\sum_{k=1}^\infty a_k^g\cos kx+b_k^g\sin kx
\end{align*}
Zusammen mit dem Ansatz f"ur $u(t,x)$ als Fourierreihe finden wir jetzt
die Bedingungen f"ur $t=0$
\begin{align*}
\frac{a_0(0)}2+\sum_{k=1}^\infty a_k(0)\cos kx +b_k(0)\sin kx
&=
\frac{a_0^f}2+\sum_{k=1}^\infty a_k^f\cos kx+b_k^f\sin kx\\
\\
\frac{a_0'(0)}2+\sum_{k=1}^\infty a_k'(0)\cos kx+b_k'(0)\sin kx
&=
\frac{a_0^g}2+\sum_{k=1}^\infty a_k^g\cos kx+b_k^g\sin kx
\end{align*}
Koeffizientenvergleich liefert jetzt die Anfangsbedingungen f"ur die
Funktionen $a_k$ und $b_k$:
\begin{align*}
a_k(0)&=a_k^f&b_k(0)&=b_k^f\\
a_k'(0)&=a_k^g&b_k'(0)&=b_k^g\\
\end{align*}
Zusammen mit den fr"uher gefundenen L"osungen gilt also
\begin{align*}
c_0&=a_0^f&&&A_k^a&=a_k^f&&&&&A_k^b&=b_k^f\\
m_0&=a_0^g&&&kB_k^a&=a_k^g&\Rightarrow B_k^a&=\frac1ka_k^g&&&kB_k^b&=b_k^g&\Rightarrow B_k^b&=\frac1kb_k^g
\end{align*}
Die vollst"andige L"osung ist damit
\begin{align*}
u(t,x)=\frac{a_0^gt+a_0^f}2
+\sum_{k=1}^\infty
\biggl(a_k^f\cos kt+\frac1ka_k^g\sin kt\biggr)\cos kx
+
\biggl(b_k^f\cos kt+\frac1kb_k^g\sin kt\biggr)\sin kx.
\end{align*}

\subsection{Inhomogene Wellengleichung}
Das Verf"ahren l"asst sich auch die inhomogene Wellengleichung
\[
\partial_t^2u-\partial_x^2u=f
\]
verallgemeinern. Die Funktion $f(t,x)$ l"asst sich nat"urlich ebenso wie
$u$ als Fourierreihe entwickeln:
\[
f(t,x)=\frac{a_0^f(t)}2+\sum_{k=1}^\infty a_k^f(t)\cos kx+b_k^f(t)\sin kx.
\]
Setzt man dies zusammen mit der Entwicklung f"ur $u(t,x)$ in die
Differentialgleichung ein, findet man die Gleichungen
\begin{align*}
a''_k(t)+k^2a_k(t)&=a_k^f(t)\\
b''_k(t)+k^2b_k(t)&=b_k^f(t),
\end{align*}
also ein System von gew"ohnlichen inhomogenen linearen Differentialgleichungen.

\subsection{Lehren aus dem Einf"uhrungsbeispiel}
Durch die Transformation auf die Fourier-Koeffizienten verschwindet die
Ableitung nach $x$,  stattdessen bleibt nur noch eine gew"ohnliche
Differentialgleichung, die zweifache Ableitung nach $x$ wurde zur
algebraischen Operation der Multiplikation mit $-k^2$. Diese Transformation
war m"oglich, weil der $x$-Definitionsbereich ein Interval war.
F"ur andere Definitionsbereiche wird die Fourier-Transformation nicht
geeignet sein.

\section{Laplace-Transformation Crash-Course}
F"ur $[0,\infty[$ als Definitionsbereich ist die Laplace-Transformation
die geeignete Transformation. Dieser Abschnitt enth"alt einen Kurzabriss
der wesentlichen Ideen der Theorie der Laplace-Transformation.
F"ur uns wesentlich ist dabei, dass die Transformation die Ableitungen
in algebraische Operationen umwandelt. Eine algebraische Gleichung ist
einfacher zu l"osen, die R"ucktransformation liefert daraus die L"osung.

\subsection{Definition}
\begin{definition}
Die Laplace-Transformierte ${\cal L}f$ einer Funktion
$f\colon[0,\infty[\to\mathbb R$ ist die Funktion
\[
({\cal L}f)(s)=\int_0^\infty f(t)e^{-st}\,dt.
\]
\end{definition}
Offenbar ist ${\cal L}f$ f"ur beliebige beschr"ankte Funktionen $f$ definiert.

\subsection{Beispiele}
Drei Beispiele sollen das Rechnen mit der Laplace-Transformation
illustrieren.
\begin{enumerate}
\item
Sei $f(t)=c$ eine konstante Funktion. Die Laplace-Transformation
\[
({\cal L}f)(s)
=
\int_0^\infty ce^{-st}\,dt
=
\left[
-\frac{c}{s}e^{-st}
\right]_0^\infty=\frac{c}{s}.
\]

\item
Sei jetzt $f(t)=e^{-kt}$, die Laplace-Transformation ist
\[
({\cal L}f)(s)=\int_0^\infty e^{-kt}e^{-st}\,dt
=\int_0^{\infty}e^{-(s+k)t}\,dt
=
\left[
- \frac1{s+k}e^{-(s+k)t}
\right]_0^\infty
=\frac1{s+k}.
\]

\item
Wir berechnen jetzt die Laplace-Transformierte einer Ableitung, also
$({\cal L}f')(s)$:
\begin{align*}
({\cal L}f')(s)
&=
\int_0^\infty f'(t)e^{-st}\,dt
\\
&=
\left[f(t)e^{-ts}\right]_0^\infty
+
\int_0^\infty sf(t)e^{-st}\,dt
\\
&=
-f(0)+s\int_0^\infty f(t)e^{-st}\,dt
\\
&=s({\cal L}f)(s)-f(0).
\end{align*}
Wie versprochen wandelt die Laplace-Transformation die Ableitung
ein eine algebraische Operation um.
\end{enumerate}

\subsection{L"osung einer gew"ohnlichen Differentialgleichung}
Wir l"osen die Differentialgleichung
\[
\dot x(t)+px(t)=f(t)
\]
mit der Anfangsbedingung $x(0)=0$.
Die Laplace-Transformation der Differentialgleichung ist
\[
s({\cal L}x)(s)-x(0)+p({\cal L}x)(s)=({\cal L}f)(s).
\]
Zu bestimmen ist $x$, diese Gleichung ist eine algebraische Gleichung
f"ur ${\cal L}x$, die man auch nach ${\cal L}x$ aufl"osen kann:
\[
({\cal L}x)(s)
=
\frac{({\cal L}f)(s)}{s+p}
\]
Um zu zeigen, wie mit dieser Methode die vollst"andige L"osung gefunden
werden kann, f"uhren wir die R"ucktransformation f"ur $f(t)=q$ durch.
Zun"achst ist in diesem Fall
\[
({\cal L}f)(s)=\frac{q}{s}
\]
und damit 
\[
({\cal L}x)(s)
=
\frac{q}{s(s+p)}=\frac{q}{p}\frac{1}{s}-\frac{q}{p}\frac{1}{s+p}
\]
Beide Termen auf der rechten Seite k"onnen mit den Beispielen im vorangegangenen
Abschnitt r"ucktransformiert werden:
\[
x(t)=\frac{q}{p}-\frac{q}{p}e^{-pt}=\frac{q}{p}(1-e^{-pt}).
\]

\subsection{L"osung einer partiellen Differentialgleichung}
Dieses Verfahren funktioniert auch f"ur partielle Differentialgleichungen.
Als Beispiel l"osen wir die quaslineare Differentialgleichung
\[
\frac{\partial u}{\partial t}+x\frac{\partial u}{\partial x}=x
\]
$u$ ist definiert f"ur $t\ge0$, $x\ge 0$ und wir verlangen die Randbedingungen
\begin{align*}
u(x,0)&=0\qquad x>0\\
u(0,t)&=0\qquad t>0\\
\end{align*}
Die Laplacetransformation der Differentialgleichung ist
\[
s({\cal L}u)(s,x)-u(0,x)+x\frac{\partial}{\partial x}({\cal L}u)(s,x)=\frac{x}{s},
\]
sie hat die Ableitung nach der Zeit zum Verschwinden gebracht. F"ur jedes
$s$ kann man dies als eine gew"ohnliche Differentialgleichung f"ur die
Funktion $x\mapsto ({\cal L}u)(s,x)$ mit der Anfangsbedingung
$({\cal L}u)(s,0)=0$ betrachten.
Die L"osung mit den Standardmethoden 
ergibt
\[
({\cal L}u)(s,x)=\frac{x}{s(s+1)}=\frac{x}{s}-\frac{x}{s+1}.
\]
Beide Terme k"onnen r"ucktransformiert werden, die L"osung ist
\[
u(t,x)=x-xe^{-t}=x(1-e^{-t}).
\]
Wir berechnen zur Kontrolle die Ableitungen
\begin{align*}
\frac{\partial u}{\partial t}
&=
xe^{-t}
\\
\frac{\partial u}{\partial x}
&=
1-e^{-t},
\end{align*}
damit 
\[
xe^{-t}+x(1-e^{-t})=x,
\]
die Differentialgleichung ist also erf"ullt, und wegen
\begin{align*}
u(x,0)
&=
0
\\
u(0,t)
&=
0
\end{align*}
sind auch die Randbedingungen erf"ullt.

\section{W"armeleitung auf einem Ring}
\rhead{W"armeleitung}
Wir betrachten jetzt die W"armeleitungsgleichung auf einem Ring
mit gegebenen Anfangswerten. Die Temperaturverteilung auf dem
Ring beschreiben wir als eine periodische Funktion auf dem Interval
$[-\pi,\pi]$. Gesucht ist daher eine L"osung der W"armelteitungsgleichung
mit den Anfangsbedingungen
\begin{align*}
\partial_t u(t,x)&=\partial_x^2 u(t,x) &&\forall(t,x)\in\mathbb R\times[-\pi,\pi]\\
u(0,x)&=f(x)&& \forall x\in[-\pi,\pi]\\
\end{align*}

\subsection{Fouriertransformation auf dem Interval}
Wie im einf"uhrenden Beispiel gehen wir jetzt zu Fourierkoeffizienten
"uber.
Der Einfachheit halber verwenden wir komplexe Fourierkoeffizienten
\[
c_k=\hat f(k)=\frac1{2\pi}\int_{-\pi}^{\pi}e^{-ikx}f(x)\,dx
\]
Wenden wir die Transformation auf die urspr"ungliche Differentialgleichung
an, erhalten wir
\begin{align}
\frac1{2\pi}\int_{-\pi}^{\pi}e^{-ikx}\partial_t u(t,x)\,dx
&=\frac1{2\pi}\int_{-\pi}^{\pi}e^{-ikx}\partial_x^2 u(t,x)\,dx
\notag
\\
\partial_t\frac1{2\pi}\int_{-\pi}^{\pi}e^{-ikx} u(t,x)\,dx
&=
\frac1{2\pi}
\underbrace{
\left[
e^{-ikx}\partial_x u(t,x)
\right]_{-\pi}^{\pi}}_{=0}
+
ik\frac1{2\pi}\int_{-\pi}^{\pi}e^{-ikx}\partial_x u(t,x)\,dx
\notag
\\
&=
0 + 
ik\frac1{2\pi}
\underbrace{
\left[
e^{-ikx}u(t,x)
\right]_{-\pi}^{\pi}}_{=0}
-
k^2\frac1{2\pi}\int_{-\pi}^{\pi}e^{-ikx} u(t,x)\,dx
\notag
\\
&=
-
k^2\frac1{2\pi}\int_{-\pi}^{\pi}e^{-ikx} u(t,x)\,dx
\notag
\\
\partial_t\hat u(t,k)&=-k^2 \hat u(t,k)
\label{fouriertransformiert}
\end{align}
Aus der partiellen Differentialgleichung ist ein System von gew"ohnlichen
Differentialgleichungen f"ur $t\ge 0$ geworden.

\subsection{Laplacetransformation}
Durch Laplace-Transformation der Gleichung (\ref{fouriertransformiert})
erhalten wir die Gleichung
\begin{align*}
\int_0^{\infty} \partial_t \hat u(t,k)e^{-st}\,dt
&=
-k^2\int_0^{\infty}\hat u(t,k)e^{-st}\,dt
\\
\left[\hat u(t,k)e^{-st}\right]_0^{\infty}
+
s\int_0^{\infty}\hat u(t,k)e^{-st}\,dt
&=
-k^2{\cal L}\hat u(s,k)
\\
\hat u(0,k)-s{\cal L}\hat u(s,k)&=k^2{\cal L}\hat u(s,k).
\\
{\cal L}\hat u(s,k)&=\frac{\hat u(0,k)}{s+k^2}
\end{align*}

\subsection{R"ucktransformation}
Um die allgemeine L"osung der Differentialgleichung zu finden,
muss man jetzt r"ucktransformieren, zun"achst nach Laplace:
\begin{align*}
\hat u(t,k)&={\cal L}^{-1}\left(\frac1{s+k^2}\right)\hat u(0,k)
\\
&=e^{-k^2t}\hat u(0,k)
\end{align*}
und dann auch noch nach Fourier, also durch Bilden der Fourierreihe
\begin{align*}
u(t,x)&=\sum_{k\in\mathbb Z}e^{-k^2t}\hat u(0,k)e^{ikx}
\\
&=\sum_{k\in\mathbb Z}\hat u(0,k)e^{ikx-k^2t}
\end{align*}

\section{Diffusion}
\rhead{Diffusion}
Wir betrachten jetzt die Diffusionsgleichung auf der Halbebene
$(t,x)\in[0,\infty)\times \mathbb R$
\[
\partial_tu-\partial_x^2u=f(t,x)
\]
mit den Anfangsbedingungen
\[
u(0,x)=g(x)\quad \forall x\in\mathbb R.
\]
Wir wenden jetzt die jeweils an den Definitionsbereich angepasste
Transformation durch.
\subsection{Fouriertransformation}
Die Fouriertransformation f"ur eine Funktion $g(x)$ von $x$ ist
\[
\hat g(k)=\frac1{\sqrt{2\pi}}\int_{\mathbb R}g(x)e^{-ikx}\,dx.
\]
Die Fouriertransformation macht aus den gegebenen Gleichungen
\begin{align*}
\partial_t \hat u(t,k)+k^2\hat u(t,k)&=\hat f(t,k)\\
\hat u(0,k)&=\hat g(k)
\end{align*}
\subsection{Laplace-Transformation}
In $t$-Richtung wenden wir eine Laplace-Transformation an
\[
\hat u(0,k)-(s-k^2){\cal L}\hat u(s,k)={\cal L}\hat f(s,k)
\]
Diese Gleichung kann nach ${\cal L}\hat u(s,k)$ aufl"osen:
\[
{\cal L}\hat u(s,k)=\frac{\hat u(0,k)-{\cal L}\hat f(s,k)}{s-k^2}
\]
\subsection{L"osung des Anfangswertproblems}
Durch R"ucktransformation k"onnen jetzt auch die L"osungen wieder
gewonnen werden.
\begin{align*}
\hat u(t,k)&={\cal L}^{-1}
\frac{\hat u(0,k)-{\cal L}\hat f(s,k)}{s-k^2}
\\
u(t,x)&={\cal F}^{-1}{\cal L}^{-1}
\left(\frac{\hat u(0,k)-{\cal L}\hat f(s,k)}{s-k^2}\right)(t,x)
\end{align*}

\section{Transformation bei beliebigen Gebieten}
Mit Hilfe von Integraltransformationen k"onnen partielle
Differentialgleichungen  gel"ost werden, wenn die Transformation
an das Definitionsgebiet angepasst ist:
\begin{center}
\begin{tabular}{cl}
Definitionsgebiet&Transformation\\
\hline
$[0,\infty)$&Laplacetransformation\\
$\mathbb R$&Fouriertransformation\\
$[-\pi,\pi]$&Fourierreihen
\end{tabular}
\end{center}
Die Separationsmethode zeigt uns aber auch, wie dieses 
Verfahren verallgemeinert werden kann. Die W"armeleitungsgleichung
soll auf einem Gebiet $G\subset \mathbb R^n$ gel"ost werden, d.~h.~es
soll eine L"osung der Differentialgleichung
\[
\partial_t u(t,x)=\Delta u(t,x)
\]
gefunden werden f"ur $(t,x)\in [0,\infty)\times G$ mit der Anfangsbedingung
\[
u(0,x)=u_0(x) \quad x\in G
\]
und der Randbedingung
\[
au(t,x)+b\partial_nu(t,x)=0\quad (t,x)\in[0,\infty)\times \partial G
\]
gefunden werden. Kapitel \ref{chapter-separation} suggeriert, dass
es eine Familie von Funktionen $u_k(x)$ gibt, die der Gleichung
\[
\Delta u_k=\lambda_k u_k
\]
und der Randbedingung gen"ugen, und mit denen sich die 
Anfangsbedingung als Reihe schreiben l"asst:
\[
u_0(x)=\sum_{k=1}^\infty a_ku_k(x).
\]
Dann kann man die L"osung der W"armeleitungsgleichung sofort
hinschreiben:
\[
u(t,x)=\sum_{k=1}e^{-\lambda_kt}u_k(x)
\]
Die Funktionen spielen die Rolle von $e^{ikx}$ in der Fouriertheorie.

Die moderne Mathematik kennt eine allgemeine Theorie der harmonischen
Analyse auf sogenannten Lie-Gruppen, welche die Fourier- und 
Laplace-Theorie verallgemeinert und vereinheitlicht, und welche
weitere Gebiete zu behandeln erlaubt. Zum Beispiel erm"oglicht
die Entwicklung von Funktionen auf einer ($n$-dimensionalen)
Kugel mit sogenannten Kugelfunktionen die Idee der Fourierreihen
auf einem Interval auf h"oherdimensionale Gebiete zu verallgemeinern.

\section{Zusammenfassung: das Wichtigste in K"urze}
\begin{enumerate}
\item Der "Ubergang von Funktionen zu Fourierreihen verwandelt
eine partielle Differentialgleichung in eine Famile gew"ohnlicher
Differentialgleichung f"ur die einzelnen Fourier-Koeffizienten.
\item Integraltransformationen k"onnen ein partielle Differentialgleichung
in eine Familie partieller Differentialgleichungen mit weniger Variablen
oder sogar gew"ohnlicher Differentialgleichungen verwandeln.
\item Integraltransformationen und die R"ucktransformationen k"onnen
Formeln f"ur die L"osungen gewisser partieller Differentialgleichungen
liefern, und damit die Frage beantworten, f"ur welche Randwertvorgaben
die Gleichungen gut gestellt sind.
\end{enumerate}
