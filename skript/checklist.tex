%
% checklist.tex -- Checkliste zur Pruefungsvorbereitung
%
% (c) 2009 Prof. Dr. Andreas Mueller, HSR
% $Id: skript.tex,v 1.34 2008/11/02 22:46:16 afm Exp $
%
\documentclass[a4paper,12pt,twocolumn]{article}
\usepackage{geometry}
\geometry{papersize={200mm,280mm},total={180mm,250mm}}
\usepackage{german}
\usepackage{times}
\usepackage{alltt}
\usepackage{verbatim}
\usepackage{fancyhdr}
\usepackage{amsmath}
\usepackage{amssymb}
\usepackage{amsfonts}
\usepackage{amsthm}
\usepackage{textcomp}
\usepackage{graphicx}
\usepackage{array}
\usepackage{ifthen}
\usepackage{multirow}
\usepackage{txfonts}
\usepackage{paralist}
\begin{document}
\title{Prüfungsvorbereitungscheckliste:\\ Partielle Differentialgleichungen, Theoretischer Teil}
\date{}
\maketitle
\section{Begriffe}
\begin{compactenum}
\item Gebiet
\item Rand
\item partielle Ableitung
\item partielle Differentialgleichung
\item Laplace-Operator
\item Ordung
\item lineare PDGL
\item Funktion $F$ einer PDGL
\item Normalableitung
\item Randbedingung
\item Dirichlet-Randbedingung
\item Neumann-Randbedingung
\item Anfangswerte
\item Cauchy-Problem
\item Randwertproblem
\item Wellengleichung
\item Wärmeleitungsgleichung
\item quasilineare PDGL
\item Charakteristik
\item Charakteristiken und Lösbarkeit
\item Separation
\item Überlagerungsprinzip
\item Eigenwertproblem
\item Laplace-Transformation
\item Fourier-Transformation
\item Symbolmatrix
\item Klassifikation der linearen PDGL zweiter Ordnung
\item hyperbolische PDGL
\item elliptische PDGL
\item parabolische PDGL
\item Laplace-Gleichung
\item harmonische Funktion
\item Maximum-Prinzip für harmonische Funktionen
\item Greensche Funktion
\item Mittelwerteigenschaft
\item Streifen
\item Charakteristiken einer hyperbolischen PDGL
\end{compactenum}
\section{Fragen}
\begin{compactenum}
\item Wie löst man eine quaslineare PDGL 1.~Ordnung mit Charakteristiken?
\item Wie funktioniert das Lösungsverfahren mit Separation?
\item Wie funktioniert Lösungsverfahren mit Transformation?
\item Geben Sie je ein Anwendungsbeispiel einer elliptischen, parabolischen und
hyperbolischen PDGL.
\item Warum sind die Lösungen einer elliptischen PDGL in einem
zusammenhnängenden und beschränkten Gebiet eindeutig?
\item Wie kann man Charakteristiken verwenden, um zu beurteilen, ob ein
hyperbolisches Differentialgleichungsproblem gut gestellt ist?
\item Wie breiten sich Unstetigkeitsstellen einer Lösung einer
hyperbolischen PDGL aus?
\end{compactenum}
\vfill
\end{document}
