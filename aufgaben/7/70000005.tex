Bei der numerischen L"osung des Poissonproblems auf dem Kreisgebiet
mit den Randbedingungen
\[
u= x^{\frac13}+y^{\frac13}\qquad \text{f"ur $x^2+y^2=1$}
\]
ergab sich der Wert $u(\frac12,\frac13)=1.7820$.  Kann das sein?

\ifthenelse{\boolean{loesungen}}{
\begin{loesung}
Der maximale Wert von
$f(x,y)= x^{\frac13}+y^{\frac13}$
wird dort erreicht, wo $x$ und $y$
gleich gross sind, wie man zum Beispiel mit der Methode der
Lagrange-Multiplikatoren erkennen kann. Der Wert an dieser Stelle
ist
\[
u\biggl(
\pm\frac{\sqrt{2}}2
,
\pm\frac{\sqrt{2}}2
\biggr)
=\begin{cases}
0
&\qquad\text{verschiedene Vorzeichen}\\
\pm 2\cdot 2^{-\frac16}=\pm1.781797\dots
&\qquad\text{gleiche Vorzeichen}
\end{cases}
\]
Der Wert im Punkt $(\frac12,\frac13)$ ist also gr"osser als
der gr"osste Randwert, was dem Maximum-Prinzip widerspricht.

Zur Methode der Lagrange-Multiplikatoren: man sucht das Maximum der
Funktion $f$ unter der Nebenbedingung $g(x,y)=x^2+y^2=1$.
Dazu ist $\lambda$ zu finden so dass
$\operatorname{grad}f=\lambda\operatorname{grad}g$
gilt:
\begin{align*}
\frac13x^{-\frac23}&=\lambda 2x\\
\frac13y^{-\frac23}&=\lambda 2y\\
\end{align*}
L"ost man beide Gleichungen nach $x$ und $y$ auf, ergibt sich
\begin{align*}
x&=(6\lambda)^{-\frac35}\\
y&=(6\lambda)^{-\frac35}
\end{align*}
Ausserdem sind die Punkte $(0,\pm 1)$ und $(\pm 1,0)$ weitere
Kandidaten f"ur die Extremstellen, die Funktionswerte von $f$ sind dort
aber $\pm 1$.
\end{loesung}
}{}

