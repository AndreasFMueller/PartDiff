Auf dem Gebiet
\[
\Omega = \{ (x,y)\,|\, 1 < x^2 + y^2\}
\]
soll die folgende partielle Differentialgleichung in Polarkoordinaten
\begin{equation}
\frac{\partial^2 u}{\partial r^2}
+
\frac{\partial^2 u}{\partial \varphi^2}
=
0
\label{90000015:dgl}
\end{equation}
gelöst werden.
Auf welchem Teil des Randes $x^2 +y^2 = 1$ müssen Randwerte vorgegeben
werden, damit die Lösung $u(x,y)$ für Punkte mit $x>0$ und $x^2 + y^2 = 4$
eindeutig bestimmt ist?

\begin{loesung}
Es ist einfacher, auch das Gebiet in Polarkoordinaten zu schreiben.
Es gilt
\[
\Omega = \{ (r,\varphi)\,|\, r>1\}.
\]
Die Differentialgleichung \eqref{90000015:dgl} hat das Symbol
\[
A
=
\begin{pmatrix}
1&0\\
0&1
\end{pmatrix}
\qquad\Rightarrow\qquad
\det A=-1<0
\]
und ist daher elliptisch.
Die Lösung einer elliptischen Differentialgleichung in einem bestimmten
Punkt ist nur dann eindeutig bestimmt, wenn in jedem beliebigen Randpunkt
Randwerte vorgegeben sind.
Genau durch diese Eigenschaft unterscheiden sich elliptische
partielle Differentialgleichungen fundamental von den
hyperbolischen, bei denen ein durch
den Verlauf der Charakteristiken bestimmer Teil des Randes bereits die
Lösung festlegen kann.

Leider ist das nicht gut genug, denn das Gebiet ist in der vorliegenden
Aufgabe unbeschränkt, während der Satz~7.2 im Skript und damit die davon
abhängenden Sätze über die Eindeutigkeit der Lösung auch nur für beschränkte
Gebiete gelten können.
In der Tat ist die Funktion  $r=\sqrt{x^2+y^2}$ eine Lösung der
Differentialgleichung, denn
\[
\frac{\partial^2 r}{\partial r^2} + \frac{\partial^2 r}{\partial \varphi^2}=0.
\]
Man kann also zu jeder Lösung $u$, die die Randbeingung auf dem ganzen Rand
$r=1$ erfüllt, die Funktion $r$ hinzuaddieren und damit eine Lösung erhalten,
die die gleiche Randbedingung erfüllt.
Die Funktion $r$ wächst aber beliebig an.
Um Eindeutigkeit der Lösung zu garantieren muss man also zusätzlich zur
Randbedingung bei $r=1$ eine Bedingung für das Verhalten bei sehr grossen $r$
stellen, zum Beispiel dass die Lösungen für $r\to\infty$ gegen $0$ gehen.

Alternativ hätte man argumentieren können, dass die Greensche Funktion 
die Lösung eindeutig garantieren würde.
Doch die Greensche Funktion verlangt ja als Voraussetzung ebenfalls,
dass die Lösung eindeutig sein muss, ist also ebenfalls von einer
Bedingung im Unendlichen abhängig.
Diese wird aber meist implizit gegeben, so dass dieses Argument
tatsächlich funktionieren könnte.

Ein weiterer Lösungsansatz besteht darin, dass man die Differentialgleichung
auf dem Gebiet explizit löst.
Dies ist mit einem Separationsansatz
\[
u(r,\varphi) = R(r) \cdot \Phi(\varphi)
\]
möglich.
Die Separation ergibt die beiden Gleichungen
\begin{align*}
\Phi''(\varphi)&=-\mu^2\Phi(\varphi)
\\
R''(r)&=\mu^2R(r).
\end{align*}
Darin ist der Wert der Separationskonstanten bereits so gewählt, dass sich
die Lösung der $\Phi$-Gleichung einfach schreiben lässt.
Man findet dafür
\[
\Phi(\varphi) =
\begin{cases}
\sin\mu\phi\\
\cos\mu\phi
\end{cases}
\]
Dies können jedoch nur Lösungen sein, wenn sie $2\pi$-periodisch sind,
also muss $\mu$ eine ganze Zahl sein, die wir im Folgenden wieder $n$
schreiben.

Für die Funktion $R$ finden wir
\[
R(r) = e^{\pm nr}.
\]
Die Lösung der Differentialgleichung hat also die allgemeine
Form
\[
u(r,\varphi)
=
\frac{a_0}2
+
\sum_{n=1}^\infty
r^{-nr}(a_n^- \cos n\varphi + b_n^- \sin n\varphi)
+
\sum_{n=1}^\infty
r^{+nr}(a_n^+ \cos n\varphi + b_n^+ \sin n\varphi)
\]
Die Koeffizienten $a_0$, $a_n^\pm$ und $b_n^\pm$ müsssen noch bestimmt
werden.
Dies kann auf zwei Arten geschehen.
\begin{enumerate}
\item Man könnte Funktionswerte von $u$ und von der Normalableitung für
$r=1$ vorgeben.
Die Normalableitung fällt mit der Ableitung nach $r$ zusammen.
Dann wäre nämlich
\begin{align*}
u(1,\varphi)
&=
\frac{a_0}2
+
\sum_{n=1}^\infty ((a_n^-+a_n^+)\cos n\varphi + (b_n^-+b_n^+)\sin n\varphi)
\\
\frac{\partial u}{\partial r}(1,\varphi)
&=
\sum_{n=1}^\infty ((-a_n^-+a_n^+)\cos n\varphi + (-b_n^-+b_n^+)\sin n\varphi)
\end{align*}
Vorgabe dieser Randwerte würde also $a_n^-\pm a_n^+$ und $b_n^-\pm b_n^+$
festlegen, und damit alle Koeffizienten.
\item
Man könnte durch eine Bedingung im Unendlichen die Terme der Form $e^{nr}$
ausschliessen, dadurch würden die Koeffizienten $a_n^+=0$ und $b_n^+=0$.
Es wären nur noch die Koeffizienten $a_n^-$ und $b_n^-$ zu bestimmen,
und das wiederum wäre durch vorgabe der Randwerte $u(1,\varphi)$
machbar.
\qedhere
\end{enumerate}
\end{loesung}

\begin{bewertung}
Elliptischer Differentialoperator ({\bf E}) 2 Punkte,
Randwerte müssen auf dem ganzen Rand vorgegeben werden ({\bf R}) 2 Punkte,
Hinweis auf das Problem mit dem unbeschränkten Gebiet ({\bf U}) 2 Punkte.
Da der letzte Schritt im Unterricht nur am Rande behandelt wurde, wird der 
mögliche Wegfall der Punkte {\bf U} durch Anpassung der Notenskala
kompensiert.

Für den Zugang über die Separationslösung wurden Punkte wie folgt vergeben:
Separationsansatz ({\bf A}) 1 Punkt,
Separation ({\bf M}) 1 Punkt,
Lösung der Separierten Gleichungen ({\bf L}) 1 Punkt,
Bedingung, dass $n$ ganzzahlig ({\bf N}) sein muss 1 Punkt,
Schlussfolgerung über nötige Randwertvorgaben ({\bf R}) 2 Punkte.
\end{bewertung}

