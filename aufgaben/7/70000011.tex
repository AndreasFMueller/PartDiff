On the domain
\[
\Omega = \{ (x,y)\,|\, 1 < x^2 + y^2\},
\]
the partial differential equation
\begin{equation}
\frac{\partial^2 u}{\partial r^2}
+
\frac{\partial^2 u}{\partial \varphi^2}
=
0
\label{90000015:dgl}
\end{equation}
is to be solved.
On what portion of the boundary
$x^2 +y^2 = 1$
do we need to specify boundary values in order for the the solution $u(x,y)$
to be uniquely determined in the points with $x>0$ and $x^2 + y^2 = 4$?

\begin{loesung}
The domain is more simply described in polar coordinates.
We have
\[
\Omega = \{ (r,\varphi)\,|\, r>1\}.
\]
The differential equation \eqref{90000015:dgl} has the symbol
\[
A
=
\begin{pmatrix}
1&0\\
0&1
\end{pmatrix}
\qquad\Rightarrow\qquad
\det A=1>0
\]
and thus it is elliptic.
The solution of the differential equation in a given point is uniquely
determined if boundary values are prescribed for all boundary points.
This property distinguishes the elliptic partial differential equations
fundamentally from hyperbolic ones.
For the latter the characteristics through the point determine the
part of the boundary that can influence the value of the solution.

Unfortunately, this is not enough.
The domain in this problem is not bounded, but theorem 7.2 in the
lecture notes and the theorems regarding uniqueness of the solution
derived from it only hold for bounded domains.

In fact the function $r=\sqrt{x^2+y^2}$ is a solution because
\[
\frac{\partial^2 r}{\partial r^2} + \frac{\partial^2 r}{\partial \varphi^2}=0.
\]
% XXX
For each solution $u$ that satisfies the boundary conditions on
the complete border $r=1$, we can add any of the functions $a(r-1)$, which
satisfy homogeneous boundary conditions and the homogeneous equation.
In this manner, we can find infinitely many different solutions
of the partial differential equation.

The problem is that the functions $a(r-1)$ increase beyond all bounds
for $r\to\infty$.
To guarantee uniqueness of the solution, we therefore need to add a
condition that prevents the solution from increasing, e.~g.~by requiring
that $\lim_{r\to\infty}u=0$.

Alterantively we could argue that Green's function guarantees uniqueness
of the solution.
But the construction of Green's function requires that the solution must
be unique in the first place.
This, again depends on a condition at infinity.
Such a condition is often given implicitely, so that the argument
could be made to work.

A third attempt at answering the question could be to try to solve
the equation explictly.
Use the separation ansatz
\[
u(r,\varphi) = R(r) \cdot \Phi(\varphi).
\]
The separation gives the equations
\begin{align*}
\Phi''(\varphi)&=-\mu^2\Phi(\varphi)
\\
R''(r)&=\mu^2R(r).
\end{align*}
The separation are already chosen in such a way that the solutions
of the $\Phi$-equation can be written in a simple form.
One finds
\[
\Phi(\varphi) =
\begin{cases}
\sin\mu\phi\\
\cos\mu\phi
\end{cases}
\]
This can only be a solution, if $\Phi$ is $2\pi$-periodic, so that
$\mu$ must be an integer, which we will write $n$ in the following.

For the function $R(r)$ we find
\[
R(r) = e^{\pm nr}.
\]
The solution of the differential equation thus has the general form
\[
u(r,\varphi)
=
\frac{a_0}2
+
\sum_{n=1}^\infty
e^{-nr}(a_n^- \cos n\varphi + b_n^- \sin n\varphi)
+
\sum_{n=1}^\infty
e^{+nr}(a_n^+ \cos n\varphi + b_n^+ \sin n\varphi).
\]
The coefficients 
$a_0$, $a_n^\pm$ and $b_n^\pm$ still need to be determined.

This can be in two different ways.

\begin{enumerate}
\item
We could specify the values of $u$ and the normal derivatives at $r=1$.
The normal derivative coincides with the derivative with respect to $r$.
Given this data,
\begin{align*}
u(1,\varphi)
&=
\frac{a_0}2
+
\sum_{n=1}^\infty ((a_n^-e^{-n}+a_n^+e^n)\cos n\varphi + (b_n^-e^{-n}+b_n^+e^n)\sin n\varphi)
\\
\frac{\partial u}{\partial r}(1,\varphi)
&=
\sum_{n=1}^\infty ((-na_n^-e^{-n}+na_n^+e^n)\cos n\varphi + (-nb_n^-e^{-n}+nb_n^+e^n)\sin n\varphi).
\end{align*}
This boundary data thus determines
$a_n^-e^{-n}\pm a_n^+e^n$ and $b_n^-e^{-n}\pm b_n^+e^n$
and hence all coefficients.
\item
We could specify a condition at infinity that exclude the terms of the
form $e^{nr}$.
This would lead to the coefficients $a_n^+$ and $b_n^+$ being
excluded.
Only the coefficients $a_n^-$ and $b_n^-$ would still need to be
determined, which could again be done based on boundary values 
$u(1,\varphi)$.
\qedhere
\end{enumerate}
\end{loesung}

