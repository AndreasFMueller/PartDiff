From the Navier-Stokes-equation one can derive that the
flow velocity $v(r,\varphi)$ of a laminar stationary flow of an
incompressible fluid through a tube with circular cross section
of radius $1$ satisfies the partial differential equation
\[
\Delta v=c
\]
on the tube cross section described in polar coordiantes as the domain
\[
\Omega = \{ (r,\varphi)\,|\,0 < r < 1,0 < \varphi < 2\pi\}.
\]
\begin{teilaufgaben}
\item
The fluid attaches to the wall.
Formulate this as a boundary condition for the function $v(r,\varphi)$.
\item
Find a solution for $v(r,\varphi)$.
({\it Hint:} try a solution that depends polynomially on $r$).
\item
Is this the only solution?
\end{teilaufgaben}

\begin{loesung}
\begin{teilaufgaben}
\item
If the fluid sticks to the wall, then the flow velocity has to 
vanish at the bounary.
This can be written as
\[
v(1,\varphi)=0,\qquad\forall 0\le \varphi\le 2\pi.
\]
\item
The Laplace operator in polar coordinates is
\[
\Delta
=
\frac1r\frac{\partial}{\partial r}r\frac{\partial}{\partial r}
+\frac1{r^2}\frac{\partial^2}{\partial\varphi^2}.
\]
We are looking for a particular solution $v_p(r,\varphi)$.
We can assume that the solution is rotationally symmetric so that
$v_p$ does not depend on $\varphi$, 
$v_p(r,\varphi)=v_p(r)$.
This leads to
\[
\Delta v=
\frac1r\frac{\partial}{\partial r}r
\frac{\partial v_p}{\partial r}
+\frac1{r^2}\frac{\partial^2v_p}{\partial\varphi^2}
=
\frac1r\frac{\partial}{\partial r}rv_p'(r)
=
\frac1rv_p'(r)+v_p''(r)=c.
\]
Following the hint, we are looking for a solution in the form of a
polynomial.
For a monomial $r^k$ it follows that
\[
\left(\frac1r\frac{d}{dr}+\frac{d^2}{dr^2}\right)r^k
=kr^{k-2}+k(k-1)r^{k-2}
=k^2 r^{k-2}.
\]
The only monomial that has a constant derivative is $r^2$.
So we substitute $v(r)=ar^2$ in the differential equation and compute
\[
\frac1r2ar+2a
=
4a
=c
\quad
\Rightarrow
\quad
a=\frac{c}{4}.
\]
This particular differential equation thus turns into
\[
v_p(r,\varphi)=\frac{c}{4}r^2.
\]
But this function does not satisfy the boundary conditions because of
$v_p(1,\varphi)=\frac{c}{4}$.

We thus need a solution $v_h(r,\varphi)$ of the homogenous equation
\[
\Delta v_h=0
\]
with boundary condition $v_h(1,\varphi)=-\frac{c}{4}$.
The constant function
\[
v_h(r,\varphi)=-\frac{c}{4}
\]
does this.
This gives the final solution
\[
v(r,\varphi)=v_p(r,\varphi)+v_h(r,\varphi)=\frac{c}{4}r^2 -\frac{c}{4}=
\frac{c}{4}(r^2-1).
\]
\item
An elliptic partial differential equation like
$\Delta u=c$ with Dirichlet boundary conditions only has a single 
solution.
\qedhere
\end{teilaufgaben}
\end{loesung}
