Aus der Navier-Stokes-Gleichung kann man ableiten, dass die
Strömungsgeschwindigkeit $v(r,\varphi)$ einer laminaren,
stationären Strömung einer inkompressiblen Flüssigkeit
durch ein Rohr mit kreisförmigem Querschnitt mit Radius 1
die partielle Differentialgleichung
\[
\Delta v=c
\]
auf dem in Polarkoordinaten beschriebenen Gebiet des Rohrquerschnittes
\[
\Omega = \{ (r,\varphi)\,|\,0 < r < 1,0 < \varphi < 2\pi\}
\]
erfüllt.
\begin{teilaufgaben}
\item Die Flüssigkeit haftet an der Wand. Formulieren Sie dies
als Randbedingungen für die Funktion $v(r,\varphi)$
\item Finden Sie eine Lösung für $v(r,\varphi)$. ({\it Hinweis:}
die Lösung hängt polynomiell vom Radius ab.)
\item Ist dies die einzige mögliche Lösung?
\end{teilaufgaben}

\begin{loesung}
\begin{teilaufgaben}
\item Dass die Flüssigkeit an der Wand des Rohres haftet, hat zur
Folge, dass dort die Geschwindigkeit verschwindet, es ist also
\[
v(1,\varphi)=0,\qquad\forall 0\le \varphi\le 2\pi.
\]
\item Der Laplace-Operator in Polarkoordinaten lautet:
\[
\Delta
=
\frac1r\frac{\partial}{\partial r}r\frac{\partial}{\partial r}
+\frac1{r^2}\frac{\partial^2}{\partial\varphi^2}
\]
Wir suchen zunächst eine partikuläre Lösung $v_p(r,\varphi)$.
Wir können davon ausgehen, dass die Lösung rotationssymmetrisch
sein wird, also $v_p$ gar nicht von $\varphi$ abhängen wird,
$v_p(r,\varphi)=v_p(r)$. Damit bekommen wir
\[
\Delta v=
\frac1r\frac{\partial}{\partial r}r
\frac{\partial v_p}{\partial r}
+\frac1{r^2}\frac{\partial^2v_p}{\partial\varphi^2}
=
\frac1r\frac{\partial}{\partial r}rv_p'(r)
=
\frac1rv_p'(r)+v_p''(r)=c.
\]
Gemäss Hinweis ist ein Polynom gesucht. Für ein Monom $r^k$
gilt
\[
\left(\frac1r\frac{d}{dr}+\frac{d^2}{dr^2}\right)r^k
=kr^{k-2}+k(k-1)r^{k-2}
=k^2 r^{k-2}
\]
Das einzige Monom, dessen so berechnete ``Ableitung'' konstant wird, ist
$r^2$.
Wir
setzen also den Ansatz $v(r)=ar^2$
in die Differentialgleichung ein und berechnen
\[
\frac1r2ar+2a
=
4a
=c
\quad
\Rightarrow
\quad
a=\frac{c}{4}
\]
Diese partikuläre Lösung lautet also
\[
v_p(r,\varphi)=\frac{c}{4}r^2.
\]
Sie erfüllt allerdings wegen $v_p(1,\varphi)=\frac{c}{4}$ die
Randbedingung nicht. Wir brauchen daher noch eine
Lösung $v_h(r,\varphi)$ der Differentialgleichung
\[
\Delta v_h=0
\]
mit der Randbedingung $v_h(1,\varphi)=-\frac{c}{4}$. Die konstante
Funktion
\[
v_h(r,\varphi)=-\frac{c}{4}
\]
leistet dies. Somit ist die endgültige Lösung:
\[
v(r,\varphi)=v_p(r,\varphi)+v_h(r,\varphi)=\frac{c}{4}r^2 -\frac{c}{4}=
\frac{c}{4}(r^2-1).
\]
\item Eine elliptische PDGL wie $\Delta u=c$ hat mit vorgegebenen
Dirichlet-Randwerten nur eine Lösung. Zur Erinnerung: Der Grund
war das Maximum-Prinzip.
Gäbe es nämlich eine zweite Lösung $\bar v(r,\varphi)$ mit gleichen
Randwerten, wäre
$v-\bar v$ eine Lösung der Gleichung
\[
\Delta (v-\bar v)=0
\]
also eine harmonische Funktion. Die Randwerte von $v-\bar v$ sind $0$.
Da eine harmonische Funktion das Maximum auf dem Rand annimmt, ist
$v-\bar v=0$, die  Lösung ist also eindeutig.
\qedhere
\end{teilaufgaben}
\end{loesung}
