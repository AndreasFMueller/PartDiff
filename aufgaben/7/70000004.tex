Beweisen Sie folgende Aussage: eine nicht konstante harmonische Funktion in der
Ebene hat weder ein Maximum noch ein Minimum.

\begin{loesung}
H"atte die Funktion ihr Maximum im Punkt $x_0$, dann m"usste
sie auch ein Maximum in dem Kreisgebiet um $x_0$ mit Radius $r$
haben. Das Maximumprinzip sagt aber, dass die Funktion dann das
Maximum auf dem Rand des Kreises haben m"usste. Das geht nur,
wenn die Funktion auf dem Kreis dieselben Werte hat wie im
Punkt $x_0$, wenn sie also konstant ist. Nach Voraussetzung ist
sie das aber nicht, also kann es kein solches Maximum, und analog
auch kein Minimum geben.
\end{loesung}

