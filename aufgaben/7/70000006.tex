Verwenden Sie die Greensche Funktion f"ur das Interval $I=[-1,1]$,
um die Differentialgleichung
\[
u''(x)=1-x^2
\]
auf $I$ mit Randbedingungen $u(-1)=u(1)=0$
zu l"osen.

\begin{loesung}
In der Vorlesung wurde die Greensche Funktion f"ur das Interval
$[0,1]$ konstruiert. Als erstes m"ussen wir daher die Greensche
Funktion f"ur das Interval $[-1,1]$ anpassen. Dies kann auf
verschiedene Arten geschehen.
Einerseits kann man die L"osung aus dem
Skript nehmen, und durch Skalieren in die richtige Form bringen.
Man k"onnte auch einfach die Rechnung im Skript nochmals f"ur
das Interval $[-1,1]$ durchf"uhren. Drittens kann man
die Greensche Funktion aus singul"anren L"osungen bestimmen.

In der Vorlesung wurde gezeigt, dass die Greensche Funktion
aus den singul"aren L"osungen $\sigma(x,\xi)$ berechnet werden
kann. Dazu muss man $\sigma(x,\xi)$ mit einer in $x$ harmonischen
Funktion erg"anzen, so dass die Summe die Randbedingung erf"ullt.
Im vorliegenden Fall ist also $u_h$ f"ur
\[
G(x,\xi)=\sigma(x,\xi)+u_h(x,\xi)
=
{\textstyle\frac12}|x-\xi|+u_h(x,\xi)
\]
zu suchen, und $u_h$ muss harmonisch sein. In einer Dimension
heisst das,
$u_h(x,\xi)=ax+b$, wobei $a$ und $b$ so zu bestimmen sind, dass die
Randbedingung erf"ullt wird.
\begin{align*}
G(-1,\xi)&={\textstyle\frac12}|-1-\xi|-a+b&={\textstyle\frac12}(1+\xi)-a+b&=0\\
G(1,\xi)&={\textstyle\frac12}|1-\xi|+a+b&={\textstyle\frac12}(1-\xi)+a+b&=0
\end{align*}
Aus Summe und Differenz dieser beiden Gleichungen erh"alt man
schliesslich
\begin{align*}
1+2b&=0&\Rightarrow\qquad b&=\frac12\\
\xi-2a&=0&\Rightarrow\qquad a&=\frac{\xi}2
\end{align*}
Daraus folgt f"ur die Greensche Funktion
\begin{align*}
G(x,\xi)
&=
{\textstyle\frac12}(|x-\xi|+x\xi+1)
\\
&=
\begin{cases}
{\textstyle\frac12}(x-\xi+x\xi + 1)
&\qquad x>\xi\\
{\textstyle\frac12}(\xi-x+x\xi + 1)
&\qquad x<\xi
\end{cases}
\end{align*}
Diese Form ist f"ur die Berechnung der L"osung nicht optimal. Man kann
aber auch schreiben:
\begin{align*}
G(x,\xi)
&=
(x-\xi)\vartheta(x-\xi)-{\textstyle\frac12}(x+1)(1-\xi)
\\
&=
\begin{cases}
x-\xi
+
{\textstyle\frac12}(\xi-x+x\xi-1)
=
{\textstyle\frac12}(x-\xi+x\xi-1)
&\qquad x>\xi\\
{\textstyle\frac12}(\xi-x+x\xi-1)
&\qquad x<\xi
\end{cases}
\end{align*}
Damit ist die Greensche Funktion gefunden.

Mit dieser Greenschen Funktion kann jetzt die L"osung der
Differentialgleichung berechnet werden:
\begin{align*}
u(x)&=\int_{-1}^x
(x-\xi)\vartheta(x-\xi)-{\textstyle\frac12}(x+1)(1-\xi)
)(1-\xi^2)\,d\xi
\\
&=
\int_{-1}^1(x-\xi)\vartheta(x-\xi)(1-\xi^2)\,d\xi
-{\textstyle\frac12}(x+1)\int_{-1}^1(1-\xi)(1-\xi^2)\,d\xi
\\
&=
\int_{-1}^x(x-\xi)(1-\xi^2)\,d\xi
-{\textstyle\frac12}(x+1)\int_{-1}^1(1-\xi)(1-\xi^2)\,d\xi
\\
&=
-\frac1{12}(x^4-6x^2+5)
\end{align*}
wobei die Berechnung des Integrals mit Mathematica erfolgte.
Man kann durch Einsetzen nachpr"ufen, dass
$u(-1)=u(1)=0$ und dass $u''(x)=1-x^2$, die Greensche Funktion
liefert also tats"achlich eine L"osung des Problems.

Nun noch die Berechnung der Greenschen Funktion als Skalierung
der bereits berechneten Funktion. Die Greensche Funktion $G(x,\xi)$ f"ur
das Interval $[0,1]$
$G(x,\xi)$ hat die Eigenschaft, dass die partielle Funktion
$x\mapsto G(x,\xi)$ f"ur jedes $\xi$ die Randbedingungen erf"ullt,
und ausserdem
\[
\frac{\partial^2}{\partial x^2}G(x,\xi)=\delta(x-\xi).
\]
Der Ausdruck $\frac12(x+1)$ bildet $-1$ auf $0$ und $1$ auf $1$
ab, setzt man ihn in die Greensche Funktion $K(x,\xi)$ ein,
die f"ur das Interval $[0,1]$ berechnet wurde,
erh"alt man eine Funktion
\[
G(x,\xi)=K({\textstyle\frac12}(x+1),{\textstyle\frac12}(\xi+1)),
\]
welche
f"ur jeden Wert von $\xi$ die Randbedingung erf"ullt, also
$K(\frac12(-1+1),\frac12(\xi+1))=G(0,\frac12(\xi+1))=0$ und
$G(\frac12(1+1),\frac12(\xi+1))=G(1,\frac12(\xi+1))=0$.
Die Bedingung, dass die zweite Ableitung eine Dirac-Distribution
sein muss, wird jedoch verletzt. Wir "uberlegen uns, wie die
die Skalierung den Wert der Ableitung ver"andert.
Die Ableitung von $G$ nach der ersten Variablen ist:
\begin{align*}
\frac{\partial}{\partial x}G(x,\xi)
&=
\frac{\partial}{\partial x}K({\textstyle\frac12}(x+1),{\textstyle\frac12}(\xi+1))\frac{\partial}{\partial x}\frac12(x+1)
=
\frac12\frac{\partial }{\partial x}K({\textstyle \frac12}(x+1), {\textstyle\frac12}(\xi+1))
\end{align*}
Die Funktion
\[
x\mapsto K(x,\xi)=\begin{cases}
(x-\xi)-x(1-\xi)=\xi(x-1)&\qquad x>\xi
\\
-x(1-\xi)&\qquad x<\xi
\end{cases}
\]
ist in jedem Teilinterval $x>\xi$ und $x<\xi$ eine lineare Funktion. Die Ableitung
ist also eine st"uckweise konstante Funktion.
Die Ableitung davon ist "uberall $0$, ausser an der Stelle $x=\xi$,
insbesondere ist die Ableitung der skalierten Ableitung gleich gross
wie die Ableitung der unskalierten Ableitung, der Faktor $\frac12$ ist
also alles, was kompensiert werden muss.

Die Greensche Funktion f"ur dieses Problem wird damit
\begin{align*}
G(x,\xi)&=
2K({\textstyle\frac12}(x+1),{\textstyle\frac12}(\xi+1))
\\
&=
2
({\textstyle\frac12}(x+1)-\textstyle{\frac12}(\xi+1))
\vartheta({\textstyle\frac12}(x+1)-{\textstyle\frac12}(\xi+1))
-2{\textstyle\frac12}(x+1)(1-{\textstyle\frac12}(\xi+1))
\\
&=
(x-\xi)\vartheta(x-\xi)-{\textstyle\frac12}(x+1)(1-\xi)
\end{align*}
\end{loesung}
