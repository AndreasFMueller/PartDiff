Gegeben ist das Gebiet
\[
\Omega=\{
(x,y)\,|\,\pi < xy < 2\pi, x < \pi,y < \pi
\}
\]
und der Differentialoperator
\[
Lu=
x^2\frac{\partial^2u}{\partial x^2}
+xy\frac{\partial^2u}{\partial x\partial y}
+y^2\frac{\partial^2u}{\partial y^2}
+3x^2y^2u.
\]
\begin{teilaufgaben}
\item Zeichnen Sie das Gebiet $\Omega$.
\item Ist $L$ elliptisch, hyperbolisch oder parabolisch?
\item Zeigen Sie, dass $u(x,y)=\sin(xy)$ eine L"osung der
Differentialgleichung
\[
Lu=xy\cos(xy)
\]
auf dem Gebiet $\Omega$ ist mit Randwerten
\begin{align*}
u(x,y)&=0            &&\text{$xy=\pi$ oder $xy=2\pi$},\\
u(\pi,y)&=\sin(\pi y)&&1\le y\le 2,\\
u(x,\pi)&=\sin(\pi x)&&1\le x\le 2.
\end{align*}
\item Warum ist dies die einzige L"osung im Teilgebiet mit $x>0$, $y>0$?
\end{teilaufgaben}

\begin{loesung}
\begin{teilaufgaben}
\item
Das Gebiet wird im ersten Quadranten von zwei Hyperbeln und zwei Strecken berandet:
\begin{center}
\includeagraphics[width=0.6\hsize]{bild-1}
\end{center}
Es gibt aber auch noch einen Teil im dritten Quadranten, der mit dem Teilgebiet
im ersten Quadranten nicht verbunden ist. Das Teilgebiet im
dritten Quadranten ist unbeschr"ankt, das ganze Gebiet zwischen
den beiden Hyperbeln ist eingschlossen.
\item
Die Symbolmatrix ist
\[
A=\begin{pmatrix}
x^2&\frac12xy\\
\frac12xy&y^2
\end{pmatrix}
\]
Um zu beurteilen, ob die beiden Eigenwerte gleiches Vorzeichen haben,
gen"ugt es, die Determinante zu berechnen:
\[
\det(A)=x^2y^2-\frac14x^2y^2=\frac34x^2y^2>0,
\]
der Operator ist also elliptisch.
\item
Es muss gepr"uft werden, ob die Differentialgleichung erf"ullt ist,
und ob die Randbedingungen erf"ullt sind.
Die partiellen Ableitungen sind
\begin{align*}
\frac{\partial u}{\partial x}&=y\cos(xy)
&
\frac{\partial^2 u}{\partial x^2}&=-y^2\sin(xy)
\\
&&\frac{\partial^2 u}{\partial x\partial y}&=\cos(xy)-xy\sin(xy)
\\
\frac{\partial u}{\partial y}&=x\cos(xy)
&
\frac{\partial^2 u}{\partial y^2}&=-x^2\sin(xy)
\end{align*}
Die Differentialgleichung wird damit zu
\begin{align*}
Lu&=-x^2y^2\sin(xy)+xy(\cos(xy)-xy\sin(xy))-x^2y^2\sin(xy)+3x^2y^2\sin(xy)\
\\
&=xy\cos(xy),
\end{align*}
ist also erf"ullt.
F"ur die Randbedingungen gilt an den gekr"ummten Randst"ucken
\begin{align*}
xy&=\pi&\Rightarrow\quad u(x,y)&=\sin(xy)=\sin\pi=0\\
xy&=2\pi&\Rightarrow\quad u(x,y)&=\sin(xy)=\sin2\pi=0
\end{align*}
und an den geraden St"ucken
\begin{align*}
u(\pi,y)&=\sin(\pi y)
&u(x,\pi)&=\sin(x\pi),
\end{align*}
die Randbedingungen
sind also auch erf"ullt. Damit ist $u$ eine L"osung der PDGL.
\item
F"ur L"osungen elliptischer Differentialgleichungen gilt das Maximumprinzip.
Ein weitere L"osung $\bar u(x,y)$ der Gleichung erg"abe
$v=u-\bar u$ mit
$Lu=0$, wobei $v$ auf dem Rand des Gebietes verschwindet.
Da $v$ dort das Maximum haben m"usste, folgt $v=0$ oder $u=\bar u$,
es gibt also gar keine andere L"osung.

Allerdings gilt dieses Argument nur f"ur beschr"ankte Gebiete, also nur
f"ur den Teil im ersten Quadranten. Dort ist die L"osung tas"achlich
eindeutig. F"ur das Teilgebiet im dritten Quadranten kann es
jedoch noch weitere L"osungen geben.
\qedhere
\end{teilaufgaben}
\end{loesung}
