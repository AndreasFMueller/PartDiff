In the lectures, Green's function for the problem
\[
\begin{aligned}
\Delta u&= f & &\text{in $\Omega$}\\
       u&= g & &\text{auf $\partial\Omega$}
\end{aligned}
\]
on the one dimensional domain
$\Omega=(0,1)$ was derived.
It turned out to be
\[
G(x,\xi) = (x-\xi)\vartheta(x-\xi) -x(1-\xi).
\]
Use Green's function to solve the problem
\[
\begin{aligned}
u''(x)& = e^x& &\text{in $(0,1)$}\\
  u(0)& =  20& &\\
  u(1)& =  16.& &
\end{aligned}
\label{70000010:problem}
\]

\begin{loesung}
Green's function finds the solution of the problem~\eqref{70000010:problem}
via the formula
\begin{equation}
u(x)
=
\int_0^1 G(x,\xi) f(\xi)\,d\xi
+
\int_{\partial[0,1]}\frac{\partial G}{\partial \xi} g(\xi)\,dn.
\label{70000010:green}
\end{equation}
(Theorem~7.8 in the lecture notes). 

The left hand integral is
\begin{align*}
\int_0^1 G(x,\xi) f(\xi)\,d\xi
&=
\int_0^1 ( (x-\xi)\vartheta(x-\xi) -x(1-\xi)) e^{\xi} \,d\xi
\\
&=
\int_0^1 (x-\xi)\vartheta(x-\xi) e^{\xi} \,d\xi
-
\int_0^1 x(1-\xi) e^{\xi} \,d\xi
\\
&=
\int_0^x (x-\xi) e^{\xi} \,d\xi
-
\int_0^1 x(1-\xi) e^{\xi} \,d\xi.
\end{align*}
In this expression, the integrands
$e^\xi$ and $\xi e^\xi$ have the following antiderivatives:
\begin{align*}
\int e^\xi\,d\xi&=e^\xi + C\\
\int \xi e^\xi\,d\xi&=\xi e^\xi - \int e^\xi\,d\xi = (\xi - 1)e^\xi + C.
\end{align*}
We can use these to compute the integrals:
\begin{align*}
\int_0^1 G(x,\xi) f(\xi)\,d\xi
&=
\left[
xe^\xi -(\xi - 1)e^\xi
\right]_0^x
-
x\left[
e^\xi-(\xi - 1)e^\xi
\right]_0^1
\\
&=
\left[
xe^x-(x-1)e^x
-
(x-(-1))
\right]
-
x
\left[
e - (1-(-1))
\right]
\\
&=
e^x-x-1
-
x(e-2)
\\
&=
e^x-x-1
-xe+2x
\\
&=
e^x+x-xe-1.
\end{align*}
As expected, this function does not satisfy the boundary conditions.
When we substitute the values $x=0$ and $x=1$, we get
\begin{align*}
\int_0^1 G(0,\xi) f(\xi)\,d\xi
&=0\qquad\text{and}
\\
\int_0^1 G(1,\xi) f(\xi)\,d\xi
&=0.
\end{align*}
The correct boundary conditions are satisfied by the second integral
in \eqref{70000010:green}.
To compute it, we first need to find the derivative of Green's function
at least at the positions $\xi=0$ and $\xi=1$.
We get
\begin{align*}
\frac{\partial G}{\partial\xi}
&=(-1)\vartheta(x-\xi) - x(-1)
=\begin{cases}
-1+x=-(1-x)
\qquad&\text{for $\xi=0$}
\\
x
\qquad&\text{for $\xi=1$.}
\end{cases}
\end{align*}
Substituting this in the second integral of formula~\eqref{70000010:green},
we get
\begin{align*}
\int_{\partial[0,1]}\frac{\partial G}{\partial \xi} g(\xi)\,dn
&=
\frac{\partial G(x,1)}{\partial\xi} g(1)
-
\frac{\partial G(x,0)}{\partial\xi} g(0)
=
xg(1) + (1-x)g(0)
=
(1-x)a
+
xb.
\end{align*}
The minus sign in the second expression stems from the fact that
the normal $n$ needs to point outward. 
At $\xi=0$, this is the negative direction, $n=-1$.
All combined we now can write down the solution:
\[
u(x)
=
e^x+x(1-e)-1
+20(1-x)+16x.
\]
To verify the solution, we compute:
\begin{align*}
u'(x)
&=
e^x+(1-e)-20+16,
\\
u''(x)
&=
e^x,
\\
u(0)
&=
20,
\\
u(1)
&=
16,
\end{align*}
so $u$ really is the solution of the problem posed.
\end{loesung}

