In der Vorlesung wurde die Greensche Funktion für das Problem
\[
\begin{aligned}
\Delta u&= f & &\text{in $\Omega$}\\
       u&= g & &\text{auf $\partial\Omega$}
\end{aligned}
\]
für das eindimensionale Gebiet  $\Omega=[0,1]$ hergeleitet.
Sie lautet 
\[
G(x,\xi) = (x-\xi)\vartheta(x-\xi) -x(1-\xi).
\]
Verwenden Sie die Greensche Funktion, um das Problem
\[
\begin{aligned}
u''(x)& = e^x& &\text{in $(0,1)$}\\
  u(0)& =  20& &\\
  u(1)& =  16& &
\end{aligned}
\label{70000010:problem}
\]
zu lösen.

\begin{loesung}
Die Greensche Funktion liefert die Lösung für das
Problem~\eqref{70000010:problem} über die Formel
\begin{equation}
u(x)
=
\int_0^1 G(x,\xi) f(\xi)\,d\xi
+
\int_{\partial[0,1]}\frac{\partial G}{\partial \xi} g(\xi)\,dn
\label{70000010:green}
\end{equation}
(Satz~7.8 im Skript). 

Das linke Integral ist
\begin{align*}
\int_0^1 G(x,\xi) f(\xi)\,d\xi
&=
\int_0^1 ( (x-\xi)\vartheta(x-\xi) -x(1-\xi)) e^{\xi} \,d\xi
\\
&=
\int_0^1 (x-\xi)\vartheta(x-\xi) e^{\xi} \,d\xi
-
\int_0^1 x(1-\xi) e^{\xi} \,d\xi
\\
&=
\int_0^x (x-\xi) e^{\xi} \,d\xi
-
\int_0^1 x(1-\xi) e^{\xi} \,d\xi.
\end{align*}
In diesem Ausdruck kommen die Integranden $e^\xi$ und $\xi e^\xi$ vor,
die die folgenden Stammfunktionen haben:
\begin{align*}
\int e^\xi\,d\xi&=e^\xi + C\\
\int \xi e^\xi\,d\xi&=\xi e^\xi - \int e^\xi\,d\xi = (\xi - 1)e^\xi + C.
\end{align*}
Dies kann man jetzt verwenden, um die Integrale zu berechnen:
\begin{align*}
\int_0^1 G(x,\xi) f(\xi)\,d\xi
&=
\left[
xe^\xi -(\xi - 1)e^\xi
\right]_0^x
-
x\left[
e^\xi-(\xi - 1)e^\xi
\right]_0^1
\\
&=
\left[
xe^x-(x-1)e^x
-
(x-(-1))
\right]
-
x
\left[
e - (1-(-1))
\right]
\\
&=
e^x-x-1
-
x(e-2)
\\
&=
e^x-x-1
-xe+2x
\\
&=
e^x+x-xe-1.
\end{align*}
Diese Funktion erfüllt wie zu erwarten die Randbedingungen nicht, denn
wenn wir die Werte $x=0$ und $x=1$ einsetzen, erhalten wir
\begin{align*}
\int_0^1 G(0,\xi) f(\xi)\,d\xi
&=0\qquad\text{und}
\\
\int_0^1 G(1,\xi) f(\xi)\,d\xi
&=0.
\end{align*}
Die korrekten Randbedingungen werden durch das zweite Integral in
\eqref{70000010:green} erfüllt, für die wir jedoch erst die
Ableitung der Greenschen Funktion wenigstens an den Stellen $0$ und $1$
berechnen müssen:
\begin{align*}
\frac{\partial G}{\partial\xi}
&=(-1)\vartheta(x-\xi) - x(-1)
=\begin{cases}
-1+x=-(1-x)
\qquad&\text{für $\xi=0$}
\\
x
\qquad&\text{für $\xi=1$.}
\end{cases}
\end{align*}
Setzt man dies in das zweite Integral der Formel~\eqref{70000010:green} ein,
erhält man
\begin{align*}
\int_{\partial[0,1]}\frac{\partial G}{\partial \xi} g(\xi)\,dn
&=
\frac{\partial G(x,1)}{\partial\xi} g(1)
-
\frac{\partial G(x,0)}{\partial\xi} g(0)
=
xg(1) + (1-x)g(0)
=
(1-x)a
+
xb.
\end{align*}
Das Minuszeichen im zweiten Ausdruck rührt daher, dass die die Normale $n$
nach aussen zeigen muss, bei $\xi=0$ als in die negative Richtung, $n=-1$.
Zusammen kann man jetzt die Lösung hinschreiben:
\[
u(x)
=
e^x+x(1-e)-1
+20(1-x)+16x.
\]
Zur Kontrolle rechnen wir nach:
\begin{align*}
u'(x)
&=
e^x+(1-e)-20+16,
\\
u''(x)
&=
e^x,
\\
u(0)
&=
20,
\\
u(1)
&=
16,
\end{align*}
somit ist $u$ tatsächlich die Lösung des gestellten Problems.
\end{loesung}

