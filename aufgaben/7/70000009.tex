XXX
Die partielle Differentialgleichung
\begin{equation}
(2-x^2)\frac{\partial^2 u}{\partial x^2}
+
2x^2y^2\frac{\partial^2 u}{\partial x\partial y}
+
(2-y^2)\frac{\partial^2 u}{\partial y^2}
=0
\label{70000008:pdgl}
\end{equation}
soll im Gebiet $\Omega=\{(x,y)\,|\, 0< x<1,\; 0<y<1\}$ gel"ost
werden. Auf dem Rand von $\Omega$ soll $u$ die folgenden Werte
annehmen:
\begin{align*}
&
	&u(x,1)&=\sin({\textstyle\frac\pi4}(1-x))
		&&
			&&\text{f"ur $0<x<1$}\\
u(0,y)&=\sin({\textstyle\frac\pi4}y)
	&&
		&u(1,y)&=\sin({\textstyle\frac\pi4}(1-y))
			&&\text{f"ur $0<y<1$}\\
&
	&u(x,0)&=\sin({\textstyle\frac\pi4} x)
		&&                
			&&\text{f"ur $0<x<1$}
\end{align*}
Bei der numerischen Berechnung der L"osung wird ein Wert
$u(\frac12,\frac12)=0.72$ gefunden. Kann das sein?

\begin{hinweis}
Versuchen Sie nicht, die partielle Differentialgleichung zu l"osen.
\end{hinweis}

\begin{loesung}
Eine Aussage "uber die Gr"osse einzelner Werte l"asst sich f"ur
elliptische partielle Differentialgleichungen machen, da f"ur diese das
Maximumprinzip gilt.

Wir pr"ufen daher zun"achst, ob eine elliptische
partielle Differentialgleichung vorliegt. Dazu ben"otigen wir die
Symbolmatrix, die in diesem Fall 
\[
A=\begin{pmatrix}
2-x^2 &x^2y^2\\
x^2y^2&2-y^2
\end{pmatrix}
\]
ist. Die partielle Differentialgleichung ist elliptisch, wenn die
beiden Eigenwerte das gleiche Vorzeichen haben, wenn also die Determinante
dieser Matrix positiv ist:
\[
\det A=\underbrace{(2-x^2)}_{> 1}\underbrace{(2-y^2)}_{> 1}-\underbrace{x^4y^4}_{< 1}>0
\]
Man beachte dazu, dass im Inneren des Gebietes, und nur dies ist relevant,
die angegebenen Ungleichungen erf"ullt sind, die Determinante ist also
positiv.
Daraus folgt, dass die partielle Differentialgleichung (\ref{70000008:pdgl})
elliptisch ist.

Das Maximum-Prinzip besagt, dass die Funktionswerte von $u$ im Inneren
des Gebietes die Randwerte nicht "ubersteigen oder unterschreiten
k"onnen. Die Randwerte sind Funktionswerte der Funktion $f(t)=\sin(\frac\pi4t)$
f"ur $0<t<1$. Da die Funktion $\sin(\frac\pi4t)$ monoton wachsend ist, ist das
Maximum $\sin\frac\pi4=\frac{\sqrt{2}}2=0.7071067811865$.
Die L"osung $u$ darf daher im Gebiet $\Omega$ nie gr"osser als
$\frac{\sqrt{2}}2$ werden. Der numerisch gefundene Wert im Punkt
$(\frac12,\frac12)$ widerspricht also dem Maximumprinzip.
\end{loesung}

\begin{diskussion}
Diese Aufgabe ist analog zu \ref{70000005} in der Aufgabensammlung.
\end{diskussion}

\begin{bewertung}
Symbolmatrix ({\bf S}) 1 Punkt,
Determinante und Spur ({\bf D}) 1 Punkt,
Vorzeichen der Eigenwerte ({\bf V}) 1 Punkt,
Klassifikation als elliptische PDGL ({\bf E}) 1 Punkt,
Maximumprinzip ({\bf M}) 1 Punkt,
Widerspruch zum Maximumprinzip ({\bf W}) 1 Punkt.
\end{bewertung}
