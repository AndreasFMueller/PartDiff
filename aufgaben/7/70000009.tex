Die partielle Differentialgleichung
\[
\frac{\partial^2 u}{\partial x^2}
+
2e^{-x}
\frac{\partial^2 u}{\partial x\partial y}
+
\frac{\partial^2 u}{\partial y^2}
=
0
\]
soll im Gebiet $\Omega=\{(x,y)\,|\, 0<x<2, 0<y<2\}$ gel"ost werden.
Auf dem Rand von $\Omega$ soll $u$ die folgenden Werte annehmen:
\begin{align*}
      &    &\quad u(x,2)&=\sqrt{2-x}&\quad       &        &&\text{f"ur $0\le x\le 2$}\\
u(0,y)&=\sqrt{y}&\quad       &        &\quad u(2,y)&=\sqrt{2-y}&&\text{f"ur $0\le y\le 2$}\\
      &    &\quad u(x,0)&=\sqrt{x}    &\quad       &        &&\text{f"ur $0\le x\le 2$}
\end{align*}
Bei der numerischen Berechnung wird ein Wert $u(0.1,0.9)=1.42$ gefunden.
Kann das sein?

\begin{hinweis}
Versuchen Sie nicht, die partielle Differentialgleichung zu l"osen.
\end{hinweis}

\begin{loesung}
Eine Aussage "uber die Gr"osse der Funktionswerte der L"osung in
einzelnen Punkten des Gebietes l"asst sich f"ur elliptische
partielle Differentialgleichungen machen, da f"ur diese das Maximumprinzip
gilt.

Wir pr"ufen daher zun"achst, ob eine elliptische partielle
Differentialgleichung vorliegt. Dazu ben"otigen wir die Symbolmatrix,
die in diesem Fall
\[
A=\begin{pmatrix}
1&e^{-x}\\
e^{-x}&1
\end{pmatrix}
\]
ist. Die Determinante ist
\[
\det(A)=1-e^{-2x}>0
\]
wegen
$e^{-2x}<1$
f"ur $x>0$, die beiden Eigenwerte von $A$ haben also das gleiche Vorzeichen,
die Differentialgleichung ist elliptisch.

Die gr"ossten Randwerte findet man bei $(0,2)$ und $(2,0)$, nach dem
Maximumprinzip
k"onnen auch die Funktionswerte der L"osung diese Werte nicht "ubersteigen.
Die L"osung kann also nirgends gr"osser sein als
$u(0,2)=u(2,0)=\sqrt{2}=1.4142\dots$.
Insbesondere ist $u(0.1,0.9)=1.42>\sqrt{2}$ nicht m"oglich.
\end{loesung}

\begin{bewertung}
Symbolmatrix ({\bf S}) 1 Punkt,
Determinante ({\bf D}) 1 Punkt,
Vorzeichen der Eigenwerte ({\bf V}) 1 Punkt,
Klassifikation als elliptische PDGL ({\bf E}) 1 Punkt,
Maximumprinzip ({\bf M}) 1 Punkt,
Widerspruch zum Maximumprinzip ({\bf W}) 1 Punkt.
\end{bewertung}
