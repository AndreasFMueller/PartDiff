Die partielle Differentialgleichung
\[
\frac{\partial^2 u}{\partial x^2}
+
2e^{-x}
\frac{\partial^2 u}{\partial x\partial y}
+
\frac{\partial^2 u}{\partial y^2}
=
0
\]
soll im Gebiet $\Omega=\{(x,y)\,|\, 0<x<2, 0<y<2\}$ gelöst werden.
Auf dem Rand von $\Omega$ soll $u$ die folgenden Werte annehmen:
\begin{align*}
      &    &\quad u(x,2)&=\sqrt{2-x}&\quad       &        &&\text{für $0\le x\le 2$}\\
u(0,y)&=\sqrt{y}&\quad       &        &\quad u(2,y)&=\sqrt{2-y}&&\text{für $0\le y\le 2$}\\
      &    &\quad u(x,0)&=\sqrt{x}    &\quad       &        &&\text{für $0\le x\le 2$}
\end{align*}
Bei der numerischen Berechnung wird ein Wert $u(0.1,0.9)=1.42$ gefunden.
Kann das sein?

\begin{hinweis}
Versuchen Sie nicht, die partielle Differentialgleichung zu lösen.
\end{hinweis}

\begin{loesung}
Eine Aussage über die Grösse der Funktionswerte der Lösung in
einzelnen Punkten des Gebietes lässt sich für elliptische
partielle Differentialgleichungen machen, da für diese das Maximumprinzip
gilt.

Wir prüfen daher zunächst, ob eine elliptische partielle
Differentialgleichung vorliegt. Dazu benötigen wir die Symbolmatrix,
die in diesem Fall
\[
A=\begin{pmatrix}
1&e^{-x}\\
e^{-x}&1
\end{pmatrix}
\]
ist. Die Determinante ist
\[
\det(A)=1-e^{-2x}>0
\]
wegen
$e^{-2x}<1$
für $x>0$, die beiden Eigenwerte von $A$ haben also das gleiche Vorzeichen,
die Differentialgleichung ist elliptisch.

Die grössten Randwerte findet man bei $(0,2)$ und $(2,0)$, nach dem
Maximumprinzip
können auch die Funktionswerte der Lösung diese Werte nicht übersteigen.
Die Lösung kann also nirgends grösser sein als
$u(0,2)=u(2,0)=\sqrt{2}=1.4142\dots$.
Insbesondere ist $u(0.1,0.9)=1.42>\sqrt{2}$ nicht möglich.
\end{loesung}

\begin{bewertung}
Symbolmatrix ({\bf S}) 1 Punkt,
Determinante ({\bf D}) 1 Punkt,
Vorzeichen der Eigenwerte ({\bf V}) 1 Punkt,
Klassifikation als elliptische PDGL ({\bf E}) 1 Punkt,
Maximumprinzip ({\bf M}) 1 Punkt,
Widerspruch zum Maximumprinzip ({\bf W}) 1 Punkt.
\end{bewertung}
