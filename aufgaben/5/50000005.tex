Einen Stab ist bis zur Zeit $t=0$ mit zwei
W"armereservoirs auf den Temperaturen $-1$ bei $x=-\frac{\pi}2$ und
$1$ bei $x=\frac{\pi}2$ verbunden,
so dass sich ein station"arer Zustand eingestellt hat (siehe
"Ubung 1). Zur Zeit $t=0$ werden die Reservoirs entfernt und der
Stab wird sich selbst "uberlassen. Insbesondere kann durch die
Enden keine W"arme mehr abgeleitet werden. L"osen sie die
W"armeleitungsgleichung
\[
\frac{\partial u}{\partial t}=\frac{\partial^2 u}{\partial x^2}
\]
f"ur diesen Fall.

\begin{hinweis}
Verwenden sie f"ur die
``Dreiecksfunktion''
\[
d(x)
=
\begin{cases}
\displaystyle-2-\frac{2x}{\pi}&\qquad \displaystyle-\frac{\pi}2\le x\\
\displaystyle\frac{2x}{\pi}&\qquad \displaystyle-\frac{\pi}2\le x\le \frac{\pi}2\\
\displaystyle2-\frac{2x}{\pi}&\qquad \displaystyle x\le\frac{\pi}2
\end{cases}
\]
die folgende nur Sinus-Funktionen
verwendende Fourier-Reihe
\[
d(x)=\sum_{n=0}^\infty \frac{8(-1)^n}{\pi^2(2n+1)^2}\sin (2n+1)x
\]
\end{hinweis}

\begin{hinweis}
Diese Aufgabe wurde bereits als \ref{40000005} mit der Separationsmethode
gel"ost.
\end{hinweis}

\begin{loesung}
Die L"osung mit der Transformationsmethode
verwendet Fourier-Transformation
entlang der $x$-Achse und Laplace-Transformation entlang der Zeit-Achse.
Wir schreiben $\hat u(t,k)$ f"ur die Fourier-Transformation und ${\cal L} u$
f"ur die Laplace-Transformation.

Die Anfangsbedingungen sind ungerade, so dass auch die L"osung der
Differntialgleichung f"ur alle Zeiten ungerade sein wird. Die
Randbedingung sagen ausserdem, dass
$\partial_xu(t,-\frac{\pi}2)=\partial_xu(t,\frac{\pi}2)=0$ gilt.
Dies bedeutet, dass man die Funktion durch Spiegelung an
$-\frac{\pi}2$ bzw.~$\frac{\pi}2$ zu einer $2\pi$-periodischen
Funktion auf ganz $\mathbb R$ fortsetzen kann. Insbesondere l"asst
sich $u(t,x)$ mit einer Sinusreihe entwickeln. Seien $\hat u(t,k)$
die Fourier-Sinus-Koeffizienten. Damit haben wir als Differentialgleichung
der $\hat u(t,k)$
\[
\partial_t\hat u(t,k)=-k^2\hat u(t,k)
\]
Diese Gleichung kann man jetzt nach Laplace transformieren:
\begin{align*}
s{\cal L}\hat u(s,k) - \hat u(0,k)&=-k^2 {\cal L}\hat u(s,k)\\
(s+k^2){\cal L}\hat u(s,k)&=\hat u(0,k)\\
{\cal L}\hat u(s,k)&=\frac{\hat u(0,k)}{s+k^2}
\end{align*}
R"ucktransformation ergibt:
\[
\hat u(t,k)=\hat u(0,k) e^{-k^2t}
\]
Jetzt sind nur noch die Fourierkoeffizienten zu bestimmen, die kann
man aber dem Hinweis entnehmen:
\[
\hat u(0,2n+1)=
\frac{8(-1)^n}{\pi^2(2n+1)^2}
\]
und damit die endg"ultige L"osung durch Summieren der Fourierreihen bekommen:
\[
u(t,x)=
\sum_{n=0}^\infty \frac{8(-1)^n}{\pi^2(2n+1)^2}e^{-(2n+1)^2t}\sin(2n+1)x
\]
Eine graphische Darstellung der L"osung ist in Abbildung~\ref{50000005:bild}
zu finden.
\begin{figure}
\begin{center}
\includeagraphics[width=\hsize]{a3.pdf}
\end{center}
\caption{L"osung der W"armeleitungsgleichung (Aufgabe~\ref{50000005})
\label{50000005:bild}}
\end{figure}
\end{loesung}
