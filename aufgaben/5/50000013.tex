Gegeben ist die Differentialgleichung
\begin{equation}
\frac{\partial^3 u}{\partial t^3}
+
\frac{\partial u}{\partial x}
=
0
\label{50000013:dgl}
\end{equation}
auf dem Gebiet
\[
\Omega = \{ (x,t)\,|\, t>0\wedge x > 0\}
\]
mit Randbedingungen
\begin{equation}
\begin{aligned}
u(x,0)&=0,
&
\frac{\partial u}{\partial t}(x,0)
&=
0,
&
\frac{\partial^2 u}{\partial t^2}(x,0)
&=0,
\\
u(0,t)&=1.
\end{aligned}
\end{equation}
\begin{teilaufgaben}
\item
Führen Sie Laplace-Transformation nach der Variablen $t$ durch und
stellen Sie eine Differentialgleichung für ${\cal L}u(x,s)$ samt
Anfangsbedingungen auf.
\item
Bestimmen Sie die Funktion ${\cal L}u(x,s)$.
\end{teilaufgaben}

\begin{hinweis}
Die gewöhnliche Differentialgleichung $y'+ay=b$ mit Anfangsbedingung $y(0)=c$
hat die allgemeine Lösung
\[
y(x) = \biggl(c-\frac{b}{a}\biggr)e^{-ax} + \frac{b}{a}.
\]
\end{hinweis}

\begin{loesung}
\begin{teilaufgaben}
\item
Die Laplace-Transformation der Differentialgleichung \eqref{50000013:dgl}
ergibt
\begin{align*}
s^3 {\cal L}u(x,s)
-s^2 u(x,0)
-s\frac{\partial u}{\partial t}(x,0)
-\frac{\partial^2 u}{\partial t^2}(x,0)
+
\frac{\partial}{\partial x}{\cal L}u(x,s)
&=
0.
\end{align*}
Die Randbedingungen besagen, dass der zweite und dritte Term wegfallen
und dass der vierte Term $=1$ ist.
Die Differentialgleichung wird damit zu
\begin{align*}
s^3 {\cal L}u(x,s)
-1
+
\frac{\partial}{\partial x}{\cal L}u(x,s)
&=
0
\\
\Rightarrow\qquad
\frac{\partial}{\partial x}{\cal L}u(x,s)
+
s^3 {\cal L}u(x,s)
&=
1.
\end{align*}
Dies ist eine mit $s$ parametrisierte gewöhnliche Differentialgleichung
erster Ordnung für die Funktion ${\cal L}u(x,s)$.

Die Anfangsbedingung für diese Differentialgleichung erhält man durch
Laplace-Transformation der Randbedingung $u(0,t)=1$, die Laplace-Transformation
der Konstanten ist
\[
{\cal L}u(0,s) = \frac1s.
\]

Die Differentialgleichung wird etwas übersichtlicher, wenn man
$y_s(x) = {\cal L}u(x,s)$ schreibt, dann wird sie nämlich zu
\begin{equation}
y_s'(x) + s^3 y_s(x)=1
\qquad\text{mit Anfangsbedingung}\qquad
y_s(0)=\frac1s.
\label{50000013:final}
\end{equation}

\item
Die Differentialgleichung \eqref{50000013:final} ist eine Differentialgleichung
der im Hinweis gegebenen Art mit $a=s^3$, $b=1$ und $c=1/s$.
Die Lösung ist daher
\[
{\cal L}u(x,s)
=
y_s(x)
=
\biggl(\frac1s-\frac1{s^3}\biggr)e^{-s^3x} + \frac{1}{s^3}.
\qedhere
\]
\end{teilaufgaben}
\end{loesung}

