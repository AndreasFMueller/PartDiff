The transformation method depends on the domain matching the requirements
of the transform.
In the following domains, figure out which transformation can be used
to reduce the number of derivatives in a linear partial differential
equation on these domains.
\begin{teilaufgaben}
\item
$\Omega = \{ (x,y) \mid -\pi < x <  \pi\wedge y > 0\}$
\item
$\Omega = \{ (x,y) \mid -\pi < x <  \pi\wedge y > 0\}$ with
functions $u(x,y)$ that satisfy periodic boundary conditions
$u(-\pi,y)=u(\pi,y)$.
\item
Using polar coordinates on the entire plane results in the domain
$\Omega = \{ (\varphi,r) \mid \varphi \in \mathbb{R}\wedge r > 0\}$
with functions $u(\varphi,r)$ satisfying the periodicity condition
$u(\varphi+2\pi,r)=u(\varphi,r)$.
\item
The first quadrant can be described as
$\Omega=\{(x,y)\mid x>0\wedge y>0\}$
in kartesian coordinates or as
$\Omega= \{(\varphi,r)\mid r>0\wedge 0< \varphi<\frac{\pi}2\}$
in polar coordinates.
\end{teilaufgaben}


\begin{loesung}
\begin{teilaufgaben}
\item
The Laplace transform can be used on the variable $y$.
\item
Again, the Laplace transform can be used on the variable $y$.
But in addition, Fourier series can be used for the dependency 
on the variable $x$.
\item
The Laplace transform can be used on on the variable $r$, converting
the problem into an ordinary differential equation for periodic functions.
Alternatively, Fourier series can be used on the $\varphi$-coordinate.
\item
Using kartesian coordinates, both variables $x$ and $y$ can be transformed
using the Laplace transform.
In polar coordinates, only the $r$-coordinate can be transformed using
the Laplace transform.
\qedhere
\end{teilaufgaben}
\end{loesung}

