Betrachten Sie die Differentialgleichung
\begin{equation}
\frac{\partial u}{\partial t}
+
\frac{\partial^4 u}{\partial x^4}
=
0
\end{equation}
auf dem Gebiet
\[
\Omega = \{ (t,x)\,|\, t>0, x\in\mathbb R\}
\]
mit den Bedingungen
\begin{align*}
u(0,x) &= 0,\\
u(t,0) &= 1,\\
\frac{\partial}{\partial x}u(t,0) &=0,\\
\frac{\partial^2}{\partial x^2}u(t,0) &=0,\\
\frac{\partial^3}{\partial x^3}u(t,0) &=0.
\end{align*}
für $x\in\mathbb R$ und $t > 0$.
\begin{teilaufgaben}
\item
Verwenden Sie die Laplace-Transformation und finden Sie eine gewöhnliche
Differentialgleichung samt Anfangsbedingungen, mit deren Hilfe das
Problem gelöst werden kann.
\item
Ist das Problem gut gestellt?
\end{teilaufgaben}

\begin{hinweis}
Es wird nicht verlangt, dass Sie die gewöhnliche Differentialgleichung lösen.
\end{hinweis}

%\begin{hinweis}
%Die Differentialgleichungen $y''''+ay=0$ hat die Lösungen
%\begin{center}
%\begin{tabular}{|
%>{$}l<{$}|
%>{$}l<{$}|
%>{$}l<{$}|
%>{$}l<{$}|
%>{$}l<{$}|}
%\hline
%y(x)&y'(x)&y''(x)&y'''(x)&y''''(x)\\
%\hline
%e^{bx}\sin bx 
%	&be^{bx}(\sin bx +\cos bx)
%		&2b^2e^{bx}\cos bx
%			&2b^3(\cos bx-\sin bx)
%				&-4b^4e^{bx}\sin bx
%\\
%e^{-bx}\sin bx
%	&be^{-bx}(\cos bx-\sin bx)
%		&-2b^2e^{-bx}\cos bx
%			&2b^3(\sin bx+\cos bx)
%				&-4b^4e^{-bx}\sin bx
%\\
%e^{bx}\cos bx 
%	&be^{bx}(\cos bx-\sin bx)
%		&-2b^2e^{bx}\sin bx
%			&-2b^3e^{bx}(\sin bx-\cos bx)
%				&-4b^4e^{bx}\cos bx
%\\
%e^{-bx}\cos bx
%	&-be^{-bx}(\sin bx-\cos bx)
%		&2b^2 e^{-bx}\sin bx
%			&2b^3(\cos bx-\sin bx)
%				&-4b^4e^{-bx}\cos bx
%\\
%\hline
%\end{tabular}
%\end{center}
%\end{hinweis}

\begin{loesung}
\begin{teilaufgaben}
\item
Die Laplace-Transformation der Differentialgleichung ergibt
\[
s{\mathscr{L}u} - u(0,x) + \frac{\partial^4}{\partial x^4}{\mathscr{L}u} = 0.
\]
Der zweite Term fällt wegen der Anfangsbedingungen weg.
Mit der Notation $y_s(x) = \mathscr{L}u(s,x)$ wird die Gleichung zu
\begin{equation}
y_s''''(x)+sy_s(x)=0.
\label{50000015:eq}
\end{equation}

Die Laplacetransformation der übrigen Bedingungen liefert
\begin{align*}
\mathscr{L} u(s,0)&=\frac1s
\\
\frac{\partial}{\partial x}\mathscr{L}u(s,0)&= 0\\
\frac{\partial^2}{\partial x^2}\mathscr{L}u(s,0)&= 0\\
\frac{\partial^3}{\partial x^3}\mathscr{L}u(s,0)&= 0
\end{align*}
Wir suchen also eine Lösungsfunktion $y_s(x)$ für die die Differentialgleichung
\eqref{50000015:eq}, die die Anfangsbedingungen
\begin{equation}
y_s(0)=\frac1s,\qquad
y'_s(0)=0,\qquad
y''_s(0)=0
\qquad\text{und}\qquad
y'''_s(0)=0
\label{50000015:anf}
\end{equation}
erfüllt.
\item
Die Differentialgleichung \eqref{50000015:eq} ist mit den gegebenen
Anfangsbedingungen \eqref{50000015:anf} eindeutig lösbar, also ist
das Problem gut gestellt.
\qedhere
\end{teilaufgaben}
\end{loesung}

\begin{bewertung}
\begin{teilaufgaben}
\item
Laplace-Transformation der DGL ({\bf L}) 1 Punkt,
Vereinfachung $u(0,x)=0$ ({\bf V}) 1 Punkt,
Laplace-Transformation der Anfangsbedingung $u(t,0)=1$ ({\bf A}) 1 Punkt,
Laplace-Transformation für die Anfangsbedingungen für die Ableitungen
({\bf D}) je 1 Punkt,
\item
Problem gut gestellt ({\bf G}) 1 Punkt.
\end{teilaufgaben}
\end{bewertung}
