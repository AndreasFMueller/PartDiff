Auf dem Gebiet $\Omega = \{(x,t)\,|\, x,t > 0\}$ ist die Differentialgleichung
\begin{equation}
\frac{\partial u}{\partial x}
+
\frac{\partial u}{\partial t}
=
1
\label{50000009:dgl}
\end{equation}
gegeben mit den Randbedingungen
\begin{equation}
\begin{aligned}
u(x,0)&=x^2&\text{für}\quad x&>0\\
u(0,t)&=t  &\text{für}\quad t&>0
\end{aligned}
\end{equation}
Finden Sie eine Lösung.
(Aufgabe von Tobias Plüss, HSLU 2015)

\begin{proof}[Lösung mit Laplace-Transformation]
Die Laplace-Transformation der Variablen $t$ verwandelt die
Differentialgleichung \eqref{50000009:dgl} in eine Differentialgleichung
\[
\frac{\partial U}{\partial x}+sU=\frac1s+x^2
\]
für die Funtion $U(x,s)=({\cal L}u)(x,s)$. Wir betrachten Sie
als eine gewöhnliche inhomogene lineare Differentialgleichung,
und lösen Sie mit dem üblichen Verfahren. Dazu muss erst die Lösung 
der homogenen Differentialgleichung
\[
U' + sU = 0
\]
gefunden werden, also
\[
U'=-sU
\qquad\Rightarrow\qquad
U=ke^{-xs}.
\]
Die inhomogene Gleichung wird dann durch Variation der Konstanten $k$ gelöst,
man findet
\[
k' = e^{sx}\biggl(\frac1s+x^2\biggr).
\]
Die Funktion $k$ kann mit Hilfe partieller Integration gefunden werden,
man bekommt
\[
k=e^{sx}\frac{s^2x^2-2sx+s+2}{s^3}+C
\]
und damit für die Lösung
\begin{equation}
U=\frac{s^2x^2-2sx+s+2}{s^3}+Ce^{-sx}.
\label{50000009:allg}
\end{equation}
Die Konstante $C$ ist noch aus der Anfangsbedingung zu bestimmen. Aus
$u(0,t)=t$ bekommt man durch Laplace-Transformation
\[
U(0,s)=\frac1{s^2}.
\]
Durch Einsetzen von $x=0$ in die allgemeine Lösung~\eqref{50000009:allg}
findet man
\[
U(0,s)=\frac{s+2}{s^3}+C=\frac1{s^2}
\qquad\Rightarrow\qquad
C= \frac1{s^2} - \frac{s+2}{s^3}.
\]
Daraus folgt
\[
U(x,s)=\frac{s^2x^2-2sx+s+2}{s^3}
+
e^{-sx}\biggl(\frac1{s^2}-\frac{s+2}{s^3}\biggr).
\]
Dies ist die Laplace-Transformierte der Lösung $u(x,t)$, $U$ muss jetzt
also noch zurücktransformiert werden.
Die Rechnung ergibt
\[
u(x,t)=x^2-2xt+t+t^2-(t-x)^2\theta(t-x),
\]
wobei $\theta(t-x)$ die Heaviside-Funktion ist.
Man kann dies noch etwas vereinfachen:
\begin{align}
u(x,t)
&=
t+(t-x)^2 -(t-x)^2\theta(t-x)
\notag
\\
&=
t+(t-x)^2(1- \theta(t-x))
\notag
\\
&=t+(x-t)^2\theta(x-t).
\label{50000009:laplaceloesung}
\end{align}
Das Auftreten der Heaviside-Funktion könnte dazu verleiten anzunehmen,
die Lösungsfunktion sei nicht stetig. Da die Heaviside-Funktion
aber mit einer stetigen Funktion multipliziert wird, die genau dort
verschwindet, wo die Heaviside-Funktion ihre Sprungstelle hat, 
ist das Resultat stetig. Man kann es auch so schreiben:
\[
u(x,t)=\begin{cases}
t&\qquad x<t\\
t+(x-t)^2&\qquad\text{sonst}.
\end{cases}
\]
für $x=t$ geben beide Teilfunktionen den Wert $t$, die Funktion $u(x,t)$
ist also stetig.
\end{proof}

\begin{proof}[Lösung mit Charakteristiken]
\begin{figure}
\centering
\includeagraphics[]{domain-1.pdf}
\caption{Projektion der Charakteristiken auf die $x$-$t$-Ebene
\label{50000009:charakteristiken}}
\end{figure}
\begin{figure}
\centering
\includeagraphics[width=0.6\hsize]{loes.jpg}
\caption{Lösungsfläche der Differentialgleichung~\ref{50000009:dgl}
mit Charakteristiken
\label{50000009:loesung}}
\end{figure}
Die Differentialgleichung ist quasilinear und von erster Ordnung, man kann
sie also auch mit Charakteristiken lösen.
Die Differentialgleichung der Charakteristiken mit Parameter $\tau$ ist
\[
\begin{aligned}
\dot x&=1& &\Rightarrow& x(\tau)&=\tau + x_0\\
\dot t&=1& &\Rightarrow& t(\tau)&=\tau + t_0\\
\dot u&=1& &\Rightarrow& u(\tau)&=\tau + u_0
\end{aligned}
\]
Den Verlauf der Projektionen der Charakteristiken auf die $x$-$t$-Ebene zeigt
Abbildung~\ref{50000009:charakteristiken}.
Wir müssen die Charakteristiken danach unterscheiden, ob sie den unteren
oder den linken Rand des Gebietes treffen.

Für Charakteristiken, die den linken Rand des Gebietes treffen, ist $x_0=0$
und $u_0=u(0,t_0)=t_0$, also
\[
\begin{aligned}
x&=\tau\\
t&=x+t_0&&\Rightarrow& t_0&=t-x\\
u&=x+(t-x)=t.
\end{aligned}
\]
Aus der Bedingung $t_0>0$ lesen wir ab, dass diese Lösung für Punkte
mit $t>x$ gilt.

Für die Charakteristiken, die den unteren Rand des Gebietes treffen,
ist $t_0=0$ und $u_0=u(x_0,0)=x_0^2$, also
\[
\begin{aligned}
t&=\tau\\
x&=t+x_0&&\Rightarrow&x_0=x-t\\
u&=t+x_0^2=t+(x-t)^2.
\end{aligned}
\]
Aus der Bedingung $x_0>0$ lesen wir diesmal ab, dass diese Lösung für Punkte
mit $t < x$ gilt.

Will man beide Lösungsteile in einer einzigen Funktion schreiben, kann
man dazu die Heaviside-Funktion verwenden. Die zweite Lösung hat ja
einen zusätzlichen Term $(x-t)^2$, der genau dann auftritt, wenn $t<x$
gilt, oder wenn $\theta(x-t)=1$ ist.
Die Lösung ist daher
\begin{equation}
u(x,t)=t+(x-t)^2\theta(x-t),
\label{50000009:charakteristikenloesung}
\end{equation}
was mit der Lösung~\eqref{50000009:laplaceloesung} übereinstimmt.
\end{proof}

