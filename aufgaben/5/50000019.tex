The Mellin transform is an integral transform applicable to functions
defined on $\mathbb R^+$ defined by
\[
(Mf)(\mu)
=
\int_0^\infty f(x) x^\mu\,dx.
\]
\begin{teilaufgaben}
\item
Compute the Mellin transform of a derivative $f'(x)$ and relate it to the
Mellin transform of $f$.
\item
Apply the Mellin transform to the differential equation
\[
f'(x)+f(x)=0.
\]
\item
We are looking for a function $f(x)$ that decays quickly enough so that
all the integrals exist for $\mu>0$.
Use this information to simplify the equation obtained in b).
\item
What type of equation do you get?
\end{teilaufgaben}

\begin{loesung}
\begin{teilaufgaben}
\item
We use partial integration to compute the Mellin transform of $f'(x)$
\begin{align*}
(Mf')(\mu)
&=
\int_0^\infty f'(x)x^\mu\,dx
=
\biggl[ f(x)x^\mu \biggr]_0^\infty
-
\mu
\int_0^\infty f(x)x^{\mu-1}\,dx
\\
&=
\biggl[ f(x)x^\mu \biggr]_0^\infty
-
\mu \,Mf(\mu-1)
\end{align*}
\item
Applying the Mellin transform to the differential equation gives
\begin{align*}
(Mf')(\mu) + (Mf)(\mu)
&=
\bigl[f(x)x^\mu\bigr]_0^\infty - \mu (Mf)(\mu-1) + (Mf)(\mu) = 0
\end{align*}
\item
The condition means that the first term on the right hand side must
vanish, so the equation is now
\[
- \mu (Mf)(\mu-1) + (Mf)(\mu) = 0
\qquad\Rightarrow\qquad
(Mf)(\mu) = \mu (Mf)(\mu-1).
\]
\item
The equation is a difference equation and does no longer contain
any derivatives.
\qedhere
\end{teilaufgaben}
\end{loesung}







