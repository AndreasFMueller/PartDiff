Auf dem Gebiet $\Omega=\{(x,t)\,|\, x>0,t > 0\}$
ist die Differentialgleichung
\begin{equation}
\frac{\partial u}{\partial x}+x\frac{\partial u}{\partial t}=t
\label{50000007:gleichung}
\end{equation}
gegeben. Die L"osung soll die Randbedingungen
\[
u(0,t)=e^t
\qquad
\text{und}
\qquad
u(x,0)=g(x)
\]
erf"ullen.
\begin{teilaufgaben}
\item Verwenden Sie die Laplace-Transformation nach $t$,
um (\ref{50000007:gleichung}) in eine Familie gew"ohnlicher
Differentialgleichungen zu verwandeln.
\item
Ist die Funktion $u(x,t)$ durch die Randbedingungen eindeutig bestimmt?
\item
L"osen Sie die in a) gefundene gew"ohnliche Differentialgleichung f"ur den Fall
$g(x)=0$, soweit dies in geschlossener Form\footnote{In geschlossener Form
l"osen heisst, dass man eine Formel f"ur die L"osung angeben kann, die 
nur die wohlbekannten Funktionen verwendet, und in der keine nicht
ausgerechneten Integrale mehr stehen.
} m"oglich ist.
\end{teilaufgaben}

\begin{hinweis}
Anfangsbedingungen nicht vergessen!
\end{hinweis}

\begin{loesung}
\begin{teilaufgaben}
\item
Die Laplace-Transformation von (\ref{50000007:gleichung}) ergibt
\begin{align}
\frac{\partial}{\partial x}{\cal L}u(x,s)
+x\bigl(-u(x,0)+s{\cal L}u(x,s)\bigr)&=\frac1{s^2}
\notag
\\
\frac{\partial}{\partial x}{\cal L}u(x,s)
+xs{\cal L}u(x,s)&=\frac1{s^2}+xg(x)
\label{50000007:transformiert}
\end{align}
Darin haben wir die Laplace-Transformierte von $t$ verwendet:
\begin{align*}
\int_0^\infty e^{-st}t\,dt
&=
\left[
-\frac1{s} e^{-st}t
\right]_0^\infty
+\frac1s\int_0^\infty e^{-st}\,dt
=\frac1s\left[
-\frac1se^{-st}
\right]_0^\infty=\frac1{s^2}.
\end{align*}
Schreibt man $y_s(x)={\cal L}u(x,s)$, wird deutlich,
dass (\ref{50000007:transformiert}) eine mit $s$ parametrisierte
Familie von gew"ohnlichen, linearen, inhomogenen Differentialgleichungen
\begin{equation}
y_s'(x)+xsy_s(x)=\frac1{s^2}-xg(x)
\label{50000007:family}
\end{equation}
erster Ordnung ist.

Die Anfangsbedingung f"ur die Funktion $y_s(x)$ an der Stelle $x=0$
muss aus $u(0,t)=e^t$ ermittelt werden.
Die Laplace-Transformierte von $e^t$ ist
\begin{align*}
\int_0^\infty e^{-st}e^t\,dt
&=
\int_0^\infty e^{t(1-s)}\,dt
=\left[\frac1{1-s}e^{t(1-s)}\right]_0^\infty=\frac1{s-1}.
\end{align*}
Es muss also gelten
\[
y_s(0)=\frac1{s-1}.
\]
\item
Aus der Theorie der linearen partiellen Differentialgleichungen weiss man,
dass $y_s(x)$ f"ur alle $x$ existiert und eindeutig bestimmt ist. 
Daher wird auch $u(x,t)$ eindeutig bestimmt sein.

Man kann auch mit Hilfe von Charakteristiken zu diesem Schluss kommen, 
(\ref{50000007:gleichung}) ist n"amlich auch eine quasilineare partielle
Differentialgleichung erster Ordnung.
Das Differentialgleichungssystem f"ur die Charakteristiken mit Parameter $p$
ist
\begin{align*}
\dot x&=1&&\Rightarrow&x(p)&=p+x_0\\
\dot t&=x&&\Rightarrow&t(p)&=\frac12p^2+x_0p+t_0\\
\dot u&=t&&\Rightarrow&u(p)&=\frac16p^3+\frac12x_0p^2+t_0p+u_0
\end{align*}
Mit $p=x-x_0$ aus der ersten Gleichung findet man
\[
t(x)=\frac12(x-x_0)^2+x_0(x-x_0)+t_0=\frac12(x^2-x_0^2)+t_0.
\]
Man muss sich jetzt "uberlegen, ob mit solchen Charakteristiken
jeder beliebige Punkt des Gebietes "uberdeckt werden kann.
Ausgehend von Punkten auf der $x$-Achse, d.~h.~mit $t_0=0$
erreicht man alle Punkte in $\{(x,t)\,|\, t\le \frac12x^2\}$.
Ausgehend von Punkten auf der $t$-Achse, d.~h.~mit $x_0=0$
erreicht man alle Punkte in $\{(x,t)\,|\, t\ge \frac12x^2\}$.
Dies kann man zum Beispiel auch aus der Darstellung
der Charakteristiken in Abbildung~\ref{50000007:bild}
ablesen.
\begin{figure}
\begin{center}
\includeagraphics[width=0.6\hsize]{domain-1.pdf}
\end{center}
\caption{Charakteristiken der partiellen
Differentialgleichung~(\ref{50000007:gleichung}).
Die fett ausgezogene Charakteristik trennt die Einflussgebiet der
Randbedingungen auf der $x$- und $t$-Achse.
\label{50000007:bild}}
\end{figure}
\begin{figure}
\centering
\includeagraphics[width=0.5\hsize]{loes.jpg}
\caption{L"osunsgfl"ache der Differentialgleichung~(\ref{50000007:gleichung}).
\label{50000007:loesungsflaeche}}
\end{figure}
\item
Die homogene Differentialgleichung zu (\ref{50000007:family}) l"asst sich
mit Separation l"osen:
\begin{align*}
y_s'  + xsy_s&=0
\\
\int\frac{dy_s}{y_s}&=-\int xs\,dx
\\
y_s&=C_se^{-\frac12sx^2}
\end{align*}
Die L"osung der inhomogenen Gleichung kann jetzt durch Verwendung der
Methode der Variation der Konstanten gefunden werden, was auf die
Gleichung 
\[
C'_s(x)e^{-\frac12sx^2}=\frac1{s^2}
\]
f"uhrt.
Das Integral 
\[
\int_0^x\frac1{s^2}e^{\frac12sx^2}\,dx
=\frac1{s^2}
\int_0^xe^{\frac12sx^2}\,dx
\]
kann jedoch nicht in geschlossener Form ausgewertet werden, so dass der
Versuch der analytischen L"osung an dieser Stelle abgebrochen werden muss.
\qedhere
\end{teilaufgaben}
\end{loesung}

\begin{bewertung}
Laplace-Transformation von $t$ ({\bf T}) 1 Punkt,
Transformation der Gleichung ({\bf G}) 1 Punkt,
Laplace-Transformation der Anfangsbedingung ({\bf A}) 1 Punkt,
Eindeutigkeit in Teilaufgabe b) ({\bf E}) 1 Punkt,
Separation der homogenen Differentialgleichung ({\bf S}) 1 Punkt,
L"osung der homogenen Gleichung ({\bf L}) 1 Punkt.
\end{bewertung}
