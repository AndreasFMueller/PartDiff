Auf dem Gebiet $\Omega=\{(x,t)\,|\, x>1,t > 0\}$
ist die Differentialgleichung
\begin{equation}
x\frac{\partial u}{\partial x}+\frac{\partial u}{\partial t}=0
\label{50000008:gleichung}
\end{equation}
gegeben. Die Lösung soll die Randbedingungen
\[
u(1,t)=1-\cos at
\qquad
\text{und}
\qquad
u(x,0)=0
\]
erfüllen ($a > 0$).
\begin{teilaufgaben}
\item Verwenden Sie die Laplace-Transformation nach $t$,
um \eqref{50000008:gleichung} in eine Familie gewöhnlicher
Differentialgleichungen zu verwandeln.
\item 
Ist die Lösung der Differentialgleichung~\eqref{50000008:gleichung} durch
die Randbedingungen eindeutig festgelegt?
\item
Lösen Sie die in a) gefundene gewöhnliche Differentialgleichung,
soweit dies in geschlossener Form\footnote{In geschlossener Form
lösen heisst, dass man eine Formel für die Lösung angeben kann, die 
nur die wohlbekannten Funktionen verwendet, und in der keine nicht
ausgerechneten Integrale mehr stehen.
} möglich ist.
\end{teilaufgaben}

\begin{hinweis}
Die Laplacetransformierte der Funktion $t\mapsto\cos at$ ist
\[
s\mapsto
\frac{s}{s^2+a^2}.
\]
Anfangsbedingungen nicht vergessen!
\end{hinweis}

\begin{loesung}
\begin{teilaufgaben}
\item
Die Laplacetransformation nach der Variablen $t$
ergibt für die Differentialgleichung
\[
x\frac{\partial{\cal L}u(x,s)}{\partial x}
-
u(x,0)+{\cal L}u(x,s)=0.
\]
Dank der Randbedingung $u(x,0)=0$ wird die Differentialgleichung
homogen:
\[
x\frac{\partial{\cal L}u(x,s)}{\partial x}
+
s{\cal L}u(x,s)=0.
\]
Die Anfangsbedingung für ${\cal L}u(x,s)$ für $x=1$ ist die
Laplacetransformierte der Kosinusfunktion
\[
{\cal L}u(1,s)=\frac1s-\frac{s}{s^2+a^2} = \frac{a^2}{s(s^2+a^2)}.
\]
In der letzten Form erkennt man auch, dass die Laplacetransformierte
für alle $s$ positiv ist.
\item
Die Theorie der gewöhnlichen Differentialgleichungen garantiert, dass
die Funktionen ${\cal L}u(x,s)$ eindeutig festgelegt sind, damit folgt
auch, dass $u(x,t)$ eindeutig festgelegt ist.
\item
Schreiben wir $y_s(x)={\cal L}u(x,s)$ erhalten wir die Differentialgleichung
in etwas übersichtlicherer Form
\begin{equation}
xy_s'(x)+sy_s(x)=0.
\label{50000008:sdgl}
\end{equation}
Diese gewöhnliche Differentialgleichung kann mit Separation der
Variablen gelöst werden:
\begin{align*}
\frac{y_s'}{y_s}&=-\frac{s}{x}
\\
\frac{d}{dx}\log y_s&=-s\frac{d}{dx}\log x
\\
\log y_s&=-s\log x+\operatorname{const}
\\
y_s(x)&=Cx^{-s}.
\end{align*}
Die Anfangsbedingung für $x=1$ liefert
\begin{equation}
y_s(1)=C=\frac{s}{s^2+a^2}
\qquad\Rightarrow\qquad
y_s(x)
=
\biggl(\frac1s-\frac{s}{s^2+a^2}\biggr)x^{-s}
=
\frac{a^2}{s(s^2+a^2)}x^{-s}.
\label{50000008:loesung}
\end{equation}
Die rechte Seite ist immer positiv, ansonsten wäre die oben durchgeführte
Lösung der Differentialgleichung nicht zulässig.
\qedhere
\end{teilaufgaben}
\end{loesung}

\begin{diskussion}
Die Rücktransformation von \eqref{50000008:loesung} lässt sich
ebenfalls durchführen, verlangt aber vertiefte Kenntnisse der
Laplace-Transformation, und war daher nicht verlangt.

Den Faktor $x^{-s}$ auf der rechten Seite von \eqref{50000008:loesung}
kann man auch als $e^{-s\log x}$
schreiben, damit wird klar, dass $u(x,t)$ nur eine um $\log x$ verschobene
Version der Funktion $u(1,t)$ ist. 
Die zweite Verschiebeeigenschaft der Laplace-Transformation lautet nämlich
\[
{\cal L}\bigl(f(t-\log x)\theta(t-\log x)\bigr)
=
e^{-s\log x}{\cal L}(f(t)),
\]
wobei $\theta(t)$ die Heaviside-Funktion
\[
\theta(t)=\begin{cases}1\qquad&t \ge 0\\ 0&\text{sonst}\end{cases}
\]
ist. Daraus kann man ablesen, dass
\begin{equation}
u(x,t)=\begin{cases}
1-\cos a(t-\log x)\qquad
&t \ge \log x\\
0
&\text{sonst}
\end{cases}
\label{50000008:ruecktransformiert}
\end{equation}
die Lösung der Differentialgleichung \eqref{50000008:gleichung} ist.
\begin{figure}
\centering
\includeagraphics[]{domain-1.pdf}
\caption{Charakteristiken der Differentialgleichung~\eqref{50000008:gleichung}
\label{50000008:charakteristiken}}
\end{figure}
\begin{figure}
\centering
\includeagraphics[width=\hsize]{loes.jpg}
\caption{3D-Darstellung der Lösungsfläche der
Differentialgleichung~\eqref{50000008:gleichung}, $x$-Achse nach vorne.
\label{50000008:loesung}}
\end{figure}


Diese quasilineare partielle Differentialgleichung kann natürlich auch
mit Hilfe von Charakteristiken gelöst werden.
Die Differentialgleichungen der Charakteristiken sind
\begin{equation}
\begin{aligned}
\dot x&=x&
&\Rightarrow&
x(\tau)&=x_0e^\tau
\\
\dot t&=1&
&\Rightarrow&
t(\tau)&=\tau + t_0
\\
\dot u&=0&
&\Rightarrow&
u(\tau)&=u_0
\end{aligned}
\label{50000008:chardgl}
\end{equation}
Die Randbedingungen legen die Lösung eindeutig fest, wie man aus
Abbildung~\ref{50000008:charakteristiken} ablesen kann.

Die Charakteristik durch den Punkt $(1,0)$, in Abildung~\ref{50000008:charakteristiken} etwas breiter eingezeichnet, spielt eine besondere Rolle.
Funktionswerte in Punkten oberhalb dieser Charakteristik sind durch die 
Randbedingung $u(1,t)=1-\cos at$ festgelegt, Funktionswerte in Punkten
unterhalb dieser Charakteristik sind $0$.
Für diese Charakteristik ist $x_0=1$ und $t_0=0$, die Charakteristik
hat daher die Gleichung $t=\log x$, wie man aus den ersten zwei Gleichungen
von \eqref{50000008:chardgl} ablesen kann.

Aus der zweiten Gleichung von \eqref{50000008:chardgl}
liest man $\tau=t-t_0$ ab, so dass 
$x=x_0e^{t-t_0}$ folgt.
Für die Randbedingung für $x=1$ folgt $x=e^{t-t_0}$
oder $t_0=t-\log x$ und damit die Lösung 
\[
u(\tau)=u_0=1-\cos at_0=1-\cos a(t-\log x).
\]
Da für Punkte unterhalb der Charakteristik $t=\log x$ der Funktionswert $0$
sein muss, erhält man wieder \eqref{50000008:ruecktransformiert} als Lösung.
\end{diskussion}

\begin{bewertung}
Transformation der Gleichung ({\bf G}) 1 Punkt,
Laplace-Transformation der Anfangsbedingung ({\bf A}) 1 Punkt,
Eindeutigkeit in Teilaufgabe b) ({\bf E}) 1 Punkt,
Separation der homogenen Differentialgleichung \eqref{50000008:sdgl}
({\bf S}) 1 Punkt,
Lösung der homogenen Gleichung ({\bf L}) 1 Punkt,
Einsetzen der Anfangsbedingungen \eqref{50000008:loesung} ({\bf R}) 1 Punkt.
\end{bewertung}
