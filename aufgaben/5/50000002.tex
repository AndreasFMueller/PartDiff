Finden Sie eine im ganzen Definitionsgebiet 
$G=\mathbb R\times [0,\infty[$
beschränkte Lösung
des Poisson-Problems
\begin{align*}
\Delta u&=0,&(x,y)&\in G
\\
u(x,0)&=f(x)&x&\in\mathbb R=\partial G.
%\partial_yu(x,0)&=g(x)&x&\in\mathbb R.
\end{align*}
Berechnen sie die Lösung explizit, wenn die Funktion $f$ eine
$\delta$-Funktion im Punkt $0$ ist. Verwenden Sie dazu die 
Fourier-Transformation entlang der $x$-Achse.

\begin{loesung}
Durch Fourier-Transformation in der Variablen $x$ wird die Gleichung
zu
\begin{align*}
-k^2\hat u(k,y)+\frac{\partial^2\hat u}{\partial y^2}(k,y)&=0\\
\frac{\partial^2\hat u}{\partial y^2}(k,y)&= k^2\hat u(k,y)
\end{align*}
Diese mit $k$ parametrisierte Familie gewöhnlicher Differentialgleichung
für die Funktion
$y\mapsto \hat u(k,y)$ hat als Lösung
\[
\hat u(k,y)=\begin{cases}
e^{-|k|y}\\
e^{|k|y}
\end{cases}
\]
Die zweite Lösung ist nicht beschränkt, kann als für die Lösung
des Problems nicht verwendet werden. Für $\hat u(k,y)$ kommt also
nur die erste Lösung in Frage, die noch mit einer von $k$ abhängigen
Konstanten multipliziert werden muss:
\begin{equation}
\hat u(k,y)=g(k)e^{-|k|y}.
\label{50000002:anf}
\end{equation}

Um die Anfangsbedingung $u(x,0)=f(x)$ zu erfüllen, berechnen wir
die Fourier-Transformierte und erhalten $\hat u(k,0)=\hat f(k)$.
Setze man $y=0$ in \eqref{50000002:anf} ein, erhält man
\[
\hat u(k,0)=g(k)=\hat f(k),
\]
also ist 
\[
\hat u(k,y)=\hat f(k)e^{-|k|y}.
\]
Um die Lösung zu finden, ist davon wieder die Rücktransformierte
zu berechnen. Auf der rechten Seite steht das Produkt von zwei
Fouriertransformierten, nämlich von $f$ und von der Funktion
\[
x\mapsto \frac{2y}{x^2+y^2}
\]
also ist die Rücktransformierte die Faltung dieser beiden
Funktionen
\[
u(x,y)=\int_{-\infty}^\infty 
f(\xi) 
\frac{2y}{(x-\xi)^2+y^2}
\,d\xi
=
2y\int_{-\infty}^\infty\frac{f(\xi)}{(x-\xi)^2+y^2}\,d\xi.
\]

Für eine $\delta$-Funktion an der Stelle $0$ ist das Integral der
Wert des Integranden an der Stelle $\xi=0$, also
\[
u(x,y)=2y\frac{1}{x^2+y^2}=\frac{2y}{x^2+y^2}
\]

Man könnte auch versuchen, die Lösung mit der Laplace-Transformation
zu berechnen. Die Transformation der Differentialgleichung ergibt
\begin{align*}
({\cal L}\partial_x^2u)(x,s)
+
s^2({\cal L}u)(x, s)-u(x,0)-s\partial_xu(x,0)
&=0\\
\partial_x^2({\cal L}u)(x,s)+s^2({\cal L}u)(x, s)-u(x,0)-s\partial_xu(x,0)=0
\end{align*}
Die Anfangsbedingung liefert $u(x,0)=f(x)$, also
\[
\partial_x^2({\cal L}u)(x,s)+s^2({\cal L}u)(x, s)-f(x)-s\partial_xu(x,0)=0
\]
Dies ist eine Differentialgleichung zweiter Ordnung für jedes $s$, schreiben
wir $({\cal L}u)=y_s(x)$, und zur Abkürzung $g(x)=\partial_xu(x,0)$,
lautet sie
\begin{align*}
v_s''(x)+s^2v_s(x)-f(x)-sg(x)&=0,\\
v_s''(x)+s^2v_s(x)&=f(x)+sg(x)
\end{align*}
Damit wir diese Gleichung lösen können, brauchen wir eine
Anfangsbedingung, eine solche liegt uns jedoch nicht vor, weil
das Gebiet in $x$-Richtung in beide Richtungen unendlich ausgedehnt
ist.
Bestenfalls können wir also eine Funktion $h(s)$ als Anfangsbedingung
zu verwenden. Wir müssen also die Gleichung
\[
v''(x)-a^2v(x)=w(x)
\]
mit $a>0$
lösen können. Eine Lösung der homogenen Gleichung ist
\[
v(x)=A\cosh ax+B\sinh(-ax).
\]

Variation der Konstanten ergibt die Gleichung
\begin{align*}
A''(x)e^{ax}+aA(x)e^{ax}+a^2A(x)e^{ax}
+
B''(x)e^{-bx}-aB(x)e^{-ax}+b^2B(x)e^{-ax}
-a^2A(x)e^{ax}-b^2B(x)e^{-ax}&=w(x)
\\
(A''(x)+aA'(x))e^{ax}
+
(B''(x)-aB'(x))e^{-ax}
&=w(x)
\end{align*}
\end{loesung}
