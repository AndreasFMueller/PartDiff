The partial differential equation
\begin{equation}
\frac{\partial^2 u}{\partial r^2}
+
\frac{\partial u}{\partial r}
+
\frac{\partial^2 u}{\partial \varphi^2}
=
2
\label{50000022:eqn}
\end{equation}
on the domain $\Omega = \{ (r,\varphi)\,|\,0<\varphi<\pi/2\wedge r>0\}$
in polar coordinates with boundary conditions
\begin{align}
u(r,0)                       &= e^{-r}
\label{50000022:bottom}
\\
u(r,{\textstyle\frac{\pi}2}) &= 1 
\label{50000022:boundary}
\end{align}
is to be solved using the transformation method.
Note that for $r=0$, the polar coordinate system ist somewhat singular,
as all pairs $(0,\varphi)$ refer to the same point.
To deal with this problem, we add additional boundary conditions
compatible with~\eqref{50000022:boundary}:
\begin{equation}
\begin{aligned}
u(0,\varphi) &= 1
\\
\frac{\partial u}{\partial r}(0,\varphi) &= 0
\end{aligned}
\label{50000022:additional}
\end{equation}
for $0<\varphi<\frac{\pi}2$.

\begin{teilaufgaben}
\item
Use the Laplace transform with respect to $r$ to obtain ordinary differential
equations and boundary conditions.
\item
Is the solution to the partial differential equation unique?
\end{teilaufgaben}


\begin{loesung}
\begin{teilaufgaben}
\item
First we compute the Laplace transform with respect to $r$ of the
differential equation~\eqref{50000022:eqn}, we call the transformed variable
$s$:
\begin{align*}
s^2 (\mathscr{L}u)(s,\varphi)
   -s \underbrace{u(0,\varphi)}_{\textstyle=1}
   -\underbrace{\frac{\partial u}{\partial r}(0,\varphi)}_{\textstyle=0}
+s(\mathscr{L}u)(s,\varphi) - \underbrace{u(0,\varphi)}_{\textstyle=1}
+
\frac{\partial^2}{\partial\varphi^2}(\mathscr{L}u)(s,\varphi)
&=
\frac2s
\\
\Rightarrow\hspace{6cm}
\frac{\partial^2}{\partial\varphi^2}(\mathscr{L}u)(s\varphi)
+
(s^2+s) (\mathscr{L}u)(s,\varphi)
&=
\frac2s
+s+1
\end{align*}
Writing $y_s(\varphi)=\mathscr{L}u(s,\varphi)$ and considering
$s$ just a parameter, these can be seen as a family
\begin{equation}
y''_s(\varphi) +s(s+1) y_s(\varphi) = \frac{2}s+s+1
\label{50000022:transformedeqn}
\end{equation}
of linear ordinary
differential equations of second order.

We also need boundary conditions for each of these ordinary differential
equations, which we can get from \eqref{50000022:bottom} and
\eqref{50000022:boundary} by using the Laplace transform once more:
\begin{align*}
\mathscr{L}u(s,0) &= \frac{1}{s+1}
\\
\mathscr{L}u(s,{\textstyle\frac{\pi}2}) &= \frac1s
\end{align*}
So $y_s(\varphi)$ are solutions of~\eqref{50000022:transformedeqn}
satisfying the boundary conditions
\[
y_s(0)=\frac{1}{s+1}
\qquad\text{and}\qquad
y_s({\textstyle\frac{\pi}2})=\frac{1}s.
\]
\item
The homegeneous equation of the transformed equation is
\[
y_s''(\varphi) + (s^2+s) y_s(\varphi)=0,
\]
which has the solutions
\[
\cos(\!\sqrt{s^2+s}\,\varphi)
\qquad\text{and}\qquad
\sin(\!\sqrt{s^2+s}\,\varphi).
\]
The constant
\[
y_s(x) = \frac{2/s+s+1}{s(s+1)}
\]
is a particular solution for the nonhomogeneous equation.
Using these solutions, the solution $y_s(\varphi)$ can be uniquely determined
from the boundary conditions.
Since there can only be one Laplace transform, the solution $u(r,\varphi)$
also must be unique.
\qedhere
\end{teilaufgaben}
\end{loesung}

\begin{bewertung}
\begin{teilaufgaben}
\item
Laplace transform of differential equation ({\bf L}) 2 points,
simplify using additional boundary conditions \eqref{50000022:additional}
({\bf S}) 1 point,
Laplace transforms of boundary conditions ({\bf B}) 2 points,
\item
uniqueness ({\bf U}) 1 point.
\end{teilaufgaben}
\end{bewertung}

