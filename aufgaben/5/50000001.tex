Lösen Sie die Wärmeleitungsgleichung
\[
\partial_t u=\kappa\partial_x^2u
\]
für $t>0$ und $x>0$ mit den Rand- und Anfangsbedingungen
\begin{align*}
u(t,0)&=c,\\
u(0,x)&=0.
\end{align*}

\begin{loesung}
Die Bedingung $t>0$ zeigt an, dass die Laplace-Transformation
der Variablen $t$ in die Variable $s$ zum Erfolg führen könnte.
Wir schreiben $\cal L$ für die Laplace-Transformation von $s$.
Die transformierte Differentialgleichung ist
\[
s{\cal L}u(s, x)-u(0,x)=\kappa \partial_x^2{\cal L}u(s,x).
\]
Setzt man die Anfangsbedingung $u(0,x)=0$ ein, wird daraus
\[
s{\cal L}u(s, x)=\kappa \partial_x^2{\cal L}u(s,x),
\]
also eine mit $s$ parametrisierte Familie von gewöhnlichen
Differentialgleichung der Funktion $x\mapsto {\cal L}u(s,x)$.
Etwas umgestellt ist sie äquivalent zu
\[
\partial_x^2{\cal L}u(s,x)
=
\frac{s}{\kappa}{\cal L}u(s, x),
\]
mit den Lösungen
\[
{\cal L}u(s,x)=\begin{cases}
C_1(s)e^{\sqrt{\frac{s}{\kappa}}x}\\
C_2(s)e^{-\sqrt{\frac{s}{\kappa}}x}
\end{cases}
\]
Da die physikalisch sinnvollen Lösungen der Wärmeleitungsgleichung
beschränkt sein müssen, kommt nur die zweite Lösung in Frage.

Die Randbedingung muss ebenfalls transformiert werden:
\[
{\cal L}u(s,0)=\frac{c}{s},
\]
dies ist die Anfangsbedingung für die Differentialgleichung,
\[
{\cal L}u(s,0)=C_2(s)=\frac{c}{s}.
\]
Somit ist die Laplace-Transformierte der gesuchten Lösung
\[
{\cal L}u(s,x)=\frac{c}{s}e^{-\sqrt{\frac{s}{\kappa}}x}
\]
Diese Funktion muss jetzt zurück transformiert werden.
Dazu schreiben wir die rechte Seite zunächst noch um:
\[
{\cal L}u(s,x)=c\cdot \frac{1}{s}e^{-k\sqrt{s}}
\]
mit $k=x/\sqrt{\kappa}$.
Aus der
Tabelle der Laplace-Transformationen entnimmt man, dass die
Rücktransformierte
\[
u(t,x)
=
c \operatorname{erfc}\biggl(
\frac{k}{2\sqrt{t}}
\biggr)
=
c \operatorname{erfc}\biggl(
\frac{x}{2\sqrt{\kappa t}}
\biggr)
\]
ist.
\end{loesung}
