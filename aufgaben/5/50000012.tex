Verwenden Sie Laplace-Transformation nach $t$ um
im Gebiet $\Omega=\{(x,t)\,|\, x > 1 \text{ und } t > 0\}$ eine
L"osung der Differentialgleichung
\begin{equation}
\frac{\partial u}{\partial t} +3\frac{\partial u}{\partial x} = -\frac{u}{x}
\label{50000012:dgl}
\end{equation}
mit den Randbedingungen
\begin{equation}
\begin{aligned}
u(1,t)&=t& &\text{f"ur $t>0$},\\
u(x,0)&=0& &\text{f"ur $x > 1$}
\end{aligned}
\label{50000012:rb}
\end{equation}
zu finden.

\begin{hinweis}
Sie d"urfen die Laplace-Transformierte der L"osung stehen lassen, wenn Sie
die R"ucktransformation nicht finden.
\end{hinweis}

\begin{loesung}
Die Laplace-Transformierte der Differentialgleichung ist
\[
s{\cal L}u(x,s) - \underbrace{u(x,0)}_{\displaystyle =0}
+ 3\frac{\partial }{\partial x}{\cal L}u(x,s)
=
-\frac1x{\cal L}u(x,s).
\]
Der besseren "Ubersicht halber k"urzen wir ${\cal L}u(x,s)=y_s(x)$ ab
und erhalten
\begin{align*}
3y_s'(x) + sy_s(x) &= -\frac1x y_s(x)
\\
y_s'(x)&=-\frac13\biggl(s+\frac1x\biggr) y_s(x)
\\
\frac{d}{dx} \log y_s(x) &= -\frac13\biggl(s+\frac1x\biggr).
\end{align*}
Durch Integration der letzten Zeile finden wir
\begin{align*}
\log y_s(x) &= -\frac{sx}{3}-\frac13\log x + k(s)
\\
y_s(x) &= C(s) e^{-sx/3}\frac1{\sqrt[3]{x}}\qquad\text{mit}\quad C(s)=e^{k(s)}.
\end{align*}
Darin ist die Konstante $C(s)$ aus den Randbedingungen f"ur $x=1$
noch zu ermitteln.
Die Laplace-Transformiere von $u(1,t)$ ist
\begin{align*}
{\cal L}u(1,s)
&=
\int_0^\infty u(1,t)e^{-st}\,dt
=
\int_0^\infty \underset{\downarrow}{t}\cdot\underset{\uparrow}{e^{-st}}\,dt
=
\biggl[-\frac{t}se^{-st}\biggr]_0^\infty
+\int_0^\infty \frac1s e^{-st}\,dt
=
\biggl[ -\frac1{s^2}e^{-st}\biggr]_0^\infty=\frac1{s^2}.
\end{align*}
Setzt man dies in die Randbedingung f"ur $y_s(1)$ ein, erh"alt man
\[
{\cal L}u(1,s) = \frac1{s^2} = y_s(1)=C(s) e^{-s/3}
\qquad
\Rightarrow
\qquad
C(s) = \frac{e^{s/3}}{s^2}.
\]
Damit kann man jetzt auch die Laplace-Transformierte der L"osung der gegeben
partiellen Differentialgleichung hinschreiben:
\[
{\cal L}u(x,s)
=
y_s(x)
=
\frac{e^{s/3}}{s^2}e^{-sx/3}\frac1{\sqrt[3]{x}}
=
\frac1{s^2}e^{\frac{s}3(1-x)}\frac1{\sqrt[3]{x}}.
\qedhere
\]
%\begin{figure}
%\centering
%\includeagraphics[width=0.7\hsize]{loes.jpg}
%\caption{L"osungsfl"ache der Differentialgleichung~(\ref{50000012:dgl})
%in Aufgabe~\ref{50000011}
%\label{50000012:graph}}
%\end{figure}
\end{loesung}

%\begin{diskussion}
%\end{diskussion}

\begin{bewertung}
Laplacetransformierte Differentialgleichung ({\bf D}) 1 Punkt,
Einsetzen der zweiten Randbedingung ({\bf Z}) 1 Punkt,
Laplacetransformierte der ersten Randbedingung als Anfangsbedingung
der gew"ohnlichen Differentialgleichung ({\bf R}) 1 Punkt,
L"osung der homogenen Differentialgleichung ({\bf H}) 1 Punkt,
Bestimmung der Konstanten $C$ ({\bf C}) 1 Punkt,
(Laplace-Transformierte der) L"osung ({\bf L}) 1 Punkt.
\end{bewertung}
