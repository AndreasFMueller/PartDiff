Betrachten Sie die Differentialgleichung
\begin{equation}
\frac{\partial u}{\partial t}
+
e^{x^2}
\frac{\partial^2 u}{\partial x^2}
=
0
\end{equation}
auf dem Gebiet
\[
\Omega = \{ (t,x)\,|\, t>0, x\in\mathbb R\}
\]
mit den Randbedingungen
\begin{align*}
u(0,x) &= 0,\\
u(t,0) &= e^{-t},\\
\frac{\partial}{\partial x}u(t,0) &=0.
\end{align*}
für $x\in\mathbb R$ und $t > 0$.
\begin{teilaufgaben}
\item
Verwenden Sie die Laplace-Transformation und finden Sie eine gewöhnliche
Differentialgleichung samt Anfangsbedingungen, mit deren Hilfe das
Problem gelöst werden kann.
\item
Ist das Problem gut gestellt?
\end{teilaufgaben}

\begin{hinweis}
Es wird nicht verlangt, dass Sie die gewöhnliche Differentialgleichung lösen.
\end{hinweis}

\begin{loesung}
\begin{teilaufgaben}
\item
Die Laplace-Transformation nach $t$ (nur diese Variable hat den für die
Laplace-Transformation nötigen Definitionsbereich $\mathbb R_+$)
der Differentialgleichung ergibt
\[
s\mathcal{L}u(s,x) - u(0,x)
+ e^{x^2}\frac{\partial^2}{\partial x^2}\mathcal{L}u(s,x)
= 0.
\]
Der zweite Term fällt wegen der Anfangsbedingungen weg.
Mit der Notation $y_s(x) = \mathcal{L}u(s,x)$ wird die Gleichung zu
\begin{equation}
y_s''(x)+se^{-x^2}y_s(x)=0.
\label{50000016:eq}
\end{equation}

Die Laplacetransformation der übrigen Bedingungen liefert
\[
\begin{aligned}
u(t,0)&=e^{-t}
&&\Rightarrow&
\mathcal{L} u(s,0)&=\frac1{s+1},
\\
\frac{\partial}{\partial x}u(t,0)&=0
&&\Rightarrow&
\frac{\partial}{\partial x}\mathcal{L}u(s,0)&= 0.
\end{aligned}
\]
Wir suchen also eine Lösungsfunktion $y_s(x)$ für die Differentialgleichung
\eqref{50000016:eq}, die die Anfangsbedingungen
\begin{equation}
y_s(0)=\frac1{s+1}
\qquad\text{und}\qquad
y'_s(0)=0
\label{50000016:anf}
\end{equation}
erfüllt.
\item
Die Differentialgleichung \eqref{50000016:eq} ist für $s>0$ mit den gegebenen
Anfangsbedingungen \eqref{50000016:anf} eindeutig lösbar, also ist
das Problem gut gestellt.
\qedhere
\end{teilaufgaben}
\end{loesung}

\begin{bewertung}
\begin{teilaufgaben}
\item
Laplace-Transformation der DGL ({\bf L}) 1 Punkt,
Vereinfachung $u(0,x)=0$ ({\bf V}) 1 Punkt,
Laplace-Transformation der Anfangsbedingung $u(t,0)=1$ ({\bf A}) 1 Punkt,
Laplace-Transformation für die Anfangsbedingungen für die Ableitungen
({\bf D}) je 1 Punkt,
\item
Problem gut gestellt ({\bf G}) 1 Punkt,
Begründung ({\bf B}) 1 Punkt.
\end{teilaufgaben}
\end{bewertung}
