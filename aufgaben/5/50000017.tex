For the second order partial differential equation
\begin{equation}
-x\frac{\partial u}{\partial t}
+
\frac{\partial^2 u}{\partial x^2}
=
0
\label{50000017:equation}
\end{equation}
on the domain
\[
\Omega = \{ (t,x)\in \mathbb R^2 \,|\, \text{$t>0$ and $x>0$}\}
\]
boundary conditions
\[
\begin{aligned}
u(0,x) &= 0
&&&&
&&\text{for $x>0$}
\\
u(t,0) &= e^{-t}
&&\text{and}&
\frac{\partial u}{\partial x}(t,0) &= 1
&&\text{for $t>0$}
\end{aligned}
\]
are given.
\begin{teilaufgaben}
\item
Convert the partial differential equation \eqref{50000017:equation}
into a family of ordinary differential equations using the
Laplace transform with respect to the variable $t$.
\item
Assume that there are solutions $A(x)$ and $B(x)$ of the
ordinary differential equation $y''-xy=0$ with
initial conditions $A(0)=1$, $A'(0)=0$,
$B(0)=0$, $B'(0)=1$.
Show that the functions $A(\sqrt{s}x)$ and $B(\sqrt{s}x)$
are solutions of the ordinary differential equation obtained in a).
\item
Express the Laplace transform $\mathscr{L}u(s,x)$ by means of the functions
$A(x)$ and $B(x)$.
\end{teilaufgaben}


\begin{loesung}
\begin{teilaufgaben}
\item
The Laplace transform with respect to the variable $t$ turns the differential
equation into
\[
-
x\biggl(
-u(0,x) + s(\mathscr{L}u)(s,x)
\biggr)
+\frac{\partial^2}{\partial x^2} (\mathscr{L}u)(s,x)
=
0.
\]
Writing $y_s(x) = (\mathscr{L})(s,x)$ simplifies the equation to
the family of ordinary differential equations
\begin{equation}
y_s''(x)  -sxy_s(x) = 0.
\label{50000017:ode}
\end{equation}
The initial conditions $f(t)=e^{-t}$ and $g(t)=1$ also need to be transformed:
\begin{align*}
\mathscr{L}f(s)
&=
\int_0^\infty e^{-t} e^{-st}\,dt
=
\int_0^\infty e^{-t(1+s)} \,dt
=
\biggl[
\frac{-1}{1+s}e^{-t(1+s)}
\biggr]_0^\infty
=
\frac1{1+s}
\\
\mathscr{L}g(s)
&=
\int_0^\infty 1\cdot e^{-st}\,dt
=
\biggl[ -\frac1se^{-st}\biggr]_0^\infty
=
\frac1s.
\end{align*}
So we have to find solutions $y_s(x)$ of the differential equation
\eqref{50000017:ode} with initial conditions
\[
y_s(x) = \frac1{s+1}
\qquad
\text{and}
\qquad
y'_s(x) = \frac{1}{s}.
\]
\item
The functions $A(\sqrt{s}x)$ and $B(\sqrt{s}x)$ satisfy
\begin{align*}
\frac{d^2}{dx^2}A(\sqrt{s}x)
-
sxA(\sqrt{s}x)
&=
sA''(\sqrt{s}x)-sxA(\sqrt{s}x)
=
s(\underbrace{A''(\sqrt{s}x)-xA(\sqrt{s}x}_{\displaystyle=0})
=
0,
\\
\frac{d^2}{dx^2}B(\sqrt{s}x)
-
sxB(\sqrt{s}x)
&=
sB''(\sqrt{s}x)-sxB(\sqrt{s}x)
=
s(\underbrace{B''(\sqrt{s}x)-xB(\sqrt{s}x}_{\displaystyle=0})
=
0.
\end{align*}
The expressions above the curly braces are $0$ because they are just
the ordinary 
differential equation $y''-xy=0$ for the functions $A(x)$ and $B(x)$.
So the functions $A(\sqrt{s}x)$ and $B(\sqrt{s}x)$ are solutions of
the equation \eqref{50000017:ode}.
\item
To find $\mathscr{L}u(s,x)$, we have to linearly combine the
Functions $A(\sqrt{s}x)$ and $B(\sqrt{s}x)$ in such a way that the
initial conditions are satisfied for all $s$.
We use coefficients $a(s)$ and $b(s)$  dependent on $s$ for this purpose
and write 
\begin{align*}
y_s(x)
&=
a(s)
A(\sqrt{s}x) +
b(s)
B(\sqrt{s}x).
\end{align*}
We substitute $y_s(x)$ into the initial conditions:
\begin{align*}
y_s(0) &= a(s) A(0) + b(s) B(0) = a(s) = \frac{1}{s+1}
\\
y'_s(0)&= a(s)\sqrt{s}A'(0) + b(s) \sqrt{s}B'(0) = b(s)\sqrt{s} = \frac{1}{s}
\end{align*}
so
\[
\mathscr{L}u(s,x)
=
\frac1{s+1} A(\sqrt{s}x)
+
\frac{1}{s^{\frac32}} B(\sqrt{s}x)
\]
is the solution of the Laplace transformed equation of
\eqref{50000017:equation}.
\qedhere
\end{teilaufgaben}
\end{loesung}

\begin{diskussion}
The function $y''-xy=0$ is also known as Airy's equation.
The Airy functions $\operatorname{Ai}(x)$ and $\operatorname{Bi}(x)$
are explicit solutions of Airy's equation, but
they do not satisfy the initial conditions of the functions $A(x)$ and $B(x)$
above.
However, such functions can easily be constructed from the Airy functions.
\end{diskussion}

\begin{bewertung}
\begin{teilaufgaben}
\item
Transformed equation ({\bf E}) 1 point,
transformed boundary values for $f(t)$ ({\bf F}) 1 point,
transformed boundary values for $g(t)$ ({\bf G}) 1 point,
\item
verification that $A(\sqrt{s}x)$ and $B(\sqrt{s}x)$ satisfy Airy's
equation ({\bf A}) 1 point,
\item
use initial condition to determine the linear combination ({\bf I}) 1 point,
linear combination and solution for Laplace transformed solution ({\bf L})
1 point.
\end{teilaufgaben}
\end{bewertung}


