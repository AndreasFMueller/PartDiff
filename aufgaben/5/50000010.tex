Auf dem Gebiet $\Omega=\{(x,t)\,|\, x>1,t > 0\}$
ist die Differentialgleichung
\begin{equation}
x\frac{\partial u}{\partial x}+\frac{\partial u}{\partial t}=0
\label{50000010:gleichung}
\end{equation}
gegeben. Die Lösung soll die Randbedingungen
\[
u(1,t)=1-J_0(t)
\qquad
\text{und}
\qquad
u(x,0)=0
\]
erfüllen.
Dabei ist $J_0(t)$ eine sogenannte Bessel-Funktion, von der man nur
die Laplace-Transformation kennt:
\[
{\cal L}J_0(s)=\frac1{\sqrt{s^2+1}}
\]
Sie sieht ein bisschen aus wie $\sin(t)/t$, aber die genaue Kenntnis der
Funktion ist für die Lösung der Aufgabe nicht erforderlich.
\begin{teilaufgaben}
\item Verwenden Sie die Laplace-Transformation nach $t$,
um \eqref{50000010:gleichung} in eine Familie gewöhnlicher
Differentialgleichungen zu verwandeln.
\item 
Ist die Lösung der Differentialgleichung~\eqref{50000010:gleichung} durch
die Randbedingungen eindeutig festgelegt?
\item
Lösen Sie die in a) gefundene gewöhnliche Differentialgleichung,
soweit dies in geschlossener Form möglich ist.
In geschlossener Form
lösen heisst, dass man eine Formel für die Lösung angeben kann, die 
nur die wohlbekannten Funktionen verwendet, und in der keine nicht
ausgerechneten Integrale mehr stehen.
\end{teilaufgaben}

\begin{hinweis}
Anfangsbedingungen nicht vergessen!
\end{hinweis}

\begin{loesung}
Der Graph der Besselfunktion $J_0$ ist in Abbildung~\ref{50000010:besselj0}
dargestellt.
\begin{teilaufgaben}
\item
Die Laplacetransformation nach der Variablen $t$
ergibt für die Differentialgleichung
\[
x\frac{\partial{\cal L}u(x,s)}{\partial x}
-
u(x,0)+s{\cal L}u(x,s)=0.
\]
Dank der Randbedingung $u(x,0)=0$ wird die Differentialgleichung
homogen:
\[
x\frac{\partial{\cal L}u(x,s)}{\partial x}
+
s{\cal L}u(x,s)=0.
\]
Die Anfangsbedingung für ${\cal L}u(x,s)$ für $x=1$ ist die
Laplacetransformierte von $1-J_0(t)$, also
\[
{\cal L}u(1,s)=\frac1s-\frac{1}{\sqrt{s^2+1}}.
\]
Weil $\sqrt{s^2+1}>\sqrt{s^2}=s$ ist, kann man folgern, dass
die Laplacetransformierte ${\cal L}u(1,s)$ für alle $s$ negativ ist.
\item
Die Theorie der gewöhnlichen Differentialgleichungen garantiert, dass
die Funktionen ${\cal L}u(x,s)$ eindeutig festgelegt sind, damit folgt
auch, dass $u(x,t)$ eindeutig festgelegt ist.
\item
Schreiben wir $y_s(x)={\cal L}u(x,s)$ erhalten wir die Differentialgleichung
in etwas übersichtlicherer Form
\begin{equation}
xy_s'(x)+sy_s(x)=0.
\label{50000010:sdgl}
\end{equation}
Diese gewöhnliche Differentialgleichung kann mit Separation der
Variablen gelöst werden:
\begin{align*}
\frac{y_s'}{y_s}&=-\frac{s}{x}
\\
\frac{d}{dx}\log y_s&=-s\frac{d}{dx}\log x
\\
\log y_s&=-s\log x+\operatorname{const}
\\
y_s(x)&=Cx^{-s}.
\end{align*}
Die Anfangsbedingung für $x=1$ liefert
\begin{equation}
y_s(1)=C=\frac1s-\frac{1}{\sqrt{s^2 + 1}}
\qquad\Rightarrow\qquad
y_s(x)
=
\biggl(\frac1s-\frac{1}{\sqrt{s^2+1}}\biggr)x^{-s}.
\label{50000010:loesung}
\end{equation}
Die rechte Seite ist immer positiv, ansonsten wäre die oben durchgeführte
Lösung der Differentialgleichung nicht zulässig.
\end{teilaufgaben}
\end{loesung}

\begin{diskussion}
Die Rücktransformation von \eqref{50000010:loesung} lässt sich
ebenfalls durchführen, verlangt aber vertiefte Kenntnisse der
Laplace-Transformation, und war daher nicht verlangt.

Den Faktor $x^{-s}$ auf der rechten Seite von \eqref{50000010:loesung}
kann man auch als $e^{-s\log x}$
schreiben, damit wird klar, dass $u(x,t)$ nur eine um $\log x$ verschobene
Version der Funktion $u(1,t)$ ist. 
Die zweite Verschiebeeigenschaft der Laplace-Transformation lautet nämlich
\[
{\cal L}\bigl(f(t-\log x)\theta(t-\log x)\bigr)
=
e^{-s\log x}{\cal L}(f(t)),
\]
wobei $\theta(t)$ die Heaviside-Funktion
\[
\theta(t)=\begin{cases}1\qquad&t \ge 0\\ 0&\text{sonst}\end{cases}
\]
ist. Daraus kann man ablesen, dass
\begin{equation}
u(x,t)=\begin{cases}
1-J_0(t-\log x)\qquad
&t \ge \log x\\
0
&\text{sonst}
\end{cases}
\label{50000010:ruecktransformiert}
\end{equation}
die Lösung der Differentialgleichung \eqref{50000010:gleichung} ist.
\begin{figure}
\centering
\includeagraphics[]{domain-2.pdf}
\caption{Graph der Besselfunktion $J_0(t)$
\label{50000010:besselj0}}
\end{figure}
\begin{figure}
\centering
\includeagraphics[]{domain-1.pdf}
\caption{Charakteristiken der Differentialgleichung~\eqref{50000010:gleichung}
\label{50000010:charakteristiken}}
\end{figure}
\begin{figure}
\centering
\includeagraphics[width=\hsize]{loes.jpg}
\caption{3D-Darstellung der Lösungsfläche der
Differentialgleichung~\eqref{50000010:gleichung}, $x$-Achse nach rechts.
\label{50000010:loesung}}
\end{figure}


Diese quasilineare partielle Differentialgleichung kann natürlich auch
mit Hilfe von Charakteristiken gelöst werden.
Die Differentialgleichungen der Charakteristiken sind
\begin{equation}
\begin{aligned}
\dot x&=x&
&\Rightarrow&
x(\tau)&=x_0e^\tau
\\
\dot t&=1&
&\Rightarrow&
t(\tau)&=\tau + t_0
\\
\dot u&=0&
&\Rightarrow&
u(\tau)&=u_0
\end{aligned}
\label{50000010:chardgl}
\end{equation}
Die Randbedingungen legen die Lösung eindeutig fest, wie man aus
Abbildung~\ref{50000010:charakteristiken} ablesen kann.

Die Charakteristik durch den Punkt $(1,0)$, in Abildung~\ref{50000010:charakteristiken} etwas breiter eingezeichnet, spielt eine besondere Rolle.
Funktionswerte in Punkten oberhalb dieser Charakteristik sind durch die 
Randbedingung $u(1,t)=1-J_0(t)$ festgelegt, Funktionswerte in Punkten
unterhalb dieser Charakteristik sind $0$.
Für diese Charakteristik ist $x_0=1$ und $t_0=0$, die Charakteristik
hat daher die Gleichung $t=\log x$, wie man aus den ersten zwei Gleichungen
von \eqref{50000010:chardgl} ablesen kann.

Aus der zweiten Gleichung von \eqref{50000010:chardgl}
liest man $\tau=t-t_0$ ab, so dass 
$x=x_0e^{t-t_0}$ folgt.
Für die Randbedingung für $x=1$ folgt $x=e^{t-t_0}$
oder $t_0=t-\log x$ und damit die Lösung 
\[
u(\tau)=u_0=1-J_0(t_0)-1=1-J_0(t-\log x)-1.
\]
Da für Punkte unterhalb der Charakteristik $t=\log x$ der Funktionswert $0$
sein muss, erhält man wieder \eqref{50000010:ruecktransformiert} als Lösung.
\end{diskussion}

\begin{bewertung}
Transformation der Gleichung ({\bf G}) 1 Punkt,
Laplace-Transformation der Anfangsbedingung ({\bf A}) 1 Punkt,
Eindeutigkeit in Teilaufgabe b) ({\bf E}) 1 Punkt,
Separation der homogenen Differentialgleichung \eqref{50000010:sdgl}
({\bf S}) 1 Punkt,
Lösung der homogenen Gleichung ({\bf L}) 1 Punkt,
Einsetzen der Anfangsbedingungen \eqref{50000010:loesung} ({\bf R}) 1 Punkt.
\end{bewertung}



