Verwenden Sie Laplace-Transformation nach $t$ um
im Gebiet $\Omega=\{(x,t)\,|\, x > 1 \text{ und } t > 0\}$ eine
L"osung der Differentialgleichung
\begin{equation}
x^2\frac{\partial u}{\partial t}+\frac{\partial u}{\partial t}=0
\label{50000011:dgl}
\end{equation}
mit den Randbedingungen
\begin{equation}
\begin{aligned}
u(1,t)&=t& &\text{f"ur $t>0$},\\
u(x,0)&=0& &\text{f"ur $x > 1$}
\end{aligned}
\label{50000011:rb}
\end{equation}
zu finden.

\begin{hinweis}
Sie d"urfen die Laplace-Transformierte der L"osung stehen lassen, wenn Sie
die R"ucktransformation nicht finden.
\end{hinweis}

\begin{loesung}
Die Laplace-Transformierte der Differentialgleichung~(\ref{50000011:dgl}) ist
\begin{equation}
x^2\frac{\partial}{\partial x}{\cal L}u(x,s) + {\cal L}u(x,s) - u(x,0)=0.
\end{equation}
Der letzte Term f"allt wegen der zweiten Randbedingung weg.
Wir schreiben $y_s(x)={\cal L}u(x,s)$, und suchen jetzt also eine L"osung
der gew"ohnlichen Differentialgleichung
\begin{equation}
x^2y_s(x) + sy_s(x)=0.
\label{50000011:odgl}
\end{equation}
Die Anfangsbedingung von $y_s(x)$ f"ur $x=1$ entsteht durch
Laplace-Transformation der ersten Randbedingung in~(\ref{50000011:rb}) ist
\begin{align*}
{\cal L}u(1,s)
&=
\int_0^\infty u(1,t)e^{-st}\,dt
=
\int_0^\infty \underbrace{t}_{\downarrow}\underbrace{e^{-st}}_{\uparrow}\,dt
\\
&=
\biggl[
t\cdot \frac1{-s}e^{-st}
\biggr]_0^\infty
+\int_0^\infty \frac1se^{-st}\,dt
=
\biggl[
-\frac1{s^2}e^{-st}
\biggr]_0^\infty=\frac1{s^2}.
\end{align*}
Die allgemeine L"osung der gew"ohnlichen Differentialgleichung~(\ref{50000011:odgl}) ist
\begin{align*}
\frac{y_s'}{y_s}&=-\frac{s}{x^2}
\\
\frac{d}{dx} \log y_s &= -\frac{s}{x^2}=\frac{d}{dx}\frac{s}{x}
\\
y_s&=C(s)e^{s/x}.
\end{align*}
Die Konstante $C(s)$ muss aus der Anfangsbedingung bestimmt werden:
\[
y_s(1)={\cal L}u(1,s)=\frac1{s^2}=C(s)e^{s}
\qquad\Rightarrow\qquad
C(s)=\frac1{s^2}e^{-s}.
\]
Damit ist die Differentialgleichung~(\ref{50000011:odgl}) gel"ost, wir haben
\begin{equation}
y_s(x)
=
{\cal L}u(x,s)
=
\frac1{s^2}e^{\frac{s}{x}-s}
=
\frac1{s^2}e^{s(1-\frac1x)}.
\label{50000011:sol}
\end{equation}
Die R"ucktransformation war nicht verlangt, l"asst sich aber mit der
Verschiebungsformel bestimmen.
Diese besagt, dass $e^{-as}F(s)$ die Laplace-Transformierte von 
$f(t-a)$ ist, wenn $F(s)$ die Laplace-Transformierte von $f(t)$ ist.
Diese Situation liegt in (\ref{50000011:sol}) mit
\[
a = -\biggl(1-\frac1x\biggr)
\qquad\text{und}\qquad
F(s)=\frac1{s^2}
\]
vor.
Die L"osung ist daher
\[
u(x,t)
=
\begin{cases}
t-1+\frac1x
&\qquad t \ge 1-\frac1x
\\
0
&\qquad t < 1-\frac1x
\end{cases}
\]
Wir kontrollieren dieses Resultat durch Einsetzen, f"ur $t>1-\frac1x$
erhalten wir
\begin{align*}
\frac{\partial u}{\partial x}
&=-\frac1{x^2}
\\
\frac{\partial u}{\partial t}
&=1,
\end{align*}
und damit
\[
x^2\frac{\partial u}{\partial x}+\frac{\partial u}{\partial t}
=
x^2\cdot \biggl(-\frac1{x^2}\biggr) + 1 = -1 + 1=0,
\]
die Differentialgleichung ist also erf"ullt.
F"ur die Randbedingung erh"alt man
\[
u(1,t)=t-1+\frac11=t,
\]
somit ist auch die Randbedingung erf"ullt.
\begin{figure}
\centering
\includeagraphics[width=0.7\hsize]{loes.jpg}
\caption{L"osungsfl"ache der Differentialgleichung~(\ref{50000011:dgl})
in Aufgabe~\ref{50000011}
\label{50000011:graph}}
\end{figure}
\end{loesung}

\begin{diskussion}
Die Differentialgleichung ist ausserdem eine quaslineare partielle
Differentialgleichung erster Ordnung, die mit Charakteristiken gel"ost
werden kann.
Die Differentialgleichungen f"ur die Charkateristiken mit Parameter $s$
sind 
\[
\begin{aligned}
\dot x&=x^2 &&       &   &     \\
\dot t&=1   &&\Rightarrow& t &= s  \\
\dot u&=0   &&\Rightarrow& u &= u_0
\end{aligned}
\]
Die erste Gleichung kann mit Separation gel"ost werden:
\begin{align*}
\int \frac{dx}{x^2}
&=
\int dt+C
\\
-\frac1x
&=
t+C
\end{align*}
Damit f"ur $t=0$ der Wert $x_0$ entsteht, muss $C=-\frac1{x_0}$
gew"ahlt werden, es gilt also f"ur die im Punkt $(x_0,0)$ beginnende
Charakteristik 
\begin{align*}
-\frac1x&=s-\frac1{x_0}\\
       t&=s\\
       u&=0
\end{align*}
oder
\[
t= \frac1{x_0} - \frac1x.
\]
Da die Randbedingung entlang der $x$-Achse verschwindet ist
klar, dass im Teilgebiet $t<\frac1{x_0}-\frac1x$
die Funktion $u$ identisch verschwindet.

F"ur Charakteristiken, die im Punkt $(1,t_0)$ beginnen, muss man dagegen
$C=-1-t_0$ w"ahlen, und es gilt
\begin{align*}
-\frac1x&=s-1-t_0\\
       t&=s      \\
       u&=t_0.
\end{align*}
Daraus muss $t_0$ und $s$ eliminert werden, und es muss $u$ durch $x$ und $t$
ausgedr"uckt werden.
Man findet
\begin{align*}
-\frac1x&=t-1-u
\\
\Rightarrow\qquad
u&=t-1+\frac1x,
\end{align*}
also genau die L"osung, die auch mit der Laplace-Transformation gefunden wurde.
\end{diskussion}

\begin{bewertung}
Laplacetransformierte Differentialgleichung ({\bf D}) 1 Punkt,
Einsetzen der zweiten Randbedingung ({\bf Z}) 1 Punkt,
Laplacetransformierte der ersten Randbeidnung als Anfangsbedingung
der gew"ohnlichen Differentialgleichung ({\bf R}) 1 Punkt,
L"osung der homogenen Differentialgleichung ({\bf H}) 1 Punkt,
Bestimmung der Konstanten $C$ ({\bf C}) 1 Punkt,
(Laplace-Transformierte der) L"osung ({\bf L}) 1 Punkt.
\end{bewertung}
