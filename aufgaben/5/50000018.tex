For the second order partial differential equation 
\begin{equation}
x^2 \frac{\partial^2 u}{\partial t^2}
+
\frac{\partial^2 u}{\partial x^2}
=
0
\label{50000018:equation}
\end{equation}
on the domain 
\[
\Omega
= 
\{ (t,x)\in\mathbb R^2\,|\, t>0 \text{ and } x > 0\}
\]
boundary conditions
\[
\begin{aligned}
u(0,x) &= 0 & &\text{and}& \frac{\partial u}{\partial t}(0,x) &= 0 & &\text{for $x>0$}\\
u(t,0) &= \pi &&\text{and}&\frac{\partial u}{\partial x}(t,0)&=e^{-2t} &&\text{for $t>0$}
\end{aligned}
\]
are given.
\begin{teilaufgaben}
\item
Convert the partial differential equation \eqref{50000018:equation} into
a family of ordinary differential equations using the Laplace transform.
\item
Let $A(x)$ be any differentiable function and let $y(x)=A(\!\sqrt{s}x)$.
Compute $y(0)$ and $y'(0)$.
\item
Now let $A(x)$ be a solution of the ordinary differential equation
$A''+x^2A=0$, such functions are so called parabolic cylinder functions.
We know from the theory of ordinary differential equations that there are
two solutions $A(x)$ and $B(x)$ of $A''+x^2A=0$ with initial conditions
$A(0)=1$, $A'(0)=0$, $B(0)=0$ and $B'(0)=1$.
It can then be shown then that functions $x\mapsto A(\!\sqrt{s}x)$
and $x\mapsto B(\!\sqrt{s}x)$  are solutions of the equation found in a).
Express the Laplace transform $\mathcal{L}u(s,x)$ by means of the functions
$A(\!\sqrt{s}x)$ and $B(\!\sqrt{s}x)$.
\end{teilaufgaben}

\begin{hinweis}
Don't forget the initial conditions in a).
In b), you don't have to show that $A(\!\sqrt{s}x)$ is a solution of the
ordinary differential equation found in a), you can simply assume that.
\end{hinweis}

\begin{loesung}
\begin{teilaufgaben}
\item
The Laplace transform of the equation \eqref{50000018:equation} is
\[
x^2\biggl(
-u(0,x) -s\frac{\partial u}{\partial t}(0,x) +s^2\mathcal{L}u(s,x)
\biggr)
+
\frac{\partial^2}{\partial x^2} (\mathcal{L}u)(s,x)
=
0.
\]
Substituting the boundary conditions for $t=0$ gives
\begin{equation}
x^2s^2\mathcal{L}u(s,x)
+
\frac{\partial^2}{\partial x^2} (\mathcal{L}u)(s,x)
=
0
\qquad\Rightarrow\qquad
y_s''(x) + x^2s^2y_s(x)=0
\label{50000018:ode}
\end{equation}
using the customary notation $y_s(x)=(\mathcal{L}u)(s,x)$.

The initial conditions
$f(t)=\pi$
and
$g(t)=e^{-2t}$
also need to be transformed:
\begin{align*}
\mathcal{L}f(s)
&=
\int_0^\infty \pi\cdot e^{-st}\,dt
=
\biggl[ -\frac{\pi}{s}e^{-st}\biggr]_0^\infty
=
\frac{\pi}{s}
\\
\mathcal{L}g(s)
&=
\int_0^\infty e^{-2t} e^{-st}\,dt
=
\int_0^\infty e^{-t(2+s)} \,dt
=
\biggl[
\frac{-1}{2+s}e^{-t(2+s)}
\biggr]_0^\infty
=
\frac1{2+s}.
\end{align*}
So we have to find solutions $y_s(x)$ of the differential equation
\eqref{50000018:ode} with initial conditions
\[
y_s(x) = \frac{\pi}{s}.
\qquad
\text{and}
\qquad
y'_s(x) = \frac1{s+2}
\]
\item
It is clear that $y(0)=A(\!\sqrt{s}\cdot 0)=A(0)$.
For the derivative, we use the computation above, which showed
\[
y'(x) = \frac{d}{dx} A(\!\sqrt{s}x) = \sqrt{s}A'(\!\sqrt{s}x)
\qquad\Rightarrow\qquad
y'(0)=\sqrt{s}A'(\!\sqrt{s}\cdot 0)=\sqrt{s}A'(0).
\]
\item
We construct a linear combination 
\[
y_s(x) = aA(\!\sqrt{s}x) + b B(\!\sqrt{s}x)
\]
and we have to choose $a$ and $b$ in such way that the initial conditions
are satisfied.
So we compute the initial values
\begin{align*}
y_s(0) &= aA(\!\sqrt{s}0) + bB(\!\sqrt{s}\cdot 0) = aA(0) + bB(0) = a
&&\Rightarrow& a &= \frac{\pi}s,
\\
y_s'(0)&= a\sqrt{s} A'(0) + b\sqrt{s} B'(0) = b\sqrt{s}
&&\Rightarrow& b &= \frac{1}{\!\sqrt{s}(s+2)}.
\end{align*}
This allows us to write the function $y_s(x)$ as
\[
y_s(x) = (\mathcal{L}u)(s,x)
=
\frac{\pi}{s}
A(\!\sqrt{s}x)
+
\frac{1}{\!\sqrt{s}(s+2)}B(\!\sqrt{s}x).
\qedhere
\]
\end{teilaufgaben}
\end{loesung}

\begin{bewertung}
\begin{teilaufgaben}
\item
Transformed equation ({\bf E}) 1 point,
transformed boundary values for $f(t)$ ({\bf F}) 1 point,
transformed boundary values for $g(t)$ ({\bf G}) 1 point,
\item
derivatives ({\bf D}) 1 point,
\item
use initial condition to determine the linear combination ({\bf I}) 1 point,
linear combination and solution for Laplace transformed solution ({\bf L})
1 point.
\end{teilaufgaben}
\end{bewertung}


