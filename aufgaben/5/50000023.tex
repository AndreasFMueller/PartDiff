By differentiating the definition of the Laplace-transform with respect to $s$,
one can immediately derive the formula
\begin{equation}
\frac{d}{ds}\mathcal{L}f(s)
=
\frac{d}{ds}\int_0^\infty e^{-st} f(t)\,dt
=
\int_0^\infty -t e^{-st} f(t)\,dt
=
\mathcal{L}(-tf(t)) (s)
\quad\Rightarrow\quad
\mathcal{L}(tf(t)) (s) = -\frac{d}{ds}\mathcal{L}f(s).
\label{50000023:deriv}
\end{equation}
This means that the Laplace transform of $tf(t)$ is up to a sign the
derivative of the Laplace transform of $f(t)$.
\begin{teilaufgaben}
\item
Use the identity \eqref{50000023:deriv} to Laplace-transform the partial
differential equation
\begin{equation}
\frac{\partial^2u}{\partial x^2}(x,y)
+x\,u(x,y)
+\frac{\partial u}{\partial y}(x,y) = e^{-x}
\label{50000023:eqn}
\end{equation}
on the domain $\Omega=\{(x,y)\,|\,x>0\}$ with respect to the variable $x$.
The boundary conditions for a solution $u(x,y)$ for equation
\eqref{50000023:eqn} are
\begin{align*}
u(0,y) &= g(y)\\
\frac{\partial u}{\partial x}(u,y) &= 0
\end{align*}
for $y\in\mathbb R$.
\item
When using the transformation method, one usually expects to get an
ordinary differential equation.
This does not quite work in this example, but it is possible
to simplify the equation~\eqref{50000023:eqn}
somewhat to an equation for which we have a solution method.
What type of partial differential equation have you found?
\end{teilaufgaben}

\begin{hinweis}
You are not requested to derive formula~\eqref{50000023:deriv}, you can
simply use it to solve part a) and b).
\end{hinweis}

\begin{loesung}
\begin{teilaufgaben}
\item
The Laplace transform of the differential equation is
\begin{align*}
s^2(\mathcal{L}u)(s,y)
-
s\underbrace{\frac{\partial u}{\partial x}(0,y)}_{\textstyle=0}
-
\underbrace{u(0,y)}_{\textstyle=g(y)}
-\frac{\partial }{\partial s}(\mathcal{L}u)(s,y)
+
\frac{\partial }{\partial y}(\mathcal{L}u)(s,y)
&=
\frac1{s+1},
\end{align*}
where the fourth term uses the formula~\eqref{50000023:deriv}.
Boundary conditions for $u(x,y)$ can be used to simplify the
second and third term to give
\[
s^2(\mathcal{L}u)(s,y)-g(y) - \frac{\partial}{\partial s}(\mathcal{L}u)(s,y)
+
\frac{\partial}{\partial y}(\mathcal{L}u)(s,y)
=
\frac1{s+1}.
\]
We abbreviate the transformed function $u$ as $U(s,y) = \mathcal{L}u(s,y)$
and write the equation again as
\begin{equation}
s^2 U(s,y) - g(y) -\frac{\partial U}{\partial s}(s,y)
+ \frac{\partial U}{\partial y}(s,y)
=
\frac1{s+1}.
\label{50000023:firstorder}
\end{equation}
So we have managed to reduce the order, it is now a partial differential
equation of first order.
\item
By regrouping the equation \eqref{50000023:firstorder}  and collecting
the derivatives on the left hand side, we get
\begin{equation}
\frac{\partial U}{\partial s} - \frac{\partial U}{\partial y}
=
s^2 U - g(y) - \frac{1}{s+1}.
\label{50000023:qlpde}
\end{equation}
The equation~\eqref{50000023:qlpde} is a linear partial differential
equation of first order, which we could solve using the method of
characteristics.
\qedhere
\end{teilaufgaben}
\end{loesung}


\begin{bewertung}
\begin{teilaufgaben}
\item
Laplace transform of differential equation ({\bf L}) 2 points (1 point
for each side of the equation),
simplification using the boundary conditions ({\bf B}) 2 points,
1 point,
using the derivative equation~\eqref{50000023:deriv} to reduce the
equation further ({\bf D}) 1 point.
\item
Quasilinear PDE of first order ({\bf Q}) 1 point.
\end{teilaufgaben}
\end{bewertung}

