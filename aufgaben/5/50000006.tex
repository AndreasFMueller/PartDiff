Gegeben ist die Differentialgleichung
\[
\frac{\partial u}{\partial x}+\frac{\partial u}{\partial t}=1
\]
auf dem Gebiet $G=\{ (x,t)\,|\,t>0\}$ mit Anfangsbedingung
\[
u(x,0)=g(x).
\]
\begin{teilaufgaben}
\item
Wandeln Sie die partielle Differentialgleichung mit Hilfe der
Laplace-Transformation f"ur die Variable $t$ in eine Familie
gew"ohnlicher Differentialgleichungen um.
\item
L"osen Sie die partielle Differentialgleichung mit Hilfe der Charakteristiken.
\item 
Bestimmen Sie die Laplace-Transformation der in b) gefunden L"osung, und
pr"ufen Sie nach, ob sie eine L"osung der in a) gefunden Familie von
gew"ohnlichen Differentialgleichungen ist.
\end{teilaufgaben}

\begin{loesung}
\begin{teilaufgaben}
\item
Anwendung der Laplace-Transformation auf die Variable $t$ der
Differentialgleichung gibt die Differentialgleichung
\[
\frac{\partial}{\partial x}{\cal L}u(x,s)
+s{\cal L}u(x,s)-u(x,0)=\frac1s.
\]
Die rechte Seite ist die Laplace-Transformierte der Konstanten
$1$, also
\[
\int_0^\infty e^{-st}\,dt
=
\left[
-\frac1s e^{-st}
\right]_0^\infty=\frac1s.
\]
Schreibt man $y_s(x)={\cal L}u(x,s)$, dann hat man
f"ur jedes $s$  die Differentialgleichung
\[
y'_s(x)+sy_s(x)-g(x)=\frac1s
\]
zu l"osen.
\item
Die Charakteristiken haben die Differentialgleichung
\[
\frac{d}{d\tau}\begin{pmatrix}
x\\t\\u
\end{pmatrix}
=\begin{pmatrix}
1\\1\\1
\end{pmatrix}.
\]
mit den L"osungen
\begin{align*}
x&=\tau + x_0\\
t&=\tau + t_0\\
u&=\tau + u_0
\end{align*}
Anwendung der Randbedingungen liefert nacheinander
\begin{align*}
t&=\tau\\
x&=t+x_0
&\Rightarrow&x_0=x-t\\
u&=t+g(x_0)=t+g(x-t).
\end{align*}
Durch Einsetzen in die Differentialgleichung kann man "uberpr"ufen,
dass $u(x,t)=t+g(x-t)$ tats"achlich eine L"osung ist:
\[
\frac{\partial u}{\partial x}+\frac{\partial u}{\partial t}
=
g'(x-t)+1-g'(x-t)=1.
\]
\item
Die Laplace-Transformierte der Funktion $t\mapsto t+g(x-t)$ ist
\[
y_s(x)=\int_0^\infty (t+g(x-t))e^{-st}\,dt.
\]
Um diese Funktion in die Differentialgleichung einsetzen zu k"onnen,
braucht man die Ableitung
\begin{align*}
y'_s(x)=\int_0^\infty g'(x-t)e^{-st}\,dt
&=\left[-g(x-t)e^{-st}\right]_0^\infty
-\int_0^\infty sg(x-t)e^{-st}\,dt
\\
&=g(x)-s\int_0^\infty g(x-t) e^{-st}\,dt
\end{align*}
Setzt man dies in die Differntialgleichung ein, erh"alt man
\begin{align*}
y'_s(x)+sy_s(x)-g(x)&=g(x)-s\int_0^\infty g(x-t)e^{-st}\,dt
\\
&\qquad
+s\int_0^\infty (t+g(x-t)) e^{-st}\,dt-g(x)
\\
&=s\int_0^\infty te^{-st}\,dt
\\
&=s\left[-\frac{t}{s}e^{-st}\right]_0^\infty
+s\int_0^\infty\frac{1}{s}e^{-st}\,dt
\\
&=\left[-\frac1s e^{-st}\right]_0^\infty
=\frac1s,
\end{align*}
die Differentialgleichung ist also erf"ullt.
\end{teilaufgaben}
\end{loesung}
