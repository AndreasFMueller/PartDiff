Use the Laplace transform to reduce the partial differential equation
\begin{equation}
\frac{\partial u}{\partial x}
+
\frac{1}{\tan y}\frac{\partial u}{\partial y}
=
u
\label{50000025:eqn}
\end{equation}
on the domain
\[
\Omega
=
\biggl\{
(x,y)
\;\bigg|\;
x > 0, 1 < y < \frac32
\biggr\}
\]
with boundary conditions
\begin{equation}
\left.
\begin{aligned}
u(x,1) &= e^{-ax} &&\text{for} & 0<\mathstrut&x\\
u(0,y) &= y       &&\text{for} & 1<\mathstrut&y<\frac32
\end{aligned}
\quad\right\}
\label{50000025:boundary}
\end{equation}
for some $a>0$
to a family of ordinary differential equations with suitable boundary
conditions.
Do the boundary conditions~\eqref{50000025:boundary} uniquely determine
the solution?

\begin{loesung}
Applying the Laplace transform to the variable $x$ in the partial
differential equation~\eqref{50000025:eqn} equation gives
\[
s(\mathscr{L}u)(s,y)
-
\underbrace{u(0,y)}_{\displaystyle=y}
+
\cot y \frac{\partial}{\partial y}(\mathscr{L}u)(s,y)
=
(\mathscr{L}u)(s,y).
\]
The first boundary condition must also be transformed and gives
\[
(\mathscr{L}u)(s,1)
=
(\mathscr{L}e^{-ax})(s)
=
\frac{1}{s+a}.
\]
If we write $f_s(y)=(\mathscr{L}u)(s,y)$ and consider $s$ as a mere
parameter, we get the ordinary differential equation
\begin{equation}
\cot y \cdot f'_s(y) 
+(s-1) f_s(y)
=
y
\label{50000025:ordinary}
\end{equation}
on the interval $y\in [1,\frac32)$ with initial condition
\[
f_s(1) = \frac{1}{s+a}.
\]
The differential equation~\eqref{50000025:ordinary} is a linear
differential equation of first order with bounded coefficients
in the specified domain, so it has a unique solution.
From this we can conclude that the solution of the original 
partial differential equation also has a unique solution.
\end{loesung}

\begin{bewertung}
Choice of variable to transform ({\bf X}) 1 point,
transform of the equation ({\bf E}) 1 point,
simplification using the boundary condition ({\bf U}$\mathstrut_0$) 1 point,
transform of the other boundary condition ({\bf B}) 1 point,
reformulation as an ordinary differential equation problem ({\bf O}) 1 point,
argument for well-posedness ({\bf W}) 1 point.
\end{bewertung}
