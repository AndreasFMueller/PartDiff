%
% aufgabensammlung.tex -- exercise collection
%               gehalten an der Hochschule Rapperswil im Wintersemester 09
%
% (c) 2009-2019 Prof. Dr. Andreas Mueller, HSR
%
\documentclass[a4paper,12pt]{book}
\usepackage[utf8]{inputenc}
\usepackage[T1]{fontenc}
%\usepackage{german}
\usepackage{times}
\usepackage{geometry}
\geometry{papersize={210mm,297mm},total={160mm,240mm},top=31mm,bindingoffset=15mm}
\usepackage{alltt}
\usepackage{verbatim}
\usepackage{fancyhdr}
\usepackage{amsmath}
\usepackage{amssymb}
\usepackage{amsfonts}
\usepackage{amsthm}
\usepackage{textcomp}
\usepackage{graphicx}
\usepackage{array}
\usepackage{ifthen}
\usepackage{multirow}
\usepackage{txfonts}
\usepackage{mathrsfs}
\usepackage[all]{xy}
\usepackage{algorithm}
\usepackage{algorithmic}
\usepackage{makeidx}
\usepackage{paralist}
\usepackage{hyperref}
\usepackage{environ}
\usepackage{color}
\usepackage{tikz}
%%%%%%%%%%%%%%%%%%%%%%%
%% Copyleft
%% Walter A. Kehowski
%% Department of Mathematics
%% Glendale Community College
%% walter.kehowski@gcmail.maricopa.edu
%% \begin{linsys}{2}
%% -x & + & 4y & = & 8\\
%% -3x & - & 2y & = & 6
%% \end{linsys}
%%%%%%%%%%%%%%%%%%%%%%%
%\makeatletter
%% math-mode column types ------------------
\newcolumntype{\linsysR}{>{$}r<{$}}
\newcolumntype{\linsysL}{>{$}l<{$}}
\newcolumntype{\linsysC}{>{$}c<{$}}
\newenvironment{linsys}[1]{%
\begin{tabular}{*{#1}{\linsysR@{\;}\linsysC}@{\;}\linsysR}}%
{\end{tabular}}
%\makeatother
\endinput

\makeindex
\begin{document}
\pagestyle{fancy}
\lhead{Exercise collection}
\rhead{}
\frontmatter
\newcommand\HRule{\noindent\rule{\linewidth}{1.5pt}}
\begin{titlepage}
\vspace*{\stretch{1}}
\HRule
\vspace*{2pt}
\begin{flushright}
{\Huge
Partial differential equations\\
\bigskip
Exercise collection}
\end{flushright}
\HRule
\begin{flushright}
\vspace{30pt}
\LARGE
Andreas Müller
\end{flushright}
\vspace*{\stretch{2}}
\begin{center}
OST Ostschweizer Fachhochschule, Rapperswil, 2008-2022
\end{center}
\end{titlepage}
\hypersetup{
	colorlinks=true,
	linktoc=all,
	linkcolor=blue
}
\rhead{Contents}
\tableofcontents
\newenvironment{beispiel}[1][Example]{%
\begin{proof}[#1]%
\renewcommand{\qedsymbol}{$\bigcirc$}
}{\end{proof}}
\mainmatter
\input uebungen.tex
\setboolean{loesungen}{true}
\aufgabetoplevel{./}

\chapter{Introductory examples}
\rhead{Introductory examples}
\input 1.tex
\chapter{Terminology and Notation}
\rhead{Terminology and Notation}
\input 2.tex
\chapter{Geometry of partial differential equations of first order
\label{chapter:separation}}
\rhead{PDE of first order}
\input 3.tex
\chapter{Separation of Variables\label{chapter:separation}}
\rhead{Separation}
\input 4.tex
\chapter{Transformation}
\rhead{Transformation}
Some problems from chapter~\ref{chapter:separation}
can also be solved with the transform method.
They are shown again in this chapter with an alternative solution.
\bigskip

\input 5.tex
\chapter{Partial differential equations of second order}
\rhead{PDE of second order}
\input 6.tex
\chapter{Elliptic partial differential equations}
\rhead{Elliptic PDE}
\input 7.tex
\chapter{Parabolic partial differential equations}
\rhead{Parabolic PDE}
\input 8.tex
\chapter{Hyperbolic partial differential equations}
\rhead{Hyperbolic PDE}
\input 9.tex
%\chapter{Nichtlineare partielle Differentialgleichungen}
%\rhead{Nichtlineare PDGL}
%\input a.tex
\chapter{Exam problems for PDE}
\rhead{Numerics of PDE}
\input b.tex
\end{document}
