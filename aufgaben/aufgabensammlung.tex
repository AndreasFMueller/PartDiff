%
% aufgabensammlung.tex -- exercise collection
%
% (c) 2009-2023 Prof. Dr. Andreas Mueller, HSR/OST
%
\documentclass[a4paper,12pt]{book}
\usepackage[utf8]{inputenc}
\usepackage[T1]{fontenc}
\usepackage{times}
\usepackage{geometry}
\geometry{papersize={210mm,297mm},total={160mm,240mm},top=31mm,bindingoffset=15mm}
\usepackage{alltt}
\usepackage{verbatim}
\usepackage{fancyhdr}
\usepackage{amsmath}
\usepackage{amssymb}
\usepackage{amsfonts}
\usepackage{amsthm}
\usepackage{textcomp}
\usepackage{graphicx}
\usepackage{array}
\usepackage{ifthen}
\usepackage{multirow}
\usepackage{txfonts}
\usepackage{mathrsfs}
\usepackage[all]{xy}
\usepackage{algorithm}
\usepackage{algorithmic}
\usepackage{makeidx}
\usepackage{paralist}
\usepackage{hyperref}
\usepackage{environ}
\usepackage{color}
\usepackage{tikz}
%%%%%%%%%%%%%%%%%%%%%%%
%% Copyleft
%% Walter A. Kehowski
%% Department of Mathematics
%% Glendale Community College
%% walter.kehowski@gcmail.maricopa.edu
%% \begin{linsys}{2}
%% -x & + & 4y & = & 8\\
%% -3x & - & 2y & = & 6
%% \end{linsys}
%%%%%%%%%%%%%%%%%%%%%%%
%\makeatletter
%% math-mode column types ------------------
\newcolumntype{\linsysR}{>{$}r<{$}}
\newcolumntype{\linsysL}{>{$}l<{$}}
\newcolumntype{\linsysC}{>{$}c<{$}}
\newenvironment{linsys}[1]{%
\begin{tabular}{*{#1}{\linsysR@{\;}\linsysC}@{\;}\linsysR}}%
{\end{tabular}}
%\makeatother
\endinput

\newboolean{pruefung}
\setboolean{pruefung}{false}
%
% packages.tex
%
% (c) 2023 Prof Dr Andreas Müller
%
\usepackage[utf8]{inputenc}
\usepackage[T1]{fontenc}
\usepackage{times}
\usepackage{geometry}
\geometry{papersize={210mm,297mm},total={160mm,240mm},top=31mm,bindingoffset=15mm,marginparwidth=9mm}
\usepackage{alltt}
\usepackage{verbatim}
\usepackage{fancyhdr}
\usepackage{amsmath}
\usepackage{amssymb}
\usepackage{amsfonts}
\usepackage{amsthm}
\usepackage{textcomp}
\usepackage{graphicx}
\usepackage{array}
\usepackage{ifthen}
\usepackage{multirow}
\usepackage{txfonts}
\usepackage{mathrsfs}
\usepackage{mathtools}
\usepackage[all]{xy}
\usepackage{algorithm}
\usepackage{algorithmic}
\usepackage{makeidx}
\usepackage{paralist}
\usepackage[colorlinks=true]{hyperref}
\usepackage{environ}
\usepackage{color}
\usepackage{tikz}
\usepackage{etoolbox}
\usepackage{xr}
\usepackage{xstring}
%%%%%%%%%%%%%%%%%%%%%%%
%% Copyleft
%% Walter A. Kehowski
%% Department of Mathematics
%% Glendale Community College
%% walter.kehowski@gcmail.maricopa.edu
%% \begin{linsys}{2}
%% -x & + & 4y & = & 8\\
%% -3x & - & 2y & = & 6
%% \end{linsys}
%%%%%%%%%%%%%%%%%%%%%%%
%\makeatletter
%% math-mode column types ------------------
\newcolumntype{\linsysR}{>{$}r<{$}}
\newcolumntype{\linsysL}{>{$}l<{$}}
\newcolumntype{\linsysC}{>{$}c<{$}}
\newenvironment{linsys}[1]{%
\begin{tabular}{*{#1}{\linsysR@{\;}\linsysC}@{\;}\linsysR}}%
{\end{tabular}}
%\makeatother
\endinput


\makeindex
\begin{document}
%
% front.tex
%
% (c) 2023 Prof Dr Andreas Müller
%
\pagestyle{fancy}
\lhead{Exercise collection}
\rhead{}
\frontmatter
\newcommand\HRule{\noindent\rule{\linewidth}{1.5pt}}
\begin{titlepage}
\vspace*{\stretch{1}}
\HRule
\vspace*{2pt}
\begin{flushright}
{\Huge
Partial differential equations\\
\bigskip
Exercise collection}
\end{flushright}
\HRule
\begin{flushright}
\vspace{30pt}
\LARGE
Andreas Müller
\end{flushright}
\vspace*{\stretch{2}}
\begin{center}
OST Ostschweizer Fachhochschule, Rapperswil, 2008-2023
\end{center}
\end{titlepage}
\hypersetup{
	colorlinks=true,
	linktoc=all,
	linkcolor=blue
}
\rhead{Contents}
\tableofcontents

\mainmatter
%
% uebung.tex -- gemeinsame Makros fuer Uebungsblaetter
%
% (c) 2006 Prof. Dr. Andreas Mueller, HSR
% $Id: uebung.tex,v 1.3 2008/01/06 14:05:56 afm Exp $
%
\newcounter{uebungsaufgabe}
\newboolean{loesungen}
% environment fuer uebungsaufgaben
\newenvironment{uebungsaufgaben}{
\begin{list}{\arabic{uebungsaufgabe}.}
  {\usecounter{uebungsaufgabe}
  \setlength{\labelwidth}{2cm}
  \setlength{\leftmargin}{0pt}
  \setlength{\labelsep}{5mm}
  \setlength{\rightmargin}{0pt}
  \setlength{\itemindent}{0pt}
}}{\end{list}\vfill\pagebreak}
% Teilaufgaben
\newenvironment{teilaufgaben}{
\begin{enumerate}
\renewcommand{\labelenumi}{\alph{enumi})}
}{\end{enumerate}}
% Loesung
\def\swallow#1{
%nothing
}
\NewEnviron{loesung}{%
\begin{proof}[Solution]%
\renewcommand{\qedsymbol}{$\bigcirc$}
\BODY
\end{proof}}
\NewEnviron{diskussion}{
\begin{proof}[Discusson]
\renewcommand{\qedsymbol}{}
\BODY
\end{proof}
}
\NewEnviron{bewertung}{\relax}
\def\keineloesungen{%
\RenewEnviron{loesung}{\relax}
\RenewEnviron{diskussion}{\relax}
}
\def\bewertungen{
\RenewEnviron{bewertung}{\begin{proof}[Valuation]
\renewcommand{\qedsymbol}{}
\BODY
\end{proof}}
}

% Hinweis
\newenvironment{hinweis}{%
\renewcommand{\qedsymbol}{}
\begin{proof}[Hint]}{\end{proof}}
% Aufgabe aus der Sammlung wiedergeben
\newcounter{problemcounter}[chapter]
\def\aufgabepath{./}
\def\ainput#1{\input\aufgabepath/#1}
\def\verbatimainput#1{\expandafter\verbatiminput{\aufgabepath/#1}}
\def\aufgabetoplevel#1{%
\expandafter\def\expandafter\inputpath{#1}%
\let\aufgabepath=\inputpath
}
\def\includeagraphics[#1]#2{\expandafter\includegraphics[#1]{\aufgabepath#2}}
% \aufgabe
%\newcommand{\aufgabe}[2]{%
%\refstepcounter{problemcounter}%
%\label{#2}
%\bigskip{\parindent0pt\strut}\hbox{\bf\theproblemcounter. }%
%\marginpar{\raggedright\tiny #2}%
%\expandafter\def\csname aufgabepath\endcsname{\inputpath/#1/#2/}%
%\expandafter\input{\inputpath#1/#2.tex}
%\bigskip
%}
\newcommand{\aufgabe}[1]{%
\StrRemoveBraces{#1}[\FirstChar]%
\StrChar{\FirstChar}{1}[\FirstChar]%
  \expandafter\def\csname themalist\endcsname{}
  \refstepcounter{problemcounter}%
  \label{#1}
  \bigskip{\parindent0pt\strut}\hbox{\bf\theproblemcounter. }%
  \marginpar{\raggedright\tiny #1}%
  \expandafter\def\csname currentaufgabe\endcsname{#1}%
  \expandafter\def\csname aufgabepath\endcsname{\inputpath/\FirstChar/#1/}%
  \expandafter\input{\inputpath\FirstChar/#1.tex}
  %\medskip
  \bigskip

}
\renewcommand\theproblemcounter{\thechapter.\arabic{problemcounter}}
% oft benutzte Macros
\def\blank{\text{\textvisiblespace}}
\newenvironment{beispiel}[1][Example]{%
\begin{proof}[#1]%
\renewcommand{\qedsymbol}{$\bigcirc$}
}{\end{proof}}


\setboolean{loesungen}{true}
\aufgabetoplevel{./}

\chapter{Introductory examples}
\rhead{Introductory examples}
\input{work/1.tex}
\chapter{Terminology and Notation}
\rhead{Terminology and Notation}
\input{work/2.tex}
\chapter{Geometry of partial differential equations of first order
\label{chapter:separation}}
\rhead{PDE of first order}
\input{work/3.tex}
\chapter{Separation of Variables\label{chapter:separation}}
\rhead{Separation}
\input{work/4.tex}
\chapter{Transformation}
\rhead{Transformation}
Some problems from chapter~\ref{chapter:separation}
can also be solved with the transform method.
They are shown again in this chapter with an alternative solution.
\bigskip

\input{work/5.tex}
\chapter{Partial differential equations of second order}
\rhead{PDE of second order}
\input{work/6.tex}
\chapter{Elliptic partial differential equations}
\rhead{Elliptic PDE}
\input{work/7.tex}
\chapter{Parabolic partial differential equations}
\rhead{Parabolic PDE}
\input{work/8.tex}
\chapter{Hyperbolic partial differential equations}
\rhead{Hyperbolic PDE}
\input{work/9.tex}
%\chapter{Nichtlineare partielle Differentialgleichungen}
%\rhead{Nichtlineare PDGL}
%\input{a.tex}
\chapter{Exam problems for PDE}
\rhead{Numerics of PDE}
\input{work/b.tex}
\end{document}
