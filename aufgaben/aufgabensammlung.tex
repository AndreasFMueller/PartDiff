%
% aufgabensammlung.tex -- exercise collection
%
% (c) 2009-2023 Prof. Dr. Andreas Mueller, HSR/OST
%
\documentclass[a4paper,12pt]{book}
\usepackage[utf8]{inputenc}
\usepackage[T1]{fontenc}
\usepackage{times}
\usepackage{geometry}
\geometry{papersize={210mm,297mm},total={160mm,240mm},top=31mm,bindingoffset=15mm}
\usepackage{alltt}
\usepackage{verbatim}
\usepackage{fancyhdr}
\usepackage{amsmath}
\usepackage{amssymb}
\usepackage{amsfonts}
\usepackage{amsthm}
\usepackage{textcomp}
\usepackage{graphicx}
\usepackage{array}
\usepackage{ifthen}
\usepackage{multirow}
\usepackage{txfonts}
\usepackage{mathrsfs}
\usepackage[all]{xy}
\usepackage{algorithm}
\usepackage{algorithmic}
\usepackage{makeidx}
\usepackage{paralist}
\usepackage{hyperref}
\usepackage{environ}
\usepackage{color}
\usepackage{tikz}
\input{../common/linsys.tex}
\newboolean{pruefung}
\setboolean{pruefung}{false}
%
% packages.tex
%
% (c) 2023 Prof Dr Andreas Müller
%
\usepackage[utf8]{inputenc}
\usepackage[T1]{fontenc}
\usepackage{times}
\usepackage{geometry}
\geometry{papersize={210mm,297mm},total={160mm,240mm},top=31mm,bindingoffset=15mm,marginparwidth=9mm}
\usepackage{alltt}
\usepackage{verbatim}
\usepackage{fancyhdr}
\usepackage{amsmath}
\usepackage{amssymb}
\usepackage{amsfonts}
\usepackage{amsthm}
\usepackage{textcomp}
\usepackage{graphicx}
\usepackage{array}
\usepackage{ifthen}
\usepackage{multirow}
\usepackage{txfonts}
\usepackage{mathrsfs}
\usepackage{mathtools}
\usepackage[all]{xy}
\usepackage{algorithm}
\usepackage{algorithmic}
\usepackage{makeidx}
\usepackage{paralist}
\usepackage[colorlinks=true]{hyperref}
\usepackage{environ}
\usepackage{color}
\usepackage{tikz}
\usepackage{etoolbox}
\usepackage{xr}
\usepackage{xstring}
\input{../common/linsys.tex}

\makeindex
\begin{document}
%
% front.tex
%
% (c) 2023 Prof Dr Andreas Müller
%
\pagestyle{fancy}
\lhead{Exercise collection}
\rhead{}
\frontmatter
\newcommand\HRule{\noindent\rule{\linewidth}{1.5pt}}
\begin{titlepage}
\vspace*{\stretch{1}}
\HRule
\vspace*{2pt}
\begin{flushright}
{\Huge
Partial differential equations\\
\bigskip
Exercise collection}
\end{flushright}
\HRule
\begin{flushright}
\vspace{30pt}
\LARGE
Andreas Müller
\end{flushright}
\vspace*{\stretch{2}}
\begin{center}
OST Ostschweizer Fachhochschule, Rapperswil, 2008-2023
\end{center}
\end{titlepage}
\hypersetup{
	colorlinks=true,
	linktoc=all,
	linkcolor=blue
}
\rhead{Contents}
\tableofcontents

\mainmatter
\input{macros/uebungen.tex}
\setboolean{loesungen}{true}
\aufgabetoplevel{./}

\chapter{Introductory examples}
\rhead{Introductory examples}
\input{work/1.tex}
\chapter{Terminology and Notation}
\rhead{Terminology and Notation}
\input{work/2.tex}
\chapter{Geometry of partial differential equations of first order
\label{chapter:separation}}
\rhead{PDE of first order}
\input{work/3.tex}
\chapter{Separation of Variables\label{chapter:separation}}
\rhead{Separation}
\input{work/4.tex}
\chapter{Transformation}
\rhead{Transformation}
Some problems from chapter~\ref{chapter:separation}
can also be solved with the transform method.
They are shown again in this chapter with an alternative solution.
\bigskip

\input{work/5.tex}
\chapter{Partial differential equations of second order}
\rhead{PDE of second order}
\input{work/6.tex}
\chapter{Elliptic partial differential equations}
\rhead{Elliptic PDE}
\input{work/7.tex}
\chapter{Parabolic partial differential equations}
\rhead{Parabolic PDE}
\input{work/8.tex}
\chapter{Hyperbolic partial differential equations}
\rhead{Hyperbolic PDE}
\input{work/9.tex}
%\chapter{Nichtlineare partielle Differentialgleichungen}
%\rhead{Nichtlineare PDGL}
%\input{a.tex}
\chapter{Exam problems for PDE}
\rhead{Numerics of PDE}
\input{work/b.tex}
\end{document}
