Bestimmen Sie den Rand der folgenden Gebiete
\begin{teilaufgaben}
\item
$\Omega=\{(x,y)\in\mathbb R^2\,|\, x^2+y^2<25\}$.
\item
$\Omega=\{(x,y)\in\mathbb R^2\,|\, |x|+|y|<1\}$.
\end{teilaufgaben}

\begin{loesung}
\begin{teilaufgaben}
\item
$\Omega$ ist die Kreisscheibe mit Radius $5$, der Rand ist die Menge
\[
\partial\Omega=\{(x,y)\in\mathbb R^2\,|\, x^2+y^2=25\},
\]
als ein Kreis mit Radius $5$ (Abbildung \ref{20000003:fig} rechts).
\item
Da $|x|=|-x|$ und $|y|=|-y|$ muss $\Omega$ symmetrisch bez"uglich
Spiegelungen an der $x$- und $y$-Achse sein. Es gen"ugt daher,
den Teil im ersten Quadranten zu bestimmen.  Ein Punkt $(x,y)$
im ersten Quadranten ist genau dann in $\Omega$, wenn $x+y<1$ ist,
also $y<1-x$. Es geh"oren also alle Punkte ``unter'' der Geraden
$y=1-x$ dazu. $\Omega$ besteht also aus den Punkten des
Quadrates mit den Ecken $(1,0)$, $(0,1)$, $(-1,0)$ und $(0,-1)$,
und der Rand $\partial\Omega$ ist der Rand dieses Quadrates
(Abbildung \ref{20000003:fig} links).
\end{teilaufgaben}
\begin{figure}
\begin{center}
\includeagraphics[]{graphs-1}
\qquad
\includeagraphics[]{graphs-2}
\end{center}
\caption{Gebiete und Rand f"ur Aufgabe \ref{20000003} a) (links)
und b) (rechts)\label{20000003:fig}}
\end{figure}
\end{loesung}
