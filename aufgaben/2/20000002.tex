Welche der folgenden Mengen sind Gebiete
\begin{teilaufgaben}
\item $\Omega=\{(x,y)\in\mathbb R^2\,|\, \text{$x$ hat eine reelle Quadratwurzel}\}$
\item $\Omega=\{(x,y)\in\mathbb R^2\,|\, x^2+y^2>0\}$.
\end{teilaufgaben}


\begin{loesung}
\begin{teilaufgaben}
\item
Genau die Zahlen $x\ge 0$ haben eine reelle Quadratwurzel, $\Omega$ ist
also eigentlich die Menge
\[
\Omega=\{(x,y)\in\mathbb R^2\,|\, x\ge 0\}.
\]
Dies ist die rechte Halbebene inklusive der $y$-Achse. Da der Rand
auch $\Omega$ enthalten ist, ist $\Omega$ kein Gebiet.
\item
$\Omega$ umfasst alle Punkte ausser dem Nullpunkt. Ist $r$ der Abstand
des Punktes $(x,y)$ von $(0,0)$, dann ist offenbar der Ball $B((x,y),r)$
immer noch in der Menge $\Omega$ enthalten. $\Omega$ ist also eine offene
Menge, $\Omega$ ist ein Gebiet.
\end{teilaufgaben}
\end{loesung}
