Welche der folgenden Mengen sind Gebiete?
\begin{teilaufgaben}
\item $\Omega=\{(x,y)\in\mathbb R^2\,|\, \text{$x$ hat eine reelle Quadratwurzel}\}$
\item $\Omega=\{(x,y)\in\mathbb R^2\,|\, x^2+y^2>0\}$.
\end{teilaufgaben}


\begin{loesung}
\begin{teilaufgaben}
\item
Genau die Zahlen $x\ge 0$ haben eine reelle Quadratwurzel, $\Omega$ ist
also eigentlich die Menge
\[
\Omega=\{(x,y)\in\mathbb R^2\,|\, x\ge 0\}.
\]
Dies ist die rechte Halbebene inklusive der $y$-Achse. Da der Rand
auch $\Omega$ enthalten ist, ist $\Omega$ kein Gebiet.
(Abbildung \ref{20000002:fig} links).
\item
$\Omega$ umfasst alle Punkte ausser dem Nullpunkt. Ist $r$ der Abstand
des Punktes $(x,y)$ von $(0,0)$, dann ist offenbar der Ball $B((x,y),r)$
immer noch in der Menge $\Omega$ enthalten. $\Omega$ ist also eine offene
Menge, $\Omega$ ist ein Gebiet.

Der Rand des Gebietes $\Omega$ ist also nur der Nullpunkt. F"ugt man
aber den Rand hinzu, erh"alt man die ganze Ebene, die wieder eine
Gebiet ist. Man darf also nicht sagen, wenn man einem Gebiet den Rand
hinzuf"ugt, w"urde daraus etwas, was kein Gebiet ist. Man muss sich immer
auf die Definition st"utzen, dass Gebiete offene Mengen sein m"ussen.
(Abbildung \ref{20000002:fig} rechts).
\qedhere
\end{teilaufgaben}
\begin{figure}
\begin{center}
\includeagraphics[]{graphs-1}
\qquad
\includeagraphics[]{graphs-2}
\end{center}
\caption{Teilmengen $\Omega\subset\mathbb R^2$ f"ur Aufgaben \ref{20000002} a)
(links) 
und b) (rechts)\label{20000002:fig}}
\end{figure}
\end{loesung}
