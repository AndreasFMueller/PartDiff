Bestimmen Sie von den folgenden Differentialgleichungen die Ordnung
und geben Sie an ob Sie linear oder quasilinear sind.
\begin{teilaufgaben}
\item $\displaystyle \left(\frac{\partial u}{\partial x}\right)^2+\left(\frac{\partial u}{\partial y}\right)^2=0$
\item $\displaystyle
\left(\frac{\partial}{\partial x}\right)^2u+\left(\frac{\partial}{\partial y}\right)^2u=0$
\item $\displaystyle
\left(\frac{\partial}{\partial x}+\frac{\partial}{\partial y}\right)u=0
$
\item $\displaystyle
u\left(\frac{\partial}{\partial x}+\frac{\partial}{\partial y}\right)u=0
$
\item $\displaystyle
\partial_1\partial_2\partial_3\dots\partial_n u=0
$
\item $\displaystyle
\partial_1u\partial_2u\partial_3\dots u\partial_n u=0
$
\item $\displaystyle
(\partial_1u)(\partial_2u)(\partial_3u)\dots (\partial_n u)=0
$
\item $\displaystyle
\left(\frac{\partial}{\partial x}+\frac{\partial}{\partial y}\right)u^2=0
$
\end{teilaufgaben}

\begin{loesung}
Eine PDGL ist quasilinear, wenn die Ableitungen linear vorkommen.
In der Teilaufgabe h) sieht man dies erst nach Anwendung der Produktregel:
\[
\left(\frac{\partial}{\partial x}+\frac{\partial}{\partial y}\right)u^2
=
2u\frac{\partial u}{\partial x}+2u\frac{\partial u}{\partial y}=0.
\]
\begin{center}
\begin{tabular}{|c|c|c|c|}
\hline
Gleichung&Ordnung&linear&quasilinear\\
\hline
a)&1&nein&nein\\
b)&2&ja&ja\\
c)&1&ja&ja\\
d)&1&nein&ja\\
e)&$n$&ja&ja\\
f)&$n$&nein&nein\\
g)&1&nein&nein\\
h)&1&nein&ja\\
\hline
\end{tabular}
\end{center}
\end{loesung}
