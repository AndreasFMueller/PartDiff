Teilen sie die folgenden partiellen Differentialgleichungen
in elliptische, parabolische und hyperbolische ein und markieren
Sie auch diejenigen, die nicht in dieses Klassifikationsschema passen.
\begin{teilaufgaben}
\item $\partial_x\partial_y\partial_z u=0$
\item $\partial_x^2u=\partial_y^2u$
\item $\partial_x^2u+4\partial_x\partial_yu+\partial_y^2u=0$
\item $\partial_x^2u-4\partial_x\partial_yu+\partial_y^2u=0$
\item $\partial_x^2u-\partial_x\partial_yu+\partial_y^2u=0$
\item $\Delta u=\kappa\partial_t^2u$, wobei $\kappa\in \mathbb R$ ist.
\end{teilaufgaben}

\begin{loesung}
\begin{teilaufgaben}
\item Dies ist eine PDGL dritter Ordnung, passt also nicht ins
Klassifikationsschema.
\item Dies ist die PDGL
\[
\frac{\partial^2u}{\partial x^2}-\frac{\partial^2 u}{\partial y^2}=0,
\]
die Koeffizientenmatrix der zweiten Ableitungen ist also
\[
\begin{pmatrix}
1&0\\0&-1
\end{pmatrix}
\]
welche zwei Eigenwerte mit verschiedenen Vorzeichen hat, diese Gleichung
ist also hyperbolisch.
\item Die Koeffizientenmatrix ist
\[
A=\begin{pmatrix}
1&2\\2&1
\end{pmatrix}.
\]
Die Eigenwerte k"onnen mit Hilfe der charakteristischen Gleichung
gefunden werden:
\begin{align*}
\det(A-\lambda E)
&=
\left|\begin{matrix} 1-\lambda&2\\2&1-\lambda\end{matrix}\right|
\\
&=(1-\lambda)^2-4\\
&=\lambda^2-2\lambda-3=0
\end{align*}
welche die Nullstellen $\lambda_1=3$ und $\lambda_2=-1$ hat.
Insbesondere ist die Gleichung also hyperbolisch.
\item
In diesem Fall ist das charakteristische Polynom das selbe wie in
der vorangegangenen Teilaufgabe, also ist auch diese Differentialgleichung
hyperbolisch.
\item Die Koeffizientenmatrix
\[
\begin{pmatrix}
1&-\frac12\\
-\frac12&1
\end{pmatrix}
\]
hat das charakteristische Polynom
\[
(1-\lambda)^2-\frac14=\lambda^2-2\lambda-\frac34
\]
mit den Nullstellen $\lambda_1=\frac32$ und $\lambda_2=\frac12$,
diese Gleichung ist also elliptisch.
\item
Bringt man alle Terme auf eine Seite, wird daraus die Differentialgleichung
\[
\partial_1^2u+\dots\partial_n^2u-\kappa\partial_t^2u=0
\]
mit der Koeffizientenmatrix
\[
\begin{pmatrix}
1&0&\dots&0\\
0&1&\dots&0\\
\vdots&\vdots&\ddots&\vdots\\
0&0&\dots&-\kappa\\
\end{pmatrix}
\]
also hyperbolisch f"ur $\kappa > 0$ und elliptisch f"ur $\kappa < 0$.
F"ur $\kappa=0$ ist die Gleichung parabolisch.
\end{teilaufgaben}
\end{loesung}
