Die Funktion $y(x)$ erf"ulle die Differentialgleichung
\begin{equation}
y''+\beta y' + \gamma y=0.
\label{60000003:vorgabe}
\end{equation}
Setzen Sie $u(x)=e^{k x}y(x)$.
Wie m"ussen Sie $k$ und $\alpha$ w"ahlen, damit $u$ L"osung der
Differentialgleichung
\begin{equation}
u''+\alpha u=0
\label{60000003:ziel}
\end{equation}
ist?
Diese Aufgabe illustriert, wie die erste Ableitung in
\eqref{60000003:vorgabe} mit Hilfe eines Exponentialfaktors $e^{kx}$
wegtransformiert werden kann.

\begin{loesung}
Wir berechnen die Ableitungen von $u$:
\begin{align*}
u'(x)&=y'(x)e^{kx}+y(x)ke^{kx}\\
u''(x)&=y''(x)e^{kx}+2y'(x)ke^{kx}+k^2y(x)e^{kx}
\end{align*}
Die Differentialgleichung  \eqref{60000003:ziel} wird damit
zu
\[
0
=
u''(x)+\alpha u(x)
=
y''(x)e^{kx}+2y'(x)ke^{kx}+k^2y(x)e^{kx}
+ \alpha y(x)e^{kx}
=
\bigl(
y''+2k y' +(k^2+\alpha)y
\bigr)
e^{kx}
\]
Diese Differentialgleichung ist erf"ullt wenn $2k=\beta$ und
$k^2+\alpha=\gamma$, also muss man w"ahlen
\[
\begin{aligned}
k&=\frac{\beta}2
&
&\text{und}&
\alpha=\gamma-k^2=\gamma-\frac{\beta^2}4.
\end{aligned}
\]
Diese L"osung zeigt, dass der Term erster Ordnung in einer gew"ohnlichen
Differentialgleichung wegtransformiert werden kann.
\end{loesung}

