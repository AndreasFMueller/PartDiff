Welche der folgenden partiellen Differentialoperatoren ist
elliptisch, hyperbolisch, parabolisch?
\begin{teilaufgaben}
\item \[
\frac{\partial^2 u}{\partial x^2}
+2\frac{\partial u}{\partial x}\frac{\partial u}{\partial y}
+\frac{\partial^2u}{\partial y^2}
\]
\item \[
\frac{\partial^2 u}{\partial x^2}
+\frac{\partial^2 u}{\partial x\partial z}
+\frac{\partial^2 u}{\partial z^2}
\]
\item \[
2\frac{\partial^2u}{\partial^2x}
+2\frac{\partial^2u}{\partial^2y}
+2\frac{\partial^2u}{\partial^2z}
-2\frac{\partial^2u}{\partial x\partial y}
-2\frac{\partial^2u}{\partial y\partial z}
\]
\item Ein partieller Differentialoperator in den Variablen $x$ und $y$,
so dass die Koeffizienten von $\partial_x^2$ und $\partial_y^2$
verschiedenes Vorzeichen haben.
\item \[
2\frac{\partial^2u}{\partial x\partial z}
+\frac{\partial^2u}{\partial x^2}
+2\frac{\partial^2u}{\partial x\partial y}
+\frac{\partial^2u}{\partial z^2}
+2\frac{\partial^2u}{\partial y\partial z}
+\frac{\partial^2u}{\partial y^2}
\]
\item
\[
\frac{\partial^2u}{\partial x\partial y}
+\frac{\partial^2u}{\partial y\partial z}
+\frac{\partial^2u}{\partial z\partial x}
\]
\end{teilaufgaben}

\begin{loesung}
\begin{teilaufgaben}
\item Dies ist kein linearer Differentialoperator, er kann also nicht
wie gewünscht klassifiziert werden.
\item Die Symbolmatrix ist
\[
\begin{pmatrix}
1&0&\frac12\\
0&0&0\\
\frac12&0&1
\end{pmatrix}
\]
Das charakteristische Polynom ist die Determinante von
\[
\left|\,\begin{matrix}
1-\lambda&0&\frac12\\
0&-\lambda&0\\
\frac12&0&1-\lambda
\end{matrix}
\,\right|,
\]
die man nach der zweiten Zeile entwickeln kann:
\[
-\lambda\cdot\left|\,
\begin{matrix}1-\lambda&\frac12\\\frac12&1-\lambda\end{matrix}
\,\right|
=
-\lambda\biggl((1-\lambda)^2-\frac14\biggr)
=
-\lambda\left(\lambda^2-2\lambda+\frac34\right)
=
-\lambda(\lambda-\frac12)(\lambda-\frac32).
\]
Die Matrix hat also zwei positive Eigenwerte und den Eigenwert $0$,
der Operator ist also parabolisch.
\item Die Symbolmatrix ist
\[
\begin{pmatrix}
2&-1&0\\
-1&2&-1\\
0&-1&2
\end{pmatrix}
\]
mit dem charakteristischen Polynom
\begin{align*}
\left|\,
\begin{matrix}
2-\lambda&-1&0\\
-1&2-\lambda&-1\\
0&-1&2-\lambda
\end{matrix}\,\right|
&=(2-\lambda^3)+2(2-\lambda)
\\
&=(2-\lambda)((2-\lambda)^2+2)
\\
&=(2-\lambda)(\lambda^2-4\lambda+6).
\end{align*}
Am konstanten Term im letzten Faktor kann man ablesen, dass
seine zwei Nullstellen das gleiche Vorzeichen haben. Da der
Koeffizient von $\lambda$ negativ ist, kann man ablesen, dass
beide positiv sind. Da auch der erste Faktor eine positive
Nullstelle hat, sind alle Nullstellen positiv, der Operator ist
also elliptisch.
\item Die Symbolmatrix eines solchen Differentialoperators hat die Form
\[
\begin{pmatrix}a&b\\b&c\end{pmatrix}
\]
mit Determinante $ac-b^2$. Da $a$ und $c$ verschiedene Vorzeichen
haben, ist $ac< 0$. Da $b^2\ge 0$, ist $ac-b^2<0$, also haben
die beiden Eigenwerte verschiedene Vorzeichen, der Differentialoperator
ist hyperbolisch.
\item Die Symbolmatrix ist
\[
\begin{pmatrix}1&1&1\\1&1&1\\1&1&1\end{pmatrix}
\]
Sie hat den Rang 1, also hat sie einen doppelten Eigenwert $0$,
und fällt damit aus dem Klassifikationsschema. Die parabolischen
Operatoren, die als einzige verschwindende Eigenwerte haben dürfen,
haben höchstens einen.
\item Die Symbolmatrix dieses Differentialoperators ist
\[
\begin{pmatrix}
0&1&1\\
1&0&1\\
1&1&0\\
\end{pmatrix}
\]
Das charakteristische Polynom ist
\[
\left|\,
\begin{matrix}
-\lambda&1&1\\
1&-\lambda&1\\
1&1&-\lambda\\
\end{matrix}
\,\right|
=-\lambda^3+3\lambda+2
\]
Für $\lambda=2$ nimmt es den Wert $2$ an, also müssen alle Eigenwerte
von $0$ verschieden sein. Die Spur der Matrix ist $0$, die Summe der
Eigenwerte also auch. Daher können nicht alle Eigenwerte das gleiche
Vorzeichen haben. Da die Determinante positiv ist, müssen zwei Eigenwerte
negativ sein und einer positiv. Folglich ist der Operator hyperbolisch.
\end{teilaufgaben}
\end{loesung}
