Consider the partial differential equation
\begin{equation}
\frac{\partial u}{\partial x}
+
\frac{1}{\tan y}
\frac{\partial u}{\partial y}
=
u
\label{30000025:eqn}
\end{equation}
on the domain
\[
\Omega
=
\biggl\{
(x,y)
\;\bigg|\;
0<y<\frac{\pi}2
\biggr\}
\]
with boundary condition
\[
u(x_0,0) = f(x_0)
\]
for some boundary function $f(x_0)$.
\begin{teilaufgaben}
\item
Are the boundary values provided sufficient to uniquely determine the
solution of the partial differential equation.
\item
Find a solution $u(x,y)$ of \eqref{30000025:eqn} with the boundary function
$f(x_0)=\cos x_0$.
\end{teilaufgaben}

\begin{hinweis}
Use $\int\tan y\,dy = -\log\cos y+C$
\end{hinweis}

\ifthenelse{\boolean{loesungen}}{
\begin{figure}
\centering
\includeagraphics[]{graph.pdf}
\caption{Characteristics of the equation~\eqref{30000025:eqn} in
problem~\ref{30000025}
\label{30000025:char}}
\end{figure}
\begin{figure}
\centering
\includeagraphics[]{solution.pdf}
\caption{Solution surface for the differential equation of
problem~\ref{30000025}
\label{30000025:solution}}
\end{figure}
}{}

\begin{loesung}
\eqref{30000025:eqn} is a quasilinear partial differential equation
which can be solved using the method of characteristics.
The ordinary differential equations for the characteristics are
\[
\begin{aligned}
\dot{x} &= 1
&&
&&
&&\Rightarrow&
x&=t+D,
\\
\dot{y} &= \frac{1}{\tan y}
&&\Rightarrow&
\int \tan y\,dy &= \int dt + C
&&\Rightarrow&
-\log\cos y &= t + C,
\\
\dot{u} &= u
&&
&&
&&\Rightarrow&
u&=u_0e^t.
\end{aligned}
\]
The characteristic curves should start on the $x$-axis for $t=0$,
i.~e. the second line implies
\[
0 + C
=
-\log\cos 0
=
-\log 1
=
0
\qquad\Rightarrow\qquad
-\log\cos y = t.
\]
As the characteristic is supposed to start in the point $(x_0,0)$,
the first line implies that $D=x_0$ or $x=t+x_0$.
By substituting $t$ in the second equation we get
\begin{equation}
x = x_0 - \log\cos y.
\label{30000025:xy}
\end{equation}

\begin{teilaufgaben}
\item
From \eqref{30000025:xy} or from the figure~\ref{30000025:char} we
see that the characteristics cover all of the domain, which means
that the boundary conditions uniquely determine the solution to the
differential equation.
\item
To find the solution, we substitute $x_0=x+\log\cos y$ and $u_0=\cos x_0$ 
into the $u$-equation above and get
\[
u(x,y)
=
u_0 e^t
=
\cos x_0\cdot
e^{-\log\cos y}
=
\frac{\cos (x+\log\cos y)}{\cos y}.
\]
To verify the solution, we compute the derivatives
\begin{align*}
\frac{\partial u}{\partial x}
&=
-\frac{\sin(x+\log\cos y)}{\cos y}
\\
&=
\frac{\sin y \sin(x+\log\cos y)}{\cos^2 y}
+
\frac{\sin y \cos(x+\log\cos y)}{\cos^2 y}
\\
%(%i4) diff(u,y)
%           sin(y) sin(log(cos(y)) + x)   sin(y) cos(log(cos(y)) + x)
%(%o4)      --------------------------- + ---------------------------
%                        2                             2
%                     cos (y)                       cos (y)
\frac{\partial u}{\partial y}
&=
\tan y
\biggl(
\frac{\sin(x+\log\cos y)}{\cos y}
+
\frac{\cos(x+\log\cos y)}{\cos y}
\biggr)
\\
&=
\tan y
\biggl(
\frac{\sin(x+\log\cos y)}{\cos y}
+
u
\biggr)
\intertext{implying}
\frac{\partial u}{\partial x}
+
\frac{1}{\tan y}
\frac{\partial u}{\partial y}
&=
-\frac{\sin(x+\log\cos y)}{\cos y}
+
\frac{\sin(x+\log\cos y)}{\cos y}
+
u
=
u,
\end{align*}
so the differential equation holds.

For the boundary conditions, we substitute $x_0$ in $u$ and compute
\[
u(x_0,0) = \frac{\cos(x_0+\log\cos 0)}{\cos 0} = \cos x_0,
\]
so the boundary conditions are satisfied as well.
\qedhere
\end{teilaufgaben}
\end{loesung}

\begin{diskussion}
It may be noteworthy that the coefficient of $u_y$ in the differential
equation diverges as $y$ goes to $0$.
This implies that it does not make sense to try to evaluate the
differential equation in points of the boundary, but by definition
the equation only has to hold in the interior of $\Omega$.

Another consequence is that for the equation to hold, the $y$-derivative
of $u$ has to tend to $0$ at least as fast as $\tan y$ so that the
second term of the differential equation remains bounded.
This becomes evident in figure~\ref{30000025:solution}, which shows that
the solution surface leaves the initial curve horizontally in the
$y$-direction.
\end{diskussion}

\begin{bewertung}
Differential equation for characteristics ({\bf D}) 1 point,
solution to the $Y$-equation ({\bf Y}) 1 point,
determination of the constant $C$ ({\bf C}) 1 point,
solution to the other two equations ({\bf X}) 1 point,
elimination of $x_0$ and $t$ ({\bf E}) 1 point,
solution of the equation ({\bf S}) 1 point.
\end{bewertung}
