%
% domain.tex -- Template für standalone TIKZ Bilder
%
% (c) 2019 Prof Dr Andreas Müller, Hochschule Rapperswil
%
\documentclass[tikz,12pt]{standalone}
\usepackage{amsmath}
\usepackage{times}
\usepackage{txfonts}
\usepackage{pgfplots}
\usepackage{csvsimple}
\usepackage{ifthen}
\usetikzlibrary{arrows,intersections,math,calc}
\begin{document}
\def\skala{2}
\newboolean{fein}
\setboolean{fein}{true}
\begin{tikzpicture}[>=latex,thick,scale=\skala]

\fill[color=blue!10] (0,-2.1) rectangle ({3.14159/2},2.1);

\definecolor{darkgreen}{rgb}{0,0.6,0}

\draw[->] (0,-2.1) -- (0,2.2) coordinate[label={left:$y$}];
\draw[->] ({-0.1/\skala},0) -- (1.8,0) coordinate[label={$x$}];

\def\xmin{0.05}

\begin{scope}
	\clip (0,-2.1) rectangle (1.6,2.1);

	\foreach \ynull in {-2,-1,...,5}{
		\draw[color=red]
			plot[domain=1:\xmin,samples=100]
				({3.14159*asin(\x)/180},{ln(\x)+\ynull});
	}

\ifthenelse{\boolean{fein}}{
	\foreach \ynull in {-1.8,-1.6,...,4.8}{
		\draw[color=red,line width=0.1pt]
			plot[domain=1:\xmin,samples=100]
				({3.14159*asin(\x)/180},{ln(\x)+\ynull});
	}
}{}

\end{scope}

\draw ({3.14159/2},{-0.1/\skala}) -- ({3.14159/2},{0.1/\skala});
\node at ({3.14159/2},0) [below right] {$\displaystyle\frac{\pi}2$};

\foreach \y in {-2,-1,0,1,2}{
	\draw ({-0.1/\skala},\y) -- ({0.1/\skala},\y);
	\node at (0,\y) [left] {$\y$};
}

\draw[color=darkgreen] ({3.14159/2},-2.1) -- ({3.14159/2},2.1);

\end{tikzpicture}
\end{document}

