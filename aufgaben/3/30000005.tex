L"osen sie die Differentialgleichungen
\[
\sqrt{x}\partial_x u+\sqrt{y}\partial_yu=1
\]
im Gebiet $\Omega=\{(x,y)\,|\, x,y>0\}$
mit der Anfangsbedingungen
\begin{align*}
u(x,0)&=\sin x\\
u(0,y)&=-\sin y
\end{align*}

\begin{loesung}
Es liegt eine quasilineare Differentialgleichung vor, die wir mit
Charakteristiken l"osen k"onnen. Dazu m"ussen die gew"ohnlichen
Differentialgleichungen
\begin{align*}
\dot x &=\sqrt{x} \\
\dot y &=\sqrt{y} \\
\dot u &=1
\end{align*}
gel"ost werden. Die letzte ist die einfachste, $u(t)=u_0 + t$. Dabei
ist $u_0$ der Wert, der am Anfangspunkt der Charakteristik, also
im Punkt $(x_0,0)$ angenommen wird, $u_0=\sin x_0$.

Die Differentialgleichung $\dot x=\sqrt{x}$ kann mit Separation
gel"ost werden:
\begin{align*}
\frac{dx}{dt}&=\sqrt{x}\\
\int\frac{dx}{\sqrt{x}}&=\int\,dt\\
2\sqrt{x}&=t+C_1
\end{align*}
und analog
\[
2\sqrt{y}=t+C_2
\]
Die Konstanten $C_1$ und $C_2$ w"ahlen eine Charakteristik aus,
die bei $(C_1,C_2)$ beginnt.
F"ur die L"osung der partiellen Differentialgleichung brauchen wir
jetzt Charakterstiken, die von $(0,y_0)$  ausgehen und solche, die
von $(x_0,0)$ ausgehen.

Im ersten Fall bekommen wir
\begin{align*}
2\sqrt{x}&=t\\
2\sqrt{y}&=t+2\sqrt{y_0}\\
u&=-\sin y_0 + t
\end{align*}
und m"ussen zu einem Punkt $(x,y)$ das zugeh"orige $t$ und $y_0$
finden.
Aus $t=2\sqrt{x}$ findet man $\sqrt{y_0}=\sqrt{y}-\sqrt{x}$, dies
ist nat"urlich nur m"oglich, wenn $y>x$. F"ur $u$ findet man dann
\[
u(x,y)=-\sin y_0+2\sqrt{x}=-\sin(\sqrt{y}-\sqrt{x})^2 +2\sqrt{x}.
\]
Im zweiten Fall bekommen wir
\begin{align*}
2\sqrt{x}&=t+2\sqrt{x_0}\\
2\sqrt{y}&=t\\
u&=\sin x_0+t
\end{align*}
und analog zum ersten Fall bekommt man $\sqrt{x_0}=\sqrt{x}-\sqrt{y}$
und
\[
u(x,y)=\sin(\sqrt{x}-\sqrt{y})^2+2\sqrt{y}
\]
wobei diese Formel nur f"ur $y\le x$ gilt. Insgesamt bekommt man also
\[
u(x,y)=
\begin{cases}
-\sin(\sqrt{y}-\sqrt{x})^2 +
2\sqrt{x}&\qquad y \ge x\\
\sin(\sqrt{y}-\sqrt{x})^2 +
2\sqrt{y}&\qquad x \ge y
\end{cases}
\]
Insbesondere stellt man fest, dass die L"osung auf der Winkelhalbierenden
nicht differenzierbar sein kann.

Kontrolle:
\begin{align*}
\frac{\partial u}{\partial x}
&=
+\begin{cases}
2\cos(\sqrt{y}-\sqrt{x})^2(\sqrt{y}-\sqrt{x})\frac1{2\sqrt{x}}
+\frac1{\sqrt{x}}&\qquad y> x\\
-2\cos(\sqrt{y}-\sqrt{x})^2(\sqrt{y}-\sqrt{x})\frac1{2\sqrt{x}}
&\qquad x> y
\end{cases}
\\
\frac{\partial u}{\partial x}
&=
+\begin{cases}
-2\cos(\sqrt{y}-\sqrt{x})^2(\sqrt{y}-\sqrt{x})\frac1{2\sqrt{y}}
&\qquad y> x\\
2\cos(\sqrt{y}-\sqrt{x})^2(\sqrt{y}-\sqrt{x})\frac1{2\sqrt{y}}
+\frac1{\sqrt{y}}&\qquad x> y\\
\end{cases}
\\
\sqrt{x}
\frac{\partial u}{\partial x}
+
\sqrt{y}
\frac{\partial u}{\partial x}
&=
\begin{cases}
\sqrt{x}\frac1{\sqrt{x}}=1&\qquad y> x\\
\sqrt{y}\frac1{\sqrt{y}}=1&\qquad x> y\\
\end{cases}
\end{align*}
Die Differentialgleichung ist also "uberall erf"ullt. Die L"osung ist
zwar nicht differentierbar entlang der
Winkelhalbierenden. Der Graph von $u$ ist aber trotzdem eine
glatte Fl"ache.
\end{loesung}
