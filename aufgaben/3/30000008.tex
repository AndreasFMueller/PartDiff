Auf dem Gebiet $\Omega = \{ (x,y)\,|\, x^2+y^2 < 1\}$ ist die
partielle Differentialgleichung 
\begin{equation}
x\frac{\partial u}{\partial x}+x^2\frac{\partial u}{\partial y}=0
\label{30000008:dgl}
\end{equation}
gegeben. Uns interessiert aber nicht die Lösung (daher sind auch keine
Randbedingungen vorgegeben), sondern die Frage der eindeutigen Lösbarkeit.
\begin{teilaufgaben}
\item
Ist die Differentialgleichung eindeutig lösbar, wenn man Randwerte für
alle Randpunkte $(x_0,y_0)$ des Gebietes vorgibt, für die $x_0<0$ ist?
\item
Ist die Differentialgleichung eindeutig lösbar, wenn man Randwerte für
alle Randpunkte $(x_0,y_0)$ des Gebietes vorgibt, für die $y_0<0$ ist?
\end{teilaufgaben}

\begin{loesung}
Dies ist eine quasilineare partielle Differentialgleichung, die mit dem
Verfahren der Charakteristiken gelöst werden kann.
Das Gebiet ist die Einheitskreisscheibe, Randwerte sind in a) vorgegeben
auf der linken Hälfte des Einheitskreises, in b) auf der unteren Hälfte.

Ob die Gleichung durch Vorgabe der Randwerte auf dem linken oder unteren Rand
eindeutig gelöst werden kann, kann genau dann bejaht werden, wenn die 
von den vorgegebenen Randpunkten
ausgehenden Charakteristiken das ganze Gebiet $\Omega$ überdecken.

Die Differentialgleichungen für die Charakteristiken durch den Punkt
$(x_0,y_0,u_0)$ und ihre Lösungen sind
\begin{align*}
\dot x&=x&&\Rightarrow&x(t)&=x_0e^t,\\
\dot y&=x^2&&\Rightarrow&y(t)&=\frac12x^2+y_0-\frac12x_0^2,\\
\dot u&=0&&\Rightarrow&u(t)&=u_0.
\end{align*}
Die zweite Differentialgleichung wird gelöst, indem man auf der rechten
Seite die Lösung der ersten Gleichung einsetzt:
\[
\dot y=x_0^2e^{2t}
\quad\Rightarrow\quad
y(t)
=y_0 + \int_0^tx_0^2e^{2\tau}\,d\tau
=y_0 + \left[\frac12x_0^2e^{2\tau}\right]_0^t
=y_0 + \frac12x_0^2e^{2t}-\frac12x_0^2.
=y_0 + \frac12x^2-\frac12x_0^2.
\]
Die Charakteristiken sind also Parabeln mit Scheitelpunkt auf der
$y$-Achse.
\begin{figure}
\begin{center}
\includeagraphics[width=0.6\hsize]{domain-1.pdf}
\end{center}
\caption{Charakteristiken (rot) im Gebiet $\Omega$ der Differentialgleichung
\eqref{30000008:dgl}.\label{30000008:domain}}
\end{figure}
Die Abbildung~\ref{30000008:domain} zeigt die Lösungskurven.

Die Charakteristiken schneiden den rechten und linken Rand mindestens ein mal.
Insbesondere kann in b) also die Lösung nicht eindeutig bestimmt sein, denn
die obere Hälfte des Einheitskreises wird überdeckt von
Charakteristiken, die den unteren Rand des Kreises nicht schneiden,
ohne Vorgabe von Randwerten auf dem oberen Rand des Kreises ist die
Lösung der Differentialgleichung dort nicht bestimmt.

Für a) könnte einzig beim Punkt $(0,-1)$ ein Problem auftreten.
Die Charakteristik durch den
Punkt $(0,-1)$ darf den linken Rand nicht schneiden, da sonst die 
nahe bei $(0,-1)$ vorgegebenen Randwerte anderen Randwerten widersprechen
könnten.

Die Parabel durch den Punkt $(0,-1)$ hat die Gleichung $y=-1+\frac12x^2$.
Es muss also untersucht werden, ob für $x\ne 0$
\[
-\sqrt{1-x^2} > -1+\frac12x^2
\]
ist. Das Quadrat der rechten Seite ist
\[
(-1+\frac12x^2)^2=1-x^2 +\frac14x^4 > 1-x^2=\bigl(-\sqrt{1-x^2}\bigr)^2,
\]
(Achtung: beim Quadrieren einer Ungleichung von negativen Grössen ändert das
Ungleichheitszeichen)
insbesondere kann die Charakterstik durch $(0,-1)$ den Einheitskreis
in keinem anderen Punkt schneiden.
Die Randwerte auf dem linken Rand des Einheitskreises
bestimmen die Lösung der Differentialgleichung eindeutig.
\end{loesung}

\begin{bewertung}
Quasilineare PDGL ({\bf Q}) 1 Punkt,
Differentialgleichungen der Charakteristiken ({\bf D}) 1 Punkt,
Lösung der Differentialgleichungen für $x(t)$ ({\bf X})
und $y(t)$ ({\bf Y}) je 1 Punkt,
Verwendung der Charakteristikenmethode zur Beurteilung, ob die
vorgegebenen Randwerte ausreichend sind in a) ({\bf A}) 1 Punkt,
Begründung, warum die Randwerte in b) nicht ausreichen ({\bf B}) 1 Punkt.
\end{bewertung}
