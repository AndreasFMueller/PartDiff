Auf dem Gebiet $\Omega = \{ (x,y)\,|\, x^2+y^2 < 1\}$ ist die
partielle Differentialgleichung 
\begin{equation}
x\frac{\partial u}{\partial x}+x^2\frac{\partial u}{\partial y}=0
\label{30000008:dgl}
\end{equation}
gegeben. Uns interessiert aber nicht die L"osung (daher sind auch keine
Randbedingungen vorgegeben), sondern die Frage der eindeutigen L"osbarkeit.
\begin{teilaufgaben}
\item
Ist die Differentialgleichung eindeutig l"osbar, wenn man Randwerte f"ur
alle Randpunkte $(x_0,y_0)$ des Gebietes vorgibt, f"ur die $x_0<0$ ist?
\item
Ist die Differentialgleichung eindeutig l"osbar, wenn man Randwerte f"ur
alle Randpunkte $(x_0,y_0)$ des Gebietes vorgibt, f"ur die $y_0<0$ ist?
\end{teilaufgaben}

\begin{loesung}
Dies ist eine quasilineare partielle Differentialgleichung, die mit dem
Verfahren der Charakteristiken gel"ost werden kann.
Das Gebiet ist die Einheitskreisscheibe, Randwerte sind in a) vorgegeben
auf der linken H"alfte des Einheitskreises, in b) auf der unteren H"alfte.

Ob die Gleichung durch Vorgabe der Randwerte auf dem linken oder unteren Rand
eindeutig gel"ost werden kann, kann genau dann bejaht werden, wenn die 
von den vorgegebenen Randpunkten
ausgehenden Charakteristiken das ganze Gebiet $\Omega$ "uberdecken.

Die Differentialgleichungen f"ur die Charakteristiken durch den Punkt
$(x_0,y_0,u_0)$ und ihre L"osungen sind
\begin{align*}
\dot x&=x&&\Rightarrow&x(t)&=x_0e^t,\\
\dot y&=x^2&&\Rightarrow&y(t)&=\frac12x^2+y_0-\frac12x_0^2,\\
\dot u&=0&&\Rightarrow&u(t)&=u_0.
\end{align*}
Die zweite Differentialgleichung wird gel"ost, indem man auf der rechten
Seite die L"osung der ersten Gleichung einsetzt:
\[
\dot y=x_0^2e^{2t}
\quad\Rightarrow\quad
y(t)
=y_0 + \int_0^tx_0^2e^{2\tau}\,d\tau
=y_0 + \left[\frac12x_0^2e^{2\tau}\right]_0^t
=y_0 + \frac12x_0^2e^{2t}-\frac12x_0^2.
=y_0 + \frac12x^2-\frac12x_0^2.
\]
Die Charakteristiken sind also Parabeln mit Scheitelpunkt auf der
$y$-Achse.
\begin{figure}
\begin{center}
\includeagraphics[width=0.6\hsize]{domain-1.pdf}
\end{center}
\caption{Charakteristiken (rot) im Gebiet $\Omega$ der Differentialgleichung
(\ref{30000008:dgl}).\label{30000008:domain}}
\end{figure}
Die Abbildung~\ref{30000008:domain} zeigt die L"osungskurven.

Die Charakteristiken schneiden den rechten und linken Rand mindestens ein mal.
Insbesondere kann in b) also die L"osung nicht eindeutig bestimmt sein, denn
die obere H"alfte des Einheitskreises wird "uberdeckt von
Charakteristiken, die den unteren Rand des Kreises nicht schneiden,
ohne Vorgabe von Randwerten auf dem oberen Rand des Kreises ist die
L"osung der Differentialgleichung dort nicht bestimmt.

F"ur a) k"onnte einzig beim Punkt $(0,-1)$ ein Problem auftreten.
Die Charakteristik durch den
Punkt $(0,-1)$ darf den linken Rand nicht schneiden, da sonst die 
nahe bei $(0,-1)$ vorgegebenen Randwerte anderen Randwerten widersprechen
k"onnten.

Die Parabel durch den Punkt $(0,-1)$ hat die Gleichung $y=-1+\frac12x^2$.
Es muss also untersucht werden, ob f"ur $x\ne 0$
\[
-\sqrt{1-x^2} > -1+\frac12x^2
\]
ist. Das Quadrat der rechten Seite ist
\[
(-1+\frac12x^2)^2=1-x^2 +\frac14x^4 > 1-x^2=\bigl(-\sqrt{1-x^2}\bigr)^2,
\]
(Achtung: beim Quadrieren einer Ungleichung von negativen Gr"ossen "andert das
Ungleichheitszeichen)
insbesondere kann die Charakterstik durch $(0,-1)$ den Einheitskreis
in keinem anderen Punkt schneiden.
Die Randwerte auf dem linken Rand des Einheitskreises
bestimmen die L"osung der Differentialgleichung eindeutig.
\end{loesung}

\begin{bewertung}
Quasilineare PDGL ({\bf Q}) 1 Punkt,
Differentialgleichungen der Charakteristiken ({\bf D}) 1 Punkt,
L"osung der Differentialgleichungen f"ur $x(t)$ ({\bf X})
und $y(t)$ ({\bf Y}) je 1 Punkt,
Verwendung der Charakteristikenmethode zur Beurteilung, ob die
vorgegebenen Randwerte ausreichend sind in a) ({\bf A}) 1 Punkt,
Begr"undung, warum die Randwerte in b) nicht ausreichen ({\bf B}) 1 Punkt.
\end{bewertung}
