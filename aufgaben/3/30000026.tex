On the domain
\(
\Omega = \{(x,y) \mid x>1\},
\)
the partial differential equation
\begin{equation}
x
\frac{\partial u}{\partial x}
+
u
\frac{\partial u}{\partial y}
=
y
\label{30000026:pde}
\end{equation}
is given with homogeneous boundary conditions
\[
u(1,y) = 0, \qquad y\in \mathbb{R}.
\]
Find the solution $u(x,y)$.

\begin{hinweis}
It may be useful to note that the system of ordinary differential
equations
\[
\left.
\begin{aligned}
\dot{X} &= Y\\
\dot{Y} &= X
\end{aligned}
\right\}
\qquad\text{has solutions}\qquad
\left\{
\begin{aligned}
X &= A\cosh t + B\sinh t\\
Y &= A\sinh t + B\cosh t.
\end{aligned}
\right.
\]
\end{hinweis}

\begin{loesung}
\begin{figure}
\centering
\includeagraphics[]{solution.pdf}
\caption{Solution surface of the partial differential equation
\eqref{30000026:pde}.
The characteristic curves are drawn in red, the Cauchy initial curve
in green.
The projection of the characteristic curves into the $x$-$y$-plane is
shown in light red.
\label{30000026:fig}}
\end{figure}
This is a quasilinear partial differential equation of first order
which can be solved using the method of characteristics.
The differential equation for the characteristics is
\begin{equation}
\frac{d}{dt}
\begin{pmatrix*}
x\\
y\\
u
\end{pmatrix*}
=
\begin{pmatrix*}
x\\
u\\
y
\end{pmatrix*}
\qquad\Rightarrow\qquad
\dot{x} = x \qquad\Rightarrow\qquad x(t) = x_0e^t.
\end{equation}
The boundary is specified by $x=1$, which means that in order for the 
characteristic curve to start in point $(1,y_0)$, then we must have
$1=x_0$.

The two functions $u(t)$ und $y(t)$ satisfy the equations
\begin{equation}
\left.
\begin{aligned}
\dot{y} &= u \\
\dot{u} &= y
\end{aligned}
\;
\right\}
\qquad\Rightarrow\qquad
\left\{\;
\begin{aligned}
\ddot{y}=y\\
\ddot{u}=u
\end{aligned}
\right.
\label{30000026:chardgl}
\end{equation}
These equations have linear combinations of $\sinh t$ und $\cosh t$
as solutions.
Let
\[
u(t) = A\cosh t + B\sinh t
\]
then the boundary condition $u(0)=0$ means that 
\[
0 = u(0) = A\cosh 0 + B\sinh 0 = A
\qquad\Rightarrow\qquad
u(t) = B\sinh t.
\]
From the equation $\dot{u}=y$ we can derive
\[
y(t) = A\sinh t + B\cosh t = B\cosh t.
\]
On the other hand, $B$ has to be chosen so that the characteristic
curve starts at $y_0 = y(0) = B\cosh 0 = B$.
This means that we now have a complete solution 
\begin{align*}
x(t) &= e^t        &&\Rightarrow&    t &= \log x\bigg.\\
y(t) &= y_0\cosh t &&\Rightarrow&  y_0 &= \frac{y}{\cosh\log x}\\
u(t) &= y_0\sinh t &&\Rightarrow&    u &= \frac{y}{\cosh\log x}\sinh\log x
= y\tanh\log x.
\end{align*}
We thus have found the solution
\begin{equation}
u(x,y) = y\tanh\log x
\label{30000026:eqn:sol}
\end{equation}
of the partial differential equation~\eqref{30000026:pde}.

The solution \eqref{30000026:eqn:sol} can be further simplified
by observing that
\[
\tanh t
=
\frac{e^t-e^{-t}}{e^t+e^{-t}}
= 
\frac{x-x^{-1}}{x+x^{-1}}
=
\frac{x^2-1}{x^2+1}.
\]
Thus the solution gets the even easier form
\[
u(x,y) = y\frac{x^2-1}{x^2+1}.
\]

We can verify this solution by computing the partial derivatives
\begin{align*}
\frac{\partial u}{\partial x}
&=
y\operatorname{sech}^2(\log x)\cdot \frac{1}{x}
\\
\frac{\partial u}{\partial y}
&=
\tanh\log x
\end{align*}
and substituting them in the partial differential equation
\begin{align*}
x\frac{\partial u}{\partial x}
+
u\frac{\partial u}{\partial y}
&=
y\operatorname{sech}^2(\log x)
+
y\tanh^2(\log x)
\\
&=
y\biggl(
\frac{1}{\cosh^2\log x}
+
\frac{\sinh^2\log x}{\cosh^2\log x}
\biggr)
\\
&=
y\frac{1+\sinh^2\log x}{\cosh^2\log x}
\\
&=
y\frac{\cosh^2\log x}{\cosh^2\log x}
=
y
\end{align*}
For the boundary condition we substitute $x=1$ und $y=y_0$ into
$u$ and obtain
\[
u(1,y_0) = y_0\tanh\log 1 = y_0\tanh 0 = 0,
\]
so the boundary condition is satisfied as well.
For the second form of the solution, the verification is even easier.
\end{loesung}

\begin{bewertung}
Differential equations for characteristics ({\bf C}) 1 point,
solution for $x(t)$ including $x_0$ from boundary condition $x(0)=1$
({\bf X}) 1 point,
general solution for $y(t)$ and $u(t)$ using the hint ({\bf H}) 1 point,
using the boundary condition $u(0)=0$ to determining $A=0$ ({\bf A}) 1 point,
using the boundary condition $y(0)=y_0$ to determine $B=y_0$ ({\bf B}) 1 point,
eliminating $t$ and $y_0$ to get $u(x,y)$ ({\bf U}) 1 point.
\end{bewertung}
