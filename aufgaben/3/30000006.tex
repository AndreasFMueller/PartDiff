Betrachten Sie die partielle Differentialgleichung
\[
\frac{\partial u}{\partial x}
+x
\frac{\partial u}{\partial y}
=
\sin u
\]
auf dem Einheitsquadrat $\Omega=(0,1)^2=\{(x,y)\,|\, 0<x,y<1\}$.
Welche der folgenden Randwertvorgaben f"uhrt zu einem gut gestellten
Problem?
\begin{teilaufgaben}
\item
Randwerte auf $\{(x,0)\,|\,x\in [0,1]$ und $\{(0,y)\,|\,y\in[0,1]\}$
\item
Randwerte auf $\{(x,1)\,|\,x\in [0,1]$ und $\{(1,y)\,|\,y\in[0,1]\}$
\item
Randwerte auf $\partial\Omega$
\item
Randwerte auf $\{(x,1)\,|\,x\in [0,1]$ und $\{(0,y)\,|\,y\in[0,1]\}$
\item
Randwerte auf
$\{(x,0)\,|\,x\in [0,1]\}$,
$\{(x,1)\,|\,x\in [0,1]\}$ und 
$\{(0,y)\,|\,y\in[0,\frac12)\}$
\end{teilaufgaben}

\begin{loesung}
Die Gleichung ist nichtlinear, aber quasilinear.
Die Charakteristiken geben dar"uber Auskunft, wo Randwerte vorgegeben
werden m"ussen.
Die Gleichung der Charakteristiken ist 
\begin{equation}
\frac{d}{dt}\begin{pmatrix}x\\y\\u\end{pmatrix}
=
\begin{pmatrix}
1\\x\\\sin u
\end{pmatrix}
\qquad
\Rightarrow
\qquad
\begin{aligned}
\dot x &= 1\\
\dot y &= x\\
\dot u &=\sin u
\end{aligned}
\end{equation}
Die erste Gleichung hat als L"osung
\[
x(t) = t+x_0.
\]
Daraus bekommt man als L"osung der zweiten Gleichung
\[
y(t) = \frac12 t^2+x_0t+y_0.
\]
Mit $t=x-x_0$ kann man die Variable $t$ eliminieren, und
bekommt
\[
y=\frac12(x-x_0)^2+x_0(x-x_0)+y_0=\frac12x^2 + y_0-\frac12x_0^2.
\]
Die Projektionen der Charakteristiken in die $x$-$y$-Ebene sind also
Parabeln der Form $y=\frac12x^2+c$.

\begin{figure}
\begin{center}
\includeagraphics[]{domain-1.pdf}
\end{center}
\caption{Projektion der Charakteristiken\label{30000006:char}}
\end{figure}
Die Randbedingungen m"ussen also auf einer Teilmenge $R\subset \partial\Omega$
vorgegeben werden so, dass die Projektionen der Charakterstiken, die das
Gebiet ber"uhren, mit $R$ genau einen Punkt gemeinsam haben (Abbildung~\ref{30000006:char}). 
\begin{teilaufgaben}
\item Der linke und untere Rand schneidet jede Parabel genau einmal. 
Jede Parabel, die das Gebiet ber"uhrt, trifft den linken oder unteren
Rand. Damit ist das Problem gut gestellt.
\item Gut gestellt, analog zu a)
\item Jede Parabel schneidet den Rand zweimal, die Randwerte k"onnen nicht
beliebig vorgegeben werden, da sie sich widersprechen k"onnen.
\item Die Parabeln unterhalb von $y=\frac12x^2$ schneiden den Rand
nicht, im Teilgebiet $\{(x,y)\,|\, y<\frac12x^2\}$ ist die L"osung also
nicht eindeutig bestimmt.
\item Die Bedinung ist erf"ullt, allerdings ist nicht sichergestellt,
dass die Teile der L"osung, die durch die Randwerte auf dem oberen 
Rand bestimmt werden, mit der L"osung zusammenpassen, die durch
die Randwerte auf dem unteren Rand und dem Interval $[0,1)$ auf
der $y$-Achse betimmt ist. Dazu muss zus"atzlich der Grenzwert
der Randwerte im Punkt $(0,\frac12)$ mit dem Randwert in $(1,1)$
"ubereinstimmen, und ausserdem muss eine Bedingung f"ur die Ableitungen
gelten, damit die L"osung entlang der Charakteristik $y=\frac12(x^2+1)$
differenzierbar wird. M"oglicherweise erh"alt man also nur eine
schwache L"osung.
\end{teilaufgaben}
\end{loesung}
