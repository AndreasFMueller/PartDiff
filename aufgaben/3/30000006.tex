Consider the partial differential equation
\[
\frac{\partial u}{\partial x}
+x
\frac{\partial u}{\partial y}
=
\sin u
\]
on the unit square
$\Omega=(0,1)^2=\{(x,y)\,|\, 0<x,y<1\}$.
Which of the following boundary conditions lead to a well posed
problem?
\begin{teilaufgaben}
\item
Boundary values $\{(x,0)\,|\,x\in [0,1]\}$ and $\{(0,y)\,|\,y\in[0,1]\}$
\item
Boundary values $\{(x,1)\,|\,x\in [0,1]\}$ and $\{(1,y)\,|\,y\in[0,1]\}$
\item
Boundary values on all of $\partial\Omega$
\item
Boundary values $\{(x,1)\,|\,x\in [0,1]\}$ and $\{(0,y)\,|\,y\in[0,1]\}$
\item
Boundary values
$\{(x,0)\,|\,x\in [0,1]\}$,
$\{(x,1)\,|\,x\in [0,1]\}$ and 
$\{(0,y)\,|\,y\in[0,\frac12)\}$
\end{teilaufgaben}

\begin{loesung}
The equation is not linear, but quasilinear.
The characteristics will tell us where boundary values need to be
specified.
The equation for the characteristics is
\begin{equation}
\frac{d}{dt}\begin{pmatrix}x\\y\\u\end{pmatrix}
=
\begin{pmatrix}
1\\x\\\sin u
\end{pmatrix}
\qquad
\Rightarrow
\qquad
\begin{aligned}
\dot x &= 1\\
\dot y &= x\\
\dot u &=\sin u.
\end{aligned}
\end{equation}
The first equation has the solution
\[
x(t) = t+x_0.
\]
From this, we can get
\[
y(t) = \frac12 t^2+x_0t+y_0
\]
as a solution for the second equation.
With $t=x-x_0$ we can eliminate the variable $t$ and end up with
\[
y=\frac12(x-x_0)^2+x_0(x-x_0)+y_0=\frac12x^2 + y_0-\frac12x_0^2.
\]
The projections of the characteristics in the $x$-$y$-plane are
all parabolas of the form $y=\frac12x^2+c$.

\begin{figure}
\begin{center}
\includeagraphics[]{domain-1.pdf}
\end{center}
\caption{Projection of characteristics to the $x$-$y$-plane
(problem~\ref{30000006})
\label{30000006:char}}
\end{figure}
Boundary conditions need to be specified on a subset
$R\subset \partial\Omega$ such that the projection of the
characteristics that touch the domain have exactly one
point in common with $R$
(figure~\ref{30000006:char}). 
\begin{teilaufgaben}
\item
The left and bottom boundaries intersect each parabola exactly once.
Each parabola that touches the domain also intersects the left and bottom
boundary.
This means that the problem is well posed.
\item
Well posed, just as in a)
\item
Each parabola intersects the boundary twice, so the boundary values
cannot be prescribed arbitrarily without contraticting themselves.
This problem is in general not well posed.
\item
The parabolas below $y=\frac12x^2$ do not intersect the left and top
boundaries, so the solution is not uniquely determined in
$\{(x,y)\,|\, y<\frac12x^2\}$.
\item 
The condition is satisfied, but we have to ensure that the part of
the solution determined by the top boundary values and the
part determined by the bottom boundary have to fit together along
the curve $y=\frac12(x^2+1)$.
For this we have to require that the limit of the boundary values
in $(0,\frac12)$ has to be the same as the limit in $(1,1)$.
If this condition is satisfied, we get a uniquely determined 
(weak) solution
\end{teilaufgaben}
In all the subproblems where the boundary consists of multiple segments,
it may happen that the derivatives don't match and that we only get
a weak solution.
\end{loesung}
