Auf dem Gebiet $\Omega=\{(x,y)\;|\;0<x<\frac{\pi}2,0<y<\frac{\pi}{2}\}$
ist die partielle Differentialgleichung
\begin{equation}
\frac{1}{\cos x}\frac{\partial u}{\partial x}+\frac{\partial u}{\partial y}=0
\label{30000010:gleichung}
\end{equation}
gegeben.
\begin{teilaufgaben}
\item Finden Sie eine Lösung, die die Randbedingung
\[
u(0,y)=\cos y
\]
erfüllt.
\item Ist die Lösung durch diese Randbedingung eindeutig festgelegt?
\end{teilaufgaben}

\begin{loesung}
Es handelt sich um eine quasilineare Differentialgleichung erster Ordnung,
die gestellten Fragen können mit Hilfe von Charakteristiken beantwortet
werden.
\begin{teilaufgaben}
\item
Die Differentialgleichungen der Charakteristiken mit Parameter $t$
sind
\begin{align}
\dot x &= \frac1{\cos x}\label{30000010:1}\\
\dot y &= 1\label{30000010:2}\\
\dot u &= 0\label{30000010:3}
\end{align}
Die erste Gleichung kann mit Separation der Variablen gelöst werden:
\begin{align*}
\frac{dx}{dt}&=\frac1{\cos x}
&&\Rightarrow&
\int \cos x\,dx&=\int\, dt
&&\Rightarrow&
\sin x&=t + C
&&\Rightarrow&
x&=\arcsin(t + C)
\end{align*}
Die Randwerte sind für $x=0$ vorgegeben, wir parametrisieren die
Charakteristiken daher so, dass sie für $t=0$ auf der $y$-Achse
im Punkt $(0,y_0)$ beginnen. Dazu muss $C=0$ sein.

Die zweite Gleichung \eqref{30000010:2} hat als Lösung $y=t+y_0$, und
die dritte \eqref{30000010:3} hat $u=u_0=\cos y_0$ als Lösung.
Man muss also aus den Gleichungen
\begin{align*}
x&=\arcsin t\\
y&=t+y_0\\
u&=\cos y_0
\end{align*}
die Variablen $t$ und $y_0$ eliminieren.
Aus der ersten Gleichung bekommt man $t=\sin x$, und damit aus der zweiten
$y_0=y-\sin x$. Einsetzen in die dritte Gleichung liefert
\begin{equation}
u=\cos(y-\sin x).
\end{equation}
Zur Kontrolle rechnen wir nach:
\begin{align*}
\frac{\partial u}{\partial x}&=\sin(y-\sin x)\cos x
\\
\frac{\partial u}{\partial y}&=-\sin(y-\sin x)
\\
\frac1{\cos x}\frac{\partial u}{\partial x}+\frac{\partial u}{\partial y}
&=
\sin(y-\sin x)-\sin(y-\sin x)=0.
\end{align*}
Diese Funktion erfüllt auch die Randbedingung
\begin{equation*}
u(0,y)=\cos(y-\sin 0)=\cos y.
\end{equation*}
Die Funktion $u(x,y)=\cos(y-\sin x)$ ist also eine Lösung der
Differentialgleichung.
\item
\begin{figure}
\centering
\includeagraphics[width=0.45\hsize]{domain-1.pdf}
\caption{Charakteristiken der Differentialgleichung \eqref{30000010:gleichung},
im blauen Teilgebiet ist die Lösung durch die Randwerte (grün) nicht
festgelegt.
\label{30000010:char}}
\end{figure}
\begin{figure}
\centering
\includeagraphics[width=0.7\hsize]{loes.jpg}
\caption{3D-Darstellung der Lösung, soweit sie durch die Randbedingung
eindeutig bestimmt ist. Die $y$-Achse zeigt nach vorne.
\label{30000010:loesungsflaeche}}
\end{figure}
Die Eindeutigkeit kann ebenfalls mit Hilfe der Charakteristiken entschieden
werden.
Die Charaketeristiken in der $x$-$y$-Ebene haben die Gleichung
\[
y=y_0 + \sin x,
\]
sie sind in Abbildung~\ref{30000010:char} dargestellt.
Man erkennt, dass die Werte im Teilgebiet $y < \sin x$ durch die
Anfangswerte nicht festgelegt sind.
Der durch die Randbedingungen festgelegte Teil der Lösungsfläche ist
in Abbildung~\ref{30000010:loesungsflaeche} dargestellt.

Ist $g\colon\mathbb R\to\mathbb R$ eine beliebige differenzierbare Funktion mit
$g(y)=\cos y$ für $y>0$, dann ist $g(y-\sin x)$ eine Lösung der
Differentialgleichung \eqref{30000010:gleichung}, die die Randbedingung
erfüllt. Da die Funktion $g(y)$ für $y<0$ beliebig gewählt werden
kann, gibt es beliebig viele Lösungen der Differentialgleichung,
die die Randbedingung erfüllen.
\qedhere
\end{teilaufgaben}
\end{loesung}

\begin{bewertung}
Quasilineare PDGL ({\bf Q}) 1 Punkt,
Differentialgleichungen der Charakteristiken ({\bf D}) 1 Punkt,
Lösung der Differentialgleichungen für $x(t)$ ({\bf X})
und $y(t)$ ({\bf Y}) je 1 Punkt,
Lösung der partiellen Differentialgleichung ({\bf U}) 1 Punkt,
Verwendung der Charakteristikenmethode zur Beurteilung ob
vorgegebenen Randwerte die Lösung eindeutig bestimmen ({\bf B}) 1 Punkt.
\end{bewertung}

