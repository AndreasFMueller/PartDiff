Betrachten Sie das Cauchy-Problem f"ur die Differentialgleichung
\[
y\frac{\partial u}{\partial x}+x\frac{\partial u}{\partial y}=0
\]
und die Anfangsbedingung $u(0,y)=y^2$.
\begin{teilaufgaben}
\item Bestimmen Sie die Charakteristiken.
\item Welcher Teil der $x$-$y$-Ebene wird von Charakteristiken "uberdeckt,
die von der $y$-Achse ausgehen?
\item L"osen Sie das Cauchy-Problem. "Uberpr"ufen sie die L"osung durch
Einsetzen in die Differentialgleichung.
\item Betrachten Sie die Funktion
\[
v(x,y)=\begin{cases}
u(x,y)&\qquad |y|\ge |x|\\
(x^2-y^2)^2&\qquad |x|<|y|
\end{cases}
\]
Rechnen Sie nach, dass diese Funktion ebenfalls die Differentialgleichung
erf"ullt
die aber offenbar von $u(x,y)$ verschieden ist (zum Beispiel ist
$u(1,0)=-1$ und $v(1,0)=1$). Ist sie eine L"osung?
\end{teilaufgaben}

\begin{loesung}
Die Anfangsbedingung in dieser Aufgabe ist auf der $y$-Achse vorgegeben.
Zur Vermeidung von Verwechslungen bezeichnen wir die $y$-Werte auf
der $y$-Achse wieder mit $y_0$, "ahnlich wie in Aufgabe \ref{30000001}.
Die Methode der Charakteristiken wird also Funktionen $x(y_0,t)$,
$y(y_0,t)$ und $u(y_0,t)$ derart, dass die Differentialgleichungen
\begin{equation}
\begin{aligned}
\dot x(y_0, t)&=y\\
\dot y(y_0, t)&=x\\
\dot u(y_0, t)&=0
\end{aligned}
\label{30000002:chareq}
\end{equation}
erf"ullt sind mit den Anfangsbedingungen 
\begin{align*}
x(y_0, 0)&=0\\
y(y_0, 0)&=y_0\\
u(y_0, 0)&=f(y_0)
\end{align*}
wobei $f(y_0)$ die Anfangsfunktion entlang der $y$-Achse ist.
\begin{teilaufgaben}
\item
Aus der Gleichung (\ref{30000002:chareq}) kann man ablesen,
dass alle Charakteristiken parallel zur $x$-$y$-Ebene verlaufen.
Man kann also, um die Frage zu beantworten, nur die Differentialgleichungen
f"ur die beiden Funktionen $x$ und $y$ l"osen.

Leitet man die erste
Gleichung ab und setzt die zweite Gleichung ein, erh"alt man die
die Differentialgleichung
\[
\ddot x(t)=x(t)
\]
mit den L"osungen $x(t)=a\cosh t$ und $x(t)=b\sinh t$.
Da wir Kurven
suchen, die auf der $y$-Achse im Punkt $(0,y_0)$ beginnen, muss
\[
\begin{pmatrix}
x(t)\\y(t)
\end{pmatrix}
=
y_0
\begin{pmatrix}
\sinh t\\
\cosh t
\end{pmatrix}
\]
sein.
Diese Funktionen erf"ullen die Gleichung $y^2-x^2=y_0^2$, sie stellen
also Hyperbeln durch die Punkte $(0,y_0)$ dar.
\item
Die L"osungskurven sind Hyperbeln mit den Asymptoten $y=\pm x$, die
Charaketerisitiken k"onnen die Asymptoten nicht "uberschreiten, es wird
also nur der Bereich $\{(x,y)|\;|x|<|y|\}$ der $x$-$y$-Ebene "uberdeckt.
\item
Die Gleichung f"ur $u$ liefert $u(y_0, t)=y_0^2$.
Nach dem zweiten Schritt in der Methode der Charakteristiken
m"ussen wir jetzt $t$ und $y_0$ aus den Gleichungen
\begin{align*}
x&=y_0\sinh t\\
y&=y_0\cosh t\\
u&=y_0^2
\end{align*}
eliminieren. F"ur die hyperbolischen Funktionen gilt
\[
y^2-x^2=y_0^2(\cosh^2t-\sinh^2t)=y_0^2.
\]
Setzen wir dies in der dritten Gleichung ein, erhalten wir
\[
u(x,y)=y^2-x^2.
\]
Tats"achlich kann man durch Ableiten nachrechnen
\begin{align*}
\frac{\partial u}{\partial x}&=-2x\\
\frac{\partial u}{\partial y}&=2y\\
\Rightarrow
y\frac{\partial u}{\partial x}+x\frac{\partial u}{\partial y}&=-2xy+2xy=0,
\end{align*}
Die Differentialgleichung ist also erf"ullt.
\item
Die Ableitungen f"ur $|x|>|y|$ sind
\begin{align*}
\frac{\partial v}{\partial x}
&=2(x^2-y^2)2x\\
\frac{\partial v}{\partial x}
&=2(x^2-y^2)(-2y)
\end{align*}
Daraus bekommen wir
\[
y\frac{\partial v}{\partial x}
+
x \frac{\partial v}{\partial x}
=
4xy(x^2-y^2)
-4xy(x^2-y^2)=0,
\]
$v$ erf"ullt also die Differentialgleichung.
Trotzdem ist es keine
L"osung, denn die Ableitungen passen an der Nahtstelle nicht zusammen.
Auf der Geraden $x=y$ gilt n"amlich:
\begin{align*}
\frac{\partial}{\partial x}(y^2-x^2)&=-2x\\
\frac{\partial}{\partial x}(x^2-y^2)^2&=2(x^2-y^2)2x=0
\end{align*}
f"ur $x>0$ sind die Ableitungen also verschieden.
\end{teilaufgaben}

Man kann f"ur diese Differentialgleichung f"ur geeignete Anfangsbedingungen
verschiedene L"osungen konstruieren. Seien $f$ und $g$ irgendwelche
differenzierbaren Funktionen mit $f(0)=g(0)$ und $f'(0)=g'(0)=0$,
und betrachten wir die Funktion
\[
u(x,y)=\begin{cases}
f(y^2-x^2)&\qquad |y|\ge|x|\\
g(x^2-y^2)&\qquad |x|\le|y|
\end{cases}
\]
An der Nahtstelle passen die Werte zusammen, wenn $|x|=|y|$ ist $x^2-y^2$,
d.~h.~$f(y^2-x^2)=f(0)=g(0)=g(x^2-y^2)$, es spielt also keine Rolle, welche
der Definitionen man w"ahlt.
Auch die Ableitungen passen zusammen:
\begin{align*}
\frac{\partial}{\partial x}f(y^2-x^2)&=-2xf'(y^2-x^2)=-2xf'(0)=0
\\
\frac{\partial}{\partial x}g(x^2-y^2)&=2xg'(y^2-x^2)=2xg'(0)=0
\\
\frac{\partial}{\partial y}f(y^2-x^2)&=2yf'(y^2-x^2)=2yf'(0)=0
\\
\frac{\partial}{\partial y}g(x^2-y^2)&=-2yg'(y^2-x^2)=-2yg'(0)=0
\end{align*}
Die Funktion $u$ ist also stetig differenzierbar.
Eingesetzt in die Differentialgleichung ergibt sich
\[
y\partial_xu+x\partial_yu
=
\begin{cases}
-yf'(y^2-x^2)2x+xf'(y^2-x^2)2y=0&\qquad |y|>|x|\\
yg'(x^2-y^2)2x-xg'(x^2-y^2)2y=0&\qquad |y|<|x|
\end{cases}
\]
Sie ist also eine L"osung. Sie erf"ullt die Anfangsbedingung
$u(0,y)=f(y^2)$, f"ur jede beliebige Funktion $g$. Das Cauchyproblem
ist also nicht eindeutig l"osbar.

Ursache daf"ur ist nat"urlich, dass man von der $y$-Achse aus mit
Charakteristiken nur den Bereich $|y|\ge |x|$ erreichen kann.
Die L"osung mit Charakteristiken kann die Unterschiedlichen L"osungen
im Bereich $|y|<|x|$ gar nicht ``sehen''.
Die Funktion $u$ erf"ullt entlang der $x$-Achse eine Anfangsbedingung
der Form $u(x,0)=g(x^2)$, diese Werte beeinflussen die L"osung aber
nur im Bereich $|y|<|x|$. Sofern die L"osungen an der Nahtstelle
zusammenpassen, was durch $f'(0)=g'(0)=0$ sichergestellt wird,
kann man also die L"osung unendlich viele L"osungen dieses Cauchy-Problems
finden.
\end{loesung}

