Consider the Cauchy problem for the partial diffrential equation
\begin{equation}
y\frac{\partial u}{\partial x}+x\frac{\partial u}{\partial y}=0
\label{30000002:1}
\end{equation}
with boundary conitions $u(0,y)=y^2$.
It could be considered on the domain
$\Omega_+=\{ (x,y)\in\mathbb R^2\,|\, x > 0\}$
or on the domain
$\Omega_-=\{(x,y)\in\mathbb R^2\,|\, x <0\}$,
both wit a boundary condition specified on the $y$-axis.
But solutions on those domains can be stitched together to give a solution
in all of $\mathbb R^2=\overline{\Omega_+\cup\Omega_-}$ with the single
initial data along the $y$-axis.
It thus makes sense to talk about a solution of 
\eqref{30000006:1} in the domain $\Omega=\mathbb R^2$.
\begin{teilaufgaben}
\item
Find the characteristics.
\item
Which part of the $x$-$y$-plane is covered by characteristics that emanate
from the $y$-axis?
\item
Solve the Cauchy problem.
Verify your solution by substituting into the partial differential equation
\eqref{30000002:1}.
\item
Consider the function
\[
v(x,y)=\begin{cases}
u(x,y)&\qquad |y|\ge |x|\\
(x^2-y^2)^2&\qquad |y|<|x|.
\end{cases}
\]
Obviously, $u(x,y)$ and $v(x,y)$ are different, e.~g.~$u(1,0)=-1$
and $v(1,0)=1$.
Verify that the function $v(x,y)$ also is a solution of the partial
differential equation which is different from $u(x,y)$.
Is it a solution?
\end{teilaufgaben}

\begin{loesung}
The initial condition in this problem is specified on the $y$-axis.
To prevent confusions, we will such an initial point by $y_0$, similar
to what we did in problem \ref{30000001}.
The method of characteristics will find functions $x(y_0,t)$,
$y(y_0,t)$ and $u(y_0,t)$ such that the partial differential equation
\begin{equation}
\begin{aligned}
\dot x(y_0, t)&=y\\
\dot y(y_0, t)&=x\\
\dot u(y_0, t)&=0
\end{aligned}
\label{30000002:chareq}
\end{equation}
is satisfied with initial conditions
\begin{align*}
x(y_0, 0)&=0\\
y(y_0, 0)&=y_0\\
u(y_0, 0)&=f(y_0),
\end{align*}
where $f(y_0)$ gives the Cauchy initial values along the $y$-axis.
\begin{teilaufgaben}
\item
From equation \eqref{30000002:chareq} we can read off that the characteristics
are all parapllel to the $x$-$y$-plane.
Thus to find the characteristics, we can solve the equations for $x$ and $y$
independently of $u$.

Deriving the first equation and substituting it in the second equation
gives the ordinary differential equation
\[
\ddot x(t)=x(t)
\]
with solutions
$x(t)=a\cosh t$ and $x(t)=b\sinh t$.
We are looking for curves that emanate from points $(0,y_0)$ from
the $y$-axis, we have to take the $\sinh$-function for the solution,
giving
\[
\begin{pmatrix}
x(t)\\y(t)
\end{pmatrix}
=
y_0
\begin{pmatrix}
\sinh t\\
\cosh t
\end{pmatrix}.
\]
This function satisfies the equation $y^2-x^2=y_0^2$, it thus
parameterices hyperbolas through the points $(0,y_0)$.
\item
The solution curves are hyperbolas with asymptotes $y=\pm x$,
the characteristics cannot cross the asymptotes, so only the part
$\{(x,y)\in\mathbb R^2\;|\;|x|<|y|\}$ of the $x$-$y$-plane is covered by these
characteristics.
\item
The equation for $u$ gives $u(y_0, t)=y_0^2$.
The second step in the method of characteristics requires that the
variables $t$ and $y_0$ be eliminated from the equations
\begin{align*}
x&=y_0\sinh t\\
y&=y_0\cosh t\\
u&=y_0^2.
\end{align*}
For the hyperbolic functions we have
\[
y^2-x^2=y_0^2(\cosh^2t-\sinh^2t)=y_0^2.
\]
Substituting this in the third equation gives the solution
\[
u(x,y)=y^2-x^2.
\]
In fact, by computing the derivatives, we get
\begin{align*}
\frac{\partial u}{\partial x}&=-2x\\
\frac{\partial u}{\partial y}&=2y\\
\Rightarrow
y\frac{\partial u}{\partial x}+x\frac{\partial u}{\partial y}&=-2xy+2xy=0,
\end{align*}
so the differential equation is satisfied.
\begin{figure}
\centering
\begin{tikzpicture}[>=latex]
\node at (0,0) {\includeagraphics[width=16cm]{solution.jpg}};
\node at (-0.7,11) {$z$};
\node at (2.2,-1.0) {$x$};
\node at (7.3,5.3) {$y$};
\end{tikzpicture}
\caption{Two solution surfaces for equation~\eqref{30000002:1}.
The Cauchy initial curve (green) does not uniquely determine the solution,
as the characteristics (red) do not cover the complete domain.
At the asymptotes (pink) any solution (orange example) can be attached,
as explained in
part d) of problem~\ref{30000002}.
\label{30000002:solutionfigure}}
\end{figure}
\item
The deriviatives for $|x|>|y|$ are
\begin{align*}
\frac{\partial v}{\partial x}
&=2(x^2-y^2)2x\\
\frac{\partial v}{\partial x}
&=2(x^2-y^2)(-2y)
\end{align*}
This implies
\[
y\frac{\partial v}{\partial x}
+
x \frac{\partial v}{\partial x}
=
4xy(x^2-y^2)
-4xy(x^2-y^2)=0,
\]
so $v$ satisifies the partial differential equation.
In spite of this, $v$ is not a solution, because the pieces do not
fit together diffrentiably along the  lines $y=\pm x$.
On the line $x=y$, we have
\begin{align*}
\frac{\partial}{\partial x}(y^2-x^2)&=-2x\\
\frac{\partial}{\partial x}(x^2-y^2)^2&=2(x^2-y^2)2x=0.
\end{align*}
In particular, for $x>0$, the derivatives are different.
\end{teilaufgaben}

One can easily construct many different solutions for this equation
(see figure~\ref{30000002:solutionfigure}).
Let $f$ and $g$ be differentiable functions with $f(0)=g(0)$
and $f'(0)=g'(0)=0$, and consider the function
\[
u(x,y)=\begin{cases}
f(y^2-x^2)&\qquad |y|\ge|x|\\
g(x^2-y^2)&\qquad |y|\le|x|.
\end{cases}
\]
The values fit together along $|x|=|y|$, because then $x^2-y^2=0$,
i.~e.~$f(y^2-x^2)=f(0)=g(0)=g(x^2-y^2)$, so it does not matter which
definition we choose.
This even is true for the derivatives:
\begin{align*}
\frac{\partial}{\partial x}f(y^2-x^2)&=-2xf'(y^2-x^2)=-2xf'(0)=0
\\
\frac{\partial}{\partial x}g(x^2-y^2)&=2xg'(y^2-x^2)=2xg'(0)=0
\\
\frac{\partial}{\partial y}f(y^2-x^2)&=2yf'(y^2-x^2)=2yf'(0)=0
\\
\frac{\partial}{\partial y}g(x^2-y^2)&=-2yg'(y^2-x^2)=-2yg'(0)=0
\end{align*}
This means that $u$ is continuously differentiable.
Substituting it into the differential equation gives
\[
y\partial_xu+x\partial_yu
=
\begin{cases}
-yf'(y^2-x^2)2x+xf'(y^2-x^2)2y=0&\qquad |y|>|x|\\
yg'(x^2-y^2)2x-xg'(x^2-y^2)2y=0&\qquad |y|<|x|,
\end{cases}
\]
which means it is a solution.
It satisfies the initial condition $u(0,y)=f(y^2)$ for every function $g$.
By choosing different functions $g$ subject to the conditions noted,
we can construct arbitrarily many solutions.

The reason for this is of course that the characteristics emanating from
the $y$-axis cover only the domain with $|y|> |x|$.
The part $|y|<|x|$ is ``invisible'' to these characteristics.
\end{loesung}

