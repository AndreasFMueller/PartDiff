%
% graph.tex -- Template für standalone TIKZ Bilder
%
% (c) 2019 Prof Dr Andreas Müller, Hochschule Rapperswil
%
\documentclass[tikz,12pt]{standalone}
\usepackage{amsmath}
\usepackage{times}
\usepackage{txfonts}
\usepackage{pgfplots}
\usepackage{csvsimple}
\usetikzlibrary{arrows,intersections,math,calc}
\begin{document}
\newboolean{fein}
\setboolean{fein}{true}
\def\skala{3}
\begin{tikzpicture}[>=latex,thick,scale=\skala]

\def\tmin{0.02}

\definecolor{darkgreen}{rgb}{0,0.6,0}

\fill[color=blue!10] (-2.1,0) rectangle (2.1,{3.14159/2});

\begin{scope}
\clip (-2.1,0) rectangle (2.1,{3.14159/2});
	\foreach \xnull in {-5,-4,...,2}{
		\draw[color=red] plot[domain=1:\tmin,samples=100]
			({-ln(\x)+\xnull},{3.14159*acos(\x)/180});
	}
\ifthenelse{\boolean{fein}}{
	\foreach \xnull in {-5.8,-5.6,...,2}{
		\draw[color=red,line width=0.1pt]
			plot[domain=1:\tmin,samples=100]
				({-ln(\x)+\xnull},{3.14159*acos(\x)/180});
	}
}{}
\end{scope}

\draw[->] (-2.1,0) -- (2.3,0) coordinate[label={$x$}];
\draw[->] (0,-0.1) -- (0,1.8) coordinate[label={right:$y$}];

\foreach \x in {-2,-1,1,2}{
	\draw (\x,{-0.1/\skala}) -- (\x,{0.1/\skala});
	\node at (\x,0) [below] {$\x$};
}
\draw ({-0.1/\skala},{3.14159/2}) -- ({0.1/\skala},{3.14159/2});
\node at (0,{3.14159/2}) [above left] {$\frac\pi2$};

\draw[color=darkgreen,line width=1.4pt] (-2.1,0) -- (2.1,0);

\end{tikzpicture}
\end{document}

