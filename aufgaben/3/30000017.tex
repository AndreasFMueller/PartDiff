Gegeben ist die Differentialgleichung
\begin{equation}
\frac{\partial u}{\partial x}
+
\cos x\, \frac{\partial u}{\partial y}
=
1
\label{30000017:eq}
\end{equation}
auf dem Gebiet 
\[
\Omega=\{ (x,y) \,|\,
0<x<\infty
\;\text{und}\;
-1<y<1
\}.
\]
mit der Randbedingung
\[
u(0,y) = y^3.
\]
\begin{teilaufgaben}
\item
Finden Sie eine Lösung der Differentialgleichung \eqref{30000017:eq}.
\item
Ist die Lösung durch die Vorgaben eindeutig bestimmt?
\end{teilaufgaben}

\begin{loesung}
\begin{figure}
\centering
\includeagraphics[]{char.pdf}
\caption{Charakteristiken der Differentialgleichung
\eqref{30000017:eq}.
Das hellrote Gebiet wird nicht von Charakteristiken überdeckt, die von der
Randbedingung festgelegt werden.
\label{30000017:char}}
\end{figure}
Es handelt sich um eine quasilineare partielle Differentialgleichung erster
Ordnung, die mit der Methode der Charakteristiken gelöst werden kann.
\begin{teilaufgaben}
\item
Die Differentialgleichung der Charakteristiken
\[
\begin{aligned}
\dot x&= 1      &&\Rightarrow &x&=t+x_0\\
\dot y&= \cos x &&            & &      \\
\dot u&= 1      &&            & &
\end{aligned}
\]
Da die Charakteristiken am linken Rand des Gebietes beginnen sollen,
können wir $x_0=0$ wählen, also $x=t$. 
Die erste Differentialgleichung hat als Lösung
\[
y(t) = \int \cos x(t)\,dt = \int \cos t\,dt = \sin t + C.
\]
Die Konstante $C$ muss so gewählt werden, dass die Kurve bei $y_0$
beginnt, also $y_0=y(0)=\sin 0 + C=C$ oder $C=y_0$.
Somit muss $y(t)=y_0+\sin t$ verwendet werden.
Die letzte Differentialgleichung hat als Lösung
\[
u(t) = t + u_0.
\]
Der Wert von $u_0$ wird durch die Randedingung $u_0=u(y_0,0)=y_0^3$
festgelegt.
Wir müssen also aus den Gleichungen
\begin{align*}
y&=y_0+\sin x\\
u(x,y)&=x+y_0^3
\end{align*}
die Grösse $y_0$ eliminieren.
Es ist $y_0=y-\sin x$, also
\begin{equation}
u(x,y)
=
x + (y-\sin x)^3.
\label{30000017:l}
\end{equation}

Wir überprüfen dieses Resultat durch Einsetzen in die Differentialgleichung.
Die Ableitungen sind
\begin{align*}
%(%i2) u:x+(y-sin(x))^3
%                                           3
%(%o2)                          (y - sin(x))  + x
%(%i3) diff(u,x)
%                                                   2
%(%o3)                     1 - 3 cos(x) (y - sin(x))
%(%i4) diff(u,y)
%                                              2
%(%o4)                           3 (y - sin(x))
\frac{\partial u}{\partial x}
&=
1 - 3 \cos x\cdot(y-\sin x)^2
\\
\frac{\partial u}{\partial y}
&=
3(y-\sin x)^2
\end{align*}
Einsetzen in die Differentialgleichung ergibt
\[
1 - 3 \cos x\cdot(y-\sin x)^2
+
\cos x
\cdot
(
3(y-\sin x)^2
)
=
1,
\]
also erfüllt~\eqref{30000017:l} die Differentialgleichung.
Einsetzen von $x=0$ liefert
\[
u(x,y)
=
0 + (y-\sin 0)^3 = y^3.
\]
die Anfangsbedingung ist also auch erfüllt.
\item
Die Charaketeristiken sind im Grundriss die Kurven $(x, y_0+\sin x)$, also
Sinus-Kurven in Abhängigkeit von $x$.
Die vom Interval $(-1,1)$ auf der $y$-Achse ausgehenden Kurven überdecken
nicht das ganze
Gebiet $\Omega$, damit ist die Lösung nicht eindeutig bestimmt
(Abbildung~\ref{30000017:char}).
\qedhere
\end{teilaufgaben}
\end{loesung}

\begin{bewertung}
\begin{teilaufgaben}
\item
Quasilinear DGL 1. Ordnung ({\bf Q}) 1 Punkt,
Charakteristikengleichung ({\bf C}) 1 Punkt,
die ``einfachen'' Lösungen für $x$ und $u$ ({\bf E}) 1 Punkt,
Lösung für $y(t)$ ({\bf Y}) 1 Punkt,
Lösung für $u(x,y)$ (Elimination von $y_0$, {\bf U}) 1 Punkt,
\item
Charakteristiken überdecken nicht das ganze Gebiet ({\bf G}) 1 Punkt.
\end{teilaufgaben}
\end{bewertung}
