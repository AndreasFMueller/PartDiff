Sei $0 < y_0 < 1$ und $\Omega$ das Gebiet
\[
\Omega=\{ (x,y)\,|\, 0<x<1\wedge y_0 < y < 1\}.
\]
Die Funktion $u$ erfüllt in $\Omega$ die Differentialgleichung
\begin{equation}
\frac1x \frac{\partial u}{\partial x}
+
y\frac{\partial u}{\partial y}
=
0
\label{30000012:dgl}
\end{equation}
und die Randbedingung 
\[
u(x,y_0)=x^2,\qquad 0<x<1.
\]
\begin{teilaufgaben}
\item
Finden Sie eine Funktion $u$ mit diesen Eigenschaften.
\item
Ist die Lösung $u$ durch die gegebenen Randbedingungen eindeutig bestimmt?
\end{teilaufgaben}

\begin{loesung}
Dies ist eine quasilineare partielle Differentialgleichung erster Ordnung,
die mit der Methode der Charakteristiken gelöst werden kann.
Die Differentialgleichungen für die Charakteristiken sind
\[
\begin{aligned}
\dot x &= \frac1x &&           &        &=         \\
\dot y &=       y &&\Rightarrow&   y(t) &= y_0 e^t \\
\dot u &= 0       &&\Rightarrow&   u(t) &= u_0.
\end{aligned}
\]
Die Gleichung für $x(t)$ kann gelöst werden, indem man sie zunächst mit
$x$ multipliziert und dann bemerkt, dass $2x\dot x$ die Ableitung von $x^2$
ist.
Die Lösung ist daher
\begin{align*}
\dot x &= \frac1x
&&\Rightarrow&
x\dot x &= 1
&&\Rightarrow&
\frac{d}{dt}x^2&=2x\dot x=2
&&\Rightarrow&
x^2(t) &= 2t + x_0^2
&&\Rightarrow&
x(t)&=\sqrt{2t+x_0^2}.
\end{align*}
\begin{figure}
\centering
\includeagraphics[]{domain-1.pdf}
\caption{Definitionsgebiet der Differentialgleichung~\ref{30000013:dgl}
\label{30000013:domain}}
\end{figure}
\begin{teilaufgaben}
\item
Um eine Lösungsfunktion zu finden, müssen die Variablen $x_0$ und $t$ aus den
Gleichungen
\begin{align*}
x(t)&=\sqrt{2t+x_0^2}\\
y(t) &= y_0e^t\\
u(t)&=u_0 = x_0^2
\end{align*}
eliminiert werden, und $u$ muss durch $x$ und $y$ ausgedrückt werden.
Der Parameter $y_0$ darf stehen bleiben.

Zunächst kann man die zweite Gleichung nach $t$ und die erste nach $x_0^2$
auflösen: 
\begin{align*}
t&=\log\frac{y}{y_0}\\
x_0^2&=x^2-2t.
\end{align*}
Zusammen mit der Gleichung für $u(t)$ finden wir
\[
u(x,y) = x^2 - 2\log\frac{y}{y_0}.
\]
Zur Kontrolle leiten wir diese Funktion nach $x$ und $y$ ab
\begin{align*}
\frac{\partial u}{\partial x}
&=
2x
\\
\frac{\partial u}{\partial y}
&=
-2\frac{y_0}{y}\cdot\frac1{y_0}=-\frac2y
\end{align*}
und setzen dies in die Differentialgleichung ein.
Wir erhalten
\[
\frac1x\frac{\partial u}{\partial x} + y\frac{\partial u}{\partial y}
=
\frac1\cdot (2x)
+
y\cdot \biggl(-\frac{2}{y}\biggr)
=
0,
\]
die Funktion $u$ erfüllt also tatsächlich die Differentialgleichung.
Aber auch die Randwerte sind korrekt:
\[
u(x, y_0)=x^2-2\log\frac{y_0}{y_0}=x^2,
\]
also erfüllt $u(x,y)$ auch die Randbedingungen.
\item
Im Grundriss erfüllen die Charakteristiken die Gleichung
\[
t = \frac12 (x^2-x_0^2)
\qquad\Rightarrow\qquad
y(x)=y_0\exp\biggl(\frac12 (x^2-x_0^2)\biggr)
\]
Diese Charakteristik beginnt im Punkt $(x_0,y_0)$.
Die Charakteristiken legen also nur Funktionswerte von $u$ zwischen
den Charakteristiken für $x_0=0$ und $x_0=1$ fest, also zwischen den Kurven
\[
y(x) = y_0\exp\frac12 x^2
\qquad\text{und}\qquad
y(x) = y_0\exp\biggl(\frac12 (x^2-1)\biggr)
\]
Abbildung~\ref{30000013} zeigt die beiden Kurven und das Gebiet $\Omega$.
Es ist offensichtlich, dass die Charakteristiken nicht ganz $\Omega$
überdecken.
Somit legen die gegeben Randbedingungen die Lösung nicht eindeutig fest.
\qedhere
\end{teilaufgaben}
\end{loesung}

\begin{diskussion}
Diese Aufgabe ist die an der $45^\circ$-Geraden gespiegelte Version der
Aufgabe~\ref{30000012}.
Die dortige Beobachtung bezüglich Lösbarkeit mit einem Separationsansatz ist
daher hier ebenfalls anwendbar.
\end{diskussion}

\begin{bewertung}
Methode der Charakteristiken ({\bf M}) 1 Punkt,
Differentialgleichungen der Charakteristiken ({\bf D}) 1 Punkt,
Lösung der Differentialgleichung für $x$ ({\bf X}) 1 Punkt,
Einsetzen der Randbedingung ({\bf R}) 1 Punkt,
Elimination der Variablen $y_0$ und $t$ ({\bf E}) 1 Punkt,
ist Lösung eindeutig bestimmt ({\bf B}) 1 Punkt.
\end{bewertung}
