Sei $0 < x_0 < 1$ und $\Omega$ das Gebiet
\[
\Omega=\{ (x,y)\,|\, x_0<x<1\wedge 0 < y < 1\}.
\]
Die Funktion $u$ erfüllt in $\Omega$ die Differentialgleichung
\begin{equation}
x\frac{\partial u}{\partial x}
+
\frac1y\frac{\partial u}{\partial y}
=
0
\label{30000012:dgl}
\end{equation}
und die Randbedingung 
\[
u(x_0,y)=y^2,\qquad 0<y<1.
\]
\begin{teilaufgaben}
\item
Finden Sie eine Funktion $u$ mit diesen Eigenschaften.
\item
Ist die Lösung $u$ durch die gegebenen Randbedingungen eindeutig bestimmt?
\end{teilaufgaben}

\begin{loesung}
Dies ist eine quasilineare partielle Differentialgleichung erster Ordnung,
die mit der Methode der Charakteristiken gelöst werden kann.
Die Differentialgleichungen für die Charakteristiken sind
\[
\begin{aligned}
\dot x &= x       &&\Rightarrow&   x(t) &= x_0 e^t \\
\dot y &= \frac1y &&           &        &          \\
\dot u &= 0       &&\Rightarrow&   u(t) &= u_0.
\end{aligned}
\]
Die Gleichung für $y(t)$ kann gelöst werden, indem man sie zunächst mit
$y$ multipliziert und dann bemerkt, dass $2y\dot y$ die Ableitung von $y^2$
ist.
Die Lösung ist daher
\begin{align*}
\dot y &= \frac1y
&&\Rightarrow&
y\dot y &= 1
&&\Rightarrow&
\frac{d}{dt}y^2&=2y\dot y=2
&&\Rightarrow&
y^2(t) &= 2t + y_0^2
&&\Rightarrow&
y(t)&=\sqrt{2t+y_0^2}.
\end{align*}
\begin{figure}
\centering
\includeagraphics[]{domain-1.pdf}
\caption{Definitionsgebiet der Differentialgleichung~\ref{30000012:dgl}
\label{30000012:domain}}
\end{figure}
\begin{teilaufgaben}
\item
Um eine Lösungsfunktion zu finden, müssen die Variablen $y_0$ und $t$ aus den
Gleichungen
\begin{align*}
x(t) &= x_0e^t\\
y(t)&=\sqrt{2t+y_0^2}\\
u(t)&=u_0 = y_0^2
\end{align*}
eliminiert werden, und $u$ muss durch $x$ und $y$ ausgedrückt werden.
Der Parameter $x_0$ darf stehen bleiben.

Zunächst kann man die erste Gleichung nach $t$ und die zweite nach $y_0^2$
auflösen: 
\begin{align*}
t&=\log\frac{x}{x_0}\\
y_0^2&=y^2-2t.
\end{align*}
Zusammen mit der Gleichung für $u(t)$ finden wir
\[
u(x,y) = y^2 - 2\log\frac{x}{x_0}.
\]
Zur Kontrolle leiten wir diese Funktion nach $x$ und $y$ ab
\begin{align*}
\frac{\partial u}{\partial x}
&=
-2\frac{x_0}{x}\cdot\frac1{x_0}=-\frac2x
\\
\frac{\partial u}{\partial y}
&=
2y
\end{align*}
und setzen dies in die Differentialgleichung ein.
Wir erhalten
\[
x\frac{\partial u}{\partial x} + \frac1y\frac{\partial u}{\partial y}
=
x\cdot \biggl(-\frac{2}{x}\biggr)
+ \frac1y\cdot (2y)
=
0,
\]
die Funktion $u$ erfüllt also tatsächlich die Differentialgleichung.
Aber auch die Randwerte sind korrekt:
\[
u(x_0,y)=y^2-2\log\frac{x_0}{x_0}=y^2,
\]
also erfüllt $u(x,y)$ auch die Randbedingungen.
\item
Im Grundriss erfüllen die Charakteristiken die Gleichung
\[
t = \log\frac{x}{x_0}
\qquad\Rightarrow\qquad
y(x)=\sqrt{\log\frac{x}{x_0}+y_0^2}.
\]
Die Charakteristik beginnt im Punkt $(x_0,y_0)$, insbesondere legen die
gegebenen Randwerte nur Funktionswerte von $u$ zwischen den Charakteristiken
für $y_0=0$ und $y_0=1$ fest, also zwischen den Kurven
\[
y(x) = \sqrt{\log\frac{x}{x_0}}
\qquad\text{und}\qquad
y(x) = \sqrt{\log\frac{x}{x_0}+1}.
\]
Abbildung~\ref{30000012} zeigt die beiden Kurven und das Gebiet $\Omega$.
Es ist offensichtlich, dass die Charakteristiken nicht ganz $\Omega$
überdecken.
Somit legen die gegeben Randbedingungen die Lösung nicht eindeutig fest.
\qedhere
\end{teilaufgaben}
\end{loesung}

\begin{diskussion}
Zufälligerweise kann man diese Aufgabe auch mit einem Separationsansatz
mit einer Summe $u(x,y)= X(x) + Y(y)$ lösen.
Setzt man den Ansatz in die Differentialgleichung \eqref{30000012:dgl} ein,
erhält man
\[
xX'(x) +\frac1yY'(y)=0,
\]
was man leicht separieren kann:
\[
xX'(x) = - \frac1yY'(y).
\]
Beide Seiten müssen daher konstant sein, wir nennen die Konstante $k$.
Wir müssen jetzt zwei gewöhnliche Differentialgleichungen mit dem Paramter
$k$ lösen, die aber sofort durch Integration gelöst werden k"Onnen:
\begin{align*}
X'(x)&=\frac{k}{x}&&\Rightarrow&X(x)&=k\log x + C_x      \\
Y'(y)&=-ky        &&\Rightarrow&Y(y)&=-\frac{k}2y^2 + C_y
\end{align*}
Wir haben daher für jedes $k$ die Lösung
\[
u_k(x,y)=-\frac{k}2 y^2 + k\log x + C
\]
gefunden.
Aus diesen Lösungen muss jetzt durch "Uberlagerung eine Lösung 
gebaut werden, die auch die Randbedingungen erfüllt.
An dieser Stelle versagt die Methode normalerweise, aber im vorliegenden
Fall hat man wegen der speziellen Form der Randbedingungen Glück, des
gerade eine der Funktion $u_k$ die Lösung ist, man muss also
nur noch $k$ und $C$ bestimmen.

An der Stelle $(x_0,0)$ muss der Wert $0$ entstehen, also
\[
u_k(x_0,0) = k\log x_0 + C=0
\qquad\Rightarrow\qquad C=-k\log x_0.
\]
Damit kann man die Lösung jetzt auch
\[
u_k(x,y)=-\frac{k}{2}y^2 +k\log\frac{x}{x_0}
\]
schreiben.
Diese Lösung erfüllt die Randbedingung im Punkt $(x_0,0)$, aber noch
nicht für alle $(x_0,y)$.
Setzt man die Randbedingung ein, erhält man die Gleichung
\begin{equation}
u_k(x_0,y_0) = -\frac{k}{2}y_0^2 +k\log\frac{x_0}{x_0} = -\frac{k}{2}y_0^2=y_0^2.
\label{30000012:k}
\end{equation}
Daraus können wir ablesen, dass $k=-2$ sein muss, und damit bekommen
wir tatsächlich die Lösung
\[
u_{-2}(x,y)=y^2 -2\log\frac{x}{x_0}.
\]
Man beachte, dass dies für jede andere Randbedingung aber nicht mehr
funktioniert, weil die Gleichung \eqref{30000012:k} dann nicht mehr
ein einziges $k$, liefert.
Und natürlich kann man mit dieser Methode auch nicht entscheiden, ob
das Problem gut gestellt ist.

In der Prüfung haben viel die Funktion
\[
v(x,y)=\biggl(y^2-2\log\frac{x}{x_0}\biggr)^2
\]
gefunden.
Die Ursache dafür war, dass der $y_0$-Wert, der für die Randbedingung
$u(x_0,y_0)=y_0^2$ und die Integrationskonstante für $y_0$, die
unsorgfältigerweise auch mit $y_0$ bezeichnet wurde, nicht
auseinandergehalten wurden.
Unglücklich ist dabei, dass diese Funktion die Differentialgleichung
erfüllt.
Dies ist nicht so überraschend: ist $u$ eine Lösung einer homogenen
linearen partiellen Differentialgleichung erster Ordnung, dann ist auch
$u^\alpha$ eine Lösung, denn es ist
\begin{align*}
a(x,y)\frac{\partial u^{\alpha}}{\partial x}
+
b(x,y)\frac{\partial u^{\alpha}}{\partial x}
&=
a(x,y)\alpha u^{\alpha-1}\frac{\partial u}{\partial x}
+
b(x,y)\alpha u^{\alpha-1}\frac{\partial u}{\partial y}
=
\alpha u^{\alpha-1}\underbrace{
\biggl(a(x,y)\frac{\partial u}{\partial x}
+
b(x,y)\frac{\partial u}{\partial y}\biggr)}_{\displaystyle=0}=0.
\end{align*}
Um sich zu versichern, dass man wirklich die Lösung gefunden hat, muss
man also unbedingt auch die Randbedingung prüfen.
\end{diskussion}

\begin{bewertung}
Methode der Charakteristiken ({\bf M}) 1 Punkt,
Differentialgleichungen der Charakteristiken ({\bf D}) 1 Punkt,
Lösung der Differentialgleichung für $y$ ({\bf Y}) 1 Punkt,
Einsetzen der Randbedingung ({\bf R}) 1 Punkt,
Elimination der Variablen $y_0$ und $t$ ({\bf E}) 1 Punkt,
ist Lösung eindeutig bestimmt ({\bf B}) 1 Punkt.
\end{bewertung}
