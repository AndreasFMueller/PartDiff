Sei $0 < x_0 < 1$ und $\Omega$ des Gebiet
\[
\Omega=\{ (x,y)\,|\, x_0<x<1\wedge 0 < y < 1\}.
\]
Die Funktion $u$ erf"ullt in $\Omega$ die Differentialgleichung
\begin{equation}
x\frac{\partial u}{\partial x}
+
\frac1y\frac{\partial u}{\partial y}
=
0
\label{30000012:dgl}
\end{equation}
und die Randbedingung 
\[
u(x_0,y)=y^2,\qquad 0<y<1.
\]
\begin{teilaufgaben}
\item
Finden Sie eine Funktion $u$ mit diesen Eigenschaften.
\item
Ist die L"osung $u$ durch die gegebenen Randbedingungen eindeutig bestimmt?
\end{teilaufgaben}

\begin{loesung}
Dies ist eine quasilineare partielle Differentialgleichung erster Ordnung,
die mit der Methode der Charakteristiken gel"ost werden kann.
Die Differentialgleichungen f"ur die Charakteristiken sind
\[
\begin{aligned}
\dot x &= x       &&\Rightarrow&   x(t) &= x_0 e^t \\
\dot y &= \frac1y &&           &        &          \\
\dot u &= 0       &&\Rightarrow&   u(t) &= u_0
\end{aligned}
\]
Die Gleichung f"ur $y(t)$ kann gel"ost werden, indem man sie zun"achst mit
$y$ multipliziert und dann bemerkt, dass $2y\dot y$ die Ableitung von $y^2$
ist.
Die L"osung ist daher
\begin{align*}
\dot y &= \frac1y
&&\Rightarrow&
y\dot y &= 1
&&\Rightarrow&
\frac{d}{dt}y^2&=2y\dot y=2
&&\Rightarrow&
y^2(t) &= 2t + y_0^2
&&\Rightarrow&
y(t)&=\sqrt{2t+y_0^2}.
\end{align*}
\begin{figure}
\centering
\includeagraphics[]{domain-1.pdf}
\caption{Definitionsgebiet der Differentialgleichung~\ref{30000012:dgl}
\label{30000012:domain}}
\end{figure}
\begin{teilaufgaben}
\item
Um eine L"osungsfunktion zu finden, m"ussen die Variablen $y_0$ und $t$ aus den
Gleichungen
\begin{align*}
x(t) &= x_0e^t\\
y(t)&=\sqrt{2t+y_0^2}\\
u(t)&=u_0 = y_0^2
\end{align*}
Zun"achst kann man die erste Gleichung nach $t$ und die zweite nach $y_0^2$
aufl"osen: 
\begin{align*}
t&=\log\frac{x}{x_0}\\
y_0^2&=y^2-2t
\end{align*}
Zusammen mit der Gleichung f"ur $u(t)$ finden wir
\[
u(x,y) = y^2 - 2\log\frac{x}{x_0}.
\]
Zur Kontrolle leiten wir diese Funktion nach $x$ und $y$ ab
\begin{align*}
\frac{\partial u}{\partial x}
&=
-2\frac{x_0}{x}\cdot\frac1{x_0}=-\frac2x
\\
\frac{\partial u}{\partial y}
&=
2y
\end{align*}
und setzen dies in die Differentialgleichung ein.
Wir erhalten
\[
x\frac{\partial u}{\partial x} + \frac1y\frac{\partial u}{\partial y}
=
x\cdot \biggl(-\frac{2}{x}\biggr)
+ \frac1y\cdot (2y)
=
0,
\]
die Funktion $u$ erf"ullt also tats"achlich die Differentialgleichung.
Aber auch die Randwerte sind korrekt:
\[
u(x_0,y)=y^2-2\log\frac{x_0}{x_0}=y^2,
\]
also erf"ullt $u(x,y)$ auch die Randbedingungen.
\item
Im Grundriss erf"ullen die Charakteristiken die Gleichung
\[
t = \log\frac{x}{x_0}
\qquad\Rightarrow\qquad
y(x)=\sqrt{\log\frac{x}{x_0}+y_0^2}
\]
Die Charakteristik beginnt im Punkt $(x_0,y_0)$, insbesondere legen die
gegebenen Randwerte nur Funktionswerte von $u$ zwischen den Charakteristiken
f"ur $y_0=0$ und $y_0=1$ fest, also zwischen den Kurven
\[
y(x) = \sqrt{\log\frac{x}{x_0}}
\qquad\text{und}\qquad
y(x) = \sqrt{\log\frac{x}{x_0}+1}.
\]
Abbildung~\ref{30000012} zeigt die beiden Kurven und das Gebiet $\Omega$.
Es ist offensichtlich, dass die Charakteristiken nicht ganz $\Omega$
"uberdecken.
Somit legen die gegeben Randbedingungen die L"osung nicht eindeutig fest.
\qedhere
\end{teilaufgaben}
\end{loesung}
