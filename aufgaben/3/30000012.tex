Sei $0 < x_0 < 1$ und $\Omega$ das Gebiet
\[
\Omega=\{ (x,y)\,|\, x_0<x<1\wedge 0 < y < 1\}.
\]
Die Funktion $u$ erf"ullt in $\Omega$ die Differentialgleichung
\begin{equation}
x\frac{\partial u}{\partial x}
+
\frac1y\frac{\partial u}{\partial y}
=
0
\label{30000012:dgl}
\end{equation}
und die Randbedingung 
\[
u(x_0,y)=y^2,\qquad 0<y<1.
\]
\begin{teilaufgaben}
\item
Finden Sie eine Funktion $u$ mit diesen Eigenschaften.
\item
Ist die L"osung $u$ durch die gegebenen Randbedingungen eindeutig bestimmt?
\end{teilaufgaben}

\begin{loesung}
Dies ist eine quasilineare partielle Differentialgleichung erster Ordnung,
die mit der Methode der Charakteristiken gel"ost werden kann.
Die Differentialgleichungen f"ur die Charakteristiken sind
\[
\begin{aligned}
\dot x &= x       &&\Rightarrow&   x(t) &= x_0 e^t \\
\dot y &= \frac1y &&           &        &          \\
\dot u &= 0       &&\Rightarrow&   u(t) &= u_0.
\end{aligned}
\]
Die Gleichung f"ur $y(t)$ kann gel"ost werden, indem man sie zun"achst mit
$y$ multipliziert und dann bemerkt, dass $2y\dot y$ die Ableitung von $y^2$
ist.
Die L"osung ist daher
\begin{align*}
\dot y &= \frac1y
&&\Rightarrow&
y\dot y &= 1
&&\Rightarrow&
\frac{d}{dt}y^2&=2y\dot y=2
&&\Rightarrow&
y^2(t) &= 2t + y_0^2
&&\Rightarrow&
y(t)&=\sqrt{2t+y_0^2}.
\end{align*}
\begin{figure}
\centering
\includeagraphics[]{domain-1.pdf}
\caption{Definitionsgebiet der Differentialgleichung~\ref{30000012:dgl}
\label{30000012:domain}}
\end{figure}
\begin{teilaufgaben}
\item
Um eine L"osungsfunktion zu finden, m"ussen die Variablen $y_0$ und $t$ aus den
Gleichungen
\begin{align*}
x(t) &= x_0e^t\\
y(t)&=\sqrt{2t+y_0^2}\\
u(t)&=u_0 = y_0^2
\end{align*}
eliminiert werden, und $u$ muss durch $x$ und $y$ ausgedr"uckt werden.
Der Parameter $x_0$ darf stehen bleiben.

Zun"achst kann man die erste Gleichung nach $t$ und die zweite nach $y_0^2$
aufl"osen: 
\begin{align*}
t&=\log\frac{x}{x_0}\\
y_0^2&=y^2-2t.
\end{align*}
Zusammen mit der Gleichung f"ur $u(t)$ finden wir
\[
u(x,y) = y^2 - 2\log\frac{x}{x_0}.
\]
Zur Kontrolle leiten wir diese Funktion nach $x$ und $y$ ab
\begin{align*}
\frac{\partial u}{\partial x}
&=
-2\frac{x_0}{x}\cdot\frac1{x_0}=-\frac2x
\\
\frac{\partial u}{\partial y}
&=
2y
\end{align*}
und setzen dies in die Differentialgleichung ein.
Wir erhalten
\[
x\frac{\partial u}{\partial x} + \frac1y\frac{\partial u}{\partial y}
=
x\cdot \biggl(-\frac{2}{x}\biggr)
+ \frac1y\cdot (2y)
=
0,
\]
die Funktion $u$ erf"ullt also tats"achlich die Differentialgleichung.
Aber auch die Randwerte sind korrekt:
\[
u(x_0,y)=y^2-2\log\frac{x_0}{x_0}=y^2,
\]
also erf"ullt $u(x,y)$ auch die Randbedingungen.
\item
Im Grundriss erf"ullen die Charakteristiken die Gleichung
\[
t = \log\frac{x}{x_0}
\qquad\Rightarrow\qquad
y(x)=\sqrt{\log\frac{x}{x_0}+y_0^2}.
\]
Die Charakteristik beginnt im Punkt $(x_0,y_0)$, insbesondere legen die
gegebenen Randwerte nur Funktionswerte von $u$ zwischen den Charakteristiken
f"ur $y_0=0$ und $y_0=1$ fest, also zwischen den Kurven
\[
y(x) = \sqrt{\log\frac{x}{x_0}}
\qquad\text{und}\qquad
y(x) = \sqrt{\log\frac{x}{x_0}+1}.
\]
Abbildung~\ref{30000012} zeigt die beiden Kurven und das Gebiet $\Omega$.
Es ist offensichtlich, dass die Charakteristiken nicht ganz $\Omega$
"uberdecken.
Somit legen die gegeben Randbedingungen die L"osung nicht eindeutig fest.
\qedhere
\end{teilaufgaben}
\end{loesung}

\begin{diskussion}
Zuf"alligerweise kann man diese Aufgabe auch mit einem Separationsansatz
mit einer Summe $u(x,y)= X(x) + Y(y)$ l"osen.
Setzt man den Ansatz in die Differentialgleichung (\ref{30000012:dgl}) ein,
erh"alt man
\[
xX'(x) +\frac1yY'(y)=0,
\]
was man leicht separieren kann:
\[
xX'(x) = - \frac1yY'(y).
\]
Beide Seiten m"ussen daher konstant sein, wir nennen die Konstante $k$.
Wir m"ussen jetzt zwei gew"ohnliche Differentialgleichungen mit dem Paramter
$k$ l"osen, die aber sofort durch Integration gel"ost werden k"onnen:
\begin{align*}
X'(x)&=\frac{k}{x}&&\Rightarrow&X(x)&=k\log x + C_x      \\
Y'(y)&=-ky        &&\Rightarrow&Y(y)&=-\frac{k}2y^2 + C_y
\end{align*}
Wir haben daher f"ur jedes $k$ die L"osung
\[
u_k(x,y)=-\frac{k}2 y^2 + k\log x + C
\]
gefunden.
Aus diesen L"osungen muss jetzt durch "Uberlagerung eine L"osung 
gebaut werden, die auch die Randbedingungen erf"ullt.
An dieser Stelle versagt die Methode normalerweise, aber im vorliegenden
Fall hat man wegen der speziellen Form der Randbedingungen Gl"uck, des
gerade eine der Funktion $u_k$ die L"osung ist, man muss also
nur noch $k$ und $C$ bestimmen.

An der Stelle $(x_0,0)$ muss der Wert $0$ entstehen, also
\[
u_k(x_0,0) = k\log x_0 + C=0
\qquad\Rightarrow\qquad C=-k\log x_0.
\]
Damit kann man die L"osung jetzt auch
\[
u_k(x,y)=-\frac{k}{2}y^2 +k\log\frac{x}{x_0}
\]
schreiben.
Diese L"osung erf"ullt die Randbedingung im Punkt $(x_0,0)$, aber noch
nicht f"ur alle $(x_0,y)$.
Setzt man die Randbedingung ein, erh"alt man die Gleichung
\begin{equation}
u_k(x_0,y_0) = -\frac{k}{2}y_0^2 +k\log\frac{x_0}{x_0} = -\frac{k}{2}y_0^2=y_0^2.
\label{30000012:k}
\end{equation}
Daraus k"onnen wir ablesen, dass $k=-2$ sein muss, und damit bekommen
wir tats"achlich die L"osung
\[
u_{-2}(x,y)=y^2 -2\log\frac{x}{x_0}.
\]
Man beachte, dass dies f"ur jede andere Randbedingung aber nicht mehr
funktioniert, weil die Gleichung (\ref{30000012:k}) dann nicht mehr
ein einziges $k$, liefert.
Und nat"urlich kann man mit dieser Methode auch nicht entscheiden, ob
das Problem gut gestellt ist.

In der Pr"ufung haben viel die Funktion
\[
v(x,y)=\biggl(y^2-2\log\frac{x}{x_0}\biggr)^2
\]
gefunden.
Die Ursache daf"ur war, dass der $y_0$-Wert, der f"ur die Randbedingung
$u(x_0,y_0)=y_0^2$ und die Integrationskonstante f"ur $y_0$, die
unsorgf"altigerweise auch mit $y_0$ bezeichnet wurde, nicht
auseinandergehalten wurden.
Ungl"ucklich ist dabei, dass diese Funktion die Differentialgleichung
erf"ullt.
Dies ist nicht so "uberraschend: ist $u$ eine L"osung einer homogenen
linearen partiellen Differentialgleichung erster Ordnung, dann ist auch
$u^\alpha$ eine L"osung, denn es ist
\begin{align*}
a(x,y)\frac{\partial u^{\alpha}}{\partial x}
+
b(x,y)\frac{\partial u^{\alpha}}{\partial x}
&=
a(x,y)\alpha u^{\alpha-1}\frac{\partial u}{\partial x}
+
b(x,y)\alpha u^{\alpha-1}\frac{\partial u}{\partial y}
=
\alpha u^{\alpha-1}\underbrace{
\biggl(a(x,y)\frac{\partial u}{\partial x}
+
b(x,y)\frac{\partial u}{\partial y}\biggr)}_{\displaystyle=0}=0.
\end{align*}
Um sich zu versichern, dass man wirklich die L"osung gefunden hat, muss
man also unbedingt auch die Randbedingung pr"ufen.
\end{diskussion}

\begin{bewertung}
Methode der Charakteristiken ({\bf M}) 1 Punkt,
Differentialgleichungen der Charakteristiken ({\bf D}) 1 Punkt,
L"osung der Differentialgleichung f"ur $y$ ({\bf Y}) 1 Punkt,
Einsetzen der Randbedingung ({\bf R}) 1 Punkt,
Elimination der Variablen $y_0$ und $t$ ({\bf E}) 1 Punkt,
ist L"osung eindeutig bestimmt ({\bf B}) 1 Punkt.
\end{bewertung}
