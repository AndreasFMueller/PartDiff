Auf dem Gebiet $\Omega = \{ (x,y)\,|\, x>0\;\text{und}\;y > 0\}$
ist die partielle Differentialgleichung
\begin{equation}
\frac1x\frac{\partial u}{\partial x}+x^2\frac{\partial u}{\partial y}=0
\label{30000007:pdgl}
\end{equation}
gegeben. Uns interessiert aber nicht die L"osung (daher ist auch keine
Randbedingungen vorgegeben), sondern die Frage der eindeutigen L"osbarkeit.
Ist die Differentialgleichung eindeutig l"osbar, wenn man Randwerte 
f"ur Punkte $(x_0,0)$ mit $x_0>0$ vorgibt?

\begin{loesung}
Dies ist eine quasilineare partielle Differentialgleichung, die mit
dem Verfahren der Charakteristiken gel"ost werden kann. Ob die Gleichung
durch Vorgabe der Randwerte auf der $x$-Achse eindeutig gel"ost werden
kann, kann genau dann bejaht werden, wenn die von der $x$-Achse
ausgehenden Charakteristiken das ganze Gebiet $\Omega$
"uberdecken.

Die Differentialgleichungen der Charakterstiken mit Kurvenparameter $t$ sind 
\begin{align*}
\dot x(t)&=\frac1x\\
\dot y(t)&=x^2\\
\dot u(t)&=0
\end{align*}
In der ersten Differentialgleichung kommt nur $x(t)$ vor, sie ist unabh"angig
von den anderen Gleichungen. Eine L"osung kann mit Separation gefunden werden:
\begin{align*}
\frac{dx}{dt}&=\frac1x\\
\int x\,dx&=\int \,dt\\
\frac12x^2&= t + c&\Rightarrow x&=\sqrt{2t + C}
\end{align*}
Mit dieser L"osung kann jetzt auch die zweite Differentialgleichung
gel"ost werden, sie lautet
\begin{align}
\frac{dy}{dt}&=x^2=2t+C \notag \\
y&=t^2+Ct + D \label{30000007:ysol}
\end{align}
Die Gleichung f"ur $u$ besagt, dass $u(t)$ konstant ist, also den
Wert im Anfangspunkt der Charakteristik hat: $u(t)=u_0=g(x_0)$ (wenn
$u(x_0,0)=g(x_0)$ vorgegeben wird).

Wenn Randwerte nur auf der $x$-Achse vorgegeben sind, dann m"ussen
wir die Charakteristiken so parametrisieren, dass sie f"ur $t=0$
in einem Randpunkt $x_0$ auf der $x$-Achse beginnen, dass also
$y(0)=0$ ist. Daraus folgt aber mit (\ref{30000007:ysol}), dass
$y(0)=0=D$. Der zugeh"orige $x$-Wert ist $x(0)=\sqrt{C}$, somit
gilt $C=x_0^2$.

\begin{figure}
\begin{center}
\includeagraphics[width=0.6\hsize]{domain-1.pdf}
\end{center}
\caption{Von der $x$-Achse ausgehende Charakteristiken der
Differentialgleichung (\ref{30000007:pdgl})
\label{30000007:chardom}}
\end{figure}
Damit haben wir jetzt die Gleichungen
\begin{align*}
x&=\sqrt{2t+x_0^2}\\
y&=t^2+x_0^2t = t(t+x_0^2).
\end{align*}
Die erste Gleichung kann man nach $t$ aufl"osen:
\[
t=\frac12(x^2-x_0^2),
\]
und dieses $t$ in die zweite Gleichung einsetzen:
\begin{equation}
y=t(t+x_0^2)=\frac12(x^2-x_0^2)\biggl(\frac12(x^2-x_0^2)+x_0^2\biggr)
=\frac14(x^2-x_0^2)(x^2+x_0^2)=\frac14(x^4-x_0^4).
\label{30000007:char}
\end{equation}
Insbesondere gilt also immer $y\le\frac14x^4$, das Gebiet oberhalb
der Kurve $y=\frac14x^4$ wird von den Charakteristiken also gar nicht
erreicht. Die L"osung in diesem Teil des Gebietes ist erst bestimmt,
wenn auch noch Randwert auf der $y$-Achse vorgegeben werden. Die
Abbildung \ref{30000007:chardom} verdeutlicht diesen Sachverhalt.
\end{loesung}

\begin{diskussion}
Diese Aufgabe ist analog zur Aufgabe \ref{30000006}, die in
"Ubung 2 als Aufgabe 3 gestellt war.

Die Aufgabe verlangt nicht, dass man die L"osung in dem Teil des
Gebietes bestimmt, in dem sie durch eine Randbedingung
$u(x_0,0)=g(x_0)$ eindeutig bestimmt ist. Dennoch kann man die
L"osung aus der Gleichung (\ref{30000007:char}) der Charakteristik
jetzt relativ leicht finden. Dazu l"ost man (\ref{30000007:char})
zun"achst nach $x_0$ auf:
\[
x_0^4 = 4y-x^4.
\]
Die L"osung ist dann
\begin{equation}
u(x,y)=g(x_0)
=
g(\sqrt[4]{4y-x^4}).
\label{30000007:sol}
\end{equation}

Zur Kontrolle setzen wir dies in die Differentialgleichung ein.
Zun"achst sind die partiellen Ableitungen
\begin{align*}
\frac{\partial u}{\partial x}
&=
g'\bigl(\sqrt[4]{4y-x^4}\bigr) (4y-x^4)^{-\frac34}
\cdot (-4x^3)
\\
\frac{\partial u}{\partial y}
&=
g'\bigl(\sqrt[4]{4y-x^4}\bigr) (4y-x^4)^{-\frac34}
\cdot 4
\end{align*}
Eingesetzt in die Differentialgleichung ergibt sich:
\[
\frac1x\frac{\partial u}{\partial x}+x^2\frac{\partial u}{\partial y}=
g'\bigl(\sqrt[4]{4y-x^4}\bigr) (4y-x^4)^{-\frac34}
\cdot
\biggl(
-4x^3\cdot\frac1x+4x^2
\biggr)=0,
\]
(\ref{30000007:sol}) ist also tats"achlich eine L"osung der
Differentialgleichung
(\ref{30000007:pdgl}).

Man kann nat"urlich auch f"ur den Teil $y>\frac14x^4$ des
Gebietes eine L"osung zur Randbedingung $u(0,y_0)=h(y_0)$ finden.
Man findet unter Anwendung der gleichen Methode wie oben 
\[
u(x,y)=h(y-{\textstyle\frac14}x^4).
\]
Aus den beiden Teill"osungen l"asst sich jetzt eine L"osung im
ganzen Gebiet $\Omega$ zusammensetzen: 
\[
u(x,y)=\begin{cases}
g(\sqrt[4]{4y-x^4})&\qquad \text{f"ur $y<\frac14x^4$}\\
h(y-\frac14x^4)&\qquad \text{f"ur $y>\frac14x^4$}\\
\end{cases}
\]
Falls $g(0)=h(0)$ ist $u$ stetig. W"ahlt man f"ur $h$ eine glatte
Funktion, die zusammen mit all ihren Ableitungen im Punkt $0$
verschwindet, und f"ur $g$ die "uberall verschwindende Funktion,
d.~h.~$g(x_0)=0$ f"ur alle $x_0>0$, dann ist die zugeh"orige L"osung
ebenfalls glatt. dann verschwindet auf der $x$-Achse,
kann aber je nach Wahl f"ur $y>\frac14x^4$ beliebige Werte
annehmen, insbesondere erkennt man, dass die L"osung durch die
Werte entlang der $x$-Achse nicht eindeutig bestimmt ist.
\end{diskussion}

\begin{bewertung}
Quasilineare PDGL ({\bf Q}) 1 Punkt,
Differentialgleichungen der Charakteristiken ({\bf D}) 1 Punkt,
L"osung der Differentialgleichungen f"ur $x(t)$ ({\bf X}) 
und $y(t)$ ({\bf Y}) je 1 Punkt,
Verwendung der Charakteristikenmethode zur Beurteilung, ob die
vorgegebenen Randwerte ausreichend sind ({\bf M}) 1 Punkt,
Begr"undung, warum die Randwerte nicht ausreichen ({\bf R}) 1 Punkt.
\end{bewertung}
