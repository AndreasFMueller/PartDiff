Auf dem Gebiet $\Omega=\{(x,y)\;|\;0<x<1,1<y<2\}$ ist die partielle Differentialgleichung
\begin{equation}
y\frac{\partial u}{\partial x}+\frac1y\,\frac{\partial u}{\partial y}=0
\label{30000009:gleichung}
\end{equation}
gegeben.
\begin{teilaufgaben}
\item Finden Sie eine Lösung, die die Randbedingung
\[
u(x,1)=x
\]
erfüllt.
\item Ist die Lösung durch diese Randbedingung eindeutig festgelegt?
\end{teilaufgaben}

\begin{loesung}
\begin{teilaufgaben}
\item
Das Differentialgleichungssystem für die Charakteristiken ist
\begin{align}
\dot x &= y\label{30000009:1}\\
\dot y &= \frac1y\label{30000009:2}\\
\dot u &= 0\label{30000009:3}
\end{align}
Zur Bestimmung der Lösung der Differentialgleichung
mit den gegebenen Randbedingungen
suchen wir Lösungskurven, die in Randpunkten beginnen, und verwenden
die Konvention, den Anfangspunkt einer solchen Kurve mit $(x_0,y_0,u_0)$
zu bezeichnen. Auf Grund der Randbedingungen ist $y_0=1$ und $u_0=u(x_0)$.

Die Gleichung \eqref{30000009:3} ist einfach zu lösen: $u(t)=u_0$.
Aus der Gleichung \eqref{30000009:2} wird
\begin{align*}
y\dot y&=1\\
\int y\,dy&=\int dt\\
\frac12y^2&=t+C\\
y(t)&=\sqrt{2t+C}=\sqrt{2t+y_0^2}=\sqrt{2t+1}
\end{align*}
Daraus kann man jetzt auch $x(t)$ bestimmen:
\[
x(t)= \int_0^ty(\tau)\,d\tau
=\int_0^t\sqrt{2t+1}\,d\tau + x_0
=\frac13(2t+1)^{\frac{3}{2}}-\frac13+x_0
\]
Wir müssen jetzt also ausgehend von dem Gleichungssystem
\begin{align}
x&=\frac13(2t+1)^{\frac32}-\frac13+x_0\label{30000009:4}\\
y&=\sqrt{2t+1}\label{30000009:5}\\
u&=u(x_0)=x_0\label{30000009:6}
\end{align}
$u$ durch $x$ und $y$ ausdrücken. Dazu ersetzen wir zuerst in 
\eqref{30000009:4} mit Hilfe von \eqref{30000009:5} den Klammerausdruck
durch $y$:
\begin{equation}
x=\frac13y^3-\frac13+x_0\qquad\Rightarrow\qquad x_0=x-\frac13y^3+\frac13.
\label{30000009:char}
\end{equation}
Aus Gleichung \eqref{30000009:6} ergibt sich dann
\[
u(x,y)=x-\frac13y^3+\frac13.
\]
Wir kontrollieren dieses Resultat durch Nachrechnen:
\[
\left.
\begin{aligned}
\frac{\partial u}{\partial x}
&=
1
\\
\frac{\partial u}{\partial y}
&=
-y^2
\end{aligned}
\qquad
\right\}
\qquad
\Rightarrow
\qquad
y\,\frac{\partial u}{\partial x}
+\frac1y\,\frac{\partial u}{\partial y}
=
y\cdot 1-\frac1y\cdot y^2=0,
\]
die Funktion $u$ ist also tatsächlich eine Lösung der partiellen
Differentialgleichung. Auf dem Rand gilt:
\[
u(x,1)=\frac13\cdot1^3-\frac13+x=x,
\]
also ist auch die Randbedingung erfüllt.
\item
Die Charakteristiken erfüllen die Gleichung \eqref{30000009:char}, die
man auch nach $y$ auflösen kann:
\[
y=\root3\of{3(x-x_0)+1}
\]
Die vom unteren Teil des Randes ausgehenden Charakteristiken überdecken
nicht das ganze Gebiet, wie man der Abbildung \ref{30000009:charakteristiken}
entnehmen kann. Das Teilgebiet 
\[
\Omega_1=\{(x,y)\;|\; 0<x<1, \root3\of{3x+1} < y < 2\}
\]
wird nicht überdeckt, dort ist die Funktion also durch die Randbedingung
nicht festgelegt.
\qedhere
\end{teilaufgaben}
\begin{figure}
\begin{center}
\includeagraphics[width=0.4\hsize]{domain-1.pdf}
\end{center}
\caption{Charakteristiken der Differentialgleichung \eqref{30000009:gleichung}
\label{30000009:charakteristiken}}
\end{figure}
\end{loesung}

\begin{bewertung}
Quasilineare PDGL ({\bf Q}) 1 Punkt,
Differentialgleichungen der Charakteristiken ({\bf D}) 1 Punkt,
Lösung der Differentialgleichungen für $x(t)$ ({\bf X})
und $y(t)$ ({\bf Y}) je 1 Punkt,
Lösung der partiellen Differentialgleichung ({\bf U}) 1 Punkt,
Verwendung der Charakteristikenmethode zur Beurteilung ob
vorgegebenen Randwerte die Lösung eindeutig bestimmen ({\bf B}) 1 Punkt.
\end{bewertung}

\begin{diskussion}
Man kann natürlich aus der Berechnung der Charakteristiken die Lösung
auch für allgemeinere Randbedingungen der Form $u(x,1)=g(x)$ bekommen.
Statt der Gleichung \eqref{30000009:6} ist dann
$u(x_0)=g(x_0)$ zu verwenden, und man erhält als Lösung
\[
u(x,y)=g\biggl(
x- \frac13y^3+\frac13
\biggr).
\]
\end{diskussion}
