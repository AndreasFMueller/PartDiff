Solve the Cauchy problem for the partial differential equation
\[
\frac{\partial u}{\partial x}
+
\frac{\partial u}{\partial y}
=1
\]
on the domain
$\Omega = \{ (x,y)\in\mathbb R^2\,|\, y > 0\}$
with boundary values
\[
u(x,0)=\cos x.
\]

\begin{loesung}
\begin{figure}
\centering
\includeagraphics[width=\hsize]{solution.jpg}
\caption{Solution surface with Cauchy initial curve (green) and
characteristics (red) for problem \ref{30000001}
\label{30000001:solutionfigure}}
\end{figure}
The initial curve for the Cauchy problem in this case is a curve
along the $x$-axis.
To prevent confusion, we will use the symbol $x_0$ to denote a point
on the $x$-axis. 
The corresponding initial values are then $\cos x_0$.

The method of characteristics says that we have to find functions
$x(x_0, t)$, $y(x_0,t)$ and $u(x_0,t)$ that satisfy the equations
\[
\begin{pmatrix}
\dot x(x_0,t)\\
\dot y(x_0,t)\\
\dot u(x_0,t)
\end{pmatrix}
=\begin{pmatrix}
1\\1\\1
\end{pmatrix}
\]
with boundary conditions
\begin{align*}
x(x_0,0)&=x_0\\
y(x_0,0)&=0\\
u(x_0,0)&=\cos x_0.
\end{align*}
The solutions functions have the form
\[
\begin{pmatrix}
x(x_0,t)\\
y(x_0,t)\\
u(x_0,t)
\end{pmatrix}
=
\begin{pmatrix}
x_0+t\\
t\\
\cos x_0 + t
\end{pmatrix}.
\]
The second step of the method of characteristics directs us to
eliminate $t$ and $x_0$ from the equations:
\begin{align*}
x&=x_0+t\\
y&=t\\
u&=\cos x_0 + t
\end{align*}
The second equation says that we can replace $t$ by $y$:
\begin{align*}
x&=x_0+y&x_0=x-y\\
u&=\cos x_0 + y.
\end{align*}
Substituting the first equation in the second, we get the solution
\[
u=\cos(x-y)+y.
\]
We verify that this really is the solution by computing the
partial derivatives and substituting them in the equation:
\begin{align*}
\frac{\partial u}{\partial x}
&=
-\sin(x-y)
\\
\frac{\partial u}{\partial y}
&=
1+\sin(x-y)
\\
\frac{\partial u}{\partial x}
+
\frac{\partial u}{\partial y}
&=
-\sin(x-y)
+
1+\sin(x-y)
=1,
\end{align*}
So the differential equation is satisfied.
Substituting $x=x_0$ and $y=0$ gives
\[
u(x_0,0) = \cos(x_0-0)+0 = \cos(x_0),
\]
so the boundary conditions are satisfied as well.
Figure~\ref{30000001:solutionfigure} shows the solution surface
with initial curve and characteristics.
\end{loesung}

