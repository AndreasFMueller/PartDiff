L"osen Sie das Cauchy-Problem f"ur die Differentialgleichung
\[
\frac{\partial u}{\partial x}
+
\frac{\partial u}{\partial y}
=1
\]
mit den Anfangswerten
\[
u(x,0)=\cos x.
\]

\begin{loesung}
Zun"achst m"ussen wir die Charakteristiken bestimmen, also die L"osungskurven
des Vektorfeldes
\[
\begin{pmatrix}
\dot x(t)\\
\dot y(t)\\
\dot z(t)
\end{pmatrix}
=\begin{pmatrix}
1\\1\\1
\end{pmatrix}
\]
Diese sind nat"urlich von der Form
\[
\begin{pmatrix}
x(t)\\
y(t)\\
z(t)
\end{pmatrix}
=
\begin{pmatrix}x_0\\y_0\\z_0\end{pmatrix}
+
t\begin{pmatrix}1\\1\\1\end{pmatrix}
\]
Dann m"ussen wir zu einem gegebenen Punkt $(x,y)$ den zugeh"origen
Ausgangspunkt der Charakterisitik finden. Die Anfangsbedingung
bedeutet $y_0=0$, wir suchen also $x_0$ und $t$ so, dass
\[
\begin{pmatrix}
x\\y
\end{pmatrix}
=
\begin{pmatrix}
x(t)\\
y(t)
\end{pmatrix}
=
\begin{pmatrix}x_0\\0\end{pmatrix}
+
t\begin{pmatrix}1\\1\end{pmatrix}
=\begin{pmatrix}x_0+t\\t\end{pmatrix}
\]
Daraus leiten wir ab, dass $t=y$ sein muss, und weiter dass $x_0+t=x$, also
$x_0=x-y$. Somit verl"auft die Charakteristik durch den Punkt
\[
\biggl(x-y, 0,\cos (x-y)\biggr)
\]
(die $z$-Komponente ist der zugeh"orige Anfangswert). Um den Funktionswert
"uber dem Punkt $(x,y)$ zu finden, muss man jetzt der Charakteristik folgen,
d.~h.~nach der Gleichung der Charakterisitik bedeutet dies, zu jeder
Komponenten $y$ hinzuzuaddieren, man bekommt
\[
(x, y,y+\cos (x-y)) 
\Rightarrow u(x,y)=y+\cos(x-y)
\]
Wir "uberpr"ufen das Resultat noch durch nachrechnen:
\begin{align*}
\frac{\partial u}{\partial x}
&=
-\sin(x-y)
\\
\frac{\partial u}{\partial y}
&=
1+\sin(x-y)
\\
\frac{\partial u}{\partial x}
+
\frac{\partial u}{\partial y}
&=
-\sin(x-y)
+
1+\sin(x-y)
=1,
\end{align*}
die Differentialgleichung ist also erf"ullt.
\end{loesung}

