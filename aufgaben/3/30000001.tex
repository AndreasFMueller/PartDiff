L"osen Sie das Cauchy-Problem f"ur die Differentialgleichung
\[
\frac{\partial u}{\partial x}
+
\frac{\partial u}{\partial y}
=1
\]
auf dem Gebiet $\Omega = \{ (x,y)\in\mathbb R^2\,|\, y > 0\}$
mit den Anfangswerten
\[
u(x,0)=\cos x.
\]

\begin{loesung}
Die Anfangskurve f"ur das Cauchy-Problem ist in diesem Falle also die
$x$-Achse. Um Verwechslungen mit der Funktion $x$ zu vermeiden, bezeichnen
wir die Werte von $x$ auf der $x$-Achse mit $x_0$, die zugeh"origen
Anfangswerte sind dann $\cos x_0$. 

Die Methode der Charakteristiken besagt jetzt also, dass wir Funktionen
$x(x_0, t)$, $y(x_0,t)$ und $u(x_0,t)$ bestimmen m"ussen, die die
Differentialgleichungen
\[
\begin{pmatrix}
\dot x(x_0,t)\\
\dot y(x_0,t)\\
\dot z(x_0,t)
\end{pmatrix}
=\begin{pmatrix}
1\\1\\1
\end{pmatrix}
\]
erf"ullen mit der Anfangsbedingung
\begin{align*}
x(x_0,0)&=x_0\\
y(x_0,0)&=0\\
u(x_0,0)&=\cos x_0.
\end{align*}
Diese sind nat"urlich von der Form
\[
\begin{pmatrix}
x(x_0,t)\\
y(x_0,t)\\
u(x_0,t)
\end{pmatrix}
=
\begin{pmatrix}
x_0+t\\
t\\
\cos x_0 + t
\end{pmatrix}.
\]
Gem"ass dem zweiten Schritt der Methode der Charakteristiken m"ussen
wir jetzt $t$ und $x_0$ aus diesen Gleichungen eliminieren:
\begin{align*}
x&=x_0+t\\
y&=t\\
u&=\cos x_0 + t
\end{align*}
Die zweite Gleichung besagt, dass wir $t$ durch $y$ ersetzen k"onnen:
\begin{align*}
x&=x_0+y&x_0=x-y\\
u&=\cos x_0 + y.
\end{align*}
Setzen wir die erste Gleichung in die zweite ein, erhalten wir
\[
u=\cos(x-y)+y.
\]
Wir "uberpr"ufen das Resultat noch durch nachrechnen:
\begin{align*}
\frac{\partial u}{\partial x}
&=
-\sin(x-y)
\\
\frac{\partial u}{\partial y}
&=
1+\sin(x-y)
\\
\frac{\partial u}{\partial x}
+
\frac{\partial u}{\partial y}
&=
-\sin(x-y)
+
1+\sin(x-y)
=1,
\end{align*}
die Differentialgleichung ist also erf"ullt.
\end{loesung}

