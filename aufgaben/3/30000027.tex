On the domain
\(
\Omega = \{(x,y) \mid y>1\},
\)
the partial differential equation
\begin{equation}
u
\frac{\partial u}{\partial x}
+
y
\frac{\partial u}{\partial y}
=
x
\label{30000027:pde}
\end{equation}
is given with boundary condition
\[
u(x_0,1) = 1 \qquad\text{for $x_0\in\mathbb{R}$}.
\]
Find the solution $u(x,y)$.

\begin{hinweis}
It may be useful to note that the system of ordinary differential
equations
\[
\left.
\begin{aligned}
\dot{X} &= Y\\
\dot{Y} &= X
\end{aligned}
\right\}
\qquad\text{has solutions}\qquad
\left\{
\begin{aligned}
X &= A\cosh t + B\sinh t\\
Y &= A\sinh t + B\cosh t.
\end{aligned}
\right.
\]
\end{hinweis}

\begin{loesung}
\begin{figure}
\centering
\includeagraphics[]{solution.pdf}
\caption{Solution surface of the partial differential equation
\eqref{30000027:pde}.
The characteristic curves are drawn in red, the Cauchy initial curve
in green.
The projection of the characteristic curves into the $x$-$y$-plane is
shown in light red.
\label{30000027:fig}}
\end{figure}
This is a quasilinear partial differential equation of first order
which can be solved using the method of characteristics.
The differential equation for the characteristics is
\begin{equation}
\frac{d}{dt}
\begin{pmatrix*}
x\\
y\\
u
\end{pmatrix*}
=
\begin{pmatrix*}
u\\
y\\
x
\end{pmatrix*}
\qquad\Rightarrow\qquad
\dot{y} = y \qquad\Rightarrow\qquad y(t) = y_0e^t.
\end{equation}
The boundary is specified by $y=1$, which means that in order for the 
characteristic curve to start in point $(x_0,1)$, then we must have
$1=y_0$.

The two functions $u(t)$ und $x(t)$ satisfy the equations
\begin{equation}
\left.
\begin{aligned}
\dot{x} &= u \\
\dot{u} &= x
\end{aligned}
\;
\right\}
\qquad\Rightarrow\qquad
\left\{\;
\begin{aligned}
\ddot{x}=x\\
\ddot{u}=u
\end{aligned}
\right.
\label{30000027:chardgl}
\end{equation}
These equations have linear combinations of $\sinh t$ und $\cosh t$
as solutions.
Let
\[
u(t) = A\cosh t + B\sinh t
\]
then the boundary condition $u(0)=1$ means that 
\[
1 = u(0) = A\cosh 0 + B\sinh 0 = A
\qquad\Rightarrow\qquad
u(t) = \cosh t +B\sinh t.
\]
From the equation $\dot{u}=x$ we can derive
\[
x(t) = A\sinh t + B\cosh t = \sinh t + B\cosh t.
\]
On the other hand, $B$ has to be chosen so that the characteristic
curve starts at $x_0 = x(0) = \sinh t + B\cosh 0 = B$.
This means that we now have a complete solution 
\begin{align*}
y(t) &= e^t                      &&\Rightarrow&    t &= \log y\bigg.\\
x(t) &= \sinh t + x_0\cosh t &&\Rightarrow&
 x_0 &= \frac{x-\sinh\log y}{\cosh\log y}\\
u(t) &= \cosh t + x_0\sinh t &&\Rightarrow&
   u &= \cosh\log y + \frac{x-\sinh\log y}{\cosh\log y}\sinh\log y
\\
     &                           &&           &
     &= \cosh\log y - \sinh\log y\tanh\log y + x\tanh\log y
\\
     &                           &&           &
     &= \frac{\cosh^2\log y - \sinh^2\log y}{\cosh\log y}+x\tanh\log y
\\
     &                           &&           &
     &= \frac{1}{\cosh\log y} + x \tanh\log y.
\end{align*}
So we get the solution
\[
u(x,y) = \operatorname{sech}\log y + x \tanh\log y
\]
of the partial differential equation~\eqref{30000027:pde}.

We can verify this solution by computing the partial derivatives
\begin{align*}
\frac{\partial u}{\partial x}
&=
\tanh\log y
\\
\frac{\partial u}{\partial y}
&=
\frac{1}{y}\cdot
\frac{x-\sinh\log y}{\cosh^2\log y}
\end{align*}
and substituting them in the partial differential equation
\begin{align*}
u\frac{\partial u}{\partial x}
+
y\frac{\partial u}{\partial y}
&=
\frac{1+x\sinh\log y}{\cosh\log y}
\frac{\sinh\log y}{\cosh\log y}
+
\frac{x-\sinh\log y}{\cosh^2\log y}
\\
&=
\frac{
\sinh\log y+x\sinh^2\log y +x-\sinh\log y
}{\cosh^2\log y}
\\
&=
x\frac{1+\sinh^2\log y}{\cosh^2\log y}
\\
&=x\frac{\cosh^2\log y}{\cosh^2\log y} = x.
\end{align*}
For the boundary condition we substitute $x=1$ und $y=y_0$ into
$u$ and obtain
\[
u(x_0,1) = \operatorname{sech}\log 1 + x_0\tanh\log 1 = \frac{1}{\cosh 1}=1,
\]
so the boundary condition is satisfied as well.
\end{loesung}

\begin{bewertung}
Differential equations for characteristics ({\bf D}) 1 point,
solution for $y(t)$ including $y_0=1$ ({\bf Y}) 1 point,
solutions for $x(t)$ und $u(t)$ ({\bf XU}) 1 point,
using boundary condition $u(0)=1$ to determine $A=1$ ({\bf A}) 1 point,
using boundary condition $x(0)=x_0$ to determine $B=x_0$ ({\bf B}) 1 point,
eliminate $x_0$ and $t$ giving $u(x,y)$ ({\bf U}) 1 point.
\end{bewertung}
