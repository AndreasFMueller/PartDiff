Sei $0 < y_0 < 1$ gegeben und $\Omega$ sei das Gebiet
\begin{equation}
\Omega
=
\{ (x,y)\,|\,
0 < x < 1
\wedge
y_0 < y < 1
\}.
\label{30000015:gebiet}
\end{equation}
Die Funktion $u$ erfüllt in $\Omega$ die partielle Differentialgleichung
\begin{equation}
\frac{\partial u}{\partial x}
+
\frac{1}{y^2} \frac{\partial u}{\partial y}
=
1
\label{30000015:gleichung}
\end{equation}
und die Randbedingung
\[
u(x, y_0) = \cos(x), \qquad 0 < x < 1.
\]
\begin{teilaufgaben}
\item
Finden Sie eine Funktion $u$ mit diesen Eigenschaften.
\item
Ist die Lösung $u$ durch die gegebenen Randbedingungen eindeutig bestimmt?
\end{teilaufgaben}

\begin{loesung}
\begin{figure}
\centering
\includeagraphics[]{domain-1.pdf}
\caption{Charakteristiken und Definitionsbereich $\Omega$
für die Differentialgleichung~\eqref{30000015:gleichung}
auf dem Gebiet~\eqref{30000015:gebiet}.
\label{30000015:domain}
}
\end{figure}
Dies ist eine quasilineare partielle Differentialgleichung erster Ordnung, die
mit der Methode der Charakteristiken gelöst werden kann.
Die Differentialgleichungen für die Charakteristiken sind
\[
\begin{aligned}
\dot x &= 1           &&           &              &         &&\Rightarrow& x(t) &= t + D
\\
\dot y &= \frac1{y^2} &&\Rightarrow& \int y^2\,dy &=\int dt &&\Rightarrow& \frac13 y(t)^3 &= t + C
\\
\dot u &= 1           &&           &              &         &&\Rightarrow& u(t) &= t + E
\end{aligned}
\]
Damit die Charakteristiken auf der Anfangsgerade bei $y=y_0$ beginnen, muss
$C=\frac13y_0^3$ gewählt werden, also
\[
\frac13 y^3 = t + C
\qquad\Rightarrow\qquad
t = \frac13(y^3-y_0^3).
\]
Die Charakteristik, die im Punkt $(x_0,y_0)$ beginnt, wird beschrieben durch
die Wahl $D=x_0$.
Die Anfangsbedingung $u(x_0,y_0)=\cos x_0 $
verlangt dann, dass $E=\cos x_0 $.
\begin{teilaufgaben}
\item Zur Bestimmung einer Funktion $u$ mit den gegebenen Randbedingungen
ist $x_0$ aus den Gleichungen
\[
\begin{aligned}
x           &= t + x_0          &&\Rightarrow& x_0 &= x - t              \\
{\textstyle\frac13}y^3 &= t + {\textstyle\frac13}y_0^3 &&\Rightarrow& t   &= \textstyle{\frac13}(y^3-y_0^3) \\
u           &= t + \cos x_0     &&\Rightarrow& u   &= t + \cos(x-t)
\end{aligned}
\]
zu eliminieren und nach $u$ aufzulösen:
\[
u(x,y) = 
{\textstyle \frac13}(y^3-y_0^3)
+
\cos (x-{\textstyle \frac13}(y^3-y_0^3)).
\]
Diese Lösung lässt sich durch Einsetzen auch sofort verifizieren:
\begin{align*}
\frac{\partial u}{\partial x}
&=
\cos\bigl(x-{\textstyle\frac13}(y^3-y_0^3)\bigr)
\\
\frac{\partial u}{\partial y}
&=
y^2 -
\cos\bigl(x-{\textstyle\frac13}(y^3-y_0^3)\bigr)
\cdot y^2
\\
\Rightarrow
\qquad
\frac{\partial u}{\partial x}
+
\frac1{y^2}
\frac{\partial u}{\partial y}
&=
\frac1{y^2}\biggl(y^2 -
\cos\bigl(x-{\textstyle\frac13}(y^3-y_0^3)\bigr)
\cdot y^2\biggr)
+
\cos\bigl(x-{\textstyle\frac13}(y^3-y_0^3)\bigr)
=1.
\end{align*}
\item
Die Charakteristik durch $(x_0,y_0)$ erfüllt im Grundriss die Gleichung
\[
x(y)
=
x_0 + {\textstyle\frac13}(y^3-y_0^3)
\ge
x_0,
\]
wie in Abbildung~\ref{30000015:gebiet} dargestellt.
Der Graph der Funktion $(y)$ liegt für $y>y_0$ immer rechts der
Geraden $x=x_0$.
Somit sind die Werte der Funktion für $x < {\textstyle\frac13}(y^3-y_0^3)$
nicht durch die gegebenen Randwerte bestimmt (blaues Gebiet in der
Abbildung~\ref{30000015:gebiet}).
Die Lösung ist also nicht eindeutig bestimmt.
\qedhere
\end{teilaufgaben}
\end{loesung}

\begin{diskussion}
Ist $g(x)$ eine beliebige differenzierbare Funktion, die mit $\cos x$ 
für $x>0$ übereinstimmt, aber nicht für $x<0$, dann ist
\[
\tilde u(x,y) = {\textstyle\frac13}(y^3-y_0^3) + g(x-{\textstyle\frac13}(y^3-y_0^3))
\]
eine Lösung der Differentialgleichung auf einem erweiterten Gebiet
$\mathbb R \times \{y_0\}$ mit Randwerten $u(x, y_0)=g(x)$ für beliebige $x$.
Für Punkte $(x,y)\in\Omega$ mit $y<{\textstyle\frac13}(x^3-x_0^3)$
ist das Argument
der Funktion $g$ negativ, insbesondere weicht für solche Punkte $g$ von
$\cos$ ab und damit ist dort $\tilde u(x,y)\ne u(x,y)$.
\end{diskussion}

\begin{bewertung}
Methode der Charakteristiken ({\bf M}) 1 Punkt,
Differentialgleichungen der Charakteristiken ({\bf D}) 1 Punkt,
Lösung der Differentialgleichung f"ur $y$ ({\bf Y}) 1 Punkt,
Einsetzen der Randbedingung ({\bf R}) 1 Punkt,
Elimination der Variablen $y_0$ und $t$ ({\bf E}) 1 Punkt,
ist Lösung eindeutig bestimmt ({\bf B}) 1 Punkt.
\end{bewertung}



