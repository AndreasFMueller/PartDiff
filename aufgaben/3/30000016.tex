Gegeben ist die Differentialgleichung
\begin{equation}
\sin y\,\frac{\partial u}{\partial x}
+
\frac{\partial u}{\partial y}
=
1
\label{30000016:eq}
\end{equation}
auf dem Gebiet 
\[
\Omega=\{ (x,y) \,|\, -1<x<1\;\text{und}\; 0<y<\infty\}.
\]
mit der Randbedingung
\[
u(x,0) = x^2.
\]
\begin{teilaufgaben}
\item
Finden Sie eine Lösung der Differentialgleichung \eqref{30000016:eq}.
\item
Ist die Lösung durch die Vorgaben eindeutig bestimmt?
\end{teilaufgaben}

\begin{loesung}
\begin{figure}
\centering
\includeagraphics[]{char.pdf}
\caption{Charakteristiken der Differentialgleichung
\eqref{30000016:eq}.
Das hellrote Gebiet wird nicht von Charakteristiken überdeckt, die von der
Randbedingung festgelegt werden.
\label{30000016:char}}
\end{figure}
Es handelt sich um eine quasilineare partielle Differentialgleichung erster
Ordnung, die mit der Methode der Charakteristiken gelöst werden kann.
\begin{teilaufgaben}
\item
Die Differentialgleichung der Charakteristiken
\[
\begin{aligned}
\dot x&= \sin y &&&&\\
\dot y&= 1 &&\Rightarrow &y&=t+y_0\\
\dot u&= 1 &&&&
\end{aligned}
\]
Da die Charakteristiken auf dem unteren Rand des Gebietes beginnen sollen,
können wir $y_0=0$ wählen, also $y=t$. 
Die erste Differentialgleichung hat als Lösung
\[
x(t) = \int \sin y(t)\,dt = \int \sin t\,dt = -\cos t + C.
\]
Die Konstante $C$ muss so gewählt werden, dass die Kurve bei $x_0$
beginnt, also $x_0=x(0)=-\cos 0 + C=C-1$ oder $C=x_0+1$.
Somit muss $x(t)=x_0+1-\cos t$ verwendet werden.
Die letzte Differentialgleichung hat als Lösung
\[
u(t) = t + u_0.
\]
Der Wert von $u_0$ wird durch die Randedingung $u_0=u(x_0,0)=x_0^2$
festgelegt.
Wir müssen also aus den Gleichungen
\begin{align*}
x&=x_0+1-\cos y\\
u(x,y)&=y+x_0^2
\end{align*}
die Grösse $x_0$ eliminieren.
Es ist $x_0=x+\cos y$, also
\begin{equation}
u(x,y) = y + (x+\cos y-1)^2.
\label{30000016:l}
\end{equation}

Wir überprüfen dieses Resultat durch Einsetzen in die Differentialgleichung.
Die Ableitungen sind
\begin{align*}
\frac{\partial u}{\partial x}
&=
2(x+\cos y - 1),
\\
\frac{\partial u}{\partial y}
&=
1-2(x+\cos y - 1)\sin y.
\end{align*}
Einsetzen in die Differentialgleichung ergibt
\[
\sin y
2(x+\cos y - 1)
+
1-2(x+\cos y - 1)\sin y
=
1,
\]
also erfüllt~\eqref{30000016:l} die Differentialgleichung.
Einsetzen von $y$ liefert
\[
(x,y) = 0 + (x + \cos 0 - 1)^2 = x^2,
\]
die Anfangsbedingung ist also auch erfüllt.
\item
Die Charaketeristiken sind im Grundriss die Kurven $(x_0-\cos y,y)$, also
Cosinus-Kurven in Abhängigkeit von $y$.
Die vom Interval $(-1,1)$ ausgehenden Kurven überdecken nicht das ganze
Gebiet $\Omega$, damit ist die Lösung nicht eindeutig bestimmt
(Abbildung~\ref{30000016:char}).
\qedhere
\end{teilaufgaben}
\end{loesung}

\begin{bewertung}
\begin{teilaufgaben}
\item
Quasilinear DGL 1. Ordnung ({\bf Q}) 1 Punkt,
Charakteristikengleichung ({\bf C}) 1 Punkt,
die ``einfachen'' Lösungen für $y$ und $u$ ({\bf E}) 1 Punkt,
Lösung für $x(t)$ ({\bf X}) 1 Punkt,
Lösung für $u(x,y)$ (Eliminiation von $x_0$, {\bf U}) 1 Punkt,
\item
Charakteristiken überdecken nicht das ganze Gebiet ({\bf G}) 1 Punkt.
\end{teilaufgaben}
\end{bewertung}
