L"osen Sie die Differentialgleichung
\[
x^2\frac{\partial u}{\partial x}+y^2\frac{\partial u}{\partial y}=0
\]
auf dem Gebiet $\Omega=\{(x,y)\,|\,\text{$ x< 0$ und $-1\le y<0$}\}$ mit der Randbedingung
\[
u(x,-1)=\frac1x,\quad x<0.
\]

\begin{loesung}
Dies ist eine quasilineare partielle Differentialgleichung
erster Ordnung, die mit Hilfe von Charakteristiken gel"ost werden
kann.
Die Anfangswerte sind auf den Punkten $(x_0,-1)$ vorgegeben.
Das Differentialgleichungssytem f"ur die Charakteristiken ist
\begin{align*}
\dot x&=x^2\\
\dot y&=y^2\\
\dot u&=0
\end{align*}
Die Funktionen $x$, $y$ und $u$ m"ussen die Anfangsbedingungen
\begin{align*}
x(x_0,t)&=x_0\\
y(x_0,t)&=-1\\
u(x_0,t)&=\frac1{x_0}
\end{align*}
erf"ullen.

Die ersten beiden Differentialgleichungen sind von der selben Art, diese
gew"ohnliche Differentialgleichung l"asst sich leicht mit dem Separation l"osen:
\begin{align*}
\frac{dx}{dt}&=x^2
&\Rightarrow&
&\int\frac{dx}{x^2}&=\int dt
&\Rightarrow&
&-\frac1x&=t+C
&\Rightarrow&
&x(t)&=-\frac1{t+C_1}
\\
\frac{dy}{dt}&=y^2
&&
&&
&\Rightarrow&
&&
&&
&y(t)&=-\frac1{t+C_2}
\end{align*}
Die Anfangsbedingungen verlangen
\begin{align*}
x(x_0,0)&=x_0&&\Rightarrow&C_1&=-\frac1{x_0}&&\Rightarrow&x(t)&=-\frac1{t-\frac1{x_0}}=\frac{x_0}{1-tx_0}\\
y(x_0,0)&=-1&&\Rightarrow&C_2&=1&&\Rightarrow&y(t)&=-\frac1{t+1}
\end{align*}
Nach dem zweiten Schritt der Charakteristiken-Methode m"ussen wir jetzt
$x_0$ und $t$ aus den Gleichungen
\begin{align*}
x&=-\frac{1}{t-\frac1{x_0}},\\
y&=-\frac{1}{1+t},\\
u&=\frac1{x_0}.
\end{align*}
eliminieren.
Die zweite Gleichung liefert
\[
t=-1-\frac1{y}.
\]
Setzen wir dies und die dritte Gleichung in der Gleichung f"ur $x$ ein,
finden wir
\begin{align*}
x&=-\frac{1}{-1-\frac1y-u},\\
\frac1x&=1+\frac1y+u,\\
u(x,y)&=-1+\frac1x-\frac1y.
\end{align*}

{\bf Kontrolle:}
Tats"achlich kann man durch Einsetzen von $(x_0,-1)$ sofort nachpr"ufen,
dass $u$ die Randbedingung erf"ullt:
\[
u(x_0,-1)=\frac1{x_0}-\frac1{-1}-1=\frac1{x_0}.
\]
Aber auch die Differentialgleichung ist erf"ullt:
\[
\left.
\begin{aligned}
\frac{\partial u}{\partial x}&=-\frac1{x^2}\\
\frac{\partial u}{\partial y}&=\frac1{y^2}\\
\end{aligned}
\right\}
\quad\Rightarrow\quad
x^2\frac{\partial u}{\partial x}+y^2\frac{\partial u}{\partial y}
=
-x^2\frac1{x^2}+y^2\frac1{y^2}
=
0,
\]
die Funktion ist also tats"achlich eine L"osung.

\medskip

Alternativ kann man diese Aufgabe auch mit Hilfe eines Separationsansatzes
l"osen. Allerdings verlangt dieses L"osung Hilfsmittel, die im Bachelor-Studium
normalerweise nicht behandelt werden. Sie ist daher f"ur Studenten dieses
Kurses nicht wirklich durchf"uhrbar. Trotzdem ist unten ein Bewertungsschema
angegben, auch wenn die letzten zwei Punkte praktisch unerreichbar waren.

\medskip

Wir setzen den Separationsansatz $u(x,y)=X(x)Y(y)$ in die
Differentialgleichung ein:
\[
x^2X'(x)Y(y)+y^2Y'(y)Y(y)=0.
\]
Bringt man alle Terme mit $x$ auf die linke Seite, und alle Terme
mit $y$ auf die rechte, bekommt man
\[
x^2\frac{X'(x)}{X(x)}=- y^2\frac{Y'(y)}{Y(y)}.
\]
Da die linke Seite nur von $x$, die rechte aber nur von $y$ abh"angt,
m"ussen beide Seiten konstant sein, wir nenne die Konstante $\mu$.
So entstehen die beiden Differentialgleichungen
\begin{align*}
x^2\frac{X'(x)}{X(x)}&=\mu
&
\Rightarrow\;
\frac{d}{dx}\log X(x)&=\frac{\mu}{x^2}
&
\Rightarrow\;
\log X(x)&=-\frac{\mu}{x}
&
\Rightarrow\;
X(x)&=Ae^{-\frac{\mu}{x}}
\\
y^2\frac{Y'(y)}{Y(y)}&=-\mu
&
\Rightarrow\;
\frac{d}{dy}\log Y(y)&=-\frac{\mu}{y^2}
&
\Rightarrow\;
\log Y(y)&=\frac{\mu}{y}
&
\Rightarrow\;
Y(y)&=Be^{\frac{\mu}{y}},
\end{align*}
und damit die Teill"osungen
\[
u_\mu(x,y)=e^{\mu\bigl(-\frac1x+\frac1y\bigr)}.
\]
Aus diesen Teill"osungen muss jetzt die L"osung so linear kombiniert
werden, dass die Anfangsbedingung erf"ullt wird. Die Linearkombination
wird in diesem Fall ein Integral:
\[
u(x,y)=\int_{\mathbb R}C(\mu)
e^{\mu\bigl(-\frac1x+\frac1y\bigr)}\,d\mu,
\]
die Funktion $C(\mu)$ ist noch zu bestimmen.
Insbesondere muss gelten
\[
u(x,-1)
=
\frac1x
=
\int_{\mathbb R}C(\mu)e^{\mu\bigl(-\frac1x-1)}\,d\mu
=
\int_{\mathbb R}C(\mu)e^{-\mu\bigl(\frac1x+1)}\,d\mu
\]
Schreiben wir $\frac1x+1=s$ und $t=\mu$ wird die Bedingung
zu
\[
\int_{\mathbb R}C(t)e^{-ts}\,dt=s-1.
\]
Das Integral links ist die zweiseitige Laplace-Transformation
(im Bachelor-Studium lernt man normalerweise nur die einseitige
Laplace-Transformation kennen). Die Terme rechts haben bekannte
zweiseitige Laplace-Transformationen:
\begin{align*}
\int_{\mathbb R}-\delta(t)e^{-st}\,dt&=-1\\
\int_{\mathbb R}-\delta'(t)e^{-st}\,dt&=s
\end{align*}
Man muss also $C(t)=-\delta'(t)-\delta(t)$ verwenden, daraus bekommt
man jetzt die Funktion $u(x,y)$ durch Integration mit den Eigenschaften
von $\delta(t)$ und $\delta'(t)$:
\begin{align*}
u(x,y)
&=
\int_{\mathbb R}
(-\delta'(\mu)-\delta(\mu))
e^{-\mu\bigl(\frac1x-\frac1y\bigr)}\,d\mu
\\
&=
-\left.\frac{\partial}{\partial \mu}
e^{-\mu\bigl(\frac1x-\frac1y\bigr)}\right|_{\mu=0}
-
\left.e^{-\mu\bigl(\frac1x-\frac1y\bigr)}\right|_{\mu=0}
\\
&=
\frac1x-\frac1y-1,
\end{align*}
also das gleiche Resultat wie vorhin.
\end{loesung}
