L"osen Sie die Differentialgleichung
\[
\frac{\partial u}{\partial x}+y\frac{\partial u}{\partial y}=0
\]
auf dem Gebiet $\Omega=\{(x,y)\,|\, x> 0\}$ mit der Randbedingung
\[
u(0,y)=g(y)=y^2.
\]

\begin{loesung}
Dies ist eine quasilineare partielle Differentialgleichung
erster Ordnung, die mit Hilfe von Charakteristiken gel"ost werden
kann.
Die Anfangskurve ist entlang der $y$-Achse vorgegeben, wir bezeichnen
die zugeh"origen $y$-Werte mit $y_0$.
Das Differentialgleichungssytem f"ur die  Charaketersistiken ist
\[
\left.
\begin{aligned}
\dot x&=1\\
\dot y&=y\\
\dot u&=0
\end{aligned}
\right\}\quad\Rightarrow\quad
\begin{aligned}
x(y_0,t)&=t+C\\
y(y_0,t)&=De^t\\
u(y_0,t)&=u_0
\end{aligned}
\]
Die Anfangsbedingungen verlangen, dass
\begin{align*}
x(y_0,0)&=0&\Rightarrow&&C&=0\\
y(y_0,0)&=y_0&\Rightarrow&&D&=y_0\\
u(y_0,0)&=y_0^2&\Rightarrow&&u_0&=y_0^2
\end{align*}
Jetzt m"ussen wir die Variablen $y_0$ und $t$ aus den Gleichungen
\begin{align*}
x&=t\\
y&=y_0e^t\\
u&=y_0^2
\end{align*}
eliminieren. Setzt man die erste und die letzte Gleichung in die
quadrierte zweite ein, erh"alt man
\[
y^2=ue^{2x}\qquad\Rightarrow\quad u=y^2e^{-2x}.
\]

Alternative L"osung mit einem Separationsansatz: Man kann die DGL auch
mit einem Separationsansatz l"osen. Dazu schreibt man $u(x,y)=X(x)Y(y)$
und setzt in die Differentialgleichung ein:
\begin{align*}
X'(x)Y(y)+yX(x)Y(y)&=0\\
\Rightarrow\qquad \frac{X'(x)}{X(x)}&=-y\frac{Y'(y)}{Y(y)}
\end{align*}
Da die eine Seite nur von $x$, die andere nur von $y$ abh"angt, m"ussen
beide Seiten konstant sein, so erh"alt man zwei gew"ohnliche
Differentialgleichungen
\begin{align*}
X'(x)&=cX(x)&\Rightarrow\qquad X(x)&=K_1e^{cx}\\
Y'(y)&=\frac{c}{y}Y(y)&\Rightarrow\qquad Y(y)&=K_2y^{-c}
\end{align*}
Die zweite Gleichung wurde mit Hilfe von Separation gel"ost.
Die Anfangsbedingung f"ur $x=0$ kann offenbar erf"ullt werden,
wenn $c=-2$, dann ist die L"osung gefundene Teill"osung
\[
u(x,y)=Ke^{-2x}y^2,
\]
die Anfangsbedinung f"ur $x$ ist aber nur dann erf"ullt, wenn
\[
u(0,y_0)=Ky_0^2=g(y_0)\quad\Rightarrow\quad K=1,
\]
also ist die endg"ultige L"osung wie bei der ersten L"osung
\[
u(x,y)=e^{-2x}y^2
\qedhere
\]
\end{loesung}
