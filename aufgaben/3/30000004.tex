L"osen Sie die Differentialgleichung
\[
\frac{\partial u}{\partial x}+y\frac{\partial u}{\partial y}=0
\]
auf dem Gebiet $\Omega=\{(x,y)\,|\, x\ge 0\}$ mit der Randbedingung
\[
u(0,y)=g(y)=y^2.
\]

\begin{loesung}
Dies ist eine quasilineare partielle Differentialgleichung
erster Ordnung, die mit Hilfe von Charakteristiken gel"ost werden
kann. Das Differentialgleichungssytem f"ur die  Charaketersistiken ist
\[
\left.
\begin{aligned}
\dot x&=1\\
\dot y&=y\\
\dot u&=0
\end{aligned}
\right\}\quad\Rightarrow\quad
\begin{aligned}
x(t)&=t+C\\
y(t)&=y_0e^t\\
u(t)&=u_0
\end{aligned}
\]
Wir k"onnen $C=0$ w"ahlen, also $x=t$, und erhalten damit die
Charakteristiken in expliziter Form $y(x)=y_0e^x$ und $u(x)=u_0$.

Jetzt muss $u$ als Funktion von $x$ und $y$ ermittelt werden.
Durch den Punkt $(x,y)$ in der Ebene f"uhrt eine Charakteristik,
die im Punkt $(0,y_0)$ begonnen hat, wegen $y(x)=y_0e^x$ ist
$y_0=ye^{-x}$. F"ur $u$ gilt jetzt
\[
u(x,y)=u(0,y_0)=g(y_0)=g(ye^{-x})=y^2e^{-2x}=y^2e^{-2x}.
\]
Zur Kontrolle setzen wir diese Funktion in die urspr"ungliche
Differentialgleichung ein:
\begin{align*}
\frac{\partial u}{\partial x}
&=
-2y^2e^{-2x}
\\
\frac{\partial u}{\partial y}
&=
2ye^{-2x}
\\
\frac{\partial u}{\partial x}
+y
\frac{\partial u}{\partial y}
&=
-2y^2e^{-x}
+2y^2e^{-x}
=0,
\end{align*}
die Funktion $u$ erf"ullt also die Differentialgleichung. Sie erf"ullt
aber auch die Randbedingung, denn
\[
u(0,y_0)=y_0^2e^0=y_0^2=g(y_0).
\]
Damit ist $u(x,y)=y^2e^{-2x}$ die L"osung der gegebenen Differentialgleichung.
Alternative L"osung mit einem Separationsansatz: Man kann die DGL auch
mit einem Separationsansatz l"osen. Dazu schreibt man $u(x,y)=X(x)Y(y)$
und setzt in die Differentialgleichung ein:
\begin{align*}
X'(x)Y(y)+yX(x)Y(y)&=0\\
\Rightarrow\qquad \frac{X'(x)}{X(x)}&=-y\frac{Y'(y)}{Y(y)}
\end{align*}
Da die eine Seite nur von $x$, die andere nur von $y$ abh"angt, m"ussen
beide Seiten konstant sein, so erh"alt man zwei gew"ohnliche
Differentialgleichungen
\begin{align*}
X'(x)&=cX(x)&\Rightarrow\qquad X(x)&=K_1e^{cx}\\
Y'(y)&=\frac{c}{y}Y(y)&\Rightarrow\qquad Y(y)&=K_2y^{-c}
\end{align*}
Die zweite Gleichung wurde mit Hilfe von Separation gel"ost.
Die Anfangsbedingung f"ur $x=0$ kann offenbar erf"ullt werden,
wenn $c=-2$, dann ist die L"osung gefundene Teill"osung
\[
u(x,y)=Ke^{-2x}y^2,
\]
die Anfangsbedinung f"ur $x$ ist aber nur dann erf"ullt, wenn
\[
u(0,y_0)=Ky_0^2=g(y_0)\quad\Rightarrow\quad K=1,
\]
also ist die endg"ultige L"osung wie bei der ersten L"osung
\[
u(x,y)=e^{-2x}y^2
\]
\end{loesung}
