Auf dem Gebiet $\Omega=\{(x,y)\,|\,0<x<\frac{\pi}2,0<y<\frac{\pi}2\}$
ist die partielle Differentialgleichung
\begin{equation}
\frac{\partial u}{\partial x}
+
\frac1{\sin y}\frac{\partial u}{\partial y}
=
0
\label{30000011:equation}
\end{equation}
gegeben.
\begin{teilaufgaben}
\item
Finden Sie eine Lösung, die die Randbedingung
\[
u(x,0)=\sin x\qquad \text{für $0<x<\frac{\pi}2$}
\]
erfüllt.
\item
Ist die Lösung durch diese Randbedingung eindeutig festgelegt?
\end{teilaufgaben}

\begin{loesung}
Es handelt sich um eine quasilineare partielle Differentialgleichung erster
Ordnung, die gestellten Fragen können mit Hilfe von Charakteristiken
beantwortet werden.
\begin{teilaufgaben}
\item
Die Differentialgleichungen der Charakteristiken mit Parameter $t$ sind
\begin{align}
\dot x&=1
\label{30000011:x}
\\
\dot y&=\frac1{\sin x}
\label{30000011:y}
\\
\dot u&=0
\label{30000011:u}
\end{align}
Die Gleichungen \eqref{30000011:x} und \eqref{30000011:u} sind durch
Integration sofort zu lösen:
\begin{align*}
x&=t+x_0\\
u&=u_0
\end{align*}
Die Gleichung \eqref{30000011:y} kann durch Separation gelöst werden:
\[
\begin{aligned}
\dot y&=\frac1{\sin y}
&&\Rightarrow&
\int \sin y\,dy&=-\cos y=t+y_0
\end{aligned}
\]
Darin muss die Integrationskonstante $y_0$ so gewählt werden,
dass die Charakteristik für $t=0$ auf der $x$-Achse beginnen soll.
Setzen wir $y=0$ und $t=0$ ein, erhalten wir
\[
-\cos 0=0+y_0,
\]
wir müssen also $y_0=-1$ setzen.
Die Randbedingung liefert ausserdem $u_0=\sin x_0$. So
erhalten wir die Gleichungen
\begin{align*}
x&=t+x_0\\
\cos y&=-t+1\\
u&=\sin x_0.
\end{align*}
Aus diesen drei Gleichungen müssen wir nun noch $t$ und $x_0$ eliminieren.
Indem wir die zweite Gleichung nach $t=-\cos y + 1$ auflösen und
in der ersten einsetzen, können wir $t$ eliminieren und erhalten
\[
x=-\cos y+1+x_0.
\]
Dies können wir nach $x_0=x+\cos y-1$ auflösen und in die letzte Gleichung
einsetzen, so wird $x_0$ eliminiert:
\[
u(x,y)=\sin(x+\cos y - 1).
\]
Zur Kontrolle berechnen wir die Ableitungen
\begin{align*}
\frac{\partial u}{\partial x}
&=
\cos(x+\cos y - 1)
\\
\frac{\partial u}{\partial y}
&=
-\cos(x+\cos y - 1)
\sin y
\end{align*}
und setzen in die Differentialgleichung ein
\[
\frac{\partial u}{\partial x}
+
\frac1{\sin y}\frac{\partial u}{\partial y}
=
\cos(x+\cos y - 1)
-
\frac1{\sin y}
\cos(x+\cos y - 1)
\sin y
=
0,
\]
die Funktion $u(x,y)$ erfüllt also die Differentialgleichung. 
Die Randbedingung für $y=0$ ist $u(x_0,0)=\sin(x_0+\cos 0 - 1)=\sin x_0$
ist ebenfalls erfüllt.
\begin{figure}
\centering
\includeagraphics[width=0.6\hsize]{loes.jpg}
\caption{Lösungsfläche der partiellen Differentialgleichung
\eqref{30000011:equation}.
\label{30000011:solution}}
\end{figure}
Die Lösungsfläche ist in Abbildung~\ref{30000011:solution} dargestellt.
\item
Der Grundriss der Charakteristik durch den Punkt $(x_0,0)$ erfüllt
\[
\begin{aligned}
\cos y&=-(x-x_0)+1
&&\Rightarrow&
y&=\arccos(-x+x_0+1),
\end{aligned}
\]
also gespiegelte Arcus-Kosinus-Kurven (siehe Abbildung~\ref{30000011:char}).
Diese Kurven sind im gegebenen Gebiet $\Omega$ alle nach rechts gekrümmt,
der Teil des Gebietes oberhalb der Charakteristik durch $(0,0)$ wird also
nicht überdeckt, die Randbedingung legt dort die Lösung nicht fest.
\qedhere
\begin{figure}
\centering
\includeagraphics[]{domain-1.pdf}
\caption{Charakteristiken der Differentialgleichung \eqref{30000011:equation}.
\label{30000011:char}}
\end{figure}
\end{teilaufgaben}
\end{loesung}

\begin{bewertung}
Quasilineare PDGL ({\bf Q}) 1 Punkt,
Differentialgleichungen der Charakteristiken ({\bf D}) 1 Punkt,
Lösung der Differentialgleichungen für $x(t)$ ({\bf X})
und $y(t)$ ({\bf Y}) je 1 Punkt,
Lösung der partiellen Differentialgleichung ({\bf U}) 1 Punkt,
Verwendung der Charakteristikenmethode zur Beurteilung ob
vorgegebenen Randwerte die Lösung eindeutig bestimmen ({\bf B}) 1 Punkt.
\end{bewertung}

