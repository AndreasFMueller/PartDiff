Consider the partial differential equation
\[
\frac{\partial u}{\partial x} + \frac{\partial u}{\partial y}
=
u^2
\]
on the domain $\Omega = \{ (x,y)\in\mathbb R^2\,|\, x>0\}$, with
boundary values
\[
u(0,y) = \frac{1}{1+y^2}.
\]
\begin{teilaufgaben}
\item
Is the solution of this equation uniquely determined close to the
$y$-axis?
\item
Find a solution $u(x,y)$, possibly on a smaller domain than $\Omega$.
\item
Find the domain $\Omega_0\subset\Omega$ on which your solution is defined.
\end{teilaufgaben}

\begin{loesung}
\begin{figure}
\centering
\includeagraphics[]{domain.pdf}
\caption{Domain on which the solution \eqref{30000019:solution} is
well defined.
The green lines are characteristics.
\label{30000019:figure}}
\end{figure}%
This is a nonlinear but quasilinear partial differential equation of first
order, it can be solved using the the method of characteristics.
The ordinary differential equations
\begin{align*}
\dot{x} &= 1 &&\Rightarrow& x&=s + x_0 = s\\
\dot{y} &= 1 &&\Rightarrow& y&=s + y_0 = x + y_0&&\Rightarrow&y_0&=y-x\\
\dot{u} &= u^2
\end{align*}
\begin{teilaufgaben}
\item
To decide whether the problem is well posed we only have to know whether
the characteristics cover the whole domain.
But the characteristics are straight lines of slope $1$, these lines
cover a neighborhood of the $y$-axis, so we can conclude that the
solution is unique there, as long as the solution of the third equation 
exists.
\item
To find the solution, we need to solve the last equation.
This can be done using the method of separation of variables.
The equation is equivalent to
\begin{align*}
\int\frac{du}{u^2} &= \int \,ds
&&\Rightarrow&
-\frac1{u} &= s + C
&&\Rightarrow&
u = -\frac{1}{s+C} = -\frac{1}{s-1/u_0} = -\frac{1}{s-(1+y_0^2)}.
\end{align*}
To find the solution, we have to eliminate $y_0$ from these equations.
But from the second equation we get $y_0=y-x$ so the solution becomes
\begin{equation}
u = -\frac{1}{x-(1+(y-x)^2)}.
\label{30000019:solution}
\end{equation}
We see the solution of the equation blows up eventually.
The blow-up happens later if the inital value is smaller.
\item
The solution is only defined as long as the denominator in
\eqref{30000019:solution}
does not vanish.
By considering the solution along the characteristic, we can consider $y_0$
to be a constant and can decide how much we can increase $x$ until the
solution blows up.
This happens when 
\[
0=x-(1+y_0^2)
\qquad\Rightarrow\qquad
x=1+y_0^2.
\]
So the right hand boundary of the domain is parametrized by
\[
y_0 \mapsto (1+y_0^2,y_0+1+y_0^2).
\]
This is the red parabola in figure~\ref{30000019:figure}.
Alternatively we can write the curve as 
\[
y=
x+
\pm
\sqrt{x-1}
\]
for $x\ge 0$.
\qedhere
\end{teilaufgaben}
\end{loesung}

\begin{bewertung}
\begin{teilaufgaben}
\item
Equations for the characteristics ({\bf E}) 1 point,
solutions for the equations ({\bf S}) 1 point,
decision regarding well posedness close to the axis ({\bf W}) 1 point.
\item
Solution for $u$ ({\bf U}) 1 point,
use initial condition ({\bf I}) 1 point.
\item
Find the domain ({\bf $\Omega_0$}) 1 point.
\end{teilaufgaben}
\end{bewertung}
