% quaslinieares problem
Consider the partial differential equation
\begin{equation}
\frac{\partial u}{\partial x} + \frac{\partial u}{\partial y} = \frac{1}{3u^2}
\label{30000021:eqn}
\end{equation}
in the domain $\Omega = \{(x,y)\in\mathbb R^2\;|\; y<-x\}$. 
Find a solution that satisfies the boundary condition $u(x,y)=\sin x$ on
$\partial\Omega$.

\begin{loesung}
\begin{figure}
\centering
\definecolor{darkgreen}{rgb}{0,0.6,0}
\begin{tikzpicture}[>=latex]
\fill[color=red!10] (-4,4)--(4,-4)--(-4,-4)--cycle;
\begin{scope}
\clip (-4,-4) rectangle (4,4);
\foreach \x in {-3,...,3}{
	\draw[color=darkgreen,line width=1.4pt] (\x,{-\x})--({\x-5},{-\x-5});
}
\end{scope}
\draw[color=red,line width=1.4pt] (-4,4)--(4,-4);
\draw[->] (-4.1,0)--(4.3,0) coordinate[label={$x$}];
\draw[->] (0,-4.1)--(0,4.3) coordinate[label={right:$y$}];
\node at (-2,-2) [above left] {$\Omega$};
\node at (-2,2) [above right] {$\partial\Omega$};
\end{tikzpicture}
\caption{Domain of definition for the partial differential equation
\eqref{30000021:eqn}
\label{3000021:domain}}
\end{figure}
\begin{figure}
\centering
\begin{tikzpicture}[>=latex,thick]
\node at (0,0) {\includeagraphics[width=\hsize]{solution.jpg}};
\node at (-8.3,-1) {$x$};
\node at (0.2,-2.5) {$y$};
\node at (-0.6,2.7) {$z$};
\end{tikzpicture}
\caption{Solution surface for problem~\ref{30000021}.
\label{30000021:graph}}
\end{figure}
This is a quasidifferential equation of first order which can be solved
using the method of characteristics.
The ordinary differential equations for the characteristics and their
solutions are
\begin{equation}
\left.
\begin{aligned}
\dot x &= 1
\\
\dot y &= 1
\\
\dot u &= \frac1{3u^2}
\end{aligned}
\;
\right\}
\quad\Rightarrow\quad
\left\{
\;
\begin{aligned}
x(t) &= t + x_0
\\
y(t) &= t + y_0
\\
u(t) &=\sqrt[3]{t+u_0^3}
\end{aligned}
\right.
\label{30000020:char}
\end{equation}
Only the differential equation for $u$ is a little more complicated,
but it can easily be solved using separation:
\[
\dot{u}=\frac1{3u^2}
\quad\Rightarrow\quad
3u^2\dot u = 1
\quad\Rightarrow\quad
\frac{d}{dt}u^3=1
\quad\Rightarrow\quad
u^3 = t+C.
\]
Since $u$ must take the value $u_0=\sin^3x_0$ for $t=0$,
the value $C=u_0^3=\sin^3x_0$ has to be chosen.

The integration constants $x_0$, $y_0$ and $u_0$ must now determined
using the boundary conditions.
Points on the boundary are of the form $(x_0,-x_0)$, so $y_0=-x_0$.
The first two equations in \eqref{30000020:char} give
\begin{align*}
x-y &= t+x_0 - (t -x_0) = 2x_0
&&\Rightarrow&
x_0 &= \frac{x-y}2,
\\
x+y &= t+x_0 +(t-x_0) = 2t
&&\Rightarrow&
t&=\frac{x+y}2.
\end{align*}
Using $u_0 = \sin^3 x_0$, we can write the solution:
\[
u(x,y)
=
\sqrt[3]{t+u_0^2}
=
\sqrt[3]{\frac{x+y}2 + \sin^3\frac{x-y}2}.
\]
We verify this result and start with the boundary condition:
\[
u(x_0,-x_0)
= 
\sqrt[3]{\frac{x_0-x_0}2 + \sin^3\frac{x_0-(-x_0)}2}
=
\sin x_0.
\]
The partial differential equation also holds:
\begin{align*}
\frac{\partial u}{\partial x}
&=
\frac{1}{u(x,y)^2} \frac13
\frac{\partial}{\partial x} \biggl(
\frac{x+y}2 + \sin^3\frac{x-y}2
\biggr)
=
\frac{1}{3u(x,y)^2} \biggl(
\frac12 + 3\sin^2\frac{x-y}2\cos\frac{x-y}2
\biggr)
\\
\frac{\partial u}{\partial y}
&=
\frac{1}{u(x,y)^2} \frac13
\frac{\partial}{\partial y} \biggl(
\frac{x+y}2 + \sin^3\frac{x-y}2
\biggr)
=
\frac{1}{3u(x,y)^2} \biggl(
\frac12 - 3\sin^2\frac{x-y}2\cos\frac{x-y}2
\biggr)
\\
\Rightarrow\qquad
\frac{\partial u}{\partial x} + \frac{\partial u}{\partial y}
&=
\frac{1}{3u^2}
\end{align*}
in $\Omega$.
\end{loesung}

\begin{bewertung}
Differential equation for characteristiscs ({\bf D}) 1 point,
solutions for $x(t)$ and $y(t)$ ({\bf XY}) 1 point,
solution for $u(t)$ ({\bf U}) 1 point,
boundary or $x_0$-$y_0$-dependence ({\bf B}) 1 point,
boundary condition ({\bf C}) 1 point,
solution ({\bf S}) 1 point.
\end{bewertung}
