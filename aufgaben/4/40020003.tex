Die Hamilton-Funktion f"ur ein Federpendel mit Masse $m$ und Federkonstante $k$
ist
\[
H=\frac1{2m}p^2+\frac{k}2x^2.
\]
Der erste Term ist die kinetische Energie ($p=mv\;\Rightarrow\;p^2/2m=\frac12mv^2$),
der zweite die in der Feder gespeicherte potentielle Energie.
Die Hamilton-Jacobi-Theorie erm"oglicht,
die f"ur dieses Problem optimalen Koordinaten $Q$, $P$ zu finden.
\begin{teilaufgaben}
\item Stellen Sie die Hamilton-Jacobi-Differentialgleichung auf ({\it Hinweis:}
Konsultieren Sie die Pr"asentation ``Separation'')
\item F"uhren Sie die Separation f"ur die Hamilton-Jacobi-Differentialgleichung
durch. ({\it Hinweis:} Verwenden Sie einen Summen-Ansatz)
\item L"osen Sie die gew"ohnlichen Differentialgleichungen.
({\it Hinweis:} rechnen Sie
die ``komplizierten'' Integrale noch nicht aus, sie werden sp"ater wieder
abgeleitet, und sind dann einfacher auszuwerten.)
\item Finden Sie die neue Koordinate $Q$, l"osen Sie sie nach $t$
auf. ({\it Hinweis:} Pr"asentation)
\item Versuchen Sie, die verallgemeinerten Koordinaten $Q$ und $P$ physikalisch
zu interpretieren.
\end{teilaufgaben}
Mit dieser Beschreibung des Federpendels k"onnte in einem n"achsten, jedoch f"ur
diese Vorlesung nicht relevanten Schritt der Einfluss einer Nichtlinearit"at
der Feder auf Frequenz und Schwingungsform mit Hilfe der St"orungstheorie
berechnet werden.

\begin{loesung}
\begin{teilaufgaben}
\item Gesucht wird eine Funktion $S(x,P,t)$, aus der sich sp"ater die
Koordinaten berechnen lassen:
\begin{align*}
p&=\frac{\partial S}{\partial x} & Q=\frac{\partial S}{\partial P}.
\end{align*}
Sie erf"ullt die Differentialgleichung
\begin{equation}
\frac1{2m}\biggl(\frac{\partial S}{\partial x}\biggr)^2+\frac{k}2x^2
+\frac{\partial S}{\partial t}=0
\label{40020003:hjdgl}
\end{equation}
\item
Wir verwenden einen Separationsansatz $S=S_1(x)+S_2(t)$, denn $x$ und $t$
sind die einzigen Variablen, nach denen in (\ref{40020003:hjdgl})
abgeleitet wird.
Setzt man diesen Ansatz in die Differentialgleichung (\ref{40020003:hjdgl}) ein, bekommt man
\begin{align*}
\frac1{2m}\biggl(\frac{dS_1(x)}{dx}\biggr)^2+\frac{k}2x^2+\frac{dS_2(t)}{dt}&=0
\\
\frac1{2m}\biggl(\frac{dS_1(x)}{dx}\biggr)^2+\frac{k}2x^2&=-\frac{dS_2(t)}{dt}
\end{align*}
Die linke Seite h"angt nur von $x$ ab, die rechte nur von $t$, beide m"ussen
daher konstant sein. Wir bezeichnen diese Konstante mit $P$ und erhalten
die separierten gew"ohnlichen Differentialgleichungen:
\begin{align*}
-\frac{dS_2(t)}{dt}&=P&
&\Rightarrow&
S_2(t)&=-Pt\\
\frac{dS_1(x)}{dx}&=\sqrt{m(2P-kx^2)}&
&\Rightarrow&
S_1(x)&=\int\sqrt{m(2P-kx^2)}\,dx
%\\
%&&&&&=
%\frac12x\sqrt{P-mkx^2}+\frac{P}{2\sqrt{km}}\operatorname{arc}\tan
%\biggl(
%\frac{x\sqrt{km}}{\sqrt{P-kmx^2}}
%\biggr)
\end{align*}
\item
Jetzt kann man durch Ableiten nach $P$ die neuen Koordinaten finden:
\begin{align*}
Q=\frac{\partial S}{\partial P}
&=\frac{\partial S_1(x)}{\partial P}+\frac{\partial S_2(t)}{\partial P}
\notag
\\
&=\frac{\partial}{\partial P}\int\sqrt{m(2P-kx^2)}\,dx+\frac{\partial}{\partial P}(-Pt)
\notag
\\
&=\int \frac{m}{\sqrt{m(2P-kx^2)}} \,dx-t
\label{40020003:integral}
\\
&=\sqrt{\frac{m}{k}}\operatorname{arc}\tan\frac{\sqrt{k}\,x}{\sqrt{2P-kx^2}} -t
\notag
\end{align*}
Das Integral (\ref{40020003:integral}) entnimmt man einer Formelsammlung oder
berechnet es mit Hilfe eines Computer-Algebra-Systems.
\item
Dies kann man jetzt nach $x$ aufl"osen, wobei wir zur Vereinfachung der
Formeln $y=\sqrt{k/2P}\,x$ schreiben:
\begin{align*}
\tan\biggl(\sqrt{\frac{k}{m}}(t+Q)\biggr)
%&
=\frac{\sqrt{k}x}{\sqrt{2P-kx^2}}
%\\
%&
=\frac{\sqrt{\frac{k}{2P}}x}{\sqrt{1-\frac{k}{2P}x^2}}
%\\
%&
=\frac{y}{1-y^2}
\end{align*}
Nun gilt aber, wie man zum Beispiel in der Trigonometrie-Formelsammlung auf der
Wikipedia findet:
\[
\tan \varphi=\frac{\sin \varphi}{\sqrt{1-\sin^2 \varphi}},
\]
wobei in unserem Fall $\varphi=\sqrt{\frac{k}{m}}(t+Q)$.
Also folgt $y=\sin \varphi$ oder
\[
\sqrt{\frac{k}{2P}}\,x
=\sin\biggl( \sqrt{\frac{k}{m}}(t+Q)\biggr)
\quad\Rightarrow\quad
x=\sqrt{\frac{2P}{k}}\sin\biggl( \sqrt{\frac{k}{m}}(t+Q)\biggr)
\]
Die Auslenkung folgt also einer Sinusschwingung mit Kreisfrequenz $\sqrt{k/m}$.
\item
Jetzt kann man die Bedeutung der Koordinaten $Q$ und $P$ wieder interpretieren.
$Q$ ist die Nulldurchgangszeit. Der Koeffizient der Sinus-Funktion ist 
die maximale Auslenkung $x_{\text{max}}$, also ist
\[
x_{\text{max}}=\sqrt{\frac{2P}{k}}
\qquad\Rightarrow\qquad
P=\frac12kx_{\text{max}}^2,
\]
$P$ hat also die Bedeutung der Energie.
\end{teilaufgaben}
\end{loesung}
