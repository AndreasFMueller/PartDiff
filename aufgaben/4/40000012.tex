Auf dem Gebiet $\Omega=\{(x,y)\;|\;0<x,0 < y, 1 < x^2+y^2< 4\}$
ist die partielle Differentialgleichung
\begin{equation}
\frac{\partial^2 u}{\partial x^2}+\frac{\partial^2u}{\partial y^2}=\lambda u
\label{40000012:gleichung}
\end{equation}
gegeben.
\begin{teilaufgaben}
\item
F"uhren Sie Separation in Polarkoordinaten durch.
\item
Finden Sie f"ur $\lambda=0$ eine L"osung von
(\ref{40000012:gleichung}), die den Randbedingungen
\begin{align*}
u(x,0)&= 0&u(x,y)&=\sin 2\varphi \qquad \text{f"ur $x^2+y^2=1$ oder $4$}\\
u(0,y)&= 0&      &
\end{align*}
gen"ugt.
Darin ist $\varphi$ der Polarwinkel, es gilt $\tan\varphi=\frac{y}{x}$.
\item
Wo sind die Maxima der in Teilaufgabe b) gefundenen Funktion $u(x,y)$?
\end{teilaufgaben}

\begin{hinweis}
Suchen Sie Radialfunktionen $R(r)$ in der Form $R(r)=r^{\alpha}$.
\end{hinweis}

\begin{loesung}
\begin{figure}
\centering
\includeagraphics[]{domain-1.pdf}
\caption{Definitionsgebiet $\Omega$ der Differentialgleichung
(\ref{40000012:gleichung})
\label{40000012:gebiet}
}
\end{figure}
Das Gebiet ist in Abbildung~\ref{40000012:gebiet} dargestellt.
Wir arbeiten in Polarkoordinaten, da l"asst sich der Laplace-Operator als
\[
\Delta u
=
\frac1r\frac{\partial}{\partial r}r\frac{\partial}{\partial r}
+
\frac1{r^2}
\frac{\partial^2}{\partial\varphi^2}
\]
ausdr"ucken.
\begin{teilaufgaben}
\item
Wir setzen einen Separationsansatz der Form $u(r,\varphi)=R(r)\Phi(\varphi)$
in die Differentialgleichung ein und erhalten
\begin{align*}
\frac1r\frac{d}{dr}\bigl(rR'(r)\bigr)\Phi(\varphi)
+
\frac1{r^2}
\Phi''(\varphi)R(r)
&=
\lambda R(r)\Phi(\varphi),
\\
\biggl(\frac1rR'(r)+R''(r)\biggr)\Phi(\varphi)
- \lambda R(r)\Phi(\varphi)
&=
-
\frac1{r^2}
R(r)\Phi''(\varphi),
\\
\frac{r^2R''(r)+rR'(r)-\lambda r^2 R(r)}{R(r)}
&=
-\frac{\Phi''(\varphi)}{\Phi(\varphi)}.
\end{align*}
Da die linke Seite nur von $r$, die rechte aber nur von $\varphi$ abh"angt,
m"ussen beide Seiten konstant sein und wir
erhalten die zwei Gleichungen
\begin{align}
r^2R''(r)+rR'(r)-\lambda r^2 R(r)&=\phantom{-}\mu R(r)
\label{40000012:erste}
\\
\Phi''(\varphi)&=-\mu\Phi(\varphi)
\label{40000012:zweite}
\end{align}
mit einer noch zu bestimmenden Konstanten $\mu$.
\item
Die Funktionen $\Phi(\varphi)$ m"ussen die Randbedingungen
\[
\Phi(0)=\Phi({\textstyle\frac{\pi}2})=0
\]
erf"ullen.
Die Randbedingungen auf den $x$- und $y$-Achsen verlangen, dass f"ur $\Phi$
oszillierende L"osungen verwendet werden m"ussen, was nur m"oglich ist,
wenn $\mu>0$ ist. Wir setzen daher $\mu=m^2$. Damit bekommen wir f"ur die
$\Phi(\varphi)$
\[
\Phi(\varphi)=\begin{cases}\sin m\varphi,\\\cos m\varphi.\end{cases}
\]
Die Kosinus-L"osung erf"ullt die Randbedingung auf der $x$-Achse nicht,
und die Sinus-L"osung erf"ullt die Randbedingung auf der $y$-Achse nur,
wenn $m=2k$ mit $k\in\mathbb Z$ ist.
Wir haben also die Teill"osungen
\[
\Phi_k(\varphi)=\sin 2k\varphi, \quad\varphi\in[0,{\textstyle \frac{\pi}2}]
\]
gefunden.

F"ur jedes $k$ m"ussen jetzt noch Funktionen $R_k(r)$ gefunden werden.
Man kann aber jetzt schon sagen, dass wegen der speziellen Randbedingungen
auf den kreisf"ormigen Teilen des Randes nur die L"osung $k=1$ ben"otigt wird.

\begin{figure}
\centering
\includeagraphics[width=0.8\hsize]{loes.jpg}
\caption{3D-Darstellung der L"osungsfunktion
\label{40000012:3d}}
\end{figure}
Aufgrund des Hinweises versuchen wir die Radialfunktion in der Form
$R(r)=r^\alpha$, und setzen dies in (\ref{40000012:erste}) ein:
\begin{align*}
r^2\alpha(\alpha-1)r^{\alpha-2}+r\alpha r^{\alpha-1}&=4k^2r^{\alpha}
\\
(\alpha(\alpha-1)+\alpha -4k^2)r^{\alpha}&=0
\\
(\alpha^2 -4k^2)r^{\alpha}&=0
\\
\alpha^2&=4k^2
\\
\alpha&=\pm2k
\end{align*}
Wir wissen bereits, dass nur der Fall $k=1$ ben"otigt wird, und suchen
daher Koeffizienten $A$ und $B$ derart, dass
\[
Ar^2+Br^{-2}=1\qquad \text{f"ur $r=1,2$.}
\]
Dies f"uhrt auf das Gleichungssystem
\[
\begin{linsys}{3}
 A&+&       B&=&1\\
4A&+&\frac14B&=&1
\end{linsys}
\]
mit den L"osungen
\[
A=\frac15,\qquad B=\frac45.
\]
Die gesuchte L"osung der Differentialgleichung ist also
\[
u(r,\varphi)=\biggl(\frac15r^2+\frac45r^{-2}\biggr)\sin 2\varphi.
\]
Die Abbildung~\ref{40000012:3d} zeigt eine 3D-Darstellung der L"osungsfunktion.
Der folgende Maxima-Code verifiziert, dass dies die korrekte L"osung ist.
\verbatimainput{check.maxima}
\item
Die Funktion $u$ ist wegen $\lambda=0$ eine harmonische Funktion, ihre
Maxima und Minima sind daher auf dem Rand.
Es reicht daher, die Randbedingungen zu konsultieren, wir finden die
Maxima in den Punkten
\[
\biggl(\frac{\sqrt{2}}{2},\frac{\sqrt{2}}2\biggr),\qquad
(\sqrt{2},\sqrt{2}).
\qedhere
\]
\end{teilaufgaben}
\end{loesung}

\begin{bewertung}
Differentialgleichung f"ur $R$ ({\bf R}) 1 Punkt,
Differentialgleichung f"ur $\Phi$ ({$\mathbf \Phi$}) 1 Punkt,
L"osung der Gleichung f"ur $\phi$ ($\textbf{L}_\Phi$) 1 Punkt,
L"osung der Gleichung f"ur $R$ mit dem Potenzansatz ($\textbf{L}_R$) 1 Punkt,
Einsetzen der Randwerte/Koeffizientenvergleich ({\bf B}) 1 Punkt,
Maximumprinzip und Maximalstellen ({\bf M}) 1 Punkt.
\end{bewertung}


