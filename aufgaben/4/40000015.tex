Das Eigenwertproblem 
\[
\Delta u = \lambda u
\]
\begin{figure}
\centering
\includeagraphics[]{domain-1.pdf}
\caption{Gebiet f"ur das Eigenwertproblem von Aufgabe~\ref{40000015}.
\label{40000015:gebiet}}
\end{figure}%
mit homogenen Randbedingungen soll auf dem in Abbildung~\ref{40000015:gebiet}
dargestellten Gebiet
mit Hilfe eines Separationsansatzes gel"ost werden.
Daf"ur geeignete Koordinaten sind sogenannte elliptische Koordinaten
$(\mu,\nu)\in \mathbb R^+\times [0,2\pi]$, mit der Koordinatenumrechnung
\begin{align*}
x&=\cosh\mu\cos\nu
\\
y&=\sinh\mu\sin\nu
\end{align*}
In einer guten Formelsammlung finden Sie den Laplace-Operator ausgedr"uckt
in diesen Koordinaten, er ist
\[
\Delta
=
\frac1{\sinh^2\mu+\sin^2\nu}
\biggl(
\frac{\partial^2}{\partial \mu^2}+\frac{\partial^2}{\partial\nu^2}
\biggr).
\]
F"uhren Sie einen Separationsansatz durch und stellen Sie gew"ohnliche
Differentialgleichungen und Randbedingungen auf, die das gestellte
Eigenwertproblem zu l"osen gestatten.

\begin{loesung}
Wir verwenden den Separationsansatz $u(\mu,\nu)=M(\mu)N(\nu)$ und
setzen diesen in die Differentialgleichung
\[
\Delta u
=
\frac1{\sinh^2\mu+\sin^2\nu}
\biggl(
M''(\mu)N(\nu) + M(\mu)N''(\nu)
\biggr)
=
\lambda M(\mu)N(\nu).
\]
Um die Variablen zu separieren dividieren wir durch $M(\mu)N(\nu)$
und multiplizieren mit $\sinh^2\mu+\sin^2\nu$, wir erhalten
\[
\frac{M''(\mu)}{M(\mu)}
+
\frac{N''(\nu)}{N(\nu)}
=
\lambda\sinh^2\mu + \lambda \sin^2\nu.
\]
Jetzt k"onnen die Variablen getrennt werden:
\[
\frac{M''(\mu)}{M(\mu)}-\lambda\sinh^2\mu
=
-\frac{N''(\nu)}{N(\nu)}+\lambda\sin^2\nu.
\]
Da die linke Seite nur von $\mu$, die rechte aber nur von $\nu$ abh"angt,
m"ussen beide Seiten konstant sein.
Wir nennen die Konstante $\kappa$.
Damit erhalten wir zwei gew"ohnliche Differentialgleichungen
\begin{align*}
\frac{M''(\mu)}{M(\mu)}-\lambda\sinh^2\mu &=  \kappa
&
M''(\mu)-(\lambda\sinh^2\mu+\kappa)M(\mu)&=0
\\
\frac{N''(\nu)}{N(\nu)}-\lambda\sin^2\nu &= -\kappa
&
N''(\nu)-(\lambda\sin^2\nu+\kappa)N(\nu)&=0.
\end{align*}
Die zugeh"origen Randbedingungen sind
\begin{align*}
M(0)&=0&
N({\textstyle\frac{2\pi}{8}})&=0
\\
M(1)&=0&
N({\textstyle\frac{5\pi}{8}})&=0
\end{align*}
\end{loesung}

\begin{bewertung}
Separationsansatz mit Produkt $u(\mu,\nu) = M(\mu)N(\nu)$ ({\bf A}) 1 Punkt,
Einsetzen in die Differentialgleichung ({\bf E}) 1 Punkt,
Trennung der Variablen ({\bf T}) 1 Punkt,
Differentialgleichung f"ur $M(\mu)$ ({\bf M}) 1 Punkt,
Differentialgleichung f"ur $N(\nu)$ ({\bf N}) 1 Punkt,
Randbedingungen f"ur $S$ und $T$ ({\bf R}) 1 Punkt.
\end{bewertung}

