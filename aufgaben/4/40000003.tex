Betrachten Sie die Differentialgleichung
\[
\frac{\partial^2}{\partial x\partial y}u=0
\]
definiert in ganz $\mathbb R^2$.
\begin{teilaufgaben}
\item
Finden Sie die allgemeinen Lösungen der Gleichung
mit Hilfe eines Separationsansatzes.
\item
Bestimmen Sie eine Lösung in der rechten Halbebene
\[
H=\{(x,y)\,|\, x\ge 0\},
\]
die ausserdem die Randbedingung
\[
u(0,y)=e^{-y^2}
\]
erfüllt. Ist sie eindeutig?
\item
Überprüfen Sie Ihre Lösung durch Einsetzen.
\item
Finden Sie alle Lösungen die ausserdem die Bedingung
\begin{equation}
u(t,t)=\sin t
\label{40000003:rb}
\end{equation}
erfüllen. Hier geben wir Werte der Lösung auf einer Geraden
mitten durch das Gebiet vor, also nicht unbedingt in der Form,
wie wir das bisher bei PDGL gemacht haben. Man kann sich aber
auch das Gebiet aus zwei Halbebenen
\begin{align*}
H_1&=\{(x,y)\,|\,x\le y\}&&\text{links der $45^\circ$-Geraden}
\\
H_2&=\{(x,y)\,|\,x\ge y\}&&\text{rechts der $45^\circ$-Geraden}
\end{align*}
zusammengesetzt denken, wir suchen in jedem dieser Teilgebiete eine
Lösung mit Randbedingung \eqref{40000003:rb}.
\label{40000003:halbebenen}
\item
Wird die Lösung in der vorangegangenen Teilaufgabe eindeutig, wenn sie
zusätzlich
\[
\frac{\partial u}{\partial x}(t,t)=\frac{\partial u}{\partial y}(t,t)
\]
verlangen?
\item
Finden Sie eine Lösung der Diffentialgleichung, die
die Bedingung
\begin{align*}
u(t,t)&=f(t)\\
\frac{\partial}{\partial n}&u(t,t)=g(t)
\end{align*}
erfüllt. ({\it Hinweis:} Bemerkung zur Teilaufgaben d))
\item
Wenden Sie eine Drehung des Koordinatensystems um $45^\circ$
auf die Differentialgleichung an, und zeigen Sie,
dass sie in die Wellengleichung übergeht.
\end{teilaufgaben}

\begin{loesung}
\begin{teilaufgaben}
\item
Wir setzen die Lösung $u(x,y)$ in der Form
\[
u(x,y)=X(x)Y(y)
\]
an und setzen in die Differentialgleichung ein:
\[
\frac{\partial^2}{\partial x\partial y}u(x,y)
=
\frac{\partial}{\partial x}X(x)Y'(y)
=
X'(x)Y'(y)=0
\]
Dies bedeutet, dass jeweils eine der Funktionen
konstant ist, die andere jedoch beliebig.
Somit ist jede Funktion, die nur von einer Variablen
abhängig ist, eine Lösung, die allgemeine Lösung
ist
\[
u(x,y)=f(x)+g(y).
\]
\item
Die Anfangsbedingung besagt, dass $Y(y)$ für die gesuchte
Lösung nicht konstant sein kann, also muss in diesem Fall
$X(x)$ konstant sein. Wir können $X(x)=1$ wählen.

Es bleibt jetzt noch $Y(y)$ so zu bestimmen, dass die Anfangsbedingung
erfüllt ist. Wegen $u(0,y)=Y(y)=e^{-y^2}$ folgt
\[
u(x,y)=e^{-y^2}.
\]
\item
Die gefundene Lösung hängt nicht von $x$ ab. Beim Einsetzen
in die Differentialgleichung muss eine partielle Ableitung nach
$x$ gebildet werden, die natürlich verschwinden muss. Damit ist
die Differentialgleichung erfüllt.
\item
Eine Lösung ist von der Form
$f(x)+g(y),$
d.~h.~die Anfangsbedingung ist $f(t)+g(t)=\sin t$ oder $g(t)=\sin t-f(t)$.
Folglich ist jede Funktion
\[
u(x,y)=f(x)+\sin y-f(y)
\]
eine Lösung, die auch die Anfangsbedingung erfüllt.
\item
Die Ableitungen der allgemeinen Lösung sind
\[
\left.
\begin{aligned}
\frac{\partial u}{\partial x}
&=
f'(x)
\\
\frac{\partial u}{\partial y}
&=
-f'(y)+\cos y
\end{aligned}
\quad
\right\}
\quad
\Rightarrow
\quad
2f'(t)=\cos t
\Rightarrow f(t)=\frac12\sin t
\]
Damit ist die Lösung eindeutig bestimmt:
\[
u(x,y)=\frac12\sin x+\sin y-\frac12\sin y=\frac12(\sin x+\sin y).
\]
\item
Die Ableitung senkrecht auf der Geraden $y=x$ ist die Richtungsableitung
in Richtung $(1,-1)$, also
\[
\frac{\partial}{\partial x}u(t,t)-\frac{\partial}{\partial y}u(t,t)=g(t)
\]
Wir wissen bereits, dass $u$ die Summe von Funktionen sein muss, die
nur von jeweils einer Variablen abhängen, also
$u(x,y)=X(x)+Y(y)$. setzt man dies in die Anfangsbedingungen
ein, erhält man
\begin{align*}
X(t)+Y(t)&=f(t)\\
X'(t)-Y'(t)&=g(t)
\end{align*}
Ist $G(t)$ eine Stammfunktion von $g$, folgt daraus
\begin{align*}
X(t)+Y(t)&=f(t)\\
X(t)-Y(t)&=G(t) + C
\end{align*}
Die Lösung dieses linearen Gleichungssystems ergibt
\[
X(t)=\frac12(f(t)+G(t)+C),\qquad 
Y(t)=\frac12(f(t)-G(t)-C).
\]
Daraus lässt sich jetzt auch die Lösung gewinnen:
\begin{align*}
u(x,y)&=X(x)+Y(y)
\\
&=
\frac12(f(x)+G(x)+C)
+
\frac12(f(y)-G(y)-C)
\\
&=
\frac12(f(x)+f(y))+\frac12(G(x)+G(y)).
\end{align*}
Die Lösung ist also wieder eindeutig bestimmt.
\item
Bei einer Drehung des Koordinatensystems um $45^\circ$ gehen
$x$ und $y$ in $\xi=\frac{\sqrt{2}}2(x+y)$ und $\eta=\frac{\sqrt{2}}2(x-y)$
über. Umgekehrt lassen sich auch die Koordinaten $x$ und $y$ durch
die neuen Koordinaten $\xi$ und $\eta$ ausdrücken:
\[
x=\frac{\sqrt{2}}2(\xi-\eta),\qquad y=\frac{\sqrt{2}}2(\xi+\eta).
\]
Damit kann die Lösung $u$ jetzt auch in den Koordinaten $(\xi,\eta)$
geschrieben werden. Die Ableitungen nach $x$ und $y$
einer Funktion von $\xi$
und $\eta$ sind
\begin{align*}
\frac{\partial}{\partial x}u(\xi,\eta)
&=
\frac{\partial u}{\partial \xi}\frac{\partial \xi}{\partial x}
+
\frac{\partial u}{\partial \eta}\frac{\partial \eta}{\partial x}
=
\frac{\sqrt{2}}2\frac{\partial u}{\partial\xi}
+\frac{\sqrt{2}}2\frac{\partial u}{\partial\eta}
=
\frac{\sqrt{2}}2\left(\frac{\partial}{\partial \xi}+\frac{\partial}{\partial\eta}\right)u
\\
\frac{\partial}{\partial y}u(\xi,\eta)
&=
\frac{\partial u}{\partial \xi}\frac{\partial \xi}{\partial y}
+
\frac{\partial u}{\partial \eta}\frac{\partial \eta}{\partial y}
=
\frac{\sqrt{2}}2\frac{\partial u}{\partial\xi}
-\frac{\sqrt{2}}2\frac{\partial u}{\partial\eta}
=
\frac{\sqrt{2}}2\left(\frac{\partial}{\partial \xi}-\frac{\partial}{\partial\eta}\right)u
\end{align*}
Wendet man beide Ableitungen hintereinander an, wie es die
Differentialgleichung verlangt, erhält man
\begin{align*}
0=\frac{\partial^2}{\partial x\partial y}u
&=
\frac{\sqrt{2}}2
\left(\frac{\partial}{\partial \xi}+\frac{\partial}{\partial\eta}\right)
\frac{\sqrt{2}}2
\left(\frac{\partial}{\partial \xi}-\frac{\partial}{\partial\eta}\right)u
\\
&=
\frac12
\left(
\frac{\partial^2}{\partial x^2}-\frac{\partial^2}{\partial y^2}
\right)u
\end{align*}
Dies ist bis auf den Faktor $\frac12$, der ohnehin kein Rolle spielt,
die Wellengleichung.
\qedhere
\end{teilaufgaben}
\end{loesung}
