Betrachten Sie die Differentialgleichung
\[
\frac{\partial^2}{\partial x\partial y}u=0
\]
definiert in ganz $\mathbb R^2$.
\begin{teilaufgaben}
\item
Finden sie die allgemeinen L"osungen der Gleichung
mit Hilfe eines Separationsansatzes.
\item
Bestimmen Sie eine L"osung in der rechten Halbebene
\[
H=\{(x,y)\,|\, x\ge 0\},
\]
die ausserdem die Randbedingung
\[
u(0,y)=e^{-y^2}
\]
erf"ullt. Ist sie eindeutig?
\item
"Uberpr"ufen Sie Ihre L"osung durch Einsetzen.
\item
Finden Sie alle L"osungen die ausserdem die Bedingung
\begin{equation}
u(t,t)=\sin t
\label{40000003:rb}
\end{equation}
erf"ullen. Hier geben wir Werte der L"osung auf einer Geraden
mitten durch das Gebiet vor, also nicht unbedingt in der Form,
wie wir das bisher bei PDGL gemacht haben. Man kann sich aber
auch das Gebiet aus zwei Halbebenen
\begin{align*}
H_1&=\{(x,y)\,|\,x\le y\}&&\text{links der $45^\circ$-Geraden}
\\
H_2&=\{(x,y)\,|\,x\ge y\}&&\text{rechts der $45^\circ$-Geraden}
\end{align*}
zusammengesetzt denken, wir suchen in jedem dieser Teilgebiete eine
L"osung mit Randbedingung (\ref{40000003:rb}).
\label{40000003:halbebenen}
\item
Wird die L"osung in der vorangegangenen Teilaufgabe eindeutig, wenn sie
zus"atzlich
\[
\frac{\partial u}{\partial x}(t,t)=\frac{\partial u}{\partial y}(t,t)
\]
verlangen?
\item
Finden Sie eine L"osung der Diffentialgleichung, die
die Bedingung
\begin{align*}
u(t,t)&=f(t)\\
\frac{\partial}{\partial n}&u(t,t)=g(t)
\end{align*}
erf"ullt. ({\it Hinweis:} Bemerkung zur Teilaufgaben d))
\item
Wenden Sie eine Drehung des Koordinatensystems um $45^\circ$
auf die Differentialgleichung an, und zeigen Sie,
dass sie in die Wellengleichung "ubergeht.
\end{teilaufgaben}

\ifthenelse{\boolean{loesungen}}{
\begin{loesung}
\begin{teilaufgaben}
\item
Wir setzen die L"osung $u(x,y)$ in der Form
\[
u(x,y)=X(x)Y(y)
\]
an und setzen in die Differentialgleichung ein:
\[
\frac{\partial^2}{\partial x\partial y}u(x,y)
=
\frac{\partial}{\partial x}X(x)Y'(y)
=
X'(x)Y'(y)=0
\]
Dies bedeutet, dass jeweils eine der Funktionen
konstant ist, die andere jedoch beliebig.
Somit ist jede Funktion, die nur von einer Variablen
abh"angig ist, eine L"osung, die allgemeine L"osung
ist
\[
u(x,y)=f(x)+g(y).
\]
\item
Die Anfangsbedingung besagt, dass $Y(y)$ f"ur die gesuchte
L"osung nicht konstant sein kann, also muss in diesem Fall
$X(x)$ konstant sein. Wir k"onnen $X(x)=1$ w"ahlen.

Es bleibt jetzt noch $Y(y)$ so zu bestimmen, dass die Anfangsbedingung
erf"ullt ist. Wegen $u(0,y)=Y(y)=e^{-y^2}$ folgt
\[
u(x,y)=e^{-y^2}.
\]
\item
Die gefundene L"osung h"angt nicht von $x$ ab. Beim Einsetzen
in die Differentialgleichung muss eine partielle Ableitung nach
$x$ gebildet werden, die nat"urlich verschwinden muss. Damit ist
die Differentialgleichung erf"ullt.
\item
Eine L"osung ist von der Form
$f(x)+g(y),$
d.~h.~die Anfangsbedingung ist $f(t)+g(t)=\sin t$ oder $g(t)=\sin t-f(t)$.
Folglich ist jede Funktion
\[
u(x,y)=f(x)+\sin y-f(y)
\]
eine L"osung, die auch die Anfangsbedingung erf"ullt.
\item
Die Ableitungen der allgemeinen L"osung sind
\[
\left.
\begin{aligned}
\frac{\partial u}{\partial x}
&=
f'(x)
\\
\frac{\partial u}{\partial y}
&=
-f'(y)+\cos y
\end{aligned}
\quad
\right\}
\quad
\Rightarrow
\quad
2f'(t)=\cos t
\Rightarrow f(t)=\frac12\sin t
\]
Damit ist die L"osung eindeutig bestimmt:
\[
u(x,y)=\frac12\sin x+\sin y-\frac12\sin y=\frac12(\sin x+\sin y).
\]
\item
Die Ableitung senkrecht auf der Geraden $y=x$ ist die Richtungsableitung
in Richtung $(1,-1)$, also
\[
\frac{\partial}{\partial x}u(t,t)-\frac{\partial}{\partial y}u(t,t)=g(t)
\]
Wir wissen bereits, dass $u$ die Summe von Funktionen sein muss, die
nur von jeweils einer Variablen abh"angen, also
$u(x,y)=X(x)+Y(y)$. setzt man dies in die Anfangsbedingungen
ein, erh"alt man
\begin{align*}
X(t)+Y(t)&=f(t)\\
X'(t)-Y'(t)&=g(t)
\end{align*}
Ist $G(t)$ eine Stammfunktion von $g$, folgt daraus
\begin{align*}
X(t)+Y(t)&=f(t)\\
X(t)-Y(t)&=G(t) + C
\end{align*}
Die L"osung dieses linearen Gleichungssystems ergibt
\[
X(t)=\frac12(f(t)+G(t)+C),\qquad 
Y(t)=\frac12(f(t)-G(t)-C).
\]
Daraus l"asst sich jetzt auch die L"osung gewinnen:
\begin{align*}
u(x,y)&=X(x)+Y(y)
\\
&=
\frac12(f(x)+G(x)+C)
+
\frac12(f(y)-G(y)-C)
\\
&=
\frac12(f(x)+f(y))+\frac12(G(x)+G(y)).
\end{align*}
Die L"osung ist also wieder eindeutig bestimmt.
\item
Bei einer Drehung des Koordinatensystems um $45^\circ$ gehen
$x$ und $y$ in $\xi=\frac{\sqrt{2}}2(x+y)$ und $\eta=\frac{\sqrt{2}}2(x-y)$
"uber. Umgekehrt lassen sich auch die Koordinaten $x$ und $y$ durch
die neuen Koordinaten $\xi$ und $\eta$ ausdr"ucken:
\[
x=\frac{\sqrt{2}}2(\xi-\eta),\qquad y=\frac{\sqrt{2}}2(\xi+\eta).
\]
Damit kann die L"osung $u$ jetzt auch in den Koordinaten $(\xi,\eta)$
geschrieben werden. Die Ableitungen nach $x$ und $y$
einer Funktion von $\xi$
und $\eta$ sind
\begin{align*}
\frac{\partial}{\partial x}u(\xi,\eta)
&=
\frac{\partial u}{\partial \xi}\frac{\partial \xi}{\partial x}
+
\frac{\partial u}{\partial \eta}\frac{\partial \eta}{\partial x}
=
\frac{\sqrt{2}}2\frac{\partial u}{\partial\xi}
+\frac{\sqrt{2}}2\frac{\partial u}{\partial\eta}
=
\frac{\sqrt{2}}2\left(\frac{\partial}{\partial \xi}+\frac{\partial}{\partial\eta}\right)u
\\
\frac{\partial}{\partial y}u(\xi,\eta)
&=
\frac{\partial u}{\partial \xi}\frac{\partial \xi}{\partial y}
+
\frac{\partial u}{\partial \eta}\frac{\partial \eta}{\partial y}
=
\frac{\sqrt{2}}2\frac{\partial u}{\partial\xi}
-\frac{\sqrt{2}}2\frac{\partial u}{\partial\eta}
=
\frac{\sqrt{2}}2\left(\frac{\partial}{\partial \xi}-\frac{\partial}{\partial\eta}\right)u
\end{align*}
Wendet man beide Ableitungen hintereinander an, wie es die
Differentialgleichung verlangt, erh"alt man
\begin{align*}
0=\frac{\partial^2}{\partial x\partial y}u
&=
\frac{\sqrt{2}}2
\left(\frac{\partial}{\partial \xi}+\frac{\partial}{\partial\eta}\right)
\frac{\sqrt{2}}2
\left(\frac{\partial}{\partial \xi}-\frac{\partial}{\partial\eta}\right)u
\\
&=
\frac12
\left(
\frac{\partial^2}{\partial x^2}-\frac{\partial^2}{\partial y^2}
\right)u
\end{align*}
Dies ist bis auf den Faktor $\frac12$, der ohnehin kein Rolle spielt,
die Wellengleichung.
\end{teilaufgaben}
\end{loesung}
}{}

