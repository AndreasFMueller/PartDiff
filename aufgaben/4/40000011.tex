Auf dem Gebiet $\Omega=\{(x,y)\;|\;0<x<\pi,y>1\}$
ist die partielle Differentialgleichung
\begin{equation}
\frac{\partial^2 u}{\partial x^2}-y\frac{\partial u}{\partial y}=0
\label{40000011:gleichung}
\end{equation}
gegeben.
Finden Sie eine Lösung von \eqref{40000011:gleichung}, die den
Randbedingungen
\begin{align*}
\frac{\partial}{\partial x}u(0,y)=
\frac{\partial}{\partial x}u(\pi,y)&=0&y&> 1\\
u(x,1)&=\cos2x&0&<x<\pi
\end{align*}
genügt.

\begin{loesung}
Wir lösen die partielle Differentialgleichung mit Hilfe eines 
Separationsansatzes $u(x,y)=X(x)\cdot Y(y)$.
Einsetzen in \eqref{40000011:gleichung} und Separieren ergibt
\begin{align*}
X''(x)Y(y)-yX(x)Y'(y)&=0\\
\frac{X''(x)}{X(x)}&=y\frac{Y'(y)}{Y(y)}.
\end{align*}
Da die linke Seite nur von $x$, die rechte nur von $y$ abhängt, müssen
beide Seiten konstant sein, wir nennen die gemeinsame Konstante $-\mu^2$,
was sich als praktisch erweisen wird.

Die Teilgleichung
\[
\frac{X''(x)}{X(x)}=-\mu^2
\qquad
\Leftrightarrow
\qquad
X''(x)=-\mu^2X(x)
\]
ist eine Schwingungsdifferentialgleichung, sie hat $\sin\mu x$ und $\cos\mu x$
als Lösungen.
Ausserdem müssen aber auch die homogenen Randbedingungen
erfüllt sein.
Die Lösung $\sin\mu x$ hat am linken Rand die Ableitung $\mu$,
die die homogene Randbedingung nicht erfüllt, es bleibt nur $\cos\mu x$.
Seine Ableitung am rechten Rand muss ebenfalls verschwinden:
\[
-\mu\sin\mu \pi=0
\qquad
\Rightarrow
\qquad
\mu\in\mathbb N.
\]

Damit können wir jetzt auch die Gleichung für $Y$ vereinfachen
\[
y\, \frac{Y'(y)}{Y(y)}=-\mu^2
\qquad
\Leftrightarrow
\qquad
\frac{dY}{dy}=-\mu^2\frac{Y}{y}
\]
und mit Separation lösen
\begin{align*}
\int\frac{dY}{Y}&=-\mu^2\int\frac{dy}y\\
\log Y&=-\mu^2\log y+K\\
Y(y)&=Cy^{-\mu^2}.
\end{align*}
Jetzt können wir die Lösung der Differentialgleichung als Linearkombination
der gefundenen Teillösungen ansetzen:
\[
u(x,y)=\sum_{n=0}^\infty a_ny^{-n^2}\cos nx.
\]
Es bleibt jetzt nur noch, mit der verbleibenden Randbedingung die Koeffizienten
$a_n$ zu bestimmen. Einsetzen der Randbedingung ergibt:
\[
u(x,1)=\sum_{n=0}^\infty a_n\cos nx=\cos 2x,
\]
woraus man mit Koeffizientenvergleich ablesen kann,
dass $a_2=1$ und alle anderen Koeffizienten verschwinden.
Die gesuchte Lösung ist also
\[
u(x,y)=y^{-4}\cos 2x.
\]
Wir kontrollieren dieses Resultat durch Einsetzen:
\[
\left.
\begin{aligned}
\frac{\partial u}{\partial x}&=-2y^{-4}\sin 2x
\\
\frac{\partial^2 u}{\partial x^2}&=-4y^{-4}\cos 2x
\\
\frac{\partial u}{\partial y}&=-4y^{-5}\cos 2x
\end{aligned}\qquad
\right\}
\qquad
\Rightarrow
\qquad
\frac{\partial^2u}{\partial x^2}-y\frac{\partial u}{\partial y}=
-4y^4\cos 2x-y(-4y^{-5}\cos 2x)=0,
\]
die Randbedingungen sind offensichtlich auch erfüllt.
\end{loesung}

\begin{bewertung}
Separationsansatz ({\bf A}) 1 Punkt,
Separation  ({\bf S}) 1 Punkt,
Wahl einer Konstanten ({\bf K}) 1 Punkt,
Lösung der Teilgleichungen ({\bf L}) 1 Punkt,
Einsetzen der Randwerte ({\bf R}) 1 Punkt,
Koeffizientenvergleich und Lösung ({\bf K}) 1 Punkt.
\end{bewertung}


