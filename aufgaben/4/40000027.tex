Consider the domain $\Omega$ as depicted in figure~\ref{40000027:domain}.
\begin{figure}[h]
\centering
\includeagraphics[]{domain.pdf}
\caption{Domain for problem~\ref{40000027} with $\xi$-$\eta$-coordinate
grid.
\label{40000027:domain}}
\end{figure}
The differential equation
\begin{equation}
\Delta u = \frac{\lambda}{r^2} u,
\label{40000027:dgl}
\end{equation}
where $r^2= x^2 +y^2$, is to be solved with homogeneous boundary
conditions.

There is a $\xi$-$\eta$-coordinate system with coordinates
\[
\left.
\begin{aligned}
\xi  &= \frac{ x}{x^2+y^2}\\
\eta &= \frac{-y}{x^2+y^2}
\end{aligned}
\quad\right\}
\qquad\text{and}\qquad
\left\{\quad
\begin{aligned}
x &= \frac{\xi}{\xi^2+\eta^2}\\
y &= \frac{-\eta}{\xi^2+\eta^2},
\end{aligned}
\right.
\]
in which the the domain has the description
\begin{equation}
\Omega
=
\{ (\xi,\eta)\mid 1<\xi<2\text{ and } -1< \eta< 0\}.
\label{40000027:domainxieta}
\end{equation}
Also, the Laplace operator has the form
\[
\Delta
=
(\xi^2+\eta^2)
\biggl(\frac{\partial^2}{\partial \xi^2}
+
\frac{\partial^2}{\partial \eta^2}\biggr)
\]
in these coordinates.
\begin{teilaufgaben}
\item
Write the differential equation completely in $\xi$-$\eta$-coordinates.
\item
Using these coordinates, reduce the differential equation~\eqref{40000027:dgl}
to two ordinary differential equations with suitable boundary conditions.
\end{teilaufgaben}

\begin{loesung}
\begin{teilaufgaben}
\item
To write the differential equation in $\xi$-$\eta$-coordinates, we first
express the factor $1/r^2$ in these coordinates:
\[
\frac{1}{r^2}
=
\frac{1}{x^2+y^2}
=
\frac{1}{
\displaystyle
\frac{\xi^2}{(\xi^2+\eta^2)^2}
+
\frac{\eta^2}{(\xi^2+\eta^2)^2}
}
=
\frac{(\xi^2+\eta^2)^2}{\xi^2+\eta^2}
=
\xi^2+\eta^2.
\]
Using the expression for the Laplace operator in $\xi$-$\eta$-coordinates,
the differential equation becomes
\[
(\xi^2+\eta^2)
\biggl(
\frac{\partial^2 u}{\partial\xi^2}
+
\frac{\partial^2 u}{\partial\eta^2}
\biggr)
=
\lambda(\xi^2+\eta^2) u.
\]
Because $\xi^2+\eta^2>0$, we can divide by it and get the differential
equation
\begin{equation}
\frac{\partial^2 u}{\partial \xi^2}
+
\frac{\partial^2 u}{\partial \eta^2}
=
\lambda u.
\label{40000027:transformed}
\end{equation}
on the domain $\Omega$ expressed as in \eqref{40000027:domainxieta},
i.~e.~on a rectangle in the $\xi$-$\eta$-plane.
\item
The equation~\eqref{40000027:transformed}
has in fact been solved in class and it turns out
that the solutions are products of trigonometric functions.
The solution is summarized below.

This equation can easily be solved using the separation ansatz
$u(\xi,\eta)=X(\xi)Y(\eta)$, which leads to
\[
X''(\xi)Y(\eta)+X(\xi)Y''(\eta) = X(\xi)Y(\eta),
\]
and after division by $X(\xi)Y(\eta)$
\[
\frac{X''(\xi)}{X(\xi)}
+
\frac{Y''(\eta)}{Y(\eta)}
=
\lambda.
\]
Separation gives the two equations
\begin{equation}
\frac{X''(\xi)}{X(\xi)}
=
\lambda
-
\frac{Y''(\eta)}{Y(\eta)},
\label{40000027:dgl}
\end{equation}
where each sides depends only on one variable.
Each side is therefore equal to a constant $\mu$, giving
the separated equations
\[
X''(\xi) = \mu X(\xi)
\qquad\text{and}\qquad
Y''(\eta) = (\lambda -\mu) Y(\eta).
\]
The homogeneous boundary conditions for $u(x,y)$ become
the homogeneous boundary conditions
\[
\begin{aligned}
X(1)&=0\\
X(2)&=0
\end{aligned}
\qquad\text{and}\qquad
\begin{aligned}
Y(-1)&=0\\
Y(0)&=0
\end{aligned}
\]
for $X(\xi)$ and $Y(\eta)$.
\qedhere
\end{teilaufgaben}
\end{loesung}

\begin{diskussion}
The differential equations~\eqref{40000027:dgl} can easily be solved using
trigonometric functions.
The equation for $X(\xi)$ has solutions
\[
X(\xi) = \cos \sqrt{-\mu}\xi
\qquad\text{and}\qquad
X(\xi) = \sin \sqrt{-\mu}\xi,
\]
but the homogeneous boundary conditions can only be satisfied if $\sqrt{-\mu}$
ist an integer multiple of $\pi$ or $\mu=-k^2\pi^2$.
Similarly, the equation for $Y(\eta)$ has solutions
$\sin \sqrt{\mu-\lambda}\eta$, but again, to satisfy the homogeneous
boundary conditions, $\sqrt{\mu-\lambda}$ must be an integer multiple of
$\pi$.
Therefore
\[
\lambda - \mu = -l^2\pi^2
\qquad\Rightarrow\qquad
\lambda = -(l^2 + k^2)\pi^2.
\]
So the solutions the equation are
\[
u_{kl}(\xi,\eta)
=
X(\xi)Y(\eta)
=
\sin k\pi\xi \sin l\pi\eta.
\qedhere
\]
\end{diskussion}

\begin{bewertung}
\begin{teilaufgaben}
\item
$r$ in $\xi$-$\eta$-coordinates ({\bf R}) 1 point,
equation \eqref{40000027:dgl} in $\xi$-$\eta$-coordinates ({\bf D}) 1 point,
\item
Separation ansatz ({\bf S}) 1 point,
equation with separated variables ({\bf V}) 1 point,
separated differential equations ({\bf E}) 1 point,
boundary conditions ({\bf B}) 1 point.
\end{teilaufgaben}
\end{bewertung}
