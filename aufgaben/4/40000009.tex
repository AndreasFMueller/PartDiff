Im Streifen $\Omega=\{ (x,t)\, |\, 0<x<\pi, t > 0\}$ soll die partielle
Differentialgleichung
\[
\frac{\partial u}{\partial t}
=
\frac{\partial^2 u}{\partial x^2}+2\frac{\partial u}{\partial x}
\]
gel"ost werden. Finden Sie eine L"osung $u(x,t)$, die Randwerte
\begin{align*}
u(0,t)&=u(\pi,t)=0&&\text{f"ur $t>0$}\\
u(x,0)&=e^{-x}\sin x&&\text{f"ur $0<x<\pi$}
\end{align*}
hat.

\begin{loesung}
Ein Separationsansatz $u(x,t)=X(x)T(t)$ ergibt die separierte Gleichung
\[
\frac{T'(t)}{T(t)}
=
\frac{X''(x)+2X'(x)}{X(x)}.
\]
Da die linke Seite nur von $t$, die rechte aber nur von $x$ abh"angt,
m"ussen beide Seiten konstant $=\lambda$ sein. Die linke Seite wird
damit zu
\[
T' = \lambda T
\]
Die partielle Differentialgleichung ist parabolisch, wir erwarten also
das Verhalten der L"osungen wie bei der W"armeleitungsgleichung, insbesondere
erwarten wir, dass die Amplituden mit der Zeit abnehmen, was einer
negativen Konstanten f"ur $\lambda$ entspricht. Wir schreiben daher
$\lambda=-\mu^2$ f"ur die Konstante.
Die L"osungen f"ur die Funktion $T$ sind daher $T=T_0e^{-\mu^2t}$.

Die rechte Seite f"uhrt jetzt auf die Differentialgleichung
\[
X''+2X'+\mu^2X=0.
\]
Diese gew"ohnliche lineare Differentialgleichung zweiter Ordnung
kann mit dem Standardverfahren gel"ost werden. Das charakteristische
Polynom 
\[
\lambda^2+2\lambda +\mu^2=0
\]
hat die Nullstellen
\[
\lambda_{\pm}=-1\pm\sqrt{1-\mu^2}.
\]
A priori wissen wir nichts "uber das Vorzeichen des Radikanden,
f"ur positive Radikanden w"aren L"osungen der Form
\begin{equation}
e^{x(-1+ \sqrt{\mu^2-1})}
\qquad
\text{oder}
\qquad
e^{x(-1-\sqrt{\mu^2-1})}
\label{40000009:exp}
\end{equation}
oder
\begin{equation}
e^{-x}\sinh(x\sqrt{\mu^2-1})
\qquad
\text{oder}
\qquad
e^{-x}\cosh(x\sqrt{\mu^2-1}),
\label{40000009:hyp}
\end{equation}
f"ur negative Radikanden aber solche der Form
\begin{equation}
e^{-x}\sin(x\sqrt{1-\mu^2})
\qquad
\text{oder}
\qquad
e^{-x}\cos(x\sqrt{1-\mu^2})
\label{40000009:trig}
\end{equation}
m"oglich. Da unsere L"osung als Anfangsbedingung f"ur $t=0$ die
Funktion $e^{-x}\sin x$ haben muss, sind (\ref{40000009:exp}) 
und (\ref{40000009:hyp}) nicht geeignet, und wir
w"ahlen (\ref{40000009:trig}).

Wir m"ussen aber auch die Randwertbedingungen
$X(0)=X(\pi)=0$
befriedigen k"onnen.
Da $\cos 0=1$ ist, kommt die $\cos$-L"osung
in (\ref{40000009:trig}) nicht mehr in Frage.
Aber auch die $\sin$-L"osung ist nur
zul"assig, wenn $\sin(\pi\sqrt{\mu^2-1})=0$, also
$\sqrt{\mu^2-1}=k$ eine ganze Zahl ist.
Daraus ergibt sich $\mu=\sqrt{k^2+1}$.

Wir suchen jetzt also eine
L"osung der Differentialgleichung in der Form
\[
u(x,t)=\sum_{k=1}^\infty b_ke^{-(k^2+1)t}e^{-x}\sin kx.
\]
Diese L"osung muss auch noch die Randbedingung f"ur $t=0$ im
Bereich $0<x<\pi$ erf"ullen:
\[
u(x,0)=\sum_{k=1}^\infty b_ke^{-x}\sin kx=e^{-x}\sin x.
\]
Daraus schliessen wir, dass die Koeffizienten
\[
b_k=\begin{cases}
1&\qquad\text{f"ur $k=1$}\\
0&\qquad\text{sonst}
\end{cases}
\]
sind. Die L"osung ist also
\begin{equation}
u(x,t)=
e^{-2t-x}\sin x.
\label{30000009:solution}
\end{equation}
\end{loesung}

\begin{diskussion}
Zur Kontrolle kann man die L"osung (\ref{30000009:solution})
noch in die Differentialgleichung einsetzen.
Dazu berechnen wir zun"achst die partiellen Ableitungen
\begin{align*}
\frac{\partial u}{\partial t}
&=
-2 e^{-2t-x}\sin x
\\
\frac{\partial u}{\partial x}
&=
- e^{-2t-x}\sin x + 
e^{-2t-x}\cos x
\\
\frac{\partial^2 u}{\partial x^2}
&=
e^{-2t-x}\sin x 
- e^{-2t-x}\cos x 
-e^{-2t-x}\cos x
-e^{-2t-x}\sin x
=
-2 e^{-2t-x}\cos x
\end{align*}
Eingesetzt in die Differentialgleichung:
\begin{align*}
\frac{\partial^2 u}{\partial x^2}+2\frac{\partial u}{\partial x}
&=
-2 e^{-2t-x}\cos x
+
2\cdot(
- e^{-2t-x}\sin x + 
e^{-2t-x}\cos x
)
\\
&=
-2 e^{-2t-x}\sin x 
\\
&=\frac{\partial u}{\partial t}.
\end{align*}
Asserdem gilt f"ur $t=0$:
\[
u(x,0)=e^{-x}\sin x,
\]
die Randbedingung ist also auch erf"ullt.
\end{diskussion}

\begin{bewertung}
Separationsansatz ({\bf S}) 1 Punkt,
Wahl einer Konstanten ({\bf K}) 1 Punkt,
L"osung der Gleichung f"ur $T$ ({\bf T}) 1 Punkt,
L"osung der Gleichung f"ur $X$ ({\bf X}) 1 Punkt,
Anwendung der Randbedingungen f"ur $x=0$ und $x=\pi$
zur Eingrenzung von $X$ ({\bf R}) 1 Punkt,
Anwendung der Randbedingungen f"ur $t=0$ zur Bestimmung 
der L"osung ({\bf L}) 1 Punkt.
\end{bewertung}

