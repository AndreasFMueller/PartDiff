Im Streifen $\Omega=\{ (x,t)\, |\, 0<x<\pi, t > 0\}$ soll die partielle
Differentialgleichung
\[
\frac{\partial u}{\partial t}
=
\frac{\partial^2 u}{\partial x^2}+2\frac{\partial u}{\partial x}
\]
gelöst werden. Finden Sie eine Lösung $u(x,t)$, die Randwerte
\begin{align*}
u(0,t)&=u(\pi,t)=0&&\text{für $t>0$}\\
u(x,0)&=e^{-x}\sin x&&\text{für $0<x<\pi$}
\end{align*}
hat.

\begin{loesung}
Ein Separationsansatz $u(x,t)=X(x)T(t)$ ergibt die separierte Gleichung
\[
\frac{T'(t)}{T(t)}
=
\frac{X''(x)+2X'(x)}{X(x)}.
\]
Da die linke Seite nur von $t$, die rechte aber nur von $x$ abhängt,
müssen beide Seiten konstant $=\lambda$ sein. Die linke Seite wird
damit zu
\[
T' = \lambda T
\]
Die partielle Differentialgleichung ist parabolisch, wir erwarten also
das Verhalten der Lösungen wie bei der Wärmeleitungsgleichung, insbesondere
erwarten wir, dass die Amplituden mit der Zeit abnehmen, was einer
negativen Konstanten für $\lambda$ entspricht. Wir schreiben daher
$\lambda=-\mu^2$ für die Konstante.
Die Lösungen für die Funktion $T$ sind daher $T=T_0e^{-\mu^2t}$.

Die rechte Seite führt jetzt auf die Differentialgleichung
\[
X''+2X'+\mu^2X=0.
\]
Diese gewöhnliche lineare Differentialgleichung zweiter Ordnung
kann mit dem Standardverfahren gelöst werden. Das charakteristische
Polynom 
\[
\lambda^2+2\lambda +\mu^2=0
\]
hat die Nullstellen
\[
\lambda_{\pm}=-1\pm\sqrt{1-\mu^2}.
\]
A priori wissen wir nichts über das Vorzeichen des Radikanden,
für positive Radikanden wären Lösungen der Form
\begin{equation}
e^{x(-1+ \sqrt{1-\mu^2})}
\qquad
\text{oder}
\qquad
e^{x(-1-\sqrt{1-\mu^2})}
\label{40000009:exp}
\end{equation}
oder
\begin{equation}
e^{-x}\sinh(x\sqrt{1-\mu^2})
\qquad
\text{oder}
\qquad
e^{-x}\cosh(x\sqrt{1-\mu^2}),
\label{40000009:hyp}
\end{equation}
für negative Radikanden aber solche der Form
\begin{equation}
e^{-x}\sin(x\sqrt{\mu^2-1})
\qquad
\text{oder}
\qquad
e^{-x}\cos(x\sqrt{\mu^2-1})
\label{40000009:trig}
\end{equation}
möglich. Da unsere Lösung als Anfangsbedingung für $t=0$ die
Funktion $e^{-x}\sin x$ haben muss, sind \eqref{40000009:exp}
und \eqref{40000009:hyp} nicht geeignet, und wir
wählen \eqref{40000009:trig}.

Wir müssen aber auch die Randwertbedingungen
$X(0)=X(\pi)=0$
befriedigen können.
Da $\cos 0=1$ ist, kommt die $\cos$-Lösung
in \eqref{40000009:trig} nicht mehr in Frage.
Aber auch die $\sin$-Lösung ist nur
zulässig, wenn $\sin(\pi\sqrt{\mu^2-1})=0$, also
$\sqrt{\mu^2-1}=k$ eine ganze Zahl ist.
Daraus ergibt sich $\mu=\sqrt{k^2+1}$.

Wir suchen jetzt also eine
Lösung der Differentialgleichung in der Form
\[
u(x,t)=\sum_{k=1}^\infty b_ke^{-(k^2+1)t}e^{-x}\sin kx.
\]
Diese Lösung muss auch noch die Randbedingung für $t=0$ im
Bereich $0<x<\pi$ erfüllen:
\[
u(x,0)=\sum_{k=1}^\infty b_ke^{-x}\sin kx=e^{-x}\sin x.
\]
Daraus schliessen wir, dass die Koeffizienten
\[
b_k=\begin{cases}
1&\qquad\text{für $k=1$}\\
0&\qquad\text{sonst}
\end{cases}
\]
sind. Die Lösung ist also
\begin{equation}
u(x,t)=
e^{-2t-x}\sin x.
\label{30000009:solution}
\end{equation}
\end{loesung}

\begin{diskussion}
Zur Kontrolle kann man die Lösung \eqref{30000009:solution}
noch in die Differentialgleichung einsetzen.
Dazu berechnen wir zunächst die partiellen Ableitungen
\begin{align*}
\frac{\partial u}{\partial t}
&=
-2 e^{-2t-x}\sin x
\\
\frac{\partial u}{\partial x}
&=
- e^{-2t-x}\sin x + 
e^{-2t-x}\cos x
\\
\frac{\partial^2 u}{\partial x^2}
&=
e^{-2t-x}\sin x 
- e^{-2t-x}\cos x 
-e^{-2t-x}\cos x
-e^{-2t-x}\sin x
=
-2 e^{-2t-x}\cos x
\end{align*}
Eingesetzt in die Differentialgleichung:
\begin{align*}
\frac{\partial^2 u}{\partial x^2}+2\frac{\partial u}{\partial x}
&=
-2 e^{-2t-x}\cos x
+
2\cdot(
- e^{-2t-x}\sin x + 
e^{-2t-x}\cos x
)
\\
&=
-2 e^{-2t-x}\sin x 
\\
&=\frac{\partial u}{\partial t}.
\end{align*}
Asserdem gilt für $t=0$:
\[
u(x,0)=e^{-x}\sin x,
\]
die Randbedingung ist also auch erfüllt.
\end{diskussion}

\begin{bewertung}
Separationsansatz ({\bf S}) 1 Punkt,
Wahl einer Konstanten ({\bf K}) 1 Punkt,
Lösung der Gleichung für $T$ ({\bf T}) 1 Punkt,
Lösung der Gleichung für $X$ ({\bf X}) 1 Punkt,
Anwendung der Randbedingungen für $x=0$ und $x=\pi$
zur Eingrenzung von $X$ ({\bf R}) 1 Punkt,
Anwendung der Randbedingungen für $t=0$ zur Bestimmung 
der Lösung ({\bf L}) 1 Punkt.
\end{bewertung}

