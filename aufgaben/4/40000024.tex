On the domain $\Omega = \{ (x,y)\,|\, 1 < x^2+y^2 < 4\}$ the partial
differential equation
\begin{equation}
\Delta u = r^2 u,
\label{40000024:pde}
\end{equation}
where $r^2=x^2+y^2$,
is given with boundary conditions
\begin{equation}
\begin{aligned}
u(x,y) &= \phantom{\textstyle\frac12}x  & &\text{für $x^2+y^2 = 1$}\\
u(x,y) &= \phantom{\textstyle\frac12}y  & &\text{für $x^2+y^2 = 4$.}
\end{aligned}
\label{40000024:boundarypolar}
\end{equation}
\begin{teilaufgaben}
\item
Is this equation elliptic, parabolic or hyperbolic?
\item
Use separation to obtain ordinary differential equations that can be
used to find solutions of the partial differential equation.
\item
Solve one of the ordinary differential equations you have obtained
in b)
\item
Determine the possible values of the separation constant.
\item
Write the general solution as a superposition.
\end{teilaufgaben}

\begin{loesung}
Since the domain is a ring, we have to use polar coordinates.
The boundary conditions in polar coordinates are 
\[
\begin{aligned}
u(1,\varphi) &= \phantom{2}\cos\varphi\\
u(2,\varphi) &= 2\sin\varphi.
\end{aligned}
\]
\begin{teilaufgaben}
\item
Since the operator in question is the Laplace operator, the equation is
elliptic.
\item
We use an ansatz of the form $u=R(r)\Phi(\varphi)$
and substitute the ansatz into the partial differential equation, which
we have to rewrite with the well known form of the Laplace operator
in polar coordinates:
\begin{align*}
\Delta u
=
\biggl(
\frac{\partial^2}{\partial r^2}
+
\frac1r\frac{\partial}{\partial r}
+
\frac1{r^2}\frac{\partial^2}{\partial\varphi^2}
\biggr)
R(r)\Phi(\varphi)
&=
r^2R(r)\Phi(\varphi)
\\
R''(r)\Phi(\varphi)
+\frac1r R'(r)\Phi(\varphi)
+\frac1{r^2} R(r)\Phi''(\varphi)
&=r^2R(r)\Phi(\varphi)
\\
R''(r)\Phi(\varphi)
+\frac1r R'(r)\Phi(\varphi)
-r^2R(r)\Phi(\varphi)
&=
-\frac1{r^2} R(r)\Phi''(\varphi)
\\
\frac{r^2R''(r)+rR'(r)-r^4R(r)}{R(r)}
&=
-\frac{\Phi''(\varphi)}{\Phi(\varphi)}
\end{align*}
Since the left hand side depends only on $r$ and the right only on $\varphi$,
both sides are constant, we call this constant $\lambda^2$.

The equation for $R(r)$ becomes
\begin{align}
r^2R''(r) + rR'(r) -r^4R(r) &= \lambda^2 R(r)
\notag
\\
R''(r) + \frac1rR'(r) - (r^4+\lambda^2) R(r) &=0
\label{40000024:Rdiff}
\end{align}
This is an linear ordinary differential equation which will in general
have two linearly independent solutions $R_{\lambda,1}(r)$ and
$R_{\lambda,2}(r)$.
Finding a solution is a little more complicated.
The power series method (also known as the method of Frobenius)
allows to find such solutions.
Unfortunately, there is no expression of the solutions in terms of
well known functions.

For the function $\Phi(\varphi)$, the ordinary differential equation becomes
\begin{equation}
\Phi''(\varphi) = -\lambda^2\Phi(\varphi),
\label{40000024:phieqn}
\end{equation}
which is a oscillation differential equation.

\item
The equation \eqref{40000024:phieqn} has solutions
\begin{equation}
\cos(\lambda\varphi)
\qquad\text{and}\qquad
\sin(\lambda\varphi).
\label{40000024:phisolution}
\end{equation}

\item
Since a solution of \eqref{40000024:phieqn} must be $2\pi$-periodic,
the $\lambda$ in \eqref{40000024:phisolution} must be an integer.
We thus can write $\lambda=k$.
In this step we have already used all homogneous boundary conditions of
the problem.
\item
The solution of the equation~\eqref{40000024:pde} can now constructed
as a superposition in the form
\begin{equation}
\begin{aligned}
u(r,\varphi)
&=
a_{0,1} R_{0,1}(r) + a_{0,2} R_{0,2}(r)
\\
&\qquad
+
\sum_{k=1}^\infty
\bigl(
(a_{k,1}R_{k,1}(r) + a_{k,2}R_{k,2}(r))
\cos k\varphi
+
(b_{k,1}R_{k,1}(r) + b_{k,2}R_{k,2}(r))
\sin k\varphi
\bigr)
\end{aligned}
\label{40000024:series}
\qedhere
\end{equation}
\end{teilaufgaben}
\end{loesung}

\begin{diskussion}
The following discussion shows that we can in fact find out a little more
about the solution.

For any fixed radius $r$, the series \eqref{40000024:series} is 
a Fourier series.
On the boundary, the values $R_{k,1}(1)$, $R_{k,2}(1)$, $R_{k,1}(2)$ and
$R_{k,2}(2)$ become coefficients of a linear system of equations to
determine the coefficients $a_{k,i}$ and $b_{k,i}$.
The boundary conditions in polar form~\eqref{40000024:boundarypolar}
are already very simple Fourier series, name fourier series without
nonvanishing terms for $k=1$.

If the boundary conditions were arbitrary functions
\begin{align*}
u(1,\varphi)
&=
f(\varphi)
&&\text{with Fourier series} &
f(\varphi)
&=
\frac{a^f_0}2 + \sum_{k=1}^\infty (a^f_k\cos k\varphi + b^f_k\sin k\varphi)
\\
u(2,\varphi)
&=
g(\varphi)
&&\text{with Fourier series}&
g(\varphi)
&=
\frac{a^g_0}2 + \sum_{k=1}^\infty (a^g_k\cos k\varphi + b^g_k\sin k\varphi),
\end{align*}
then we could conclude that the equations
\[
\begin{aligned}
r&=1:&&&
R_{k,1}(1) a_{k,1} + R_{k,2}(1)a_{k,2} &= a_k^f
\\
 &   &&&
R_{k,1}(1) b_{k,1} + R_{k,2}(1)b_{k,2} &= b_k^f
\\[5pt]
r&=2:&&&
R_{k,1}(2) a_{k,1} + R_{k,2}(2)a_{k,2} &= a_k^g
\\
 &   &&&
R_{k,1}(2) b_{k,1} + R_{k,2}(2)b_{k,2} &= b_k^g
\end{aligned}
\]
must hold.
These are two linear systems of equations.

In the particular case of the problem, most of the coefficients
$a_k^f$, $b_k^f$, $a_k^g$ and $b_k^g$ are zero.
If both coefficents on the right hand side of the linear system
of equations are zero, so are the corresponding coefficents
$a_{k,1}$, $a_{k,2}$, $b_{k,1}$ and $b_{k,2}$.
This happens in all equations except for $k=1$.

For $k=1$, the equations are
\[
\begin{linsys}{3}
a_{1,1}R_{1,1}(1) &+& a_{1,2}R_{1,2}(1) &=& 1 \\
a_{1,1}R_{1,1}(2) &+& a_{1,2}R_{1,2}(2) &=& 0
\end{linsys}
\qquad\text{and}\qquad
\begin{linsys}{3}
b_{1,1}R_{1,1}(1) &+& b_{1,2}R_{1,2}(1) &=& 0\phantom{.} \\
b_{1,1}R_{1,1}(2) &+& b_{1,2}R_{1,2}(2) &=& 1.
\end{linsys}
\]
They are not homogeneous, so there will be a unique solution for
the coefficients $a_{1,1}$, $a_{1,2}$, $b_{1,1}$ and $b_{1,2}$.
This then gives the final solution of the equation in the form
\[
u(r,\varphi)
=
(
a_{1,1}R_{1,1}(r)
+
a_{1,2}R_{1,2}(r)
)
\cos\varphi
+
(
b_{1,1}R_{1,1}(r)
+
b_{1,2}R_{1,2}(r)
)
\sin\varphi.
\]
So it is sufficient to find the solutions of the equation
\eqref{40000024:Rdiff}
for $\lambda=1$ in order to give a complete solution of the
partial differential equation.
Unfortunately, there is no closed form expression for the solutions
of
\eqref{40000024:Rdiff}.
\end{diskussion}

\begin{bewertung}
\begin{teilaufgaben}
\item Elliptic ({\bf E}) 1 point,
\item separation ansatz, must match domain shape, i.~e.~use polar
coordinates ({\bf A}) 1 point,
separated equations ({\bf G}),
\item solution of oscillation equation ({\bf S}) 1 point,
\item separation constant from periodicity ({\bf P}) 1 point,
\item superposition formula for $u$ ({\bf U}) 1 point.
\end{teilaufgaben}
\end{bewertung}
