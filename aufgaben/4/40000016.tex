Gegeben ist die nichtlineare partielle Differentialgleichung 
zweiter Ordnung
\begin{equation}
\biggl(1+\biggl(\frac{\partial u}{\partial y}\biggr)^2\biggr)
\frac{\partial^2 u}{\partial x^2}
-2
\frac{\partial u}{\partial y}
\frac{\partial u}{\partial x}
\frac{\partial^2 u}{\partial x\partial y}
+
\biggl(1+\biggl(\frac{\partial u}{\partial x}\biggr)^2\biggr)
\frac{\partial^2 u}{\partial y^2}
\label{40000016:dgl}
\end{equation}
für die Funktion $u(x,y)$ auf dem Gebiet $\Omega =\mathbb R^2$.
Die Funktion $u$ soll zudem die Eigenschaften
\[
u(0,0)=0
\qquad\text{und}\qquad
\operatorname{grad} u(0,0)=0
\]
haben.
\begin{teilaufgaben}
\item
Führen Sie Separation durch mit Hilfe eines Ansatzes der Form
\begin{equation}
u(x,y) = X(x) + Y(y)
\label{40000016:ansatz}
\end{equation}
und stellen Sie Differentialgleichungen und Anfangsbedingungen
für $X(x)$ und $Y(y)$ auf.
\item
Legen die Vorgaben die Separationslösung eindeutig fest?
\end{teilaufgaben}

\begin{loesung}
\begin{teilaufgaben}
\item
Die partiellen Ableitungen von $u$ nach dem Ansatz \eqref{40000016:ansatz}
sind:
\begin{equation*}
\begin{aligned}
\frac{\partial u}{\partial x}
&=
X'(x),
&
\frac{\partial u}{\partial y}
&=
Y'(y),
\\
\frac{\partial^2 u}{\partial x^2}
&=
X''(x),
&
\frac{\partial^2 u}{\partial y^2}
&=
Y''(y),
&
\frac{\partial^2 u}{\partial x\partial y}
&=
0.
\end{aligned}
\end{equation*}
Einsetzen derselben in die Differentialgleichung \eqref{40000016:dgl} 
ergibt
\begin{equation}
(1+Y'(y)^2)\, X''(x) + (1+X'(x)^2)\, Y''(y) = 0.
\end{equation}
Bringt man Terme mit $x$ auf die linke Seite und Terme mit $y$ auf die
rechte Seite, erhält man
\begin{equation}
\frac{X''(x)}{1+X'(x)^2}
=
-\frac{Y''(y)}{1+Y'(y)^2}.
\end{equation}
Da die Variablen separiert sind, kann man schliessen, dass beide Seiten
der Gleichung konstant sein müssen.
Mit der neuen Konstanten $c$ erhält man also die Gleichungen
\begin{align*}
\frac{X''(x)}{1+X'(x)^2} &= c
&
&\text{und}&
\frac{Y''(y)}{1+Y'(y)^2} &= -c
\\
\intertext{mit den Anfangsbedingungen}
X(0) &= 0
&&&
Y(0) &= 0
\\
X'(0) &= 0
&&&
Y'(0) &= 0.
\end{align*}
\item
Diese Anfangsbedingungen legen die Lösung $X(x)$ und $Y(y)$ eindeutig
fest, und damit auch die Separationslösung $u(x,y)=X(x) + Y(y)$.
\qedhere
\end{teilaufgaben}
\end{loesung}

\begin{diskussion}
Die Differntialgleichung \eqref{40000016:dgl} ist die Bedingung, dass der
Graph der Funktion $u(x,y)$ mittlere Krümmung $0$ hat, er wird damit eine
Minimalfläche.
Diese Minimalfläche wurde von H.~F.~Scherk 1834 gefunden.
\end{diskussion}
