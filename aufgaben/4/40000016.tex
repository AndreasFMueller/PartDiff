Gegeben ist die nichtlineare partielle Differentialgleichung 
zweiter Ordnung
\begin{equation}
\biggl(1+\biggl(\frac{\partial u}{\partial y}\biggr)^2\biggr)
\frac{\partial^2 u}{\partial x^2}
-2
\frac{\partial u}{\partial y}
\frac{\partial u}{\partial x}
\frac{\partial^2 u}{\partial x\partial y}
+
\biggl(1+\biggl(\frac{\partial u}{\partial x}\biggr)^2\biggr)
\frac{\partial^2 u}{\partial y^2}
=0
\label{40000016:dgl}
\end{equation}
für die Funktion $u(x,y)$ auf dem Gebiet $\Omega =\mathbb R^2$.
Die Funktion $u$ soll zudem die Eigenschaften
\[
u(0,0)=0
\qquad\text{und}\qquad
\operatorname{grad} u(0,0)=0
\]
haben.
\begin{teilaufgaben}
\item
Führen Sie Separation durch mit Hilfe eines Ansatzes der Form
$
u(x,y) = X(x) + Y(y)
$
und stellen Sie Differentialgleichungen und Anfangsbedingungen
für $X(x)$ und $Y(y)$ auf.
\item
Legen die Vorgaben die Separationslösung eindeutig fest?
\end{teilaufgaben}

\begin{loesung}
\begin{teilaufgaben}
\item
Die partiellen Ableitungen von $u$ entsprechend dem Ansatz sind:
\begin{equation*}
\begin{aligned}
\frac{\partial u}{\partial x}
&=
X'(x),
&
\frac{\partial u}{\partial y}
&=
Y'(y),
\\
\frac{\partial^2 u}{\partial x^2}
&=
X''(x),
&
\frac{\partial^2 u}{\partial y^2}
&=
Y''(y),
&
\frac{\partial^2 u}{\partial x\partial y}
&=
0.
\end{aligned}
\end{equation*}
Einsetzen derselben in die Differentialgleichung \eqref{40000016:dgl} 
ergibt
\begin{equation}
(1+Y'(y)^2)\, X''(x) + (1+X'(x)^2)\, Y''(y) = 0.
\label{40000016:dgl2}
\end{equation}
Bringt man Terme mit $x$ auf die linke Seite und Terme mit $y$ auf die
rechte Seite, erhält man
\begin{equation}
\frac{X''(x)}{1+X'(x)^2}
=
-\frac{Y''(y)}{1+Y'(y)^2}.
\label{40000016:sep}
\end{equation}
Da die Variablen separiert sind, kann man schliessen, dass beide Seiten
der Gleichung konstant sein müssen.
Mit der neuen Konstanten $c$ erhält man also die Gleichungen
\begin{align}
\qquad\qquad\qquad\qquad
\frac{X''(x)}{1+X'(x)^2} &= c
&
&\text{und}&
\frac{Y''(y)}{1+Y'(y)^2} &= -c
\qquad\qquad\qquad\qquad
\label{40000016:loes}
\\
\intertext{mit den Anfangsbedingungen}
X(0) &= 0
&&&
Y(0) &= 0
\notag
\\
X'(0) &= 0
&&&
Y'(0) &= 0.
\notag
\end{align}
\item
Diese Anfangsbedingungen legen die Lösung $X(x)$ und $Y(y)$
der gewöhnlichen Differentialgleichungen
\eqref{40000016:loes}
eindeutig
fest aber nicht die Konstante $c$.
Es gibt also eine ganze Familie von Separationslösungen, die durch
die Konstante $c$ parametrisiert sind.
\end{teilaufgaben}

Man beachte, dass die Anfangsbedingungen $X(0)=0$ und $Y(0)=0$ nicht aus
den gegebenen Informationen folgen.
Es folgt aber, dass $X(0) + Y(0) = 0$ sein muss.
Ist aber $X(x)$ eine Lösung mit $X(0)=a$, dann ist $\tilde X(x) = X(x)-a$
eine Lösung mit $\tilde X(0)=0$.
Damit man mit diesem $X(x)$ eine Separationslösung konstruieren kann,
müssten $Y(y)$ eine Lösung mit $Y(0)=-a$ sein, dann ist aber
wiederum $\tilde Y(y) = Y(y)+a$ eine Lösung mit $\tilde Y(0)=0$.
Die Summe der Funktionen ist jeweils eine Separationslösung, aber die
beiden Summen
\[
\tilde X(x)+\tilde Y(y)
=
X(x)-a + Y(y)+1
=
X(x)+Y(y)
\]
sind gleich.
Eine andere Anfangsbedingung für $X(0)$ führt also zwar auf
andere Funktionen $X(x)$ und $Y(y)$ aber trotzdem auf die gleiche
Separationslösung.
\qedhere
\end{loesung}

\begin{diskussion}
Die Differentialgleichung \eqref{40000016:dgl} ist die Bedingung, dass der
Graph der Funktion $u(x,y)$ mittlere Krümmung $0$ hat, er wird damit eine
Minimalfläche.
Diese Minimalfläche wurde von H.~F.~Scherk 1834 gefunden.
\end{diskussion}

\begin{bewertung}
Einsetzen des Separationsansatzes ({\bf E}) 1 Punkt,
Differentialgleichung \eqref{40000016:dgl2} ({\bf D}) 1 Punkt,
Separation \eqref{40000016:sep} ({\bf S}) 1 Punkt,
Einzelgleichungen ({\bf C}) 1 Punkt,
Anfangsbedingungen ({\bf A}) 1 Punkt,
Eindeutigkeit ({\bf B}) 1 Punkt.
\end{bewertung}




