On the domain
\begin{equation}
\Omega = \{ (x,y)\mid y>0\;\text{and}\; x^2+y^2< 1\},
\end{equation}
consider the partial differential equation
\begin{equation}
\Delta u =  -\lambda u
\label{40000028:pde}
\end{equation}
with homogeneous boundary conditions.
Obtain two ordinary differential equations with initial or boundary
conditions from which solutions to the equation~\eqref{40000028:pde}
can be constructed.
Also find all solutions of one of those equations.

\begin{loesung}
Using polar coordinates, the domain becomes
\[
\Omega = \{ (\varphi,r)\mid 0<\varphi<\pi\;\text{and}\; 0<r<1\}.
\]
This means that a separation ansatz of the form
$u(r,\varphi)=R(r)\Phi(\varphi)$ can be used.
The Laplace operator in polar coordinates is
\[
\Delta
=
\frac1r \frac{\partial}{\partial r}r\frac{\partial }{\partial r}
+
\frac1{r^2}\frac{\partial^2}{\partial r^2}.
\]
Using the ansatz, we get
\[
\Delta u
=
\frac{1}{r}\frac{\partial}{\partial r} rR'(r)\Phi(\varphi)
+
\frac{1}{r^2}R(r)\Phi''(\varphi)
=
\frac{1}{r}(R'(r)+rR''(r))\Phi(\varphi)
+
\frac{1}{r^2}R(r)\Phi''(\varphi)
=
-\lambda R(r)\Phi(\varphi)
\]
Dividing by $u(r,\varphi)$ and multiplying by $r^2$ gives
\[
\frac{rR'(r)+r^2R''(r)+\lambda r^2R(r)}{R(r)}
+
\frac{\Phi''(\varphi)}{\Phi(\varphi)}
=
0.
\]
This allows to separate the variables $r$ and $\varphi$, giving
\[
\frac{rR'(r)+r^2R''(r)+\lambda r^2R(r)}{R(r)}
=-
\frac{\Phi''(\varphi)}{\Phi(\varphi)}
\;\Rightarrow
\left\{
\;
\begin{aligned}
rR'(r)+r^2R''(r)+\lambda r^2R(r) &= \mu R(r) \\
\Phi''(\varphi) &= -\mu \Phi(\varphi).
\end{aligned}
\right.
\]
The boundary conditions are
\begin{equation}
\Phi(0)=\Phi(\pi)=0 
\qquad\text{and}\qquad
   R(0)=R(1)= 0.
\end{equation}
The $\Phi$-equation is a harmonic oscillation, so the solution must be
\[
\Phi(\varphi) = A \cos\sqrt{\mu}\varphi + B \sin\sqrt{\mu}\varphi.
\]
At the boundary $\varphi=0$, $0=\Phi(0)=A\cos\sqrt{\mu}0 = A$.
At the boundary $\varphi=\pi$ we get
\[
\Phi(\pi) = B\sin\sqrt{\mu}\pi = 0
\qquad\Rightarrow\qquad
\sqrt{\mu}\in\mathbb{N}
\qquad\Rightarrow\qquad
\mu = k^2,\;k\in\mathbb{N}.
\]
So all the solutions are the functions $\Phi_k(\varphi)=\sin k\varphi$.
This implies that $R_k(r)$  must be a solution of the equation
\[
r^2R''(r) + rR'(r) +(\lambda r^2-k^2) R(r) = 0.
\]
This is the Bessel equation which can be solved using Bessel functions.
\end{loesung}

\begin{bewertung}
Separation ansatz ({\bf A}) 1 point,
equation for $R(r)$ ({\bf R}) 1 point,
boundary conditions for $R(r)$ ({\bf B}) 1 point,
equation for $\Phi(\varphi)$ with boundary conditions ($\Phi$) 1 point,
solution for $\Phi(\varphi)$ ({\bf S}) 1 point,
admissible values of $\mu$ ({\bf M}) 1 point.
\end{bewertung}
