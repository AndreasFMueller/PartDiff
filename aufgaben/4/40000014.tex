Auf dem Gebiet $\Omega\subset\mathbb R^2$ berandet von den vier Parabeln
\[
\begin{aligned}
y&=\frac12\biggl( \frac{x^2}{s^2}-s^2\biggr)&&\text{mit}&s&=\frac12,1
	&&\qquad\text{({\color{green}grüne} Kurven)}\\
\text{und}\qquad
y&=\frac12\biggl(-\frac{x^2}{t^2}+t^2\biggr)&&\text{mit}&t&=1,\sqrt{2}
	&&\qquad\text{({\color{red}rote} Kurven)}
\end{aligned}
\]
\begin{figure}
\centering
\includeagraphics[]{domain-1.pdf}
\caption{Gebiet $\Omega$ für Aufgabe~\ref{40000014}
\label{40000014:domain}}
\end{figure}%
(Abbildung~\ref{40000014:domain})
soll das Eigenwertproblem
\begin{equation}
\Delta u = \lambda u
\label{40000014:dgl}
\end{equation}
mit homogenen Randbedingungen gelöst werden.
Aus der Wikipedia erfahren Sie, dass die genannten Kurven 
die Koordinatenlinien sogenannter parabolischer Koordinaten $(s,t)$ sind,
die mit den Formeln
\begin{align*}
x&=st\\
y&=\frac12(t^2-s^2)
\end{align*}
in kartesische Koordinaten umgerechnet werden können.
Ausserdem lässt sich in diesen Koordinaten der Laplace-Operator als
\[
\Delta u = \frac1{s^2 + t^2}\biggl(\frac{\partial^2 u}{\partial s^2}+\frac{\partial^2u}{\partial t^2}\biggr)
\]
schreiben.
Verwenden Sie diese Informationen für einen Separationsansatz, und
stellen Sie Differentialgleichungen und Randbedingungen dafür auf.

\begin{hinweis}
Es wird nicht verlangt, die Differentialgleichungen zu lösen.
\end{hinweis}

\begin{loesung}
Zunächst überprüfen wir, dass die Parabeln tatsächlich Koordinatenlinien
sind.
Dazu lösen wir die erste Gleichung der Koordinatensystemdefinition nach $s$
bzw.~nach $t$ auf, und erhalten
\[
s=\frac{x}{t}
\qquad\text{bzw.}\qquad
t=\frac{x}{s}
\]
und setzen dies in die zweite Gleichung ein. 
So erhalten wir
\[
y=\frac12\biggl(\frac{x^2}{s^2}-s^2\biggr)
\qquad\text{bzw.}\qquad
y=\frac12\biggl(t^2 - \frac{x^2}{t^2}\biggr),
\]
also genau die vorgegebenen Gleichungen.

Für das Eigenwertproblem machen wir nun einen Separationsansatz in 
parabolischen Koordinaten, d.~h.~wir setzen an $u(s,t)=S(s)T(t)$, und
setzen dies in die Differentialgleichung \eqref{40000014:dgl} ein.
Wir erhalten
\[
\Delta u
=
\frac1{s^2+t^2}\bigl(S''(s)T(t) + S(s)T''(t)\bigr) = \lambda S(s)T(t).
\]
Wir dividieren durch $S(s)T(t)$ und multiplizieren mit $s^2+t^2$, so
bekommen wir
\[
\frac{S''(s)}{S(s)} + \frac{T''(t)}{T(t)}=\lambda s^2 + \lambda t^2.
\]
In dieser Form können wir die Separation durchführen:
\[
\frac{S''(s)}{S(s)} - \lambda s^2 = - \frac{T''(t)}{T(t)} + \lambda t^2.
\]
Da die linke Seite nur von $s$, die rechte nur von $t$ abhängt, müssen
beide Seiten konstant sein, wir nennen die Konstante $\mu$.
So erhalten wir die zwei Differentialgleichungen
\begin{equation}
S''(s)=(\lambda s^2+\mu)S(s)
\qquad\text{und}\qquad
T''(t) = (\lambda t^2-\mu)T(t).
\label{40000014:sepdgl}
\end{equation}
Dies sind gewöhnliche Differentialgleichungen zweiter Ordnung für die
Funktionen $S(s)$ und $T(t)$.
Die Funktionen sollen homogene Randbedingungen erfüllen, es muss daher
verlangt werden, dass
\begin{equation}
S({\textstyle \frac12})=S(1)=0
\qquad\text{und}\qquad
T(1)=T(\sqrt{2})=0.
\label{40000014:seprb}
\end{equation}
Die Eigenwerte sind diejenigen Werte von $\lambda$, für die es ein passendes
$\mu$ gibt, so dass sich Funktion $S(s)$ und $T(t)$ finden lassen, die
die Differentialgleichungen~\eqref{40000014:sepdgl} und die
Randbedingungen~\eqref{40000014:seprb} erfüllen.
\end{loesung}

\begin{bewertung}
Separationsansatz mit Produkt $u(s,t) = T(t)S(s)$ ({\bf A}) 1 Punkt,
Einsetzen in die Differentialgleichung ({\bf E}) 1 Punkt,
Trennung der Variablen ({\bf A}) 1 Punkt,
Differentialgleichung für $T(t)$ ({\bf T}) 1 Punkt,
Differentialgleichung für $S(s)$ ({\bf S}) 1 Punkt,
Randbedingungen für $S$ und $T$ ({\bf R}) 1 Punkt.
\end{bewertung}



