Auf dem Quadrat
\begin{equation}
\Omega = \{ (x,y)\,|\, 0<x<\pi \;\text{und}\; 0 < y < \pi \}
\label{40000019:domain}
\end{equation}
soll die Differentialgleichung
\begin{equation}
\Delta u = \cosh(\!\sqrt{2}x\,)\cdot \sin y
\label{40000019:pde}
\end{equation}
mit den Randbedingungen $u=0$ auf $\partial\Omega$ gelöst werden.
Die Lösung soll in der Form $u=u_p+u_h$ gefunden werden, wobei
$u_h$ eine Lösung der homogenen Gleichung $\Delta u_h=0$ ist.
\begin{teilaufgaben}
\item
Finden sie eine partikuläre Lösung $u_p$.
\item
Bestimmen Sie die Randbedingungen von $u_h$.
\item
Verwenden Sie einen Separationsansatz für $u_h$, stellen Sie die
Differentialgleichungen und Randbedinungen für die Faktoren auf.
\item
Lösen Sie die Differentialgleichungen.
\item
Führen Sie den Koeffizientenvergleich durch und bestimmen sie $u_h$.
\end{teilaufgaben}


\begin{loesung}
\begin{teilaufgaben}
\item
Dies ist eine inhomogene Differentialgleichung, wir brauchen also zunächst
eine partikuläre Lösung.
Die rechte Seite der Differentialgleichung ist eine Funktion, die bis auf
einen konstanten Faktor die Differentialgleichung erfüllt.
Es gilt nämlich
\[
\Delta (
\cosh \!\sqrt{2}x\cdot \sin y
)
=
2\cosh \!\sqrt{2}x\cdot \sin y
-\cosh\!\sqrt{2}x\cdot\sin y
=
\cosh \!\sqrt{2}x\cdot \sin y
\]
Also ist
\[
u_p(x,y)
=
\cosh \!\sqrt{2}x\cdot \sin y
\]
eine partikuläre Lösung der Differentialgleichung.
\item
\begin{figure}
\centering
\includeagraphics[]{domain.pdf}
\caption{Gebiet für die Aufgaben~\ref{40000019}
\label{40000019:gebiet}}
\end{figure}
Auf dem Rand des Gebietes $\Omega$ sind die Funktionswerte von $u_p$:
\[
u_p(x,0) = u_p(x,\pi)=0,
\quad
u_p(0,y) = \sin y
\quad\text{und}\quad
u_p(\pi,y) = \cosh\!\sqrt{2}x\cdot \sin y.
\]
Wir suchen daher eine Lösung der homogenen Gleichung $\Delta u_h = 0$
mit den Randbedingungen
\[
u_h(0,y) = u_h(\pi,y) = 0,
\quad
u_h(0,y) = -\sin y
\quad\text{und}\quad
u_h(\pi,y) = -\cosh\!\sqrt{2}\pi\cdot \sin y
\]
(Siehe auch Abbildung~\ref{40000019:gebiet}.)
\item
Zur Bestimmung der Lösung $u_h$ der homogenen Gleichung verwenden wir einen
Separationsansatz, wir setzen
\[
u(x,y) = X(x)\cdot Y(y).
\]
Diesen Ansatz setzen wir in die Differentialgleichung ein und erhalten
\[
X''(x)Y(y) + X(x)Y''(y)=0
\qquad\Rightarrow\qquad
\frac{X''(x)}{X(x)} = -\frac{Y''(y)}{Y(y)}
\]
Die Separationsmethode besagt, dass die Seiten der Gleichung konstant
sein müssen, also
\[
X''(x) = -\mu X(x)
\qquad\text{und}\qquad
Y''(y) = \mu Y(y).
\]
Je nach Vorzeichen von $\mu$ sind die Lösungen trigonometrische Funktionen
oder Exponentialfunktionen, die Randbedingungen müssen über das Vorzeichen
von $\mu$ entscheiden.
\item
Die homogenen Randbedingungen lauten für $Y$: 
\[
Y(0)=0
\qquad\text{und}\qquad
Y(\pi)=0.
\]
Diese lassen sich nur mit der Lösung $\sin kx$ mit ganzzahligem $k$
erfüllen.
Demzufolge hat die Gleichung für $X(x)$ die Lösungen $e^{kx}$ und $e^{-kx}$.
Die allgemeine Lösung für $u_h$ ist daher
\[
u_h(x,y)
=
\sum_{k=1}^\infty (a_k e^{kx} + b_k e^{-kx})\sin ky
\]
mit noch zu bestimmenden Koeffizienten $a_k$ und $b_k$.
Diese Lösung erfüllt bereits die Randbedingungen für $y=0$ und $y=\pi$.
\item
Es bleiben also noch die Randbedingungen für $x=0$ und $x=\pi$ zu erfüllen:
\begin{equation}
\begin{aligned}
u_h(0,y)   &= \sum_{k=1}^\infty \sin ky \cdot (a_k + b_k)&&=-\sin y
\\
u_h(\pi,y) &= \sum_{k=1}^\infty \sin ky \cdot (a_ke^{k\pi} + b_ke^{-k\pi}) &&=-\cosh\!\sqrt{2}\pi\cdot \sin y
\end{aligned}
\end{equation}
Koeffizientenvergleich zeigt, dass diese
Gleichungen nur für $k=1$ erfüllt sein können, daher muss
\begin{equation}
\begin{linsys}{2}
       a_1&+&        b_1&=& -1\\
e^{\pi}a_1&+&e^{-\pi}b_1&=& -\cosh\!\sqrt{2}\pi 
\end{linsys}
\end{equation}
gelten.
Alle anderen Koeffizienten $a_k$ und $b_k$ mit $k\ne 1$ verschwinden.
Die Koeffizienten $a_1$ und $b_1$ kann man zum Beispiel mit der Kramerschen
Regel finden:
\begin{align*}
a_1
&=
\frac{-e^{-\pi}+\cosh\!\sqrt{2}\pi}{e^{-\pi}-e^{\pi}},
\\
b_1
&=
\frac{-\cosh\!\sqrt{2}\pi-e^{\pi}}{e^{-\pi}-e^{\pi}}.
\end{align*}
Damit erhalten wir für die Lösung der homogenen Gleichung
\[
u_h(x,y)
=
\frac{-e^{-\pi}+\cosh\!\sqrt{2}\pi}{e^{-\pi}-e^{\pi}}\cdot e^{kx} \sin y
+
\frac{-\cosh\!\sqrt{2}\pi-e^{\pi}}{e^{-\pi}-e^{\pi}}\cdot e^{-kx} \sin y
\]
und für die Lösung der Differentialgleichung
\[
u
=
u_p + u_h
=
\cosh \!\sqrt{2}x\cdot \sin y
+
\frac{ \sin y}{e^{-\pi}-e^{\pi}}
\cdot \biggl(
(-e^{-\pi}+\cosh\!\sqrt{2}\pi) e^{kx}
+
(-\cosh\!\sqrt{2}\pi-e^{\pi}) e^{-kx}
\biggr).
\qedhere
\]
\end{teilaufgaben}
\end{loesung}

\begin{bewertung}
\begin{teilaufgaben}
\item
Partikuläre Lösung ({\bf P}) 1 Punkt,
\item
Randbedingungen für $u_h$ ({\bf R}) 1 Punkt,
\item
Separation ({\bf S}) 1 Punkt,
\item
Lösungen für $Y(y)$ und Ganzzahligkeitsbedingung ({\bf Y}) 1 Punkt,
Lösungen für $X(x)$ ({\bf X}) 1 Punkt,
\item
Koeffizientenvergleich und Lösung ({\bf K}) 1 Punkt.
\end{teilaufgaben}
\end{bewertung}




