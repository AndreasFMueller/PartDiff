L"osen Sie die partielle Differentialgleichung
\[
\frac{\partial^2}{\partial x^2}u+\frac{\partial^2}{\partial y^2}u
+2\frac{\partial }{\partial y}u+u=0
\]
auf dem Gebiet $0 < x < \pi$, $y > 0$ mit den Randbedingungen
\[
u(0,y)=u(\pi,y)=0
\quad
\quad
u(x,0)=\sin x,\quad \frac{\partial }{\partial n}u(x,0)=\sin 3x
\]

\begin{loesung}
Ein Separationsansatz $u(x,y)=X(x)Y(y)$ liefert
\begin{align*}
X''(x)Y(y)+X(x)Y''(y)+2X(x)Y'(y)+X(x)Y(y)&=0
\\
\frac{Y''(y)+2Y'(y)+Y(y)}{Y(y)}
&=
-\frac{X''(x)}{X(x)}
\end{align*}
Daraus folgen die beiden gew"ohnlichen Differentialgleichungen
\begin{align*}
X''(x)&=-\lambda X(x)\\
Y''(y)+2Y'(y)+(1-\lambda)Y(y)&=0
\end{align*}
Die erste Gleichung hat f"ur positives $\lambda$ die L"osungen
$\sin\sqrt{\lambda}x$, diese erf"ullen die Randbedingung f"ur $x=0$
und $x=\pi$ aber nur, wenn $\sqrt{\lambda}$ eine Ganzzahl ist, also
$\lambda=n^2$. F"ur negatives $\lambda$ gibt es keine L"osung,
die die Randbedingungen respektiert. Daher wird die zweite Gleichung
\[
Y''(y)+2Y'(y)+(1-n^2)Y(y)=0
\]
Dies ist eine lineare Differentialgleichung mit konstanten
Koeffizienten, die mit dem "ublichen Verfahren gel"ost werden
kann. Das charakteristische Polynom ist
\[
\omega^2+2\omega+(1-n^2)=0,
\]
es hat die Nullstellen
\[
\omega_{\pm}=-1\pm{\sqrt{1-(1-n^2)}}=-1\pm n.
\]
Die allgmeine L"osung lautet also
\[
u_n(x,y)=(a_{n,+}e^{(-1+n)y}+a_{n,-}e^{(-1-n)y})\sin nx.
\]
Die Exponentialfaktoren lassen sich aber eleganter
mit den hyperbolischen Funktionen ausdr"ucken:
\[
u_n(x,y)=(a_n\cosh ny+b_n\sinh ny)e^{-y}\sin nx.
\]
Tats"achlich kann man nachrechen, dass diese Funktionen
tats"achlich L"osungen sind. Die Ableitungen sind
\begin{align*}
\frac{\partial}{\partial x}u_n(x,y)
&=
n (a_n\cosh ny+b_n\sinh ny)e^{-y}\cos nx
\\
\frac{\partial^2}{\partial x^2}u_n(x,y)
&=
-n^2 (a_n\cosh ny+b_n\sinh ny)e^{-y}\sin nx
\\
\frac{\partial}{\partial y}u_n(x,y)
&=
-(a_n\cosh ny+b_n\sinh ny)e^{-y}\sin nx
+
n(a_n\sinh ny+b_n\cosh ny)e^{-y}\sin nx
\\
\frac{\partial^2}{\partial y^2}u_n(x,y)
&=
(a_n\cosh ny+b_n\sinh ny)e^{-y}\sin nx
-2n(a_n\sinh ny+b_n\cosh ny)e^{-y}\sin nx
\\
&\qquad
+n^2(a_n\cosh ny+b_n\sinh ny)e^{-y}\sin nx
\\
\end{align*}
Setzt man diese in die Differentialgleichung
ein, ergibt sich
\begin{align*}
-n^2 (a_n\cosh ny+b_n\sinh ny)e^{-y}\sin nx&\\
+(a_n\cosh ny+b_n\sinh ny)e^{-y}\sin nx&\\
-2n(a_n\sinh ny+b_n\cosh ny)e^{-y}\sin nx&\\
+n^2(a_n\cosh ny+b_n\sinh ny)e^{-y}\sin nx&\\
+2(
-(a_n\cosh ny+b_n\sinh ny)e^{-y}\sin nx&\\
+
n(a_n\sinh ny+b_n\cosh ny)e^{-y}\sin nx
)&\\
+
(a_n\cosh ny+b_n\sinh ny)e^{-y}\sin nx
&=0
\end{align*}
In dieser Form ist die Anfangsbedingung etwas "ubersichtlicher
zu erf"ullen, weil  $\sinh n0=0$ und $\cosh n0=1$. Die Normalableitung
am Rand $y=0$ ist die Ableitung in Richtung $y$, also
\begin{align*}
u_n(x,0)&=a_n\sin nx
\\
\frac{\partial}{\partial n}u_n(x,0)&=(-a_n+nb_n)\sin nx
\end{align*}
Man kann also zun"achst aus den Werten f"ur $y=0$ das $a_n$
ermitteln, und dann mit Hilfe der Normalableitungen f"ur $y=0$
daraus die $b_n$.

Im vorliegenden Fall ist $a_1=1$ der einzige $a$-Koeffizient,
der nicht verschwindet. Andererseits braucht man zur Erf"ullung
der Bedingung an die Ableitung den Fall $n=3$. Es folgt
\begin{align*}
a_1&=1&&\\
-a_1+1\cdot b_1&=0&\Rightarrow\quad b_1&=1\\
a_3&=0&&\\
-a_3+3b_3&=1&\Rightarrow\quad b_3&=\frac13
\end{align*}
Offenbar ist die gesuchte L"osung also
\[
u(x,y)=e^{-y}(\sin x \cosh y +
\sin x\sinh y
+
\frac13\sin 3x \sinh 3y),
\]
wie man auch durch Nachrechnen zum Beispiel mit einem Computeralgebrasystem
"uberpr"ufen kann.
\end{loesung}

