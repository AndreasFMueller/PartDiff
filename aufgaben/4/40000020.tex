The air column in a wind instrument with conical bore can be described
as a cone with center at the origin where the mouthpiece is located.
The oboe fits this model quite well, a saxophone also has a conical bore
although twisted into an S-shape.
The most suitable coordinate system for this problem is spherical coordinates.
One can safely assume that the density of the air does not depend on the angles
$\varphi$ and $\vartheta$, so that it sufficies to model the air
density $\varrho$ as a function of time and the radius alone, i.~e.~as
a function $\varrho(t,r)$.
More precisely, $\varrho(t,r)$ is the density deviation with respect 
to the density of the air surrounding the instrument.
The wave equation then becomes
\begin{equation}
\frac{\partial^2}{\partial t^2}\varrho
-
a^2\biggl(
\frac{\partial^2}{\partial r^2}
+\frac2r\frac{\partial}{\partial r}
\biggr)\varrho
=
0,
\label{40000020:equation}
\end{equation}
where $a$ is the speed of sound.
The boundary conditions are 
\begin{align*}
\frac{\partial\varrho}{\partial r}(t,0)&=0&&\text{(instrument closed at mouthpiece)}
\\
\varrho(t,2\pi)&=0&&\text{(density of surrounding air at the bell of the instrument)}
\end{align*}
In addition, initial conditions at $t=0$ are given as
\[
\begin{aligned}
\varrho(0,r) &= f(r)
&
&\text{and}&
\frac{\partial}{\partial t} \varrho(0,r) &= g(r)
\end{aligned}
\]
for $r\in(0,2\pi)$.

\begin{teilaufgaben}
\item
Perform separation of variables and find ordinary differential equations
that can be used to find a solution for this partial differential equation.
Solve one of those equations.
\item
Write the factor $R(r)$ in the separation as $p(r)/r$ and use this
to simplify the equation for $R(r)$ into an equation for $p(r)$.
\ifthenelse{\boolean{loesungen}}{
\item
Solve the equation for $R(r)$ and determine the possible frequencies
of the oboe.
}{}
\end{teilaufgaben}

\begin{loesung}
\begin{teilaufgaben}
\item
We substitute the separation ansatz $\varrho(t,r)=T(t)R(r)$ in the
differential equation \eqref{40000020:equation} and get
\begin{align*}
T''(t) R(r)
-a^2
\biggl(
T(t)R''(t) + \frac2r T(t)R'(r)
\biggr)
&=
0
\\
\frac{T''(t)}{T(t)}
&=
a^2
\biggl(
R''(t) + \frac2r R'(r)
\biggr)
\frac1{R(r)}.
\end{align*}
In this equation, $t$ appears only on the left hand side and $r$ appears
only on the right hand side, the variables are thus separated.
Accordingly, both sides must be constant, we call the constant $-\mu^2$,
because we expect oscillatory solutions.
The equation for $T(t)$ then becomes
\[
T''(t) = -\mu^2 T(t)
\qquad
\text{with solutions}
\qquad
T_\mu(t) = A_\mu \cos \mu t + B_\mu \sin \mu t.
\]
The equation for $R(r)$ becomes
\[
R''(r) +\frac2r R'(r) + \frac{\mu^2}{a^2} R(r) = 0,
\]
which is an ordinary linear differential equation of second
order with boundary conditions
\[
\begin{aligned}
R'(0)&=0
&&\text{and}&
R(2\pi)&=0.
\end{aligned}
\]
\item
The derivatives of $R(r)=p(r)/r$ are
\begin{align*}
R'(r)
&=
\frac{rp'(r)-p(r)}{r^2}
\qquad\text{and}\\
R''(r)
&=
\frac{r^2p''(r) -2rp'(r) + 2p(r)}{r^3}
.
\end{align*}
Substituting these into the differential equation gives
\begin{align*}
0=
R''(r) +\frac2r R'(r) + \frac{\mu^2}{a^2} R(r)
&=
\frac{r^2p''(r) -2rp'(r) + 2p(r)}{r^3}
+2
\frac{rp'(r)-p(r)}{r^2}
+\frac{\mu^2}{a^2}
\frac{p(r)}{r}
\\
&=
\frac1r\biggl( p''(r) +\frac{\mu^2}{a^2} p(r)\biggr)
\\
\Leftrightarrow\qquad
p''(r)
&=
-\frac{\mu^2}{a^2} p(r)
\qquad\Rightarrow\qquad
p(r)
=
C_\mu \cos\frac{\mu}{a} r + D_\mu \sin\frac{\mu}{a} r
\end{align*}
\item
Thus the solution for $R(r)$ becomes
\[
R(r)
= 
C_\mu \frac1r \cos\frac{\mu}{a} r + D_\mu \frac1r \sin\frac{\mu}{a} r
\]
We have to find the derivative of this function at $r=0$.
The second term is an even function, so the derivative at $r=0$ is 0.
The first term does not even have a limit when $r\to 0$, so we conclude
that $C_\mu=0$.

At $r=2\pi$,
\[
R(2\pi)=D_\mu\frac1{2\pi}\sin\frac{\mu}{a}2\pi = 0,
\]
so $\frac{\mu}a 2\pi$ must be an integer multiple of $\pi$, or
\[
\frac{2\pi}{a}\mu\in\pi\mathbb Z
\quad\Rightarrow\quad
\frac{2\mu}{a}\in\mathbb Z
\quad\Rightarrow\quad
\mu = \frac{ka}{2}\quad \text{with $k\in\mathbb Z$}.
\]
Thus the possible frequencies are integer multiples of $a/2$.
\end{teilaufgaben}
\end{loesung}

\begin{diskussion}
Subproblem c) was not part of the exam
\end{diskussion}

\begin{bewertung}
\begin{teilaufgaben}
\item
Separation principle ({\bf S}) 1 point,
solution for $T(t)$ ({\bf T}) 1 point,
separated equation for $R(r)$ ({\bf R}) 1 point,
boundary conditions for $R(r)$  ({\bf B}) 1 point,
\item
derivatives of $R(r)=p(r)/r$ ({\bf D}) 1 point,
separated equation for $p(r)$ ({\bf P}) 1 point.
\end{teilaufgaben}
\end{bewertung}



