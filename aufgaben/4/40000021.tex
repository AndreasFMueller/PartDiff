A siren needs a horn for impedance matching with the surrounding
atmosphere.
The shape of the horn determines, among other things, the directional
characteristics of the emitted sound wave.
A horn with a narrow but tall opening (taller than the wave length of
the emitted sound wave) will have a more horizontal emission characteristic
due to diffraction, which is more suitable for the intended purpose of
a siren.
So a wedge shaped horn is tried, which is the domain
\[
\Omega = \{ (r,\varphi,z) \;|\; 0 < r < r_0, 0< \varphi <\varphi_0, 0 < z < h\}
\]
in cylindrical coordinates, as depicted in the following image.
\begin{center}
\includeagraphics[]{siren.pdf}
\end{center}
A sound wave can be modelled by the wave equation
\[
\frac{\partial^2 \varrho}{\partial t^2} - a^2\Delta \varrho=0.
\]
To evaluate this design,
we are looking for a solution that is independent of $z$ and $\varphi$.
The boundary conditions are homogeneous Neumann conditions at $r=0$ and 
homogeneous Dirichlet conditions at $r=r_0$.

\begin{teilaufgaben}
\item
Write down the differential equation and boundary conditions needed
for this problem expressed in the coordinates $t$ and $r$.
\item
Perform separation of variables to find ordinary differential equations
that can be sued to find a solution of the partial differential equation.
\item
Solve the time equation.
\item
Write the equation for the $r$-dependence as an ordinary, linear, homogeneous
differential equation and specify the boundary conditions.
\end{teilaufgaben}

\begin{loesung}
\begin{teilaufgaben}
\item
The Laplace operator in cylindrical coordinates is
\[
\Delta = 
\frac{1}{r}\frac{\partial}{\partial r}
\biggl(r\frac{\partial}{\partial r}\biggr)
+
\frac1{r^2}\frac{\partial^2}{\partial \varphi^2}
+
\frac{\partial^2}{\partial z^2}.
\]
The last two terms are not needed, because we assume that the solution
does not depend on $z$ and $\varphi$.
So the wave equation becomes
\[
0
=
\frac{\partial^2\varrho}{\partial t^2}
-
a^2\frac{\partial}{\partial r}\biggl(r\frac{\partial\varrho}{\partial r}\biggr)
=
\frac{\partial^2\varrho}{\partial t^2}
-
a^2
\biggl(
\frac{\partial\varrho}{\partial r}
+
r
\frac{\partial^2\varrho}{\partial r^2}
\biggr).
\]
The boundary condition at the sharp edge of the wedge is
\[
\frac{\partial\varrho}{\partial r}(t,0)=0,
\]
the one at the opening is
\[
\varrho(t, r_0) = 0.
\]
\item
We try the separation ansatz
\[
\varrho(t,r) = T(t) \cdot R(r).
\]
Substituting this into the differential equation gives
\[
T''(t) R(r) - a^2 (R'(r)T(t) + rR''(r)T(t)) = 0.
\]
Moving the second term to the right hand side and dividing by $R(r)T(t)$
gives the separated equation
\[
\frac{T''(t)}{T(t)}
=
a^2 \frac{R'(r)+rR''(r)}{R(r)}
=
-\lambda^2,
\]
both sides being constant.
\item
The time equation is 
\[
T'' = -\lambda^2 T,
\]
which is just an oscillation equation with solutions
$A\cos\lambda t+B\sin\lambda t$.
\item
By dividing by $a^2$ and multiplying by $R(r)$ we get the equation
\[
rR''(r) + R'(r) +\frac{\lambda^2}{a^2}R(r)=0.
\]
The boundary conditions are
\begin{align*}
R'(0) &= 0,\\
R(r_0) &= 0.
\qedhere
\end{align*}
\end{teilaufgaben}
\end{loesung}

\begin{bewertung}
\begin{teilaufgaben}
\item Differential equation ({\bf E}) 1 point,
boundary conditions ({\bf B}) 1 point.
\item
Substitution ({\bf S}) 1 point,
separation ({\bf S}) 1 point
\item
Solution to time equation ({\bf T}) 1 point.
\item
Equation with boundary conditions for $R$ ({\bf R}) 1 point.
\end{teilaufgaben}
\end{bewertung}

