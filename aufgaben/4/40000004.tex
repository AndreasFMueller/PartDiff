Lösen Sie die Wellengleichung
\[
\frac{\partial^2 u}{\partial t^2}=\frac{\partial^2 u}{\partial x^2}
\]
für eine zwischen $x=0$ und $x=\pi$ gespannte Saite,
die zur Zeit $t=0$ in Ruhelage ist, aber die Geschwindigkeit
\[
\frac{\partial u}{\partial t}(0, x)
=
\sin^3 x=\frac34\sin x-\frac14\sin 3x
\]
hat.

\begin{hinweis}
Diese Aufgabe kann auch mit der Transformationsmethode gelöst werden,
siehe dazu die Aufgabe \ref{50000004}.
\end{hinweis}

\begin{loesung}
Die Separation der Wellengleichung in $u(t,x)=T(t)X(x)$ ergibt die
zwei Gleichungen
\[
X''(x)=-\mu X(x)\qquad\text{und}\qquad T''(t)=-\mu T(t),
\]
welche als Lösungen die Funktionen
\begin{align*}
X(x)&=\sin\sqrt{\mu}x & T(t)=\sin\sqrt{\mu}t\\
X(x)&=\cos\sqrt{\mu}x & T(t)=\cos\sqrt{\mu}t
\end{align*}
haben, wobei $\mu$ eine positive Zahl sein muss.

Die Randbedingungen für $x=0$ und $x=\pi$ können allerdings nur
für die Funktionen $\sin\sqrt{\mu}x$ erfüllt werden, und auch nur dann,
wenn $\sqrt{\mu}$ eine ganze Zahl ist. Die allgmeine Lösung der Gleichung
hat daher die Form
\[
u(t,x)
=
\sum_{n=1}^\infty a_n\sin nx \sin nt+\sum_{n=1}^\infty b_n\sin nx\cos nt
\]
Darin müssen die Koeffizienten $a_n$ und $b_n$ mit Hilfe der 
Anfangsbedingung zur Zeit $t=0$ bestimmt werden. Man hat
\begin{align*}
u(0,x)&=
\sum_{n=1}^\infty b_n\sin nx
\\
&=0\quad\Rightarrow\quad b_n=0
\\
\frac{\partial u}{\partial t}(0,x)
&=
\sum_{n=1}^\infty a_nn\sin nx
\\
&=
\sin^3 x=\frac34\sin x-\frac14\sin 3x
\end{align*}
Daraus folgt durch Vergleich der Koeffizienten
\begin{align*}
a_1&=\frac34\\
a_3&=-\frac{1}{12}\\
a_k&=0\qquad\text{für $k\ne 1,3$}
\end{align*}
Damit ist die Lösung der Differentialgleichung
\[
u(t,x)
= \frac34\sin x \sin t -\frac1{12}\sin 3x \sin 3t.
\qedhere
\]
\end{loesung}
