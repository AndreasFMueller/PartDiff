L"osen Sie die Wellengleichung
\[
\frac{\partial^2 u}{\partial t^2}=\frac{\partial^2 u}{\partial x^2}
\]
f"ur eine zwischen $x=0$ und $x=\pi$ gespannte Saite,
die zur Zeit $t=0$ in Ruhelage ist, aber die Geschwindigkeit
\[
\frac{\partial u}{\partial t}(0, x)
=
\sin^3 x=\frac34\sin x-\frac14\sin 3x
\]
hat.

\begin{loesung}
Die {\bf Separation} der Wellengleichung in $u(t,x)=T(t)X(x)$ ergibt die
zwei Gleichungen
\[
X''(x)=-\mu X(x)\qquad\text{und}\qquad T''(t)=-\mu T(t),
\]
welche als L"osungen die Funktionen
\begin{align*}
X(x)&=\sin\sqrt{\mu}x & T(t)=\sin\sqrt{\mu}t\\
X(x)&=\cos\sqrt{\mu}x & T(t)=\cos\sqrt{\mu}t
\end{align*}
haben, wobei $\mu$ eine positive Zahl sein muss.

Die Randbedingungen f"ur $x=0$ und $x=\pi$ k"onnen allerdings nur
f"ur die Funktionen $\sin\sqrt{\mu}x$ erf"ullt werden, und auch nur dann,
wenn $\sqrt{\mu}$ eine ganze Zahl ist. Die allgmeine L"osung der Gleichung
hat daher die Form
\[
u(t,x)
=
\sum_{n=1}^\infty a_n\sin nx \sin nt+\sum_{n=1}^\infty b_n\sin nx\cos nt
\]
Darin m"ussen die Koeffizienten $a_n$ und $b_n$ mit Hilfe der 
Anfangsbedingung zur Zeit $t=0$ bestimmt werden. Man hat
\begin{align*}
u(0,x)&=
\sum_{n=1}^\infty b_n\sin nx
\\
&=0\quad\Rightarrow\quad b_n=0
\\
\frac{\partial u}{\partial t}(0,x)
&=
\sum_{n=1}^\infty a_nn\sin nx
\\
&=
\sin^3 x=\frac34\sin x-\frac14\sin 3x
\end{align*}
Daraus folgt durch Vergleich der Koeffizienten
\begin{align*}
a_1&=\frac34\\
a_3&=-\frac{1}{12}\\
a_k&=0\qquad\text{f"ur $k\ne 1,3$}
\end{align*}
Damit ist die L"osung der Differentialgleichung
\[
u(t,x)
= \frac34\sin x \sin t -\frac1{12}\sin 3x \sin 3t.
\]

Alternativ wollen wir jetzt auch noch die L"osung mit der
{\bf Transformationsmethode} darstellen. Die Fourier-Transformation
von Funktionen auf dem Interval $[0,\pi]$, die am Rande
des Intervals verschwinden, ist die Entwicklung in eine Fourier-Sinus-Reihe,
also die Transformation 
\[u(t,x) \mapsto \hat u(t, k)= C\int_0^{\pi} u(t,x)\sin kx\, dx\]
mit einer geeigneten Normierungskonstanten $C$, die im vorliegenden
Fall allerdings unwichtig ist (weil die Berechnung der Fourierreihe
im Hinweis bereits durchgef"uhrt ist).

Die Transformation der Differentialgleichung 
\begin{equation}
\frac{\partial^2}{\partial t^2}\hat u(t,k)
=-k^2\hat u(t,k)
\label{gleichungfueruk}
\end{equation}
Die Fouriertransformation der Anfangsbedingung ist
\begin{align}
\hat u(0,k)&=0\label{40000004:anfbedu}\\
\frac{\partial}{\partial t}\hat u(0,k)&=
\begin{cases}
\frac34&\qquad k= 1\\
-\frac14&\qquad k= 3\\
0&\qquad \text{sonst}
\end{cases}
\label{40000004:anfbedudot}
\end{align}
Die Gleichung
(\ref{gleichungfueruk})
ist eine gew"ohnliche Differentialgleichung f"ur jeden
einzelnen Fourierkoeffizienten, mit allgemeiner L"osung
\[
\hat u(t,k)=A_k\cos kt+B_k\sin kt
\]
Darin sind die Koeffizienten $A_k$ und $B_k$ aus den Anfangsbedingungen
(\ref{40000004:anfbedu}) und (\ref{40000004:anfbedudot}) zu bestimmen:
\begin{align*}
\hat u(0,k)&=A_k=0\\
\partial_t \hat u(0,k)&=kB_k=
\begin{cases}
\frac34&\qquad k= 1\\
-\frac14&\qquad k= 3\\
0&\qquad \text{sonst}
\end{cases}
\end{align*}
Insbesondere gilt
\[
B_1=\frac23,\qquad B_3=-\frac1{12}
\]
Damit sind die zeitabh"angigen Fourierkoeffizienten $\hat u(t,k)$ bestimmt:
\[
\hat u(t,k)=
\begin{cases}
\frac34\sin t &\qquad k= 1\\
-\frac1{12}\sin 3t&\qquad k= 3\\
0&\qquad \text{sonst}
\end{cases}
\]
Summiert man die Fourier-Reihe, ergibt sich
\[
u(t,x)=\sum_{k>0}\hat u(t,k) \sin kx
= \frac34\sin x \sin t -\frac1{12}\sin 3x \sin 3t.
\]
\end{loesung}
