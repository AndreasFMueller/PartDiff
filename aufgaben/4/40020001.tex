Wenden Sie die Hamilton-Jacobi-Theorie auf die Bewegung eines
zweidimensionalen Federpendels mit der Hamilton-Funktion 
\[
H(x,y,p_x,p_y)=\frac1{2m}(p_x^2+p_y^2)+K(x^2 + y^2)
\]
an. Stellen Sie die Hamilton-Jacobi-Differentialgleichung auf
und l"osen Sie sie mit Hilfe eines Separationsansatzes.

\begin{loesung}
Das ``Kochrezept'' der Hamilton-Jacobi-Theorie besagt, dass eine
Funktion $S(t, x, y)$ gesucht werden muss, die L"osung
der Differentialgleichung
\[
\frac1{2m}\biggl(
\biggl(\frac{\partial S}{\partial x}\biggr)^2
+
\biggl(\frac{\partial S}{\partial y}\biggr)^2
\biggr)+K(x^2+y^2)=\frac{\partial S}{\partial t}
\]
sein muss. Wer verwenden einen Separationsansatz der Form
\[
S(t,x,y)=S_1(t)+S_2(x)+S_3(y)
\]
und setzen diesen in die Differentialgleichung ein.
Wir erhalten
\[
\frac1{2m}\bigl( S_2'(x)^2+S_3'(y)^2 \bigr)+K(x^2+y^2)=S_1'(t).
\]
Darin ist die Variable $t$ bereits separiert, sie kommt alleine auf
der rechten Seite vor. Also m"ussen beide Seiten der Gleichung konstant
sein, wir nennen die Konstante $P_1$. Es folgt dann, dass 
$S_3(t)=P_1t.$

Trennen wir in der verbleibenden linken Seite die Variablen $x$ und $y$,
erhalten wir
\[
\frac1{2m}S_2'(x)^2+Kx^2
=
P_1-
\frac1{2m}S_3'(y)^2-Ky^2
\]
Wieder m"ussen beide Seiten konstant sein, wir nennen die Konstanten $P_2$
und erhalten die Differentialgleichungen
\begin{align*}
S_2'(x)&=\sqrt{2mP_2-2mKx^2}=\sqrt{2mK}\sqrt{\frac{P_2}{K}-x^2}\\
S_3'(y)&=\sqrt{2m(P_1-P_2)-2mKy^2}=\sqrt{mK}\sqrt{\frac{P_1-P_2}{K}-y^2}
\end{align*}
Die L"osung findet man durch Integration, in beiden F"allen muss ein 
Integral der Form
\[
\int \sqrt{A-x^2}\,dx
\]
berechnet werden.
Da dies nur f"ur $A>0$ m"oglich ist, schreiben wir $A=a^2$.
In einer guten Formelsammlung finden wir die Stammfunktion
\[
\int\sqrt{a^2-x^2}\,dx
=
\frac{a^2}2\arcsin\frac{x}{a}+\frac{x}2\sqrt{a^2-x^2}
\]
Damit kann man jetzt die Teilfunktionen $S_2(x)$ und $S_3(y)$
zusammenbauen:
\begin{align*}
S_2(x)
%&=
%\sqrt{m}\,\sqrt{K}\,\left({{x\,\sqrt{{{{\it P_2}}\over{K}}-x^2}
% }\over{2}}+{{\arcsin \left({{x\,\sqrt{K}}\over{\sqrt{{\it P_2}}}}
% \right)\,{\it P_2}}\over{2\,K}}\right)
%\\
&=
\frac{x\sqrt{m}}2\sqrt{P_2-Kx^2}
+
\frac{P_2}{2}\sqrt{\frac{m}{K}}\arcsin x\sqrt{\frac{K}{P_2}}
\\
S_3(y)
%&=
%\sqrt{m}\,\sqrt{K}\,\left(-{{\arcsin \left({{y\,\sqrt{K}}\over{
% \sqrt{{\it P_1}-{\it P_2}}}}\right)\,{\it P_2}}\over{2\,K}}+{{y\,
% \sqrt{{{{\it P_1}-{\it P_2}}\over{K}}-y^2}}\over{2}}+{{{\it P_1}\,
% \arcsin \left({{y\,\sqrt{K}}\over{\sqrt{{\it P_1}-{\it P_2}}}}
% \right)}\over{2\,K}}\right)
%\\
%&=
%\sqrt{mK}\biggl(
%\frac{y}2\sqrt{\frac{P_1-P_2}{K}-y^2}
%+
%\frac{P_1-P_2}{2K}\arcsin y\sqrt{\frac{K}{P_1-P_2}}
%\biggr)
%\\
&=
\frac{y\sqrt{m}}2\sqrt{P_1-P_2-Ky^2}
+
\frac{P_1-P_2}2 \sqrt{\frac{m}{K}}\arcsin y\sqrt{\frac{K}{P_1-P_2}}
\end{align*}
Die gesuchte L"osungsfunktion ist dann die Summe dieser drei Terme.
\end{loesung}

\begin{diskussion}
Die Berechnung der neuen Koordinaten $Q_1$ und $Q_2$ ist in diesem
Fall ziemlich kompliziert. Man kann sich aber mindestens "uberlegen,
dass wegen der Terme mit $\arcsin$ wohl periodische L"osungen zu
erwarten sind, was ja auch der Erfahrung entspricht.

Die Berechnung der L"osung kann mit dem folgenden Maxima\footnote{Maxima
ist eine freie Computer-Algebra-Software, die von
http://maxima.sourceforge.net heruntergeladen werden kann}-Programm
einfach durchgef"uhrt werden:
\verbatimainput{federpendel.maxima}
\end{diskussion}
