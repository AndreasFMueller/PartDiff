L"osen Sie die Eikonal-Gleichung
\begin{equation}
\biggl( \frac{\partial u}{\partial x} \biggr)^2
+
\biggl( \frac{\partial u}{\partial x} \biggr)^2
=
y
\label{4000008:eikonal}
\end{equation}
Im Gebiet $\Omega=\{(x,y)\,|\, y > 0\}$.
Finden Sie eine L"osung, deren Gradient im Punkt $(0,1)$ parallel
zur $x$-Achse ist.
Die L"osung beschreibt wie in Aufgabe~\ref{4000007} die
Phase einer Wellenfront, die auf der $y$-Achse beginnt.
Berechnen Sie die Orthogonaltrajektorien, also die Strahlrichtung.

\begin{hinweis}
Dies ist ein einfaches Modell f"ur die Lichtreflektion in einer
heissen Luftschicht unmittelbar "uber dem von der Sonne aufgeheizten
Boden. Die Hitze reduziert die optische Dichte.
\end{hinweis}

\begin{loesung}
Die Differentialgleichung  l"asst sich mit einem Separationsansatz
$u(x,y)=X(x)+Y(y)$
l"osen:
\[
X'(x)^2+Y'(y)^2=y
\qquad
\Rightarrow
\qquad
X'(x)^2=y-Y'(y)^2
\]
Die linke Seite h"angt nur von $x$ ab, die rechte nur von $y$, also
sind beide konstant.
Wir nennen die gemeinsame Konstante, die offensichtlich positiv sein
muss $a^2$.

F"ur $X(x)$ finden wir dann
\[
X'(x)^2=a^2
\qquad\Rightarrow\qquad
X(x)=ax
\]
und f"ur $Y(y)$
\[
Y'(x)^2=y-a^2
\qquad\Rightarrow\qquad
Y(y)=\int \sqrt{y-a^2}\,dy=\frac23(y-a^2)^{\frac32}
\]
Die L"osung ist also
\[
u(x,y)=ax + \frac23(y-a^2)^{\frac32}.
\]

Der Gradient dieser L"osung ist
\[
\operatorname{grad}u(x,y)=\begin{pmatrix}a\\y-a^2\end{pmatrix}
\]
Es wurde verlangt, dass im Punkt $(0,1)$ der Gradient parallel zur
$x$-Achse sein soll, also die $y$-Komponenten verschwindet. Daraus
folgt $0=y-a^2=1-a^2$, also $a=1$.

Die Orthogonaltrajektorien sind die Integralkurven des Gradienten,
wir suchen also Funktionen $x(s)$ und $y(s)$ mit der Eigenschaft
\begin{align*}
\frac{dx}{ds}&=1\\
\frac{dy}{ds}&=y-1
\end{align*}
Diese beiden Differentialgleichungen sind entkoppelt, sie k"onnen
separat gel"ost werden. Die erste f"ur $x(s)$ ist trivial:
\[
x(s)=s + x_0,
\]
wobei $x_0$ eine Integrationskonstante ist.
Die zweite ist inhomogen. Die homogene Gleichung $y'=y$ hat die
L"osung $y(s)=Ce^s$, also kann man eine partikul"are L"osung
der inhomogene durch Variation
der Konstanten, also mit dem Ansatz $y(s)=C(s)e^s$ finden.
Einsetzen in die Differentialgleichung ergibt
\begin{align*}
y'(s)=C'(s)e^s+C(s)e^s&=C(s)e^s+1
\\
\Rightarrow\qquad
C'(s)e^s&=1\\
C'(s)&=e^{-s}\\
C(s&)=-e^{-s}
\end{align*}
Damit ist die allgemeine L"osung der Differentialgleichung f"ur $y(s)$
\[
y(s)=(y_0+1)e^s-1,
\]
wobei wir die Integrationskonstante so gew"ahlt haben, dass $y(s)=y_0$.

Der L"osung $u$ der Eikonalgleichung entspricht also eine Schar
von
vom Punkt $(x_0,y_0)$ ausgehenden
Lichtstrahlen, die
durch
\begin{align*}
x(s)&=s+x_0\\
y(s)&=(y_0+1)e^s-1
\end{align*}
beschrieben werden. F"ur $x_0=0$, also Lichtstrahlen, die von der
$y$-Achse ausgehen, finden wir also
\begin{equation}
\left.
\begin{aligned}
x(s)&=s\\
y(s)&=(y_0+1)e^s-1
\end{aligned}
\right\}
\qquad\Rightarrow\qquad
y(x)=(y_0+1)e^x-1
\end{equation}
Der Lichtstrahl, der horizontal durch den Punkt $(0,1)$ verl"auft,
kr"ummt sich also exponentiell in $y$-Richtung.
Dies ist das bekannte Ph"anomen der Lichtspiegelung in einer
heisser Luftschicht.
\end{loesung}
