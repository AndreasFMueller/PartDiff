L"osen Sie die Eikonal-Gleichung
\begin{equation}
\biggl( \frac{\partial u}{\partial x} \biggr)^2
+
\biggl( \frac{\partial u}{\partial y} \biggr)^2
=
y
\label{4000008:eikonal}
\end{equation}
Im Gebiet $\Omega=\{(x,y)\,|\, y > 0\}$.
Finden Sie eine L"osung, deren Gradient im Punkt $(0,1)$ parallel
zur $x$-Achse ist.
Die L"osung beschreibt wie in Aufgabe~\ref{4000007} die
Phase einer Wellenfront, die auf der $y$-Achse beginnt.
Berechnen Sie die Orthogonaltrajektorien, also die Strahlrichtung.

\begin{hinweis}
Dies ist ein einfaches Modell f"ur die Lichtreflektion in einer
heissen Luftschicht unmittelbar "uber dem von der Sonne aufgeheizten
Boden. Die Hitze reduziert die optische Dichte.
\end{hinweis}

\begin{loesung}
Die Differentialgleichung  l"asst sich mit einem Separationsansatz
$u(x,y)=X(x)+Y(y)$
l"osen:
\[
X'(x)^2+Y'(y)^2=y
\qquad
\Rightarrow
\qquad
X'(x)^2=y-Y'(y)^2
\]
Die linke Seite h"angt nur von $x$ ab, die rechte nur von $y$, also
sind beide konstant.
Wir nennen die gemeinsame Konstante, die offensichtlich positiv sein
muss $a^2$.

F"ur $X(x)$ finden wir dann
\[
X'(x)^2=a^2
\qquad\Rightarrow\qquad
X(x)=ax
\]
und f"ur $Y(y)$
\[
Y'(x)^2=y-a^2
\qquad\Rightarrow\qquad
Y(y)=\int \sqrt{y-a^2}\,dy=\frac23(y-a^2)^{\frac32}
\]
Die L"osung ist also
\[
u(x,y)=ax + \frac23(y-a^2)^{\frac32}.
\]

Der Gradient dieser L"osung ist
\[
\operatorname{grad}u(x,y)=\begin{pmatrix}a\\\sqrt{y-a^2}\end{pmatrix}
\]
Es wurde verlangt, dass im Punkt $(0,1)$ der Gradient parallel zur
$x$-Achse sein soll, also die $y$-Komponenten verschwindet. Daraus
folgt $0=y-a^2=1-a^2$, also $a=1$.

Die Orthogonaltrajektorien sind die Integralkurven des Gradienten,
wir suchen also Funktionen $x(s)$ und $y(s)$ mit der Eigenschaft
\begin{align*}
\frac{dx}{ds}&=1\\
\frac{dy}{ds}&=\sqrt{y-1}
\end{align*}
Diese beiden Differentialgleichungen sind entkoppelt, sie k"onnen
separat gel"ost werden. Die erste f"ur $x(s)$ ist trivial:
\[
x(s)=s + x_0,
\]
wobei $x_0$ eine Integrationskonstante ist.
Da die Kurve vom Punkt $(0,1)$ ausgehen soll, muss $x_0=0$ sein.

Die zweite ist nicht linear, sie kann aber mit Separation gel"ost
werden:
\begin{align*}
y'&=\sqrt{y-1}
\\
\int \frac{dy}{\sqrt{y-1}}&=\int ds
\\
2\sqrt{y-1}&=s+s_0
\\
y&=1+\frac12(s+s_0)^2
\end{align*}
mit der Integrationskonstanten.
Da f"ur $s=0$ die Kurve durch den Punkt $(0,1)$ gehen soll, folgt
wieder $s_0=0$.

Die gesuchte Kurve hat daher die Parameterdarstellung
\[
\begin{pmatrix}
x(s)\\y(s)
\end{pmatrix}
=
\begin{pmatrix}
s\\
1+\frac12s^2
\end{pmatrix}.
\]
Da $x=s$ gilt, kann man $y$ auch direkt durch $x$ ausdr"ucken:
\[
y=\frac12x^2+1.
\]
Der Lichtstrahl, der horizontal durch den Punkt $(0,1)$ verl"auft,
kr"ummt sich also quadratisch in $y$-Richtung.
Dies ist das bekannte Ph"anomen der Lichtspiegelung in einer
heisser Luftschicht.
\end{loesung}

\begin{diskussion}
\begin{figure}
\centering
\includeagraphics[]{fatamorgana-1.pdf}
\caption{Lichtstrahlen durch den Punkt $(x_0,y_0)$ f"ur verschiedene
Werte des Separationsparameters $a$.
\label{40000008:fata}}
\end{figure}
Die Gleichung der Orthogonaltrajektorien l"asst sich nat"urlich noch
allgemeiner l"osen, wir suchen die Trajektorien, die durch den Punkt
$(x_0,y_0)$ verlaufen.
Das Gleichungssystem f"ur die Funktionen $x(s)$ und $y(s)$ wird
\begin{align*}
x'&=a\\
y'&=\sqrt{y-a^2}
\end{align*}
Die L"osung f"ur $x(s)$, die f"ur $s=0$ den Wert $x_0$ annimmt, ist
\[
x(s)=as+x_0.
\]
Die Gleichung f"ur $y(s)$ kann wieder mit Separation gel"ost werden,
man findet
\[
y(s)=a^2+\frac12(s+s_0)^2.
\]
Damit $y(0)=y_0$ wird, muss 
\[
s_0=\pm \sqrt{2(y_0-a^2)}
\]
gew"ahlt werden.
Die allgemeine Form der L"osung ist dann
\begin{align*}
x(s)&=as+x_0
\\
y(s)&=a^2 + \frac12\left(s\pm\sqrt{2(y_0-a^2)}\right)^2
\end{align*}
Alle L"osungskurven sind also Parabeln durch den Punkt $(x_0,y_0)$,
$a^2$ ist der kleinstm"ogliche Wert von $y$ ist (Abbildung~\ref{40000008:fata}).
Der Wert $a=0$ liefert einen Lichtstrahl parallel zur $y$-Achse, f"ur $a^2=y_0$
verl"auft der Strahl durch den Punkt $(x_0,y_0)$ horizontal.
\end{diskussion}

