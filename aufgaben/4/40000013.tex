Auf dem Gebiet 
\[
\Omega = \{(x,y)\,|\, 0<x, 0 < y < x\cdot \tan 60^\circ, 1 < x^2 +y^2 < 4\}
\]
ist die partielle Differentialgleichung
\begin{equation}
\frac{\partial^2u}{\partial x^2}
+
\frac{\partial^2u}{\partial y^2}
=
\lambda u
\label{40000013:equation}
\end{equation}
gegeben.
\begin{teilaufgaben}
\item
F"uhren Sie Separation in Polarkoordinaten $(r,\varphi)$ durch.
\item
Finden Sie f"ur $\lambda=0$ eine L"osung von (\ref{40000013:equation}), die
den Randbedingungen
\begin{align*}
u(x,0)&=0&u(x,y)=\sin3\varphi\qquad \text{f"ur $x^2+y^2=1$ oder $4$}\\
u(x,x\cdot \tan 60^\circ)&=0
\end{align*}
gen"ugt.
\item
Wo sind die Maxima der in b) gefundenen Funktion $u(x,y)$?
\end{teilaufgaben}

\begin{hinweis}
Suchen Sie Radialfunktionen $R(r)$ in der Form $R(r)=r^\alpha$.
\end{hinweis}

\begin{loesung}
Das Gebiet ist ein $60^\circ$-Ausschnitt aus einem Kreisring.

\begin{figure}
\centering
\includeagraphics[]{domain-1.pdf}
\caption{Definitionsgebiet $\Omega$ der Differentialgleichung
(\ref{40000013:equation})
\label{40000013:gebiet}
}
\end{figure}
Das Gebiet ist in Abbildung~\ref{40000013:gebiet} dargestellt.
Wir arbeiten in Polarkoordinaten, da l"asst sich der Laplace-Operator als
\[
\Delta u
=
\frac1r\frac{\partial}{\partial r}r\frac{\partial}{\partial r}
+
\frac1{r^2}
\frac{\partial^2}{\partial\varphi^2}
\]
ausdr"ucken.
\begin{teilaufgaben}
\item
Wir setzen einen Separationsansatz der Form $u(r,\varphi)=R(r)\Phi(\varphi)$
in die Differentialgleichung ein und erhalten
\begin{align*}
\frac1r\frac{d}{dr}\bigl(rR'(r)\bigr)\Phi(\varphi)
+
\frac1{r^2}
\Phi''(\varphi)R(r)
&=
\lambda R(r)\Phi(\varphi),
\\
\biggl(\frac1rR'(r)+R''(r)\biggr)\Phi(\varphi)
- \lambda R(r)\Phi(\varphi)
&=
-
\frac1{r^2}
R(r)\Phi''(\varphi),
\\
\frac{r^2R''(r)+rR'(r)-\lambda R(r)}{R(r)}
&=
-\frac{\Phi''(\varphi)}{\Phi(\varphi)}.
\end{align*}
Da die linke Seite nur von $r$, die rechte aber nur von $\varphi$ abh"angt,
m"ussen beide Seiten konstant sein und wir
erhalten die zwei Gleichungen
\begin{align}
r^2R''(r)+rR'(r)-\lambda R(r)&=\phantom{-}\mu R(r)
\label{40000013:erste}
\\
\Phi''(\varphi)&=-\mu\Phi(\varphi)
\label{40000013:zweite}
\end{align}
mit einer noch zu bestimmenden Konstanten $\mu$.
\item
Die Funktionen $\Phi(\varphi)$ m"ussen die Randbedingungen
\[
\Phi(0)=\Phi({\textstyle\frac{\pi}3})=0
\]
erf"ullen.
Die Randbedingungen auf der $x$-Achse und der $60^\circ$-Geraden verlangen,
dass f"ur $\Phi$
oszillierende L"osungen verwendet werden m"ussen, was nur m"oglich ist,
wenn $\mu>0$ ist. Wir setzen daher $\mu=m^2$. Damit bekommen wir f"ur die
$\Phi(\varphi)$
\[
\Phi(\varphi)=\begin{cases}\sin m\varphi,\\\cos m\varphi.\end{cases}
\]
Die Kosinus-L"osung erf"ullt die Randbedingung auf der $x$-Achse nicht,
und die Sinus-L"osung erf"ullt die Randbedingung auf der $60^\circ$-Geraden nur,
wenn $m=3k$ mit $k\in\mathbb Z$ ist.
Wir haben also die Teill"osungen
\[
\Phi_k(\varphi)=\sin 3k\varphi, \quad\varphi\in[0,{\textstyle \frac{\pi}3}]
\]
gefunden.

F"ur jedes $k$ m"ussen jetzt noch Funktionen $R_k(r)$ gefunden werden.
Man kann aber jetzt schon sagen, dass wegen der speziellen Randbedingungen
auf den kreisf"ormigen Teilen des Randes nur die L"osung $k=1$ ben"otigt wird.

\begin{figure}
\centering
\includeagraphics[width=0.8\hsize]{loes.jpg}
\caption{3D-Darstellung der L"osungsfunktion
\label{40000013:3d}}
\end{figure}
Aufgrund des Hinweises versuchen wir die Radialfunktion in der Form
$R(r)=r^\alpha$, und setzen dies in (\ref{40000013:erste}) ein:
\begin{align*}
r^2\alpha(\alpha-1)r^{\alpha-2}+r\alpha r^{\alpha-1}&=9k^2r^{\alpha}
\\
(\alpha(\alpha-1)+\alpha -9k^2)r^{\alpha}&=0
\\
(\alpha^2 -9k^2)r^{\alpha}&=0
\\
\alpha^2&=9k^2
\\
\alpha&=\pm3k
\end{align*}
Wir wissen bereits, dass nur der Fall $k=1$ ben"otigt wird, und suchen
daher Koeffizienten $A$ und $B$ derart, dass
\[
Ar^3+Br^{-3}=1\qquad \text{f"ur $r=1,2$.}
\]
Dies f"uhrt auf das Gleichungssystem
\[
\begin{linsys}{3}
 A&+&       B&=&1\\
8A&+&\frac18B&=&1
\end{linsys}
\]
mit den L"osungen
\[
A=\frac19,\qquad B=\frac89.
\]
Die gesuchte L"osung der Differentialgleichung ist also
\begin{equation}
u(r,\varphi)=\biggl(\frac19r^3+\frac89r^{-3}\biggr)\sin 3\varphi.
\label{40000013:solution}
\end{equation}
Die Abbildung~\ref{40000013:3d} zeigt eine 3D-Darstellung der L"osungsfunktion.
\item
Die Funktion $u$ ist wegen $\lambda=0$ eine harmonische Funktion, ihre
Maxima und Minima sind daher auf dem Rand.
Es reicht daher, die Randbedingungen zu konsultieren, wir finden die
Maxima in den Punkten mit Polarwinkel $30^\circ$, wo die Sinus-Funktion
in (\ref{40000013:solution}) den Wert $1$ annimmt.
Dies sind die Punkte
\[
(\cos 30^\circ, \sin 30^\circ)=\biggl(
\frac{\sqrt{3}}{2}
,
\frac12
\biggr),\qquad
\text{und}\qquad
(
2\cos 30^\circ
,
2\sin 30^\circ
)=(
\sqrt{3}
,
1
).
\qedhere
\]
\end{teilaufgaben}
\end{loesung}

\begin{bewertung}
Differentialgleichung f"ur $R$ ({\bf R}) 1 Punkt,
Differentialgleichung f"ur $\Phi$ ({$\mathbf \Phi$}) 1 Punkt,
L"osung der Gleichung f"ur $\phi$ ($\textbf{L}_\Phi$) 1 Punkt,
L"osung der Gleichung f"ur $R$ mit dem Potenzansatz ($\textbf{L}_R$) 1 Punkt,
Einsetzen der Randwerte/Koeffizientenvergleich ({\bf B}) 1 Punkt,
Maximumprinzip und Maximalstellen ({\bf M}) 1 Punkt.
\end{bewertung}


