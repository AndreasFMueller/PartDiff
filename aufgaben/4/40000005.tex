Einen Stab ist bis zur Zeit $t=0$ mit zwei
Wärmereservoirs auf den Temperaturen $-1$ bei $x=-\frac{\pi}2$ und
$1$ bei $x=\frac{\pi}2$ verbunden,
so dass sich ein stationärer Zustand eingestellt hat (siehe
Übung 1). Zur Zeit $t=0$ werden die Reservoirs entfernt und der
Stab wird sich selbst überlassen. Insbesondere kann durch die
Enden keine Wärme mehr abgeleitet werden. Lösen sie die
Wärmeleitungsgleichung
\[
\frac{\partial u}{\partial t}=\frac{\partial^2 u}{\partial x^2}
\]
für diesen Fall.

\begin{hinweis}
\begin{figure}
\centering
\includeagraphics[]{dreieck-1.pdf}
\caption{Graph der Dreiecksfunktion von Aufgabe~\ref{40000005}.
\label{40000005:dreieck}}
\end{figure}
Verwenden sie für die
``Dreiecksfunktion'' (Abbildung~\ref{40000005:dreieck})
\[
d(x)
=
\begin{cases}
\displaystyle-2-\frac{2x}{\pi}&\qquad -\pi<x\le \displaystyle-\frac{\pi}2\\
\displaystyle\frac{2x}{\pi}&\qquad \displaystyle-\frac{\pi}2\le x\le \frac{\pi}2\\
\displaystyle2-\frac{2x}{\pi}&\qquad \displaystyle \frac{\pi}2<x\le\pi
\end{cases}
\]
die folgende nur Sinus-Funktionen
verwendende Fourier-Reihe
\[
d(x)=\sum_{n=0}^\infty \frac{8(-1)^n}{\pi^2(2n+1)^2}\sin (2n+1)x
\]
\end{hinweis}

\begin{hinweis}
Eine Lösung dieser Aufgabe mit der Transformationsmethode ist als
Aufgabe \ref{50000005} dargestellt.
\end{hinweis}

\begin{loesung}
Wir kontrollieren zunächst die angegebenen Fourierkoeffizienten.
Da die Dreiecksfunktion eine ungerade Funktion ist, kommen in ihrer
Fourier-Reihe nur Sinus-Terme vor. Die Koeffizienten werden mit
den üblichen Fourier-Integralen berechnet:
\begin{align*}
b_k&=\frac{1}{\pi}\int_{-\pi}^{\pi} d(x)\sin kx\,dx\\
&=\frac{2}{\pi}\int_{0}^{\pi} d(x)\sin kx\,dx
\end{align*}
Weil die Sinus-Funktionen mit geradem $k$ auf dem Interval $[0,\pi]$
bezüglich des Punktes $x=\frac{\pi}2$ ungerade sind, verschwinden die
zugehörigen Fourier-Koeffizienten, es bleiben also nur noch die
Koeffizienten für ungerades $k$ zu berechnen. Sei also $k$ ungerade, dann
ist
\begin{align*}
b_k
&=
\frac{4}{\pi}\int_0^{\frac{\pi}2} \frac{2}{\pi}x\sin kx\,dx
\\
&=
\frac{8}{\pi^2}\int_0^{\frac{\pi}2} x\sin kx\,dx
\\
&=
\frac{8}{\pi^2}\left[ -\frac{1}kx\cos kx+\frac1{k^2}\sin kx\right]_0^{\frac{\pi}2}
\end{align*}
Für ungerades $k$ verschwindet der Kosinus-Term, so dass nur der Sinus-Term
stehen bleibt. Dieser ist $1$ für $k=1,5,9,\dots$ und $-1$ für $k=3,7,11,\dots$.
Schreibt man $k=2n+1$, dann ist
\[
b_k=b_{2n+1}=\frac{(-1)^n8}{\pi^2(2n+1)^2},
\]
wie behauptet.

Jetzt kann man die Differentialgleichung mit dem Separationsansatz
lösen. Dazu setzt man $u(t,x)=T(t)X(x)$ an und setzt dies in die
Differentialgleichung ein:
\[
T'(t)X(t)=T(t)X''(x)
\]
Falls $X(x)\ne 0$ und $T(t)\ne 0$ kann man dividieren:
\[
\frac{T'(t)}{T(t)}=\frac{X''(x)}{X(x)}
\]
Da die rechte Seite nur von $x$, die linke aber nur von $t$ abhängt,
müssen beide Seiten gleich einer Konstanten $\lambda$ sein, also
\[
T'(t)=\lambda T(t),\qquad X''(x)=\lambda X(x).
\]
Die Temperatur ist eine positive Grösse, die mit der Zeit abnimmt,
dies ist nur möglich wenn $\lambda$ eine negative Zahl $\lambda=-\mu$
sein, worin $\mu$ positiv sein soll. Dann kann man die gewöhnlichen
Differentialgleichungen lösen:
\begin{align*}
T(t)&=e^{-\mu t}&X(x)&=\sin\sqrt{\mu}x\\
    &            &X(x)&=\cos\sqrt{\mu}x
\end{align*}

Die Lösungen müssen ausserdem die Randbedingungen erfüllen.
In diesem Fall drücken die Randbedingungen aus, dass der Stab isoliert ist,
also keine Wärme über die Enden hinweg fliessen kann. Dazu ist nötig,
dass dort kein Temperaturgefälle herscht, wir haben also
\[\frac{\partial u}{\partial x}=0\quad\text{für $x=\pm\frac{\pi}2$}.\]
Für die Sinus-Lösungen bedeutet dies
\[
X'\biggl(\frac{\pi}2\biggr)=\sqrt{\mu}\cos \sqrt{\mu}\frac{\pi}2=0,
\]
also muss $\sqrt{\mu}$ in diesem Falle eine ungerade Zahl sein,
zum Beispiel $\sqrt{\mu}=2n+1$ oder
$\mu=(2n+1)^2$.
Für die Kosinus-Lösungen folgt ähnlich
\[
X'\biggl(\frac{\pi}2\biggr)=-\sqrt{\mu}\sin\sqrt{\mu}\frac{\pi}2=0,
\]
Also muss $\sqrt{\mu}$ eine gerade Zahl sein, zum Beispiel $\sqrt{\mu}=2n$ oder
$\mu=4n^2$.
Die Lösungen, die die Anfangsbedingung
erfüllen, sind also
\[
e^{-4n^2t}\cos 2nx\qquad\text{und}\qquad e^{-(2n+1)^2t}\sin(2n+1)x.
\]

Jetzt müssen wir die allgemeine Lösung der Differentialgleichung aus
diesen Einzellösungen linear kombinieren, also
\[
u(t,x)=
\sum_{n=0}^\infty a_ne^{-2nt}\cos 2nx +\sum_{n=0}^\infty b_n e^{-(2n+1)^2t}\sin(2n+1)x,
\]
und die Koeffizienten so wählen, dass die Anfangsbedingung erfüllt wird.

Da die Anfangsbedingung eine ungerade Funktion is, brauchen wir die Kosinus-Terme,
die ja alle gerade sind, nicht. Es bleiben also die $b_n$ zu bestimmen, indem
man $t=0$ einsetzt. Man erhält
\begin{align*}
u(0,x)&=\sum_{n=0}^\infty b_n\sin(2n+1)x=d(x)
\\
\Rightarrow
b_n&=\frac{8(-1)^n}{\pi^2(2n+1)^2}.
\end{align*}
Die Lösung der Wärmeleitungsgleichung ist also
\[
u(t,x)=
\sum_{n=0}^\infty \frac{8(-1)^n}{\pi^2(2n+1)^2}e^{-(2n+1)^2t}\sin(2n+1)x
\]
Eine graphische Darstellung der Lösung ist in Abbildung~\ref{40000005:bild}
zu finden.
\begin{figure}
\begin{center}
\includeagraphics[width=\hsize]{waerme.jpg}
\end{center}
\caption{Lösung der Wärmeleitungsgleichung (Aufgabe \ref{40000005})\label{40000005:bild}}
\end{figure}
\end{loesung}
