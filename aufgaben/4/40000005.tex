Einen Stab ist bis zur Zeit $t=0$ mit zwei
W"armereservoirs auf den Temperaturen $-1$ bei $x=-\frac{\pi}2$ und
$1$ bei $x=\frac{\pi}2$ verbunden,
so dass sich ein station"arer Zustand eingestellt hat (siehe
"Ubung 1). Zur Zeit $t=0$ werden die Reservoirs entfernt und der
Stab wird sich selbst "uberlassen. Insbesondere kann durch die
Enden keine W"arme mehr abgeleitet werden. L"osen sie die
W"armeleitungsgleichung
\[
\frac{\partial u}{\partial t}=\frac{\partial^2 u}{\partial x^2}
\]
f"ur diesen Fall.

{\it Hinweis:} Verwenden sie f"ur die
``Dreiecksfunktion''
\[
d(x)
=
\begin{cases}
\displaystyle-2-\frac{2x}{\pi}&\qquad \displaystyle-\frac{\pi}2\le x\\
\displaystyle\frac{2x}{\pi}&\qquad \displaystyle-\frac{\pi}2\le x\le \frac{\pi}2\\
\displaystyle2-\frac{2x}{\pi}&\qquad \displaystyle x\le\frac{\pi}2
\end{cases}
\]
die folgende nur Sinus-Funktionen
verwendende Fourier-Reihe
\[
d(x)=\sum_{n=0}^\infty \frac{8(-1)^n}{\pi^2(2n+1)^2}\sin (2n+1)x
\]

\begin{loesung}
Wir kontrollieren zun"achst die angegebenen Fourierkoeffizienten.
Da die Dreiecksfunktion eine ungerade Funktion ist, kommen in ihrer
Fourier-Reihe nur Sinus-Terme vor. Die Koeffizienten werden mit
den "ublichen Fourier-Integralen berechnet:
\begin{align*}
b_k&=\frac{1}{\pi}\int_{-\pi}^{\pi} d(x)\sin kx\,dx\\
&=\frac{2}{\pi}\int_{0}^{\pi} d(x)\sin kx\,dx
\end{align*}
Weil die Sinus-Funktionen mit geradem $k$ auf dem Interval $[0,\pi]$
bez"uglich des Punktes $x=\frac{\pi}2$ ungerade sind, verschwinden die
zugeh"origen Fourier-Koeffizienten, es bleiben also nur noch die
Koeffizienten f"ur ungerades $k$ zu berechnen. Sei also $k$ ungerade, dann
ist
\begin{align*}
b_k
&=
\frac{4}{\pi}\int_0^{\frac{\pi}2} \frac{2}{\pi}x\sin kx\,dx
\\
&=
\frac{8}{\pi^2}\int_0^{\frac{\pi}2} x\sin kx\,dx
\\
&=
\frac{8}{\pi^2}\left[ -\frac{1}kx\cos kx+\frac1{k^2}\sin kx\right]_0^{\frac{\pi}2}
\end{align*}
F"ur ungerades $k$ verschwindet der Kosinus-Term, so dass nur der Sinus-Term
stehen bleibt. Dieser ist $1$ f"ur $k=1,5,9,\dots$ und $-1$ f"ur $k=3,7,11,\dots$.
Schreibt man $k=2n+1$, dann ist
\[
b_k=b_{2n+1}=\frac{(-1)^n8}{\pi^2(2n+1)^2},
\]
wie behauptet.

Jetzt kann man die Differentialgleichung mit dem {\bf Separationsansatz}
l"osen. Dazu setzt man $u(t,x)=T(t)X(x)$ an und setzt dies in die
Differentialgleichung ein:
\[
T'(t)X(t)=T(t)X''(x)
\]
Falls $X(x)\ne 0$ und $T(t)\ne 0$ kann man dividieren:
\[
\frac{T'(t)}{T(t)}=\frac{X''(x)}{X(x)}
\]
Da die rechte Seite nur von $x$, die linke aber nur von $t$ abh"angt,
m"ussen beide Seiten gleich einer Konstanten $\lambda$ sein, also
\[
T'(t)=\lambda T(t),\qquad X''(x)=\lambda X(x).
\]
Die Temperatur ist eine positive Gr"osse, die mit der Zeit abnimmt,
dies ist nur m"oglich wenn $\lambda$ eine negative Zahl $\lambda=-\mu$
sein, worin $\mu$ positiv sein soll. Dann kann man die gew"ohnlichen
Differentialgleichungen l"osen:
\begin{align*}
T(t)&=e^{-\mu t}&X(x)&=\sin\sqrt{\mu}x\\
    &            &X(x)&=\cos\sqrt{\mu}x
\end{align*}

Die L"osungen m"ussen ausserdem die Randbedingungen erf"ullen.
In diesem Fall dr"ucken die Randbedingungen aus, dass der Stab isoliert ist,
also keine W"arme "uber die Enden hinweg fliessen kann. Dazu ist n"otig,
dass dort kein Temperaturgef"alle herscht, wir haben also
\[\frac{\partial u}{\partial x}=0\quad\text{f"ur $x=\pm\frac{\pi}2$}.\]
F"ur die Sinus-L"osungen bedeutet dies
\[
X'\biggl(\frac{\pi}2\biggr)=\sqrt{\mu}\cos \sqrt{\mu}\frac{\pi}2=0,
\]
also muss $\sqrt{\mu}$ in diesem Falle eine ungerade Zahl sein,
zum Beispiel $\sqrt{\mu}=2n+1$ oder
$\mu=(2n+1)^2$.
F"ur die Kosinus-L"osungen folgt "ahnlich
\[
X'\biggl(\frac{\pi}2\biggr)=-\sqrt{\mu}\sin\sqrt{\mu}\frac{\pi}2=0,
\]
Also muss $\sqrt{\mu}$ eine gerade Zahl sein, zum Beispiel $\sqrt{\mu}=2n$ oder
$\mu=4n^2$.
Die L"osungen, die die Anfangsbedingung
erf"ullen, sind also
\[
e^{-4n^2t}\cos 2nx\qquad\text{und}\qquad e^{-(2n+1)^2t}\sin(2n+1)x.
\]

Jetzt m"ussen wir die allgemeine L"osung der Differentialgleichung aus
diesen Einzell"osungen linear kombinieren, also
\[
u(t,x)=
\sum_{n=0}^\infty a_ne^{-2nt}\cos 2nx +\sum_{n=0}^\infty b_n e^{-(2n+1)^2t}\sin(2n+1)x,
\]
und die Koeffizienten so w"ahlen, dass die Anfangsbedingung erf"ullt wird.

Da die Anfangsbedingung eine ungerade Funktion is, brauchen wir die Kosinus-Terme,
die ja alle gerade sind, nicht. Es bleiben also die $b_n$ zu bestimmen, indem
man $t=0$ einsetzt. Man erh"alt
\begin{align*}
u(0,x)&=\sum_{n=0}^\infty b_n\sin(2n+1)x=d(x)
\\
\Rightarrow
b_n&=\frac{8(-1)^n}{\pi^2(2n+1)^2}.
\end{align*}
Die L"osung der W"armeleitungsgleichung ist also
\[
u(t,x)=
\sum_{n=0}^\infty \frac{8(-1)^n}{\pi^2(2n+1)^2}e^{-(2n+1)^2t}\sin(2n+1)x
\]

Die L"osung mit der {\bf Transformationsmethode}
verwendet Fourier-Transformation
entlang der $x$-Achse und Laplace-Transformation entlang der Zeit-Achse.
Wir schreiben $\hat u(t,k)$ f"ur die Fourier-Transformation und ${\cal L} u$
f"ur die Laplace-Transformation.

Die Anfangsbedingungen sind ungerade, so dass auch die L"osung der
Differntialgleichung f"ur alle Zeiten ungerade sein wird. Die
Randbedingung sagen ausserdem, dass
$\partial_xu(t,-\frac{\pi}2)=\partial_xu(t,\frac{\pi}2)=0$ gilt.
Dies bedeutet, dass man die Funktion durch Spiegelung an
$-\frac{\pi}2$ bzw.~$\frac{\pi}2$ zu einer $2\pi$-periodischen
Funktion auf ganz $\mathbb R$ fortsetzen kann. Insbesondere l"asst
sich $u(t,x)$ mit einer Sinusreihe entwickeln. Seien $\hat u(t,k)$
die Fourier-Sinus-Koeffizienten. Damit haben wir als Differentialgleichung
der $\hat u(t,k)$
\[
\partial_t\hat u(t,k)=-k^2\hat u(t,k)
\]
Diese Gleichung kann man jetzt nach Laplace transformieren:
\begin{align*}
s{\cal L}\hat u(s,k) - \hat u(0,k)&=-k^2 {\cal L}\hat u(s,k)\\
(s+k^2){\cal L}\hat u(s,k)&=\hat u(0,k)\\
{\cal L}\hat u(s,k)&=\frac{\hat u(0,k)}{s+k^2}
\end{align*}
R"ucktransformation ergibt:
\[
\hat u(t,k)=\hat u(0,k) e^{-k^2t}
\]
Jetzt sind nur noch die Fourierkoeffizienten zu bestimmen, die kann
man aber dem Hinweis entnehmen:
\[
\hat u(0,2n+1)=
\frac{8(-1)^n}{\pi^2(2n+1)^2}
\]
und damit die endg"ultige L"osung durch Summieren der Fourierreihen bekommen:
\[
u(t,x)=
\sum_{n=0}^\infty \frac{8(-1)^n}{\pi^2(2n+1)^2}e^{-(2n+1)^2t}\sin(2n+1)x
\]
\begin{figure}
\begin{center}
\includeagraphics[width=\hsize]{a3.pdf}
\end{center}
\caption{L"osung der W"armeleitungsgleichung}
\end{figure}
\end{loesung}
