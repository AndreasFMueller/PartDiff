In the domain $\Omega=\{(x,y)\mid x^2+y^2<1\}$ the differential equation
\begin{equation}
\frac{\partial^2u}{\partial x^2} + \frac{\partial^2 u}{\partial y^2}
+m(x^2+y^2)u=0
\label{40000026:pde}
\end{equation}
with homogeneous Dirichlet boundary conditions is given.
Reduce the solution of the problem to two ordinary differential 
equations.
Also specify the boundary conditions for these two differential equations
and solve one of them.
Determine the possible values of constants that you might have introduced
in the process.

\begin{diskussion}
Unfortunately, there was a typo in the printed exam, the function $u$
was missing in the last term on the left hand side of the equation
\eqref{40000026:pde}.
This lead to a nonhomogeneous equation, which, by the method discussed
in in class, should first be simplified by subtracting a particular
solution.
To find a particular solution to the equation $\Delta u+mr^2=0$ we try
a guess $u_p(r)=r^4-1$ and obtain $\Delta u_p=16r^2$, which at least
works for $m=-16$.
Subtracting this particular solution leaves a homogenous problem with
homogeneous boundary conditions as it was discussed in class.
The remaining equation is the homogeneous equation $\Delta u=0$ which has
been discussed in class.
This, although being perfectly in the realm of what has been taught
during the lectures, is a bit less straightforward than the originally 
intended problem.

This oversight was taken into account in grading: errors related to the
second term were essentially ignored, even equations that did not completely
separate away these terms were counted, if the differential terms were
handled correctly.
An additional point was given if the need for polar coordinates was 
recognized.
Thus 3 points were given for things that were completely unrelated
to the erroneous term.
An additional point was given if it was pointed out that the term
as it was printed poses a problem.

The domain used in this problem was discussed in class on slides 19--22,
where the separation method without the additional term $m^2(x^2+y^2)u$
was discussed.
Also, the Laplace operator in polar coordinates was introduced there.
\end{diskussion}

\begin{loesung}
We use polar coordinates $(r,\varphi)$, in which the Laplace operator
has the form
\[
\Delta 
=
\frac{\partial^2}{\partial r^2}
+
\frac1r
\frac{\partial }{\partial r}
+
\frac1{r^2}
\frac{\partial^2}{\partial\varphi^2}.
\]
The differential equation becomes
\[
\frac{\partial^2 u}{\partial r^2}
+
\frac1r
\frac{\partial u}{\partial r}
+
\frac1{r^2}
\frac{\partial^2 u}{\partial\varphi^2}
+mr^2u
=
0.
\]
A separation ansatz of the form
$u(r,\varphi)= R(r)\cdot \Phi(\varphi)$
leads to the equation
\[
R''(r)\Phi(\varphi)
+
\frac1r R'(r)\Phi(\varphi)
+
\frac{1}{r^2} R(r)\Phi''(\varphi)
+
mr^2 R(r)\Phi(\varphi)
=
0.
\]
To separate the variables, the third term needs to be moved 
to the right hand side:
\[
R''(r)\Phi(\varphi)
+
\frac1r R'(r)\Phi(\varphi)
+
mr^2 R(r)\Phi(\varphi)
=
-
\frac{1}{r^2} R(r)\Phi''(\varphi).
\]
After multiplication by $r^2$ and division by 
$R(r)\Phi(\varphi)$ we get
\[
\frac{r^2R''(r) + rR'(r)+mr^4R(r)}{R(r)}
=
-\frac{\Phi''(\varphi)}{\Phi(\varphi)},
\]
where the variables have been separated.
This implies that both sides need tob e constant.
We use $\mu^2$ as the separation constant and get the two
ordinary differential equations
\begin{align*}
r^2R''(r)+rRr'(r)+(mr^4-\mu^2)R(r)&=0
&
&\text{und}&
\Phi''^(\varphi)&=-\mu^2 \Phi(\varphi)
\intertext{with boundary conditions}
 R(1)&=0         &&& \Phi(2\pi)  &= \Phi(0) \\
R'(0)&=0         &&& \Phi'(2\pi) &= \Phi'(0).
\end{align*}
The differential equation for $\Phi(\varphi)$ has the solutions
$\cos\mu\varphi$ and $\sin\mu\varphi$, but the boundary conditions
on $\Phi(\varphi)$ are only satisfied if $\mu$ is an integer.
\end{loesung}

\begin{diskussion}
The differential equation for $R(r)$ has no easy and well known solution
but a power series solution can be found as follows using Perron's
power series method.
Substituting
\begin{align*}
R(r)
&=
\sum_{k=0}^\infty a_kr^k
&R'(r) 
&=
\sum_{k=0}^\infty (k-1)a_kr^{k-1}
\\
&&
R''(r)
&=
\sum_{k=0}^\infty (k-1)ka_kr^{k-2}
\end{align*}
into the differential equation gives
\[
\sum_{k=0}^\infty
k(k-1)
a_k
r^k
+
\sum_{k=0}^\infty
ka_k r^k
-
\sum_{k=0}^\infty
\mu^2 a_kr^k
+
\sum_{k=4}^\infty
m a_{k-4}r^k
=
0.
\]
For the terms $k\ge 4$ we get
\begin{align*}
\sum_{k=4}^\infty \bigl(k(k-1)a_k+ka_k -\mu^2 a_k+ma_{k-4}\bigr)r^k
&=
\sum_{k=4}^\infty \bigl((k^2-\mu^2)a_k + ma_{k-4}\bigr) r^k
&&\Rightarrow&
(k^2-\mu^2) a_k&= m a_{k-4}.
\intertext{For $k<4$ we have}
(k^2-\mu^2)a_k&=0.
\end{align*}
Because $\mu$ is an integer, precisely one of the factors 
$(k^2-\mu^2)$ will be zero.
This implies that
\[
a_k 
=
\begin{cases}
\displaystyle
\frac{m}{k^2-\mu^2}a_{k-4}&\qquad\text{for $k\ge 4$ and $k\ne \mu$}\\
\text{arbitrary}&\qquad\text{for $k=\mu$}
\end{cases}
\]
In particular, all coefficients $a_{\mu-4l}=0$ for any natural number.
It follows that $\mu$ is the lowest index, for which $a_k=0$.

This implies that $a_\mu$ can be chosen, lets call it $a$, then all the
other nonzero coefficients are
\[
a_{\mu+4l}
=
a
\cdot
\frac{m}{(\mu+4l)^2-\mu^2}
\cdot
\frac{m}{(\mu+4(l-1))^2-\mu^2}
\cdots
=
a
\frac{m}{8l\mu+16l^2}
\cdot
\frac{m}{8(l-1)\mu+16(l-1)^2}
\cdots
\frac{m}{8\mu +16}.
\]
Unfortunatley, this is still not an easily identifiable function, but
it offers a way to numerically computing a solution.
\end{diskussion}

\begin{bewertung}
Need for polar coordinates ({\bf P}) 1 point,
Laplace operator in polar coordinates ({\bf L}) 1 point,
Separation ansatz ({\bf A}) 1 point,
separated equations ({\bf E}) 1 point,
boundary conditions for the separated equations ({\bf B}) 1 point,
solution of the $\Phi(\varphi)$-equation ({\bf S}) 1 point,
determination of $\mu$ ({\bf M}) 1 point.
\end{bewertung}

