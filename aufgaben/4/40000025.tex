On the rectangular domain
$\Omega = \{(x,y)\in\mathbb R^2\,|\, 1<x<2\wedge -\pi<y<\pi\}$,
the differential equation
\begin{equation}
\Delta u = x^2u
\label{40000025:eqn}
\end{equation}
is given with boundary conditions
\begin{equation}
\left.
\begin{aligned}
u(1,y) &= \sin y\\
u(2,y) &= \cos y
\end{aligned}
\quad
\right\}
\quad
\text{for $y\in[-\pi,\pi]$}
\qquad
\text{and}
%\qquad
\qquad
\left.
\begin{aligned}
u(x,-\pi) &= u(x,\pi)\\
\frac{\partial u}{\partial y}(x,-\pi) &= \frac{\partial u}{\partial y}(x,\pi)
\end{aligned}
\quad\right\}
\quad
\text{for $x\in[1,2]$.}
\end{equation}
The equations on the right means that the function $y\mapsto u(x,y)$
can be extended to a differentiable $2\pi$-periodic function for all $y$.

\begin{teilaufgaben}
\item
Is the equation elliptic, parabolic or hyperbolic?
\item
Use separation to obtain ordinary differential equations that can be
used to find solutions of they partial differential equation.
\item
Solve one of the ordinary differential equations you have obtained in b).
\item
Determine the possible values of the separation constant.
\item
Write the general solution as a superposition.
\end{teilaufgaben}

\begin{loesung}
\begin{teilaufgaben}
\item
The domain is a rectangle, so wie have work on cartesian coordinates, in
which the Laplace operator has the form
\[
\Delta
=
\frac{\partial^2}{\partial x^2}
+
\frac{\partial^2}{\partial y^2}.
\]
This is probably the most famous elliptic partial differential operator.
\item
We substitute the ansatz $u(x,y) = X(x) Y(y)$ into the differential
equation \eqref{40000025:eqn} and obtain
\begin{align*}
\Delta u
=
\frac{\partial^2 u}{\partial x^2}
+
\frac{\partial^2 u}{\partial y^2}
=
X''(x)Y(y) + X(x)Y''(y)
&=
x^2 X(x)Y(x)
\\
\frac{X''(x)-x^2X(x)}{X(x)} &= \frac{Y''(y)}{Y(y)}.
\end{align*}
The left hand side only depends on $x$, the right hand side only on $y$,
so both sides must be constant.
We call the constant $-\lambda^2$, this particular form willl prove to
be ideal later.
We get the two separated differential equations
\begin{align}
X''(x)-(x^2+\lambda^2)X(x)&=0
&
Y''(y)=-\lambda^2 Y(y).
\label{40000025:separated}
\end{align}
\item
The second equation is an oscillation equation with solutions
\[
\sin(\lambda y)
\qquad\text{and}\qquad
\cos(\lambda y).
\]
\item
The boundary conditions mean that the solutions are $2\pi$-periodic,
which implies that $\lambda$ must be an integer, we will write them
$\lambda=k$, $k\in\mathbb N$.
\item
For every possible value $k$, there are two linearly independent 
solutions $X_{k,1}(x)$ und $X_{k,2}(x)$ for the left hand equation in
\eqref{40000025:separated}.
The general solution for $k$ is a linear combination of the form
$a_{k,1}X_{k,1}(x) + a_{k,2}X_{k,2}(x)$.
This has to be multiplied by the solutions for $Y(y)$ equation.
As we have two linearly independent solutions, we need two
linear combinations of $X_{k,1}(x)$ and $X_{k,2}(x)$.
This leads to the series
\begin{equation}
\begin{aligned}
u(x,y)
&=
\frac{a_{0,1}X_{0,1}(x)+a_{0,2}X_{0,2}(x)}{2}
\\
&\qquad
+\sum_{k=1}^\infty
\bigl(
(a_{k,1}X_{k,1}(x)+a_{k,2}X_{k,2}(x))\cos ky
+
(b_{k,1}X_{k,1}(x)+b_{k,2}X_{k,2}(x))\sin ky
\bigr).
\end{aligned}
\label{40000025:series}
\qedhere
\end{equation}
\end{teilaufgaben}
\end{loesung}

\begin{diskussion}
For each value of $x$, the series expansion~\eqref{40000025:series}
is a Fourier series with the value if the linear combination of the
$X$-functions as coefficient.
To match them up with the boundary conditions for $u(x,y)$ for $x=1$ and
$x=2$, we have to expand those boundary conditions as Fourier series
as well.
Let the boundary conditions be
\begin{align*}
u(1,y) &= f(y)
&&\text{with Fourier series}&
f(y)
&=
\frac{a^f_0}{2} + \sum_{k=1}^\infty \bigl(a_k^f\cos ky+b_k^f\sin ky\bigr)
\\
u(2,y)&=g(y)
&&\text{with Fourier series}&
g(y)
&=
\frac{a^g_0}{2} + \sum_{k=1}^\infty \bigl(a_k^g\cos ky+b_k^g\sin ky\bigr).
\end{align*}
They have to coincide with the Fourier coefficients of $u(x,y)$ on the
boundaries,
i.~e.~we get the equations
\begin{align*}
a_{k,1}X_{k,1}(1)+a_{k,2}X_{k,2}(1) &= a_k^f \\
a_{k,1}X_{k,1}(2)+a_{k,2}X_{k,2}(2) &= a_k^g
\\[5pt]
b_{k,1}X_{k,1}(1)+b_{k,2}X_{k,2}(1) &= b_k^f \\
b_{k,1}X_{k,1}(2)+b_{k,2}X_{k,2}(2) &= b_k^g
\end{align*}
for the coefficients, which allow to uniquely determine them once the
values of the $X$-functions for $x=1$ and $x=2$ are known.
\end{diskussion}


\begin{bewertung}
\begin{teilaufgaben}
\item Elliptic ({\bf E}) 1 point,
\item separation ansatz using cartesian coordinates ({\bf A}) 1 point,
separated equations ({\bf S}) 1 point,
\item solution of oscillation equation ({\bf O}) 1 point,
\item separation constant from periodicity ({\bf P}) 1 point,
\item superposition formula~\eqref{40000025:series}
 or $u$ ({\bf U}) 1 point.
\end{teilaufgaben}
\end{bewertung}

