Wenden Sie die Hamilton-Jacobi-Theorie auf das
zweidimensionalen Federpendel in Polarkoordinaten mit der Hamilton-Funktion 
\[
H(r,\varphi,p_r, L)=\frac1{2m}p_r^2+\frac1{2mr^2}L^2+Kr^2
\]
an.  $L$ ist der Drehimpuls, $mr^2$ das Trägheitsmoment
der Punktmasse.
Führen Sie die Separation der Hamilton-Jacobi-Differentialgleichung,
und lösen Sie sie, soweit möglich.

\begin{hinweis}
Versuchen Sie nicht, die auftretenden Integrale auszurechnen, dies
ist nur unter Verwendung sogenannter elliptischer Integrale möglich. 
\end{hinweis}

\begin{loesung}
Die Hamilton-Jacobi Differentialgleichung für die Funktion
$S(t,r,\varphi)$ lautet in diesem Fall
\begin{equation}
\frac1{2m}\biggl(\frac{\partial S}{\partial r}\biggr)^2
+\frac1{2mr^2}\biggl(\frac{\partial S}{\partial \varphi}\biggr)^2
+Kr^2
=
\frac{\partial S}{\partial t}
\label{40020002:hjdgl}
\end{equation}
Wir verwenden den Separationsansatz
\[
S(t,r,\varphi)=S_1(t)+S_2(r)+S_3(\varphi)
\]
und erhalten nach Einsetzen in die Differentialgleichung
\begin{align*}
\frac1{2m}S_2'(r)^2+\frac1{2mr^2}S_3'(\varphi)^2 + Kr^2&=S_1'(t).
\end{align*}
Es folgt wieder sofort, dass $S_1(t)=P_1t$.
Die Separation von $r$ und $\varphi$ ergibt dann
\[
\frac1{2m}S_3'(\varphi)^2=r^2\biggl(P_1-\frac1{2m}S_2'(r)^2-Kr^2\biggr)
\]
Die linke Seite hängt nur von $\varphi$ ab, die rechte nur von $r$,
also sind beide konstant, wir nennen die Konstante $P_2$ und finden
\[
S_3(\varphi)
=
\sqrt{2mP_2}\varphi.
\]
Der zweite Term in \eqref{40020002:hjdgl} ist die Rotationsenergie,
mit $P_2$ ausgedrückt wir sie
\[
\frac1{2mr^2}{L^2}
=
\frac1{2mr^2}\biggl(\frac{\partial S}{\partial \varphi}\biggr)^2
=
\frac1{2mr^2}S_3'(\varphi)^2=\frac1{r^2}P_2
\qquad\Rightarrow\qquad
P_2=\frac{L^2}{2m},
\]
bis auf einen konstanten Faktor ist $P_2$ also die bekannte Erhaltungsgrösse
des Drehimpulses.

Für $S_3(r)$ findet man jetzt den Ausdruck
\[
S_2'(r)^2=2m\biggl(P_1-\frac1{r^2}P_2-Kr^2\biggr),
\]
oder
\begin{equation}
S_2(r)
=
\sqrt{2m}\int\sqrt{P_1-\frac1{r^2}P_2-Kr^2}\,dr
=
\sqrt{2m}
\int
\frac1r
\sqrt{P_1r^2-P_2-Kr^4}\,dr.
\label{40020002:integral}
\end{equation}
Leider ist dieses Integral nicht elementar auswertbar, wir stecken also
in einer Sackgasse.
\end{loesung}

\begin{diskussion}
Immerhin kann man sich überlegen, unter welchen Voraussetzung das
Integral
\eqref{40020002:integral}
überhaupt sinnvoll ist. Der Radikand muss positiv sein, also
\begin{equation}
P_1r^2-P_2-Kr^4\ge 0.
\label{40020002:bedingung}
\end{equation}
Andererseits wird der Integrand für grosse $r$ wegen $K>0$ auf
jeden Fall negativ.  Die Bedingung \eqref{40020002:bedingung}
legt also fest, in welchem Bereich $r$ varieren kann. Die Nullstellen
des quadratischen Polynoms sind
\[
r_\pm=\frac{1}{2K}\left(P_1\pm\sqrt{P_1^2-4P_2K}\right).
\]
Eine Lösung ist nur möglich, wenn
\[
P_1^2-4P_2K\ge 0
\]
ist.
Falls $P_1^2=4P_2K$ ist überhaupt keine Variation möglich, es muss
sich also eine Kreisbahn ergeben. Für eine Kreisbahn mit Radius $r$ 
und und Kreisfrequenz $\omega$ ist der Drehimpuls $L=mr^2\omega$.
Setzen wir dies in die Kreisbahnbedingung ein, folgt
über $P_1$ und $P_2$ wissen, folgt
\begin{align*}
P_1^2&=4P_2K
=
4\frac{L^2}{2m}K
=
4\frac{m^2r^4\omega^2}{2m}K
=
4 \frac{mr^2\omega^2}{2}  \cdot Kr^2
=
4 E_{\text{kin}}\cdot E_{\text{pot}}
\end{align*}
Andererseits wissen wir auch, dass $P_1=E_{\text{kin}}+E_{\text{pot}}$,
also auch
\begin{align*}
P_1^2&=
E_{\text{kin}}^2
+ 2 
E_{\text{kin}}
E_{\text{pot}}
+
E_{\text{pot}}^2
\\
0&=
E_{\text{kin}}^2
- 2 
E_{\text{kin}}
E_{\text{pot}}
+
E_{\text{pot}}^2
=
(E_{\text{kin}}
-
E_{\text{pot}})^2
\\
\Rightarrow\qquad
E_{\text{kin}}
&=
E_{\text{pot}}
\\
\frac12mr^2\omega^2&=Kr^2
\\
\omega&=\sqrt{\frac{2K}{m}}
\end{align*}
Da auf der Kreisbahn die kinetische und die potentielle Energie
gleich gross sind, ist die Gesamtenergie einer Kreisbahn mit Radius
$r$ immer $Kr^2$.
\end{diskussion}
