Consider the partial differential equation
\begin{equation}
x^2 \frac{\partial^2 u}{\partial x^2}
+
\frac{\partial^2 u}{\partial y^2}
=
-\frac14
u
\label{40000022:eqn}
\end{equation}
on the domain 
\[
\Omega
=
\{
(x,y)\in\mathbb R^2
\;|\;
0<x<1\wedge 0<y<\pi
\}.
\]
Find a solution with boundary conditions
\[
\begin{aligned}
u(x,0) &= 0 &&\text{and}& u(x,\pi) &= 0 &&\text{for} &x&\in(0,1)
\\
u(0,y) &= 0 &&\text{and}& u(1,y) &=\sin 2y &&\text{for} &y&\in(0,\pi).
\end{aligned}
\]

\begin{hinweis}
For the dependence of the solution on $x$, try an ansatz of the form
$X(x)=x^r$
and note that only solutions with $r>0$ can be used.
\end{hinweis}

\begin{loesung}
\begin{figure}
\centering
\begin{tikzpicture}[>=latex,thick]
\node at (0,0) {\includeagraphics[width=16.8cm]{solution.jpg}};
\node at (8.2,-0.9) {$x$};
\node at (-8.1,0.2) {$y$};
\node at (5.6,3.4) {$z$};
\end{tikzpicture}
\caption{Solution surface of the differential equation
\eqref{40000022:eqn}
\label{40000022:surface}}
\end{figure}
We use separation with the ansatz $u(x,y)=X(x)Y(y)$.
Substituting the ansatz into the equation gives
\[
x^2 X''(x) Y(y) + X(x) Y''(y) = -\frac14X(x) X(x) Y(y).
\]
Dividing by $X(x)Y(y)$ allows to separate the equation:
\[
x^2 \frac{X''(x)}{X(x)} + \frac{1}{4} = - \frac{Y''(y)}{Y(y)}.
\]
Since the left hand side only depends on $x$ and the right hand side only
depends on $y$, both have to be constant, we call the common constant
$\mu^2$.
We then obtain the two equations
\begin{align*}
x^2X''(x)&=(-{\textstyle\frac{1}{4}}+\mu^2) X(x)
&&\text{and}
&
Y''(y) &= -\mu^2 Y(y)
\intertext{with homogeneous boundary conditions}
X(0)&=0
&&\text{and}
&
Y(0)&=Y(\pi)=0.
\end{align*}
The equation for $Y(y)$ has solutions $\sin(\mu y)$ and $\cos(\mu y)$.
The boundary conditions only leave $Y_k(y)=\sin ky$ where $k$ is an
integer $k>0$.
Note that $k=0$ gives the trivial solution in which we are not
interested.

For the equation for $X(x)$ we try solutions of the form $X(x)=x^r$.
Substituting this into the equation gives
\[
x^2 r(r-1)x^{r-2} = \bigl(-{\textstyle\frac{1}{4}} + k^2\bigr) x^r
\]
which is only possible if $r(r-1) = (1 + k^2)$.
There are two solutions corresponding to 
the solutions of the quadratic equation
\[
r^2-r -(k^2-{\textstyle\frac14}) = 0
\qquad\Rightarrow\qquad
r_{\pm}
=
\frac{1}{2}\pm\sqrt{\frac14+\biggl(k^2-\frac14\biggr)}
=
\frac12\pm k.
\]
Since $k>0$ is an integer, it ist at least $1$ and one of the solutions,
namely $r_-$ is always negative.
As only positive $r$ can be used, we conclude that $r_+=\frac12+k$ 
gives the solution $X_k(x) = x^{\frac12+k}$ we are interested in.

Combining the factors we get solutions
\[
u_k(x,y) = X_k(x) Y_k(y) = x^{\frac12+k} \sin(ky)
\]
which trivially satisfies the homogeneous boundary conditions.
We have to build a linear combination that also satisfies the 
boundary condition for $x=1$, i.~e.
\begin{align*}
u(x,y)
&=
\sum_{k=1}^\infty a_k x^{\frac12+k}\sin ky 
\\
\Rightarrow\qquad
u(1,y)
&=
\sum_{k=1}^\infty a_k \sin ky = \sin 2y
\\
\Rightarrow\qquad
a_k&=\begin{cases}
1&\qquad k=2\\
0&\qquad\text{otherwise}.
\end{cases}
\end{align*}
by comparing coefficients.
Thus the solution we are looking for is
\[
u(x,y) = x^{\frac52}\sin 2y.
\]
We verify that this is in fact a solution by substituting it into the
original equation. 
The partial derivatives
\begin{align*}
\frac{\partial u}{\partial x}
&=
\frac52x^{\frac32}\sin 2y
&
\frac{\partial u}{\partial y}
&=
2x^{\frac52} \cos 2y
\\
\frac{\partial^2u}{\partial x^2}
&=
\frac52\cdot\frac32 x^{\frac12}\sin 2y
&
\frac{\partial^2u}{\partial y^2}
&=
-4x^{\frac52}\sin 2y
\end{align*}
substituted into the partial differential equation~\eqref{40000022:eqn}
give
\[
x^2\frac{\partial^2u}{\partial x^2}+\frac{\partial^2 u}{\partial y^2}
=
x^2
\frac52\cdot\frac32 x^{\frac12}\sin 2y
-4x^{\frac52}\sin 2y
=
\biggl(\frac{15}4-\frac{16}{4}\biggr)
x^{\frac52}\sin 2y
=
-\frac14 x^{\frac52}\sin 2y = -\frac14u.
\]
This verifies the solution.
\end{loesung}

\begin{bewertung}
Separation ansatz ({\bf S}) 1 point,
separated equations ({\bf E}) 1 point,
solution of the equation for $Y(y)$ using homogeneous boundary conditions
({\bf Y}) 1 point,
solution of the equation for $X(x)$ ({\bf X}) 1 point,
using nonhomogeneous boundary conditions to determine coefficients ({\bf C})
1 point,
solution ({\bf U}) 1 point.
\end{bewertung}





