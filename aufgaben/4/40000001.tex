Consider the partial differential equation
\[
\frac1{1-x^2}\frac{\partial^2 u}{\partial x^2}
-
\frac{\partial^2 u}{\partial t^2}=0
\]
on the domain
\[
\Omega=\{(x,t)\,|\, x\in (-1,1), t>0\}
\]
with boundary conditions
\begin{align*}
u(t,-1)&=u(t,1)=0&t&>0\\
u(0,x)&=1-x^2&x&\in[-1,1]\\
\frac{\partial}{\partial t}u(0,x)&=0&x&\in[-1,1]\\
\end{align*}
\begin{teilaufgaben}
\item
Is this partial differential equation elliptic, parabolic and hyperbolic?
\item
Use a separation ansatz to convert the partial differential equation
into a system of ordinary differential equations.
Solve at least one of the ordinary differential equations.
\end{teilaufgaben}

\begin{loesung}
\begin{teilaufgaben}
\item
The symbol matrix of the differential operator is
\[
\begin{pmatrix}
\frac1{1-x^2}&0\\0&-1
\end{pmatrix}.
\]
As this is a diagonal matrix, the eigenvalues can be read off the diagonal.
They have different signs because $-1<x<1$ implies $1-x^2 >0$.
Therefore the partial differential equation is hyperbolic.
\item
We use the separation ansatz
\[
u(t,x)=T(t)\cdot X(t).
\]
Substituted into the partial differential equation gives
\[
\frac1{1-x^2}T(t)X''(x)-T''(t)X(x)=0.
\]
After dividing by 
$T(t)X(t)$ we get
\begin{equation}
\frac1{1-x^2}\frac{X''(x)}{X(x)}=\frac{T''(t)}{T(t)},
\label{40000001:separiert}
\end{equation}
which successfully separates the variables $x$ and $t$.

Each side of \eqref{40000001:separiert} must be constant.
If the constant were positive, we would exponentially increasing
solutions, which does not fit the wave equation.
We can therefore assume that the constant is not positive
and write it as $-\omega^2$.

For $T$ we get the ordinary differential equation
\[
T''(t)=-\omega^2 T(t)
\]
with the general solution
\[
T(t)=A\cos\omega t+B\sin\omega t.
\]
From the boundary condition $\partial_t u(0,x)=0$ it follows, that
$T'(0)=0$ or
\[
T'(0)=-A\omega\sin\omega \cdot 0+B\omega\cos\omega\cdot 0=B\omega=0.
\]
As $B=0$, the only remaining solution is
\[
T(t)=A\cos\omega t.
\]
The equation for $X$ is
\[
X''+\omega^2(1-x^2)X=0
\]
and it is not so easily solvable.
The solutions are known as the parabolic zylinder functions. 
\qedhere
\end{teilaufgaben}
\end{loesung}

