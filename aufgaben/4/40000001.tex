Betrachten Sie die partielle Differentialgleichung
\[
\frac1{1-x^2}\frac{\partial^2 u}{\partial x^2}
-
\frac{\partial^2 u}{\partial t^2}=0
\]
auf dem Gebiet
\[
\Omega=\{(x,t)\,|\, x\in (-1,1), t>0\}
\]
mit Anfangsbedingungen
\begin{align*}
u(t,-1)&=u(t,1)=0&t&>0\\
u(0,x)&=1-x^2&x&\in[-1,1]\\
\frac{\partial}{\partial t}u(0,x)&=0&x&\in[-1,1]\\
\end{align*}
\begin{teilaufgaben}
\item Ist sie elliptisch, parabolisch oder hyperbolisch?
\item F"uhren Sie die Differentialgleichung mit Hilfe eines
Separationsansatzes in ein System von gew"ohnlichen
Differentialgleichungen "uber. L"osen Sie mindestens eine der
gew"ohnlichen Differentialgleichungen.
\end{teilaufgaben}

\begin{loesung}
\begin{teilaufgaben}
\item
Die Symbolmatrix des Differentialoperators ist
\[
\begin{pmatrix}
\frac1{1-x^2}&0\\0&-1
\end{pmatrix}.
\]
Da sie diagonal ist, k"onnen die Eigenwerte von der Diagonalen
abgelesen werden. Sie haben verschiedenes Vorzeichen, wegen
$-1<x<1$ ist n"amlich $1-x^2>0$. Also ist die Differentialgleichung
hyperbolisch.
\item
Wir verwenden einen Separationsansatz der Form
\[
u(t,x)=T(t)\cdot X(t).
\]
Eingesetzt in die Differentialgleichung ergibt sich
\[
\frac1{1-x^2}T(t)X''(x)-T''(t)X(x)=0
\]
Nach Division durch $T(t)X(t)$ erhalten wir
\begin{equation}
\frac1{1-x^2}\frac{X''(x)}{X(x)}=\frac{T''(t)}{T(t)}
\label{40000001:separiert}
\end{equation}
womit die Variablen $x$ und $t$ erfolgreich separiert sind.

Jede Seite von (\ref{40000001:separiert}) muss konstant sein. W"are die
Konstante postiv, erhielten wir aus der Gleichung f"ur $T$ exponentiell
anwechsende L"osungen, was f"ur eine Wellengleichung nicht passt, wir
k"onnen also annehmen, dass die Konstante nicht positiv ist.
Daher nennen wir sie $-\omega^2$.

F"ur $T$ ergibt sich die Gleichung
\[
T''(t)=-\omega^2 T(t)
\]
mit der allgemeinen L"osung
\[
T(t)=A\cos\omega t+B\sin\omega t.
\]
Aus der Anfangsbedingung $\partial_t u(0,x)=0$ folgt jedoch, dass
$T'(0)=0$ sein muss, also
\[
T'(0)=-A\omega\sin\omega \cdot 0+B\omega\cos\omega\cdot 0=B\omega=0.
\]
Da also $B=0$, ist die einzig m"ogliche L"osung
\[
T(t)=A\cos\omega t.
\]
Die Gleichung f"ur $X$ ist
\[
X''+\omega^2(1-x^2)X=0
\]
ist nicht so einfach l"osbar, die L"osungen heissen die parabolischen
Zylinderfunktionen.
\qedhere
\end{teilaufgaben}
\end{loesung}

