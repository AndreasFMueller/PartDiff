Finden Sie eine beschr"ankte L"osung der Differentialgleichung
\begin{equation}
\frac1{1+t^2}\frac{\partial u}{\partial t}=\frac{\partial^2 u}{\partial x^2}
\label{40000010:dgl}
\end{equation}
auf dem Gebiet $\Omega=\{(x,t)\,|\, 0<x<\pi,t>0\}$ 
mit den Randbedingungen
\begin{align*}
u(0,t)=u(\pi,0)&=0&&\text{f"ur $t > 0$}\\
u(x,0)&=\sin x&&\text{f"ur $0<x<\pi$}.
\end{align*}

\begin{loesung}
Diese Differentialgleichung kann mit einem Seprationsansatz gel"ost werden.
Wir schreiben $u(x,t)=X(x)T(t)$ und setzen dies in die Differentialgleichung
ein:
\begin{align*}
\frac{\partial u}{\partial t}&=X(x)T'(t)\\
\frac{\partial^2 u}{\partial x^2}&=X''(x)T(t)\\
\end{align*}
und erhalten die separierte Differentialgleichung
\[
\frac1{1+t^2}\frac{T'(t)}{T(t)}=\frac{X''(x)}{X(x)}=\mu.
\]
Die Differentialgleichung f"ur $T$ kann mit der Separationsmethode
gel"ost werden:
\begin{align*}
T'(t)&=\mu (1+t^2)T(t)\\
\int\frac{dT}{T}&=\mu\int 1+t^2\,dt\\
\log|T|&=\mu(t+\frac13t^3) +C\\
T(t)&=K\exp\mu\biggl(t+\frac13t^3\biggr)
\end{align*}
Diese L"osungsfunktion ist nur dann beschr"ankt, kann also nur dann
f"ur die gesuchte L"osung in Frage kommen, wenn $\mu <0$.

Die Differentialgleichung f"ur $X$ ist
\begin{align*}
X''(x)&=\mu X(x)\\
\Rightarrow\qquad
X(x)&=A_\mu \cos(\sqrt{-\mu}x) + B_\mu\sin(\sqrt{-\mu}x).
\end{align*}
Die Randbedingungen verlangen, dass
\begin{align*}
X(0)&=A_\mu=0,\\
X(\pi)&=A_\mu\cos(\sqrt{-\mu}\pi)+B_\mu\sin(\sqrt{-\mu}\pi)=B_\mu\sin(\sqrt{-\mu}\pi).
\end{align*}
Da $B_\mu\ne 0$ sein muss, wenn "uberhaupt eine L"osung gefunden werden
soll, muss $\sin(\sqrt{-\mu}\pi)=0$ sein, d.~h.~$\mu$ muss eine negative ganze
Zahl sein, wir schreiben $k=\sqrt{-\mu}$ oder $\mu=-k^2$.
Damit haben wir L"osungen
\[
u_k(x,t)=K_k e^{-k^2(t+\frac13t^3)}\sin kx
\]
gefunden, die jetzt zu einer L"osung zusammengesetzt werden m"ussen, die
die Randbedingung f"ur $t=0,$ $0<x<\pi$ erf"ullt. Dazu setzen wir an:
\[
u(x,t)=\sum_{k=1}^\infty K_k e^{-k^2(t+\frac13t^3)}\sin kx
\]
und setzen die Anfangsbedingung ein:
\[
u(x,0)=\sin x=\sum_{k=1}^\infty K_k \sin kx.
\]
Koeffizientenvergleich zeigt, dass $K_1=1$ sein muss und alle anderen $K_k$
verschwinden m"ussen, die gesuchte L"osung ist also
\[
u(x,t)=e^{-(t+\frac13t^3)}\sin x.
\qedhere
\]
\end{loesung}

\begin{diskussion}
Zur Kontrolle setzen wir diese Funktion in die Differentialgleichung
ein:
\begin{align*}
\frac{\partial u}{\partial t}&=
-(1+t^2)
e^{-(t+\frac13t^3)}
\sin x
&&\Rightarrow&
\frac1{1+t^2}\frac{\partial u}{\partial t}&=
-e^{-(t+\frac13t^3)}
\sin x
\\
\frac{\partial u}{\partial x}&=
e^{-(t+\frac13t^3)}
\cos x
&&\Rightarrow&
\frac{\partial^2 u}{\partial x^2}&=
-e^{-(t+\frac13t^3)}
\sin x
\end{align*}
Die Gleichung \eqref{40000010:dgl} ist also erf"ullt.
Aber auch die Randbedingungen sind erf"ullt:
\begin{align*}
u(0,t)&=e^{-(t+\frac13t^3)}\sin 0=0\\
u(\pi,t)&=e^{-(t+\frac13t^3)}\sin\pi=0\\
u(x,0)&=\sin x,
\end{align*}
$u(x,t)$ ist also genau die gesuchte L"osung.
\end{diskussion}

\begin{bewertung}
Separationsansatz ({\bf S}) 1 Punkt,
Wahl einer Konstanten ({\bf K}) 1 Punkt,
L"osung der Gleichung f"ur $T$ ({\bf T}) 1 Punkt,
L"osung der Gleichung f"ur $X$ ({\bf X}) 1 Punkt,
Anwendung der Randbedingungen f"ur $x=0$ und $x=\pi$
zur Eingrenzung von $X$ ({\bf R}) 1 Punkt,
Anwendung der Randbedingungen f"ur $t=0$ zur Bestimmung
der L"osung ({\bf L}) 1 Punkt.
\end{bewertung}

