In der geometrischen Optik spielt die sogenannte Eikonal-Gleichung
\begin{equation}
\biggl( \frac{\partial u}{\partial x}\biggr)^2
+
\biggl( \frac{\partial u}{\partial y}\biggr)^2
=
n(x,y)^2
\label{40000007:eikonal}
\end{equation}
eine wichtige Rolle.
Eine L"osung $u(x,y)$ beschreibt die
Ausbreitung einer Welle in einem Medium mit der optischen
Dichte  $n(x,y)$, dabei sind Kurven gleicher Werte von $u(x,y)$
die Kurven gleicher Phase, entsprechen also den Wellenfronten.
Insbesondere steht der Gradient von $u$ immer senkrecht auf
der Wellenfront, zeigt also in Richtung der Lichtausbreitung.
\begin{teilaufgaben}
\item
L"osen Sie die Gleichung
(\ref{40000007:eikonal}) mit Hilfe eines Separationsansatzes
der Form $u(x,y)=X(x) + Y(y)$
f"ur 
\[
n(x,y)=
\begin{cases}n_1&\qquad x< 0\\
n_2&\qquad x>0.
\end{cases}
\]
\item
Berechnen Sie die Winkel $\alpha_1$ und $\alpha_2$ zwischen den Gradienten
Ihrer L"osung und der $x$-Achse f"ur $x<0$ und $x>0$. Bestimmen Sie 
$\sin\alpha_1/\sin\alpha_2$.
\end{teilaufgaben}

\begin{loesung}
In unserem Fall h"angt $n$ nicht von $y$ ab, ist also eigentlich
nur eine Funktion $n(x,y)=n(x)$ von $x$.
\begin{teilaufgaben}
\item
Wir verwenden den Separationsansatz
\[
u(x,y)=X(x) + Y(y).
\]
Einsetzen in die Differentialgleichung
(\ref{40000007:eikonal}) ergibt in separierter Form
\[
n(x)^2-X'(x)^2=Y'(y)^2.
\]
Die linke Seite h"angt nur von $x$ ab, die rechte nur von $y$,
also m"ussen beide konstant sein.
Wir nennen die gemeinsame Konstante, die offenbar positiv sein
muss, $\lambda^2$.
F"ur $Y(y)$ finden wir die Differentialgleichung
\[
Y'(y)=\lambda \qquad\Rightarrow\qquad Y(y)=\lambda y.
\]
F"ur $X(x)$ ergibt sich dagegen
\[
X'(x)=\sqrt{n(x)^2-\lambda^2}.
\]
Jetzt m"ussen wir die F"alle $x<0$ und $x>0$ unterscheiden.
In jedem dieser F"alle ist $n(x)$ konstant, es gilt dort
also
\[
X'(x)=\sqrt{n_i^2-\lambda^2}
\qquad
\Rightarrow
\qquad
X(x)=x\sqrt{n_i^2-\lambda^2}.
\]
Die L"osung ist also
\[
u(x,y)=\lambda y + x\sqrt{n_i^2-\lambda^2}
\]
wobei f"ur $x<0$ der Wert $n_1$ f"ur $n_i$ zu nehmen ist.
\item
Der Gradient ist
\[
\operatorname{grad}u(x,y)
=
\begin{pmatrix}
\sqrt{n_i^2-\lambda^2}\\
\lambda
\end{pmatrix}
\]
F"ur den Winkel zwischen $x$-Achse und Gradient gilt
\[
\tan\alpha_i
=
\frac{\lambda}{\sqrt{n_i^2-\lambda^2}}
=
\frac1{\sqrt{\frac{n_i^2}{\lambda^2}-1}}.
\]
Das ist gleichbedeutend mit
\[
\sin\alpha_i=\frac{\lambda}{n_i}.
\]
Daraus folgt jetzt
\[
\frac{ \sin\alpha_1}{\sin\alpha_2}=\frac{n_2}{n_1},
\]
dies ist das Brechungsgesetz von Snellius.
\end{teilaufgaben}
\end{loesung}

\begin{diskussion}
Die L"osung $u(x,y)$ der Eikonalgleichung ist die Phase einer Welle
$A(x,y)e^{iu(x,y)}$ in einer N"aherung f"ur sehr kleine Wellenl"ange.
Die Eikonalgleichung kann aus den elektromagnetischen Feldgleichungen
abgleitet werden.
In diese N"aherung sind Beugungseffekte nicht beobachtbar, und das
Brechungsgesetzt gilt exakt.
\end{diskussion}
