Reduce the problem of solving the partial differential equation
\[
x^2\frac{\partial^2 u}{\partial x^2}
+x\frac{\partial u}{\partial x}
+y^2\frac{\partial^2 u}{\partial y^2}
+y\frac{\partial u}{\partial y}
=\lambda u
\]
on the domain
$\Omega=(1,2)\times(1,2)=\{ (x,y)\in\mathbb R^2\,|\, 1<x,y<2\}$
with boundary conditions
$u=0$ on $\partial\Omega$
to a set of problems for ordinary differential equations using a
separation ansatz.

\begin{hinweis}
Don't forget the initial conditions for the ordinary differential
equations.
You are not required to actually solve the differential equations.
\end{hinweis}

\begin{loesung}
We use the ansatz
$u(x,y)=X(x)Y(y)$ and substitute it into the partial differential
equation:
\[
x^2X''(x)Y(y)
+xX'(x)Y(y)
+y^2X(x)Y''(y)
+yX(x)Y'(y)=\lambda X(x)Y(y).
\]
Dividing by $X(x)Y(y)$ gives
\[
x^2\frac{X''(x)}{X(x)}
+x\frac{X'(x)}{X(x)}
+y^2\frac{Y''(y)}{Y(y)}
+y\frac{Y'(y)}{Y(y)}=\lambda.
\]
By moving the terms with $y$ to the right hand side, we get
\[
x^2\frac{X''(x)}{X(x)}
+x\frac{X'(x)}{X(x)}
=
\lambda
-y^2\frac{Y''(y)}{Y(y)}
-y\frac{Y'(y)}{Y(y)}.
\]
As the left hand side only depends on $x$ and the right hand side only
on $y$, both sides must be constant.
Naming the constant $\mu$, the equations become
\begin{align*}
x^2\frac{X''(x)}{X(x)}
+x\frac{X'(x)}{X(x)}&=\mu
&\Rightarrow&
&
x^2X''(x)+xX'(x)-\mu X(x)&=0
\\
y^2\frac{Y''(y)}{Y(y)}
+y\frac{Y'(y)}{Y(y)}&=\lambda-\mu
&\Rightarrow&
&
y^2Y''(y)+yY'(y)+(\mu-\lambda) Y(y)&=0.
\end{align*}
with boundary conditions
\begin{align*}
X(1)&=0&Y(1)&=0\\
X(2)&=0&Y(2)&=0
\end{align*}
The ordinary differential equations happen to be Euler equations,
which can be transformed to linear differential equations with constant
coefficients using a suitable transformation.
\end{loesung}

\begin{diskussion}
The boundary conditions may be a little bit confusing.
For ordinary differential equations, one expects to linearly combine
the solutions in such a way that the boundary conditions are satisfied.
The necessary coefficients are usually found from a linear system of
equations that arises when the general solution is substituted into
the boundary conditions.
In the present case, however, there is only the zero solution of
the linear system of equations.
But we have to remember that the problem has two more parameters
$\lambda$ and $\mu$ that can be chosen in such a way that the 
solution does not identically vanish.
\end{diskussion}
