Führen Sie die Lösung des partiellen Differentialgleichung
\[
x^2\frac{\partial^2 u}{\partial x^2}
+x\frac{\partial u}{\partial x}
+y^2\frac{\partial^2 u}{\partial y^2}
+y\frac{\partial u}{\partial y}
=\lambda u
\]
auf dem Gebiet
$\Omega=(1,2)\times(1,2)=\{ (x,y)\in\mathbb R^2\,|\, 1<x,y<2\}$
mit den Randbedingungen $u=0$ auf $\partial\Omega$
mit Hilfe eines Separationsansatzes auf gewöhnliche Differentialgleichungen
zurück.

\begin{hinweis}
Vergessen Sie Anfangs-/Randbedingungen für die gewöhnlichen
Differentialgleichungen nicht.
\end{hinweis}

\begin{loesung}
Wir verwenden den Ansatz $u(x,y)=X(x)Y(y)$ und setzen dies in die
Differentialgleichung ein:
\[
x^2X''(x)Y(y)
+xX'(x)Y(y)
+y^2X(x)Y''(y)
+yX(x)Y'(y)=\lambda X(x)Y(y)
\]
Division durch $X(x)Y(y)$ ergibt
\[
x^2\frac{X''(x)}{X(x)}
+x\frac{X'(x)}{X(x)}
+y^2\frac{Y''(y)}{Y(y)}
+y\frac{Y'(y)}{Y(y)}=\lambda
\]
Schafft man alle Terme in $y$ auf die rechte Seite, ergibt sich
\[
x^2\frac{X''(x)}{X(x)}
+x\frac{X'(x)}{X(x)}
=
\lambda
-y^2\frac{Y''(y)}{Y(y)}
-y\frac{Y'(y)}{Y(y)}
\]
Da die linke Seite nur von $x$ abhängt, die rechte Seite aber nur
von $y$, sind beide Seiten Konstanten, also gelten die
Gleichungen
\begin{align*}
x^2\frac{X''(x)}{X(x)}
+x\frac{X'(x)}{X(x)}&=\mu
&\Rightarrow&
&
x^2X''(x)+xX'(x)-\mu X(x)&=0
\\
y^2\frac{Y''(y)}{Y(y)}
+y\frac{Y'(y)}{Y(y)}&=\lambda-\mu
&\Rightarrow&
&
y^2Y''(y)+yY'(y)+(\mu-\lambda) Y(y)&=0
\end{align*}
mit den Randbedingungen
\begin{align*}
X(1)&=0&Y(1)&=0\\
X(2)&=0&Y(2)&=0
\end{align*}
Die Differentialgleichungen sind übrigens eulersche Differentialgleichungen,
sie können mit einer geeigneten Transformation in lineare
Differentialgleichungen mit konstanten Koeffizienten umgeformt werden.
\end{loesung}

\begin{diskussion}
Möglicherweise verwirren die Randbedingungen etwas.
Man erwartet doch bei einer gewöhnlichen Differentialgleichung, dass
man die Lösung durch Linearkombination so zusammenbauen muss, dass die
Randbedingungen erfüllt sind.
Die dazu benötigten Koeffizienten findet man typischerweise mit einem
linearen Gleichungssystem, welches für die vorliegenden Randbedingungen
nur eine Lösung hat: die Nulllösung.
Man muss sich aber daran erinnern, dass in diesem Problem noch zwei weitere
Parameter $\lambda$ und $\mu$ vorkommen, die so gewählt werden können,
dass die Randbedingungen auch für eine Lösung erfüllt sind, die nicht
identisch verschwindet.
\end{diskussion}
