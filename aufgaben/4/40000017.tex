Die biharmonische Gleichung
\begin{equation}
\Delta\Delta u =0
\label{50000014:bi}
\end{equation}
soll
%auf dem Gebiet
%\[
%\Omega = \{(x,y)\,|\, 0<x<1\wedge 0<y<1\}
%\]
gelöst werden.
Man kann sie auch als
\begin{equation}
\frac{\partial^4u}{\partial x^4}
+
2\frac{\partial^4u}{\partial x^2\partial y^2}
+
\frac{\partial^4u}{\partial y^4}
=
0
\label{50000014:bi2}
\end{equation}
schreiben.
Verwenden Sie einen Separationsansatz der Form $u(x,y)=X(x)Y(y)$
und leiten Sie gewöhnliche Differentialgleichungen für $X(x)$ und $Y(y)$
her, mit denen das Problem gelöst werden kann.

\begin{hinweis}
Unmittelbar nach Einsetzen des Separationsansatzes lässt sich die 
Differentialgleichung~\eqref{50000014:bi2} noch nicht separieren.
Schreiben Sie sie in der Form
\[
E(x) + F(x)G(y) + H(y) = 0
\]
und leiten Sie nach $x$ und nach $y$ ab.
Zeigen Sie dann, dass mindestens einer der Terme $F(x)$ und $G(x)$ konstant
sein muss, schreiben Sie $F(x)=\lambda_1$ im ersten Fall und $G(y)=\lambda_2$ 
im zweiten Fall.
In jedem dieser zwei Fälle wird die Separation möglich.

Die Differentialgleichungen müssen nicht gelöst werden.
\end{hinweis}

\begin{loesung}
Einsetzen des Separationsansatzes $u(x,y)=X(x)Y(y)$ in 
\eqref{50000014:bi2} ergibt
\[
X^{(4)}(x)Y(y) + 2X''(x) Y''(y) + X(x)Y^{(4)}=0.
\]
Nach Division durch $X(x)Y(y)$ können wir diese Gleichung leider noch
nicht separieren:
\begin{equation}
\frac{X^{(4)}(x)}{X(x)}
+
2\frac{X''(x)}{X(x)}\cdot \frac{Y''(y)}{Y(y)}
+
\frac{Y^{(4)}(y)}{Y(y)}
=
0
\label{50000014:sep}
\end{equation}
Gemäss Hinweis schreiben wir
\begin{align*}
E(x)&=\frac{X^{(4)}(x)}{X(x)},
&
F(x)&=\frac{X''(x)}{X(x)},
&
G(y)&=\frac{Y''(y)}{Y(y)},
&
H(y)&=\frac{Y^{(4)}(y)}{Y(y)}
\end{align*}
und erhalten die Differentialgleichung in der Form
\[
E(x) + F(x) G(y) + H(y) = 0.
\]
Leitet man nach $x$ und $y$ ab, erhält man
\[
F'(x) G'(y)=0
\]
Das Produkt ist $0$, wenn einer der Faktoren verschwindet, wir haben
daher die beiden im Hinweis versprochenen Fälle 
\[
\begin{aligned}
F'(x)&=0&&\Rightarrow& F(x)&=\lambda_1&&\Rightarrow& X''(x)&=\lambda_1 X(x)
\qquad\text{und}
\\
G'(y)&=0&&\Rightarrow& G(y)&=\lambda_2&&\Rightarrow& Y''(y)&=\lambda_2 Y(y).
\end{aligned}
\]
Wir betrachten zunächst den ersten Fall, also $X''(x)=\lambda_1 X(x)$.
Einsetzen in die Differentialgleichung \eqref{50000014:sep} liefert
die Gleichung
\[
\frac{X^{(4)}(x)}{X(x)}
+
2\lambda_1\frac{Y''(y)}{Y(y)}
+
\frac{Y^{(4)}(y)}{Y(y)}
=0
\qquad\Rightarrow\qquad
-\frac{X^{(4)}(x)}{X(x)}
=
2\lambda_1\frac{Y''(y)}{Y(y)}
+
\frac{Y^{(4)}(y)}{Y(y)}.
\]
Da die linke Seite nur von $x$, die rechte aber nur von $y$ abhängt, müssen
beide konstant sein, wir schreiben $\mu_1$ für diese Konstante und
erhalten das System von Differentialgleichungen
\begin{align*}
X''(x)&=\lambda_1 X(x)
\\
X^{(4)}(x)&=-\mu_1 X(x)
\\
Y^{(4)}(y)
+
2\lambda_1 Y''(y)
&=
\mu_1 Y(y).
\end{align*}
Dies sind alles linear Differentialgleichungen mit konstanten Koeffizienten,
die mit Standardmethoden gelöst werden können.

Im zweiten Fall erhalten wir nach Einsetzen von $G(y)=\lambda_2$ in
die Differentialgleichung~\eqref{50000014:sep}
\[
\frac{X^{(4)}(x)}{X(x)}
+
2\frac{X''(x)}{X(x)}\lambda_2
+
\frac{Y^{(4)}(y)}{Y(y)}
=0
\qquad\Rightarrow\qquad
-\frac{X^{(4)}(x)}{X(x)}
-
2\frac{X''(x)}{X(x)}\lambda_2
=
\frac{Y^{(4)}(y)}{Y(y)}.
\]
Da die linke Seite nur von $x$, die rechte aber nur von $y$ abhängt, müssen
beide konstant sein, wir schreiben $-\mu_2$ für diese Konstante und erhalten
das Differentialgleichungssystem
\begin{align*}
Y''(y)&=\lambda_2 Y(y)
\\
Y^{(4)}&=-\mu_2 Y(y)
\\
X^{(4)}(x) + 2\lambda_2 X''(x) &= \mu_2 X(x).
\end{align*}
Damit haben wir für jeden Fall ein System von gewöhnlichen
Differentialgleichungen zur Bestimmung von $X(x)$ bzw.~$Y(y)$
gefunden.
\end{loesung}




