The solution of a differential equation is only determined if
complete boundary conditions are specified.
In this problem you study how the solution changes if the boundary
conditions change.
Solve the differential equation
\[
y''(x)=1
\]
on the interval $[0,1]$ with the following boundary conditions:
\begin{teilaufgaben}
\item $y(0)=0, y(1)=1$
\item $y(0)=0, y'(1)=-1$
\item $y'(0)=0, y(1)=1$
\item $y'(0)=1, y'(1)=-1$
\end{teilaufgaben}

\begin{loesung}
As a common first step we determine the general solution of the
equation which we then apply the each of the partial problem cases.

We first lookfor the solution of the homogeneous equation
$y''=0$.
Linear functions satisfy this equation, so the general solution of
the homogeneous problem is
\[
y_h(x)=Ax+B.
\]
A particular solution is easy to guess, we can use
\[
y_p(x)=\frac12x^2\Rightarrow y_p'(x)=x\Rightarrow y_p''(x)=1.
\]
The general solution of the differential equation now becomes
\[
y(x)=\frac12x^2+Ax+B, \text{\quad mit\quad} y'(x)=x+A.
\]
The values and derivatives of the solution at the interval ends are
\begin{center} 
\begin{tabular}{|c|cc|}
\hline
&0&1\\
\hline
$y$&$B$&$\frac12 +A+B$\\
$y'$&$A$&$A+1$\\
\hline
\end{tabular}
\end{center}
With these results, the problem parts a)--d) are now easy to tackle:
\begin{teilaufgaben}
\item
We get the conditions
\begin{align*}
B&=0\\
\frac12+A+B&=1
\end{align*}
from which we derivce $B=0$ and $A=\frac12$, which gives
\[
y(x)=\frac12x(x+1)
\]
as the solution.
The solution can easily be verified by substituting into the
original differential equation and the boundary conditions.
\item
In this case, the conditions are
\begin{align*}
B&=0\\
A+1&=-1\\
\end{align*}
from which we conclude $B=0$ and $A=-2$, giving
\[
y(x)=\frac12x^2-2x.
\]
\item
Conditions:
\begin{align*}
A&=0\\
\frac12+A+B&=1
\end{align*}
which implies $A=0$ and $B=\frac12$ or
\[
y(x)=\frac12(x^2+1).
\]
\item In this case the conditions are
\begin{align*}
A&=1\\
A+1&=-1
\end{align*}
The difference of these two equations is $-1=2$, which is obviously wrong.
This means that the boundary conditions cannot be met simultaneously,
and that there cannot be a solution.
\qedhere
\end{teilaufgaben}
\end{loesung}
