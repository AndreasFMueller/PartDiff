Die L"osung einer Differentialgleichung ist erst bestimmt, wenn
die Randbedingungen vollst"andig festgelegt sind. In dieser Aufgabe
studieren Sie, wie sich die L"osungen "andern, wenn man die Randbedingungen
"andert. Gegeben ist die gew"ohnliche Differentialgleichung
\[
y''(x)=1.
\]
L"osen Sie die Gleichung im Interval $[0,1]$
unter den folgenden Randbedingungen:
\begin{teilaufgaben}
\item $y(0)=0, y(1)=1$
\item $y(0)=0, y'(1)=-1$
\item $y'(0)=0, y(1)=1$
\item $y'(0)=1, y'(1)=-1$
\end{teilaufgaben}

\begin{loesung}
Zun"achst finden wir die allgemeine L"osung dieser Gleichung.
Dazu muss die L"osung der homogenen Gleichung gefunden werden,
also von $y''=0$. Lineare Funktionen erf"ullen diese Gleichung.
Die allgemeine L"osung der homogenen Gleichung ist also
\[
y_h(x)=Ax+B.
\]
Jetzt wird noch eine partikul"are L"osung der inhomogenen
Gleichung ben"otigt, ein solche ist aber einfach zu
erraten:
\[
y_p(x)=\frac12x^2\Rightarrow y_p'(x)=x\Rightarrow y_p''(x)=1.
\]
Die allgemeine L"osung der Differentialgleichung ist also
\[
y(x)=\frac12x^2+Ax+B, \text{\quad mit\quad} y'(x)=x+A
\]
Die Funktionswerte und Ableitungen am linken und rechten
Rand des Intervals sind
\begin{center}
\begin{tabular}{|c|cc|}
\hline
&0&1\\
\hline
$y$&$B$&$\frac12 +A+B$\\
$y'$&$A$&$A+1$\\
\hline
\end{tabular}
\end{center}
Mit diesen Resultaten ist es jetzt leicht, die Teilaufgaben anzupacken:
\begin{teilaufgaben}
%\item Wir erhalten die Bedingungen
%\begin{align*}
%B&=1\\
%A&=1
%\end{align*}
%also die L"osung $y(x)=\frac12x^2+x+1.$
\item Wir erhalten die Bedingungen
\begin{align*}
B&=0\\
\frac12+A+B&=1
\end{align*}
woraus wir sofort $B=0$ und $A=\frac12$ ablesen, also
\[
y(x)=\frac12x(x+1).
\]
Die Korrektheit dieser L"osung kann durch Einsetzen sofort
verifiziert werden.
\item In diesem Fall lauten die Bedingungen
\begin{align*}
B&=0\\
A+1&=-1\\
\end{align*}
woraus $B=0$ und $A=-2$ folgt, also
\[
y(x)=\frac12x^2-2x.
\]
\item
Bedingungen:
\begin{align*}
A&=0\\
\frac12+A+B&=1
\end{align*}
was $A=0$ und $B=\frac12$ zur Folge hat, also
\[
y(x)=\frac12(x^2+1).
\]
\item In diesem Fall werden die Bedingungen
\begin{align*}
A&=1\\
A+1&=-1
\end{align*}
Bildet man die Differenz dieser beiden Gleichungen
erh"alt man $-1=2$, was offensichtlich falsch ist. Diese
Randbedingungen lassen sich also nicht erf"ullen.
\qedhere
\end{teilaufgaben}
\end{loesung}
