Ein Balken wird an den beiden Enden bei $x=\pm l$ von einem Thermostat
auf verschiedenen Temperaturen $T_\pm$ gehalten. Nach einiger Zeit
hat sich ein stationärer Zustand eingestellt, also eine Temperaturverteilung
$T(t,x)$, die gar nicht von $t$ abhängt. Wie sieht diese Temperaturverteilung
aus?

\begin{loesung}
Die Temperaturverteilung muss die Wärmeleitungsgleichung erfüllen,
\[
\frac{\partial}{\partial t}T(t,x)=
a^2\frac{\partial^2}{\partial x^2}T(t,x)
\]
aber auch die Bedingung
\[
\frac{\partial}{\partial t}T(t,x)=0.
\]
$T(t,x)$ ist also nur eine Funktion $T(x)$, welche die Differentialgleichung
\[
a^2T''(x)=0
\]
erfüllt. Diese Gleichung hat die allgemeine Lösungen
\[
T(x)=Ax+B,
\]
die Konstanten $A$ und $B$ müssen so gewählt werden, dass
die Randbedingungen erfüllt sind:
\[
T(-l)=T_-\qquad T(l)=T_+
\]
was uns auf das Gleichungssystem
\begin{align*}
Al+B&=T_+\\
-Al+B&=T_-
\end{align*}
führt.
Summe und Differenz der beiden Gleichungen liefert
\begin{align*}
B&=\frac{T_++T_-}{2},
\\
A&=\frac{T_+-T_-}{2l}.
\end{align*}
Die beiden Grössen $A$ und $B$ haben eine natürliche physikalische
Bedeutung: $B$ ist die Durchschnitts\-temperatur und $A$ ist der
Temperaturgradient.
\end{loesung}
