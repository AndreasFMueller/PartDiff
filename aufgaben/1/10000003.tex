A beam is held at constant temperatures $T_\pm$ at both ends $x=\pm l$
by a thermostat.
After some time a stationary state is reached where the temperature
distribution
$T(t,x)=T(x)$ no longer depends on time.
Find this temperature distribution.

\begin{loesung}
The temperature distribution $T(x)$ must satisfy the wave equation
\[
\frac{\partial}{\partial t}T(t,x)=
a^2\frac{\partial^2}{\partial x^2}T(t,x).
\]
Stationarity means that the solution does not depend on time or that
the partial derivative with respect to $t$ vanishes as in
\[
\frac{\partial}{\partial t}T(t,x)=0.
\]
So $T(x)$ is a twice continuously differentiable function that 
satisfies
\[
a^2T''(x)=0.
\]
This equation has the general solutions
\[
T(x)=Ax+B,
\]
where the constants $A$ and $B$ must be chosen in such a way that the
boundary conditions 
\[
T(-l)=T_-\qquad T(l)=T_+
\]
hold.
This leads to the linear system of equations
\begin{align*}
Al+B&=T_+\\
-Al+B&=T_-
\end{align*}
for the unknowns $A$ and $B$
The sum and the difference of these equations easily gives the solution as
\begin{align*}
B&=\frac{T_++T_-}{2},
\\
A&=\frac{T_+-T_-}{2l}.
\end{align*}
Both terms have a natural physical interpretation.
$B$ is the average temperature and $A$ is the temperature gradient
of the stationary solution.
\end{loesung}
