Ein Balken wird an den beiden Enden bei $x=\pm l$ von einem Thermostat
auf verschiedenen Temperaturen $T_\pm$ gehalten. Nach einiger Zeit
hat sich ein station"arer Zustand eingestellt, also eine Temperaturverteilung
$T(t,x)$, die gar nicht von $t$ abh"angt. Wie sieht diese Temperaturverteilung
aus?

\ifthenelse{\boolean{loesungen}}{
\begin{loesung}
Die Temperaturverteilung muss die W"armeleitungsgleichung erf"ullen,
\[
\frac{\partial}{\partial t}T(t,x)=
a^2\frac{\partial^2}{\partial x^2}T(t,x)
\]
aber auch die Bedingung
\[
\frac{\partial}{\partial t}T(t,x)=0.
\]
$T(t,x)$ ist also nur eine Funktion $T(x)$, welche die Differentialgleichung
\[
a^2T''(x)=0
\]
erf"ullt. Diese Gleichung hat die allgemeine L"osungen
\[
T(x)=Ax+B,
\]
die Konstanten $A$ und $B$ m"ussen so gew"ahlt werden, dass
die Randbedingungen erf"ullt sind:
\[
T(-l)=T_-\qquad T(l)=T_+
\]
was uns auf das Gleichungssystem
\begin{align*}
Al+B&=T_+\\
-Al+B&=T_-
\end{align*}
f"uhrt.
Summe und Differenz der beiden Gleichungen liefert
\begin{align*}
B&=\frac{T_++T_-}{2}
\\
A&=\frac{T_+-T_-}{2l}
\end{align*}
\end{loesung}
}{}

