In der Ebene ist ein kartesisches Koordinatensystem vorgegeben. 

\vspace{2mm}

Das Gebiet $\Omega$ wird durch das Polygon mit Eckpunkten (in Reihenfolge)
\[
(0,0), \quad(2,0), \quad(2,1), \quad(3,1),  \quad(3,2),  \quad(1,2),
\quad(1,1), \quad(0,1), \quad(0,0)
\]
berandet. Die reelle Funktion $u(x,y)$ ist auf $\Omega$ definiert. Sie erfüllt im Innern von $\Omega$ die Differentialgleichung
\[
\Delta u(x,y) = 0
\]
und auf dem Rand von $\Omega$ die Bedingungen
\[
u(x,y) = 0 \qquad \text{falls} \qquad y \neq 0.
\]
Weiter sind folgende Werte von $u$ bekannt:
\[
u(1/2,0) = 1, \qquad u(3/2,0) = 2.
\]
\vspace{1mm}

Gesucht ist eine Approximation der Werte von $u$ in den vier Punkten
von $\Omega$
\[
(1/2,1/2), \qquad (3/2,1/2), \qquad (3/2,3/2), \qquad (5/2,3/2).
\]
Benutzen Sie hierzu geeignete Finite Volumina.  

\begin{loesung}
Mit 
\[
\tilde u_1 = \tilde u(1/2,1/2), \quad
\tilde u_2 = \tilde u(3/2,1/2), \quad
\tilde u_3 = \tilde u(3/2,3/2), \quad
\tilde u_4 = \tilde u(5/2,3/2), 
\]
sowie 
\[
A = \left[\begin{array}{rrrr} 
7 & -1 & 0 & 0 \\
-1 & 6 & -1 & 0 \\
0 & -1 & 6 & -1 \\ 
0 & 0 & -1 & 7 \end{array}\right]
\qquad
\text{und}
\qquad
\underline{b} =  \left(\begin{array}{r} 2 \\ 4 \\ 0 \\ 0 \end{array}\right) 
\]
gilt 
\[
A \cdot \underline{\tilde u} = \underline{b}.
\]
Es folgt 
\[
\underline{\tilde u} = (0.3934, 0.7537, 0.1287, 0.0184).
\]
\end{loesung}
