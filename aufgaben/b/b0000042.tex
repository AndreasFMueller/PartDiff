Die reelle Funktion $u(x,t)$ ist auf dem Streifen
\[
\Omega = [0, 1] \times [0,\infty)
\]
definiert.
Sie erfüllt im Innern von $\Omega$ die Wärmeleitungsgleichung
\[
u_{xx}(x,t) = u_{t}(x,t)
\]
und auf dem Rand von $\Omega$ die Bedingungen
\[
u(0,t) = e^{t}, \qquad u(1,t) = e^{1+t},
\]
sowie
\[
u(x,0) = e^x.
\]

\vspace{1mm}

Gesucht ist eine Approximation der Werte von $u$ in den zwei Punkten von
$\Omega$
\[
(1/3,2/3), \qquad  (2/3,2/3).
\]
Benutzen Sie hierzu das Verfahren von Richardson mit Schrittweiten
\[
\Delta x = 1/3
\qquad
\text{und}
\qquad
\Delta t = 1/3.
\]

\begin{loesung}
Mit
\[
A = \left[\begin{array}{rrrr} 
3 & -5 & 3 & 0 \\
0 & 3 & -5 & 3 \end{array}\right]
\qquad
\text{und}
\qquad
\underline{\tilde u}^{(0)} = (1.0000, 1.3956, 1.9477, 2.7183)
\]
folgt
\[
A \cdot \underline{\tilde u}^{(0)} = (1.8651, 2.6030)
\qquad
\text{und}
\qquad
\underline{\tilde u}^{(1)} = (1.3956, 1.8651, 2.6030, 3.7937),
\]
sowie
\[
A \cdot \underline{\tilde u}^{(1)} = (2.6703, 3.9614)
= (\tilde u(1/3,2/3), \tilde u(2/3,2/3)).
\qedhere
\]
\end{loesung}

