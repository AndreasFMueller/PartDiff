Bestimmen Sie die Knotenwerte einer Approximationslösung $\tilde u(x)$
des Randwertproblems
\[
x^2 \cdot u''(x) - 2 \cdot x \cdot u'(x) + 3 \cdot u(x) = x^2, \ \ \ x \in\, ]0,1[
\]
mit
\[
u(0) = 0 \ \ \ \text{und} \ \ \ u(1) = 1.
\]
Benutzen Sie hierzu zentrale Finite Differenzen mit Schrittweite $\Delta x = 1/4$. 

\begin{loesung}
Mit 
\[
\tilde u_1 = \tilde u(1/4), \ \tilde u_2 = \tilde u(1/2), \ \tilde u_3 = \tilde u(3/4),
\]
sowie 
\[
A = \left[\begin{array}{rrr} 
1 & 0 & 0 \\
6 & -5 & 2 \\
0 & 12 & -15 \end{array}\right] \ \ \text{und} \ \
\underline{b} =  \left(\begin{array}{c} 1/16 \\ 1/4 \\ -87/16 \end{array}\right)
\]
gilt
$A \cdot \underline{\tilde u} = \underline{b}$.
Es folgt $\underline{\tilde u} = (1/16, 1/4, 9/16)$.
\end{loesung}


