In der Ebene ist ein kartesisches Koordinatensystem vorgegeben. 

\vspace{2mm}

Das Gebiet $\Omega$ wird durch das Polygon mit Eckpunkten (in Reihenfolge)
\[
(0,0), \ \ (4,0), \ \ (4,2), \ \ (2,2),  \ \ (2,4),  \ \ (0,4), \ \  (0,0)
\]
berandet.
Die reelle Funktion $u(x,y)$ ist auf $\Omega$ definiert.
Sie erf"ullt im Innern von $\Omega$ die Differentialgleichung
\[
\Delta u(x,y) = 10
\]
und auf dem Rand von $\Omega$ die Bedingungen
\[
u(x,y) = 0.
\]

\vspace{1mm}

Gesucht ist eine Approximation der Werte von $u$ in den drei Punkten
von $\Omega$
\[
(1,1), \ \  (3,1), \ \ (1,3).
\]
Benutzen Sie hierzu geeignete Finite Volumina.  

\begin{loesung}
Mit 
\[
\tilde u_1 = \tilde u(1,1), \ \tilde u_2 = \tilde u(3,1), \ \tilde u_3 = \tilde u(1,3),
\]
sowie 
\[
A = \left[\begin{array}{rrr} 
- 6 & 1 & 1 \\
1 & -7 & 0 \\
1 & 0 & -7 \\
 \end{array}\right] \ \ \mbox{und} \ \
\underline{b} =  \left(\begin{array}{r} 40 \\ 40 \\ 40  \end{array}\right)
\]
gilt $A \cdot \underline{\tilde u} = \underline{b}.$ 
Es folgt 
\[
\underline{\tilde u} = (-9, -7, -7).
\qedhere
\]
\end{loesung}



