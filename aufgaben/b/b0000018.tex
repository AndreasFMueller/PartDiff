Die reelle Funktion $u(x,t)$ ist auf dem Streifen
\[
\Omega = [-1, 1] \times [0,\infty)
\]
definiert. Sie erf"ullt im Innern von $\Omega$ die W"armeleitungsgleichung
\[
u_{xx}(x,t) = u_{t}(x,t)
\]
und auf dem Rand von $\Omega$ die Bedingungen
\[
u(-1,t) = u(1,t) = 1
\]
sowie
\[
u(x,0) = x^2. 
\]

Gesucht ist eine Approximation der Werte von $u$ in den drei Punkten von
$\Omega$
\[
(-1/2,1), \ \  (0,1), \ (1/2,1).
\]
Benutzen Sie hierzu das Verfahren von Richardson mit Schrittweiten
\[
\Delta x = 1/2 \ \ \text{und} \ \  \Delta t = 1/4.
\]

\begin{loesung}
Mit
\[
A = \left[\begin{array}{rrr} 
-1 & 1 & 0 \\
1 & -1 & 1 \\ 0 & 1 & -1 \end{array}\right] \ \ \ \text{und} \ \ \   \underline{b} =  \left(\begin{array}{r} 1 \\ 0 \\ 1 \end{array}\right)
\]
sowie
\[
\underline{\tilde u}^{(0)} = (1/4, 0, 1/4)
\]
folgt
\[
\underline{\tilde u}^{(1)} =  A \cdot \underline{\tilde u}^{(0)} + \underline{b} = (3/4, 1/2, 3/4) \ \ \ \text{und} \ \ \ \underline{\tilde u}^{(2)} =  A \cdot \underline{\tilde u}^{(1)} + \underline{b} = (3/4, 1, 3/4),
\]
sowie
\[
\underline{\tilde u}^{(3)} =  A \cdot \underline{\tilde u}^{(2)} + \underline{b} = (5/4, 1/2, 5/4) \ \ \ \text{und} \ \ \ \underline{\tilde u}^{(4)} =  A \cdot \underline{\tilde u}^{(3)} + \underline{b} = (1/4, 2, 1/4),
\]
\end{loesung}

