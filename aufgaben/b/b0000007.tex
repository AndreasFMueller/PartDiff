In der Ebene ist ein kartesisches Koordinatensystem vorgegeben.

Das Gebiet $\Omega$ wird durch das Polygon mit Eckpunkten (in Reihenfolge)
\[
(0,0), \ \ (6,0), \ \ (6,2), \ \ (4,2),  \ \ (4,4),  \ \ (0,4), \ \  (0,0)
\]
berandet. Die reelle Funktion $u(x,y)$ ist auf $\Omega$ definiert.
Sie erfüllt im Innern von $\Omega$ die Differentialgleichung
\[
\Delta u(x,y) = 0
\]
und auf dem Rand von $\Omega$ die Bedingungen
\[
u(x,y) = 0 \ \ \ \mbox{falls} \ \ \ y \neq 0.
\]
Weiter sind folgende Werte von $u$ bekannt:
\[
u(1,0) = u(3,0) = u(5,0) = 1.
\]

\vspace{1mm}

Gesucht ist eine Approximation der Werte von $u$ in den fünf Punkten
von $\Omega$
\[
(1,1), \ \  (3,1), \ \ (5,1), \ \  (1,3), \ \ (3,3).
\]
Benutzen Sie hierzu geeignete Finite Volumina.


\begin{loesung}
Mit
\[
\tilde u_1 = \tilde u(1,1), \ \tilde u_2 = \tilde u(3,1), \ \tilde u_3 = \tilde u(5,1), \ \tilde u_4 = \tilde u(1,3), \ \tilde u_5 = \tilde u(3,3),
\]
sowie
\[
A = \left[\begin{array}{rrrrr} 
6 & -1 & 0 & -1 & 0 \\
-1 & 5 & -1 & 0 & -1 \\
0 & -1 & 7 & 0 & 0 \\ 
-1 & 0 & 0 & 6 & -1 \\
0 & -1 & 0 & -1 & 6 \\
 \end{array}\right] \ \ \mbox{und} \ \
\underline{b} =  \left(\begin{array}{r} 2 \\ 2 \\ 2 \\ 0 \\ 0 \end{array}\right)
\]
gilt $A \cdot \underline{\tilde u} = \underline{b}.$
Es folgt
\[
\underline{\tilde u} = (0.4465, 0.5858, 0.3694, 0.0933, 0.1132).
\]
\end{loesung}
