Die reelle Funktion $u(x,y)$ ist auf dem Quadrat
\[
\Omega = [0, 1] \times [0,1]
\]
definiert. Sie erfüllt im Innern von $\Omega$ die Bipotentialgleichung
\[
\Delta ( \Delta u(x,y)) = 4
\]
und auf dem Rand von $\Omega$ die Bedingungen
\[
u(x,y) = 0
\]
und
\[
\Delta u(x,y)
=
\left\{
\begin{array}{ccl}
x^2 - x & \text{falls} & y = 0 \\
x^2 - x & \text{falls} & y = 1 \\
y^2 - y & \text{falls} & x = 0 \\
y^2 - y & \text{falls} & x = 1
\end{array} \right.
\]
Gesucht ist eine Approximation der Werte von $u$ in den vier Punkten von
$\Omega$
\[
(1/3,1/3), \quad  (2/3,1/3), \quad (1/3,2/3), \quad (2/3, 2/3).
\]
Benutzen Sie hierzu geeignete Finite Differenzen.

\begin{loesung}
Seien
$
\tilde u_1
=
\tilde u(1/3,1/3)
$,
$
\tilde u_2
=
\tilde u(2/3,1/3)
$,
$
\tilde u_3
=
\tilde u(1/3,2/3)$,
$
\tilde u_4
=
\tilde u(1/3,2/3),
$
und
\[
A = \left[\begin{array}{rrrr} 
-4  & 1  & 1 & 0 \\
 1 & -4 & 0 & 1 \\
 1 & 0 & -4 & 1 \\ 
 0 & 1 & 1  & -4  \end{array}\right],
\]
sowie
$\underline b = (8/9, 8/9, 8/9, 8/9)$.
Aus $A \cdot \underline{c} = \underline{b}$ folgt
\[
\underline{c} = (-4/9, -4/9, -4/9, - 4/9).
\]

Sei $\underline d = 1/9 \cdot \underline c = (-4/81, -4/81, -4/81, - 4/81)$.
Aus $A \cdot \underline{\tilde u} = \underline{d}$ folgt
\[
\underline{\tilde u} = (2/81, 2/81, 2/81, 2/81).
\qedhere
\]
\end{loesung}


