Die reelle Funktion $u(x,t)$ ist auf dem Streifen
\[
\Omega = [0, 1] \times [0,\infty)
\]
definiert. Sie erfüllt im Innern von $\Omega$ die Wärmeleitungsgleichung
\[
u_{xx}(x,t) = u_{t}(x,t)
\]
und auf dem Rand von $\Omega$ die Bedingungen
\[
u(0,t) = u(1,t) = 0,
\]
sowie
\[
u(x,0) = 16 \cdot (x - x^2).
\]

\vspace{1mm}

Gesucht ist eine Approximation der Werte von $u$ in den zwei Punkten
von $\Omega$
\[
(1/3,1/2), \ \  (2/3,1/2).
\]
Benutzen Sie hierzu das Verfahren von Crank-Nicolson mit Schrittweiten
\[
\Delta x = 1/3 \ \ \mbox{und} \ \  \Delta t = 1/4.
\]

\begin{loesung}
Mit
\[
A = \left[\begin{array}{rr} 26 & -9 \\ -9 & 26 \end{array}\right]
\qquad
\text{und}
\qquad
B = \left[\begin{array}{rr} -10 & 9 \\ 9 & -10 \end{array}\right]
\]
gilt
\[
A \cdot \underline{\tilde u}^{(k+1)} = B \cdot \underline{\tilde u}^{(k)}
\qquad
\text{oder auch}
\qquad
\underline{\tilde u}^{(k+1)} = A^{-1}
\cdot B \cdot \underline{\tilde u}^{(k)}.
\]
Aus 
\[
\underline{\tilde u}^{(0)} = (32/9, 32/9)
\]
folgt
\[
\underline{\tilde u}^{(2)} = (32/2601, 32/2601).
\]
\end{loesung}

