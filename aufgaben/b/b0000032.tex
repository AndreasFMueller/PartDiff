Die reelle Funktion $u(x)$ ist auf dem Intervall
\[
\Omega = [0, 1]
\]
definiert. Sie erf"ullt im Innern von $\Omega$ die Differentialgleichung
\[
u''(x) + u(x) = 1
\]
und auf dem Rand von $\Omega$ die Randbedingungen
\[
u(0) = 0, \qquad
u(1) = 1.
\]
Berechnen Sie $a_1$ und $a_2$ im Approximationsansatz f\"ur $u$
\[
\tilde u(x) = v_0(x) + a_1 \cdot v_1(x) + a_2 \cdot v_2(x)
\]
mit
\[
v_0=(x) = x, \ \ \ v_1(x) = x^2 - x, \ \ \ v_2(x) = x^3 - x^2.
\]
Benutzen Sie hierzu die Methode der Gewichteten Residuen mit Gewichtsfunktionen
\[
w_1(x) = 1, \ \ \ w_2(x) = x.
\]


\begin{loesung}
Mit 
\[
M
=
\left[
\begin{matrix}
11/6 & 11/12 \\
11/12 & 19/20
\end{matrix}
\right]
\qquad\text{und}\qquad
\underline{m}
=
\begin{pmatrix}
1 - 1/2  \\ 1/2 - 1/3
\end{pmatrix}
\]
gilt $M \cdot \underline{a} = \underline{m}$.
Es folgt $a_1 = 0.3575, \ a_2 = -0.1695$.
\end{loesung}


