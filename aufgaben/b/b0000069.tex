\begin{teilaufgaben}
\item
\ifthenelse{\boolean{pruefung}}{(2 points)}{}
The function $u(x)$ is defined on the interval $\Omega = [0,1].$ 

The function $u(x)$ satisfies in $\Omega$ 
\[
1/9 \cdot u''(x) + 2 \cdot u(x) + 9 \cdot x^2 = 0
\]
and  $u(0) = u(1) = 0$ on the boundary of $\Omega$. 

Determine approximate values for $u(1/3)$ and $u(2/3)$. 

Use finite differences with $h = 1/3$. 

\item
\ifthenelse{\boolean{pruefung}}{(4 points)}{}
The function $u(x,t)$ is defined on the strip
$\Omega = [-1, 1] \times [0,\infty)$.

The function $u(x,t)$ satisfies in $\Omega$
\[
u_{t}(x,t) = u_{xx}(x,t)
\]
and $u(-1,t) = u(1,t) = 0$ as well as $u(x,0) = 4 \cdot (1-x^2)$
on the boundary of $\Omega$.

Determine approximate values for $u(-1/2,1/2)$, $u(0,1/2)$ and $u(1/2,1/2)$. 

Use the method of Crank-Nicolson with $\Delta x = 1/2$ and $\Delta t = 1/4.$
\end{teilaufgaben}

\begin{loesung}
\begin{teilaufgaben}
\item
Set $\tilde u_1 = \tilde u(1/3), \tilde u_2 = \tilde u(2/3)$ as well as
\[
A
=
\left[\begin{array}{rr} 0 & 1  \\ 1 & 0  \end{array}\right]
\qquad \text{and}
\underline{b}
=
\left(\begin{array}{r} -1 \\ -4 \end{array}\right).
\]
Then $A \cdot \underline{\tilde u} = \underline{b}$.
Therefore
\[
\underline{\tilde u} = (-4,-1).
\]

\item
Set
\[
\tilde u_1^{(k)}
=
\tilde u(-1/2,k/2), \tilde u_2^{(k)}
=
\tilde u(0,k/2), \tilde u_3^{(k)}
=
\tilde u(1/2,k/2)
\]
as well as
\[
F
=
\left[\begin{array}{rrr} 4 & - 1 & 0 \\ -1 & 4 & -1 \\ 0 & -1 & 4 \end{array}\right]
\qquad
\text{and}\qquad
G
=
\left[\begin{array}{rrr} 0 & 1 & 0  \\ 1 & 0 & 1 \\ 0 & 1 & 0 \end{array}\right].
\]
Then
\[
F \cdot \underline{\tilde u}^{(k+1)}
=
G \cdot \underline{\tilde u}^{(k)}.
\]
With $\underline{\tilde u}^{(0)} = (3,4,3)$ one obtains
\[
\underline{\tilde u}^{(1)}
=
(11/7,16/7,11/7)
\]
and
\[
\underline{\tilde u}^{(2)}
=
(43/49,60/49, 43/49).
\qedhere
\]
\end{teilaufgaben}
\end{loesung}

\begin{bewertung}
\begin{teilaufgaben}
\item 2 points
\item 4 points
\end{teilaufgaben}
\end{bewertung}
