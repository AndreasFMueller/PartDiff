In der Ebene ist ein kartesisches Koordinatensystem vorgegeben. 

\vspace{2mm}

Die reelle Funktion $u(x,y)$ ist auf dem Viereck
\[
\Omega = [0,1] \times [0,1]
\]
definiert. Sie erf"ullt im Innern von $\Omega$ die
Differentialgleichung 
\[
\Delta u(x,y) = 0
\]
und auf dem Rand von $\Omega$ die Bedingungen
\[
u(x,0) = u(0,y) = 0 \ \ \ \mbox{und} \ \ \ u(x,1) = x
\]
sowie
\[
\frac{\partial u}{\partial n}(1,y) = y.
\]
($\frac{\partial
u}{\partial n}$ ist die Ableitung von $u$ in Richtung der "ausseren
Normalen).

\vspace{2mm}

Gesucht ist eine Approximation der Werte von $u$ in den vier Punkten von
$\Omega$
\[
(1/3,1/3), \ \  (2/3,1/3), \ \ (1/3,2/3), \ \ (2/3,2/3).
\]
Benutzen Sie hierzu geeignete Finite Differenzen.

\begin{loesung}
Mit 
\begin{align*}
\tilde u_1&= \tilde u(1/3,1/3),&\tilde u_2&= \tilde u(2/3,1/3),&\tilde u_3&= \tilde u(1,1/3)\\
\tilde u_4&= \tilde u(1/3,2/3),&\tilde u_5&= \tilde u(2/3,2/3),&\tilde u_6&= \tilde u(1,2/3)
\end{align*}
sowie 
\[
A = \left[\begin{array}{rrrrrr} 
-4 & 1 & 0 & 1 & 0 & 0 \\
1 & -4 & 1 & 0 & 1 & 0 \\
0 & 2 & -4 & 0 & 0 & 1 \\
1 & 0 & 0 & -4 & 1 & 0 \\ 
0 & 1 & 0 & 1 & -4 & 1 \\
0 & 0 & 1 & 0 & 2 & -4 \end{array}\right]
\qquad 
\text{und}
\qquad
\underline{b} =  \left(\begin{array}{r} 0 \\ 0 \\ -2/9 \\ -3/9 \\ -6/9 \\ -13/9 \end{array}\right)
\]
gilt 
\[
A \cdot \underline{\tilde u} = \underline{b}.
\]
Es folgt 
\[
\tilde u_1 = 1/9, \tilde u_2 = 2/9, \tilde u_4 = 2/9, \tilde u_5 = 4/9
\]
\end{loesung}

