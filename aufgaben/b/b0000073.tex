\begin{teilaufgaben}
\item
The function $u(x)$ is defined on the interval $\Omega = [1,4].$ 

The function $u(x)$ satisfies in $\Omega$ 
\[
x^2 \cdot u''(x) + u(x) = 3 \cdot x^2
\]
and  $u(1) = 1$, $u(4) = 16$ on the boundary of $\Omega$. 

Determine approximate values for $u(2)$ and $u(3)$. 

Use finite differences with $h = 1$. 
\ifthenelse{\boolean{pruefung}}{\hfill{(2 points)}}{}
\item
The function $u(x,t)$ is defined on the strip
$\Omega = [0, 3] \times [0,\infty)$.

The function $u(x,t)$ satisfies in $\Omega$
\[
u_{t}(x,t) = u_{xx}(x,t)
\]

and $u(0,t) = u(3,t) = 0$ as well as $u(x,0) = 3 \cdot x - x^2$
on the boundary of $\Omega$.

Determine approximate values for $u(1,1)$ and $u(2,1)$. 

Use the explicit method of Richardson with $\Delta x = 1$ and $\Delta t = 1/4.$
\ifthenelse{\boolean{pruefung}}{\hfill{(4 points)}}{}
\end{teilaufgaben}

\begin{loesung}
\begin{teilaufgaben}
\item
By the method of finite differences we have
\[
4 \cdot (\tilde u(3) - 2 \cdot \tilde u(2) + 1) + \tilde u(2)
=
12
\qquad
\text{and}
\qquad
9 \cdot (16 - 2 \cdot \tilde u(3) + \tilde u(2)) + \tilde u(3)
=
27.
\tag{\bf 1P}
\]
The solution of this system of two linear equations is
\[
(\tilde u(2), \tilde u(3)) = (4,9).
\tag{\bf 1P}
\]
\item
We use the explicit method of Richardson with $r = 1/4$.
For the
first time step we  have
\begin{align*}
\tilde u(1, 1/4)
&=
1/4 \cdot \tilde u(0,0) + 1/2 \cdot \tilde u(1,0) + 1/4 \cdot \tilde u(2,0) 
\intertext{and}
\tilde u(2, 1/4)
&=
1/4 \cdot \tilde u(1,0) + 1/2
\cdot \tilde u(2,0) + 1/4 \cdot \tilde u(3,0).
\tag{\bf 1P}
\end{align*}
Since $u(0,t) = u(3,t) = 0$ we have in general 
\[
\left(\begin{array}{r} \tilde u(1,k/4) \\ u(2,k/4) \end{array} \right)
=
\left[ \begin{array}{rr} 1/2 & 1/4  \\ 1/4 & 1/2  \end{array}\right]^k
\cdot
\left(\begin{array}{r} \tilde u(1,0) \\ u(2,0) \end{array} \right)
\tag{\bf 1P}
\]
and in particular
\[
\left(\begin{array}{r} \tilde u(1,1) \\ u(2,1) \end{array} \right)
=
\left[\begin{array}{rr} 1/2 & 1/4  \\ 1/4 & 1/2  \end{array}\right]^4
\cdot \left(\begin{array}{r} 2 \\ 2 \end{array} \right)
\tag{\bf 1P}
\]
Therefore
\[
(\tilde u(1,1), \tilde u(2,1))
=
(81/128; 81/128) = (0.6328125, 0.6328125).
\tag{\bf 1P}
\]
\end{teilaufgaben}
\end{loesung}

\begin{bewertung}
\begin{teilaufgaben}
\item 2 points
\item 4 points
\end{teilaufgaben}
\end{bewertung}
