Die reelle Funktion $u(x,t)$ ist auf dem Streifen
\[
\Omega = [0, 1] \times [0,\infty)
\]
definiert. Sie erfüllt im Innern von $\Omega$ die Wellengleichung
\[
u_{xx}(x,t) = u_{tt}(x,t)
\]
und auf dem Rand von $\Omega$ die Bedingungen
\[
u(0,t) = u(1,t) = 0,
\]
sowie
\[
u(x,0) = 0 \ \ \ \mbox{und} \ \ \ u_t(x,0) = 9/2 \cdot (x-x^2).
\]
Gesucht ist eine Approximation der Werte von $u$ in den zwei Punkten von $\Omega$
\[
(1/3, 1), \ \  (2/3, 1).
\]
Benutzen Sie hierzu geeignete Finite Differenzen mit Schrittweiten
\[
\Delta x = 1/3 \ \ \  \mbox{und} \ \  \ \Delta t = 1/4.
\]

\begin{loesung}
Mit
\[
\tilde u_{j,k} = u(j \cdot \Delta x, k \cdot \Delta t)
\]
gilt
\[
\tilde u_{j, k+1} = \frac{9 \cdot \tilde u_{j-1, k} + 14 \cdot \tilde u_{j, k} + 9 \cdot \tilde u_{j+1, k} - 16 \cdot \tilde u_{j, k-1}}{16}
\]
und
\[
\tilde u_{1,0} = \tilde u_{2,0} = 0 \ \ \ \mbox{sowie} \ \ \ \tilde u_{1,1} = \tilde u_{2,1} = 1/4.
\]
Es folgt
\[
\tilde u_{1,2} = \tilde u_{2,2} = 23/64, \ \ \ \tilde u_{1,3} = \tilde u_{2,3} = 273/1024, \ \ \ \tilde u_{1,4} = \tilde u_{2,4} = 391/16384.
\]
\end{loesung}

