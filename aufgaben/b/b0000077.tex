(3 points + 3 points)
\begin{teilaufgaben}
\item
The function $u(x)$ is defined on the interval $\Omega = [-2,2].$ 

The function $u(x)$ satisfies in $\Omega$ 
\[
u''(x) + 2 \cdot u'(x) + u(x) = x^2 + 4 \cdot x + 2
\]
and  $u(-2) = 4, u(2) = 4$ on the boundary of $\Omega$. 

Determine approximate values for $u(-1)$, $u(0)$ and $u(1)$. 
Use central finite differences with $h = 1$. 

\item
The function $u(x,t)$ is defined on the strip
$\Omega = [-2, 2] \times [0,\infty)$.

\vspace{2mm}

The function $u(x,t)$ satisfies in $\Omega$
\[
u_{t}(x,t) = u_{xx}(x,t)
\]
and $u(-2,t) = u(2,t) = 0$ as well as $u(x,0) = 4 - x^2$ on the
boundary of $\Omega$.

Determine approximate values for $u(-1,1)$ $u(0,1)$ and $u(1,1)$. 

Use the implicit method of Richardson with $\Delta x = 1$ and
$\Delta t = 1/2.$
\end{teilaufgaben}

\begin{loesung}
\begin{teilaufgaben}
\item
By the method of finite differences we have
\begin{gather*}
(\tilde u(-2) - 2 \cdot \tilde u(-1) + \tilde u(0))
+ (- \tilde u(-2) + \tilde u(0)) + \tilde u(-1)
= -1
\\
(\tilde u(-1) - 2 \cdot \tilde u(0) + \tilde u(1))
+ (- \tilde u(-1) + \tilde u(1)) + \tilde u(0)
= 2
\\
(\tilde u(0) - 2 \cdot \tilde u(1) + \tilde u(2))
+ (- \tilde u(0) + \tilde u(2)) + \tilde u(1)
= 7
\end{gather*}
This leads to a system of three linear equations:
\begin{equation}
\begin{pmatrix*}[r]
-1 &  2 & 0 \\
 0 & -1 & 2 \\
 0 & 0 & -1
\end{pmatrix*}
\cdot
\begin{pmatrix}
\tilde u(-1) \\
\tilde u(0)  \\
\tilde u(1)
\end{pmatrix}
=
\begin{pmatrix*}[r]
-1 \\
 2 \\
-1
\end{pmatrix*}.
\tag{\bf 2P}
\end{equation}
Its solution is
\begin{equation}
(\tilde u(-1),  \tilde u(0), \tilde u(1))
=
(1, 0 , 1).
\tag{\bf 1P}
\end{equation}

\item
We use the implicit method of Richardson with $r = 1/2$. 

For the first time step we have  
\begin{equation}
\begin{pmatrix*}[r]
 3 \\
 4 \\
 3
\end{pmatrix*}
=
\begin{pmatrix}
 2   & -1/2 & 0 \\
-1/2 & 2 & -1/2 \\
 0   & -1/2 & 2
\end{pmatrix}
\cdot
\begin{pmatrix}
\tilde u(-1,1/2) \\
\tilde u(0,1/2)  \\
\tilde u(1,1/2)
\end{pmatrix}.
\tag{\bf 1P}
\end{equation}

For the second time step we have  
\begin{equation}
\begin{pmatrix*}[r]
 16/7 \\
 22/7 \\
 16/7
\end{pmatrix*}
=
\begin{pmatrix}
  2   & -1/2 &  0   \\
 -1/2 &  2   & -1/2 \\
  0   & -1/2 &  2
\end{pmatrix}
\cdot
\begin{pmatrix}
\tilde u(-1,1)  \\
\tilde u(0,1) \\
\tilde u(1,1)
\end{pmatrix}.
\tag{\bf 1P}
\end{equation}
Therefore
\begin{equation}
(\tilde u(-1,1), \tilde u(0,1), \tilde u(1,1))
=
(86/49, 120/49, 86/49).
\tag{\bf 1P}
\end{equation}
\qedhere
\end{teilaufgaben}
\end{loesung}
