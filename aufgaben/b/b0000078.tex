(2 points + 4 points)
\begin{teilaufgaben}
\item
The function $u(x,t)$ is defined on the upper half plane
$\Omega = (-\infty,\infty) \times [0,\infty)$.

The function $u(x,t)$ satisfies in $\Omega$ the equation
\[
u_{x}(x,t) + u_{t}(x,t) = 0
\]
and  $u(x,0) = x^2$ on the boundary of $\Omega$.

Determine an approximate value for $u(0,2)$. 

Use the upwind scheme with $\Delta x = 1$ and $\Delta t = 1.$

\item
The function $u(x,t)$ is defined on the strip
$\Omega = [-2, 2] \times [0,\infty)$.

The function $u(x,t)$ satisfies in $\Omega$ the equation
\[
u_{xx}(x,t) = u_{tt}(x,t)
\]
and  $u(-2,t) = 0$, $u(2,t) = 0$ on the boundary of $\Omega$. 

The function $u(x,t)$ also satisfies the initial conditions
\[
u(x,0) = 4 - x^2 \ \ \ \mbox{and} \ \ \ u_t(x,0) = 4 - x^2.
\]

Determine approximate values for $u(-1,1)$, $u(0,1)$ and $u(1,1)$  

Use finite differences with $\Delta x = 1$ and $\Delta t = 1/2.$
\end{teilaufgaben}


\begin{loesung}
\begin{teilaufgaben}
\item
We use the upwind scheme with $r = 1$.
The corresponding iteration formula is
\begin{equation}
\tilde u_{j,k+1} = \tilde u_{j-1,k}.
\tag{\bf 1P}
\end{equation}
Therefore
\begin{equation}
\tilde u(0,2) = \tilde u(-1,1) = \tilde u(-2,0)
=
4.
\tag{\bf 1P}
\end{equation}

\item
We have $f(x) = 4 - x^2$ and $g(x) = 4 - x^2$. 

Hence for the first time step we have by Taylor
\begin{equation}
\tilde u(x_j, 1/2)
=
(4 - x_j^2) \cdot 1 + (4 - x_j^2) \cdot 1/2 - 2/2 \cdot (1/2)^2
\tag{\bf 1P}
\end{equation}
where $x_{-1} = -1$, $x_0 = 0$ and $x_1 = 1$.
Therefore
\begin{equation}
\tilde u(-1, 1/2)
=
17/4, \tilde u(0, 1/2) = 23/4, \tilde u(1, 1/2)
=
17/4.
\tag{\bf 1P}
\end{equation}

For the second time step we use the leapfrog scheme with $r = 1/2$. We have
\begin{gather}
\tilde u(-1, 1)
=
1/4 \cdot \tilde u(-2, 1/2)
+
6/4 \cdot \tilde u(-1,1/2)
+
1/4 \cdot  \tilde u(0, 1/2)
-
\tilde u(-1,0)
\notag
\\
\tilde u(0, 1)
=
1/4 \cdot \tilde u(-1, 1/2)
+
6/4 \cdot \tilde u(0,1/2)
+
1/4 \cdot  \tilde u(1, 1/2)
-
\tilde u(0,0)
\notag
\\
\tilde u(1, 1)
=
1/4 \cdot \tilde u(0, 1/2)
+ 6/4 \cdot \tilde u(1,1/2)
+ 1/4 \cdot  \tilde u(2, 1/2)
-
\tilde u(1,0).
\tag{\bf 1P}
\end{gather}

This leads to
\begin{equation}
\tilde u(-1, 1)
=
77/16, \ \tilde u(0, 1) = 108/16, \ \tilde u(1, 1)
=
77/16.
\tag{\bf 1P}
\end{equation}
\end{teilaufgaben}
\end{loesung}
