Die reelle Funktion $u(x,t)$ ist auf dem Streifen
\[
\Omega = [0, 1] \times [0,\infty)
\]
definiert.
Sie erfüllt im Innern von $\Omega$ die Wärmeleitungsgleichung
\[
u_{xx}(x,t) = u_{t}(x,t)
\]
und auf dem Rand von $\Omega$ die Bedingungen
\[
u(0,t) = 2 \cdot t, \ \ u(1,t) = 1 + 2 \cdot t,$$ sowie $$u(x,0) = x^2.
\]

Gesucht ist eine Approximation der Werte von $u$ in den zwei Punkten von
$\Omega$
\[
(1/3,2/3), \ \  (2/3,2/3).
\]
Benutzen Sie hierzu das Verfahren von Richardson mit Schrittweiten
\[
\Delta x = 1/3 \ \ \mbox{und} \ \  \Delta t = 1/3.
\]

\begin{loesung}
Mit
\[
A = \left[\begin{array}{rrrr} 
3 & -5 & 3 & 0 \\
0 & 3 & -5 & 3 \end{array}\right]
\quad
\text{und}\quad
\underline{\tilde u}^{(0)} = (0, 1/9, 4/9, 1)
\]
folgt
\[
A \cdot \underline{\tilde u}^{(0)} = (7/9, 10/9)
\quad\mbox{und}\quad
\underline{\tilde u}^{(1)} = (6/9, 7/9, 10/9, 15/9),
\]
sowie
\[
A \cdot \underline{\tilde u}^{(1)} = (13/9, 16/9) = (\tilde u(1/3,2/3), \tilde u(2/3,2/3)).
\]
\end{loesung}



