Die reelle Funktion $u(x,y)$ ist auf dem Quadrat
\[
\Omega = [0, 1] \times [0,1]
\]
definiert. Sie erf"ullt im Innern von $\Omega$ die Laplacegleichung
\[
\Delta u(x,y) = 0
\]
und auf dem Rand von $\Omega$ die Randbedingung
\[
\Delta u(x,y) = \left\{
\begin{array}{ccl}
0 & \text{falls} & y = 0 \\
x & \text{falls} & y = 1 \\
0 & \text{falls} & x = 0 \\
y & \text{falls} & x = 1
\end{array} \right.
\]
Gesucht ist eine Approximation der Werte von $u$ in den vier Punkten
von $\Omega$
\[
(1/5,1/5), \ \  (3/5,1/5), \ \ (1/5,3/5), \ \  (3/5,3/5).
\]
Benutzen Sie hierzu Finite Volumina nach Voronoi.  

\begin{loesung}
Mit
\[
\tilde u_1 = \tilde u(1/5,1/5), \ 
\tilde u_2 = \tilde u(3/5,1/5), \ 
\tilde u_3 = \tilde u(1/5,3/5), \ 
\tilde u_4 = \tilde u(3/5,3/5),
\]
sowie
\[
A
=
\left[
\begin{array}{rrrr}
-6 & 1 & 1 & 0 \\
 2 & -13 & 0 & 3 \\
 2 & 0 & -13 & 3 \\
 0 & 1 & 1 & -4
\end{array}
\right]
\ \ \text{und} \ \
\underline{b}
=
\begin{pmatrix}
0 \\ -2/5 \\ -2/5 \\ -6/5
\end{pmatrix}
\]
\vspace{5mm}  
gilt $A \cdot \underline{\tilde u} = \underline{b}$.
Es folgt $\underline{\tilde u} = (1/25, 3/25, 3/25, 9/25).$
\end{loesung}

