\begin{teilaufgaben}
\item
The function $u(x,t)$ is defined on the strip
$\Omega = [0, 3] \times [0,\infty)$
satisfies in $\Omega$
\[
u_{t}(x,t) = u_{xx}(x,t)
\]
and $u(0,t) = u(3,t) = 1$ as well as $u(x,0) = 3 \cdot x - x^2 + 1$
on the boundary of $\Omega$.

Determine approximate values for $u(1,2)$ and $u(2,2)$. 

Use the explicit method of Richardson with $\Delta x = 1$ and $\Delta t = 1.$

\item
The function $u(x,t)$ is defined on the upper half plane
$\Omega = (-\infty,\infty) \times [0,\infty)$
and
%The function $u(x,t)$
satisfies in $\Omega$ the equation
\[
u_{x}(x,t) + u_{t}(x,t) = 0
\]
and  $u(x,0) = \sin(\pi/2 \cdot x)^2$ on the boundary of $\Omega$.

Determine approximate values for $u(-1,1)$, $u(0,1)$ and $u(1,1)$. 

Use the scheme of Lax-Wendroff with $\Delta x = 1$ and $\Delta t = 1/2.$
\end{teilaufgaben}

\begin{loesung}
\begin{teilaufgaben}
\item
(3 points) Set
$\tilde u_1^{(k)}
=
\tilde u(1,k), \tilde u_2^{(k)} = \tilde u(2,k)$
as well as
\[
A
=
\left[\begin{array}{rr} -1 & 1  \\ 1 & -1  \end{array}\right]
\qquad\text{and}\qquad
\underline{b}
=
\left(\begin{array}{r} 1  \\ 1   \end{array}\right).
\]
Then $\underline{\tilde u}^{(k+1)}
= A \cdot \underline{\tilde u}^{(k)} + \underline{b}$.
With $\underline{\tilde u}^{(0)} = (3,3)$ one has
$\underline{\tilde u}^{(1)} = (1,1)$ and $\underline{\tilde u}^{(2)} = (1,1)$. 

\item
(3 points) One has $\tilde u(k \cdot \Delta x, (j+1) \cdot \Delta t) =$
\[
3/8 \cdot \tilde u((k-1) \cdot \Delta x, j \cdot \Delta t) + 3/4 \cdot \tilde u(k \cdot \Delta x, j \cdot \Delta t) - 1/8 \cdot \tilde u((k+1) \cdot \Delta x, j \cdot \Delta t).
\]

With $\tilde u(k \cdot \Delta x, 0) = \sin(\pi/2 \cdot k \cdot \Delta x)^2$
one obtains
\[
\tilde u(-1,1) = 5/8, \tilde u(0,1) = 3/8, \tilde u(1,1) = 5/8.
\qedhere
\]
\end{teilaufgaben}
\end{loesung}

