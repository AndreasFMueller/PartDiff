Die reelle Funktion $u(x,t)$ ist auf dem Streifen
\[
\Omega = [0, 1] \times [0,\infty)
\]
definiert. Sie erfüllt im Innern von $\Omega$ die Wärmeleitungsgleichung
\[
u_{t}(x,t) = u_{xx}(x,t)
\]
und auf dem Rand von $\Omega$ die Bedingungen
\[
u(0,t) = 1 \qquad \text{und} \qquad  u(1,t) = 2,
\]
sowie
\[
u(x,0) = 1 + 2 \cdot x - x^2.
\]
Gesucht ist eine Approximation der Werte von $u$ in den zwei Punkten von $\Omega$
\[
(1/3, 1/4), \quad  (2/3, 1/4).
\]
Benutzen Sie hierzu das elementare Implizite Finite Differenzen Verfahren mit Schrittweiten
\[
\Delta x = 1/3 \qquad  \text{und} \qquad \Delta t = 1/12.
\]

\begin{loesung}
Seien $\tilde u_{j,k} = u(j \cdot \Delta x, k \cdot \Delta t)$, sowie
\[
E
=
\left[
\begin{array}{rr}
10/4 & - 3/4 \\
-3/4 & 10/4
\end{array}\right]
\qquad
\text{und}
\qquad
\underline{e}
=
\begin{pmatrix}
-3/4 \\ -6/4
\end{pmatrix}.
\]
Es gilt $\tilde u^{(k)} = E \cdot \tilde u^{(k+1)} + e$
und $\tilde u^{(k+1)} = E^{-1} \cdot (\tilde u^{(k)} - e).$
Aus $\tilde u^{(0)} = (14/9, 17/9)$ folgt dann
$\tilde u^{(3)} = (1.3748, 1.7081)$
\end{loesung}


