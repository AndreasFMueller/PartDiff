\begin{teilaufgaben}
\item
The function $u(x,t)$ is defined on the upper half plane
$\Omega = (-\infty,\infty) \times [0,\infty)$.

The function $u(x,t)$ satisfies in $\Omega$ the equation
\[
u_{x}(x,t) + u_{t}(x,t) = 0
\]
and  $u(x,0) = x^2$ on the boundary of $\Omega$.

Determine an approximate value for $u(0,2)$. 


Use the Lax-Wendroff scheme with $\Delta x = 1$ and $\Delta t = 2/3.$
\ifthenelse{\boolean{pruefung}}{\hfill{(3 points)}}{}
\item
The function $u(x,t)$ is defined on the strip
$\Omega = [0, 3] \times [0,\infty)$.

The function $u(x,t)$ satisfies in $\Omega$ the equation
\[
u_{xx}(x,t) = u_{tt}(x,t)
\]
and  $u(0,t) = 0$, $u(3,t) = t$ on the boundary of $\Omega$. 

The function $u(x,t)$ also satisfies the initial conditions
\[
u(x,0) = 0
\qquad\text{and}\qquad
u_t(x,0) = x/3.
\]

Determine approximate values for $u(1,1)$ and $u(2,1)$.

Use finite differences with $\Delta x = 1$ and $\Delta t = 1/2.$
\ifthenelse{\boolean{pruefung}}{\hfill{(3 points)}}{}
\end{teilaufgaben}

\begin{loesung}
\begin{teilaufgaben}
\item
%----
%
We use the scheme of Lax-Wendroff with $r = 2/3$.
For the first time step we have
\[
\tilde u(k,2/3)
=
5/9 \cdot \tilde u(k-1, 0)
+
5/9 \cdot \tilde u(k, 0)
-
1/9 \cdot u(k+1,0).
\tag{\bf 1P}
\]
As $\tilde u(k, 0) =  k^2$ we obtain
\[
u(-2,2/3) = 64/9, \quad u(-1,2/3) = 25/9, \quad u(0,2/3) = 4/9,
\]
\[
u(1,2/3) = 1/9, \quad u(2,2/3) = 16/9.
\tag{\bf 1P}
\]
Recursively we obtain
\[
u(-1,4/3) = 49/9, \ u(0,4/3) = 16/9, \ u(1,4/3) = 1/9
\]
and finally
\[
u(0,6/3) = 4.
\tag{\bf 1P}
\]
%
%----
%
%We use the scheme of Lax-Wendroff with $r = 2/3$.
%For the first time step we have
%\[
%\tilde u(k,2/3)
%=
%5/9 \cdot \tilde u(k-1, 0)
%+
%5/9 \cdot \tilde u(k, 0)
%-
%1/9 \cdot u(k+1,0).
%\tag{\bf 1P}
%\]
%As $\tilde u(k, 0) =  k^2$
%we obtain
%\[
%u(-2,2/3) = 64/9, \quad u(-1,2/3) = 25/9, \quad u(0,2/3) = 4/9, 
%\]
%\[
%u(1,2/3) = 1/9, \quad u(2,2/3) = 26/9.
%\tag{\bf 1P}
%\]
%Recursively we obtain
%\[
%u(-1,4/3) = 49/9, \quad u(0,4/3) = 16/9, \quad u(-1,2/3) = 1/9
%\]
%and finally
%\[
%u(0,6/3) = 4.
%\tag{\bf 1P} 
%\]
%
\item
We have $f(x) = 0$ and $g(x) = x/3$.
Hence for the first time step we have by Taylor
\[
\tilde u(1, 1/2)
=
1/3 \cdot 1 \cdot 1/2 = 1/6
\qquad\text{and}\qquad
\tilde u(2, 1/2) = 1/3 \cdot 2 \cdot 1/2 = 1/3.
\tag{\bf 1P}
\]
For the second time step we use the leapfrog scheme with $r = 1/2$.
We have
\[ 
\tilde u(1, 1)
=
1/4 \cdot \tilde u(0, 1/2)
+
6/4 \cdot \tilde u(1,1/2)
+
1/4 \cdot  \tilde u(2, 1/2)
-
\tilde u(1,0)
\]
and
\[
\tilde u(2, 1)
=
1/4 \cdot \tilde u(1, 1/2)
+
6/4 \cdot \tilde u(2,1/2)
+
1/4 \cdot \tilde u(3, 1/2)
-
\tilde u(2,0).
\tag{\bf 1P}
\]
This leads to
\[
\tilde u(1, 1) = 1/3, \quad \tilde u(1, 1) = 2/3
\tag{\bf 1P}
\]
\end{teilaufgaben}
\end{loesung}

\begin{bewertung}
\begin{teilaufgaben}
\item 3 points
\item 3 points
\end{teilaufgaben}
\end{bewertung}
