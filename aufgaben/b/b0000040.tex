Die reelle Funktion $u(x)$ ist auf dem Intervall
\[
\Omega = [0, 2 \cdot \pi]
\]
definiert.
Sie erfüllt im Innern von $\Omega$ die Differentialgleichung
\[
u''(x) + 2 \cdot u(x) = x
\]
und auf dem Rand von $\Omega$ die Randbedingungen
\[
u(0) = 0, \ \ u(2 \cdot \pi) = 0.
\]
Berechnen Sie $a_1, a_2, a_3$ im Approximationsansatz für $u$
\[
\tilde u(x) = a_1 \cdot v_1(x) + a_2 \cdot v_2(x) + a_3 \cdot v_3(x)
\]
mit
\[
v_1(x) = \sin(x), \ \ \ v_2(x) = \sin(2 \cdot x) \ \ \ \text{und} \ \ \  v_3(x) = \sin(3 \cdot x).
\]
Benutzen Sie hierzu die Methode von Galerkin.

\begin{loesung}
Mit
\[
G = \left[\begin{array}{ccc} 
\pi & 0 & 0 \\
0 & - 2 \cdot \pi & 0 \\
0 & 0 & -7 \cdot \pi \\
 \end{array}\right] \ \ \text{und} \ \
\underline{g}
=
\left(\begin{array}{c} -2 \cdot \pi \\ - \pi \\ - 2 \cdot \pi/3  \end{array}\right)
\]
folgt aus $G \cdot \underline{a} = \underline{g}$
\[
\underline{a} = (-2, 1/2, 2/21)
\]
und
\[
\tilde u(x) = -2 \cdot \sin(x) + 1/2 \cdot \sin(2 \cdot x) + 2/21 \cdot \sin(3 \cdot x).
\qedhere
\]
\end{loesung}
