\begin{teilaufgaben}
\item (2 points)
The function $u(x)$ is defined on the interval $\Omega = [0, 1].$

\vspace{2mm}

The function satisfies in $\Omega$ 
\[
-u''(x) = 10
\]
and $u(0) = 0$ as well as $u(1) = 0$ on the boundary of $\Omega$.  

\vspace{2mm}

Determine approximate values for $u(1/3)$ and $u(1/2)$ by the finite
element method. 

\vspace{2mm}

Use linear finite elements.

\item (4 points)
The function $u(x)$ is defined on the interval $\Omega = [0, 3].$

\vspace{2mm}

The function satisfies in $\Omega$ 
\[
x^2 \cdot u''(x) - 12 \cdot u(x) = 0
\]
and $u(0) = 0$ and $u(3) = 3$ on the boundary of $\Omega$.  

\vspace{2mm}

Determine an approximation function $\tilde u(x)$ for $u(x)$ with the
method of Galerkin.

\vspace{2mm}

Use the ansatz
$\tilde u(x)
=
x + a_1 \cdot (3 \cdot x - x^2) + a_2 \cdot (27 \cdot x - x^4)$.
\end{teilaufgaben}



\begin{loesung}
\begin{teilaufgaben}
\item
The corresponding Ritz matrix is
\begin{equation}
R
=
\left[
\begin{array}{rrrr}
3 & -3 & 0 & 0 \\
3 & 9 & - 6 & 0 \\
0 & -6 & 8 & -2 \\
0 & 0 & -2 & 2
\end{array}
\right]
\tag{1P}
\end{equation}
whereas the Ritz vector is
\[
\underline{r} = (*,10/4,10/3,*).
\]
The vector of the knot variables is
\[
\underline{a} = (0, a_1, a_2, 0)
\]
Reduction of the Ritz system
$R \cdot  \underline{a} = \underline{0}$
leads to  
\[
\left[
\begin{array}{rr}
9 & -6 \\
 -6 & 8
\end{array}
\right]
\cdot
\left(
\begin{array}{r}
a_1 \\
a_2
\end{array}
\right)
=
\left(
\begin{array}{r}
10/4 \\
10/3
\end{array}
\right).
\]
Its solution is
$\underline{a} = (1/8, 1/9)$.
Therefore we have $\tilde u(1/3) = 10/9$ and
\begin{equation}
\tilde u(1/2) = 10/8. \tag{1P}
\end{equation}

\item
Plugging
\[
\tilde u''(x) = -2 \cdot a_1 - 12 \cdot a_2 \cdot x^2
\]
and
\[
\tilde u(x)
=
x + a_1 \cdot (3 \cdot x - x^2) + a_2 \cdot (27 \cdot x - x^4)
\]
into the left side of the differential equation we obtain
\begin{equation}
-12 \cdot x + (-36 \cdot x + 10 \cdot x^2) \cdot a_1
- 324 \cdot x \cdot a_2
\tag{1P}
\end{equation}
By Galerkin $a_1$ and $a_2$ have to satisfy the equations
\begin{align}
\int_{0}^3 (-12 \cdot x + (-36 \cdot x + 10 \cdot x^2) \cdot a_1
- 324 \cdot x \cdot a_2) \cdot (3 \cdot x - x^2) \,dx
&=
0
\notag
\intertext{and}
\int_{0}^3 (-12 \cdot x + (-36 \cdot x + 10 \cdot x^2) \cdot a_1
- 324 \cdot x \cdot a_2) \cdot (27 \cdot x - x^4) \,dx
&=
0.
\tag{1P}
\end{align}
The system of two linear equations
\begin{equation}
\left[
\begin{array}{cc}
 -243/2 & -2187 \\
-28431/14 & -551124/14
\end{array}
\right]
\cdot
\left(
\begin{array}{r}
a_1 \\
a_2
\end{array}
\right)
=
\left(
\begin{array}{r}
81 \\
1458
\end{array}
\right)
\tag{1P}
\end{equation}
has the solution $(a_1, a_2) = (0,-1/27)$ and therefore the approximation
function is
\begin{equation}
\tilde u(x)
=
x - 1/27 \cdot (27 \cdot x - x^4)
=
1/27 \cdot x^4.
\tag{1P}
\end{equation}

\end{teilaufgaben}
\end{loesung}
