\begin{teilaufgaben}
\item 
The function $u(x)$ is defined on the interval $\Omega = [0, 1].$

The function satisfies in $\Omega$ 
\[
-u''(x) = 0
\]
and $u(0) = 40$ as well as $u(1) = 10$ on the boundary of $\Omega$.  

Determine approximate values for $u(1/3)$ and $u(2/3)$ by the finite element method. 

Use linear finite elements with $h = 1/3$.
\ifthenelse{\boolean{pruefung}}{\hfill{(2 points)}}{}
\item
The function $u(x)$ is defined on the interval $\Omega = [0, 1].$

The function satisfies in $\Omega$ 
\[
x \cdot u''(x) - u'(x) = 3 \cdot x^2
\]
and $u(0) = 0, \  u(1) = 0$ on the boundary of $\Omega$.  

Determine an approximation function $\tilde u(x)$ for $u(x)$ with the method of Galerkin.

Use the ansatz $\tilde u(x) = a_1 \cdot (x^2-x) + a_2 \cdot (x^3-x).$
\ifthenelse{\boolean{pruefung}}{\hfill{(4 points)}}{}
\end{teilaufgaben}

\begin{loesung}
\begin{teilaufgaben}
\item
The corresponding Ritz matrix is
\[
R
=
\left[\begin{array}{rrrr}
3 & -3 & 0 & 0 \\
-3 & 6 & - 3 & 0 \\
0 & -3 & 6 & -3 \\
0 & 0 & -3 & 3
\end{array}\right],
\tag{\bf 1P}
\]
whereas the Ritz vector $\underline{r}$ is  zero.
The vector of the knot variables is
\[
\underline{a}
=
\left(\begin{array}{r} 40 \\ a_1 \\ a_2 \\ 10 \end{array}\right).
\]
Reduction of the Ritz system
$R \cdot  \underline{a} = \underline{0}$ leads to the system 
\[
\left[\begin{array}{rr}  6 & - 3  \\  -3 & 6  \end{array}\right]
\cdot
\left(\begin{array}{r} a_1 \\ a_2 \end{array}\right)
+
\left(\begin{array}{r} -120 \\ -30 \end{array}\right)
=
\left(\begin{array}{r} 0 \\0 \end{array}\right).
\]
Its solution is $\underline{a} = (30, 20)$.
Therefore we have $\tilde u(1/3) = 30$ and
$\tilde u(2/3) = 20.$
\hfill{({\bf 1P})}

\item
Plugging
$\tilde u'(x) = a_1 \cdot (2 \cdot x - 1) + a_2 \cdot (3 \cdot x^2 - 1)$
and
$\tilde u''(x) = a_1 \cdot 2 + a_2 \cdot 6 \cdot x$
into the left side of the differential equation we obtain
\[
a_1 + a_2 \cdot (1+3 \cdot x^2).
\tag{\bf 1P}
\]
By Galerkin $a_1$ and $a_2$ have to satisfy the equations
\begin{align*}
\int_0^1 (a_1 + a_2 \cdot (1+3 \cdot x^2)) \cdot (x^2 -x) \, dx
&=
\int_0^1 3 \cdot x^2 \cdot (x^2 -x) \, dx
\intertext{and}
\int_0^1 (a_1 + a_2 \cdot (1+3 \cdot x^2)) \cdot (x^3 - x) \, dx
&=
\int_0^1 3 \cdot x^2 \cdot (x^3 - x) \ dx.
\tag{\bf 1P} 
\end{align*}
Computing the integrals we obtain the system of two linear equations
\[
\left[\begin{array}{cc} -1/8 & -19/60  \\ -1/4 & -1/2 \end{array}\right]
\cdot
\left(\begin{array}{c} a_1 \\ a_2 \end{array}\right)
=
\left(\begin{array}{c} -3/20 \\ -3/12 \end{array}\right).
\tag{\bf 1P}
\]
Its solution is $(a_1, a_2) = (-1,1)$ and therefore the approximation
function is
\[
\tilde u(x) = x^3 - x^2.
\tag{\bf 1P}
\]
\end{teilaufgaben}
\end{loesung}

\begin{bewertung}
\begin{teilaufgaben}
\item 2 points
\item 4 points
\end{teilaufgaben}
\end{bewertung}

Grade key for part 2 of the exam: $\text{grade} = \displaystyle\frac{\text{points}}4 + 1.5$
