\begin{teilaufgaben}
\item (2 points)
The domain $\Omega$ is the square $[0,2] \times [0, 2]$.
The function $u(x,y)$ is defined on $\Omega$. 

\vspace{2mm}

It satisfies in $\Omega$ 
\[
\Delta u(x,y) = x^2 + y^2.
\]
On the boundary of $\Omega$ one has $u(1,0) = u(0,1) = 0$ and
$u(2,1) = u(1,2) = 2$. 

\vspace{2mm}

Determine an approximate value for $u(1,1)$. 

\vspace{2mm}

Use finite differences with $h = 1$.

\vspace{8mm}

\item (4 points)
The domain $\Omega$ is a square with vertices
\[
(4,0), (0,4), (-4, 0), (0,-4).
\]
The function $u(x,y)$ is defined on $\Omega$.
It satisfies in $\Omega$ 
\[
\Delta u(x,y) = 0.
\]
On the boundary of $\Omega$ one has $u(x,y) = 1$ in the first quadrant
and $u(x,y) = 0$ otherwise.

\vspace{2mm}

Determine approximate values for $u(-1,-1)$ and $u(1,1)$.

\vspace{2mm}

Use finite volumes \`a la Voronoi.  
\end{teilaufgaben}


\begin{loesung}
\begin{teilaufgaben}
\item
By the five-point-star operator we have
\begin{equation}
\tilde u(2,1) + \tilde u(1,2) + \tilde u(0,1) + \tilde u(1,0)
- 4 \cdot    \tilde u(1,1)
=
2.
\tag{1P}
\end{equation}
Hence
\[
2 + 2 + 0 + 0 - 4 \cdot \tilde u(1,1)
=
2
\]
and
\begin{equation}
\tilde u(1,1) = 1/2. \tag{1P}
\end{equation}

\vspace{5mm}

\item
By the finite volume method we have
\begin{align}
4 \cdot (0 - \tilde u(-1,-1)) + (0 - \tilde u(-1,-1)) + (0 - \tilde u(-1,-1))
+ 2 \cdot (\tilde u(1,1) - \tilde u(-1,-1))
=
0,
\notag
\\
4 \cdot (1 - \tilde u(1,1)) + (0 - \tilde u(1,1)) + (0 - \tilde u(1,1))
+ 2 \cdot (\tilde u(-1,-1) - \tilde u(1,1)) = 0.
\tag{2P}
\end{align}
or simpler
\begin{align}
-8 \cdot \tilde u(-1,-1) + 2 \cdot \tilde u(1,1) &= 0,
\notag
\\
2 \cdot \tilde u(-1,-1) -8 \cdot \tilde u(1,1) &= -4.  \tag{1P}
\end{align}
Therefore
\begin{equation}
(\tilde u(-1,-1), \tilde u(1,1))
=
(2/15, 8/15). \tag{1P}
\end{equation}
\end{teilaufgaben}
\end{loesung}
