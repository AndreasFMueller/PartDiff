\begin{teilaufgaben}
\item (3 points)
The function $u(x)$ is defined on the interval $\Omega = [0,6].$ 

\vspace{2mm}

The function $u(x)$ satisfies in $\Omega$ 
\[
u''(x) + 2 \cdot u'(x) + u(x) = x^2 + 4 \cdot x + 2
\]

and  $u(0) = 0, u(6) = 36$ on the boundary of $\Omega$. 

\vspace{2mm}

Determine approximate values for $u(2)$ and $u(4)$. 

\vspace{2mm} Use central finite differences with $h = 2$. 


\item (3 points)
The function $u(x,t)$ is defined on the strip
$\Omega = [0, 6] \times [0,\infty)$.

\vspace{2mm}

The function $u(x,t)$ satisfies in $\Omega$
\[
u_{t}(x,t) = u_{xx}(x,t)
\]

and $u(0,t) = u(6,t) = 0$ as well as $u(x,0) = 6 \cdot x - x^2$ on the boundary of $\Omega$.


Determine approximate values for $u(2,1)$ and $u(4,1)$.

Use the explicit method of Richardson with $\Delta x = 2$ and $\Delta t = 1/3.$

\end{teilaufgaben}



\begin{loesung}
\begin{teilaufgaben}
\item
By the method of finite differences we have
\begin{align*}
(\tilde u(4) - 2 \cdot \tilde u(2) + \tilde u(0))
+ 2 \cdot (- \tilde u(0) + \tilde u(4))
+ 4 \cdot \tilde u(2)
&=
56
\\
(\tilde u(6) - 2 \cdot \tilde u(4) + \tilde u(2))
+ 2 \cdot (- \tilde u(2) + \tilde u(6)) + 4 \cdot \tilde u(4)
&=
136.
\end{align*}
This leads to a system of three linear equations:
\begin{equation}
\begin{pmatrix*}[r] 2 & 3 \\  -1 & 2 \end{pmatrix*}
\cdot
\begin{pmatrix} \tilde u(2)  \\   \tilde u(4)  \end{pmatrix}
=
\begin{pmatrix*}[r] 56  \\  28 \end{pmatrix*}.
\tag{2P}
\end{equation}
Its solution is
\begin{equation}
(\tilde u(2),  \tilde u(4)) = (4 , 16).
\tag{1P}
\end{equation}



\item
We use the explicit method of Richardson with $r = 1/12$. 

\smallskip

The Richardson matrix is
\begin{equation}
C
=
\begin{pmatrix*}[c] 10/12 & 1/12 \\  1/12 & 10/12 \end{pmatrix*}.
\tag{1P}
\end{equation}
The result can be obtained by iteration:
\begin{equation}
\begin{pmatrix} \tilde u(2,1)  \\ \tilde u(4,1)  \end{pmatrix}
=
\begin{pmatrix*}[c] 10/12 & 1/12 \\  1/12 & 10/12 \end{pmatrix*}^3
\cdot
\begin{pmatrix} 8  \\ 8  \end{pmatrix}.
\tag{1P}
\end{equation}
Therefore
\begin{equation}
(\tilde u(2,1), \tilde u(4,1))
=
(1331/216, 1331/216).
\tag{1P}
\end{equation}

\end{teilaufgaben}
\end{loesung}
