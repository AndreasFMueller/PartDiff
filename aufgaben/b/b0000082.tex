\begin{teilaufgaben}
\item (2 points)
The function $u(x,t)$ is defined on the upper half plane
$\Omega = (-\infty,\infty) \times [0,\infty)$.

\vspace{2mm}

The function $u(x,t)$ satisfies in $\Omega$ the equation
\[
u_{x}(x,t) + u_{t}(x,t) = 0
\]
and  $u(x,0) = x^3$ on the boundary of $\Omega$.

\vspace{2mm}

Determine an approximate value for $u(0,1)$. 

\vspace{2mm}

Use the Lax-Wendroff scheme with $\Delta x = 1$ and $\Delta t = 1/2.$

\vspace{8mm}

\item (4 points)
The function $u(x,t)$ is defined on the strip
$\Omega = [0, \pi] \times [0,\infty)$.

\vspace{2mm}

The function $u(x,t)$ satisfies in $\Omega$ the equation
\[
u_{xx}(x,t) = u_{tt}(x,t)
\]
and  $u(0,t) = 0$, $u(\pi,t) = 0$ on the boundary of $\Omega$. 

\vspace{2mm}

The function $u(x,t)$ also satisfies the initial conditions
\[
u(x,0) = \sin(x) \ \ \ \mbox{and} \ \ \ u_t(x,0) = \sin(x)^2.
\]

\vspace{2mm}

Determine approximate values for $u(\pi/2,1)$.

\vspace{2mm}

Use finite differences with $\Delta x = \pi/2$ and $\Delta t = 1/2.$

\end{teilaufgaben}


\begin{loesung}
\begin{teilaufgaben}
\item
We use the Lax-Wendroff scheme with $r = 1/2$.
The corresponding iteration formula is
\begin{equation}
\tilde u_{j,k+1}
=
3/8 \cdot \tilde u_{j-1,k} + 6/8 \cdot \tilde u_{j,k}
- 1/8 \cdot \tilde u_{j+1,k}.
\tag{1P}
\end{equation}
Therefore
\begin{equation}
\tilde u(-1,1/2)
=
-30/8,
\quad
\tilde u(0,1/2) = -4/8,
\quad
\tilde u(1,1/2) = -2/8
\quad\text{and}\quad
\tilde u(0,1) = -7/4.
\tag{1P}
\end{equation}

\vspace{5mm}

\item
We have $f(x) = \sin(x)$ and $g(x) = \sin(x)^2$. 

\smallskip

Hence for the first time step we have by Taylor
\begin{equation}
\tilde u(\pi/2, 1/2)
=
\sin(\pi/2) \cdot 1 + \sin(\pi/2)^2 \cdot 1/2
- 1/2 \cdot \sin(\pi/2) \cdot (1/2)^2.
\tag{1P}
\end{equation}
Therefore
\begin{equation}
\tilde u(\pi/2, 1/2) = 11/8.
\tag{1P}
\end{equation}
For the second time step we use the leapfrog scheme with $r = 1/\pi$.
\begin{equation}
\tilde u(\pi/2, 1)
=
1/\pi^2 \cdot \tilde u(0, 1/2)
+ 2 \cdot (1 - 1/\pi^2) \cdot \tilde u(\pi/2,1/2)
+ 1/\pi^2 \cdot \tilde u(\pi, 1/2) - \tilde u(\pi/2,0).
\tag{1P}
\end{equation}
This leads to
\begin{equation}
\tilde u(\pi/2, 1) \approx 1.471. \tag{1P}
\end{equation}
\end{teilaufgaben}
\end{loesung}

