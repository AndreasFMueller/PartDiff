\begin{teilaufgaben}
\item
The function $u(x)$ is defined on the interval $\Omega = [0, 1].$
The function satisfies in $\Omega$ 
\[
u''(x) + u(x) = 1
\]
and $u(0) = 0, \  u(1) = 0$ on the boundary of $\Omega$.  
Determine an approximation function $\tilde u(x)$ for $u(x)$ with the method of Galerkin.
Use the ansatz $\tilde u(x) = a_1 \cdot x^2 \cdot (x-1) + a_2 \cdot x \cdot (x-1)^2.$
\item
The function $u(x)$ is defined on the interval $\Omega = [0, 1].$
The function satisfies in $\Omega$ 
\[
- u''(x) = 1
\]
and $u(0) = 0, \  u(1) = 0$ on the boundary of $\Omega$.  
Determine the values of an approximation function $\tilde u(x)$ for $u(x)$ at the
nodal points by linear finite elements with $h = 1/3$.
\end{teilaufgaben}

\begin{loesung}
\begin{teilaufgaben}
\item
(3 points) Set
\[
B
=
\left[\begin{array}{rr} -13/105 & 11/420  \\ 11/420 & -13/105  \end{array}\right]
\qquad\text{and}\qquad
\underline{b}
=
\left(\begin{array}{r} -1/12 \\ 1/12 \end{array}\right).
\]
Then $B \cdot \underline{a} = \underline{b}$.
Therefore $\underline{a} = (5/9,-5/9)$ and $\tilde u(x) = 5/9 \cdot (x^2 - x)$.

\item
(3 points) Set
\[
R
=
\left[\begin{array}{rr} 6 & -3  \\ -3 & 6  \end{array}\right]
\qquad\text{and}\qquad
\underline{r}
=
\left(\begin{array}{r} 1/3 \\ 1/3 \end{array}\right).
\]
Then $R \cdot \underline{a} = \underline{r}$.
Therefore $\underline{a} = (1/9,1/9)$.
\qedhere
\end{teilaufgaben}
\end{loesung}

