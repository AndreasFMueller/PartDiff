\begin{teilaufgaben}
\item
The function $u(x,t)$ is defined on the strip
$\Omega = [0, 3] \times [0,\infty)$.
The function $u(x,t)$ satisfies in $\Omega$
\[
u_{t}(x,t) = u_{xx}(x,t)
\]
and $u(0,t) = u(3,t) = 0$ as well as $u(x,0) = 3 \cdot x - x^2$ on the boundary of $\Omega$.
Determine approximate values for $u(1,2)$ and $u(2,2)$. 
Use the method of Crank-Nicholson with $\Delta x = 1$ and $\Delta t = 1.$

\item
The function $u(x,t)$ is defined on the strip $\Omega = [0, 3] \times [0,\infty)$.
The function $u(x,t)$ satisfies in $\Omega$ the equation
\[
u_{tt}(x,t) = u_{xx}(x,t)
\]
and  $u(x,0) = 0, u_t(x,0) = x$ as well as $u(0,t) = 0, u(3,t) = 3 \cdot t$ on the boundary of $\Omega$.
Determine approximate values for $u(1,2)$ and $u(2,2)$. 
Use finite differences with $\Delta x = 1$ and $\Delta t = 1.$
\end{teilaufgaben}

\begin{loesung}
\begin{teilaufgaben}
\item
(3 points) Set $\tilde u_1^{(k)} = \tilde u(1,k), \tilde u_2^{(k)} = \tilde u(2,k)$ as well as
\[
A
=
\left[\begin{array}{rr} 1/15 & 4/15  \\ 4/15 & 1/15  \end{array}\right].
\]
Then $\underline{\tilde u}^{(k+1)} = A \cdot \underline{\tilde u}^{(k)}$.
With $\underline{\tilde u}^{(0)} = (2,2)$ one has $\underline{\tilde u}^{(1)} = (2/3,2/3)$
and $\underline{\tilde u}^{(2)} = (2/9,2/9)$. 

\item
(3 points) Set $\tilde u_1^{(k)} = \tilde u(1,k), \tilde u_2^{(k)} = \tilde u(2,k)$.
Then $\underline{\tilde u}^{(0)} = (0,0)$ and $\underline{\tilde u}^{(1)} = (1,2)$.

Now $\tilde u_1^{(2)} = 0 + \tilde u_2^{(1)} - \tilde u_1^{(0)} = 2$ and
$\tilde u_2^{(2)} = \tilde u_1^{(1)} + 3 - \tilde u_2^{(0)} = 4$.
\qedhere
\end{teilaufgaben}
\end{loesung}

