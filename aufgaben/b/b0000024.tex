Die reelle Funktion $u(x)$ ist auf dem Intervall
\[
\Omega = [0, 1]
\]
definiert. Sie erf"ullt im Innern von $\Omega$ die Differentialgleichung
\[
u''(x) + u'(x) = 1
\]
und auf dem Rand von $\Omega$ die Randbedingungen
\[
u(0) = 0, \ \ u(1) = 0.
\]
Berechnen Sie $a_1$ und $a_2$ im Approximationsansatz f\"ur $u$
\[
\tilde u(x) = a_1 \cdot v_1(x) + a_2 \cdot v_2(x)
\]
mit
\[
v_1(x) = x^3 - 2 \cdot x^2 + x \ \ \ \text{und} \ \ \  v_2(x) = x^3 - x^2.
\]
Benutzen Sie hierzu  a) die Methode von Galerkin und b) die Methode von Gauss.

\begin{loesung}
\begin{teilaufgaben}
\item
Mit 
\[
G = \left[\begin{array}{rr} -8/60 & 1/60 \\ 3/60 & -8/60  \end{array}\right] \ \ \text{und} \ \
  \underline{g} =  \left(\begin{array}{r} 5/60  \\ - 5/60 \end{array}\right)
\]
gilt $G \cdot \underline{a} = \underline{g}.$ Es folgt $a_1 = -35/61, \ a_2 = 25/61$.

\item
Mit
\[
Q = \left[\begin{array}{rr} 94/30 & 59/30 \\ 59/30 & 154/30  \end{array}\right] \ \ \text{und} \ \
  \underline{q} =  \left(\begin{array}{r} -1  \\ 1 \end{array}\right)
\]
gilt $Q \cdot \underline{a} = \underline{q}$.
Es folgt $a_1 = -426/733, \ a_2 = 306/733$.
\end{teilaufgaben}
\end{loesung}

