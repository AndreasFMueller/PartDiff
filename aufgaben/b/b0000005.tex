In der Ebene ist ein kartesisches Koordinatensystem vorgegeben.

Die reelle Funktion $u(x,y)$ ist auf dem Quadrat
\[
\Omega = [0,1] \times [0,1]
\]
definiert. Sie erfüllt im Innern von $\Omega$ die Differentialgleichung
\[
\Delta u(x,y) = 0
\]
und auf dem Rand von $\Omega$ die Bedingungen
\[
u(x,0) = 0, u(0,y) = 0, u(x,1) = x
\]
sowie
\[
\frac{\partial u}{\partial n}(1,y) = 1.
\]

Gesucht ist eine Approximation der Werte von $u$ in den vier Punkten von
$\Omega$
\[
(1/3,1/3), \ \  (2/3,1/3), \ \ (1/3,2/3), \ \ (2/3,2/3).
\]
Benutzen Sie hierzu geeignete Finite Differenzen.


\begin{loesung}
Mit
\[
\tilde u_1 = \tilde u(1/3,1/3), \ \tilde u_2 = \tilde u(2/3,1/3), \ \tilde u_3 = \tilde u(1,1/3),$$ $$\tilde u_4 = \tilde u(2/3,1/3), \ \tilde u_5 = \tilde u(2/3,2/3), \ \tilde u_6 = \tilde u(2/3,1),
\]
sowie
\[
A = \left[\begin{array}{rrrrrr} 
-4 & 1 & 0 & 1 & 0 & 0\\
1 & -4 & 1 & 0 & 1 & 0 \\
0 & 2 & -4 & 0 & 0 & 1 \\ 

1 & 0 & 0 & -4 & 1 & 0 \\
0 & 1 & 0 & 1 & -4 & 1 \\
0 & 0 & 1 & 0 & 2 & -4 \end{array}\right] \ \ \mbox{und} \ \
\underline{b} =  \left(\begin{array}{r} 0 \\ 0 \\ -2/3 \\ -1/3 \\ -2/3 \\ -5/3 \end{array}\right)
\]
gilt $A \cdot \underline{\tilde u} = \underline{b}.$ Es folgt
\[
\underline{\tilde u} \approx (0.1307, 0.2828, 0.5057, 0.2397, 0.4949, 0.7905).
\]
\end{loesung}

