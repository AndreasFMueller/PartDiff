In der Ebene ist ein kartesisches Koordinatensystem vorgegeben. 

\vspace{2mm}

Das Dreieck $\Omega$ besitzt die Eckpunkte
\[
(0,0), \ \ (4,0), \ \ (0,4).
\]

Die reelle Funktion $u(x,y)$ ist auf dem Dreieck $\Omega$ definiert. 

\vspace{2mm}

Sie erfüllt im Innern von $\Omega$ die Laplacesche Differentialgleichung 
\[
\Delta u(x,y) = 0
\]
und auf dem Rand von $\Omega$ die Dirichletschen Randbedingungen 
\[
u(x,y) = \left\{ \begin{array}{ccc} 4 - x & \mbox{falls} & y = 0
\\ 4 - y & \mbox{falls} & x = 0  \\ 0  & \mbox{sonst} & \end{array} \right.       .
\]

Gesucht ist eine Approximation der Werte von $u$ in den zwei Punkten
von $\Omega$ 
\[
(2,1), \ \ (1,2).
\]
Benutzen Sie hierzu Finite Volumina nach Voronoi.

\begin{loesung}
Mit
\[
A = \left[\begin{array}{rr} -10 & 2  \\ 2 & -10  \end{array}\right] \ \ \mbox{und} \ \
  \underline{b} =  \left(\begin{array}{r} -8 \\ -8 \end{array}\right)
\]
gilt  
\[
A \cdot \underline{\tilde u} = \underline{b}
\]
Hierbei ist
\[
\underline{\tilde u} \approx (u(2,1), u(1,2)).
\]
Es folgt 
\[
\underline{\tilde u} = (1,1).
\]
\end{loesung}
