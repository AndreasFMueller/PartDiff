Betrachten Sie die Differentialgleichung
\[
y^2\frac{\partial^2 u}{\partial x^2}
+4y\frac{\partial^2u}{\partial x\partial y}
+\frac{\partial^2 u}{\partial y^2}=0
\]
auf dem Gebiet
\[
\Omega = \{ (x,y)\,|\, y >0\}.
\]
\begin{teilaufgaben}
\item Zeigen Sie, dass die Differentialgleichung hyperbolisch ist.
\item Stellen Sie die Gleichung der Charakteristiken auf.
\item Die Differentialgleichung der Charakteristiken ist eine
gew"ohnliche Differentialgleichung erster Ordnung f"ur zwei unbekannte
Funktionen $x(s)$ und $y(s)$, faktorisieren Sie sie mit zwei Faktoren
der Form
\[
(\dot x(s)-ry(s)\dot y(s))\cdot (\dot x(s)-qy(s)\dot y(s)).
\]
Bestimmen
sie die Werte von $r$ und $q$.
\item
Finden Sie die Gleichung der Charakteristiken durch
L"osung der Differentialgleichungen
\begin{align*}
\dot x(s) - ry(s)\dot y(s)&=0\qquad\text{und}\\
\dot x(s) - qy(s)\dot y(s)&=0,
\end{align*}
verwenden Sie dabei
$\frac{d}{ds}(y(s)^2)=2y(s)\dot y(s)$.
\end{teilaufgaben}

\begin{loesung}
\begin{teilaufgaben}
\item
Die Determinante der Symbolmatrix $A$ ist
\[
A=\begin{pmatrix}
y^2&2y\\
2y&1
\end{pmatrix}
\quad\Rightarrow\quad
\det (A)=y^2-4y^2=-3y^2<0 \quad\text{in $\Omega$},
\]
also ist die Gleichung hyperbolisch.
\item
Aus den Koeffizienten der Symbolmatrix findet man die Gleichung
der Charakteristiken
\[
y(s)^2\dot y(s)^2-4y(s)\dot x(s)\dot y(s)+\dot x(s)^2=0.
\]
\item
Der Faktorisierungsansatz liefert
\begin{align*}
0&=(\dot x(s)-ry(s)\dot y(s))(\dot x(s)-qy(s)\dot y(s))
\\
&=\dot x(s)^2-(r+q)\dot x(s) y(s)\dot y(s)+rqy(s)^2\dot y(s)^2
\end{align*}
Die Zahlen $r$ und $q$ m"ussen also die Gleichungen
\begin{align*}
r+q&=4\\
rq&=1
\end{align*}
erf"ullen. Setzt man $q=1/r$ in die erste Gleichung ein, bekommt man
\begin{align*}
r+\frac1r&=4\\
r^2-4r+1&=0
\end{align*}
Diese quadratische Gleichung hat die L"osungen
\[
r
=
2\pm\sqrt{4-1}
=
\begin{cases}
\sqrt{3}+2\\
-\sqrt{3}+2
\end{cases}
\]
\item
Die gew"ohnliche Differentialgleichung ist
\[
\dot x(s)=ry(s)\dot y(s)=\frac{r}{2}\frac{d}{ds}(y(s)^2)
\]
Integriert man nach $s$ bekommt man
\[
x(s) + C=\frac{r}{2}y(s)^2.
\]
Wegen $y(s)>0$ gilt
\[
y = \sqrt{\frac{2}{r}(x+C)}.
\]
Die Charakteristiken erf"ullen also Gleichungen der Form
\[
\frac{r}2y^2-x-C=0,
\]
dies sind Parabeln.
\end{teilaufgaben}
Es wurde zwar nicht verlangt, die Charakteristiken durch einen
bestimmten Punkt $(x_0,y_0)$ zu berechnen, hier trotzdem die
L"osung. Die Charakteristik zu den Faktorn mit $r$ bzw. $q$ haben
die Konstanten $C_r$ und $C_q$ nach
\[
C_r=\frac{r}{2}y_0^2-x_0
\qquad
\text{bzw.}
\qquad
C_q=\frac{q}{2}y_0^2-x_0.
\]
Damit werden die Charakteristiken durch $(x_0,y_0)$:
\begin{align*}
y&=\sqrt{
\frac2r(x+C_r)
}
=
\sqrt{
\frac2r(x+
\frac{r}{2}y_0^2-x
)
}
=\sqrt{y_0^2+\frac2r(x-x_0)}
\qquad\text{bzw.}
\\
y&
=\sqrt{y_0^2+\frac2q(x-x_0)}
\end{align*}
\end{loesung}
