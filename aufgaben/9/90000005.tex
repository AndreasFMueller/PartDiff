Zeigen Sie, dass die partielle Differentialgleichung
\[
\partial_x\partial_y u=0
\]
auf dem Gebiet $\Omega=\mathbb R^2$ hyperbolisch ist.
Wie verlaufen die Charakteristiken?

\begin{loesung}
Die Koeffizientematrix ist
\[
A=\begin{pmatrix}0&\frac12\\\frac12&0\end{pmatrix},
\]
$A$ hat das charakteristische Polynom
$\lambda^2-\frac14=(\lambda-\frac12)(\lambda+\frac12)$ mit den Nullstellen
$\pm\frac12$. Dies zeigt, dass die Gleichung hyperbolisch ist.

Die Differentialgleichung der Charakteristiken ist
\[
\dot x\dot y=0,
\]
also $\dot x=0$ oder $\dot y=0$. Somit sind die charakteristiken Kurven
Geraden parallel zu den Koordinatenachsen.

Alternativ k"onnen die Charakteristiken wie folgt gefunden werden.
Die Richtungen der Charakteristiken sind diejenigen Vektoren $v$, die
den Ausdruck $v^tAv$ zu Null machen. Schreibt man
\[
v=\begin{pmatrix}v_1\\v_2\end{pmatrix},
\]
m"ussen die Komponenten von $v$ die Gleichung
\[
v^tAv=v_1v_2=0
\]
erf"ullen, d.~h.~$v_1=0$ oder $v_2=0$. Die Charaketeristiken sind daher
Geraden parallel zu den Koordinaten-Achsen.
\end{loesung}


