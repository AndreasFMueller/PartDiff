Finden Sie alle L"osungen der Wellengleichung
\[
\partial_t^2u-a^2\partial_x^2u=0
\]
in der Halbebene $x>0$ mit Anfangswerten
\[
u(0,x)=0,\quad x>0
\]
und Randwerten
\[
u(t,0)=\sin t,\quad t\in\mathbb R.
\]

\begin{hinweis}
Es gibt unendlich viele L"osungen.
\end{hinweis}

\begin{loesung}
In der Vorlesung wurde die d'Alembert L"osung der Wellengleichung
gezeigt. Die allgemeine L"osung der Wellengleichung
\[
u(t,x)
=
u_+(x+at)+u_-(x-at),
\]
die Funktionen $u_+$ und $u_-$ sind noch zu bestimmen. Es ist klar,
dass diese Funktionen links und rechts von $0$ v"ollig verschieden sein
k"onnen.

Die Funktion $u_+(x+at)$ beschreibt eine nach links laufende
Welle. Der Teil der Funktion f"ur negative Argumente kann
die L"osung im Bereich $x>0$ nie
beeinflussen, wir k"onnen also $u_+(x)=0$ setzen f"ur $x<0$.

Die Anfangsbedingung f"ur $t=0$ sagt uns, dass
\[
u_+(x)+u_-(x)=0
\]
sein muss f"ur $x>0$. Wir k"onnen also zum Beispiel
\[
u_+(x)=\begin{cases}u_0(x)&\qquad x\ge 0\\
0&\qquad x<0
\end{cases}
\]
setzen, wobei $u_0$ eine beliebige Funktion ist. Die Funktion $u_-$ muss
dann passend dazu wie folgt definiert werden:
\[
u_-(x)=\begin{cases}
-u_0(x)&\qquad x\ge 0\\
?&\qquad x<0
\end{cases}
\]
Der noch fehlende Teil der Funktion $u_-$ muss jetzt so definiert werden,
dass auch die zweite Randbedingung erf"ullt ist, diese lautet
\[
u_+(at)+u_-(-at)=\sin t.
\]
Schreiben wir $at=y$, oder $t=\frac{y}{a}$, dann liefert diese Bedingung
f"ur positive $t$ genau den fehlenden Teil der Definition von $u_-$:
\[
u_-(-y)=-u_+(y)+\sin \frac{y}{a}.
\]
oder
\[
u_-(y)=\begin{cases}
-u_0(y)&\qquad y\ge 0\\
-u_0(-y)-\sin{\displaystyle\frac{y}{a}}&\qquad y<0
\end{cases}
\]
\end{loesung}
