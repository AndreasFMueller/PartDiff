%
% domain.tex -- Template für standalone TIKZ Bilder
%
% (c) 2019 Prof Dr Andreas Müller, Hochschule Rapperswil
%
\documentclass[tikz,12pt]{standalone}
\usepackage{amsmath}
\usepackage{times}
\usepackage{txfonts}
\usepackage{pgfplots}
\usepackage{csvsimple}
\usetikzlibrary{arrows,intersections,math,calc}
\begin{document}
\newboolean{fein}
\setboolean{fein}{true}
\definecolor{darkgreen}{rgb}{0,0.6,0}
\def\skala{2.5}
\begin{tikzpicture}[>=latex,thick,scale=\skala]

\fill[color=blue!10] (-3.1,0.5) rectangle (3.1,3.1);

\pgfmathparse{ln((exp(1)-exp(-1))/2)}
\xdef\Ccrit{\pgfmathresult}
\pgfmathparse{2*(\Ccrit-ln((exp(0.5)-exp(-0.5))/2))}
\xdef\xmax{\pgfmathresult}

\begin{scope}
\clip (-3.1,0.5) rectangle (3.1,3.1);

\foreach \x in {-3,-2,-1,1,2,3}{
	\draw[color=red] (\x,0) -- (\x,3.1);
}

\foreach \C in {-4,-3.5,...,3.0}{
	\draw[color=darkgreen]
		plot[domain=-5.1:3.1,samples=100]
			({\x-2*\C},{ln(exp(-0.5*\x)+sqrt(exp(-\x)+1))});
}

\ifthenelse{\boolean{fein}}{

\foreach \x in {-3,-2.8,...,3}{
	\draw[color=red,line width=0.1pt] (\x,0) -- (\x,3.1);
}

\foreach \C in {-3.9,-3.8,...,3}{
	\draw[color=darkgreen,line width=0.1pt]
		plot[domain=-5.1:3.1,samples=100]
			({\x-2*\C},{ln(exp(-0.5*\x)+sqrt(exp(-\x)+1))});
}

}{}

%\xdef\Ccrit{0.2}
\fill[color=orange!20]
	plot[domain=0:3.2,samples=100]
		({\x},{ln(exp(-0.5*\x+\Ccrit)+sqrt(exp(-\x+2*\Ccrit)+1))})
		--
		(3,0) -- (0,0) -- cycle;
	
\draw[color=darkgreen]
	plot[domain=0:3.2,samples=100]
		({\x},{ln(exp(-0.5*\x+\Ccrit)+sqrt(exp(-\x+2*\Ccrit)+1))});

\end{scope}

%\def\xmax{1.7}

\draw[color=blue,line width=1.4pt] (-3.1,0.5) -- (0,0.5);
\draw[color=orange,line width=1.4pt] (0,0.5) -- (\xmax,0.5);
\draw[color=blue,line width=1.4pt] (\xmax,0.5) -- (3.1,0.5);

\draw[->] (0,-0.1) -- (0,3.3) coordinate[label={left:$y$}];
\draw[color=red,line width=1pt] (0,0.5) -- (0,3.1);
\draw[->] (-3.1,0) -- (3.3,0) coordinate[label={$x$}];

\fill[color=blue] (0,1) circle[radius=0.03];
\node[color=blue] at (0,1) [above right] {$P$};

\foreach \x in {-3,-2,-1,1,2,3}{
	\draw (\x,{-0.1/\skala}) -- (\x,{0.1/\skala});
	\node at (\x,{-0.1/\skala}) [below]  {$\x$};
}
\node at (0,0) [below left] {$O$};

\foreach \y in {2,3}{
	\draw ({-0.1/\skala},\y) -- ({0.1/\skala},\y);
}
\foreach \y in {1,2,3}{
	\node at ({-0.1/\skala},\y) [left] {$\y$};
}

\node at (-1.5,1.75) {$\Omega$};

\end{tikzpicture}
\end{document}

