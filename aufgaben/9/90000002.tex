Betrachten Sie die Differentialgleichung
\[
-(1+x)^2\frac{\partial^2u}{\partial x^2}
+(1-x^2)\frac{\partial^2u}{\partial x\partial y}
+(1-x)^2\frac{\partial^2u}{\partial y^2}=0
\]
im Gebiet
\[
\Omega=\{(x,y)\,|\, -1\le x \le 1\}.
\]
\begin{teilaufgaben}
\item Wie lautet das Symbol der Differentialgleichung? Zeigen Sie, dass sie
hyperbolisch ist.
\item Stellen Sie die Differentialgleichung der Charakteristiken auf.
\item L"osen sie die Differentialgleichung nach $\frac{dy}{dx}$ auf.
\item Betrachten Sie zwei verschiedene Charakteristiken, die sich
auf der $y$-Achse schneiden. Wie gross ist der Schnittwinkel?
\end{teilaufgaben}

\begin{hinweis}
Es wird nicht verlangt, die Differentialgleichung
der Charakeristiken zu l"osen.
\end{hinweis}

\begin{loesung}
\begin{teilaufgaben}
\item Die Symbolmatrix ist
\[
\begin{pmatrix}
-(1+x)^2&\frac12(1-x^2)\\
\frac12(1-x^2)&(1-x)^2
\end{pmatrix}
\]
mit Determinante
\begin{align*}
\left|\,\begin{matrix}
-(1+x)^2&\frac12(1-x^2)\\
\frac12(1-x^2)&(1-x)^2
\end{matrix}\,\right|&=
-(1+x)^2(1-x)^2-\frac14(1-x^2)^2
\\
&=-(1-x^2)^2-\frac14(1-x^2)
\\
&=-\frac54(1-x^2)^2<0
\end{align*}
im Inneren des Gebietes. Also ist die Differentialgleichung hyperbolisch.
\item Die Differentialgleichung der Charakteristiken ist
\[
-(1+x)^2\dot y^2-(1-x^2)\dot y\dot x+(1-x)^2\dot x^2=0
\]
Teilt man durch $\dot x^2$, erh"alt man die Gleichung
\begin{align*}
-(1+x)^2\left(\frac{\dot y}{\dot x}\right)^2
-(1-x^2)\frac{\dot y}{\dot x}
+(1-x)^2&=0\\
-(1+x)^2\left(\frac{dy}{dx}\right)^2
-(1-x^2)\frac{dy}{dx}
+(1-x)^2&=0\\
\end{align*}
\item Die Aufl"osung nach $\frac{dy}{dx}$ erfolgt mit der L"osungsformel
f"ur die quadratische Gleichung:
\begin{align*}
\frac{dy}{dx}
&=
\frac{(1-x^2)\pm\sqrt{(1-x^2)^2+4(1+x)^2(1-x)^2}}{-2(1+x)^2}
\\
&=
\frac{(1-x^2)\pm\sqrt{(1-x^2)^2+4(1-x^2)^2}}{-2(1+x)^2}
\\
&=
\frac{(1-x^2)\pm(1-x^2)\sqrt{5}}{-2(1+x)^2}
\\
&=
\frac{(1-x^2)(1\pm\sqrt{5})}{-2(1+x)^2}
\\
&=
\-\frac{1-x}{1+x}\cdot\frac{1\pm\sqrt{5}}{-2}
\end{align*}
\item
Die beiden Charakteristiken sind die L"osungen der Differentialgleichung
f"ur die verschiedenen Vorzeichen $\pm$. $\frac{dy}{dx}$ ist die
Steigung der Charakteristiken, f"ur $x=0$ ergibt sich
\[
\frac{dy}{dx}
=
-\frac{1\pm\sqrt{5}}{2}
\]
Aus diesen beiden Steigungen
k"onnen die Winkel zur $x$-Achse und damit der Schnittwinkel
berechnet werden. Allerdings ist das Produkt der beiden Steigungen
\[
\left(-\frac{1+\sqrt{5}}{2}\right)
\left(-\frac{1-\sqrt{5}}{2}\right)
=\frac{(1+\sqrt{5})(1-\sqrt{5})}{4}=
\frac{1-\sqrt{5}^2}{4}=\frac{1-5}{4}
%=\frac{-4}{4}
=-1.
\]
Die Steigungen sind also reziprok und von entgegengesetztem Vorzeichen,
woraus man schliessen kann, wie man in Analysis I lernt, dass die
beiden Kurven sich unter einem rechten Winkel schneiden.
\qedhere
\end{teilaufgaben}
\end{loesung}
