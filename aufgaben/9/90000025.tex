Consider the differential operator
\[
L
=
-2\coth y\frac{\partial^2}{\partial x\,\partial y}
+
\frac{\partial^2}{\partial y^2}
\]
in the domain
$\Omega = \{ (x,y)\mid y > \frac12\}$.
\begin{teilaufgaben}
\item
Is the operator $L$ elliptic, parabolic or hyperbolic?
\item
If $L$ is hyperbolic, can the boundary values on the part
$\{(x,\frac12)\mid x < 0\}$ of the boundary influence the value
of a solution in the point $(0,1)$.
\end{teilaufgaben}

\begin{hinweis}
The following formulae about hyperbolic functions may be useful:
\begin{align*}
\tanh x &= \frac{\sinh x}{\cosh x}
&
\int\coth x\,dx &= \log \sinh x + C
%\int\frac{dx}{\sinh x} &=
%\log\frac{e^x-1}{e^x+1}
\\
\coth x &= \frac{\cosh x}{\sinh x}
&&
\end{align*}
\end{hinweis}

\ifthenelse{\boolean{loesungen}}{
\begin{figure}[ht]
\centering
\includeagraphics[]{domain.pdf}
\caption{Domain $\Omega$ for problem~\ref{90000025} with characteristics
\label{90000025:graph}}
\end{figure}
}{}

\begin{loesung}
\begin{teilaufgaben}
\item
The symbol matrix of the operator $L$ is
\[
A=\begin{pmatrix}
    0    & -\coth y\\
-\coth y &    1
\end{pmatrix}
\]
and has determinant
\[
\det A
=
-\coth^2 y
< 0,
\]
which means that the operator is hyperbolic.
\item
The equation for the characteristics is
\begin{align*}
 2\dot{x}\dot{y} \coth y + \dot{x}^2 &= 0,
\intertext{which can be factored as}
\dot{x}\bigl( \dot{y} \coth y + {\textstyle \frac12}\dot{x}\bigr) &=0.
\end{align*}
The factorization implies that either
\begin{align*}
\dot{x} &= 0
&&\text{or}
&
y'(x) = -\frac12 \tanh y.
\end{align*}
The first equation means that the corresponding characteristics
are straight lines with constant $x$, they are drawn in red in
figure~\ref{90000025:graph}.
The second equation can be solved using separation of variables,
leading to
\begin{align*}
\int \frac{dy}{\tanh y} &= -\frac12x + C
\\
\int\coth y\,dy &= -\frac12 x + C
\\
\log\sinh y &= -\frac12x + C
\\
\sinh y&= e^{-\frac12x+C}
\\
y(x)
&=
\operatorname{arsinh}\bigl( e^{-\frac12x+C}\bigr).
\end{align*}
This means that the second set of characteristics, drawn in
green in figure~\ref{90000025:graph}, are translates
of the graph of the function
\[
y_0(x)
=
\operatorname{arsinh} e^{-\frac12x}.
\]
Since the exponential function is monotonically decreasing and the
$\sinh$-function is monotonically increasing, the function $y_0(x)$
is monotonically decreasing.

The characteristics are displayed in figure~\ref{90000025:graph},
which clearly shows that only the points in the orange interval
on the boundary of $\Omega$ can influence the value of a solution
in the point $(0,1)$.
\qedhere
\end{teilaufgaben}
\end{loesung}

\begin{diskussion}
To compute the full intervall of points on the boundary that can influence
the value in the point $(0,1)$, we have to find the value of the constant
$C_+$ such that the corresponding characteristic curve passes throught $(0,1)$.
This means that 
\[
1 = y(0) = \operatorname{arsinh}e^{C_+}
\qquad\Rightarrow\qquad
\sinh 1 = e^{C_+}
\qquad\Rightarrow\qquad
C_+ = \log\sinh 1.
\]
Furthermore, the characteristic curve passes through $y=\frac12$ for an
$x$-coordinate $x_+$ that satisfies the equation
\[
\operatorname{arsinh}e^{-\frac12x_++C_+}=\frac12
\qquad\Rightarrow\qquad
\log\sinh\frac12 = -\frac12x_++C_+
\qquad\Rightarrow\qquad
x_+
=2\biggl(C-\log\sinh\frac12\biggr).
\]
\end{diskussion}

\begin{bewertung}
Symbol matrix ({\bf S}) 1 point,
hyperbolicity ({\bf H}) 1 point,
differential equation for characteristics ({\bf D}) 1 point,
factorization and vertical characteristics ({\bf V}) 1 point,
other characteristics ({\bf R}) 1 point,
conclusion about domain of influence for the point ({\bf I}) 1 point.
\end{bewertung}
