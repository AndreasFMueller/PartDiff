Consider the differential operator 
\begin{equation}
L
=
\sinh x\,\frac{\partial^2}{\partial x^2}
+
2\cosh x\,\frac{\partial^2}{\partial x\,\partial y}
+
\sinh x\,\frac{\partial^2}{\partial y^2}
\label{90000024:eqn}
\end{equation}
in the domain $\Omega=\{(x,y)\mid x\ge \frac12\}$.
\begin{teilaufgaben}
\item
Is $L$ elliptic, parabolic or hyperbolic?
\item
Can changes in boundary values on the part $\{(\frac12,y)\mid y>0\}$ 
of the boundary $\partial\Omega$ of $\Omega$ influence the value
of a solution of the equation $Lu=0$ with Dirichlet and Neumann
boundary conditions in the point $P=(1,0)$?
\end{teilaufgaben}

\begin{hinweis}
The following formulae about hyperbolic functions may be useful:
\begin{align*}
\tanh x &= \frac{\sinh x}{\cosh x}
&
\int\frac{dx}{\sinh x} &=
\log\frac{e^x-1}{e^x+1}
\\
\coth x &= \frac{\cosh x}{\sinh x}
&\int\coth x\,dx &= \log \sinh x
\end{align*}
\end{hinweis}

\ifthenelse{\boolean{loesungen}}{
\begin{figure}[ht]
\centering
\includeagraphics[]{domain.pdf}
\caption{Domain and characterstics for the differential operator
$L$ defined as~\eqref{90000024:eqn} in problem~\ref{90000024}.
The points in the orange curvilinear triangle can influence the value
at $P=(1,0)$.
\label{90000024:char}}
\end{figure}
}{}

\begin{loesung}
\begin{teilaufgaben}
\item
The symbol matrix of the operator $L$ is
\[
A=\begin{pmatrix}
\sinh x & \cosh x\\
\cosh x & \sinh x
\end{pmatrix}
\]
has determinant
\[
\det A
=
\sinh^2x-\cosh^2x
=
-1<0,
\]
which means that the operator is hyperbolic.
\item
The differential equation for the characteristiscs is
\[
\sinh x
\cdot
\dot{y}^2
-2
\cosh x
\cdot
\dot{x}\dot{y}
+
\sinh x
\cdot
\dot{x}^2
=
0
\]
Dividing by $\dot{x}^2$ and $\sinh x$ gives
\[
(y'(x))^2
-2
\coth x\cdot y'(x)
+
1
=
0.
\]
This is a quadratic equation has the solutions
\begin{align*}
y'_\pm(x)
&=
\coth x \pm \sqrt{\coth^2 x -1}
=
\frac{\cosh x}{\sinh x}
\pm
\sqrt{
\frac{\cosh^2 x-\sinh^2}{\sinh^2x}
}
=
\frac{\cosh x}{\sinh x}
\pm
\sqrt{\frac{1}{\sinh^2 x}}
\\
&=
\frac{
\cosh x \pm 1
}{
\sinh x
}
=
\coth x \pm \frac{1}{\sinh x}.
\end{align*}
This can easily be integrated to give
\begin{align*}
y_\pm(x)
&=
\log\sinh x \pm \log\frac{e^x-1}{e^x+1} + C_{\pm}.
\end{align*}
A quick plot of $y_{\pm}(x)$ reveals that both families of characteristics
are graphs of monotonically increasing functions.
From this it follows that points on the part of the boundary in the upper
half plane cannot influence the value in points on the $x$-axis, and
a fortiori in the point $(1,0)$.
\qedhere
\end{teilaufgaben}
\end{loesung}

\begin{diskussion}
Using the equations for the characteristic curves it is possible
to determine the interval on the boundary where changes can affect
the value of the solution in $(1,0)$.
To do this, the integration constant for the two solutions needs to
be determined such that the characteristic passes through $(1,0)$.
They can be found by solving the equation
\[
0
=
\log\sinh 1 \pm \log\frac{e-1}{e+1}+C_{\pm}
\qquad\Rightarrow\qquad
C_{\pm}
=
-\log\sinh 1
\mp 
\log\frac{e-1}{e+1}
\]
By substituting
$x=\frac12$ into $y_{\pm}(x)$
with the right contant $C_\pm$ gives the interval an the line
$x=\frac12$ which can influence the point $(1,0)$.
\end{diskussion}

\begin{bewertung}
Symbol matrix ({\bf S}) 1 point,
hyperbolicity ({\bf H}) 1 point,
diffential equation for the characteristidcs ({\bf D}) 1 point,
factorization or solution for $y'(x)$ ({\bf F}) 1 point,
solution of differential equation ({\bf Y}) 1 point,
monotonicity or some other argument to explain the domain of influence
({\bf I}) 1 point.
\end{bewertung}
