On the domain $\Omega = \{(x,y)\;|\; x > 0\wedge y>-1\}$,
the partial differential equation
\begin{equation}
\frac{\partial^2u}{\partial x^2}
+
(y-1)\frac{\partial^2 u}{\partial x\,\partial y}
-
y\frac{\partial^2u}{\partial y^2}
=
0
\label{90000023:eqn}
\end{equation}
is given.
\begin{teilaufgaben}
\item
Is the equation \eqref{90000023:eqn} elliptic, parabolic or hyperbolic?
\item
Where do boundary values need to be specified in order for the solution
to be uniquely determined on the unit square
$I^2=I\times I=\{(x,y)\;|\;0<x<1\wedge 0<y<1\}$?
\end{teilaufgaben}

\begin{loesung}
\begin{teilaufgaben}
\item
The symbol matrix of the partial differential operator of 
\eqref{90000023:eqn} is
\[
A=\begin{pmatrix}
1&\frac12(y-1)\\
\frac12(y-1)&-y
\end{pmatrix}.
\]
The determinant is
\begin{align*}
\det A
&=
-y-\frac14(y-1)^2
\\
&=-\frac14(1-2y + y^2+4y)
\\
&=-\frac14(1+2y+y^2)
\\
&=-\frac14(1+y)^2.
\end{align*}
Since $y>-1$, the square on the right hand side will always be $>0$,
so the determinant will always be negative.
This implies that the partial differential operator is hyperbolic.
\item
To find the interval on the boundary that can influence values
in the unit square, we have to compute the characteristics.
The differential equation for the characteristics is
\begin{align*}
\dot{y}^2 -(y-1)\dot{y}\dot{x} -y\dot{x}^2 &= 0.
\intertext{We can turn this into an ordinary differential equation
for the function $y(x)$ by dividing by $\dot{x}$ and using
$y'=\dot{y}/\dot{x}$:}
y^{\prime 2}-(y-1)y' - y&=0.
\intertext{This equation can be factored into}
(y'-y)(y'+1)&=0.
\end{align*}
To solve the equation, we need to solve for each factor separately:
\begin{align*}
y'-y         &=0            &       y'+1 &= 0                      \\
y'           &=y            &       y'   &= -1                     \\
\frac{y'}{y} &=1            &       y    &= -x + D                 \\
\log y       &= x + c \\
           y &= Ce^x \\
\end{align*}
\begin{figure}
\centering
\def\skala{3}
\begin{tikzpicture}[>=latex,scale=\skala]
\definecolor{darkgreen}{rgb}{0,0.8,0}
\fill[color=gray!40] (0,0) rectangle (1,1);
\begin{scope}
\clip (0,-1) rectangle (3.1,2.2);
	\foreach \C in {0.05,0.1,0.2,...,2.2}{
		\draw[color=red,line width=1pt]
			plot[domain=0:3.1,samples=100] ({\x},{\C*exp(\x)});
		\draw[color=red,line width=1pt]
			plot[domain=0:3.1,samples=100] ({\x},{-\C*exp(\x)});
	}
	\foreach \D in {-0.8,-0.6,...,5.2}{
		\draw[color=darkgreen,line width=1pt]
			(0,\D) -- (3.2,{\D-3.2});
	}
	\draw[color=blue,line width=1pt] (0,2) -- (3,-1);
	\draw[color=blue,line width=1pt]
		plot[domain=0:1.8,samples=100] ({\x},{exp(\x-1)});
	\fill[color=blue] (1,1) circle[radius={0.07/\skala}];
\end{scope}
\draw[->] ({-0.1/\skala},0) -- (3.2,0) coordinate[label={$x$}];
\draw[->] (0,-1.1) -- (0,2.3) coordinate[label={left:$y$}];
\draw[color=red,line width=1pt] (0,0) -- (3.1,0);
\draw (0,-1) -- (3.2,-1);
\draw[color=blue,line width=2pt] (0,0) -- (0,2);
\draw[color=blue] ({-0.1/\skala},2) -- ({0.1/\skala},2);
\draw[color=blue] ({-0.1/\skala},0) -- ({0.1/\skala},0);
\node[color=blue] at (0,2) [left] {$2$};
\node at (0,1) [left] {$1$};
\node[color=blue] at (0,0) [left] {$0$};
\node at (1,0) [below] {$1$};
\draw (1,{-0.1/\skala}) -- (1,{0.1/\skala});

\end{tikzpicture}
\caption{Characteristics of the partial differential equation
\eqref{90000023:eqn}.
The points on the blue part of the boundary are needed to uniquely 
specify the values of the solution in the unit square.
\label{90000023:fig}}
\end{figure}%
From figure~\ref{90000023:fig} we can read off that only the boundary
values  for $y$ between $0$ and $2$ can influence the values in the
unit square.
\qedhere
\end{teilaufgaben}
\end{loesung}

\begin{bewertung}
\begin{teilaufgaben}
\item Symbol matrix ({\bf S}) 1 point,
hyperbolicity ({\bf H}) 1 point,
\item Differential equation for characteristics ({\bf D}) 1 point,
solution of two factor equations ({\bf F}) 2 points,
boundary interval that can influence the value in the corner of the
square ({\bf B}) 1 point.
\end{teilaufgaben}
\end{bewertung}

