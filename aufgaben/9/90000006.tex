Betrachten Sie die Differentialgleichung
\[
y\partial_x^2u+\partial_y^2u=0
\]
Wo ist sie elliptisch bzw.~hyperbolisch?

Berechnen Sie die charakteristischen Kurven der Differentialgleichung
im Gebiet $\Omega=\{(x,y)\in\mathbb R^2|y<0\}$.
Finden Sie die Charakteristiken, die durch den Punkt
$(1,-1)$ gehen.

\begin{loesung}
Die Differentialgleichung hat das Symbol
\[
A=\begin{pmatrix}y&0\\0&1\end{pmatrix}
\]
mit Eigenwerten $y$ und $1$, sie ist also im Gebiet $\{(x,y)\,|\,y>0\}$
elliptisch und in $\{(x,y)\,|\,y<0\}$ hyperbolisch.

Die Richtung $v$ der charakteristischen Kurven erf"ullt,
$v^tAv=0$, also $yv_1^2-v_2^2=0$. W"ahlt man eine
Parameterdarstellung $s\mapsto(x(s),y(s))$ f"ur die
charakteristischen Kurven, ist die Tangentialrichtung
$v=(x'(s),y'(s))$,
sie erf"ullen also
die Differentialgleichungen
\[
y(y')^2=-(x')^2
\]
oder
\begin{align*}
\frac{dy}{dx}&=\frac{y'}{x'}=\pm\frac1{\sqrt{-y}}.
\\
\int\sqrt{-y}\,dy&=\pm\int \,dx+C
\end{align*}
mit den L"osungen
\begin{align*}
-\frac23(-y)^{\frac32}&=\pm x+C
\\
y&=-\biggl(\frac32(\pm x+C)\biggr)^{\frac23}
\end{align*}
F"ur die L"osungskurven, die durch den Punkt $(1,-1)$ gehen,
muss die Konstante $C$ geeignet bestimmt werden:
\begin{align*}
-\frac231^{\frac32}&=(\pm1+C)\\
C&=\mp 1-\frac23
=\begin{cases}
-\frac53\\
\frac13
\end{cases}
\end{align*}
\begin{figure}
\begin{center}
\includeagraphics[width=\hsize]{7}
\end{center}
\caption{Charakteristiken durch $(1,-1)$ der PDGL $y\partial_x^2u+\partial_y^2u=0$
\label{90000006:char}}
\end{figure}
Die Graphen der Charaketeristiken sind in Abbildung \ref{90000006:char} dargestellt.
\end{loesung}
