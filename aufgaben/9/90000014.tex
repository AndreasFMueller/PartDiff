Auf dem Gebiet $\Omega=\{(x,y)\in\mathbb R^2\,|\, x>0\}$ ist die
Differentialgleichung
\begin{equation}
(\alpha + 1) \frac{\partial^2 u}{\partial x^2}
+
2\alpha\frac{\partial^2 u}{\partial x\partial y}
+
(\alpha - 1)\frac{\partial^2 u}{\partial y^2}
=
0
\label{90000014:dgl}
\end{equation}
gegeben.
Von dem Parameter $\alpha$ weiss man nur, dass $\alpha > -1$ ist.
Ausserdem sind die Randbedingungen
\[
u(0,y)=g(y)\qquad\text{und}\qquad \frac{\partial u}{\partial x}(0,y)=0
\]
gegeben, wobei $g(y)$ eine f"ur grosse Argumentwerte nur sehr aufwendig zu
berechnende Funktion ist.
\begin{teilaufgaben}
\item
Ist das Problem gut gestellt?
\item
Wegen der Schwierigkeiten bei der Berechnung der Funktion $g$
wurde eine numerische L"osung f"ur $u$ gerechnet, indem die
Randwerte f"ur $|y|>5$ null gesetzt wurden.
Diese L"osung wird nur in einem Teil von $\Omega$ mit der wahren L"osung
"ubereinstimmen.
F"ur welche Werte des Parameters $\alpha$ kann man davon ausgehen, dass die
numerische L"osung die wahre L"osung im Rechteck
$R=\{(x,y)\,|\, 0\le x\le 3,-2\le y\le 2\}$ korrekt
wiedergibt?
\end{teilaufgaben}


\begin{loesung}
\begin{figure}
\centering
\includeagraphics[]{domain-1.pdf}
\caption{Charakteristiken der Differentialgleichung~(\ref{90000014:dgl})
und Rechteck $R$ (blau). 
Die Steigung der roten Charakteristiken ist unabh"angig von $\alpha$.
Die Steigung der gr"unen Charakteristiken h"angt nach~\eqref{90000014:steigung}
von $\alpha$ ab.
F"ur die Frage in Teilaufgabe b) der Aufgabe sind die fett ausgezogenen
Charakteristiken durch die Ecken des Rechtecks massgeben.
\label{90000014:domain}}
\end{figure}
\begin{teilaufgaben}
\item
Die Symbolmatrix des Differentialoperators ist
\[
A=\begin{pmatrix}\alpha+1&\alpha \\ \alpha&\alpha - 1\end{pmatrix}
\qquad\Rightarrow\qquad
\det A=(\alpha + 1)(\alpha - 1) - \alpha^2=-1<0,
\]
die Differentialgleichung ist also f"ur alle in Frage kommenden
Werte von $\alpha$ hyperbolisch.
Das Problem ist gut gestellt, wenn die von der $y$-Achse ausgehenden
Charakteristiken das ganze Gebiet "uberdecken. 
Dazu muss die Differentialgleichung der Charakteristiken gel"ost werden,
sie lautet
\[
\dot y^2
(\alpha + 1)
-2
\dot x\dot y
\alpha
+
\dot x^2
(\alpha - 1)
=0.
\]
Dividieren wir durch $\dot x^2$, erhalten wir
\begin{equation}
(\alpha + 1)
\biggl(\frac{\dot y}{\dot x}\biggr)^2
-2\alpha \frac{\dot y}{\dot x} +(\alpha - 1)=0
\label{90000014:qgl}
\end{equation}
Der Quotient $\dot y/\dot x$ ist die Steigung $y'$ der Charakteristiken.
Die quadratische Gleichung~(\ref{90000014:qgl}) f"ur die Steigung $y'$
hat die Nullstellen
\begin{equation}
\frac{\dot y}{\dot x}=y'=
\frac{2\alpha \pm \sqrt{4\alpha^2 - 4(\alpha+1)(\alpha - 1)}}{2(\alpha +1)}
=
\frac{2\alpha \pm \sqrt{4\alpha^2 - 4(\alpha^2 - 1)}}{2(\alpha + 1)}
=
\frac{\alpha \pm 1}{\alpha + 1}
=
\begin{cases}
1&\\
\displaystyle\frac{\alpha - 1}{\alpha + 1}
\end{cases}
\label{90000014:steigung}
\end{equation}
Die Charakteristiken sind also Geraden mit Steigung $1$ oder
$(\alpha - 1)/(\alpha + 1)$.
F"ur $\alpha < 1$ wird die Steigung negativ.
F"ur Werte von $\alpha$ nahe genug bei $-1$ wird die Steigung beliebig
negativ.
In jedem Fall l"asst sich ganz $\Omega$ mit Charakteristiken "uberdecken.

\item
Aus der Graphik~\ref{90000014:domain} kann man ablesen,
dass die numerische L"osung genau dann
die L"osung im Rechteck $R$ korrekt wiedergibt, wenn von den Punkten
$(0,\pm 5)$ ausgehenden Charakteristiken das Rechteck h"ochstens noch 
streifen.
F"ur die von $(0,-5)$ ausgehende Charakteristik mit Steigung $1$ ist
dies auf jeden Fall erf"ullt.
Die vom Punkt $(0,5)$ ausgehende Charakteristik mit muss aus
Symmetriegr"unden mindestens die Steigung $-1$ haben.
Das zugeh"orige $\alpha$ ist
\[
\frac{\alpha - 1}{\alpha + 1}=-1
\qquad\Rightarrow\qquad
\alpha - 1 = -\alpha - 1
\qquad\Rightarrow\qquad
2\alpha=0
\qquad\Rightarrow\qquad
\alpha =0.
\]
\qedhere
\end{teilaufgaben}
\end{loesung}

\begin{bewertung}
Symbolmatrix ({\bf S}) 1 Punkt,
hyperbolisch ({\bf H}) 1 Punkt,
Differentialgleichungen der Charakteristiken ({\bf D}) 1 Punkt,
Steigung der Charakteristiken ({\bf M}) 1 Punkt,
Problem gut gestellt, weil Rand keine Charakteristik ({\bf G}) 1 Punkt,
maximale Steigung erreicht von $(0,-5)$ aus die Ecke $(1,-3)$ ({\bf A})
1 Punkt.
\end{bewertung}
