%
% char.tex -- Charakteristiken einer hyperbolischen DGL
%
% (c) 2018 Prof Dr Andreas Müller, Hochschule Rapperswi
%
\documentclass[tikz]{standalone}
\usepackage{amsmath}
\usepackage{times}
\usepackage{txfonts}
\usepackage[utf8]{inputenc}
\usepackage{graphics}
\usepackage{ifthen}
\usepackage{color}
\usetikzlibrary{arrows,intersections}
\begin{document}
\definecolor{darkgreen}{rgb}{0,0.6,0}

\begin{tikzpicture}[>=latex,thick,scale=6]
\fill[color=gray!20] (0,0) rectangle (1,1);
\draw[line width=0.1pt] (0,0)--(1,0)--(1,1)--(0,1)--cycle;
\draw[->] (-0.01,0) -- (1.1,0) coordinate[label={$x$}];
\draw[->] (0,-0.01) -- (0,1.1) coordinate[label={right:$y$}];
\node at (0.5,0.5) {$\Omega$};
\node at (0,0) [below left] {$0$};
\draw (1,-0.01)--(1,0.01);
\node at (1,0) [below] {$1$};
\draw (-0.01,1)--(0.01,1);
\node at (0,1) [left] {$1$};
\begin{scope}
\clip (-0.05,-0.05) rectangle (1.05,1.05);
\foreach \yoffset in {-3,-2.8,...,1}{
	\begin{scope}[yshift={\yoffset*28.5}]
		\def\rot{\draw[color=red] (0,0)
--(0.0100,0.0241)
--(0.0200,0.0481)
--(0.0300,0.0721)
--(0.0400,0.0959)
--(0.0500,0.1197)
--(0.0600,0.1434)
--(0.0700,0.1670)
--(0.0800,0.1906)
--(0.0900,0.2140)
--(0.1000,0.2374)
--(0.1100,0.2608)
--(0.1200,0.2841)
--(0.1300,0.3073)
--(0.1400,0.3304)
--(0.1500,0.3535)
--(0.1600,0.3765)
--(0.1700,0.3995)
--(0.1800,0.4223)
--(0.1900,0.4452)
--(0.2000,0.4680)
--(0.2100,0.4907)
--(0.2200,0.5134)
--(0.2300,0.5360)
--(0.2400,0.5585)
--(0.2500,0.5811)
--(0.2600,0.6035)
--(0.2700,0.6259)
--(0.2800,0.6483)
--(0.2900,0.6706)
--(0.3000,0.6929)
--(0.3100,0.7151)
--(0.3200,0.7373)
--(0.3300,0.7594)
--(0.3400,0.7815)
--(0.3500,0.8036)
--(0.3600,0.8256)
--(0.3700,0.8476)
--(0.3800,0.8695)
--(0.3900,0.8914)
--(0.4000,0.9132)
--(0.4100,0.9351)
--(0.4200,0.9568)
--(0.4300,0.9786)
--(0.4400,1.0003)
--(0.4500,1.0220)
--(0.4600,1.0436)
--(0.4700,1.0652)
--(0.4800,1.0868)
--(0.4900,1.1084)
--(0.5000,1.1299)
--(0.5100,1.1514)
--(0.5200,1.1728)
--(0.5300,1.1943)
--(0.5400,1.2157)
--(0.5500,1.2370)
--(0.5600,1.2584)
--(0.5700,1.2797)
--(0.5800,1.3010)
--(0.5900,1.3223)
--(0.6000,1.3435)
--(0.6100,1.3647)
--(0.6200,1.3859)
--(0.6300,1.4071)
--(0.6400,1.4282)
--(0.6500,1.4493)
--(0.6600,1.4704)
--(0.6700,1.4915)
--(0.6800,1.5126)
--(0.6900,1.5336)
--(0.7000,1.5546)
--(0.7100,1.5756)
--(0.7200,1.5966)
--(0.7300,1.6176)
--(0.7400,1.6385)
--(0.7500,1.6594)
--(0.7600,1.6803)
--(0.7700,1.7012)
--(0.7800,1.7221)
--(0.7900,1.7429)
--(0.8000,1.7637)
--(0.8100,1.7846)
--(0.8200,1.8054)
--(0.8300,1.8261)
--(0.8400,1.8469)
--(0.8500,1.8677)
--(0.8600,1.8884)
--(0.8700,1.9091)
--(0.8800,1.9298)
--(0.8900,1.9505)
--(0.9000,1.9712)
--(0.9100,1.9919)
--(0.9200,2.0125)
--(0.9300,2.0332)
--(0.9400,2.0538)
--(0.9500,2.0744)
--(0.9600,2.0950)
--(0.9700,2.1156)
--(0.9800,2.1362)
--(0.9900,2.1568)
--(1.0000,2.1773)
--(1.0100,2.1979)
--(1.0200,2.2184)
--(1.0300,2.2389)
--(1.0400,2.2594)
--(1.0500,2.2799)
;}

		%
% green.tex
%
% (c) 2019 Prof Dr Andreas Müller, Hochschule Rapperswil
%
\documentclass[tikz,12pt]{standalone}
\usepackage{amsmath}
\usepackage{times}
\usepackage{txfonts}
\usepackage{pgfplots}
\usepackage{csvsimple}
\usetikzlibrary{arrows,intersections,math}
\begin{document}
\begin{tikzpicture}[>=latex]

\def\s{12}

\foreach \xi in {0.025,0.075,...,1}{
	\draw[color=red!20,line width=1pt] (0,0)--({\s*\xi},{\s*\xi*(\xi-1)});
	\draw[color=red!20,line width=1pt] ({\s*\xi},{\s*\xi*(\xi-1)})--(\s,0);
	\fill[color=red!20] ({\s*\xi},{\s*\xi*(\xi-1)}) circle[radius=0.05];
}

\draw[->,line width=0.7pt] (-0.1,0)--({\s+0.6},0) coordinate[label=$x$];
\draw[->,line width=0.7pt] (0,{-0.25*\s-0.1})--(0,0.6)
	coordinate[label={$G(x,\xi)$}];

\draw[line width=0.7pt] ({\s},-0.1)--({\s},0.1);
\node at ({\s},0) [above] {$1$};
\node at (0,0) [below left] {$0$};

\def\xiv{0.325}

\draw[line width=0.5pt] ({\s*\xiv},0)--({\s*\xiv},{\s*\xiv*(\xiv-1)});
\draw[line width=0.7pt] ({\s*\xiv},-0.1)--({\s*\xiv},0.1);
\node at ({\s*\xiv},0) [above] {$\xi$};

\draw[color=red,line width=1pt] (0,0)--({\s*\xiv},{\s*\xiv*(\xiv-1)});
\draw[color=red,line width=1pt] ({\s*\xiv},{\s*\xiv*(\xiv-1)})--(\s,0);
\fill[color=red] ({\s*\xiv},{\s*\xiv*(\xiv-1)}) circle[radius=0.05];

\end{tikzpicture}
\end{document}


	\end{scope}
}
\end{scope}
\end{tikzpicture}

\end{document}
