Die Geschwindigkeit von Flachwasserwellen, also Wellen, deren Wellenl"ange
wesentlich gr"osser als die Wassertiefe $d$ ist, h"angt praktisch
nur von der Wassertiefe ab.  Die Geschwindigkeit ist $a=\sqrt{gd}$, wobei
$g$ die Erdbeschleunigung ist. Stellen Sie die Wellengleichung auf
f"ur ein eindimensionales Gebiet $[-1,1]$, mit parabolischem Profil
und maximaler Tiefe $d_0$ bei $x=0$. Berechnen Sie die Charakteristiken
durch den Punkt $(0,0)$.

\ifthenelse{\boolean{loesungen}}{
\begin{loesung}
Die Tiefe in diesem Gebiet ist von der Form $d_0(1-x^2)$, also ist
die Geschwindigkeit $a^2=gd_0(1-x^2)$ und die Wellengleichung
\[
gd_0(1-x^2)\frac{\partial^2 u}{\partial x^2}-\frac{\partial^2u}{\partial y^2}=0
\]
Die Differentialgleichung der Charakteristiken ist
\begin{align*}
gd_0(1-x^2)\dot y^2-\dot x^2&=0
\\
\dot y&=
\frac{\pm\dot x}{\sqrt{gd_0(1-x^2)}}
\\
\frac{dy}{dx}
&=
\pm\frac1{\sqrt{gd_0}}\frac1{\sqrt{1-x^2}}
\end{align*}
Diese Gleichung kann durch einfaches Integrieren gel"ost werden,
dazu muss man verwenden, dass
\[
\int \frac1{\sqrt{1-x^2}}\,dx = \arcsin x.
\]
Es folgt
\[
y(x)
=
\pm\frac1{\sqrt{gd_0}}\arcsin x + C
\]
Die Bedingung, dass die Charakteristiken durch den Punkt $(0,0)$
gehen sollen, f"uhrt auf $C=0$, also
\[
y(x)=\pm\frac1{\sqrt{gd_0}}\arcsin x.
\]
\end{loesung}
}{}

