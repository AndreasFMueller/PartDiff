On the domain $\Omega= \{(x,y)\mid x > 1\}$, the differential operator
\begin{equation}
L
=
\frac{\partial^2}{\partial x^2}
+
2x(1+x)\frac{\partial^2}{\partial x\,\partial y}
+
4x^3
\frac{\partial^2}{\partial y^2}
\label{90000026:operator}
\end{equation}
is given.
\begin{teilaufgaben}
\item
Is this operator elliptic, parabolic or hyperbolic?
\item
Can a change of a Dirichlet boundary condition in the point $Q=(1,0)$
influence the solution of a partial differential equation involving the
operator $L$ in the point $P=(\frac32,1)$?
\end{teilaufgaben}

\begin{loesung}
\begin{figure}
\centering
\begin{tikzpicture}[>=latex,thick]
\def\dx{5}
\def\dy{5}
\def\X{1.5}
\def\Y{1}
\pgfmathparse{\Y-2*\X*\X*\X/3}
\xdef\D{\pgfmathresult}
\pgfmathparse{\Y-\X*\X}
\xdef\C{\pgfmathresult}
\begin{scope}
	\fill[color=gray!10] ({\dx},{-0.7*\dy}) rectangle ({2.2*\dx},{1.2*\dy});
	\clip ({1*\dx},{-0.7*\dy}) rectangle ({2.2*\dx},{1.2*\dy});
	\fill[color=blue!20]
		plot[domain=1:\X,samples=10] 
			({\x*\dx},{(2*\x*\x*\x/3+(\D))*\dy})
		--
		plot[domain=\X:1,samples=10]
			({\x*\dx},{(\x*\x+(\C))*\dy})
		--
		cycle;
	\draw[color=darkgreen,line width=0.3pt] plot[domain=1:2.2,samples=20]
			({\x*\dx},{(\x*\x+\C)*\dy});
	\draw[color=red,line width=0.3pt] plot[domain=1:2.2,samples=20]
			({\x*\dx},{(2*\x*\x*\x/3+\D)*\dy});
	\foreach \C in {-10,-9.5,...,2}{
		\draw[color=darkgreen] plot[domain=1:2.2,samples=20]
			({\x*\dx},{(\x*\x+\C)*\dy});
	}
	\foreach \D in {-10,-9.5,...,2}{
		\draw[color=red] plot[domain=1:2.2,samples=20]
			({\x*\dx},{(2*\x*\x*\x/3+\D)*\dy});
	}
\end{scope}
\fill[color=white,opacity=1.0]
	({\X*\dx+0.05},{\Y*\dy-0.1}) rectangle ({\X*\dx+1.90},{\Y*\dy-0.70});
\draw[line width=0.2pt] (0,{\Y*\dy}) -- ({\X*\dx},{\Y*\dy});
\draw (-0.05,{\Y*\dy}) -- (0.05,{\Y*\dy});
\node at (0,{\Y*\dy}) [left] {$\Y$};
\draw[line width=0.2pt] ({\X*\dx},0) -- ({\X*\dx},{1.2*\dy});
\node at ({\X*\dx},{1.2*\dy}) [above] {$\frac32$};
\node at ({\X*\dx},{\Y*\dy}) [below right] {$P=(1,\frac32)$};
\draw[color=blue] ({1*\dx},{-0.7*\dy}) -- ({1*\dx},{1.2*\dy});
\fill[color=blue] ({\X*\dx},{\Y*\dy}) circle[radius=0.08];
\fill[color=blue] ({1*\dx},{0*\dy}) circle[radius=0.08];
\node at ({\dx},0) [above left] {$Q$};
\pgfmathparse{\Y-2*\X*\X*\X/3}
\xdef\D{\pgfmathresult}
\pgfmathparse{\Y-\X*\X}
\xdef\C{\pgfmathresult}
\draw[color=blue]
	({1*\dx-0.1},{(1+\C)*\dy})
	--
	({1*\dx+0.1},{(1+\C)*\dy});
\node at (\dx,{(1+\C)*\dy})   [left] {$-\frac{1}{4}$};
\node at (\dx,{(2/3+\D)*\dy}) [left] {$-\frac{7}{12}$};
\draw[color=blue]
	({1*\dx-0.1},{(2/3+\D)*\dy})
	--
	({1*\dx+0.1},{(2/3+\D)*\dy});
\draw[->] (-0.1,0) -- ({2.2*\dx+0.5},0) coordinate[label={$x$}];
\draw[->] (0,{-0.7*\dy-0.1}) -- (0, {1.2*\dy+0.3}) coordinate[label={$y$}];
\draw ({\X*\dx},-0.05) -- ({\X*\dx},0.05);
\end{tikzpicture}
\caption{Characteristics for the hyperbolic operator $L$ of
\eqref{90000026:operator}
\label{90000026:fig}}
\end{figure}
\begin{teilaufgaben}
\item
The symbol matrix of $L$ is
\[
A
=
\begin{pmatrix}
  1    & x(1+x) \\
x(1+x) & 4x^3
\end{pmatrix}
\qquad\Rightarrow\qquad
\det A = 4x^3 -x^2(1+x)^2
\]
Multiplying out we find
\[
\det A
=
4x^3 - x^2 -2x^3 -x^4
=
-x^2 + 2x^3 -x^4
=
-(x-x^2)^2
=
-x^2 (1-x)^2
<
0
\]
because $1-x\ne 0$ for $x>1$.
Thus  $L$ is hyperbolic.
\item
The differential equation for the characteristics is
\[
\dot{y}^2 -2x(1+x) \dot{y}\dot{x} + 4x^3\dot{x}^2 = 0
\]
By dividing by $\dot{x}$ and using $y'=\dot{y}/\dot{x}$ we can simplify
this to an ordinary differential equation
\[
y^{\prime 2} -2x(1+x)y' + 4x^3 = 0.
\]
This can be factored as
\[
(y'-2x)(y'-2x^2) = 0
\qquad\Rightarrow\qquad
y'
=
\begin{cases}
2x\\
2x^2
\end{cases}
\]
which can also be found using the quadratic formula:
\begin{align*}
y'
&=
x(1+x) \pm \sqrt{x^2(1+x)^2 - 4x^3}
\\
&=
x(1+x) \pm \sqrt{x^2+2x^3+x^4-4x^3}
\\
&=
x(1+x) \pm \sqrt{x^2-2x^3+x^4}
\\
&=
x(1+x) \pm x(1-x)
\\
&=
\begin{cases}
2x  \\
2x^2.
\end{cases}
\end{align*}
This equation can directly be integrated:
\begin{align*}
y'&=2x   &&\Rightarrow& y&=x^2 + C \\
y'&=2x^2 &&\Rightarrow& y&=\frac23 x^3 + D.
\end{align*}
Figure~\ref{90000026:fig} shows the first set of characteristics
as green curves and the second set as red curves.
We have to find the values for the constant $C$ for the two sets of
curves.
\begin{align*}
	y_0 &= x_0^2 + C         &&\Rightarrow& C &= y_0 - x_0^2 = 1-\frac{9}{4} = -\frac{5}{4}\\
	y_0 &= \frac23 x_0^3 + D &&\Rightarrow& D &= y_0 -\frac23x_0^3 = 1 - \frac{2}{3}\frac{27}{8} = 1-\frac{9}{4} = -\frac{5}{4}.
\end{align*}
The corresponding points on the boundary where $x=1$ are
\[
(1,1-{\textstyle\frac54})
=
(1,-\textstyle{\frac14})
\qquad\text{and}\qquad
(1,{\textstyle\frac23}-{\textstyle\frac54})
=(1,-{\textstyle\frac{7}{12}}).
\]
Only points $(1,y)$ with $y\in[ -\frac{1}{4}, -\frac{7}{12} ]$ can
influence the given point.
The point $Q=(1,0)$ is not one of them.
So no, a change in boundary condition in $(1,0)$ has no influence on
the value in the point $P=(\frac32,1)$.
\qedhere
\end{teilaufgaben}
\end{loesung}

\begin{bewertung}
Symbol matrix ({\bf A}) 1 point,
negative sign of the determinant and conclusion that $L$ is hyperbolic
({\bf H}) 1 point,
equation for characteristics ({\bf E}) 1 point,
factorization or solution of quadratic equation ({\bf F}) 1 point,
solution curves ({\bf S}) 1 point,
decision that $Q$ cannot influence $P$ ({\bf D}) 1 point.
\end{bewertung}

