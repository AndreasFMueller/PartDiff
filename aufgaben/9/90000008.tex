Die Differentialgleichung
\[
\frac{\partial^2 u}{\partial x^2}-y^2\frac{\partial^2u}{\partial y^2}=0
\]
soll im Gebiet
\[
\Omega=\{(x,y)\,|\,x>0,\;y>0\}
\]
gelöst werden.
In einer ersten Phase sollen nur die Werte $u(a,y)$ für ein festes 
$a > 0$ und $y$-Werte mit $0<y<b$ berechnet werden.
Es sind die Randwerte
\[
u(0,y)=g(y)=e^{y^2}\sin y
\]
vorgegeben. Die Amplitude wächst für wachsendes $y$ sehr stark an,
was die numerische Berechnung erschwert.

Der mit der Implementation betraute Programmierer beobachtet bei seinen
Tests, dass es auf die Funktionswerte von $g(y)$ für grosse
$y$ nicht ankommt und fragt Sie deshalb, ob es möglich wäre, zur
Vereinfachung der Berechnung die Anfangswerte für $y>c$ durch
$0$ zu ersetzen, und wie gross $c$ sein müsste. Was antworten Sie?

\begin{loesung}
Die partielle Differentialgleichung hat die Symbolmatrix
\[
A=
\begin{pmatrix}
1&0\\0&-y^2
\end{pmatrix}
\]
mit der Determinanten
\[
\det(A)=-y^2,
\]
die in ganz $\Omega$ negativ ist.
Die Differentialgleichung ist also hyperbolisch.

Für hyperbolische partielle Differentialgleichung kann mit Hilfe der
Charakteristiken entschieden werden, ob ein Randwert Einfluss auf einen
Funktionswert hat. Wir stellen daher zunächst die Differentialgleichung
der Charakteristiken auf:
\[
\dot y^2-y^2\dot x^2=(\dot y+y\dot x)(\dot y-y\dot x)=0
\]
Die Charakteristiken sind daher Lösungen jeweils einer der beiden
Differentialgleichungen
\begin{align*}
\dot y-y\dot x&=0&&\Rightarrow&\dot y&=y\dot x\\
\dot y+y\dot x&=0&&\Rightarrow&\dot y&=-y\dot x
\end{align*}
Diese Differentialgleichung kann durch direktes Integrieren
gelöst werden:
\begin{align*}
\frac1y\frac{dy}{dt}&=\pm\frac{dx}{dt}\\
\int \frac1y\frac{dy}{dt}\,dt&=\pm\int \frac{dx}{dt}\,dt\\
\log |y|&= \pm x + C
\end{align*}
Daraus kann man ablesen, dass die Charakteristiken die Kurven 
$y=y_0e^{\pm x}$ sind.

Mit der Kenntnis der Charakteristiken lässt sich jetzt auch die 
Frage beantworten, ob die Randwerte jenseits von $y=c$ einen Einfluss
auf die Lösungswerte $u(a,y)$ mit $0<y<b$ haben. Der Randwert
bei $y_0$ kann die Funktionswerte zwischen $y_0e^a$ und $y_0e^{-a}$
beinflussen. Wenn also sichergestellt ist, dass $y_0e^{-a} > b$
oder $y_0e^{-a} < -b$, dann kann der Randwert die genannten Funktionswerte
nicht beinflussen. Das bedeutet, dass 
\[
y_0e^{-a}>b
\qquad\Rightarrow\qquad
y_0>be^a
\]
sein muss, also muss man $c=be^a$ wählen.
\end{loesung}

\begin{bewertung}
Symbolmatrix ({\bf S}) 1 Punkt,
Differentialgleichung ist hyperbolisch ({\bf H}) 1 Punkt,
Differentialgleichung der Charakteristiken ({\bf X}) 1 Punkt,
Lösung der Differentialgleichung ({\bf L}) 1 Punkt,
Bedingung für $y_0$-Werte ({\bf B}) 1 Punkt,
Bestimmung von $c$ ({\bf C}) 1 Punkt.
\end{bewertung}
