Auf dem Gebiet $\Omega=\{(x,y)\,|\, x>1, y>1\}$ soll die partielle
Differentialgleichung
\begin{equation}
\frac{\partial^2 u}{\partial x^2}-2\kappa xy\frac{\partial^2 u}{\partial x\partial y}=0
\label{90000010:gleichung}
\end{equation}
gelöst werden mit Randbedingungen für $x=1$, also auf dem linken Rand von
$\Omega$. Der Parameter $\kappa$ ist positiv.
Hat eine Änderung der Randbedingungen für $x=1$ und $y<2$ einen
Einfluss auf die Lösung im Quadrat $\{(x,y)\,|\, 2<x<3,2<y<3\}$?

\begin{loesung}
Die Symbolmatrix dieser linearen partiellen Differentialgleichung zweiter
Ordnung ist
\[
A=\begin{pmatrix}
1&-\kappa xy\\
-\kappa xy&0
\end{pmatrix}
\qquad
\Rightarrow
\qquad
\det(A)=-\kappa^2x^2y^2<0
\]
in $\Omega$, es handelt sich also um eine hyperbolische partielle
Differentialgleichung, bei der mit Hilfe der Charakteristiken entschieden
werden kann, ob die beschriebene Änderung der Randbedingungen eine 
Auswirkung auf die Lösung im Quadrat $(2,3)^2$ hat.

Die Differentialgleichung der Charakteristiken ist
\begin{align*}
1\cdot \dot y^2 +2\kappa xy\cdot \dot x \dot y&=0,\\
\Rightarrow\qquad
\dot y(\dot y+2\kappa xy\,\dot x)&=0.
\end{align*}
Das Produkt verschwindet, wenn einer der Faktoren verschwindet, wir erhalten
also die zwei Differentialgleichungen
\begin{align*}
\dot y &= 0
&
&\text{oder}
&
\dot y&=-2\kappa xy\dot x.
\end{align*}
Die Gleichung $\dot y=0$ hat als Lösung horizontale Geraden, sie sind
in Abbildung~\ref{90000010:bild} grün eingezeichnet.

Die zweite Differentialgleichung kann mit Hilfe von $y'=\dot y/\dot x$ 
vereinfacht und dann mit Separation gelöst werden:
\begin{align*}
y'&=-2\kappa xy\\
\int \frac{dy}y&=\int -2\kappa x\,dx\\
\log y&=-\kappa x^2+K\\
y(x)&=Ce^{-\kappa x^2}
\end{align*}
Die Funktionen $Ce^{-\kappa x^2}$ sind monoton fallend, sie sind in
Abbildung~\ref{90000010:bild} rot eingezeichnet.
Insbesondere trifft
keine Charakteristik, die in Punkten $(1,y)$ mit $y<2$ beginnt das
Quadrat $(2,3)^2$.
\begin{figure}
\begin{center}
\includeagraphics[]{domain-1.pdf}
\end{center}
\caption{Charakteristiken für die hyperbolische partielle
Differentialgleichung \eqref{90000010:gleichung} für den Parameterwert
$\kappa =0.3$, das Quadrat $(2,3)^2$, in dem sich die Lösung nicht
ändern soll, ist blau hervorgehoben.
\label{90000010:bild}}
\end{figure}
\end{loesung}

\begin{bewertung}
Symbolmatrix ({\bf A}) 1 Punkt,
hyperbolische partielle Differentialgleichung ({\bf H}) 1 Punkt,
Differentialgleichung der Charakteristiken ({\bf D}) 1 Punkt,
Zwei Lösungsscharen für die Charakteristiken ({\bf L}) 2 Punkte,
Einfluss auf Lösung ({\bf E}) 1 Punkt.
\end{bewertung}



