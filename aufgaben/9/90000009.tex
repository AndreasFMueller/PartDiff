Die Differentialgleichung
\begin{equation}
\frac{\partial^2 u}{\partial x^2}
-
y\frac{\partial^2 u}{\partial y^2}
=
0
\label{90000009:dgl}
\end{equation}
soll im Gebiet
\[
\Omega = \{(x,y) \,|\, x > 0, y > 0\}
\]
gel"ost werden.
Am linken Rand soll die L"osung die Randbedingung
\[
u(0, y)=\sin e^{y^2},\quad y>0
\]
erf"ullen.
Da diese Funktion f"ur grosse Argumente sehr rasch oszilliert,
ist die numerische L"osung nicht ganz einfach.
F"ur eine erste Diskussion der L"osung w"are es jedoch ausreichend,
wenn man die Werte $u(1,y)$ f"ur $0<y<1$ bestimmen k"onnte.
Es wird behauptet, man k"onne diese Werte auch bestimmen,
wenn die Randwerte $u(0,y)=0$
gesetzt werden f"ur gen"ugend grosse $y>a$. Wie gross muss $a$
sein, damit diese Aussage stimmt?

\begin{loesung}
Die partielle Differentialgleichung hat die Symbolmatrix
\[
A=\begin{pmatrix}
1&0\\
0&-y
\end{pmatrix}
\]
mit der Determinanten
\[
\det(A) = -y,
\]
die im ganzen Gebiet $\Omega$ negativ ist. Die Differentialgleichung ist
also hyperbolisch.

F"ur hyperbolische partielle Differentialgleichungen kann mit Hilfe der
Charakteristiken entschieden werden, ob ein Randwert Einfluss auf einen
Funktionswert hat. Wir stellen daher zun"achst die Differentialgleichung
der Charakteristiken auf:
\begin{align*}
\dot y^2-y\dot x^2&=0\\
\biggl(\frac{\dot y}{\dot x}\biggr)^2&=y\\
y'&=\pm\sqrt{y}.
\end{align*}
Diese gew"ohnliche Differentialgleichung kann mit der Separationsmethode
gel"ost werden:
\begin{align*}
\int\frac{dy}{\sqrt{y}}
=\int y^{-\frac12}\,dy
=2y^{\frac12}
&=\pm\int dx
=\pm x+C\\
\Rightarrow\qquad
y&=\biggl(\frac12 x + C\biggr)^2
\end{align*}
Die Charakteristiken sind also Parabeln, die die $x$-Achse ber"uhren,
siehe Abbildung~\ref{90000009:domain}.
\begin{figure}
\begin{center}
\includeagraphics[]{domain-1.pdf}
\end{center}
\caption{Charakteristiken (rot) der Differentialgleichung (\ref{90000009:dgl}).
Die hervorgehobenen Charakteristiken durch den Punkt $(1,1)$ bestimmen,
welche Intervalle von Anfangspunkten auf der $y$-Achse die Werte 
$u(1,y)$ der L"osung f"ur $0<y<1$ beeinflussen k"onnen.
\label{90000009:domain}}
\end{figure}

Die Frage kann also beantwortet werden, indem man die Charakteristik
findet, welche durch den Punkt $(0,a)$ und $(1,1)$ geht:
\begin{align*}
1&=\biggl(\frac12\cdot 1^2 + C\biggr)^2&&\Rightarrow& \pm1&=\frac12 + C&&\Rightarrow& C&=\begin{cases}\frac12&\\-\frac32\end{cases}\\
a&=\biggl(\frac12\cdot 0 + C\biggr)^2=C^2&&           &     &            &&\Rightarrow& a&=\begin{cases}\frac14&\\\frac94&\end{cases}
\end{align*}
Der Wert $a=\frac14$ geh"ort zur Charakteristik, die im Punkt $(-1,0)$
die $x$-Achse ber"uhrt, der Wert $a=\frac94$ geh"ort zur Charakterstik,
die in $(3,0)$ ber"uhrt.
Kein Randwert mit $y>a=\frac94$ kann je einen Funktionswert $u(1,y)$ mit $0<y<1$
beeinflussen.
\end{loesung}

\begin{bewertung}
Symbolmatrix ({\bf S}) 1 Punkt,
Differentialgleichung ist hyperbolisch ({\bf H}) 1 Punkt,
Differentialgleichung der Charakteristiken ({\bf X}) 1 Punkt,
L"osung der Differentialgleichung ({\bf L}) 1 Punkt,
Bedingung f"ur Charakteristik druch Punkt $(1,1)$ ({\bf B}) 1 Punkt,
Bestimmung von $a$ ({\bf A}) 1 Punkt.
\end{bewertung}
