Auf dem Gebiet
\[
\Omega = \{ (x,y)\,|\, 1 < x^2 + y^2\}
\]
soll die partielle Differentialgleichung in Polarkoordinaten
\begin{equation}
\frac{\partial^2 u}{\partial r^2}
+
\frac{\partial^2 u}{\partial \varphi^2}
=
0
\label{90000015:dgl}
\end{equation}
gelöst werden.
Auf welchem Teil des Randes $x^2 +y^2 = 1$ müssen Randwerte vorgegeben
werden, dass die Lösung für Punkte mit $x>0$ und $x^2 + y^2 = 4$
eindeutig bestimmt ist?

\begin{loesung}
Die Differentialgleichung \eqref{90000015:dgl} hat das Symbol
\[
A
=
\begin{pmatrix}
1&0\\
0&-1
\end{pmatrix}
\qquad\Rightarrow\qquad
\det A=-1<0
\]
und ist daher hyperbolisch.
Die gestellte Frage kann daher mit Hilfe der Charakteristiken beantwortet
werden.
Die Differentialgleichung der Charakteristiken also Kurven $(r(t),\varphi(t))$
in Abhängigkeit von einem Parameter $t$ ist
\[
\dot\varphi(t)^2 - \dot r(t)^2=0.
\]
Diese Differenz von zwei Quadraten kann faktorisiert werden in
\[
\dot\varphi(t)^2 - \dot r(t)^2
=
(\dot\varphi(t) + \dot r(t))
(\dot\varphi(t) - \dot r(t)).
\]
Es folgen zwei Differentialgleichungen erster Ordnung
\[
\dot\varphi(t)\pm\dot r(t) = 0
\qquad\Rightarrow\qquad
\frac{dr}{d\varphi}
=
\frac{\dot r}{\dot \varphi}
=
\pm 1
\]
mit den Lösungen
\[
r(\varphi) =  \pm \varphi + \delta
\]
mit der Integrationskonstante $\delta$.
Diese Kurven sind Spiralen um den Nullpunkt.

Die Kurve 
\[
C
=
\{ (x,y)\,|\, x>0\wedge x^2 +y^2=4\}
\]
besteht aus Punkten mit $r=2$ und $-\frac{\pi}{2}<\varphi<\frac{\pi}2$.
Um die Lösung der Differentialgleichung auf $C$ festzulegen, müssen
Randwerte auf den Anfangspunkten vorgegeben sein, zu denen Charakteristiken
gehören, die durch $C$ gehen (Abbildung~\ref{90000015:fig}).
Dies gilt für Anfangswerte von $\varphi$ im Interval
$[-\frac{\pi}2-1,\frac{\pi}2+1]$.
\begin{figure}
\centering
\includeagraphics[]{domain-1.pdf}
\caption{Definitionsgebiet $\Omega$ und Charakteristiken der
Differentialgleichung \eqref{90000015:dgl}.
Rot eingezeichnet der Teil des Randes, auf dem die Randwerte vorgegeben
sein müssen, damit die Werte der Lösung der Differentialgleichung auf dem
roten Halbkreis eindeutig bestimmt sind.
\label{90000015:fig}}
\end{figure}
\end{loesung}


