%
% tikztemplate.tex -- Template für standalone TIKZ Bilder
%
% (c) 2019 Prof Dr Andreas Müller, Hochschule Rapperswil
%
\documentclass[tikz,12pt]{standalone}
\usepackage{amsmath}
\usepackage{times}
\usepackage{txfonts}
\usepackage{pgfplots}
\usepackage{csvsimple}
\usepackage{ifthen}
\usetikzlibrary{arrows,intersections,math}
\begin{document}
\newboolean{fein}
\setboolean{fein}{true}
\begin{tikzpicture}[>=latex,thick,scale=3]

\definecolor{darkgreen}{rgb}{0,0.6,0}
\def\xmin{0.5}

\fill[color=blue!10] (0.5,-2.10) rectangle (3.15,1.10);

\draw[->] (-0.1,0) -- (3.3,0) coordinate[label={$x$}];
\draw[->] (0,-2.1) -- (0,1.2) coordinate[label={right:$y$}];

\xdef\Cplus{0.6105}
\xdef\Cminus{-0.93338}

\pgfmathparse{ln((exp(1/2)-exp(-1/2))/2)+ln((exp(1/2)-1)/(exp(1/2)+1))+\Cplus}
\xdef\yplus{\pgfmathresult}
\pgfmathparse{ln((exp(1/2)-exp(-1/2))/2)-ln((exp(1/2)-1)/(exp(1/2)+1))+\Cminus}
\xdef\yminus{\pgfmathresult}

\begin{scope}
\clip (-0.5,-2.10) rectangle (3.15,1.10);

\foreach \C in {-4,-3,-2,-1,0,1,2}{
	\draw[color=red]
		plot[domain=\xmin:3.15,samples=100]
		({\x},{ln((exp(\x)-exp(-\x))/2)+ln((exp(\x)-1)/(exp(\x)+1))+\C});
}

\foreach \C in {-4,-3,-2,-1,0}{
	\draw[color=darkgreen]
		plot[domain=\xmin:3.15,samples=100]
		({\x},{ln((exp(\x)-exp(-\x))/2)-ln((exp(\x)-1)/(exp(\x)+1))+\C});
}

\ifthenelse{\boolean{fein}}{
\foreach \C in {-3.8,-3.6,...,2.8}{
	\draw[color=red,line width=0.1pt]
		plot[domain=\xmin:3.15,samples=100]
		({\x},{ln((exp(\x)-exp(-\x))/2)+ln((exp(\x)-1)/(exp(\x)+1))+\C});
}

\foreach \C in {-4.4,-4.2,...,0.2}{
	\draw[color=darkgreen,line width=0.1pt]
		plot[domain=\xmin:3.15,samples=100]
		({\x},{ln((exp(\x)-exp(-\x))/2)-ln((exp(\x)-1)/(exp(\x)+1))+\C});
}
}{}

\fill[color=orange!20]
	plot[domain=\xmin:1,samples=100]
		({\x},{ln((exp(\x)-exp(-\x))/2)+ln((exp(\x)-1)/(exp(\x)+1))+\Cplus})
	--
	plot[domain=1:\xmin,samples=100]
		({\x},{ln((exp(\x)-exp(-\x))/2)-ln((exp(\x)-1)/(exp(\x)+1))+\Cminus})
	--
	cycle;
\draw[color=red]
	plot[domain=\xmin:1,samples=100]
		({\x},{ln((exp(\x)-exp(-\x))/2)+ln((exp(\x)-1)/(exp(\x)+1))+\Cplus});
\draw[color=darkgreen]
	plot[domain=\xmin:1,samples=100]
		({\x},{ln((exp(\x)-exp(-\x))/2)-ln((exp(\x)-1)/(exp(\x)+1))+\Cminus});


\end{scope}

\draw[color=orange,line width=1.4pt] ({1/2},{\yminus}) -- ({1/2},{\yplus});
\draw[color=blue,line width=1.4pt] (0.5,1.10) -- ({1/2},{\yminus});
\draw[color=blue,line width=1.4pt] (0.5,-2.10) -- ({1/2},{\yplus});

\draw[color=red] ({0.5-0.1/3},\yplus) -- ({0.5+0.1/3},\yplus);
\node[color=red] at (0.5,\yplus) [left] {$y_+(\frac12)$};

\draw[color=darkgreen] ({0.5-0.1/3},\yminus) -- ({0.5+0.1/3},\yminus);
\node[color=darkgreen] at (0.5,\yminus) [left] {$y_-(\frac12)$};

\foreach \y in {-2,-1.5,-1,-0.5,0.5,1}{
	\draw ({-0.1/3},\y) -- ({0.1/3},\y);
	\node at ({-0.05/3},\y) [left] {$\y$};
}

\node at (0,0) [below left] {$O$};

\foreach \x in {1,2,3}{
	\fill[color=white,opacity=0.8]
		({\x-0.06},{0.02})
		--
		({\x+0.06},{0.02})
		--
		({\x+0.06},{0.2})
		--
		({\x-0.06},{0.2})
		-- cycle;
	\draw (\x,{-0.1/3}) -- (\x,{0.1/3});
	\node at (\x,{0.05/3}) [above] {$\x$};
}

\fill[color=blue] (1,0) circle[radius=0.03];
\node[color=blue] at (1,0) [below right] {$P$};

\node at (0.5,0.5) [above right,rotate=90] {$\partial\Omega$};
\node at (2,-0.8) {$\Omega$};

\end{tikzpicture}
\end{document}

