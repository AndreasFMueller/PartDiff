On the domain $\Omega= \{(x,y)\mid x > 0\}$, the differential operator
\begin{equation}
L
=
\frac{\partial^2}{\partial x^2}
+
(2x-1)
\frac{\partial^2}{\partial x\,\partial y}
-
2x
\frac{\partial^2}{\partial y^2}
\label{90000027:operator}
\end{equation}
is given.
\begin{teilaufgaben}
\item
Is this operator elliptic, parabolic or hyperbolic?
\item
Can a change of a Dirichlet boundary condition in the point $Q=(0,0)$
influence the solution of partial differential equation involving the
operator $L$ in the point $P=(\frac12,\frac12)$
\end{teilaufgaben}

\begin{loesung}
\begin{figure}
\centering
\begin{tikzpicture}[>=latex,thick]
\def\dx{5}
\def\dy{5}
\def\X{0.5}
\def\Y{0.5}
\pgfmathparse{\Y+\X}
\xdef\D{\pgfmathresult}
\pgfmathparse{\Y-\X*\X}
\xdef\C{\pgfmathresult}
\begin{scope}
	\fill[color=gray!10] (0,{-0.7*\dy}) rectangle ({2.2*\dx},{1.2*\dy});
	\clip ({0*\dx},{-0.7*\dy}) rectangle ({2.2*\dx},{1.2*\dy});
	\fill[color=blue!20]
		plot[domain=0:\X,samples=10] 
			({\x*\dx},{(-\x+(\D))*\dy})
		--
		plot[domain=\X:0,samples=10]
			({\x*\dx},{(\x*\x+(\C))*\dy})
		--
		cycle;
	\draw[color=darkgreen,line width=0.3pt] plot[domain=0:2.2,samples=20]
			({\x*\dx},{(\x*\x+\C)*\dy});
	\draw[color=red,line width=0.3pt] plot[domain=0:2.2,samples=20]
			({\x*\dx},{(-\x+\D)*\dy});
	\foreach \C in {-10,-9.75,...,2}{
		\draw[color=darkgreen] plot[domain=0:2.2,samples=20]
			({\x*\dx},{(\x*\x+\C)*\dy});
	}
	\foreach \D in {-1,-0.75,...,3}{
		\draw[color=red] plot[domain=0:2.2,samples=20]
			({\x*\dx},{(-\x+\D)*\dy});
	}
\end{scope}
\fill[color=white,opacity=1.0]
	({\X*\dx+0.05},{\Y*\dy-0.1}) rectangle ({\X*\dx+1.90},{\Y*\dy-0.70});
\draw[line width=0.2pt] (0,{\Y*\dy}) -- ({\X*\dx},{\Y*\dy});
\draw (-0.05,{\Y*\dy}) -- (0.05,{\Y*\dy});
\node at (0,{\Y*\dy}) [left] {$\frac12$};
\draw[line width=0.2pt] ({\X*\dx},0) -- ({\X*\dx},{1.2*\dy});
\node at ({\X*\dx},{1.2*\dy}) [above] {$\frac12$};
\node at ({\X*\dx},{\Y*\dy}) [below right] {$P=(\frac12,\frac12)$};
\node at (0,0) [above left] {$Q$};
\draw[color=blue] ({0*\dx},{-0.7*\dy}) -- ({0*\dx},{1.2*\dy});
\fill[color=blue] ({\X*\dx},{\Y*\dy}) circle[radius=0.08];
\fill[color=blue] ({0*\dx},{0*\dy}) circle[radius=0.08];
\pgfmathparse{\Y+\X}
\xdef\D{\pgfmathresult}
\pgfmathparse{\Y-\X*\X}
\xdef\C{\pgfmathresult}
\draw[color=blue] (-0.05,{\C*\dy}) -- (0.05,{\C*\dy});
\node[color=blue] at (0,{\C*\dy}) [left] {$\frac14$};
\draw[color=blue] (-0.05,{\D*\dy}) -- (0.05,{\D*\dy});
\node[color=blue] at (0,{\D*\dy}) [left] {$1$};
\draw[->] (-0.1,0) -- ({2.2*\dx+0.5},0) coordinate[label={$x$}];
\draw[->] (0,{-0.7*\dy-0.1}) -- (0,{1.2*\dy+0.3}) coordinate[label={left:$y$}];
\draw ({\X*\dx},-0.05) -- ({\X*\dx},0.05);
\draw ({\dx},-0.05) -- ({\dx},0.05);
\node at ({1*\dx},0) [below] {$1$};
\draw ({2*\dx},-0.05) -- ({2*\dx},0.05);
\node at ({2*\dx},0) [below] {$2$};
\end{tikzpicture}
\caption{Characteristics for the hyperbolic operator $L$ of
\eqref{90000027:operator}
\label{90000027:fig}}
\end{figure}
\begin{teilaufgaben}
\item
The symbol matrix of $L$ is
\[
A
=
\begin{pmatrix}
  1       & x-\frac12 \\
x-\frac12 &   -2x
\end{pmatrix}
\qquad\Rightarrow\qquad
\det A
=
-2x -(x-{\textstyle\frac12})^2.
\]
Multiplying out we find
\[
\det A
=
-2x-x^2 +x-\frac14
=
-x^2-x-\frac14
=
-(x+{\textstyle\frac12})^2
<
0
\]
because $x+\frac12\ne 0$ for $x>0$.
Thus  $L$ is hyperbolic.
\item
The differential equation for the characteristics is
\[
\dot{y}^2 -(2x-1) \dot{y}\dot{x} - 2x\dot{x}^2 = 0
\]
By dividing by $\dot{x}$ and using $y'=\dot{y}/\dot{x}$ we can simplify
this to an ordinary differential equation
\[
y^{\prime 2} -(2x-1)y' - 2x = 0.
\]
This can be factored as
\[
(y'+1)(y'-2x) = 0
\qquad\Rightarrow\qquad
y'
=
\begin{cases}
-1\\
2x
\end{cases}
\]
which can also be found using the quadratic formula:
\begin{align*}
y'
&=
(x-{\textstyle\frac12}) \pm \sqrt{(x-{\textstyle\frac12})^2+2x}
\\
&=
(x-{\textstyle\frac12}) \pm \sqrt{x^2-x+\frac14+2x}
\\
&=
(x-{\textstyle\frac12}) \pm \sqrt{x^2+x+\frac14}
\\
&=
x-{\textstyle\frac12} \pm (x+{\textstyle\frac12})
\\
&=
\begin{cases}
2x  \\
-1.
\end{cases}
\end{align*}
These equations can directly be integrated:
\[
\begin{aligned}
y'&=2x   &&\Rightarrow& y&= x^2 + C \\
y'&=-1   &&\Rightarrow& y&=-x   + D.
\end{aligned}
\]
Figure~\ref{90000027:fig} shows the first set of characteristics
as green curves and the second set as red curves.
We have to find the values for the constant $C$ for the two sets of
curves.
\[
\begin{aligned}
	y_0 &= x_0^2 + C         &
&\Rightarrow&
C &= y_0 - x_0^2 = \frac12-\frac14 = \frac{1}{4}\\
	y_0 &= -x_0 + D &
&\Rightarrow&
D &= y_0 + x_0 = \frac12 + \frac12 = 1.
\end{aligned}
\]
The corresponding points on the boundary where $x=0$ are
\[
(0,{\textstyle\frac14})
\qquad\text{and}\qquad
(0,1).
\]
Only points $(0,y)$ with $y\in[ \frac{1}{4}, 1]$ can
influence the given point.
The point $Q=(0,0)$ is not one of them.
So no, a change in boundary condition in $(0,0)$ has no influence on
the value in the point $P=(\frac12,\frac12)$.
\qedhere
\end{teilaufgaben}
\end{loesung}

\begin{bewertung}
Symbol matrix ({\bf A}) 1 point,
negative sign of the determinant and conclusion that $L$ is hyperbolic
({\bf H}) 1 point,
equation for characteristics ({\bf E}) 1 point,
factorization ({\bf F}) 1 point,
solution curves ({\bf S}) 1 point,
decision that $Q$ cannot influence $P$ ({\bf D}) 1 point.
\end{bewertung}

