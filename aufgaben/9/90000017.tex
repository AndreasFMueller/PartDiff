Betrachten Sie die Differentialgleichung
\begin{equation}
\frac{\partial^2 u}{\partial x^2}
+
2
\frac{\partial^2 u}{\partial x\partial y}
-
e^{-x}
\frac{\partial^2 u}{\partial y^2}
=
0
\label{90000017:dgl}
\end{equation}
auf dem Gebiet
\begin{equation}
\Omega
=
\left\{(x,y)\,\left|\, 0<x<1\wedge 0 <y<1\right.\right\}
\label{90000017:gebiet}
\end{equation}
mit den Randbedingungen
\[
u(x,0) = f(x)
\qquad\text{und}\qquad
u(0,y) = g(y)
\]
mit glatten Funktionen $f(x)$ und $g(y)$, die ausserdem $f(0)=g(0)$ erfüllen.
Ist das Problem gut gestellt?

\begin{hinweis}
Im Laufe der Lösung werden Sie auf eine gewöhnliche Differentialgleichung
für $y(x)$ stossen, die relativ schwierig zu lösen ist.
Versuchen Sie nicht, diese zu lösen.
Leiten Sie vielmehr aus der Differentialgleichung eine Formel für $y'(x)$ ab,
untersuchen Sie das Vorzeichen von $y'(x)$ und leiten Sie daraus ab,
ob das Problem gut gestellt ist.
\end{hinweis}

\begin{loesung}
\begin{figure}
\centering
\includeagraphics[]{char.pdf}
\caption{Charakteristiken der Differentialgleichung~\eqref{90000017:dgl}
im Gebiet $\Omega$.
Die grünen Charakteristiken überdecken nicht das ganze Gebiet.
\label{90000017:char}}
\end{figure}
Das Symbol des Differentialoperators  ist die Matrix
\[
A
=
\begin{pmatrix} 1 & 1\\ 1 & -e^{-x} \end{pmatrix}
\qquad\Rightarrow\qquad
\det A
=
-1 -e^{-x} < -1 < 0,
\]
insbesondere ist der Operator hyperbolisch.

Um zu bestimmen, ob das Problem gut gestellt ist, müssen wir daher die
Charakteristiken bestimmen.
Beide Charakteristikenscharen, die von den Randwerten ausgehen, müssen das
ganze Gebiet überdecken, damit das Problem gut gestellt ist.

Die Charakteristiken erfüllen die Differentialgleichung
\begin{align*}
\dot y^2
-
2 \cdot \dot x \dot y
-
e^{-x}\cdot \dot x^2
&=
0
\\
(y')^2
-2\cdot y'
-e^{-x} 
=
0
\end{align*}
Diese Gleichung kann mit der Lösungsformel für quadratische Gleichungen
aufgelöst werden, man findet
\begin{equation}
y' = 1 \pm \sqrt{1+e^{-x}}.
\label{90000017:ableitung}
\end{equation}
Daraus kann man ablesen, dass die Steigung der Charakteristiken nicht von
$y$ abhängt, sondern nur von $x$.
Zu jedem der Vorzeichen in \eqref{90000017:ableitung} gibt es eine Schar
von Charakteristiken, die sich nur durch eine Verschiebung in $y$-Richtung
unterscheiden.

Für das positive Vorzeichen in \eqref{90000017:ableitung} ist die
Ableitung $y' > 0$, für das negative Vorzeichen ist $y'<0$.
Die eine Schar von Charakteristiken verläuft daher von links oben
nach rechts unten, die andere Schar verläuft von links unten nach
rechts oben.
Der Verlauf der Charakteristiken ist in Abbildung~\ref{90000017:char}
dargestellt.

Die fallenden Charakteristiken (in Abbildung~\ref{90000017:char} grün
dargestellt), die vom linken Rand des Gebietes
ausgehen, können nicht das ganze Gebiet überdecken, das hellgrün
hinterlegte Teilgebiet wird nicht abgedeckt, somit ist das
Problem nicht gut gestellt.
\end{loesung}

\begin{diskussion}
Wenn man $\Omega$ auf den ganzen ersten Quadranten ausdehnt, überdecken
die Charakteristiken das ganze Gebiet.
\end{diskussion}

\begin{bewertung}
Symbol ({\bf A}) 1 Punkt,
hyperbolisch ({\bf H}) 1 Punkt,
Differentialgleichung der Charakteristiken ({\bf D}) 1 Punkt,
Quadratische Gleichung für Steigung ({\bf Q}) 1 Punkt,
Kriterium für gut gestellt ({\bf K}) 1 Punkt,
Folgerung ({\bf F}) 1 Punkt.
\end{bewertung}

