Auf dem Gebiet $\Omega=\{(x,y)\in\mathbb R^2\,|\, x>0\}$ ist die
Differentialgleichung
\begin{equation}
\sinh a \frac{\partial^2 u}{\partial x^2}
+
2\cosh a \frac{\partial^2 u}{\partial x\partial y}
+
\sinh a \frac{\partial^2 u}{\partial y^2}
=
0
\label{90000013:dgl}
\end{equation}
gegeben.
Von dem Parameter $a$ weiss man nur, dass $a>0$ ist.
Ausserdem sind die Randbedingungen
\[
u(0,y)=g(y)\qquad\text{und}\qquad \frac{\partial u}{\partial x}(0,y)=0
\]
gegeben, wobei $g(y)$ eine f"ur grosse Argumentwerte nur sehr aufwendig zu
berechnende Funktion ist.
\begin{teilaufgaben}
\item
Ist das Problem gut gestellt?
\item
Wegen der Schwierigkeiten bei der Berechnung der Funktion $g$
wurde eine numerische L"osung f"ur $u$ gerechnet, indem die
Randwerte f"ur $|y|>5$ null gesetzt wurden.
Diese L"osung wird nur in einem Teil von $\Omega$ mit der wahren L"osung
"ubereinstimmen.
F"ur welche Werte des Parameters $a$ kann man davon ausgehen, dass die
numerische L"osung die wahre L"osung im Rechteck
$R=\{(x,y)\,|\, 0\le x\le 1,-3\le y\le 3\}$ korrekt
wiedergibt?
\end{teilaufgaben}


\begin{loesung}
\begin{figure}
\centering
\includeagraphics[]{domain-1.pdf}
\caption{Charakteristiken der Differentialgleichung~(\ref{90000013:dgl})
und Rechteck $R$ (blau). 
F"ur die Frage in Teilaufgabe b) der Aufgabe ist die fett ausgezogene
rote Charakteristik durch den Punkt $(0,-5)$ massgeben.
\label{90000013:domain}}
\end{figure}
\begin{teilaufgaben}
\item
Die Symbolmatrix des Differentialoperators ist
\[
A=\begin{pmatrix}\sinh a&\cosh a \\ \cosh a&\sinh a\end{pmatrix}
\qquad\Rightarrow\qquad
\det A=\sinh^2 a - \cosh^2 a = -1<0,
\]
die Differentialgleichung ist also hyperbolisch.
Das Problem ist gut gestellt, wenn die von der $y$-Achse ausgehenden
Charakteristiken das ganze Gebiet "uberdecken. 
Dazu muss die Differentialgleichung der Charakteristiken gel"ost werden,
sie lautet
\[
\dot y^2
\sinh a
-2
\dot x\dot y
\cosh a
+
\dot x^2
\sinh a
=0.
\]
Dividieren wir durch $\sinh a $ und $\dot x^2$, erhalten wir
\begin{equation}
\biggl(\frac{\dot y}{\dot x}\biggr)^2
-2\frac{\dot y}{\dot x} \coth a + 1=0
\label{90000013:qgl}
\end{equation}
Der Quotient $\dot y/\dot x$ ist die Steigung $y'$ der Charakteristiken.
Die quadratische Gleichung~(\ref{90000013:qgl}) hat die Nullstellen
\[
\frac{\dot y}{\dot x}=y'=
\coth a\pm\sqrt{\coth^2 a-1}
.
\]
Die Charakteristiken sind also Geraden mit Steigungen
$m_{\pm}=\coth a\pm\sqrt{\coth^2a-1}$.
Da $\coth a>1$ ist f"ur $a>0$, sind beide Steigungen positiv.
In Abbildung~\ref{90000013:domain} sind die Charakteristiken in den Farben
rot und gr"un dargestellt, wobei die rote Gerade die Steigung $m_+$ hst.
Damit ist klar, dass die Charakteristiken ganz $\Omega$ "uberdecken, das
Problem ist also gut gestellt.

Alternativ kann man auch argumentieren, dass die $y$-Achse keine 
Charakteristik ist, denn dieses Cauchy-Problem w"are genau dann
nicht gut gestellt, wenn die Anfangskurve eine Charakteristik w"are.
\item
Die Charakteristiken sind Geraden mit positiver Steigung.
Die numerische L"osung gibt im verlangten Rechteck die L"osung
korrekt wieder, wenn die von $(0,-5)$ ausgehenden Charakteristiken
das Rechteck nicht schneiden.
Die Steigung von $(0,-5)$ zur rechten unteren Ecke des Rechtecks $R$
ist $2$, 
man muss also herausfinden, welches $a$ auf die diese Steigung f"uhrt.
Dies liefert die Gleichung
\begin{align*}
\coth a + \sqrt{\coth^2 a-1}&=2
\\
\coth^2 a-1&=(2-\coth a)^2=4-4\coth a +\coth^2 a
\\
-1 &=4-4\coth a
\\
\coth a&=\frac54
\qquad\Rightarrow\qquad
a=1.0986
\end{align*}
Die numerische L"osung kann also nur f"ur $a>1.0986$ die wahre L"osung in
$R$ korrekt wiedergeben.
\qedhere
\end{teilaufgaben}
\end{loesung}
