Auf dem Gebiet $\Omega=\{(x,y)\,|\, x>0, y>0\}$ soll die partielle
Differentialgleichung
\begin{equation}
\frac{\partial^2 u}{\partial x^2}-\frac1{y^2}\frac{\partial^2 u}{\partial y^2} 
+
\frac1{y^3}\frac{\partial u}{\partial y}=0
\label{90000011:gleichung}
\end{equation}
gelöst werden mit Randwertvorgaben für $x=0$, also auf dem linken Rand von
$\Omega$.
Hat eine Änderung der Randbedingungen für $x=0$ und $2 < y<3$ einen
Einfluss auf den Wert der Lösung in den Punkten
$P_1=(1,1)$,
$P_2=(1.5,3.5)$ und
$P_3=(3,3.85)$?

\begin{hinweis}
\[
y\dot y=\frac{d}{dt}\bigl(
{\textstyle \frac12}y^2
\bigr)
\]
\end{hinweis}

\begin{loesung}
Die Symbolmatrix dieser linearen partiellen Differentialgleichung zweiter
Ordnung ist
\[
A=\begin{pmatrix}
1&0\\
0&-\frac1{y^2}
\end{pmatrix}
\qquad
\Rightarrow
\qquad
\det(A)=-\frac1{y^2}<0
\]
in $\Omega$, es handelt sich also um eine hyperbolische partielle
Differentialgleichung, bei der mit Hilfe der Charakteristiken entschieden
werden kann, ob die beschriebene Änderung der Randbedingungen eine 
Auswirkung auf die den Wert der Lösung in den gegebenen Punkten hat.
\begin{figure}
\begin{center}
\includeagraphics[]{domain-1.pdf}
\end{center}
\caption{Charakteristiken für die hyperbolische partielle
Differentialgleichung \eqref{90000011:gleichung}.
Blau hervorgehoben das Gebiet, in dem sich der Wert der Lösungsfunktion
ändern wird, wenn die Anfangswerte für $x=0$ und $2<y<3$ sich ändern.
\label{90000011:bild}}
\end{figure}

Die Differentialgleichung der Charakteristiken ist
\begin{align*}
\dot y^2-\frac{1}{y^2}\dot x^2&=0,\\
\Rightarrow\qquad
\biggl(\dot y-\frac1y\dot x\biggr)\biggl(\dot y +\frac1y\dot x\biggr)=0.
\end{align*}
Das Produkt verschwindet, wenn einer der Faktoren verschwindet, wir erhalten
also die zwei Differentialgleichungen
\begin{align*}
\dot y&=\pm \frac1y \dot x
\end{align*}
oder
\begin{align*}
y\dot y&=\pm \dot x
\\
\frac{d}{dt}\biggl(\frac12y^2\biggr)&=\pm\frac{d}{dt} x
\end{align*}
mit den Lösungen
\[
\frac12y^2=\pm x + C
\]
oder aufgelöst nach $y$
\begin{equation}
y=\sqrt{D\pm 2x}.
\end{equation}
Die Charakteristiken durch den Punkt $(x_0,y_0)$ haben daher die Gleichungen
\[
y_\pm=\sqrt{y_0^2\pm 2(x-x_0)}.
\]
Eine Änderung der Randbedingung für $x=0$ und $2<y<3$ wirkt sich 
in dem blau eingezeichneten Gebiet in Abbildung~\ref{90000011:bild} aus.
Es ist also zu entscheiden, ob die genannten Punkte im Inneren dieses
Gebietes liegen. Die obere Begrenzung des Gebietes ist die
Kurve
\[
y_{\text{oben}} = \sqrt{9+2x},
\]
die untere ist
\[
y_{\text{unten}} = \sqrt{4-2x}.
\]
Einsetzen der Koordinaten der Punkte liefert
\begin{center}
\begin{tabular}{|c|c|ccccc|}
\hline
Punkt&$x$&$y_{\text{unten}}$&               &$y$ &              & $y_{\text{oben}}$\\
\hline
$P_1$&1  &$\sqrt{3}  $      &$\color{red}>$ & 1  &     $<$      & $\sqrt{11}=3.31$ \\
$P_2$&1.5&$\sqrt{2.5}$      &               &3.5 &$\color{red}>$& $\sqrt{12}=3.46$ \\
$P_3$&3  &                  &               &3.85&     $<$      & $\sqrt{15}=3.87$ \\
\hline
\end{tabular}
\end{center}
Daraus kann man ablesen, dass die Punkte $P_1$ und $P_2$ nicht im Einflussbereich
der Randbedingungsänderung liegen, der Punkt $P_3$ aber schon.
\end{loesung}

\begin{bewertung}
Symbolmatrix ({\bf A}) 1 Punkt,
hyperbolische partielle Differentialgleichung ({\bf H}) 1 Punkt,
Differentialgleichung der Charakteristiken ({\bf D}) 1 Punkt,
Zwei Lösungsscharen für die Charakteristiken ({\bf L}) 2 Punkte,
Einfluss auf Lösung ({\bf E}) 1 Punkt.
\end{bewertung}



