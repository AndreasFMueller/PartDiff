Auf dem Gebiet $\Omega = \{(x,y)\,|\, x>0,y>0\}$ soll die partielle
Differentialgleichung
\begin{equation}
\frac1{x^2}\frac{\partial^2u}{\partial x^2}
-\frac{\partial^2 u}{\partial y^2}+\frac1{y^3}\frac{\partial u}{\partial y}=0
\label{90000012:equation}
\end{equation}
gel"ost werden mit Randwertvorgaben f"ur $y=0$, also am unteren Rand von
$\Omega$.
Hat eine "Anderung der Randbedingungen f"ur $y=0$ und $2<x<3$ einen Einfluss
auf den Wert der L"osung in den Punkten 
$P_1=(1,1)$, $P_2=(3.5,1.5)$ und $P_3=(3.85,3)$?

\begin{hinweis}
\[
x\dot x=\frac{d}{dt}\bigl(
{\textstyle \frac12}x^2
\bigr)
\]
\end{hinweis}

\begin{loesung}
Die Symbolmatrix dieser linearen partiellen Differentialgleichung zweiter
Ordnung ist
\[
A=\begin{pmatrix}
\frac1{x^2}&0\\0&-1
\end{pmatrix}
\qquad
\Rightarrow
\qquad
\det(A)=-\frac1{x^2}<0,
\]
in $\Omega$, es handelt sich also um eine hyperbolische partielle 
Differentialgleichung, bei der mit Hilfe der Charakteristiken
entschieden werden kann, ob die beschriebene "Anderung der Randbedingungen
einen Auswirkung auf die L"osung in den Punkten $P_i$ hat.

Die Differentialgleichung der Charakteristiken ist
\begin{align*}
\frac1{x^2}\dot y^2-\dot x^2&=0
\\
\biggl(\frac1x \dot y-\dot x\biggr)
\biggl(\frac1x \dot y+\dot x\biggr)
&=0.
\end{align*}
Das Produkt verschwindet, wenn einer der Faktoren verschwindet, wir erhalten
also die zwei Differentialgleichungen
\[
\frac1x\dot y=\pm\dot x
\qquad
\Rightarrow
\qquad
\dot y=\pm x\dot x
\]
Nach dem Hinweis ist dies gleichbedeutend mit
\[
\frac{d}{dt}y
=
\pm
\frac{d}{dt}
\biggl(
\frac12
x^2
\biggr),
\]
woraus durch Integration 
\[
y=\pm\frac12x^2+C
\]
wird.
Die Charakteristiken sind also nach oben oder unten ge"offnete Parabeln,
die die $y$-Achse im Punkt $y=C$ schneiden.

Das von einer "Anderung der Anfangsbedingungen betroffene Gebiet wird
also begrenzt durch Parabeln, die durch die Punkte $(2,0)$ und $(3,0)$
verlaufen.
Nach links begrenzt wird das Gebiet also durch die nach unten ge"offnete
Parabel
\[
y_{\text{links}}=-\frac12x^2+2,
\]
und nach rechts durch die nach oben ge"offnete Parabel
\[
y_{\text{rechts}}=\frac12x^2-\frac{9}{2}.
\]
Ein Punkt ist in diesem Gebiet, wenn seine $y$-Koordinate
gr"osser sind als die $y$-Werte beider Begrenzungen.
Einsetzen der Koordinaten der Punkte ergibt
\begin{center}
\begin{tabular}{
|>{$}c<{$}|>{$}c<{$}|
>{$}c<{$}|
%>{$}c<{$}
>{$}c<{$}
%>{$}c<{$}
>{$}c<{$}|}
\hline
\text{Punkt}&x&y&y_{\text{links}}&y_{\text{rechts}}\\
\hline
P_1&1   & 1   & \color{red}1.5      &   -4 \\
P_2&3.5 & 1.5 & -4.125   &  \color{red}  1.625 \\
P_3&3.85& 3   &-5.41125 &    2.91125\\
\hline
\end{tabular}
\end{center}
(Werte der Begrenzungsfunktion, die die Bedinung verletzen, sind rot
hervorgehoben).
Nur der Punkt $P_3$ ist im Gebiet, $P_1$ und $P_2$ liegen knapp ausserhalb,
wie auch die Abbildung~\ref{90000012:char} zeigt.
\begin{figure}
\centering
\includeagraphics[]{domain-1.pdf}
\caption{Charakteristiken der hyperbolischen partiellen Differentialgleichung
(\ref{90000012:equation}) und Einflussgebiet einer "Anderung
der Randbedingungen.
\label{90000012:char}}
\end{figure}
\end{loesung}

\begin{bewertung}
Symbolmatrix ({\bf A}) 1 Punkt,
hyperbolische partielle Differentialgleichung ({\bf H}) 1 Punkt,
Differentialgleichung der Charakteristiken ({\bf D}) 1 Punkt,
Zwei L"osungsscharen f"ur die Charakteristiken ({\bf L}) 2 Punkte,
Einfluss auf L"osung ({\bf E}) 1 Punkt.
\end{bewertung}



