The Fokker-Planck differential equation is
\[
\frac{\partial u}{\partial t}
=
-\frac{\partial u}{\partial x}+\frac{\partial^2u}{\partial x^2}.
\]
We want to study it on the domain
\[
\Omega=\{(x,t)\,|\,  t \ge 0\}.
\]
\begin{teilaufgaben}
\item
Is this equation elliptic, parabolic or hyperbolic?
\item
Use separation to find two ordinary differential equations by which
you can then find partial solutions of the Fokker-Planck equation.
\item
Find all partial solutions for which $u(x,t)$ increases without bounds
for $t\to\infty$ and certain $x$ and which in addition satisfy
$u(t,0)=0$.
\end{teilaufgaben}

\begin{diskussion}
The Fockker-Planck equation can be used to describe the time evolution
of a probability distribution subject to diffusion and drift.
The solutions we are studying in this problem are precisely not those that
represent probability distributions, as we are looking for solutions
that increase without bounds.
The equation does allow such solutions, it is the choice of
boundary conditions that makes the difference.
\end{diskussion}

\begin{loesung}
\begin{teilaufgaben}
\item
For the symbol matrix, only the second order derivatives are
used, so it is
\[
\begin{pmatrix}
1&0\\
0&0
\end{pmatrix}
\]
with determinant $0$.
It follows that the equation is parabolic.
\item
In order to separate variables, we use the ansatz
\[
u(x,t)=X(x)T(t)
\]
and substitute it into the differential equation:
\begin{align*}
X(x)T'(t)&=-X'(x)T(t)+X''(x)T(t)\\
\frac{T'(t)}{T(t)}&=\frac{-X'(x)+X''(x)}{X(x)}.
\end{align*}
As usual, division by $T(t)X(X)$ is allowed as this expression only
vanishes on a small set.
The left hand side depends only on $t$, the right only on $x$, so both
sides must be constant.
We call this constant $\mu$.
In this way we obtain the two ordinary differential equations
\begin{align}
\frac{T'(t)}{T(t)}&=\mu&&\Rightarrow&T'(t)&=\mu T(t)
\label{80000001:Tgleichung}
\\
\frac{-X'(x)+X''(x)}{X(x)}&=\mu&&\Rightarrow&X''(x)-X'(x)-\mu X(x)&=0
\label{80000001:Xgleichung}
\end{align}
\item
The equation
\eqref{80000001:Tgleichung}
can easily be solved using the exponential function, the solution is
\[
T(t)=T(0)e^{\mu t}.
\]
This factor only grows without bounds for $t\to\infty$ if $\mu>0$.
Thus we are looking for solutions of \eqref{80000001:Xgleichung} for
positive $\mu$.

The ordinary differential equation of second order for $X(x)$ can
be solved in the usual way using the characteristic polynomial,
which in this case is
\[
\lambda^2-\lambda-\mu=0
\quad
\Rightarrow
\quad
\lambda_{\pm}(\mu)=\frac12\pm\sqrt{\frac14+\mu}.
\]
As $\mu>0$, both zeros $\lambda_{\pm}(\mu)$ of the characteristic
equation will always be real.
This leads to the general solution
\[
X(x)=
A(\mu)e^{\lambda_+(\mu)x}
+
B(\mu)e^{\lambda_-(\mu)x}.
\]
For $x=0$, the solution is expected to vanish, i.~e.
\[
X(0)=
A(\mu)e^{\lambda_+(\mu)\cdot 0}
+
B(\mu)e^{\lambda_-(\mu)\cdot 0}
=A(\mu)+B(\mu)=0
\]
or $B(\mu)=-A(\mu)$.
Thus
\begin{align*}
X(x)&=A(e^{\lambda_+(\mu)x}-e^{\lambda_-(\mu)x})
=Ae^{\frac12x}\left(
e^{x\sqrt{\frac14+\mu}}
-
e^{-x\sqrt{\frac14+\mu}}
\right)
\\
&=2Ae^{\frac12x}\sinh x\sqrt{\frac14+\mu}.
\end{align*}
Together with the solution $T(t)$ already found above we get the
partial solutions
\[
u(x,t,\mu)=
Ce^{\frac12x+\mu t}\sinh x\sqrt{\frac14+\mu}.
\]
{\bf Verification:}
To verify our solution, we substitute the function $u(x,t,\mu)$
into the partial differential equation and compute
\begin{align*}
\frac{\partial}{\partial t}u(x,t,\mu)&=
\mu Ce^{\frac12x+\mu t}\sinh x\sqrt{\frac14+\mu}
\\
\frac{\partial}{\partial x}u(x,t,\mu)&=
\frac12
Ce^{\frac12x+\mu t}\sinh x\sqrt{\frac14+\mu}
+
C\sqrt{\frac14+\mu}e^{\frac12x+\mu t}\cosh x\sqrt{\frac14+\mu}
\\
\frac{\partial^2}{\partial x^2}u(x,t,\mu)&=
\frac14
Ce^{\frac12x+\mu t}\sinh x\sqrt{\frac14+\mu}
+
C\sqrt{\frac14+\mu}e^{\frac12x+\mu t}\cosh x\sqrt{\frac14+\mu}
\\
&\qquad +
C\biggl(\frac14+\mu\biggr)e^{\frac12x+\mu t}\sinh x\sqrt{\frac14+\mu}
\\
-\frac{\partial u}{\partial x}
+
\frac{\partial^2 u}{\partial x^2}
&=
\mu Ce^{\frac12x+\mu t}\sinh x\sqrt{\frac14+\mu}
=
\frac{\partial u}{\partial t}.
\end{align*}
This shows that the function satisfies the partial differential
equation in the domain.
Since $\sinh 0=0$, the boundary conditions hold trivially.
\qedhere
\end{teilaufgaben}
\end{loesung}
