Die Fokker-Planck Differentialgleichung lautet
\[
\frac{\partial u}{\partial t}
=
-\frac{\partial u}{\partial x}+\frac{\partial^2u}{\partial x^2},
\]
wir wollen Sie auf dem Gebiet
\[
\Omega=\{(x,t)\,|\,  t \ge 0\}
\]
studieren.
\begin{teilaufgaben}
\item Ist diese Gleichung elliptisch, parabolisch oder hyperbolisch?
\item Führen Sie Separation durch und finden Sie zwei gewöhnliche
Differentialgleichungen, mit denen Sie Teillösungen für
die Fokker-Planck-Gleichung finden können.
\item Finden Sie alle Teillösungen, für die $u(x,t)$ für $t\to\infty$
und gewisse $x$ unbeschränkt wächst, und für die ausserdem $u(t,0)=0$ gilt.
\end{teilaufgaben}

\begin{loesung}
\begin{teilaufgaben}
\item Für die Symbolmatrix sind nur die zweiten Ableitungen massgebend,
daher ist sie
\[
\begin{pmatrix}
1&0\\
0&0
\end{pmatrix}
\]
mit Determinante $0$, also ist die Gleichung parabolisch.
Insbesondere erwarten wir ein Verhalten der Lösungen wie
bei der Wärmeleitungsgleichung.
\item Für die Separation machen wir den Ansatz
\[
u(x,t)=X(x)T(t)
\]
und setzen diesen in die Differentialgleichung ein:
\begin{align*}
X(x)T'(t)&=-X'(x)T(t)+X''(x)T(t)\\
\frac{T'(t)}{T(t)}&=\frac{-X'(x)+X''(x)}{X(x)}
\end{align*}
wobei die Division durch $T(t)X(X)$ erlaubt ist, weil
dieser Ausdruck nur auf einer kleinen Menge verschwindet.
Da die linke Seite nur von $t$, die rechte aber nur von $x$ abhängt,
müssen beide Seiten konstant sein, wir nennen die Konstante
$\mu$. So erhalten wir die zwei gewöhnlichen Differentialgleichungen
\begin{align}
\frac{T'(t)}{T(t)}&=\mu&&\Rightarrow&T'(t)&=\mu T(t)
\label{80000001:Tgleichung}
\\
\frac{-X'(x)+X''(x)}{X(x)}&=\mu&&\Rightarrow&X''(x)-X'(x)-\mu X(x)&=0
\label{80000001:Xgleichung}
\end{align}
\item
Die Gleichung \eqref{80000001:Tgleichung} kann leicht mit der Exponentialfunktion
gelöst werden:
\[
T(t)=T(0)e^{\mu t}.
\]
Dieser Faktor kann nur dann für $t\to\infty$ unbeschränkt wachsen, wenn
$\mu > 0$. Wir suchen also Lösungen von \eqref{80000001:Xgleichung} für
positives $\mu$.
Diese gewöhnliche lineare Differentialgleichung zweiter Ordnung wird
wie üblich mit Hilfe des charakteristischen Polynoms gelöst, welches
in diesem Fall lautet
\[
\lambda^2-\lambda-\mu=0
\quad
\Rightarrow
\quad
\lambda_{\pm}(\mu)=\frac12\pm\sqrt{\frac14+\mu}
\]
Da $\mu > 0$ ist, wird $\lambda_{\pm}(\mu)$ immer reell sein.
Damit ist die allgmeine Lösung für $X(x)$
\[
X(x)=
A(\mu)e^{\lambda_+(\mu)x}
+
B(\mu)e^{\lambda_+(\mu)x}
\]
Sie soll für $x=0$ verschwinden, also
\[
X(0)=
A(\mu)e^{\lambda_+(\mu)\cdot 0}
+
B(\mu)e^{\lambda_+(\mu)\cdot 0}
=A(\mu)+B(\mu)=0
\]
oder $B(\mu)=-A(\mu)$. Damit haben wir
\begin{align*}
X(x)&=A(e^{\lambda_+(\mu)x}-e^{\lambda_-(\mu)x})
=Ae^{\frac12x}\left(
e^{x\sqrt{\frac14+\mu}}
-
e^{-x\sqrt{\frac14+\mu}}
\right)
\\
&=2Ae^{\frac12x}\sinh x\sqrt{\frac14+\mu}.
\end{align*}
Zusammen mit der bereits gefundenen Lösung für $T(t)$ sind also
die Teillösungen
\[
u(x,t,\mu)=
Ce^{\frac12x+\mu t}\sinh x\sqrt{\frac14+\mu}.
\]
{\bf Kontrolle:} Zur Kontrolle setzen wir die Lösung $u(x,t,\mu)$
in die Differentialgleichung ein
\begin{align*}
\frac{\partial}{\partial t}u(x,t,\mu)&=
\mu Ce^{\frac12x+\mu t}\sinh x\sqrt{\frac14+\mu}
\\
\frac{\partial}{\partial x}u(x,t,\mu)&=
\frac12
Ce^{\frac12x+\mu t}\sinh x\sqrt{\frac14+\mu}
+
C\sqrt{\frac14+\mu}e^{\frac12x+\mu t}\cosh x\sqrt{\frac14+\mu}
\\
\frac{\partial^2}{\partial x^2}u(x,t,\mu)&=
\frac14
Ce^{\frac12x+\mu t}\sinh x\sqrt{\frac14+\mu}
+
C\sqrt{\frac14+\mu}e^{\frac12x+\mu t}\cosh x\sqrt{\frac14+\mu}
\\
&\qquad +
C\biggl(\frac14+\mu\biggr)e^{\frac12x+\mu t}\sinh x\sqrt{\frac14+\mu}
\\
-\frac{\partial u}{\partial x}
+
\frac{\partial^2 u}{\partial x^2}
&=
\mu Ce^{\frac12x+\mu t}\sinh x\sqrt{\frac14+\mu}
=
\frac{\partial u}{\partial t},
\end{align*}
die Funktionen erfüllen also die partielle Differentialgleichung.
Wegen $\sinh 0=0$ ist auch die Randbedingung trivial erfüllt.
\qedhere
\end{teilaufgaben}
\end{loesung}
