Stellen Sie die Lagrange-Charpits-Gleichungen f"ur die
die partielle Differentialgleichung
\[
4u\frac{\partial u}{\partial x}=\biggl(\frac{\partial u}{\partial y}\biggr)^2,
\]
auf. L"osen Sie sie soweit m"oglich.
% Randwerte: $u=1$ auf $x+2y=2$

\ifthenelse{\boolean{loesungen}}{
\begin{loesung}
Wir verwenden die Methode der Charakteristiken, dazu brauchen wir die
Funktion $F$ der Differentialgleichung:
\[
F(x,y,u,p,q)=4up-q^2.
\]
Die Ableitungen davon sind
\begin{align*}
\partial_x F&=0\\
\partial_y F&=0\\
\partial_u F&=4p\\
\partial_p F&=4u\\
\partial_q F&=-2q
\end{align*}
Die Lagrange-Charpit-Gleichungen werden damit zu
\begin{align*}
x'&=\partial_pF=4u\\
y'&=\partial_qF=-2q\\
u'&=p\partial_pF+q\partial_qF=4up-2q^2\\
p'&=-\partial_xF-p\partial_uF=-4p^2\\
q'&=-\partial_yF-q\partial_uF=-4pq
\end{align*}

Die vierte Differentialgleichung l"asst sich durch Separation integrieren:
\begin{align*}
\int \frac{dp}{p^2}=-\frac1{p}&=-4s+p_0
\end{align*}
oder 
\[
p=\frac1{4s-p_0}.
\]
Wenn aber $p$ bekannt ist, dann liefert die Differentialgleichung
f"ur $q$:
\begin{align*}
q'&=-4pq
\\
\int \frac{q'}{q}\,ds&=-\int 4p\,ds=\int\frac{-4}{p_0-4s}\,ds
\\
\log(q)&=-\log(p_0-4s) + C\\
q&=\frac{q_0}{p_0-4s}=-q_0p
\end{align*}
Da jetzt $q$ bekannt ist, kann auch $y$ bestimmt werden
\begin{align*}
y'&=-2q=\frac{2q_0}{p_0-4s}\\
y&= \frac{q_0}2\log(p_0-4s)+y_0
\end{align*}
Mit $p$ und $q$ kann man jetzt die Differentialgleichung f"ur
$u$ vereinfachen:
\begin{align*}
u'-4pu=2q^2
\end{align*}
ist eine inhomogene lineare Differntialgleichung erster Ordnung,
f"ur die es ein Standardverfahren gibt, Variation der Konstanten:
Die allgemeine L"osung ist
\[
u(s)=\frac{q_0^3}{4(p_0-4s)} + u_0(4s-p_0).
\]
Mit der ersten Differentialgleichung l"asst sich jetzt auch $x$
bestimmen:
\[
x=\int 4u\,ds=
\frac{1}{2} u_0 (p_0-4 s)^2-\frac{1}{4} q_0^3 \log(p_0-4 s).
\]
\end{loesung}
}{}
