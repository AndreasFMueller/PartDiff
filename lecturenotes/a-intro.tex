%
% a-einleitung.tex
%
% (c) 2008 Prof Dr Andreas Mueller, Hochschule Rapperswil
%
\lhead{Introduction}
\chapter*{Introduction}
\index{Mechanik!klassische}
\index{Elektrizitatslehre@Elektrizit\"atslehre}
\index{Populationsdynamik}
Gew"ohnliche Differentialgleichungen werden in der klassischen Mechanik,
in der Elektrizit"atslehre, in der Populationsdynamik, und ganz 
allgemein "uberall dort erfolgreich eingesetzt, wo die
"Anderungsrate einer skalaren Gr"osse von den aktuellen Werten dieser Gr"osse
abh"angt.
\index{Massepunkt}
\index{Newtonsches Gesetz}
Ein Massepunkt bei der Ortskoordinate $x$ "andert seine Position
gem"ass dem Newtonschen Gesetz
\[
ma=m\ddot x=F.
\]
H"angt $F$ von der aktuellen Position des Massepunktes ab, wie dies zum
Beispiel bei einem an einer Feder aufgeh"angten Gewicht der Fall ist,
liegt eine Differentialgleichung f"ur die von der Zeit abh"angige
Position $x(t)$ des Massepunktes vor:
\[
m\frac{d^2}{dt^2}x(t)=F(x(t)).
\]
In der klassischen Analysis wird gezeigt, dass solche Gleichungen bei
gegebenen Anfangsbedingungen eine eindeutig bestimmte L"osung haben.
Mit etwas zus"atzlichem Aufwand werden auch Differentialgleichungen
f"ur mehrere skalare Gr"ossen behandelt, die alle vom gleichen Parameter
(typischerweise der Zeit) abh"angen.

\index{Felder!elektrische}
\index{Temperaturverteilung}
\index{Druck}
\index{Gas}
\index{Stromungsgeschwindigkeit@Str\"omungsgeschwindigkeit}
\index{Flussigkeit@Fl\"ussigkeit}
Zur Beschreibung elektrischer Felder, der Temperaturverteilung in
einem Fest\-k"orper, des Druckes in einem Gas oder der Str"omungsgeschwindigkeit
in einer Fl"ussigkeit reicht eine einzelne oder einige wenige skalare
Gr"ossen jedoch nicht. Vielmehr muss man in den genannten Beispielen
zu jedem Zeitpunkt und an jedem Punkt des Raumes die interessierende Gr"osse
bestimmen. Sie ist also nicht mehr nur eine Funktion der Zeit, sondern eine
Funktion von Ort und Zeit.
\index{Saite!schwingende}
Die Auslenkung einer schwingenden Saite
beispielsweise ist eine Funktion $f(x,t)$, sie beschreibt, wie stark
die Saite an der Koordinate $x$ zur Zeit $t$ von der Ruhelage ausgelenkt ist.
Insbesondere hat die Saite zu jeder Zeit $t_0$ eine andere Form, die durch
die Funktion von einer Variablen
\[
x\mapsto f(x,t_0)
\]
gegeben ist.

Man k"onnte vielleicht die Hoffnung hegen, dass die zeitliche "Anderung
der Auslenkung an der Stelle $x$ nur von der Position $x$ abh"angt.
Dies kann aber nicht sein, denn wenn die Saite in einer Umgebung
des Punktes $x$ sehr stark schwankt, ``ziehen'' die ``benachbarten'' Teile
der Saite ebenfalls an dem Punkt, vor allem dann, wenn die Saite gekr"ummt
ist. Die zeitliche "Anderung der Auslenkung, also die Ableitung von $f(x,t)$
nach der Zeit, h"angt also nicht nur von $f(x,t)$ ab, sondern auch von
den Ableitungen nach $x$. Es entsteht eine Gleichung, die die verschiedenen
partiellen Ableitungen von $f$ miteinander verkn"upft, zum Beispiel in
etwa folgender Form:
\begin{align*}
a^2\frac{\partial^2}{\partial x^2}f&= \frac{\partial^2}{\partial t^2}f
\end{align*}
Kann man diese sogenannte partielle Differentialgleichung l"osen,
kann man die Form der Saite zu jedem
beliebigen zuk"unftigen Zeitpunkt voraussagen.
\index{Differentialgleichung!partielle}

\index{Differentialgleichung!gew\"ohnliche}
Bei gew"ohnlichen Differentialgleichungen hat man gelernt, dass die Gleichung
alleine die L"osung noch nicht eindeutig bestimmt, sondern dass dazu noch
weitere Angaben, sogenannte Anfangs- oder Randbedingungen n"otig sind.
Die Theorie der partiellen Differentialgleichungen muss daher auf folgende
Fragen eine Antwort liefern:
\begin{enumerate}
\item Wie muss ein Problem, welches mit Hilfe partieller Differentialgleichungen
gel"ost werden soll, "uberhaupt gestellt werden, damit die L"osung wohlbestimmt
ist?
\item Welche Eigenschaften haben die L"osungen? Wieviele Ableitungen haben
die L"osungen? Wie sieht die L"osung in der N"ahe des Randes aus, bleiben
sie stetig, oder werden Sie unbegrenzt gross?
\item Wie berechnet man die L"osung einer partiellen Differentialgleichung?
\end{enumerate}
Die Vorlesung ``Partielle Differentialgleichungen'' im Master of Sciences in
Engineering versucht diese Fragen zu beantworten. Der Kurs ist in zwei Teile
geteilt. Im ersten Teil werden die genannten Fragen theoretisch untersucht.
Leider stellt sich heraus, dass man zwar klare Aussagen dar"uber machen kann,
welche Gleichungen l"osbar sind, und wieviele L"osungen gegebenenfalls
existieren, dass man auch deren Eigenschaften sehr gut beschreiben kann,
aber leider auch nur in ganz wenigen F"allen geschlossene Formeln f"ur die
L"osungen finden kann. Mit der mathematischen Gewissheit, dass L"osungen
existieren und sich ``anst"andig benehmen'' er"offnet sich aber auch die
im zweiten Teil behandelte M"oglichkeit, die Gleichungen mit Hilfe eines
Computers numerisch zu l"osen.

Dieses Skript enth"alt die im ersten Teil der Vorlesung behandelten 
Themen. Es gliedert sich in folgende Kapitel:
\begin{enumerate}
\item Beispiele von partiellen Differentialgleichungen. Die wichtigsten
Differentialgleichungen aus Mechanik, W"armelehre und Elektrizit"atslehre
werden vorgestellt. Sie dienen in sp"ateren Kapiteln als Beispiele
f"ur theoretische "Uberlegungen wie auch f"ur L"osungsmethoden.
\item L"osung mit Separation der Variablen.
\index{Separation der Variablen}
\item L"osung mit Hilfe von Integraltransformationen
\index{Integraltransformation}
\item Elliptische partielle Differentialgleichungen
\index{Differentialgleichung!partielle!elliptische}
\item Parabolische partielle Differentialgleichungen
\index{Differentialgleichung!partielle!parabolische}
\item Hyperbolische partielle Differentialgleichungen
\index{Differentialgleichung!partielle!hyperbolische}
\item Nichtlineare Ph"anomene am Beispiel der Gleichung von Burgers.
\index{Differentialgleichung!partielle!nichtlineare}
\end{enumerate}


