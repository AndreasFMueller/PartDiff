%
% summary.tex -- XXX
%
% (c) 2019 Prof Dr Andreas Mueller
%
\section{The Separation Algorithm}
\rhead{Separation algorithm}
From these examples, we can derive a more general description of the
separation method.
A partial differential equation with independent variables
$x_1,\dots,x_n$ is to be decomposed into two
differential equations with fewer variables.
This decomposition is repeated until only one variable remains,
at which point we have an ordinary differential equation.
Some of the boundary conditions become boundary conditions for
these simpler differential equations.

In order to be able to satisfy the remaining boundary conditions,
we need to be able to form linear combinations, which only works
for linear partial differential equations.
In the following, we therefore assume that the partial differential
equation is linear.

We want to separate the variable $x_1$ in a partial differential
equation with independent variables $x_1,\dots,x_n$.
We proceed as follows.
\begin{enumerate}
\item
Choose an ansatz for the solution $u$ of the partial differential equation
in the form of a product:
\[
u(x_1,\dots,x_n)=X_1(x_1)u_1(x_2,\dots,x_n).
\]
\item
Substitute the ansatz into the differential equation.
\item
With some luck, the terms involving $x_1$ can be brought to
left hand side and all terms involving the other variables to
the right hand side.
It is acceptable to divide by $u$ because we assume that $u$ is not
identically $u$ and such a solution will only vanish on a set of
volume $0$, everywhere else it will satisfy the equation.
The equation now has the form
\[
F(x_1,X_1,X_1',\dots,X_1^{(n)})
=
G(x_2,\dots,x_n,u_1,\partial_2u_1,\dots\partial_nu_n,\dots).
\]
\item
Since the left hand side only depends on $x_1$, the right hand side
only on $x_2,\dots,x_n$, both sides must be constant.
This splits the equation into two simpler equations
\begin{equation}
\begin{aligned}
F(x_1, X_1,X_1',\dots, X_1^{(n)})&=k\\
G(x_2,\dots,x_n,u_1,\partial_2u_1,\dots\partial_nu_n,\dots)&=k,
\end{aligned}
\label{separiert}
\end{equation}
where $k$ is a new constant.
Some homogeneous boundary conditions may be used at this point to
restrict the set of valid $k$ values.

The first of these equations is an ordinary differential equation.
The second is an ordinary differential equation if $n=2$, otherwise
the process can be repeated to break the equation down further until
only ordinary differential equations remain.

\item
If
$X_1(k,x_1)$ and $u_1(k,x_2,\dots,x_n)$ are solutions of \eqref{separiert},
then
\[
u_k(x_1,\dots,x_n)=X_1(k,x_1)u_1(k,x_2,\dots,x_n)
\]
are solutions of the original partial differential equation.
The general solution then is the linear combination
\[
u(x_1,\dots,x_n)=
\sum_{k}
a_k
u_k(x_1,\dots,x_n)=X_1(k,x_1)u_1(k,x_2,\dots,x_n),
\]
where the sum is taken over all possible $k$ values.
\item
To satisfy the boundary conditions, the coefficients $a_k$ now need to
be determined.
\end{enumerate}
The method may break down in the following places:
\begin{itemize}
\item
In step 3 it is assumed that separation is possible.
This is not automatically guaranteed, but in many practical cases it
is possible to force separability by choosing a suitable coordinate
system.
\item
In step 6 it is assumed that the boundary conditions can be
represented as a linear combination of the boundary values of
the functions $u_k$.
This may turn out to be a difficult problem, as indicated by the
fact that we have to summon the highly nontrivial Fourier theory
for our examples.
\end{itemize}

