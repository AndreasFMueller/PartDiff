%
% summary.tex -- XXX
%
% (c) 2019 Prof Dr Andreas Mueller
%
\section{Zusammenfassung: Separationsverfahren}
Aus diesen Beispielen lässt sich jetzt das allgemeine Prinzip 
ableiten. Gegeben ist eine partielle Differentialgleichung
beliebiger Ordnung mit unabhängigen Variablen $x_1,\dots,x_n$.
Ziel ist, die Differentialgleichung auf eine solche mit weniger
unabhängigen Variablen zu reduzieren. Sobald man die Reduktion
bis auf eine Variable geschafft hat, hat man die partielle
Differentialgleichung in gewöhnliche Differentialgleichungen
umgewandelt, typischerweise in Randwertprobleme,
die man mit gekannten Techniken lösen kann.

Da man am Schluss die Lösung aus den Teillösungen zusammensetzen
muss, die die separierten Gleichungen liefern, ist dieses Vorgehen
nur bei linearen PDGL sinnvoll. Wir gehen also im folgenden von
einer linearen PDGL aus.

Wir gehen also von einer Differentialgleichung für die Funktion
$u(x_1,\dots,x_n)$ aus, und wollen die Variable $x_1$ separieren.
Dazu geht man wie folgt vor.
\begin{enumerate}
\item Setzt die Lösung $u$ der Differentialgleichung in der
Form eines Produktes an:
\[
u(x_1,\dots,x_n)=X_1(x_1)u_1(x_2,\dots,x_n).
\]
\item Einsetzen des Ansatzes in die Differentialgleichung.
\item
Mit etwas Glück lassen sich die Terme, die
$X_1$ und $u_1$ enthalten trennen und auf verschiedene Seiten
des Gleichheitszeichens bringen.
Da die Lösung $u\equiv 0$ nicht interessant ist, kann man
zu diesem Zweck durch $u$ dividieren, die Gleichung muss
ausserhalb der Nullstellen von $u$ immer noch erfüllt sein.
Die Gleichung hat jetzt also die Form
\[
F(x_1,X_1,X_1',\dots,X_1^{(n)})
=
G(x_2,\dots,x_n,u_1,\partial_2u_1,\dots\partial_nu_n,\dots)
\]
\item
Da die linke Seite nur von $x_1$, die rechte nur von $x_2,\dots,x_n$
abhängt, müssen beide Konstant sein, wir haben also die ursprüngliche
PDGL in zwei Differentialgleichungen zerlegt:
\begin{equation}
\begin{aligned}
F(x_1, X_1,X_1',\dots, X_1^{(n)})&=k\\
G(x_2,\dots,x_n,u_1,\partial_2u_1,\dots\partial_nu_n,\dots)&=k
\end{aligned}
\label{separiert}
\end{equation}
wobei $k$ eine Konstante ist.
Dies sind zwei Differentialgleichungen, die erste ist eine
gewöhnliche Differntialgleichung, und falls $n>2$ ist die zweite
eine partielle Differentialgleichung, die unter Umständen noch
einmal mit dem gleichen Verfahren behandelt werden muss.
Gesucht werden alle Konstanten,
für welche beide Gleichungen eine Lösung haben.
\item Sind $X_1(k,x_1)$ und $u_1(k,x_2,\dots,x_n)$ Lösungen der
Gleichungen (\ref{separiert}), dann sind 
\[
u_k(x_1,\dots,x_n)=X_1(k,x_1)u_1(k,x_2,\dots,x_n)
\]
Lösungen der ursprünglichen PDGL. Die allgmeine Lösung ist daher
eine Summe
\[
u(x_1,\dots,x_n)=
\sum_{k}
a_k
u_k(x_1,\dots,x_n)=X_1(k,x_1)u_1(k,x_2,\dots,x_n),
\]
wobei die Summe über die möglichen $k$ zu erstrecken ist.
\item
Zur Erfüllung von Randbedingungen müssen jetzt die Koeffizienten
$a_k$ bestimmt werden, für die die Randtterme korrekt werden.
\end{enumerate}
Das Verfahren kann an zwei Stellen zusammenbrechen:
\begin{itemize}
\item In Schritt 3 wird vorausgesetzt, dass die Trennung in 
Terme, die $x_1$ enthalten  und solche, die $x_1$ nicht enthalten
möglich ist. Dies ist nicht automatisch der Fall, kann aber in
vielen praktisch wichtigen Fällen durch Wahl eines geeigneten
Koordinatensystems erreicht werden.
\item In Schritt 6 wird vorausgesetzt, dass die Randbedingungen
mit Hilfe der Randwerte der Teillösungen $u_k$ erfüllt werden
können. In den Beispielen in diesem Kapitel wurde dafür jeweils
die nicht triviale Fourier-Theorie benötigt. 
\end{itemize}

