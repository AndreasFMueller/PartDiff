%
% jacobi.tex -- Anwendung: Hamilton-Jacobi-Formulierung der Mechanik 
%
% (c) 2012 Prof Dr Andreas Mueller, Hochschule Rapperswil
%
\section{Application: Hamiltonian Mechanics\label{hamilton-mechanik}}
In the examples so far, a product was the ansatz of choice, because
that allowed to satisfy boundary conditions most easily.
This section illustrates a problem where a sum ansatz is very successful
because there are no boundary conditions to satisfy.
Another interesting feature of this application is that the
equation is simple enough that the first separation step is 
always successful.

\subsection{Motivation}
\index{Newton's laws}
\index{planets}
\index{satellites}
Newtons laws of motion predict the motions of planets and satellites
with supreme precision.
A body of mass $m$ with time dependent coordinates $\vec{x}(t)$
subject to a position dependent force $\vec{F}(\vec{c})$ follows a
trajectory that is a solution of the ordinary differential equation
\begin{equation}
m\frac{d^2}{dt^2}\vec x(t)=\vec F(\vec x(t)).
\label{jacobi:newton}
\end{equation}
Starting at the initial point $\vec{x}_0$ with initial velocity
$\vec{v}_0$, we can always find the trajectory at least numerically.

But this approach has some practically important drawbacks.
In many cases, the force law is not completely known.
Satellites in low earth orbit, no more than a few 100 kilometers above
the surface of the earth, are subject to drag by the very tenuous atmosphere.
Their trajectories decay over time, but this process is very difficult to model.
To this day it is very difficult to predict when a satellite will burn
up in the atmosphere and where it will crash on the ground.

Similarly, the perihelion of the orbit of planet mercury advances
ever so slightly, an effect that is not explained in newtonian mechanics
but predicted by Einstein's theory of relativity.
The equations of general relativity are excedingly complicated.

Both these problems illustrate the common situation where one knows
the most important effects and can even solve the equation in closed
form for these cases, but in reality there are small additional
forces that have to be considered, like drag or relativistic effects.
In this situation, it may be useful to express the take the solution
of the unperturbed system, which will undoubtedly contain some integration
constants, and vary those constants to model the perturbation forces.

\begin{beispiel}
As an example, let's consider a horizontal pool table.
The movement of the pool ball is determined by the initial
point $\vec{x}_0$ and the initial velocity $\vec{v}_0$.
The ball will continue to roll in direction ov $\vec{v}_0$, but
its velocity will slowly decrease until it stops.
The solution of the equations of motion thus is a function
$t\mapsto \vec x(t, \vec x_0, \vec v_0))$,
describing the trajectory for every conceivable initial contion.

What happens if we no longer assume the table to be horizontal?
There will be an additional force acting on the ball in the
direction of steepest descent of the table.
If the slope of the table is very small, then deviation from
$\vec x(t,\vec x_0,\vec v_0))$ will be very small.
If we didn't know about the additional force, then observation
of the motion at some particular point $t$ in time would let
as believe that the ball started in a different place and with
a different velocity.
We could compensate for the deviation by changing the parameters
$\vec x_0$ and $\vec v_0$ slightly.
So the solution on the slightly sloped table can be described by the
solution function of the horizontal table with slowly varying
initial conditions $\vec x_0(t)$ and $\vec v_0(t)$ to compensate
for the slope.
\end{beispiel}

The advantage of such a description is that perturbations can
be better understood as variations of the physically meaningful
parameters of the trajectory.
The trajectory computation stays simple, because we have a
formular for it, we only have to change parameters.
We don't have to numerically integrate the solution, improving
the overall stability of our method.

Orbits of satellites are much more complicated.
While it is almost impossible to compare orbits specified
by initial points and initial velocities, there is a small set
of geometrically interesting parameters like the inclination of
the plane of the orbit, the excentricity of the orbit and so on
that perfectly describe basic keplerian orbits in an intuitive way.
Describing perturbations of the orbit as slow changes these
parameters makes the solution much easier.
Using this technique it is possible to model the solar system
over millions of years, track the changes e.~g.~in the excentricity
of earth's orbits and the inclination of earth's axis and thus
understand the Milankovic cycles of ice ages.

The task that remains is to find suitable parameters to describe
the orbits:

\begin{aufgabe}
\label{jacobi:aufgabe}
Find a set of parameters 
$Q_i$ and $P_i$ so,
so that every trajectory can be written in the form
\begin{equation}
x_i(t) = x_i(t,Q_1,\dots,Q_n,P_1,\dots,P_n).
\label{jacobi:aufgabekurve}
\end{equation}
Some of the parameters are expected to be constant or slowly varying.
\end{aufgabe}

\begin{beispiel}
Let's examine again the example of the pool ball but assume now that
there is no friction.
Then the solution to the equations of motion is
\begin{equation}
\begin{aligned}
x(t)&=x_0 + tv_{x0},\\
y(t)&=y_0 + tv_{y0}.
\end{aligned}
\label{jacobi:linear}
\end{equation}
This is already a solution close to the task we have set, as every 
possible trajectory actually is parametrized by some suitable values
of $\vec{x}_0$ and $\vec{v}_0$.

However, we can do better.
We can choose coordinates $Q_i$ and $P_i$ such that all 
the $P_i$ are energies and all the $Q_i$ are times, and these
energies and times have a intuitive significance for the
problem at hand.
The idea is that we already know that energy is conserved, so
we can expect the $P_i$ to be constants of motion.

We use  the following terms for the $P_i$
\begin{equation*}
\begin{aligned}
P_x&=\frac12mv_x^2,&
P_y&=\frac12mv_y^2
\end{aligned}
\end{equation*}
These are the contributions to the total kinetic energy by the
motion in $x$- and $y$-direction.
As the ball rolls without friction, they are conserved.

The terms $Q_x$ and $Q_y$ need to be chosen in such a way that
the equations
\eqref{jacobi:linear}
describe the actual trajectory.
By solving \eqref{jacobi:aufgabekurve} for $Q_i$, we find
\begin{equation*}
\begin{aligned}
Q_x&=-\frac{x_0}{v_{0x}},\\
Q_y&=-\frac{y_0}{v_{0y}}.
\end{aligned}
\end{equation*}
The numbers $-Q_x$ and $-Q_y$ therefore are the times at which the pool
ball crosses the $x$- and $y$-axes respectively.
It is obvious that they are constants, so we have a set of parameters
$Q_i$ and $P_i$ that are particularly well suited to describe the
trajectory.
\end{beispiel}

The more difficult question is whether every mechanical system
allows a solution as described in problem \ref{jacobi:aufgabe}.
The answer is yes, and the method to find it happends to
be a very special partial differential equation that can often
be solved using the separation method.

\subsection{Hamilton-Jacobi formalism}
\index{Hamilton-Jacobi formalism}
In the hamiltonian description of mechanics, one starts with an
expression $H(x_i, p_i)$ for the total energy in terms of positions
and momenta, called the hamilton function or hamiltonian.
In the example problem, the hamiltonian is
\begin{equation}
H(x_i,p_i)=\frac1{2m}(p_x^2+p_y^2).
\label{jacobi:hamilton:funktion}
\end{equation}
The link between coordinates and impulses is then given by the
hamiltonian differential equations of motion:
\begin{align}
\frac{d}{dt}x_i&=\frac{\partial H}{\partial p_i}
&\Rightarrow
\qquad \dot x_i&=\frac{p_i}{m}=v_i
\label{jacobi:hamilton:geschwindigkeit}
\\
\frac{d}{dt}p_i&=-\frac{\partial H}{\partial x_i}
&\Rightarrow
\qquad
\dot p_i&=m\ddot x_i=0
\label{jacobi:hamilton:newton}
\end{align}
The first set of equations
\eqref{jacobi:hamilton:geschwindigkeit}
links momenta to coordinates.
The second equation \eqref{jacobi:hamilton:newton} corresponds to
Newton's second law and says in this particular case that there
is no force acting on the ball.
By adding a potential $V(x)$ to the hamiltionan
\eqref{jacobi:hamilton:funktion}, the second equation becomes
\[
m\ddot x_i=-\frac{\partial V}{\partial x_i} = F_i(x),
\]
which is exactly Newton's second law for the force derived from 
the potential.

Hamilton found that every other set of $Q_i$ and $P_i$ is just as
suitable to describe the system, as long as the equations
\eqref{jacobi:hamilton:geschwindigkeit} and \eqref{jacobi:hamilton:newton}
continue to hold.
Jacobi and Hamilton have devised a method to find a coordinate
transformation with rather special properties.

\begin{satz}
\label{jacobi:satz}
A solution to the problem \eqref{jacobi:aufgabe} can be found from 
a function $S(x_i, t)$, which solves the partial differential equation
\begin{equation}
\frac{\partial S}{\partial t}
=
H\biggl(
x_i,
\frac{\partial S}{\partial x_i}
\biggr)
\label{jacobi:hamilton:dgl}
\end{equation}
Such a solution function necessarily contains some integration constants
$P_i$.
The partial derivatives of $S$ with respect to $P_i$ are the new coordinates
\begin{equation}
Q_i=\frac{\partial S}{\partial P_i}.
\label{jacobi:hamilton:impuls}
\end{equation}
The equations \eqref{jacobi:hamilton:impuls} contain the coordinates 
$x_i$ in addition to the $P_i$, $Q_i$ and time $t$.
By solving for the variables $x_i$, we can find the trajectory as 
a vector of functions
\[
x_i(t,Q_i,P_i).
\]
\end{satz}
Thus the parameters $(Q_i,P_i)$
describe all the possible trajectories.
Two trajectories can simply be be compared for all times by comparing the
$Q_i$ and $P_i$.
If they are close, the trajectories are close too.

A proof of this theorem is outside the scope of this course, but we
would like to show using some examples how this formalism allows to
solve the equations of motion using the separation method.

\subsection{Energy conservation}
The following special case is of particular importance.
In a closed system, energy is conserved.
Consequently, the hamiltonian $H$ does not depend on time.

We now substitute a separation ansatz in the form
\[
S(t,x_i)=S_0(t)+S_1(x_1)+\dots+S_n(x_n).
\]
into the Hamilton-Jacobi equation and get
\[
S_0'(t)=H(S_1'(x_1),\dots, S_n'(x_n)).
\]
The left hand side only depends on $t$.
On the right hand side, $t$ does not appear, which means that we have
managed to separate $t$ from the variables $x_1,\dots,x_n$.
In particular, both sides have to be constant, this constant
of course being the total energy of the system.
Let's call this constant $P_1$.

Because the derivative of $S$ with respect to $t$ is constant, we
now already know that $S$ must be of the form
\[
S(t,x_i)=P_1t + \bar S(x_i)
\]
where $\bar S(x_i)$ is a new function that does not depend on time.
Using this representation, we can find new formulae for the 
coordinates $Q_i$
\begin{align*}
Q_1&=\frac{\partial S}{\partial P_1}=t+\frac{\partial \bar S(x_i)}{\partial P_1},
\\
Q_i&=\frac{\partial \bar S}{\partial P_i},\qquad i>1
\end{align*}
The equations for $Q_i$ with $i>1$ do not contain time,
they only describe the shape of the trajetory depending on the parameters
$Q_1,\dots,Q_n,P_1,\dots,P_n$.
The first equation is the only one that contains the time.
It's second term doesn't contain time either, but only the location
coordinates.
This means that $x_i$ only depend on 
\[
t-Q_1,Q_2,\dots,Q_n, P_1,\dots,P_n.
\]
Thus instead of the form \eqref{jacobi:aufgabekurve} there also is a 
solution in the form
\[
x_i(t)=x_i(t-Q_1,Q_2, \dots,Q_n,P_1,\dots,P_n).
\]

For a point mass in three dimension we can conclude that we can
always describe a trajectory using 6 parameters
$Q_1,Q_2,Q_3,P_1,P_2,P_3$, where the Parameter $Q_1$ has 
the meaning of the point of traversal of a geometrically well defined
point on the trajectory.
For planetary trajectories, this is typically the time the plant
travels through perihelion.

\subsection{Examples}
\subsubsection{Motion without external forces}
We refer back to the hamiltonian
\[
H(p_x, p_y)=\frac1{2m}(p_x^2+p_y^2).
\]
We have to solve the Hamilton-Jacobi equation \eqref{jacobi:hamilton:dgl}
\[
\frac1{2m}\biggl(
\biggl(\frac{\partial S}{\partial x}\biggr)^2
+
\biggl(\frac{\partial S}{\partial y}\biggr)^2
\biggr)=\frac{\partial S}{\partial t}
\]
for the function $S(t,x,y)$.
We use the separation ansatz
\[
S(t,x,y)=S_1(x)+S_2(y) + S_3(t).
\]
Substituting it into the differential equation gives
\begin{equation}
\frac1{2m}( S_1'(x)^2+S_2'(y)^2)=S_3'(t).
\label{jacobi:kraeftefrei:sep1}
\end{equation}
The left hand side does not depend on $t$, the right hand side only
depends on $t$, so we both sides must be constant, we call the
constant $P_1$.
This allows us to write $S_3$  as
\[
S_3(t)=P_1t
\]
up to some integration constants for the function $S$.
So we have solved \eqref{jacobi:aufgabe} for the variable $t$.

Let's now study the right hand side of \eqref{jacobi:kraeftefrei:sep1}.
We can separate the equation using the constant $P_1$:
\[
\frac1{2m} S_1'(x)^2
=
P_1-\frac1{2m}S_2'(y)^2.
\]
The left hand side depends only on $x$, the right hand side only on $y$,
thus both sides are constant.
We call this constant $P_2$ and find the solution
\[
S_1(x)
=
\sqrt{2mP_2}x.
\]
This now also allows us to determine $S_2$, which turns out to be
\[
\frac1{2m} S_2'(y)^2
=P_1-P_2,
\]
from which we derive the solution
\[
S_2(y)
=
\sqrt{2m(P_1-P_2)}y.
\]
We combine these partial solutions to a solution of the Hamilton-Jacobi
differential equation:
\[
S(x,y,t)=
P_1t
+
\sqrt{2mP_2}x
+
\sqrt{2m(P_1-P_2)}y
\]
According to theorem~\ref{jacobi:satz} the coordinates that should be used
are
\begin{align*}
Q_1&=\frac{\partial S}{\partial P_1}
=
t + \sqrt{\frac{m}{2(P_1-P_2)}}y,\\
Q_2&=\frac{\partial S}{\partial P_2}
=
\sqrt{\frac{m}{2P_2}}x
-
\sqrt{\frac{m}{2(P_1-P_2)}}y
\end{align*}
We now can solve for $x$ and $y$ and thus express the trajectory
in terms of the $Q_i$ and $P_i$ and time $t$:
\begin{align*}
x&=\sqrt{\frac{2P_2}{m}}(Q_1+Q_2-t)\\
y&=\sqrt{\frac{2(P_1-P_2)}{m}}(Q_1-t).
\end{align*}
These are precisely the ones we expected for uniform motion.
The velocity is the derivative with respect to time, so
\begin{align*}
v_x
&=
\sqrt{\frac{2P_2}{m}}
&\Rightarrow&&
P_2&=\frac{mv_x^2}2
\\
v_y
&=
\sqrt{\frac{2(P_1-P_2)}{m}}
&\Rightarrow&&
P_1&=P_2+\frac{mv_y^2}2=\frac{m}2(v_x^2+v_y^2)=\frac12mv^2
\end{align*}
The physical meaning of $P_1$ is the kinetic energy of the system
while $P_2$ is the kinetic energy contributed by the motion component
in the $x$-direction.

\subsubsection{The cannonball problem}
By adding a gravitational potential to the previous problem we
get the cannonball problem.
A cannonball is shot from some initial point $(x_0,y_0)$  with
initial velocity $(v_{x0},v_{y0})$ and it is well known that 
it travels along parabolic arc.
This problem can be solved by the method of Hamilton and Jacobi.

The total energy gets an additional term, the potential energy,
so the hamiltonian of the new system is
\[
H(x,y,t)=\frac1{2m}(p_x^2+p_y^2)+mgy.
\]
The Hamilton-Jacobi differential equation now becomes
\[
\frac1{2m}\biggl(
\biggl(\frac{\partial S}{\partial x}\biggr)^2
+
\biggl(\frac{\partial S}{\partial y}\biggr)^2
\biggr)
+mgy=\frac{\partial S}{\partial t}.
\]
The same separation ansatz as before again leads to
\[
\frac1{2m}(S_1'(x)^2+S_2'(y)^2)+mgy=S_3'(t),
\]
from which we once more conclude that $S_3'(t)=P_1$ is constant.

We can also separate $x$ and $y$:
\begin{align*}
\frac1{2m}S_1'(x)^2&=P_1-\frac1{2m}S_2'(y)^2-mgy.
%\\
%\frac1{\sqrt{2m}}S_1'(x)&=\sqrt{P_3-\frac1{2m}S_2'(y)^2+mgy}
\end{align*}
The left hand side depends only on $y$, the right only on $x$, so
both are again constant, we call this constant $P_2$ and obtain the solution
$S_1=\sqrt{2mP_2} x$.

The right hand side can also be solved:
\begin{align*}
P_2&=
P_1-\frac1{2m}S_2'(y)^2-mgy
\\
\frac1{2m}S_2'(y)^2
&=
P_1-P_2-mgy
\\
S_2'(y)&=\sqrt{
2m(P_1-P_2-mgy)
}
\\
S_2(y)
&=
-\frac1{3m^2g}\bigl(2m(P_1-P_2-mgy)\bigr)^{\frac32}.
\end{align*}
This completes the solution:
\[
S(x,y,t)=P_1t+\sqrt{2mP_2}x
-
\frac1{3m^2g}\bigl(2m(P_1-P_2-mgy)\bigr)^{\frac32}
\]
The coordinates that we should use are now the partial derivatives
of $S$ with respect to the $P_i$.
We get one by one as
\begin{equation}
\begin{aligned}
Q_1=\frac{\partial S}{\partial P_1}
&=
t-
\frac1{g}\sqrt{\frac{2(P_1-P_2-mgy)}{m}}
\\
Q_2=\frac{\partial S}{\partial P_2}
&=
\sqrt{\frac{m}{2P_1}}x
+
\frac1{g}\sqrt{\frac{2(P_1-P_2-mgy)}{m}}.
\end{aligned}
\label{jacobi:aufloesung}
\end{equation}
The first equation can be solved for $y$:
\begin{equation}
y=
\frac{P_1-P_2}{mg}-\frac{g}{2}(Q_1-t)^2.
\label{jacobi:quadratisch}
\end{equation}
The altitude $y$ thus depends quadratically on time and $Q_1$ is
the time of highest altitude, the time of apogee, which is certainly
a geometricall well defined point on the trajectory.

The solution for $x$ becomes easier if we first take the some
of the two equations
\eqref{jacobi:aufloesung}:
\begin{align*}
Q_1+Q_2&=t+\sqrt{\frac{m}{2P_1}}x\\
x&=\sqrt{\frac{2P_1}{m}}(Q_1+Q_2-t)
\end{align*}
The $x$-coordinate increases linearly with time.
The velocity is
\[
v_x=\sqrt{\frac{2P_1}{m}}\quad\Rightarrow\quad P_1=\frac{mv_x^2}2,
\]
$P_1$ again is the contribution to total energy of the horizontal
motion component.
$Q_2$ indicates the point in time when the $x$ coordinate vanishes.

At time $t=Q_1$, the quadratic term in \eqref{jacobi:quadratisch} vanishes
and the equation can be simplified to
\[
mgy + P_2=P_1.
\]
Since $P_2$ is the kinetic energy contributed by the horizontal motion
component and  $mgy$ is the potential energy, $P_1$ again is as
expected the total energy.

