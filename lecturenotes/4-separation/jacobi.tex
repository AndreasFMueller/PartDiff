%
% jacobi.tex -- Anwendung: Hamilton-Jacobi-Formulierung der Mechanik XXX
%
% (c) 2012 Prof Dr Andreas Mueller, Hochschule Rapperswil
% $Id$
%
\section{Anwendung: Hamiltonsche Mechanik\label{hamilton-mechanik}}
In den bisherigen Bespielen wurde jeweils ein Separationsansatz mit
einem Produkt von Teilfunktion gewählt.
Dieser Abschnitt soll illustrieren, dass in einigen Fällen auch
ein Separationsansatz mit einer Summe von Teilfunktionen
zum Ziel führen kann.
Dieser Fall ist für die Anwendungen recht wichtig, denn die
dabei entstehende partielle Differentialgleichungen hat eine
genügend einfach Struktur, dass der erste Separationsschritt immer
durchgeführt werden kann.

\subsection{Motivation}
\index{Newtonsche Gesetze}
\index{Planeten}
\index{Satelliten}
Die Newtonschen Gesetze reichen vollständig, um die Bewegung von
Planeten und Satelliten vorherzusagen.
Unterliegt
ein Körper der Masse $m$ mit zeitabhängigen Koordinaten $\vec x(t)$
einer ortsabhängigen Kraft $\vec F(\vec x)$, dann muss die Bahnkurve 
$\vec x(t)$ die gewöhnliche Differentialgleichung
\begin{equation}
m\frac{d^2}{dt^2}\vec x(t)=\vec F(\vec x(t))
\label{jacobi:newton}
\end{equation}
erfüllen. Ausgehend von einem bekanten Anfangspunkt $\vec x_0$ und
der Anfangsgeschwindigkeit $\vec v_0$ lässt sich durch
lösen der Differentialgleichung die Bahnkurve bestimmen.

Diese Beschreibung hat jedoch ein paar praktisch bedeutsame Nachteile.
Oft ist das Kraftgesetz nicht exakt bekannt. Zum Beispiel werden
Satelliten in niedrigem Erdorbit\footnote{Low earth orbit, wenige 100km}
\index{Erdorbit}
von der zwar sehr dünnen, aber nicht vernachlässigbaren
Erdatmosphäre abgebremst.
\index{Erdatmosphare@Erdatmosph\äre}
\index{Merkur}
Oder der Planet Merkur verändert laufend seine Bahn um einen winzigen
Betrag, ein Effekt, den erst Albert Einstein mit seiner speziellen
Relativitätstheorie erklären konnte.
Man sieht sich also oft mit der Aufgabe konfrontiert, dass man zwar die
Bahn berechnen könnte, wenn das Kraftgesetz exakt zutreffen würde,
dass man aber die Bahnveränderungen unter dem Einfluss kleiner
Abweichungen vom exakten Kraftgesetz bestimmen sollte.
\index{Luftwiderstand}

\begin{beispiel}
\index{Billardtisch}
Als Beispiel betrachten wir ein Kugel auf einem ideal horizontal angenommenen
Billardtisch. Die Bewegung dieser Kugel wird offenbar genau beschrieben
durch den Anfangspunkt und die Geschwindigkeit $\vec x_0$ und $\vec v_0$.
Die Kugel wird in Richtung $\vec x_0$ weiterrollen, dabei
aber langsamer werden und schliesslich zum Stillstand kommen.
Es gibt also eine Kurve $t\mapsto \vec x(t, \vec x_0, \vec v_0))$,
die beiden Vektoren $\vec x_0$ und $\vec v_0$ bestimmen die
Bahn vollständig.

Was passiert, wenn der Billiardtisch nicht exakt horizontal ist?
Offenbar wirkt dann eine kleine zusätzlich Kraft auf die Billard-Kugel,
und zwar in Richtung der grössten Neigung der Billardtischplatte.
Wenn die Neigung sehr klein ist, ist die Abweichung von 
$\vec x(t,\vec x_0,\vec v_0))$ sehr gering.
Man könnte die Beschreibung diese Modifikation dadurch beschreiben,
dass man $\vec x_0$ und $\vec v_0$ leicht anpasst.
Die tatsächliche Position der Kugel zur Zeit $t$ ist die Position,
die eine Billiardkugel auf einem horizontalen Tisch ausgehend von
einem etwas anderen Ausgangspunkt $\vec x_0(t)$ mit einer
etwas anderen Ausgangsgeschwindigkeit $\vec v_0(t)$
nach Zeit $t$ erreicht hätte.
\end{beispiel}

Der Vorteil dieser Beschreibung besteht darin, dass sich die
Abweichung vom exakten Kraftgesetz direkt in der Zeitabhängigkeit
von $x_0(t)$ und $v_0(t)$ ausdrückt. Ist das Kraftgesetz exakt
erfüllt, bleiben $x_0(t)$ und $v_0(t)$ konstant. Ausserdem ist die
Berechnung der Bahn ganz einfach: zu $\vec x_0$ kommt einfach
ein Vielfaches von $\vec v_0$ hinzu.

Die Bahnen der Satelliten sind offenbar viel komplizierter. 
Ort und Geschwindigkeit sind nicht geeignet, die
Bahn auf diese Art und Weise zu charakterisieren, sie ändern laufend
dramatisch ihre Grösse. Niemand kann mit Leichtigkeit sagen, ob
die Bahn vom Punkt  $\vec x_0$ mit Anfangsgeschwindigkeit $\vec v_0$
irgendwie ähnlich aussieht wie die Bahn vom Punkt $\vec x_1$
mit Anfangsgeschwindigkeit $\vec v_1$.

\begin{aufgabe}
\label{jacobi:aufgabe}
Zu einem beliebigen mechanischen System finde man einen Satz von 
Parametern 
$Q_i$ und $P_i$ so, dass sich {\em jede}
Bahn mit Hilfe dieser Parameter in der Form
\begin{equation}
x_i(t) = x_i(t,Q_1,\dots,Q_n,P_1,\dots,P_n)
\label{jacobi:aufgabekurve}
\end{equation}
beschreiben lässt.
\end{aufgabe}

\begin{beispiel}
Betrachten wir wieder das Problem der Billardkugel, aber nehmen wir
diesmal an, dass die Kugel ohne Reibung rollt. Dann gilt
\begin{equation}
\begin{aligned}
x(t)&=x_0 + tv_{x0},\\
y(t)&=y_0 + tv_{y0}.
\end{aligned}
\label{jacobi:linear}
\end{equation}
In dieser Form ist das bereits eine Lösung der gestellten Aufgabe,
denn jede Bahnkurve lässt such durch geeignete Wahl der
Parameter $\vec x_0$ und $\vec v_0$ charakterisieren.

Man kann aber noch mehr erreichen: man kann die Koordinaten $Q_i$ und $P_i$
so wählen, dass die $P_i$ alle die Bedeutung einer Energie,
und die $Q_i$ die Bedeutung einer Zeit haben.
Die Funktion der $P_i$ können die Terme
\begin{equation*}
\begin{aligned}
P_x&=\frac12mv_x^2,&
P_y&=\frac12mv_y^2
\end{aligned}
\end{equation*}
übernehmen, also die Beiträge der Bewegungskomponenten in $x$-
bzw.~$y$-Richtung zur Energie. Da die Kugel reibungsfrei rollt, sind
diese Grössen konstant.

Die Grössen $Q_x$ und $Q_y$ müssen so gewählt werden, dass 
die Gleichungen (\ref{jacobi:linear}) die tatsächliche Bahn beschreiben.
Durch Auflösen von (\ref{jacobi:aufgabekurve}) nach $Q_i$ findet man
\begin{equation*}
\begin{aligned}
Q_x&=-\frac{x_0}{v_{0x}},\\
Q_y&=-\frac{y_0}{v_{0y}}.
\end{aligned}
\end{equation*}
Die Zahlen $-Q_x$ und $-Q_y$ sind also die Zeiten, zu denen die
Billardkugel die $x$- bzw.~$y$-Achse kreuzt.
\end{beispiel}

Die schwierigere Frage ist, ob jedes mechanische System eine
Lösung der Aufgabe \ref{jacobi:aufgabe} zulässt.
Die Antwort passt ins Thema: ja, man kann eine solche Beschreibung
finden, aber man muss dazu die Lösung einer speziellen
partiellen Differentialgleichung finden. Oft lässt sich die
Differentialgleichung mit einem Separationsansatz lösen.

\subsection{Hamilton-Jacobi-Formalismus}
\index{Hamilton-Jacobi-Formalismus}
In der Hamiltonschen Beschreibung der Mechanik geht man aus von
der Gesamtenergie $H(x_i, p_i)$ ausgedrückt durch Ort und Impuls.
Für das Beispielproblem ist
\index{Hamilton-Funktion}
\begin{equation}
H(x_i,p_i)=\frac1{2m}(p_x^2+p_y^2).
\label{jacobi:hamilton:funktion}
\end{equation}
Der Zusammenhang zwischen den Koordinaten und den Impulsen ist dann
durch die Hamiltonschen Differentialgleichungen gegeben:
\begin{align}
\frac{d}{dt}x_i&=\frac{\partial H}{\partial p_i}
&\Rightarrow
\qquad \dot x_i&=\frac{p_i}{m}=v_i
\label{jacobi:hamilton:geschwindigkeit}
\\
\frac{d}{dt}p_i&=-\frac{\partial H}{\partial x_i}
&\Rightarrow
\qquad
\dot p_i&=m\ddot x_i=0
\label{jacobi:hamilton:newton}
\end{align}
\index{Hamilton-Gleichungen}
Die erste Gleichung (\ref{jacobi:hamilton:geschwindigkeit}) stellt
nur den Zusammenhang zwischen der Geschwindigkeit und dem Impuls
her. Die zweite Gleichung (\ref{jacobi:hamilton:newton}) besagt in
diesem Fall, dass keine Kraft wirkt. Nähme man in 
(\ref{jacobi:hamilton:funktion}) noch ein Potential $V(x)$ hinzu,
würde die zweite Gleichung zu
\[
m\ddot x_i=-\frac{\partial V}{\partial x_i} = F_i(x),
\]
also genau dem Newtonschen Gesetz.

Nach Hamiliton ist jeder andere Satz von Koordinaten $Q_i$ und $P_i$
genauso geeignet zur Beschreibung des mechanischen Systems, solange die
Gleichungen 
(\ref{jacobi:hamilton:geschwindigkeit}) und (\ref{jacobi:hamilton:newton})
weiterhin Gültigkeit haben. Jacobi und Hamilton haben eine Methode
angegeben, mit der man eine Koordinatentransformation finden kann.

\begin{satz}
\label{jacobi:satz}
Eine Lösung der Aufgabe (\ref{jacobi:aufgabe}) wird gefunden mit
Hilfe einer Funktion $S(x_i, t)$, die Lösung der partiellen
Differentialgleichung
\begin{equation}
\frac{\partial S}{\partial t}
=
H\biggl(
x_i,
\frac{\partial S}{\partial x_i}
\biggr)
\label{jacobi:hamilton:dgl}
\end{equation}
erfüllt.
Eine Lösungsfunktion $S$ der Differentialgleichung enthält notwendigerweise
ein Anzahl von Integrationskonstanten $P_i$.
Die partiellen Ableitungen von $S$ nach diesen $P_i$
sind die neuen konstanten Bahnparameter $Q_i$
\begin{equation}
Q_i=\frac{\partial S}{\partial P_i}
\label{jacobi:hamilton:impuls}
\end{equation}
Die Gleichungen (\ref{jacobi:hamilton:impuls}) enthalten ausser den
Grössen $P_i$ und $Q_i$ auch die Koordinaten $x_i$ und die Zeit $t$.
Durch Auflösen nach den Variablen $x_i$ lässt sich die Bahnkurve
als Funktion
\[
x_i(t,Q_i,P_i)
\]
ausdrücken.
\end{satz}
Der Parametersatz $(Q_i,P_i)$ beschreibt also alle möglichen 
Bahnen. Zwei Bahnen können sehr einfach verglichen werden, wenn die
Paramter $Q_i$ und $P_i$ nahe beeinander sind, dann liegen auch
die Bahnen nahe beeinander.

Eine Begründung für diesen Satz liegt ausserhalb der Ziele
dieser Vorlesung, wir wollen aber mit zwei Beispielen zeigen,
dass dieser Formulismus funktioniert und die Lösung der
Bewegungsdifferentialgleichungen des Systems ermöglicht.

\subsection{Energieerhaltung}
Ein Spezialfall ist von besonderer Bedeutung. In einem abgeschlossenen
System ist die Energie erhalten, die Hamilton-Funktion $H$, die ja
die Energie darstellt, kann also nicht von der Zeit abhängen.
Verwenden wir einen Separationsansatz der Form 
\[
S(t,x_i)=S_0(t)+S_1(x_1)+\dots+S_n(x_n),
\]
wird die Hamilton-Jacobi-Differentialgleichung zu
\[
S_0'(t)=H(S_1'(x_1),\dots, S_n'(x_n)).
\]
Die linke Seite hängt nur von $t$ ab, auf der rechten Seite
kommt $t$ aber gar nicht vor, die Variablen $t$ und $x_i$ sind
separiert, beide Seiten der Gleichung müssen konstant sein.
Wir nennen die Konstante $P_1$, sie ist der Wert der Hamilton-Funktion,
also die Gesamtenergie.
Damit steht auch schon fest, dass die gesuchte Funktion $S(t, x_i)$
die Form
\[
S(t,x_i)=P_1t + \bar S(x_i)
\]
haben muss, wobei $\bar S(x_i)$ die Zeit $t$ nicht enthalten kann.
Für die Koordinaten $Q_i$ gilt daher
\begin{align*}
Q_1&=\frac{\partial S}{\partial P_1}=t+\frac{\partial \bar S(x_i)}{\partial P_1},
\\
Q_i&=\frac{\partial \bar S}{\partial P_i},\qquad i>1
\end{align*}
Die Gleichungen für $Q_i$ mit $i>1$ enthalten die Zeit nicht, sie
beschreiben also ausschliesslich die Form der Bahn in Abhängigkeit
von den Parameter $Q_1,\dots,Q_n,P_1,\dots,P_n$. Die erste
Gleichung ist die einzige, die Zeit enthält. In ihrem zweiten
Term kommt die Zeit ebenfalls nicht vor, sondern nur die Ortskoordinaten.
Die Ortskoordinaten hängen daher nur von
\[
t-Q_1,Q_2,\dots,Q_n, P_1,\dots,P_n
\]
ab. Statt der Form (\ref{jacobi:aufgabekurve}) gibt es also sogar eine
Lösung der Form
\[
x_i(t)=x_i(t-Q_1,Q_2, \dots,Q_n,P_1,\dots,P_n).
\]

Für eine Punktmasse in drei Dimensionen können wir also schliessen, 
dass es immer eine Beschreibung der Bahn mit sechs Parametern
$Q_1,Q_2,Q_3,P_1,P_2,P_3$ gibt, wobei der Parameter $Q_1$ die Bedeutung
hat, dass ein ganz bestimmter, geometrisch ausgezeichneter Bahnpunkt
zu dieser Zeit durchlaufen wird.
Bei der Beschreibung von Planetenbahnen dieser Punkt typischerweise
das Perihel, der Sonnennächste Punkt, oder bei Satelliten entsprechend
das Perigäum, der erdnächste Punkt.

\subsection{Beispiele}
\subsubsection{Bewegung ohne äusseren Krafeinfluss in zwei Dimensionen}
\index{Bewegung ohne Krafteinfluss}
Wir beginnen wieder bei der Hamilton-Funktion 
\[
H(p_x, p_y)=\frac1{2m}(p_x^2+p_y^2).
\]
Nach (\ref{jacobi:hamilton:dgl}) müssen wir jetzt also die
Differentialgleichung
\[
\frac1{2m}\biggl(
\biggl(\frac{\partial S}{\partial x}\biggr)^2
+
\biggl(\frac{\partial S}{\partial y}\biggr)^2
\biggr)=\frac{\partial S}{\partial t}
\]
für die Funktion $S(t,x,y)$ lösen. Wir verwenden dazu einen 
Separationsansatz der Form
\[
S(t,x,y)=S_1(x)+S_2(y) + S_3(t).
\]
Einsetzen in die Differentialgleichung liefert
\begin{equation}
\frac1{2m}( S_1'(x)^2+S_2'(y)^2)=S_3'(t).
\label{jacobi:kraeftefrei:sep1}
\end{equation}
Die linke Seite hängt nicht von $t$ ab, die rechte hängt
aber nur von $t$ ab, also sind beide Seiten konstant.
Wir nennen die Konstanten $P_1$. Damit lässt sich
jetzt $S_3(t)$ bestimmen, es muss eine geeignete Integrationskonstante
geben, so dass
\[
S_3(t)=P_1t
\]
damit ist die Aufgabe \ref{jacobi:aufgabe} mindestens für die Variable
$t$ bereits gelöst.

Wir untersuchen jetzt die linke Seite von (\ref{jacobi:kraeftefrei:sep1}).
Auch diese Gleichung kann man unter Verwendung der Konstanten $P_1$
separieren:
\[
\frac1{2m} S_1'(x)^2
=
P_1-\frac1{2m}S_2'(y)^2.
\]
Die linke Seite hängt nur von $x$ ab, die rechte nur von $y$, also
sind beide konstant. 
Wir nennen die Konstante $P_2$ und finden die Lösung
\[
S_1(x)
=
\sqrt{2mP_2}x.
\]
Jetzt kann man aber auch noch $S_2$ bestimmen. Es ist nämlich 
\[
\frac1{2m} S_2'(y)^2
=P_1-P_2
\]
woraus sich wie vorhin die Lösung
\[
S_2(y)
=
\sqrt{2m(P_1-P_2)}y
\]
ergibt. Damit haben wir jetzt eine Lösung der Differentialgleichung:
\[
S(x,y,t)=
P_1t
+
\sqrt{2mP_2}x
+
\sqrt{2m(P_1-P_2)}y
\]
Nach den Regeln des Satzes \ref{jacobi:satz}, sind die zu verwendenden
Koordinaten die partiellen Ableitungen:
\begin{align*}
Q_1&=\frac{\partial S}{\partial P_1}
=
t + \sqrt{\frac{m}{2(P_1-P_2)}}y,\\
Q_2&=\frac{\partial S}{\partial P_2}
=
\sqrt{\frac{m}{2P_2}}x
-
\sqrt{\frac{m}{2(P_1-P_2)}}y
\end{align*}
Jetzt kann man nach $x$ und $y$ auflösen und damit die
Bahnkurve durch die Konstanten $Q_i$ und $P_i$ und die Zeit $t$
ausdrücken:
\begin{align*}
x&=\sqrt{\frac{2P_2}{m}}(Q_1+Q_2-t)\\
y&=\sqrt{\frac{2(P_1-P_2)}{m}}(Q_1-t).
\end{align*}
Wie erwartet beschreiben diese Gleichungen eine gleichförmige
Bewegung. Die Geschwindigkeit ist die Ableitung nach der Zeit,
also
\begin{align*}
v_x
&=
\sqrt{\frac{2P_2}{m}}
&\Rightarrow&&
P_2&=\frac{mv_x^2}2
\\
v_y
&=
\sqrt{\frac{2(P_1-P_2)}{m}}
&\Rightarrow&&
P_1&=P_2+\frac{mv_y^2}2=\frac{m}2(v_x^2+v_y^2)=\frac12mv^2
\end{align*}
Die physikalische Bedeutung von $P_1$ ist also die kinetische
Energie des
Gesamtsystems, während $P_2$ die kinetische Energie darstellt, die in
der Bewegungskomponent in $x$-Richtung steckt.

\subsubsection{Schiefer Wurf}
\index{schiefer Wurf}
Wir betrachten den schiefen Wurf, der sich vom vorangegangenen
Beispiel durch ein zusätzliches Gravitationspotential
unterscheidet. Die Gesamtenergie ist jetzt
\[
H(x,y,t)=\frac1{2m}(p_x^2+p_y^2)+mgy.
\]
Die Hamilton-Jacobi-Differentialgleichung lautet jetzt
\[
\frac1{2m}\biggl(
\biggl(\frac{\partial S}{\partial x}\biggr)^2
+
\biggl(\frac{\partial S}{\partial y}\biggr)^2
\biggr)
+mgy=\frac{\partial S}{\partial t} 
\]
Der selbe Separationsansatz wie vorhin führt auch wieder zu
\[
\frac1{2m}(S_1'(x)^2+S_2'(y)^2)+mgy=S_3'(t),
\]
woraus wieder folgt, dass $S_3'(t)=P_1$ konstant ist.

Wir können aber auch $x$ und $y$ separieren:
\begin{align*}
\frac1{2m}S_1'(x)^2&=P_1-\frac1{2m}S_2'(y)^2-mgy
%\\
%\frac1{\sqrt{2m}}S_1'(x)&=\sqrt{P_3-\frac1{2m}S_2'(y)^2+mgy}
\end{align*}
Die linke Seite hängt nicht von $y$ ab, die rechte nicht von $x$, also sind
beide konstant, wir nennen die Konstante $P_2$, und bekommen
die Lösung $S_1=\sqrt{2mP_2} x$.

Die rechte Seite können wir jetzt auch lösen:
\begin{align*}
P_2&=
P_1-\frac1{2m}S_2'(y)^2-mgy
\\
\frac1{2m}S_2'(y)^2
&=
P_1-P_2-mgy
\\
S_2'(y)&=\sqrt{
2m(P_1-P_2-mgy)
}
\\
S_2(y)
&=
-\frac1{3m^2g}\bigl(2m(P_1-P_2-mgy)\bigr)^{\frac32}
\end{align*}
Damit ist jetzt eine Lösungsfunktion $S(x,y,t)$ vollständig
bekannt:
\[
S(x,y,t)=P_1t+\sqrt{2mP_2}x
-
\frac1{3m^2g}\bigl(2m(P_1-P_2-mgy)\bigr)^{\frac32}
\]
Die zu verwendenden Koordinaten bekommt man jetzt wie vorhin durch
partielle Ableitung nach den $P_i$. Man bekommt nacheinander:
\begin{equation}
\begin{aligned}
Q_1=\frac{\partial S}{\partial P_1}
&=
t-
\frac1{g}\sqrt{\frac{2(P_1-P_2-mgy)}{m}}
\\
Q_2=\frac{\partial S}{\partial P_2}
&=
\sqrt{\frac{m}{2P_1}}x
+
\frac1{g}\sqrt{\frac{2(P_1-P_2-mgy)}{m}}
\end{aligned}
\label{jacobi:aufloesung}
\end{equation}
Die erste Gleichung kann man nach $y$ auflösen:
\begin{equation}
y=
\frac{P_1-P_2}{mg}-\frac{g}{2}(Q_1-t)^2
\label{jacobi:quadratisch}
\end{equation}
Die Höhe hängt quadratisch von der Zeit ab, $Q_1$ ist die Zeit
der grössten Höhe, die Scheitelzeit.

Die Auflösung nach $x$ wird einfacher, wenn man erst die Summe der beiden
Gleichungen (\ref{jacobi:aufloesung}) bildet:
\begin{align*}
Q_1+Q_2&=t+\sqrt{\frac{m}{2P_1}}x\\
x&=\sqrt{\frac{2P_1}{m}}(Q_1+Q_2-t)
\end{align*}
Die $x$-Koordinate nimmt offenbar linear mit der Zeit zu. Die
Geschwindigkeit ist
\[
v_x=\sqrt{\frac{2P_1}{m}}\quad\Rightarrow\quad P_1=\frac{mv_x^2}2,
\]
$P_1$ ist also die kinetische Energie der Horizontalbewegung.
$Q_2$ gibt an, wie viel Zeit nach dem Scheiteldurchgang die $x$-Koordinate
verschwindet.

Zur Zeit $t=Q_1$ verschwindet der quadratische Term in
(\ref{jacobi:quadratisch}), und man kann die Gleichung vereinfachen
zu 
\[
mgy + P_2=P_1.
\]
Da $P_2$ die kinetische Energie der Horizontalbewegung ist, und $mgy$
die potentielle Energie, ist $P_1$ wie erwartet die Gesamtenergie.

