%
% ode.tex -- ordinary differential equations
%
% (c) 2019 Prof Dr Andreas Mueller
%
\section{Separation of Variables for ordinary differential equations}
The differential equation
\begin{equation}
y'-xy=0
\label{separation:ode}
\end{equation}
can be solved using separation of variables:
\begin{align*}
\frac{dy}{dx}&=xy\\
\frac1y\,dy&=x\,dx\\
\int\frac1y\,dy&=\int x\,dx\\
\log|y|&=\frac12x^2+C\\
y&=y_0e^{\frac12x^2}.
\end{align*}
The method is based on the idea to have only a single variable
on either side of the equation.
Even the derivative is formally decomposed as a fraction $dy/dx$, which
isn't really meaningful but justified by the success of the method.
This reduces the solution of the differential equation to the computation
of two integrals, i.~e.~the solution of the particularly simple
differential equation $y'=f(x)$ with solution $y(x)=\int f(x)\,dx$..
This solution also tells us what kind of initial conditions we need
for the solution to be uniquely determined.
If $y_0$ is the value of the solution at $x=x_0$, then the solution is
\[
y(x)=\int_{x_0}^xf(\xi)\,d\xi + y_0.
\]

Of course the process will not be quite as simple when we transition
to partial differential equations.
In particular the suspicious operation to separate the differentials
of $dy/dx$ has no counterpart for more than one independent variable.
However, depending on the shape of the domain, it might still be possible
to separate the variables and to transform the equation into the form
\[
\text{functions/derivatives only involving $x$}
=
\text{functions/derivatives only involving $y$}.
\]
Since the left hand side only depends on $x$ while the right hand side
depends only on $y$, both sides must be constant.
So we can write
\begin{align*}
\text{functions/derivatives only involving $x$} &= \mu
\\
\text{functions/derivatives only involving $y$} &= \mu
\end{align*}
with a new constant $\mu$ that has to be determined later.
This means that we have succeeded to reduce the partial differential
equation to two ordinary differential equations.
Since we know ``all'' about ordinary differential equations, we should
now be in a position to solve the partial differential equation and
to determine what kinds of boundary values are needed to make the
solution unique.

