%
% ode.tex -- XXX
%
% (c) 2019 Prof Dr Andreas Mueller
%
\section{Separation der Variablen für gewöhnliche Differentialgleichungen}
Die Differentialgleichung 
\begin{equation}
y'-xy=0
\label{separation:ode}
\end{equation}
kann mit Separation der Variablen gelöst werden:
\begin{align*}
\frac{dy}{dx}&=xy\\
\frac1y\,dy&=x\,dx\\
\int\frac1y\,dy&=\int x\,dx\\
\log|y|&=\frac12x^2+C\\
y&=y_0e^{\frac12x^2}
\end{align*}
Das Verfahren beruht auf dem Prinzip, auf jeder Seite der Gleichung
nur eine Variable zu haben.
Dabei zerlegt man sogar die Ableitung $dy/dx$, was man ja eigentlich
gar nicht darf.
Die Separation führt die Lösung der Differentialgleichung auf die
Berechnung zweier Integrale zurück, also eigentlich auf die
Lösung einer besonders einfachen Differentialgleichung der Form $y'=f(x)$
mit Lösung $y(x)=\int f(x)\,dx$.
Diese Form der Lösung sagt uns auch genau, was wir an Anfangsbedingungen
brauchen.
Ist $y_0$ der Wert der Lösung an der Stelle $x_0$, dann liefert
das Integral die Lösungsfunktion
\[
y(x)=\int_{x_0}^xf(\xi)\,d\xi + y_0.
\]

So einfach wird es für partielle Differentialgleichungen nicht sein.
Insbesondere für die an sich schon etwas ``anrüchige'' Operationen,
den Differentialquotienten $dy/dx$ auseinanderzureissen gibt es keinerlei
Entsprechung bei partiellen Ableitungen.
Unter geeigneten Voraussetzungen an das Gebiet und die Differentialgleichung
ist es aber immer noch denkbar, dass man die Variablen trennen kann,
also 
\[
\text{Funktionen/Ableitungen nur mit $x$} = \text{Funktionen/Ableitungen nur mit $y$}
\]
Da die linke Seite nur von $x$ abhängt, die rechte aber nur von $y$, müssen
beide Seiten konstant sein, die partielle Differentialgleichung zerfällt
in eine linke {\em gewöhnliche} Differentialgleichung für eine Funktion von
$x$ und eine rechte {\em gewöhnliche} Differentialgleichung für $y$.

Da wir über gewöhnliche Differntialgleichungen Bescheid wissen, sollte
es uns so möglich sein, eine partielle Differentialgleichung zu lösen
und herzuleiten, welche Art von Randwerten vorgegeben werden müssen, damit
die Differentialgleichung eindeutig lösbar ist.

