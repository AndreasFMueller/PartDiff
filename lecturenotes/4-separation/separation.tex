%
% separation.tex
%
% (c) 2019 Prof Dr Andreas Mueller
%
\lhead{Separation of Variables}
\rhead{}
\chapter{Separation of Variables\label{chapter-separation}}
\index{Stroboscope}
\index{standing Wave}
\begin{figure}
\begin{center}
\includegraphics[width=0.8\hsize]{../common/graphics/stringvibrlarge-10-06-06.jpg}
\end{center}
\caption{
Vibrating string
(Image by A.~Davidhazy, http://people.rit.edu/andpph/)
\label{separation:schwingendesaite}}
\end{figure}
Lighting a vibrating string with a stroboscope at the frequency of the
string makes the string seemingly stand still (see figure
\ref{separation:schwingendesaite})
The shape of the solution of the wave equation thus is the same
at periodically recurring points in time.
Measuring the elongation's dependence on time, it turns out to be
a harmonic oscillation which can be described using sine and cosine
functions.
This leads to the conjecture that the solution can be written as 
a product
\[
u(x,t)=X(x)\cdot\sin\omega t\quad\text{oder}\quad X(x)\cdot\cos\omega t.
\]
The goal of this chapter is to develop this idea into a solution
algorithm that is applicable to a suitably large class of partial
differential equations.

%
% ode.tex -- ordinary differential equations
%
% (c) 2019 Prof Dr Andreas Mueller
%
\section{Separation of Variables for ordinary differential equations}
The differential equation
\begin{equation}
y'-xy=0
\label{separation:ode}
\end{equation}
can be solved using separation of variables:
\begin{align*}
\frac{dy}{dx}&=xy\\
\frac1y\,dy&=x\,dx\\
\int\frac1y\,dy&=\int x\,dx\\
\log|y|&=\frac12x^2+C\\
y&=y_0e^{\frac12x^2}.
\end{align*}
The method is based on the idea to have only a single variable
on either side of the equation.
Even the derivative is formally decomposed as a fraction $dy/dx$, which
isn't really meaningful but justified by the success of the method.
This reduces the solution of the differential equation to the computation
of two integrals, i.~e.~the solution of the particularly simple
differential equation $y'=f(x)$ with solution $y(x)=\int f(x)\,dx$..
This solution also tells us what kind of initial conditions we need
for the solution to be uniquely determined.
If $y_0$ is the value of the solution at $x=x_0$, then the solution is
\[
y(x)=\int_{x_0}^xf(\xi)\,d\xi + y_0.
\]

Of course the process will not be quite as simple when we transition
to partial differential equations.
In particular the suspicious operation to separate the differentials
of $dy/dx$ has no counterpart for more than one independent variable.
However, depending on the shape of the domain, it might still be possible
to separate the variables and to transform the equation into the form
\[
\text{functions/derivatives only involving $x$}
=
\text{functions/derivatives only involving $y$}.
\]
Since the left hand side only depends on $x$ while the right hand side
depends only on $y$, both sides must be constant.
So we can write
\begin{align*}
\text{functions/derivatives only involving $x$} &= \mu
\\
\text{functions/derivatives only involving $y$} &= \mu
\end{align*}
with a new constant $\mu$ that has to be determined later.
This means that we have succeeded to reduce the partial differential
equation to two ordinary differential equations.
Since we know ``all'' about ordinary differential equations, we should
now be in a position to solve the partial differential equation and
to determine what kinds of boundary values are needed to make the
solution unique.


%
% idea.tex -- idea of the method
%
% (c) 2008 Prof Dr Andreas Mueller
%

\section{Idea of the method}
In applications one often has some indications from the application
domain what the solution function will most probably look like, or one
is looking for a very particular type of solution.
In partuclar, it may be known how the solution depends on one of the
variables up to a factor depending only on the other variable.
In these situations one can try to write the solution as a product
or sum of functions that depend on only one variable.

Even if one knows nothing about the solution, one can still try such
an {\em ansatz}, as the following example tries to illustrate.
Let's attempt to solve the partial differetnial equation
\begin{equation}
\frac1x
\frac{\partial u}{\partial x}
+
\frac1y
\frac{\partial u}{\partial y}
=\frac1{y^2}
,
\qquad x>1, y>1,
\label{separation:beispiel1}
\end{equation}
ignoring the boundary conditions for the time being.
We try to represent the solution as a sum of two functions which depend
on one of $x$ and $y$ only.
\begin{equation}
u(x,y)=X(x)+Y(y)
\quad\Rightarrow\quad
\begin{cases}
\quad{\displaystyle \frac{\partial u}{\partial x}}&=X'(x)\\
\\
\quad{\displaystyle \frac{\partial u}{\partial y}}&=Y'(y)\\
\end{cases}
\label{separation:beispiel1:ansatz}
\end{equation}
Substituting this into the differential equation
(\ref{separation:beispiel1})
gives the new equation
\[
\frac{X'(x)}{x}+\frac{Y'(y)}{y}=\frac1{y^2}
\]
or
\begin{equation}
\frac{X'(x)}{x}
=\frac1{y^2}
-\frac{Y'(y)}{y}.
\label{separation:beispiel1:separiert}
\end{equation}
The Form (\ref{separation:beispiel1:separiert}) has a distinct property:
the variable $x$ only appears on the left side, the variable $y$ only on
the right.
If we fix some value $y$, the right hand side cannot change, so the left
hand side must not depend on $x$.
Conversely, if we fix $x$, then the left side cannot change any more,
and thus the right hand side cannot change either.
We conclude that both sides must be the same constant.
Calling this constant $k$ we find the two ordinary differential equations
\begin{align}
\frac{X'(x)}{x}&=k
&
k&=\frac1{y^2}-\frac{Y'(y)}{y}
\label{separation:beispiel1:separiertedgl}
\end{align}
for $X(x)$ and $Y(y)$.

The differential equation for $X$ is easy to solve:
\begin{align*}
X'(x)&=kx\quad\Rightarrow\quad X(x)=
\frac12kx^2+C_x.
\end{align*}
The right equation is only slightly more complicated:
\begin{align*}
Y'(y)=\frac1y-ky
\quad\Rightarrow\quad
Y(y)=\int\frac1y-ky\,dy=
\log y-\frac12ky^2+C_y.
\end{align*}
We can now combine these functions into a solution of the initial
differential equation:
\begin{equation}
u(x,y)=
\frac12kx^2+
\log y-\frac12ky^2+C.
\label{separation:beispiel1:loesung}
\end{equation}
By varying the parameters $k$ and $C$, formula
(\ref{separation:beispiel1:loesung}) gives an infinite family of
solutions of the partial differential equation.

The values of the constants need to be determined by boundary conditions
in a manner to be studied later.

Let's summarize the method so far:
\begin{enumerate}
\item
Choose an {\em ansatz} from functions that depend from disjoint
sets of variables.
\item
Substitute into the partial differential equation and separate terms
involving the separate sets of variables.
The two sides of the equation depend on disjoint sets of variables
und must therefore be constant.
\item 
Split the equation into two coupled equations, each with a
different set of independent variables.
\item
Solve each equation individually.
\item
Put solutions together using the boundary conditions.
\end{enumerate}
As may suspected, the problem most of the time is not the solution
of the individual equations but rather the last step.
In the following sections we want to illustrate how this can be
done in a variety of examples.


%
% lpde.tex -- why linear pdes?
%
% (c) 2019 Prof Dr Andreas Mueller
%
\section{Separation for linear partial differential equations}
The base idea of the separation method produces a family of functions
that depend an some integration and separation constants.
On the boundary we are usually given some arbitrary functions.
In general it will be impossible to tune these few constants to
values that reproduce the boundary functions.
This basic version of the separation method thus is incapable of
solving a general partial differential equation due to the lack
of flexibility in the set of solutions found.

This problem changes if different solutions can be combined into new ones.
Linear combinations of solutions introduce a large set of additional
parameters to tune the solution.
In particular, Fourier theory shows that linear combinations of
basic functions can be tuned to approximate just about any periodic
function.
However, linear combinations of solutions are in general no longer
solutions the equation, except for linear partial differential
equations.
This is expressed in the following theorem.

\begin{satz}
If $u_1$ and $u_2$
are solutions of a homogeneous linear partial differential equation,
then $u_1+u_2$ and $\lambda u_1$ with $\lambda\in\mathbb R$ are also
solutions.
\end{satz}

\begin{proof}
The differential equation
\[
F(x_1,\dots,x_2,u,\frac{\partial u}{\partial x_1},\dots)=0
\]
is linear in $u$ and its derivatives, so we can expand sums and
pull factors from inside $F$:
\begin{align*}
F(x_1,\dots,x_2,u_1+u_2,\frac{\partial u_1}{\partial x_1}+\frac{\partial u_2}{\partial x_1},\dots)
&=
F(x_1,\dots,x_2,u_1,\frac{\partial u_1}{\partial x_1},\dots)
\\
&+
F(x_1,\dots,x_2,u_2,\frac{\partial u_2}{\partial x_1},\dots)=0
\\
F(x_1,\dots,x_2,\lambda u_1,\frac{\partial \lambda u_1}{\partial x_1},\dots)
&=
\lambda
F(x_1,\dots,x_2,u_1,\frac{\partial u_1}{\partial x_1},\dots)
=0
\end{align*}
Thus linear combinations of $u_1$ and $u_2$ are solutions too.
\end{proof}

Linear cominations allow us to first find as many different solutions
$u_1$, $u_2$, $u_3,\dots$ and then to use suitable coefficients $a_k$
to combine them into a solution
\[
u(x)=\sum_{i=1}^\infty a_ku_k
\]
that also satisfies the boundary conditions.

Inhomogeneous linear partial differential equations can be solved using
this method too.
As pointed out in chapter~\ref{chapter:terminology-and-notation},
we first have to find a particular solution $u_p$ which solves the
inhomogeneous partial differential equation independently
of any boundary conditions.
The separation method can be helpful for this too.
Then the problem is reduced to finding a solutions $u_h$ to the homogeneous
equations with boundary condition $g-u_p$ on $\partial\Omega$.
Using the separation method, we can build up $u_h$ as a linear combination
of solutions.

The coefficients $a_k$ for the linear combination often lead us into
Fourier theory or some generalization of it.
Such partial solutions often have immediate physical significance.
In mechanical or electrical engineering they appear as vibration modes,
in quantum mechanics as energy states.



%
% membrane.tex -- XXX
%
% (c) 2019 Prof Dr Andreas Mueller
%
\section{Schwingende rechteckige Membran}
\rhead{Rechteckige Membran}
\index{Membran!rechteckig}
Wir betrachten die Schwingung einer rechteckigen Membran, die am Rande
des Gebietes
\[
R=\{(x,y)\,|\,0\le x\le a,0\le y\le b\} =(0,a)\times(0,b)
\]
eingespannt ist. Zur Zeit $t=0$ sei die Form der Membran durch die
Funktion $f(x,y)$ gegeben.
Für beliebige Zeit $t\ge 0$ wird sie beschrieben durch eine Funktion $u(x,y,t)$,
welche der Differentialgleichung
\[
\frac1{c^2}\frac{\partial^2u}{\partial t^2}=\frac{\partial^2u}{\partial x^2}+\frac{\partial^2u}{\partial y^2}
\]
genügt mit den Anfangsbedingungen
\begin{align*}
u(x,y,0)&=f(x,y)\quad\forall 0\le x\le a,0\le y\le b,
\\
\frac{\partial}{\partial t}u(x,y,0)&=g(x,y)\quad\forall 0\le x\le a,0\le y\le b
\end{align*}
und den Randbedingungen
\begin{align*}
u(0,y,t)&=0&u(a,y,t)&=0&\forall t\ge 0,0\le y\le b,\\
u(x,0,t)&=0&u(x,b,t)&=0&\forall t\ge 0,0\le x\le a.
\end{align*}

\subsection{Separation der Zeit}
\index{Separation}
Nach der in der Einleitung motivierten Idee suchen wir Lösungen also
Produkt einer Funktion $T(t)$, die nur von der Zeit abhängt, und einer Funktion
$\varphi(x,y)$, welche nur vom Ort abhängt, also
\[
u(x,y,t)=T(t)\cdot\varphi(x,y).
\]
Leider kann ein einzelnes solches Produkt nicht alle Anfangsbedingungen
erfüllen. Wäre dies nämlich möglich, müsste $\varphi(x,y)\sim f(x,y)$
sein und alle Teile der Membran würden im Gleichtakt hin und her schwingen.
Simulationen oder physikalische Experimente zeigen aber, dass es
Anfangsbedingungen gibt, bei denen die Teile der Membran gegenläufig
schwingen.

Anderseits, muss die Lösung auf jeden Fall die Randbedingung erfüllen,
es muss also gelten
\begin{align*}
\varphi(0,y)&=0&\varphi(a,y)&=0&0\le y\le b\\
\varphi(x,0)&=0&\varphi(x,b)&=0&0\le x\le a
\end{align*}
Setzen wir diesen Ansatz für $u$ in der Wellengleichung ein,
erhalten wir
\[
\frac1{c^2}T''(t)\varphi(x,y)=T(t)\left(
\frac{\partial^2\varphi}{\partial x^2}
+
\frac{\partial^2\varphi}{\partial y^2}
\right)
\]
Wir suchen eine Funktion $u$, die nicht identisch verschwindet,
es gibt also einige Zeitpunkte $t$ und Orte $(x,y)$, an denen $T(t)$
und $\varphi(x,y)$ nicht verschwinden. An diesen Stellen kann man die
Gleichung umformen in
\begin{equation}
\frac1{c^2}\frac{T''(t)}{T(t)}
= \frac1{\varphi(x,y)}\left( \frac{\partial^2\varphi}{\partial x^2}
+ \frac{\partial^2\varphi}{\partial y^2} \right)
\label{separiertMembran}
\end{equation}
Die rechte Seite hängt nur
vom Ort ab, darf sich also nicht ändern, wenn man die Zeit $t$ variert.
Als Funktion der Zeit muss die linke Seite eine Konstante sein,
es gibt also ein $k$ mit der Eigenschaft
\[
\frac1{c^2}\frac{T''(t)}{T(t)}=k
\qquad\Leftrightarrow\qquad
T''(t)=k T(t).
\]
Diese gewöhnliche Differentialgleichung hat Lösungen der Form 
$e^{\pm\sqrt{k}t}$ für positives $k$. Für negatives $k$ sind $\sin\sqrt{k}t$ 
und $\cos\sqrt{k}t$ Lösungen.
Aus physikalischer Sicht sind nur Lösungen mit Schwingungscharakter sinnvoll,
wir können daher annehmen, dass $k<0$.
Ein solches $k$ lässt sich in der Form $k=-\lambda^2$ schreiben.
Es gibt also ein $\lambda$ mit der Eigenschaft
\[
\frac1{c^2}\frac{T''(t)}{T(t)}=-\lambda^2
\]
oder
\[
T''(t)=-c^2\lambda^2 T(t).
\]
Dies ist eine gewöhnliche Differentialgleichung zweiter Ordnung, welche mit
bekannten Methoden gelöst werden kann.
Die allgemeine Lösung dieser Gleichung ist von der Form
\[
A\cos c\lambda t+B\sin c\lambda t.
\]

\subsection{Reduktion auf ein Eigenwertproblem}
\index{Eigenwertproblem}
Die linke Seite von (\ref{separiertMembran}) hängt nur von der Zeit ab, darf sich
also nicht ändern, wenn man $x$ oder $y$ variert. Als Funktion des Ortes
muss die rechte Seite also ebenfalls konstant sein:
\begin{align*}
\frac1{\varphi(x,y)}\left(
\frac{\partial^2\varphi}{\partial x^2}
+
\frac{\partial^2\varphi}{\partial y^2}
\right)&=-\lambda^2\\
\frac{\partial^2\varphi}{\partial x^2}
+
\frac{\partial^2\varphi}{\partial y^2}
=\Delta\varphi
&=-\lambda^2
\varphi(x,y)
\end{align*}
Die gesuchte Funktion $\varphi$ ist also ein Eigenvektor des linearen
Operators $\Delta$ zum Eigenwert $-\lambda^2$.
Nur die Eigenwerte des Operator $\Delta$ kommen also für die
Zeitabhängigkeitsgleichung in Frage.

\subsection{Separation von $x$ und $y$}
\index{Separation}
Für das Eigenwertproblem können wir erneut den Separationsansatz
\[
\varphi(x,y)=X(x)\cdot Y(y)
\]
versuchen.
Einsetzen in die Differentialgleichung ergibt
\begin{align*}
X''(x)Y(x)+X(x)Y''(y)&=-\lambda^2 X(x)Y(y)
\\
\frac{X''(x)}{X(x)}+\frac{Y''(y)}{Y(y)}&=-\lambda^2
\end{align*}
Jeder der Brüche hängt nur von jeweils einer Variable ab, was nur
möglich ist, wenn beide Terme konstant sind. Damit ist das Problem
reduziert auf zwei Gleichungen
\begin{align*}
X''(x)&=-\lambda_1^2X(x)\\
Y''(y)&=-\lambda_2^2Y(y)\\
\lambda_1^2+\lambda_2^2&=\lambda^2
\end{align*}
Die allgemeinen Lösungen dieser Gleichungen, die auch die Randbedingung
bei $x=0$ bzw.~$y=0$ erfüllt, sind
\begin{align*}
X(x)&=A\sin \lambda_1x\\
Y(y)&=B\sin \lambda_2y
\end{align*}
Die Randbedingungen für $x=a$ und $y=b$ können nur erfüllt werden,
wenn $\lambda_1a$ und $\lambda_2b$ Vielfache von $\pi$ sind, also
\[
\lambda_1=\frac{k\pi}a
\qquad
\text{und}
\qquad
\lambda_2=\frac{l\pi}b
\]
Die möglichen Werte von $\lambda$ sind also
\[
\lambda_{kl}^2=\left(\frac{k^2}{a^2} + \frac{l^2}{b^2}\right)\pi^2,\qquad k,l\in\mathbb Z
\]
Damit kann man jetzt die allgemeine Lösung des Schwingungsproblems aus den
Teillösungen
\[
\varphi_{kl}(x,y)=\sin \frac{k\pi}{a}x\sin\frac{l\pi}{b}y
\]
für das Eigenwertproblem
und den Teillösungen
\[
u_{kl}(x,y,t)
=
(A_{kl}\cos c\lambda_{kl} t+
B_{kl}\sin c\lambda_{kl} t)
\sin \frac{k\pi}{a}x\sin\frac{l\pi}{b}y
\]
für das zeitabhängige Problem
zu einer allgemeinen Lösung
\begin{equation}
u(x,y,t)=\sum_{k,l}
(A_{kl}\cos c\lambda_{kl} t+
B_{kl}\sin c\lambda_{kl} t)
\sin \frac{k\pi}{a}x\sin\frac{l\pi}{b}y
\label{allgemeineloesung}
\end{equation}
zusammensetzen.

\subsection{Anfangsbedingungen}
\index{Anfangsbedingungen}
Die allgemeine Lösung muss jetzt auch noch die Anfangsbedingung erfüllen:
\begin{align*}
\sum_{k,l}A_{kl}
\sin \frac{k\pi}{a}x\sin\frac{l\pi}{b}y&=f(x,y)\\
\sum_{k,l}B_{kl}c\lambda_{kl}
\sin \frac{k\pi}{a}x\sin\frac{l\pi}{b}y&=g(x,y)\\
\end{align*}
Die Koeffizienten $A_{kl}$ und $B_{kl}$ können in einfachen Fällen mit
Koeffizientenvergleich und im Allgemeinen mit Hilfe der Theorie
der Fourierreihen berechnet werden.
\index{Fourierreihe}


%
% disk.tex -- XXX
%
% (c) 2019 Prof Dr Andreas Mueller
%
\section{Kreisgebiet}
\rhead{Kreisgebiet}
\index{Kreisgebiet}
\index{Kreisscheibe}
In diesem Abschnitt betrachten wir eine Kreisscheibe
\[
G=\{(x,y)\in\mathbb R^2|x^2+y^2 < R\}
\]
mit Radius $R$ als Definitionsbereich. Da sich dieses Gebiet durch
eine Streckung um den Faktor $\frac1R$ immer auf einen Einheitskreis
abbilden lässt, können wir ohne Verlust an Allgemeinheit vorausetzen,
dass $R=1$ ist.

Ein Kreisgebiet tritt zum Beispiel beim Problem auf, die Schwingungen
einer kreisförmigen Membran zu berechnen, wie sie bei einer Kesselpauke
vorkommen. Nach den Ergebnissen des ersten Kapitels suchen wir nach einer
Funktion $u$, welche auf $G$ die Gleichung
\[
\frac1{a^2}\frac{\partial^2 u}{\partial t^2}=\frac{\partial^2 u}{\partial x^2}+\frac{\partial^2 u}{\partial y^2}
\]
erfüllt. Wie bei der Schwingung der einer rechteckigen Platte
wird daraus mit dem Ansatz $ u(x,y,t)=u(x,y)\cdot T(t)$ ein
Eigenwertproblem:
\begin{align*}
T''(t)&=-a^2\lambda^2 T(t)\\
\Delta u(x,y)&=-\lambda^2u(x,y)
\end{align*}
Das Poissonproblem ist der Spezialfall $\lambda=0$.
\index{Poissonproblem}

\subsection{Polarkoordinaten}
\index{Polarkoordinaten}
Offenbar sind Polarkoordinaten speziell gut an das Problem angepasst, 
eine Randbedingung lässt sich zum Beispiel durch eine Funktion beschreiben,
welche nur vom Polarwinkel abhängt.
Eine schwingende kreisförmite Membran führt also auf die partielle
Differentialgleichung
\[
\frac{\partial^2u(r,\varphi)}{\partial t^2}=\Delta u(r,\varphi)
\]
mit der Randbedingung
\[
u(R,\varphi)=0,\qquad\varphi\in[0,2\pi],
\]
wobei wie oben $R$ der Radius der Membran ist.

Damit das Problem auf einem Kreisgebiet in Polarkoordinaten behandelt
werden kann,
brauchen wir einen Ausdruck für $\Delta u$ in Polarkoordinaten.
\begin{align}
x&=r\cos\varphi\\
y&=r\sin\varphi
\label{polarkoordinaten}
\end{align}
Um die Ableitungen nach $x$ und $y$ durch Ableitungen $\varphi$ und $r$ zu
ersetzen, leiten wir (\ref{polarkoordinaten}) nach $x$ und $y$ ab:
\begin{align*}
1&=
\frac{\partial r}{\partial x}\cos\varphi
-r\sin\varphi \frac{\partial\varphi}{\partial x}
&
0&=
\frac{\partial r}{\partial y}\cos\varphi
-r\sin\varphi \frac{\partial\varphi}{\partial y}
\\
0&=
\frac{\partial r}{\partial x}\sin\varphi
+r\cos\varphi \frac{\partial\varphi}{\partial x}
&
1&=
\frac{\partial r}{\partial y}\sin\varphi
+r\cos\varphi \frac{\partial\varphi}{\partial y}
\end{align*}
In Matrixschreibweise ist dies
\begin{align*}
\begin{pmatrix}1\\0\end{pmatrix}
&=
\begin{pmatrix}
\cos\varphi&-\sin\varphi\\
\sin\varphi&\cos\varphi
\end{pmatrix}
\begin{pmatrix}
\frac{\partial r}{\partial x}\\
r\frac{\partial \varphi}{\partial x}
\end{pmatrix}
&
\begin{pmatrix}0\\1\end{pmatrix}
&=
\begin{pmatrix}
\cos\varphi&-\sin\varphi\\
\sin\varphi&\cos\varphi
\end{pmatrix}
\begin{pmatrix}
\frac{\partial r}{\partial y}\\
r\frac{\partial \varphi}{\partial y}
\end{pmatrix}
\end{align*}
Die $2\times2$ Matrix ist eine Drehmatrix, die Inverse findet man, indem man
$\varphi$ durch $-\varphi$ ersetzt. Die Multiplikation auf der linken Seite
ergibt jeweils die erste bzw. zweite Spalte der Drehmatrix zum
Winkel $\varphi$:
\begin{align*}
\cos\varphi
&=\frac{\partial r}{\partial x}
&&
&
\sin\varphi
&=
\frac{\partial r}{\partial y}
&&
\\
-\sin\varphi
&=r\frac{\partial \varphi}{\partial x}
&\Rightarrow\quad
\frac{\partial\varphi}{\partial x}&=-\frac1r\sin\varphi
&
\cos\varphi
&=
r\frac{\partial\varphi}{\partial y}
&\Rightarrow\quad
\frac{\partial\varphi}{\partial y}&=\frac1r\cos\varphi
\end{align*}
Mit diesen Formeln können wir jetzt die höheren Ableitungen
von $u$ nach  $x$ und $y$ durch Ableitungen nach $r$ und $\varphi$
ersetzen.

Die partiellen Ableitungen von $\varphi$ nach $x$ und $y$ sind
\begin{align*}
\frac{\partial u}{\partial x}
&=
\frac{\partial u}{\partial r}
\frac{\partial r}{\partial x}
+
\frac{\partial u}{\partial\varphi}
\frac{\partial \varphi}{\partial x}
=
\frac{\partial u}{\partial r}
\cos\varphi
-
\frac{\partial u}{\partial\varphi}
\frac1r\sin\varphi
\\
\frac{\partial u}{\partial y}
&=
\frac{\partial u}{\partial r}
\frac{\partial r}{\partial y}
+
\frac{\partial u}{\partial\varphi}
\frac{\partial \varphi}{\partial y}
=
\frac{\partial u}{\partial r}
\sin\varphi
+
\frac{\partial u}{\partial\varphi}
\frac1r\cos\varphi
\end{align*}
Die zweiten Ableitungen sind
\begin{align*}
\frac{\partial^2u}{\partial x^2}
&=
\frac{\partial}{\partial r}
\left(
\frac{\partial u}{\partial r}
\cos\varphi
-
\frac{\partial u}{\partial\varphi}
\frac1r\sin\varphi
\right)
\frac{\partial r}{\partial x}
+
\frac{\partial }{\partial \varphi}
\left(
\frac{\partial u}{\partial r}
\cos\varphi
-
\frac{\partial u}{\partial\varphi}
\frac1r\sin\varphi
\right)
\frac{\partial\varphi}{\partial x}
\\
&=
\frac{\partial}{\partial r}
\left(
\frac{\partial u}{\partial r}
\cos\varphi
-
\frac{\partial u}{\partial\varphi}
\frac1r\sin\varphi
\right)
\cos\varphi
-
\frac{\partial }{\partial \varphi}
\left(
\frac{\partial u}{\partial r}
\cos\varphi
-
\frac{\partial u}{\partial\varphi}
\frac1r\sin\varphi
\right)
\frac1r\sin\varphi
\\
&=
\frac{\partial^2u}{\partial r^2} \cos^2\varphi
-
\frac{\partial^2u}{\partial r\partial\varphi} \frac1r\sin\varphi \cos\varphi
+
\frac{\partial u}{\partial\varphi} \frac1{r^2}\sin\varphi\cos\varphi
\\
&\quad
-
\frac{\partial^2u}{\partial\varphi\partial r}\frac1r \cos\varphi\sin\varphi
+\frac{\partial u}{\partial r}\frac1r\sin^2\varphi
+\frac{\partial^2u}{\partial\varphi^2}
\frac1{r^2}\sin^2\varphi
+\frac{\partial u}{\partial\varphi}\frac1{r^2}\cos\varphi\sin\varphi
\\
\frac{\partial^2u}{\partial y^2}
&=
\frac{\partial}{\partial r}
\left(
\frac{\partial u}{\partial r}
\sin\varphi
+
\frac{\partial u}{\partial\varphi}
\frac1r\cos\varphi
\right)
\frac{\partial r}{\partial y}
+
\frac{\partial}{\partial \varphi}
\left(
\frac{\partial u}{\partial r}
\sin\varphi
+
\frac{\partial u}{\partial\varphi}
\frac1r\cos\varphi
\right)
\frac{\partial \varphi}{\partial y}
\\
&=
\frac{\partial}{\partial r}
\left(
\frac{\partial u}{\partial r}
\sin\varphi
+
\frac{\partial u}{\partial\varphi}
\frac1r\cos\varphi
\right)
\sin\varphi
+
\frac{\partial}{\partial \varphi}
\left(
\frac{\partial u}{\partial r}
\sin\varphi
+
\frac{\partial u}{\partial\varphi}
\frac1r\cos\varphi
\right)
\frac1r\cos\varphi
\\
&=
\frac{\partial^2u}{\partial r^2}\sin^2\varphi
+\frac{\partial^2u}{\partial r\partial\varphi}\frac1r\cos\varphi\sin\varphi
-\frac{\partial u}{\partial\varphi}\frac1{r^2}\cos\varphi\sin\varphi
\\
&\quad
+
\frac{\partial^2u}{\partial\varphi\partial r}\frac1r\sin\varphi\cos\varphi
+\frac{\partial u}{\partial r}\frac1r\cos^2\varphi
+\frac{\partial^2u}{\partial \varphi^2}\frac1{r^2}\cos^2\varphi
-\frac{\partial u}{\partial \varphi}\frac1{r^2}\sin\varphi\cos\varphi
\end{align*}
\index{Laplace-Operator!in Polarkoordinaten}
Die Summe dieser zwei Terme ist die gesucht Darstellung des Laplace-Operators
in Polarkoordinaten:
\begin{align*}
\frac{\partial^2u}{\partial x^2}+\frac{\partial^2u}{\partial y^2}
&=
\frac{\partial^2u}{\partial r^2}
+\frac{\partial u}{\partial r}\frac1r
+\frac{\partial^2u}{\partial\varphi^2}\frac1{r^2}
\\
&=
\left(\frac1r\frac{\partial}{\partial r}r\frac{\partial}{\partial r}+\frac1{r^2}\frac{\partial^2}{\partial \varphi^2}\right)u
\end{align*}
Darstellungen des Laplace-Operators in weiteren Koordinatensystemen können
in jeder einigermassen vollständigen Formelsammlung gefunden werden.

\subsection{Separation der Ortsvariablen}
Die Lösung $u(r,\varphi)$ des Eigenwertproblems setzen wir wieder
als Produkt einer Funktion
$R(r)$
nur von  $r$ und einer Funktion $\Phi(\varphi)$ nur von $\varphi$ an.
Mit der im vorangegangenen Abschnitt gefundenen Formel für den Laplace-Operator
in Polarkoordinaten erhalten wir jetzt die Gleichungen
\begin{align*}
\Delta u=
\biggl(R''(r) + \frac1rR'(r)\biggr)\Phi(\varphi)
+\frac1{r^2}R(r)\Phi''(\varphi)&=-\lambda^2 R(r)\cdot\Phi(\varphi)\\
\frac{r^2R''(r)+rR'(r)}{R(r)}+\frac{\Phi''(\varphi)}{\Phi(\varphi)}
&=-\lambda^2 r^2
\\
\frac{r^2R''(r)+rR'(r)}{R(r)}+\lambda^2 r^2&=-\frac{\Phi''(\varphi)}{\Phi(\varphi)}
\end{align*}
Da die rechte Seite nur von $\varphi$ abhängt, die linke Seite aber nur von $r$,
müssen beide Seiten konstant sein, wir nennen diese Konstante $\mu^2$.
Damit sind die Variablen separiert:
\begin{align}
\Phi''(\varphi)+\mu^2\Phi(\varphi)&=0\label{phigleichung}\\
r^2R''(r)+rR'(r)+(\lambda^2 r^2-\mu^2)R(r)&=0\label{rgleichung}
\end{align}

\subsection{Lösung der separierten Differentialgleichungen}
Die allgemeine Lösung der Gleichung (\ref{phigleichung}) ist
\[
\Phi(\varphi)=A\cos\mu\varphi +B\sin\mu\varphi.
\]
Dies ist nur dann $2\pi$-periodisch, wenn $\mu$ eine ganze
Zahl ist, also $\mu=k$ mit $k\in\mathbb Z$.

Die Gleichung (\ref{rgleichung}) für $R$ bekommt damit die Form
\[
r^2R''(r)+rR'(r)+(\lambda^2 r^2-k^2)R(r)=0,
\]
sie ist verwandt mit der Besselschen Differentialgleichung.
Die Funktion $P(\varrho)=R(\varrho/\lambda)=R(r)$ hat die Ableitungen
\begin{align*}
\varrho P'(\varrho)&=\frac{\varrho}{\lambda}R'(\varrho/\lambda)=rR'(r)\\
\varrho^2 P''(\varrho)&=\frac{\varrho^2}{\lambda^2}R'(\varrho/\lambda)=r^2R''(r)
\end{align*}
und erfüllt somit die Besselsche Differentialgleichung
\[
\varrho^2P''(\varrho)+\varrho P'(\varrho)+(\varrho^2-k^2)P(\varrho).
\]
Lösungen der Besselschen Differentialgleichungen sind die Besselfunktionen
\[
P(\varrho)=J_{\pm k}(\lambda r)=R(r)
\]
Wie bei der rechteckigen Membran kann die allgemeine Lösung jetzt aus
den Teillösungen zusammengesetzt werden.


%
% initial.tex -- XXX
%
% (c) 2019 Prof Dr Andreas Mueller
%
\rhead{Anfangsbedingungen}
\section{Anfangsbedingungen}
In den bisherigen Beispielen haben wir Lösungen einer partiellen
Differentialgleichung gesucht und gefunden, welche bestenfalls einen
Teil der Randbedingungen erfüllt haben.
So haben wir zwar sichergestellt, dass die schwingende Membran eingespannt
bleibt, aber die Auslenkung der Membran zu Beginn haben wir ignoriert.

Um zu verstehen, wie die Anfangsbedingungen ebenfalls berücksichtig
werden können, betrachten wir die Wellengleichung
\[
\frac{\partial^2 u}{\partial t^2}=\frac{\partial^2 u}{\partial x^2}
\]
auf dem Gebiet $(t,x)\in\mathbb R\times [0,\pi]$
mit den Randbedingungen
\[
u(t,0)=u(t,\pi)=0.
\]
Wir verwenden den Separationsansatz
$u(t,x)=T(t)\cdot X(t)$, welcher uns wie früher dargestellt auf eine
Gleichung
\[
\frac{T''(t)}{T(t)}=\frac{X''(x)}{X(x)}=-\lambda^2
\]
führt.
Die Gleichung 
\[
X''(x)=-\lambda^2 X(x)
\]
hat als Lösung Linearkombinationen von Sinus- und Kosinusfunktionen
\[
X(x)=A\cos\lambda x+B\sin\lambda x.
\]
Damit die Anfangsbedingung am linken Rand erfüllt ist, muss $A=0$
sein. Am rechten Rand bleibt daher nur $B\sin\lambda \pi$, und wir
müssen $B\ne 0$ annehmen, da sonst die ganze Lösung verschwinden
würde. $\sin\lambda \pi$ wird aber nur dann verschwinden, wenn
$\lambda$ eine ganze Zahl ist, also
\[
X_k(x)=B\sin kx, \quad 0<k\in\mathbb Z.
\]
Die dazu passende Lösung von $T''(t)=-k^2T(t)$ hat genau die
gleiche Form, so dass die allgemeine Lösung zum Wert $\lambda=k$
\[
u_k(t,x)=\sin kx\left(A_k\cos kt+B_k\sin kt\right)
\]
ist.

Diese Teillösungen $u_k(t,x)$ erfüllen bereits die Differentialgleichung
und die Randbedingungen. Noch nicht erfüllt werden die Anfangsbedingungen
zur Zeit $t=0$. Wir geben sie in der Form
\begin{align*}
u(0,x)&=f(x)\quad x\in[0,\pi]\\
\frac{\partial u}{\partial t}(0,x)&=g(x)\quad x\in[0,\pi]
\end{align*}
vor.

Wir suchen jetzt also eine Lösung in der Form
\[
u(t,x)=\sum_{k=1}^{\infty}
\left(A_k\cos kt+B_k\sin kt\right)
\sin kx,
\]
welche die Anfangsbedingung erfüllt. Durch Einsetzen erhält
man
\begin{align*}
\sum_{k=1}^{\infty}
A_k \sin kx
&=f(x)
\\
\sum_{k=1}^{\infty}
B_kk\sin kx
&=g(x)
\end{align*}
für $x\in[0,\pi]$.
Die Lösung $u(t,x)$ kann also vollständig bestimmt werden, indem man
die Anfangsbedingungen in eine Fourier-$\sin$-Reihe entwickelt. Sind
$\hat f(k)$ und $\hat g(k)$ die Fourier-Koeffizienten, wird die
vollständige Lösung
\[
u(t,x)
=
\sum_{k=1}^{\infty}(\hat f(k)\cos kt+\hat g(k)k\sin kt)\sin kx.
\]
Mit geeigneten Voraussetzungen an die Funktionen $f$ und $g$ werden
diese Reihen konvergieren.


%
% summary.tex -- XXX
%
% (c) 2019 Prof Dr Andreas Mueller
%
\section{Zusammenfassung: Separationsverfahren}
Aus diesen Beispielen lässt sich jetzt das allgemeine Prinzip 
ableiten. Gegeben ist eine partielle Differentialgleichung
beliebiger Ordnung mit unabhängigen Variablen $x_1,\dots,x_n$.
Ziel ist, die Differentialgleichung auf eine solche mit weniger
unabhängigen Variablen zu reduzieren. Sobald man die Reduktion
bis auf eine Variable geschafft hat, hat man die partielle
Differentialgleichung in gewöhnliche Differentialgleichungen
umgewandelt, typischerweise in Randwertprobleme,
die man mit gekannten Techniken lösen kann.

Da man am Schluss die Lösung aus den Teillösungen zusammensetzen
muss, die die separierten Gleichungen liefern, ist dieses Vorgehen
nur bei linearen PDGL sinnvoll. Wir gehen also im folgenden von
einer linearen PDGL aus.

Wir gehen also von einer Differentialgleichung für die Funktion
$u(x_1,\dots,x_n)$ aus, und wollen die Variable $x_1$ separieren.
Dazu geht man wie folgt vor.
\begin{enumerate}
\item Setzt die Lösung $u$ der Differentialgleichung in der
Form eines Produktes an:
\[
u(x_1,\dots,x_n)=X_1(x_1)u_1(x_2,\dots,x_n).
\]
\item Einsetzen des Ansatzes in die Differentialgleichung.
\item
Mit etwas Glück lassen sich die Terme, die
$X_1$ und $u_1$ enthalten trennen und auf verschiedene Seiten
des Gleichheitszeichens bringen.
Da die Lösung $u\equiv 0$ nicht interessant ist, kann man
zu diesem Zweck durch $u$ dividieren, die Gleichung muss
ausserhalb der Nullstellen von $u$ immer noch erfüllt sein.
Die Gleichung hat jetzt also die Form
\[
F(x_1,X_1,X_1',\dots,X_1^{(n)})
=
G(x_2,\dots,x_n,u_1,\partial_2u_1,\dots\partial_nu_n,\dots)
\]
\item
Da die linke Seite nur von $x_1$, die rechte nur von $x_2,\dots,x_n$
abhängt, müssen beide Konstant sein, wir haben also die ursprüngliche
PDGL in zwei Differentialgleichungen zerlegt:
\begin{equation}
\begin{aligned}
F(x_1, X_1,X_1',\dots, X_1^{(n)})&=k\\
G(x_2,\dots,x_n,u_1,\partial_2u_1,\dots\partial_nu_n,\dots)&=k
\end{aligned}
\label{separiert}
\end{equation}
wobei $k$ eine Konstante ist.
Dies sind zwei Differentialgleichungen, die erste ist eine
gewöhnliche Differntialgleichung, und falls $n>2$ ist die zweite
eine partielle Differentialgleichung, die unter Umständen noch
einmal mit dem gleichen Verfahren behandelt werden muss.
Gesucht werden alle Konstanten,
für welche beide Gleichungen eine Lösung haben.
\item Sind $X_1(k,x_1)$ und $u_1(k,x_2,\dots,x_n)$ Lösungen der
Gleichungen (\ref{separiert}), dann sind 
\[
u_k(x_1,\dots,x_n)=X_1(k,x_1)u_1(k,x_2,\dots,x_n)
\]
Lösungen der ursprünglichen PDGL. Die allgmeine Lösung ist daher
eine Summe
\[
u(x_1,\dots,x_n)=
\sum_{k}
a_k
u_k(x_1,\dots,x_n)=X_1(k,x_1)u_1(k,x_2,\dots,x_n),
\]
wobei die Summe über die möglichen $k$ zu erstrecken ist.
\item
Zur Erfüllung von Randbedingungen müssen jetzt die Koeffizienten
$a_k$ bestimmt werden, für die die Randtterme korrekt werden.
\end{enumerate}
Das Verfahren kann an zwei Stellen zusammenbrechen:
\begin{itemize}
\item In Schritt 3 wird vorausgesetzt, dass die Trennung in 
Terme, die $x_1$ enthalten  und solche, die $x_1$ nicht enthalten
möglich ist. Dies ist nicht automatisch der Fall, kann aber in
vielen praktisch wichtigen Fällen durch Wahl eines geeigneten
Koordinatensystems erreicht werden.
\item In Schritt 6 wird vorausgesetzt, dass die Randbedingungen
mit Hilfe der Randwerte der Teillösungen $u_k$ erfüllt werden
können. In den Beispielen in diesem Kapitel wurde dafür jeweils
die nicht triviale Fourier-Theorie benötigt. 
\end{itemize}



\section{Summary}
\begin{enumerate}
\item
Using a suitable {\em ansatz}, a partial differential equation can be
decomposed into a set of coupled ordinary differential equations.
\item
The choice of ansatz is essential predicated on the geometry of the
domain, in particular on the coordinate system and the type of equation.
\item
For linear partial differential equations, the partial solutions found
by the previous steps can be linearly combined to give more solutions
of the initial partial differential equation.
\item
The central idea of the method is that an equation where one side
depends only on $x$ and the other side only on $y$ can only be constant.
\item
In the case of partial differential equations of second order,
which can very often be treated using a product ansatz, the separation
usually leads to an eigenvalue problem with fewer variables.
\end{enumerate}

%
% tsunami.tex
%
% (c) 2011 Prof Dr Andreas Mueller, Hochschule Rapperswil
%

\section{Anwendung: Wellenausbreitung auf einer Kugel oder der Tsunami von 2011}
\index{Tsunami}
\index{Wellenausbreitung!auf der Kugeloberfl\"ache}
Am 11.~M"arz 2011 l"oste ein Erdbeben der St"arke 9 im japanischen
Meer, das Sendai Erdbeben, einen Tsunami aus, der grosse K"ustengebiete
\index{Sendai Erdbeben}
\index{Fukushima}
Japans verw"ustete, "uber
15000 Tote forderte und Unf"alle in mehreren Kernkraftwerken
ausl"oste, wovon der Unfall in Fukushima-Daichi mit einer
teilweisen Kernschmelze der schwerwiegendste war.
Tsunamis sind von Erdbeben ausgel"oste Wellen, die im offenen
Meer unscheinbar sind, aber wegen der mit kleiner werdender Wassertiefe
geringeren Ausbreitungsgeschwindigkeit in K"ustenn"ahe grosse
Amplituden erreichen k"onnen. Die Ausbreitung solcher Wellen
kann nat"urlich mit partiellen Differentialgleichungen modelliert
und berechnet werden. Die Abbildungen \ref{tsunamiausbreitung}
und \ref{tsunamienergie}
zeigt die mit einem Computer berechnete Ausbreitung des vom
Sendai-Erdbeben erzeugten Tsunami durch den Pazifik.
Dieses Modell ber"ucksichtigt offenbar die Topographie des
Meeresbodens.

Eine direkte Berechnung der Wellenausbreitung mit der bisher
gelernten Theorie ist nat"urlich nicht m"oglich, dazu m"usste
Topographie und K"ustenlinie des Pazifik im Detail bekannt
sein. Als vereinfachtes
Modell k"onnen wir jedoch versuchen, die Wellenausbreitung auf
einer Kugeloberfl"ache zu verstehen, dies entspricht einer 
kugelf"ormigen Erde, die mit einem Meer konstanter Tiefe bedeckt
ist.
\index{Meer}

\begin{figure}
\begin{center}
\includegraphics[width=\hsize]{graphics/sendainoaa}
\end{center}
\caption{Ausbreitung des vom Sendai-Erdbeben vom 11.~M"arz 2011 
ausgel"osten Tsunami durch den Pazifik nach einer Simulation der NOAA.
\index{Pazifik}
\index{NOAA}
Hawai und andere Inseln reduzieren die Wassertiefe und damit die
Ausbreitungsgeschwindigkeit und verz"ogern damit die Ausbreitung
der Welle. Ebenfalls deutlich beobachtbar ist die Abschattung 
der Welle durch grosse Hindernisse wie Neuseeland.
\index{Neuseeland}
\label{tsunamiausbreitung}}
\end{figure}

\begin{figure}
\begin{center}
\includegraphics[width=\hsize]{graphics/sendaienergy}
\end{center}
\caption{Amplitude des Tsunami vom 11.~M"arz 2011.
Man beachte, dass durch die Wahl der Kartenprojektion 
die Grosskreise, entlang derer sich die Wellen ausbreiten,
zu S-Kurven gebogen werden. In K"ustenn"ahe nimmt die
Amplitude wegen der abnehmenden Wassertiefe und der damit
reduzierten Ausbreitungsgeschwindigkeit zu.
\label{tsunamienergie}}
\end{figure}


\subsection{Koordinaten und Randbedingungen}
Es interessieren uns nur die von einem Punkt aus erzeugte Wellen,
so wie dies bei einem Erdbeben der Fall ist. Die L"osung muss
notwendigerweise rotationssymmetrisch sein um eine Achse, die
durch den Ausgangspunkt verl"auft. 

Als Koordinatensystem auf einer Kugel verwenden wir Kugelkoordinaten
$(r,\vartheta,\varphi)$. $\vartheta$ ist die geographische Breite
vom Nordpol gemessen, der auch gleich der Ausgangspunkt der
Welle sein soll. $\varphi$ ist die geographische L"ange, wir
suchen jedoch eine L"osung, die von der geographischen Breite
unabh"angig ist. Ebenso interessiert uns der Radius $r$ nicht,
da wir uns auf die Kugeloberfl"ache beschr"anken wollen, wir
setzen daher $r=1$.

Gesucht ist also eine Funktion $u(t,\vartheta)$, welche die
Anfangsbedingungen
\begin{align*}
u(0,\vartheta)&=F(\vartheta)\\
\frac{\partial}{\partial t}u(0,\vartheta)&=G(\vartheta)
\end{align*}
erf"ullen m"ussen.

\subsection{Wellengleichung auf der Kugeloberfl"ache}
Die Wellengleichung auf der Kugeloberfl"ache entsteht als 
Einschr"ankung der dreidimensionalen Wellengleichung
\[
\frac1{c^2} \frac{\partial^2}{\partial t^2}u =\Delta u.
\]
Wir nehmen an, dass die Einheiten so gew"ahlt worden sind,
dass $c=1$ in diesen Einheiten gilt (dies erreicht man zum
Beispiel, wenn man als L"angeneinheit die in einer Zeiteinheit
zur"uckgelegte Strecke verwendet).
Der Laplace-Operator muss in Kugelkoordinaten ausgedr"uckt werden,
\index{Laplace-Operator!in Kugelkoordinaten}
\[
\Delta u
=
\frac1{r^2} \frac{\partial}{\partial r}r^2\frac{\partial}{\partial r}u
+
\frac1{r^2\sin\vartheta}
\frac{\partial}{\partial\vartheta}
\sin\vartheta
\frac{\partial}{\partial\vartheta}
u
+
\frac1{r^2\sin^2\vartheta}\frac{\partial^2}{\partial\varphi^2}u
=
\frac1{\sin\vartheta}
\frac{\partial}{\partial\vartheta}
\sin\vartheta
\frac{\partial}{\partial\vartheta}
u
\]
Die Wellengleichung lautet jetzt also noch
\begin{equation}
\frac{\partial^2u}{\partial t^2}=
\frac1{\sin\vartheta}
\frac{\partial}{\partial\vartheta}
\sin\vartheta
\frac{\partial}{\partial\vartheta}
u=0.
\label{tsunami-gleichung}
\end{equation}

\subsection{Separation}
F"ur die L"osung der Wellengleichung (\ref{tsunami-gleichung}) machen
wir jetzt den "ublichen Separationsansatz:
\[
u(t,\vartheta)=T(t)\Theta(\vartheta),
\]
und setzen ihn in die Differentialgleichung ein:
\[
T''(t)\Theta(\vartheta)=
T(t)
\frac1{\sin\vartheta}
\frac{d}{d\vartheta}
\sin\vartheta
\frac{d}{d\vartheta}\Theta(\vartheta)
\]
Da wir eine L"osung suchen, die nicht "uberall verschwindet,
d"urfen wir annehmen, dass $T$ und $\Theta$ ausser an einzelnen
Punkten nicht verschwinden, dass wir also ``meistens'' durch
$T(t)\Theta(\vartheta)$ teilen d"urfen. Damit erreichen wir
die gew"unschte Trennung der Variablen:
\begin{equation}
\frac{T''(t)}{T(t)}
=
\frac1{\Theta(\vartheta)}
\frac1{\sin\vartheta}
\frac{d}{d\vartheta}
\sin\vartheta
\frac{d}{d\vartheta}\Theta(\vartheta)
\label{tsunami-separiert}
\end{equation}
Die linke Seite ist nur von $t$ abh"angig, die rechte nur von $\vartheta$,
diese Gleichung kann also nur erf"ullt sein, wenn beide seiten konstant
sind.  Wir erhalten also zwei Gleichungen
\begin{align}
T''(t)&=mT(t)
\label{tsunami:zeitabh}
\\
\frac1{\sin\vartheta}
\frac{d}{d\vartheta}
\sin\vartheta
\frac{d}{d\vartheta}\Theta(\vartheta)
&=m\Theta(\vartheta).
\label{tsunami:winkelabh}
\end{align}
Es ist jetzt also zu ermitteln, f"ur welche Werte von $m$ die beiden
Gleichungen L"osungen haben. Dann k"onnen f"ur diese Werte von $m$ 
L"osungen der partiellen Differentialgleichung zusammengebaut werden,
mit denen sich dann beliebige L"osungen durch "Uberlangerungen
erf"ullen lassen m"ussen.

\subsection{Zeitabh"angigkeit}
Die Zeitabh"angigkeit (\ref{tsunami:zeitabh}) ist eine gew"ohnliche
Schwingungsdifferentialgleichung.
Die L"osungen sollten Schwingungscharakter haben, was nur zutrifft, wenn
$m<0$ ist. Die allgemeine L"osung ist dann
\[
T_m(t)=a_m\cos\sqrt{-m}t+b_m\sin\sqrt{-m}t
\]

\subsection{Winkelabh"angigkeit}
Die Differentialgleichung (\ref{tsunami:winkelabh}) f"ur $\Theta$
ist in dieser Form etwas unhandlich.
Daher ersetzen wir $\Theta(\vartheta)$ durch eine
Funktion $y(x)$ mit Hilfe der Substitution $x=\cos\vartheta$.
Die Ableitung nach $\vartheta$ kann mit Hilfe der Kettenregel
in eine Ableitung nach $x$ umgewandelt werden:
\[
\frac{d}{d\vartheta}\Theta(\vartheta)
=\frac{dy(x)}{dx}\frac{dx}{d\vartheta}
=-\sin\vartheta \frac{d}{dx} y(x)
\]
Setzt man dies in die Differentialgleichung ein, wird sie zu
\begin{align*}
\frac1{\sin\vartheta}
(-\sin{\vartheta})\frac{d}{dx}\sin\vartheta (-\sin\vartheta)
\frac{d}{dx}y(x)
&=
\frac{d}{dx}\sin^2\vartheta\frac{d}{dx}y(x)\\
&=
\frac{d}{dx}(1-\cos^2\vartheta)\frac{d}{dx}y(x)\\
&=
\frac{d}{dx}(1-x^2)\frac{d}{dx}y(x).
\end{align*}
Die gesuchten Funktionen sind also L"osungen der Differentialgleichung
\begin{equation}
\frac{d}{dx}(1-x^2)\frac{d}{dx}y(x)
=
my(x)
\label{tsunamieigenwertproblem}
\end{equation}
Die Funktionen $y(x)$ m"ussen im ganzen Interval $[-1,1]$ definiert
sein. Dies ist nicht unbedingt selbstverst"andlich, wie schon der Fall
$m=0$ zeigt. In diesem Fall kann man die Differntialgleichung
durch zweimaliges Integrieren l"osen:
\begin{align*}
\frac{d}{dx}(1-x^2)\frac{d}{dx}y(x)&=0\\
(1-x^2)\frac{d}{dx}y(x)&=C\\
\frac{d}{dx}y(x)&=\frac{C}{1-x^2}\\
y(x)&=C\int\frac{dx}{1-x^2}\\
&=\frac{C}2\int\frac{dx}{1-x}+\frac{C}2\int\frac{dx}{1+x}\\
&=-\frac{C}2\log(1-x)+\frac{C}2\log(1+x) +D\\
&=\frac{C}2\log\frac{1+x}{1-x} + D
\end{align*}
An beiden Intervallenden w"achst die Funktion "uber alle Grenzen,
es sei denn es sei $C=0$.

Mit Sicherheit auf dem ganzen Interval definiert w"aren Polynome,
wir k"onnten also einen Ansatz
\[
y(x)=a_0+a_1+a_2x^2+\dots a_nx^n
\]
probieren. Setzt man dies in die Differentialgleichung ein und
beh"alt nur die Terme vom Grad $x^n$, bekommt man auf der rechten
Seite von (\ref{tsunamieigenwertproblem}) $ma_nx^n$, auf
der linken Seite
\[
-\frac{d}{dx}x^2\frac{d}{dx}a_nx^n
=
-\frac{d}{dx}x^2na_nx^{n-1}
=
-\frac{d}{dx}na_nx^{n+1}
=
-n(n+1)a_nx^n
\]
Damit folgt: $m=-n(n+1)$, nur f"ur solche Werte kann
(\ref{tsunamieigenwertproblem}) ein Polynom vom Grad $n$ als L"osung
haben. Die Differentialgleichung wird jetzt zu
\begin{equation}
\frac{d}{dx}(1-x^2)\frac{d}{dx}y(x)+n(n+1)y(x)=0
\label{legendredgl}
\end{equation}

\subsection{Legendre-Polynome}
\index{Legendre-Polynom}
Die Differentialgleichung (\ref{legendredgl}) ist die Differentialgleichung
der Legendre-Polynome.
Das Legendre-Polynom $P_n(x)$ ist eine polynomiale L"osung von
(\ref{legendredgl}) mit $P_n(1)=1$.
Dies legt aber die Funktion nicht fest, es sind weitere Bedingungen
n"otig. Daher wird verlangt, dass die Polynome auch orthogonal
sein sollen, also die Bedingung
\[
\int_{-1}^1 P_k(x)P_l(x)\,dx=0\quad\text{f"ur $k\ne l$}
\]
erf"ullen. Damit werden die Polynome eindeutig.
Die ersten sechs Legendre-Polynome sind
\begin{align*}
P_0(x)&=1\\
P_1(x)&=x\\
P_2(x)&=\frac12(3x^2-1)\\
P_3(x)&=\frac12(5x^3-3x)\\
P_4(x)&=\frac18(35x^4-30x^2+3)\\
P_5(x)&=\frac18(63x^5-70x^3+15x)
\end{align*}
Ausserdem gilt
\[
\int_{-1}^1 P_k(x)^2\,dx = \frac{2}{2k+1}.
\]
Da man 
jede Funktion auf dem Interval $[-1,1]$ mit Polynomen approximieren kann,
kann man auch jede Funktion durch Linearkombinationen von Legendre-Polynomen
$P_n(x)$ schreiben. 

Die Koeffizienten kann man mit Hilfe eines Integrals finden. Setzt man
\[
f(x)=\sum_{k\ge 0} c_k P_k(x)
\]
und berechnet man das Integral
\[
\int_{-1}^1 f(x)P_l(x)\,dx
=
\sum_{k\ge 0} c_k \int_{-1}^1 P_k(x)P_l(x)\,dx
=
\frac{2c_k}{2k+1}
\]
folgt
\[
c_k=\frac{2k+1}{2}\int_{-1}^1P_k(x)f(x)\,dx.
\]
Die Koeffizienten $c_k$ sind sozusagen die ``Legendre-Koeffizienten''
der Entwicklung der Funktion $f(x)$ nach Legendre-Polynomen,
analog zu den Fourier-Koeffizienten auf einem Interval.

\subsection{Anfangsbedingungen}
\index{Anfangsbedingungen}
Unter Verwendung der Legendre-Polynome kann man jetzt die Wellengleichung
zu beliebigen Anfangsbedingungen l"osen.
Die L"osung der Differentialgleichung muss von der Form sein
\[
u(t, x)=\sum_{k=0}^{\infty}(a_k\cos \lambda_k t+b_k\sin\lambda_k t)P_k(x),
\]
wobei $\lambda_k=\sqrt{k(k+1)}$.
Die Koeffizienten m"ussen aus der Anfangsbedingung, also aus den 
Funktionen $F(\vartheta)=f(x)$ und $G(\vartheta)=g(x)$ bestimmt werden.
Die Anfangsbedingung f"ur $u(t,x)$ ergibt
\begin{align*}
u(0,x)
&=\sum_{k=0}^{\infty} a_kP_k(x)=f(x)
\end{align*}
F"ur $\partial_tu(t,x)$ ergibt sich entsprechend
\begin{align*}
\frac{\partial}{\partial t}u(0,x)
&=\sum_{k=0}^{\infty} \lambda_k b_kP_k(x)=g(x)
\end{align*}
Die Koeffizienten $a_k$ und $b_k$ kann man mit
\begin{align*}
a_k&=
\frac{2k+1}{2}\int_{-1}^1 P_k(x)f(x)\,dx
\\
b_k&=
\frac{2k+1}{2\lambda_k}\int_{-1}^1P_k(x)f(x)\,dx
\end{align*}
berechnen.

\subsection{Punktquelle}
\begin{figure}
\begin{center}
\includegraphics[width=\hsize]{graphics/tsunami0}
\end{center}
\caption{N"aherungsl"osung f"ur $N=25$ und $t=0$\label{tsunami0}}
\end{figure}
\begin{figure}
\begin{center}
\includegraphics[width=\hsize]{graphics/tsunami50}
\end{center}
\caption{N"aherungsl"osung f"ur $N=25$ und $t=1$\label{tsunami50}}
\end{figure}

Wir w"ahlen jetzt eine spezielle Anfangsbedingung:
\begin{align*}
f_\varepsilon(x)&=\begin{cases}
\frac1{\varepsilon}&\qquad 1-\varepsilon<x\le 1\\
0&\qquad\text{sonst}
\end{cases}
\\
g(x)&=0
\end{align*}
In einer kleinen Umgebung des Nordpoles ist der Wert 
$\frac1{\varepsilon}$, also sehr gross, in allen anderen Punkten $0$.
Offenbar sind die $b_k=0$, und es bleiben nur die 
$a_k$ zu berechnen. Dazu gilt:
\begin{align*}
a_k(\varepsilon)&=\frac{2k+1}{2}\int_{-1}^1P_k(x)f_\varepsilon(x)\,dx
\\
&=\frac{2k+1}{2}\int_{1-\varepsilon}^1P_k(x)\frac1{\varepsilon}\,dx
\end{align*}
Da uns nur der Grenzwert $\varepsilon\to 0$ interessiert, gehen wir
zur Grenze "uber
\begin{align*}
\lim_{\varepsilon\to 0} a_k(\varepsilon)
&=
\frac{2k+1}{2}\lim_{\varepsilon\to 0}\frac1{\varepsilon}\int_{1-\varepsilon}^1P_k(x)\,dx
\end{align*}
Mit einer Stammfunktion $I_k(x)$ von $P_k(x)$ wird dies zu
\begin{align*}
\lim_{\varepsilon\to 0} a_k(\varepsilon)
&=
\frac{2k+1}{2}\lim_{\varepsilon\to 0}\frac{I_k(1)-I_k(1-\varepsilon)}{\varepsilon}
\\
&=\frac{2k+1}{2}I_k'(1)=\frac{2k+1}{2}P_k(1)=\frac{2k+1}{2}
\end{align*}
Als L"osung bekommt man damit formal
\begin{equation}
u(t,x)
=
\sum_{k=0}^\infty \frac{2k+1}{2}P_k(x) \cos \sqrt{k(k+1)}t.
\end{equation}
Leider ist diese Reihe nicht konvergent, was angesichts der sehr
speziellen Anfangsbedingungen auch nicht zu erwarten war.
Wenn man sie aber nach $N$ Termen abbricht, und mit $\frac1{N^2}$ 
normiert, erh"alt man eine L"osungsfunktion die ein ungef"ahres
Bild f"ur die Wellenausbreitung ergibt.
In den Abbildungen \ref{tsunami0} und \ref{tsunami50} wurde die Reihe nach 25 Termen
abgebrochen.

%
% jacobi.tex -- Anwendung: Hamilton-Jacobi-Formulierung der Mechanik
%
% (c) 2012 Prof Dr Andreas Mueller, Hochschule Rapperswil
% $Id$
%
\section{Anwendung: Hamiltonsche Mechanik\label{hamilton-mechanik}}
In den bisherigen Bespielen wurde jeweils ein Separationsansatz mit
einem Produkt von Teilfunktion gew"ahlt.
Dieser Abschnitt soll illustrieren, dass in einigen F"allen auch
ein Separationsansatz mit einer Summe von Teilfunktionen
zum Ziel f"uhren kann.
Dieser Fall ist f"ur die Anwendungen recht wichtig, denn die
dabei entstehende partielle Differentialgleichungen hat eine
gen"ugend einfach Struktur, dass der erste Separationsschritt immer
durchgef"uhrt werden kann.

\subsection{Motivation}
\index{Newtonsche Gesetze}
\index{Planeten}
\index{Satelliten}
Die Newtonschen Gesetze reichen vollst"andig, um die Bewegung von
Planeten und Satelliten vorherzusagen.
Unterliegt
ein K"orper der Masse $m$ mit zeitabh"angigen Koordinaten $\vec x(t)$
einer ortsabh"angigen Kraft $\vec F(\vec x)$, dann muss die Bahnkurve 
$\vec x(t)$ die gew"ohnliche Differentialgleichung
\begin{equation}
m\frac{d^2}{dt^2}\vec x(t)=\vec F(\vec x(t))
\label{jacobi:newton}
\end{equation}
erf"ullen. Ausgehend von einem bekanten Anfangspunkt $\vec x_0$ und
der Anfangsgeschwindigkeit $\vec v_0$ l"asst sich durch
l"osen der Differentialgleichung die Bahnkurve bestimmen.

Diese Beschreibung hat jedoch ein paar praktisch bedeutsame Nachteile.
Oft ist das Kraftgesetz nicht exakt bekannt. Zum Beispiel werden
Satelliten in niedrigem Erdorbit\footnote{Low earth orbit, wenige 100km}
\index{Erdorbit}
von der zwar sehr d"unnen, aber nicht vernachl"assigbaren
Erdatmosph"are abgebremst.
\index{Erdatmosphare@Erdatmosph\"are}
\index{Merkur}
Oder der Planet Merkur ver"andert laufend seine Bahn um einen winzigen
Betrag, ein Effekt, den erst Albert Einstein mit seiner speziellen
Relativit"atstheorie erkl"aren konnte.
Man sieht sich also oft mit der Aufgabe konfrontiert, dass man zwar die
Bahn berechnen k"onnte, wenn das Kraftgesetz exakt zutreffen w"urde,
dass man aber die Bahnver"anderungen unter dem Einfluss kleiner
Abweichungen vom exakten Kraftgesetz bestimmen sollte.
\index{Luftwiderstand}

\begin{beispiel}
\index{Billardtisch}
Als Beispiel betrachten wir ein Kugel auf einem ideal horizontal angenommenen
Billardtisch. Die Bewegung dieser Kugel wird offenbar genau beschrieben
durch den Anfangspunkt und die Geschwindigkeit $\vec x_0$ und $\vec v_0$.
Die Kugel wird in Richtung $\vec x_0$ weiterrollen, dabei
aber langsamer werden und schliesslich zum Stillstand kommen.
Es gibt also eine Kurve $t\mapsto \vec x(t, \vec x_0, \vec v_0))$,
die beiden Vektoren $\vec x_0$ und $\vec v_0$ bestimmen die
Bahn vollst"andig.

Was passiert, wenn der Billiardtisch nicht exakt horizontal ist?
Offenbar wirkt dann eine kleine zus"atzlich Kraft auf die Billard-Kugel,
und zwar in Richtung der gr"ossten Neigung der Billardtischplatte.
Wenn die Neigung sehr klein ist, ist die Abweichung von 
$\vec x(t,\vec x_0,\vec v_0))$ sehr gering.
Man k"onnte die Beschreibung diese Modifikation dadurch beschreiben,
dass man $\vec x_0$ und $\vec v_0$ leicht anpasst.
Die tats"achliche Position der Kugel zur Zeit $t$ ist die Position,
die eine Billiardkugel auf einem horizontalen Tisch ausgehend von
einem etwas anderen Ausgangspunkt $\vec x_0(t)$ mit einer
etwas anderen Ausgangsgeschwindigkeit $\vec v_0(t)$
nach Zeit $t$ erreicht h"atte.
\end{beispiel}

Der Vorteil dieser Beschreibung besteht darin, dass sich die
Abweichung vom exakten Kraftgesetz direkt in der Zeitabh"angigkeit
von $x_0(t)$ und $v_0(t)$ ausdr"uckt. Ist das Kraftgesetz exakt
erf"ullt, bleiben $x_0(t)$ und $v_0(t)$ konstant. Ausserdem ist die
Berechnung der Bahn ganz einfach: zu $\vec x_0$ kommt einfach
ein Vielfaches von $\vec v_0$ hinzu.

Die Bahnen der Satelliten sind offenbar viel komplizierter. 
Ort und Geschwindigkeit sind nicht geeignet, die
Bahn auf diese Art und Weise zu charakterisieren, sie "andern laufend
dramatisch ihre Gr"osse. Niemand kann mit Leichtigkeit sagen, ob
die Bahn vom Punkt  $\vec x_0$ mit Anfangsgeschwindigkeit $\vec v_0$
irgendwie "ahnlich aussieht wie die Bahn vom Punkt $\vec x_1$
mit Anfangsgeschwindigkeit $\vec v_1$.

\begin{aufgabe}
\label{jacobi:aufgabe}
Zu einem beliebigen mechanischen System finde man einen Satz von 
Parametern 
$Q_i$ und $P_i$ so, dass sich {\em jede}
Bahn mit Hilfe dieser Parameter in der Form
\begin{equation}
x_i(t) = x_i(t,Q_1,\dots,Q_n,P_1,\dots,P_n)
\label{jacobi:aufgabekurve}
\end{equation}
beschreiben l"asst.
\end{aufgabe}

\begin{beispiel}
Betrachten wir wieder das Problem der Billardkugel, aber nehmen wir
diesmal an, dass die Kugel ohne Reibung rollt. Dann gilt
\begin{equation}
\begin{aligned}
x(t)&=x_0 + tv_{x0},\\
y(t)&=y_0 + tv_{y0}.
\end{aligned}
\label{jacobi:linear}
\end{equation}
In dieser Form ist das bereits eine L"osung der gestellten Aufgabe,
denn jede Bahnkurve l"asst such durch geeignete Wahl der
Parameter $\vec x_0$ und $\vec v_0$ charakterisieren.

Man kann aber noch mehr erreichen: man kann die Koordinaten $Q_i$ und $P_i$
so w"ahlen, dass die $P_i$ alle die Bedeutung einer Energie,
und die $Q_i$ die Bedeutung einer Zeit haben.
Die Funktion der $P_i$ k"onnen die Terme
\begin{equation*}
\begin{aligned}
P_x&=\frac12mv_x^2,&
P_y&=\frac12mv_y^2
\end{aligned}
\end{equation*}
"ubernehmen, also die Beitr"age der Bewegungskomponenten in $x$-
bzw.~$y$-Richtung zur Energie. Da die Kugel reibungsfrei rollt, sind
diese Gr"ossen konstant.

Die Gr"ossen $Q_x$ und $Q_y$ m"ussen so gew"ahlt werden, dass 
die Gleichungen (\ref{jacobi:linear}) die tats"achliche Bahn beschreiben.
Durch Aufl"osen von (\ref{jacobi:aufgabekurve}) nach $Q_i$ findet man
\begin{equation*}
\begin{aligned}
Q_x&=-\frac{x_0}{v_{0x}},\\
Q_y&=-\frac{y_0}{v_{0y}}.
\end{aligned}
\end{equation*}
Die Zahlen $-Q_x$ und $-Q_y$ sind also die Zeiten, zu denen die
Billardkugel die $x$- bzw.~$y$-Achse kreuzt.
\end{beispiel}

Die schwierigere Frage ist, ob jedes mechanische System eine
L"osung der Aufgabe \ref{jacobi:aufgabe} zul"asst.
Die Antwort passt ins Thema: ja, man kann eine solche Beschreibung
finden, aber man muss dazu die L"osung einer speziellen
partiellen Differentialgleichung finden. Oft l"asst sich die
Differentialgleichung mit einem Separationsansatz l"osen.

\subsection{Hamilton-Jacobi-Formalismus}
\index{Hamilton-Jacobi-Formalismus}
In der Hamiltonschen Beschreibung der Mechanik geht man aus von
der Gesamtenergie $H(x_i, p_i)$ ausgedr"uckt durch Ort und Impuls.
F"ur das Beispielproblem ist
\index{Hamilton-Funktion}
\begin{equation}
H(x_i,p_i)=\frac1{2m}(p_x^2+p_y^2).
\label{jacobi:hamilton:funktion}
\end{equation}
Der Zusammenhang zwischen den Koordinaten und den Impulsen ist dann
durch die Hamiltonschen Differentialgleichungen gegeben:
\begin{align}
\frac{d}{dt}x_i&=\frac{\partial H}{\partial p_i}
&\Rightarrow
\qquad \dot x_i&=\frac{p_i}{m}=v_i
\label{jacobi:hamilton:geschwindigkeit}
\\
\frac{d}{dt}p_i&=-\frac{\partial H}{\partial x_i}
&\Rightarrow
\qquad
\dot p_i&=m\ddot x_i=0
\label{jacobi:hamilton:newton}
\end{align}
\index{Hamilton-Gleichungen}
Die erste Gleichung (\ref{jacobi:hamilton:geschwindigkeit}) stellt
nur den Zusammenhang zwischen der Geschwindigkeit und dem Impuls
her. Die zweite Gleichung (\ref{jacobi:hamilton:newton}) besagt in
diesem Fall, dass keine Kraft wirkt. N"ahme man in 
(\ref{jacobi:hamilton:funktion}) noch ein Potential $V(x)$ hinzu,
w"urde die zweite Gleichung zu
\[
m\ddot x_i=-\frac{\partial V}{\partial x_i} = F_i(x),
\]
also genau dem Newtonschen Gesetz.

Nach Hamiliton ist jeder andere Satz von Koordinaten $Q_i$ und $P_i$
genauso geeignet zur Beschreibung des mechanischen Systems, solange die
Gleichungen 
(\ref{jacobi:hamilton:geschwindigkeit}) und (\ref{jacobi:hamilton:newton})
weiterhin G"ultigkeit haben. Jacobi und Hamilton haben eine Methode
angegeben, mit der man eine Koordinatentransformation finden kann.

\begin{satz}
\label{jacobi:satz}
Eine L"osung der Aufgabe (\ref{jacobi:aufgabe}) wird gefunden mit
Hilfe einer Funktion $S(x_i, t)$, die L"osung der partiellen
Differentialgleichung
\begin{equation}
\frac{\partial S}{\partial t}
=
H\biggl(
x_i,
\frac{\partial S}{\partial x_i}
\biggr)
\label{jacobi:hamilton:dgl}
\end{equation}
erf"ullt.
Eine L"osungsfunktion $S$ der Differentialgleichung enth"alt notwendigerweise
ein Anzahl von Integrationskonstanten $P_i$.
Die partiellen Ableitungen von $S$ nach diesen $P_i$
sind die neuen konstanten Bahnparameter $Q_i$
\begin{equation}
Q_i=\frac{\partial S}{\partial P_i}
\label{jacobi:hamilton:impuls}
\end{equation}
Die Gleichungen (\ref{jacobi:hamilton:impuls}) enthalten ausser den
Gr"ossen $P_i$ und $Q_i$ auch die Koordinaten $x_i$ und die Zeit $t$.
Durch Aufl"osen nach den Variablen $x_i$ l"asst sich die Bahnkurve
als Funktion
\[
x_i(t,Q_i,P_i)
\]
ausdr"ucken.
\end{satz}
Der Parametersatz $(Q_i,P_i)$ beschreibt also alle m"oglichen 
Bahnen. Zwei Bahnen k"onnen sehr einfach verglichen werden, wenn die
Paramter $Q_i$ und $P_i$ nahe beeinander sind, dann liegen auch
die Bahnen nahe beeinander.

Eine Begr"undung f"ur diesen Satz liegt ausserhalb der Ziele
dieser Vorlesung, wir wollen aber mit zwei Beispielen zeigen,
dass dieser Formulismus funktioniert und die L"osung der
Bewegungsdifferentialgleichungen des Systems erm"oglicht.

\subsection{Beispiele}
\subsubsection{Bewegung ohne "ausseren Krafeinfluss in zwei Dimensionen}
\index{Bewegung ohne Krafteinfluss}
Wir beginnen wieder bei der Hamilton-Funktion 
\[
H(p_x, p_y)=\frac1{2m}(p_x^2+p_y^2).
\]
Nach (\ref{jacobi:hamilton:dgl}) m"ussen wir jetzt also die
Differentialgleichung
\[
\frac1{2m}\biggl(
\biggl(\frac{\partial S}{\partial x}\biggr)^2
+
\biggl(\frac{\partial S}{\partial y}\biggr)^2
\biggr)=\frac{\partial S}{\partial t}
\]
f"ur die Funktion $S(t,x,y)$ l"osen. Wir verwenden dazu einen 
Separationsansatz der Form
\[
S(t,x,y)=S_1(x)+S_2(y) + S_3(t).
\]
Einsetzen in die Differentialgleichung liefert
\begin{equation}
\frac1{2m}( S_1'(x)^2+S_2'(y)^2)=S_3'(t).
\label{jacobi:kraeftefrei:sep1}
\end{equation}
Die linke Seite h"angt nicht von $t$ ab, die rechte h"angt
aber nur von $t$ ab, also sind beide Seiten konstant.
Wir nennen die Konstanten $P_1$. Damit l"asst sich
jetzt $S_3(t)$ bestimmen, es muss eine geeignete Integrationskonstante
geben, so dass
\[
S_3(t)=P_1t
\]
damit ist die Aufgabe \ref{jacobi:aufgabe} mindestens f"ur die Variable
$t$ bereits gel"ost.

Wir untersuchen jetzt die linke Seite von (\ref{jacobi:kraeftefrei:sep1}).
Auch diese Gleichung kann man unter Verwendung der Konstanten $P_3$
separieren:
\[
\frac1{2m} S_1'(x)^2
=
P_1-\frac1{2m}S_2'(y)^2.
\]
Die linke Seite h"angt nur von $x$ ab, die rechte nur von $y$, also
sind beide konstant. 
Wir nennen die Konstante $P_1$ und finden die L"osung
\[
S_1(x)
=
\sqrt{2mP_2}x.
\]
Jetzt kann man aber auch noch $S_2$ bestimmen. Es ist n"amlich 
\[
\frac1{2m} S_2'(y)^2
=P_1-P_2
\]
woraus sich wie vorhin die L"osung
\[
S_2(y)
=
\sqrt{2m(P_1-P_2)}y
\]
ergibt. Damit haben wir jetzt eine L"osung der Differentialgleichung:
\[
S(x,y,t)=
P_1t
+
\sqrt{2mP_2}x
+
\sqrt{2m(P_1-P_2)}y
\]
Nach den Regeln des Satzes \ref{jacobi:satz}, sind die zu verwendenden
Koordinaten die partiellen Ableitungen:
\begin{align*}
Q_1&=\frac{\partial S}{\partial P_1}
=
t + \sqrt{\frac{m}{2(P_1-P_2)}}y,\\
Q_2&=\frac{\partial S}{\partial P_2}
=
\sqrt{\frac{m}{2P_2}}x
-
\sqrt{\frac{m}{2(P_1-P_2)}}y
\end{align*}
Jetzt kann man nach $x$ und $y$ aufl"osen und damit die
Bahnkurve durch die Konstanten $Q_i$ und $P_i$ und die Zeit $t$
ausdr"ucken:
\begin{align*}
x&=\sqrt{\frac{2P_2}{m}}(Q_1+Q_2-t)\\
y&=\sqrt{\frac{2(P_1-P_2)}{m}}(Q_1-t).
\end{align*}
Wie erwartet beschreiben diese Gleichungen eine gleichf"ormige
Bewegung. Die Geschwindigkeit ist die Ableitung nach der Zeit,
also
\begin{align*}
v_x
&=
\sqrt{\frac{2P_2}{m}}
&\Rightarrow&&
P_2&=\frac{mv_x^2}2
\\
v_y
&=
\sqrt{\frac{2(P_1-P_2)}{m}}
&\Rightarrow&&
P_1&=P_2+\frac{mv_y^2}2=\frac{m}2(v_x^2+v_y^2)=\frac12mv^2
\end{align*}
Die physikalische Bedeutung von $P_1$ ist also die kinetische
Energie des
Gesamtsystems, w"ahrend $P_2$ die kinetische Energie darstellt, die in
der Bewegungskomponent in $x$-Richtung steckt.

\subsubsection{Schiefer Wurf}
\index{schiefer Wurf}
Wir betrachten den schiefen Wurf, der sich vom vorangegangenen
Beispiel durch ein zus"atzliches Gravitationspotential
unterscheidet. Die Gesamtenergie ist jetzt
\[
H(x,y,t)=\frac1{2m}(p_x^2+p_y^2)+mgy.
\]
Die Hamilton-Jacobi-Differentialgleichung lautet jetzt
\[
\frac1{2m}\biggl(
\biggl(\frac{\partial S}{\partial x}\biggr)^2
+
\biggl(\frac{\partial S}{\partial y}\biggr)^2
\biggr)
+mgy=\frac{\partial S}{\partial t} 
\]
Der selbe Separationsansatz wie vorhin f"uhrt auch wieder zu
\[
\frac1{2m}(S_1'(x)^2+S_2'(y)^2)+mgy=S_3'(t),
\]
woraus wieder folgt, dass $S_3'(t)=P_1$ konstant ist.

Wir k"onnen aber auch $x$ und $y$ separieren:
\begin{align*}
\frac1{2m}S_1'(x)^2&=P_1-\frac1{2m}S_2'(y)^2-mgy
%\\
%\frac1{\sqrt{2m}}S_1'(x)&=\sqrt{P_3-\frac1{2m}S_2'(y)^2+mgy}
\end{align*}
Die linke Seite h"angt nicht von $y$ ab, die rechte nicht von $x$, also sind
beide konstant, wir nennen die Konstante $P_2$, und bekommen
die L"osung $S_1=\sqrt{2mP_2} x$.

Die rechte Seite k"onnen wir jetzt auch l"osen:
\begin{align*}
P_2&=
P_1-\frac1{2m}S_2'(y)^2-mgy
\\
\frac1{2m}S_2'(y)^2
&=
P_1-P_2-mgy
\\
S_2'(y)&=\sqrt{
2m(P_1-P_2-mgy)
}
\\
S_2(y)
&=
\frac1{3m^2g}\bigl(2m(P_1-P_2-mgy)\bigr)^{\frac32}
\end{align*}
Damit ist jetzt eine L"osungsfunktion $S(x,y,t)$ vollst"andig
bekannt:
\[
S(x,y,t)=P_1t+\sqrt{2mP_2}x +
\frac1{3m^2g}\bigl(2m(P_1-P_2-mgy)\bigr)^{\frac32}
\]
Die zu verwendenden Koordinaten bekommt man jetzt wie vorhin durch
partielle Ableitung nach den $P_i$. Man bekommt nacheinander:
\begin{equation}
\begin{aligned}
Q_1=\frac{\partial S}{\partial P_1}
&=
t+
\frac1{g}\sqrt{\frac{2(P_1-P_2-mgy)}{m}}
\\
Q_2=\frac{\partial S}{\partial P_2}
&=
\sqrt{\frac{m}{2P_1}}x
-
\frac1{g}\sqrt{\frac{2(P_1-P_2-mgy)}{m}}
\end{aligned}
\label{jacobi:aufloesung}
\end{equation}
Die erste Gleichung kann man nach $y$ aufl"osen:
\begin{equation}
y=
\frac{P_1-P_2}{mg}-\frac{g}{2}(Q_1-t)^2
\label{jacobi:quadratisch}
\end{equation}
Die H"ohe h"angt quadratisch von der Zeit ab, $Q_1$ ist die Zeit
der gr"ossten H"ohe, die Scheitelzeit.

Die Aufl"osung nach $x$ wird einfacher, wenn man erst die Summe der beiden
Gleichungen (\ref{jacobi:aufloesung}) bildet:
\begin{align*}
Q_1+Q_2&=t+\sqrt{\frac{m}{2P_1}}x\\
x&=\sqrt{\frac{2P_1}{m}}(Q_1+Q_2-t)
\end{align*}
Die $x$-Koordinate nimmt offenbar linear mit der Zeit zu. Die
Geschwindigkeit ist
\[
v_x=\sqrt{\frac{2P_1}{m}}\quad\Rightarrow\quad P_1=\frac{mv_x^2}2,
\]
$P_1$ ist also die kinetische Energie der Horizontalbewegung.
$Q_2$ gibt an, wie viel Zeit nach dem Scheiteldurchgang die $x$-Koordinate
verschwindet.

Zur Zeit $t=Q_1$ verschwindet der quadratische Term in
(\ref{jacobi:quadratisch}), und man kann die Gleichung vereinfachen
zu 
\[
mgy + P_2=P_1.
\]
Da $P_2$ die kinetische Energie der Horizontalbewegung ist, und $mgy$
die potentielle Energie, ist $P_1$ die Gesamtenergie.


