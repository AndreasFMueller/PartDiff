%
% disk.tex -- XXX
%
% (c) 2019 Prof Dr Andreas Mueller
%
\section{Kreisgebiet}
\rhead{Kreisgebiet}
\index{Kreisgebiet}
\index{Kreisscheibe}
In diesem Abschnitt betrachten wir eine Kreisscheibe
\[
G=\{(x,y)\in\mathbb R^2|x^2+y^2 < R\}
\]
mit Radius $R$ als Definitionsbereich. Da sich dieses Gebiet durch
eine Streckung um den Faktor $\frac1R$ immer auf einen Einheitskreis
abbilden lässt, können wir ohne Verlust an Allgemeinheit vorausetzen,
dass $R=1$ ist.

Ein Kreisgebiet tritt zum Beispiel beim Problem auf, die Schwingungen
einer kreisförmigen Membran zu berechnen, wie sie bei einer Kesselpauke
vorkommen. Nach den Ergebnissen des ersten Kapitels suchen wir nach einer
Funktion $u$, welche auf $G$ die Gleichung
\[
\frac1{a^2}\frac{\partial^2 u}{\partial t^2}=\frac{\partial^2 u}{\partial x^2}+\frac{\partial^2 u}{\partial y^2}
\]
erfüllt. Wie bei der Schwingung der einer rechteckigen Platte
wird daraus mit dem Ansatz $ u(x,y,t)=u(x,y)\cdot T(t)$ ein
Eigenwertproblem:
\begin{align*}
T''(t)&=-a^2\lambda^2 T(t)\\
\Delta u(x,y)&=-\lambda^2u(x,y)
\end{align*}
Das Poissonproblem ist der Spezialfall $\lambda=0$.
\index{Poissonproblem}

\subsection{Polarkoordinaten}
\index{Polarkoordinaten}
Offenbar sind Polarkoordinaten speziell gut an das Problem angepasst, 
eine Randbedingung lässt sich zum Beispiel durch eine Funktion beschreiben,
welche nur vom Polarwinkel abhängt.
Eine schwingende kreisförmite Membran führt also auf die partielle
Differentialgleichung
\[
\frac{\partial^2u(r,\varphi)}{\partial t^2}=\Delta u(r,\varphi)
\]
mit der Randbedingung
\[
u(R,\varphi)=0,\qquad\varphi\in[0,2\pi],
\]
wobei wie oben $R$ der Radius der Membran ist.

Damit das Problem auf einem Kreisgebiet in Polarkoordinaten behandelt
werden kann,
brauchen wir einen Ausdruck für $\Delta u$ in Polarkoordinaten.
\begin{align}
x&=r\cos\varphi\\
y&=r\sin\varphi
\label{polarkoordinaten}
\end{align}
Um die Ableitungen nach $x$ und $y$ durch Ableitungen $\varphi$ und $r$ zu
ersetzen, leiten wir (\ref{polarkoordinaten}) nach $x$ und $y$ ab:
\begin{align*}
1&=
\frac{\partial r}{\partial x}\cos\varphi
-r\sin\varphi \frac{\partial\varphi}{\partial x}
&
0&=
\frac{\partial r}{\partial y}\cos\varphi
-r\sin\varphi \frac{\partial\varphi}{\partial y}
\\
0&=
\frac{\partial r}{\partial x}\sin\varphi
+r\cos\varphi \frac{\partial\varphi}{\partial x}
&
1&=
\frac{\partial r}{\partial y}\sin\varphi
+r\cos\varphi \frac{\partial\varphi}{\partial y}
\end{align*}
In Matrixschreibweise ist dies
\begin{align*}
\begin{pmatrix}1\\0\end{pmatrix}
&=
\begin{pmatrix}
\cos\varphi&-\sin\varphi\\
\sin\varphi&\cos\varphi
\end{pmatrix}
\begin{pmatrix}
\frac{\partial r}{\partial x}\\
r\frac{\partial \varphi}{\partial x}
\end{pmatrix}
&
\begin{pmatrix}0\\1\end{pmatrix}
&=
\begin{pmatrix}
\cos\varphi&-\sin\varphi\\
\sin\varphi&\cos\varphi
\end{pmatrix}
\begin{pmatrix}
\frac{\partial r}{\partial y}\\
r\frac{\partial \varphi}{\partial y}
\end{pmatrix}
\end{align*}
Die $2\times2$ Matrix ist eine Drehmatrix, die Inverse findet man, indem man
$\varphi$ durch $-\varphi$ ersetzt. Die Multiplikation auf der linken Seite
ergibt jeweils die erste bzw. zweite Spalte der Drehmatrix zum
Winkel $\varphi$:
\begin{align*}
\cos\varphi
&=\frac{\partial r}{\partial x}
&&
&
\sin\varphi
&=
\frac{\partial r}{\partial y}
&&
\\
-\sin\varphi
&=r\frac{\partial \varphi}{\partial x}
&\Rightarrow\quad
\frac{\partial\varphi}{\partial x}&=-\frac1r\sin\varphi
&
\cos\varphi
&=
r\frac{\partial\varphi}{\partial y}
&\Rightarrow\quad
\frac{\partial\varphi}{\partial y}&=\frac1r\cos\varphi
\end{align*}
Mit diesen Formeln können wir jetzt die höheren Ableitungen
von $u$ nach  $x$ und $y$ durch Ableitungen nach $r$ und $\varphi$
ersetzen.

Die partiellen Ableitungen von $\varphi$ nach $x$ und $y$ sind
\begin{align*}
\frac{\partial u}{\partial x}
&=
\frac{\partial u}{\partial r}
\frac{\partial r}{\partial x}
+
\frac{\partial u}{\partial\varphi}
\frac{\partial \varphi}{\partial x}
=
\frac{\partial u}{\partial r}
\cos\varphi
-
\frac{\partial u}{\partial\varphi}
\frac1r\sin\varphi
\\
\frac{\partial u}{\partial y}
&=
\frac{\partial u}{\partial r}
\frac{\partial r}{\partial y}
+
\frac{\partial u}{\partial\varphi}
\frac{\partial \varphi}{\partial y}
=
\frac{\partial u}{\partial r}
\sin\varphi
+
\frac{\partial u}{\partial\varphi}
\frac1r\cos\varphi
\end{align*}
Die zweiten Ableitungen sind
\begin{align*}
\frac{\partial^2u}{\partial x^2}
&=
\frac{\partial}{\partial r}
\left(
\frac{\partial u}{\partial r}
\cos\varphi
-
\frac{\partial u}{\partial\varphi}
\frac1r\sin\varphi
\right)
\frac{\partial r}{\partial x}
+
\frac{\partial }{\partial \varphi}
\left(
\frac{\partial u}{\partial r}
\cos\varphi
-
\frac{\partial u}{\partial\varphi}
\frac1r\sin\varphi
\right)
\frac{\partial\varphi}{\partial x}
\\
&=
\frac{\partial}{\partial r}
\left(
\frac{\partial u}{\partial r}
\cos\varphi
-
\frac{\partial u}{\partial\varphi}
\frac1r\sin\varphi
\right)
\cos\varphi
-
\frac{\partial }{\partial \varphi}
\left(
\frac{\partial u}{\partial r}
\cos\varphi
-
\frac{\partial u}{\partial\varphi}
\frac1r\sin\varphi
\right)
\frac1r\sin\varphi
\\
&=
\frac{\partial^2u}{\partial r^2} \cos^2\varphi
-
\frac{\partial^2u}{\partial r\partial\varphi} \frac1r\sin\varphi \cos\varphi
+
\frac{\partial u}{\partial\varphi} \frac1{r^2}\sin\varphi\cos\varphi
\\
&\quad
-
\frac{\partial^2u}{\partial\varphi\partial r}\frac1r \cos\varphi\sin\varphi
+\frac{\partial u}{\partial r}\frac1r\sin^2\varphi
+\frac{\partial^2u}{\partial\varphi^2}
\frac1{r^2}\sin^2\varphi
+\frac{\partial u}{\partial\varphi}\frac1{r^2}\cos\varphi\sin\varphi
\\
\frac{\partial^2u}{\partial y^2}
&=
\frac{\partial}{\partial r}
\left(
\frac{\partial u}{\partial r}
\sin\varphi
+
\frac{\partial u}{\partial\varphi}
\frac1r\cos\varphi
\right)
\frac{\partial r}{\partial y}
+
\frac{\partial}{\partial \varphi}
\left(
\frac{\partial u}{\partial r}
\sin\varphi
+
\frac{\partial u}{\partial\varphi}
\frac1r\cos\varphi
\right)
\frac{\partial \varphi}{\partial y}
\\
&=
\frac{\partial}{\partial r}
\left(
\frac{\partial u}{\partial r}
\sin\varphi
+
\frac{\partial u}{\partial\varphi}
\frac1r\cos\varphi
\right)
\sin\varphi
+
\frac{\partial}{\partial \varphi}
\left(
\frac{\partial u}{\partial r}
\sin\varphi
+
\frac{\partial u}{\partial\varphi}
\frac1r\cos\varphi
\right)
\frac1r\cos\varphi
\\
&=
\frac{\partial^2u}{\partial r^2}\sin^2\varphi
+\frac{\partial^2u}{\partial r\partial\varphi}\frac1r\cos\varphi\sin\varphi
-\frac{\partial u}{\partial\varphi}\frac1{r^2}\cos\varphi\sin\varphi
\\
&\quad
+
\frac{\partial^2u}{\partial\varphi\partial r}\frac1r\sin\varphi\cos\varphi
+\frac{\partial u}{\partial r}\frac1r\cos^2\varphi
+\frac{\partial^2u}{\partial \varphi^2}\frac1{r^2}\cos^2\varphi
-\frac{\partial u}{\partial \varphi}\frac1{r^2}\sin\varphi\cos\varphi
\end{align*}
\index{Laplace-Operator!in Polarkoordinaten}
Die Summe dieser zwei Terme ist die gesucht Darstellung des Laplace-Operators
in Polarkoordinaten:
\begin{align*}
\frac{\partial^2u}{\partial x^2}+\frac{\partial^2u}{\partial y^2}
&=
\frac{\partial^2u}{\partial r^2}
+\frac{\partial u}{\partial r}\frac1r
+\frac{\partial^2u}{\partial\varphi^2}\frac1{r^2}
\\
&=
\left(\frac1r\frac{\partial}{\partial r}r\frac{\partial}{\partial r}+\frac1{r^2}\frac{\partial^2}{\partial \varphi^2}\right)u
\end{align*}
Darstellungen des Laplace-Operators in weiteren Koordinatensystemen können
in jeder einigermassen vollständigen Formelsammlung gefunden werden.

\subsection{Separation der Ortsvariablen}
Die Lösung $u(r,\varphi)$ des Eigenwertproblems setzen wir wieder
als Produkt einer Funktion
$R(r)$
nur von  $r$ und einer Funktion $\Phi(\varphi)$ nur von $\varphi$ an.
Mit der im vorangegangenen Abschnitt gefundenen Formel für den Laplace-Operator
in Polarkoordinaten erhalten wir jetzt die Gleichungen
\begin{align*}
\Delta u=
\biggl(R''(r) + \frac1rR'(r)\biggr)\Phi(\varphi)
+\frac1{r^2}R(r)\Phi''(\varphi)&=-\lambda^2 R(r)\cdot\Phi(\varphi)\\
\frac{r^2R''(r)+rR'(r)}{R(r)}+\frac{\Phi''(\varphi)}{\Phi(\varphi)}
&=-\lambda^2 r^2
\\
\frac{r^2R''(r)+rR'(r)}{R(r)}+\lambda^2 r^2&=-\frac{\Phi''(\varphi)}{\Phi(\varphi)}
\end{align*}
Da die rechte Seite nur von $\varphi$ abhängt, die linke Seite aber nur von $r$,
müssen beide Seiten konstant sein, wir nennen diese Konstante $\mu^2$.
Damit sind die Variablen separiert:
\begin{align}
\Phi''(\varphi)+\mu^2\Phi(\varphi)&=0\label{phigleichung}\\
r^2R''(r)+rR'(r)+(\lambda^2 r^2-\mu^2)R(r)&=0\label{rgleichung}
\end{align}

\subsection{Lösung der separierten Differentialgleichungen}
Die allgemeine Lösung der Gleichung (\ref{phigleichung}) ist
\[
\Phi(\varphi)=A\cos\mu\varphi +B\sin\mu\varphi.
\]
Dies ist nur dann $2\pi$-periodisch, wenn $\mu$ eine ganze
Zahl ist, also $\mu=k$ mit $k\in\mathbb Z$.

Die Gleichung (\ref{rgleichung}) für $R$ bekommt damit die Form
\[
r^2R''(r)+rR'(r)+(\lambda^2 r^2-k^2)R(r)=0,
\]
sie ist verwandt mit der Besselschen Differentialgleichung.
Die Funktion $P(\varrho)=R(\varrho/\lambda)=R(r)$ hat die Ableitungen
\begin{align*}
\varrho P'(\varrho)&=\frac{\varrho}{\lambda}R'(\varrho/\lambda)=rR'(r)\\
\varrho^2 P''(\varrho)&=\frac{\varrho^2}{\lambda^2}R'(\varrho/\lambda)=r^2R''(r)
\end{align*}
und erfüllt somit die Besselsche Differentialgleichung
\[
\varrho^2P''(\varrho)+\varrho P'(\varrho)+(\varrho^2-k^2)P(\varrho).
\]
Lösungen der Besselschen Differentialgleichungen sind die Besselfunktionen
\[
P(\varrho)=J_{\pm k}(\lambda r)=R(r)
\]
Wie bei der rechteckigen Membran kann die allgemeine Lösung jetzt aus
den Teillösungen zusammengesetzt werden.

