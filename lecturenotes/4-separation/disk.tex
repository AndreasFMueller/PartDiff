%
% disk.tex -- 
%
% (c) 2019 Prof Dr Andreas Mueller
%
\section{Disk domain}
\rhead{Disk domain}
In this section, we attempt to solve the wave equation on the
disk shaped domain
\[
G=\{(x,y)\in\mathbb R^2|x^2+y^2 < R\}
\]
with radius $R$.
Such a domain can always be reduced to a circular domain with radius
$R=1$ by a simple homothety.
It is therefore no loss of generality to assume $R=1$ in the following.

A disk shaped domain appears e.~g.~when the vibrations of the membrane of
a drum or a microphone.
According to the results of section~\ref{subsection:separating time},
we are looking for a function of time and location in $G$.
The wave equation
\[
\frac1{a^2}\frac{\partial^2 u}{\partial t^2}
=
\frac{\partial^2 u}{\partial x^2}+\frac{\partial^2 u}{\partial y^2}
\]
as before reduces to the eigenvalue problem
\begin{align*}
T''(t)&=-a^2\lambda^2 T(t)\\
\Delta u(x,y)&=-\lambda^2u(x,y).
\end{align*}
In the specials case $\lambda=0$, this happens to be the Poisson problem.

\subsection{Polar coordinates}
\index{Polar coordinates}
Polar coordinates are obviously much more adapted to this problem than
rectangular coordinates, mainly because the boundary of the domain
is so much easier to describe in polar coordinates.
The boundary values on the circle can be described as a function
of the argument $\varphi$ only.
Assuming the speed of sound $c=1$ to simplify the computations, we
get the differential equation
\[
\frac{\partial^2u(r,\varphi)}{\partial t^2}=\Delta u(r,\varphi)
\]
with boundary conditions
\[
u(1,\varphi)=0,\qquad\varphi\in[0,2\pi].
\]

We need an expression for the Laplace operator in polar coordinates.
We start with the definition
\begin{align}
x&=r\cos\varphi\\
y&=r\sin\varphi
\label{polarkoordinaten}
\end{align}
for polar coordinates.
We have to convert derivatives with respect to $x$ and $y$
into derivatives with respect to $r$ and $\varphi$.
With this goal in mind we differentiate the equations
\eqref{polarkoordinaten} with respect to $x$ and $y$:
\begin{align*}
1&=
\frac{\partial r}{\partial x}\cos\varphi
-r\sin\varphi \frac{\partial\varphi}{\partial x}
&
0&=
\frac{\partial r}{\partial y}\cos\varphi
-r\sin\varphi \frac{\partial\varphi}{\partial y}
\\
0&=
\frac{\partial r}{\partial x}\sin\varphi
+r\cos\varphi \frac{\partial\varphi}{\partial x}
&
1&=
\frac{\partial r}{\partial y}\sin\varphi
+r\cos\varphi \frac{\partial\varphi}{\partial y}
\end{align*}
In matrix notation, this is
\begin{align*}
\begin{pmatrix}1\\0\end{pmatrix}
&=
\begin{pmatrix}
\cos\varphi&-\sin\varphi\\
\sin\varphi&\cos\varphi
\end{pmatrix}
\begin{pmatrix}
\frac{\partial r}{\partial x}\\
r\frac{\partial \varphi}{\partial x}
\end{pmatrix}
&
\begin{pmatrix}0\\1\end{pmatrix}
&=
\begin{pmatrix}
\cos\varphi&-\sin\varphi\\
\sin\varphi&\cos\varphi
\end{pmatrix}
\begin{pmatrix}
\frac{\partial r}{\partial y}\\
r\frac{\partial \varphi}{\partial y}
\end{pmatrix}
\end{align*}
This 
$2\times2$ matrix is a rotation matrix, its inverse can be found
by replacing $\varphi$ by $-\varphi$ or simply by taking the
transpose, as for any orthogonal matrix $A$, $A^{-1}=A^t$.
By multiplying out, we get terms for the following expressions for
the derivatives of polar coordinates with respect to $x$ and $y$
\begin{align*}
\cos\varphi
&=\frac{\partial r}{\partial x}
&&
&
\sin\varphi
&=
\frac{\partial r}{\partial y}
&&
\\
-\sin\varphi
&=r\frac{\partial \varphi}{\partial x}
&\Rightarrow\quad
\frac{\partial\varphi}{\partial x}&=-\frac1r\sin\varphi
&
\cos\varphi
&=
r\frac{\partial\varphi}{\partial y}
&\Rightarrow\quad
\frac{\partial\varphi}{\partial y}&=\frac1r\cos\varphi
\end{align*}
We can use these to also compute higher derivatives by
iterating the procedure.

To compute partial derivatives of $u$ we use the chain rule
\begin{align*}
\frac{\partial u}{\partial x}
&=
\frac{\partial u}{\partial r}
\frac{\partial r}{\partial x}
+
\frac{\partial u}{\partial\varphi}
\frac{\partial \varphi}{\partial x}
=
\frac{\partial u}{\partial r}
\cos\varphi
-
\frac{\partial u}{\partial\varphi}
\frac1r\sin\varphi
\\
\frac{\partial u}{\partial y}
&=
\frac{\partial u}{\partial r}
\frac{\partial r}{\partial y}
+
\frac{\partial u}{\partial\varphi}
\frac{\partial \varphi}{\partial y}
=
\frac{\partial u}{\partial r}
\sin\varphi
+
\frac{\partial u}{\partial\varphi}
\frac1r\cos\varphi
\end{align*}
The second derivatives similarly become
\begin{align*}
\frac{\partial^2u}{\partial x^2}
&=
\frac{\partial}{\partial r}
\left(
\frac{\partial u}{\partial r}
\cos\varphi
-
\frac{\partial u}{\partial\varphi}
\frac1r\sin\varphi
\right)
\frac{\partial r}{\partial x}
+
\frac{\partial }{\partial \varphi}
\left(
\frac{\partial u}{\partial r}
\cos\varphi
-
\frac{\partial u}{\partial\varphi}
\frac1r\sin\varphi
\right)
\frac{\partial\varphi}{\partial x}
\\
&=
\frac{\partial}{\partial r}
\left(
\frac{\partial u}{\partial r}
\cos\varphi
-
\frac{\partial u}{\partial\varphi}
\frac1r\sin\varphi
\right)
\cos\varphi
-
\frac{\partial }{\partial \varphi}
\left(
\frac{\partial u}{\partial r}
\cos\varphi
-
\frac{\partial u}{\partial\varphi}
\frac1r\sin\varphi
\right)
\frac1r\sin\varphi
\\
&=
\frac{\partial^2u}{\partial r^2} \cos^2\varphi
-
\frac{\partial^2u}{\partial r\partial\varphi} \frac1r\sin\varphi \cos\varphi
+
\frac{\partial u}{\partial\varphi} \frac1{r^2}\sin\varphi\cos\varphi
\\
&\quad
-
\frac{\partial^2u}{\partial\varphi\partial r}\frac1r \cos\varphi\sin\varphi
+\frac{\partial u}{\partial r}\frac1r\sin^2\varphi
+\frac{\partial^2u}{\partial\varphi^2}
\frac1{r^2}\sin^2\varphi
+\frac{\partial u}{\partial\varphi}\frac1{r^2}\cos\varphi\sin\varphi
\\
\frac{\partial^2u}{\partial y^2}
&=
\frac{\partial}{\partial r}
\left(
\frac{\partial u}{\partial r}
\sin\varphi
+
\frac{\partial u}{\partial\varphi}
\frac1r\cos\varphi
\right)
\frac{\partial r}{\partial y}
+
\frac{\partial}{\partial \varphi}
\left(
\frac{\partial u}{\partial r}
\sin\varphi
+
\frac{\partial u}{\partial\varphi}
\frac1r\cos\varphi
\right)
\frac{\partial \varphi}{\partial y}
\\
&=
\frac{\partial}{\partial r}
\left(
\frac{\partial u}{\partial r}
\sin\varphi
+
\frac{\partial u}{\partial\varphi}
\frac1r\cos\varphi
\right)
\sin\varphi
+
\frac{\partial}{\partial \varphi}
\left(
\frac{\partial u}{\partial r}
\sin\varphi
+
\frac{\partial u}{\partial\varphi}
\frac1r\cos\varphi
\right)
\frac1r\cos\varphi
\\
&=
\frac{\partial^2u}{\partial r^2}\sin^2\varphi
+\frac{\partial^2u}{\partial r\partial\varphi}\frac1r\cos\varphi\sin\varphi
-\frac{\partial u}{\partial\varphi}\frac1{r^2}\cos\varphi\sin\varphi
\\
&\quad
+
\frac{\partial^2u}{\partial\varphi\partial r}\frac1r\sin\varphi\cos\varphi
+\frac{\partial u}{\partial r}\frac1r\cos^2\varphi
+\frac{\partial^2u}{\partial \varphi^2}\frac1{r^2}\cos^2\varphi
-\frac{\partial u}{\partial \varphi}\frac1{r^2}\sin\varphi\cos\varphi
\end{align*}
Adding these two terms gives the representation of the Laplace operator
in polar coordinates that we have been looking for:
\begin{align*}
\frac{\partial^2u}{\partial x^2}+\frac{\partial^2u}{\partial y^2}
&=
\frac{\partial^2u}{\partial r^2}
+\frac{\partial u}{\partial r}\frac1r
+\frac{\partial^2u}{\partial\varphi^2}\frac1{r^2}
\\
&=
\left(\frac1r\frac{\partial}{\partial r}r\frac{\partial}{\partial r}+\frac1{r^2}\frac{\partial^2}{\partial \varphi^2}\right)u
\end{align*}
Formulae for the Laplace operator in many different coordinate systems
can be found in any reasonably complete reference.

\subsection{Separating the location variables}
The solution $u(r,\varphi)$ of the eigenvalue problem is again attempted
as the product of a function $R(r)$ of $r$ only and a function
$\Phi(\varphi)$ of $\varphi$ only.
Using the formula for the Laplace operator, the differential equation
in polar coordinates becomes
\begin{align*}
\Delta u=
\biggl(R''(r) + \frac1rR'(r)\biggr)\Phi(\varphi)
+\frac1{r^2}R(r)\Phi''(\varphi)&=-\lambda^2 R(r)\cdot\Phi(\varphi)\\
\frac{r^2R''(r)+rR'(r)}{R(r)}+\frac{\Phi''(\varphi)}{\Phi(\varphi)}
&=-\lambda^2 r^2
\\
\frac{r^2R''(r)+rR'(r)}{R(r)}+\lambda^2 r^2&=-\frac{\Phi''(\varphi)}{\Phi(\varphi)}
\end{align*}
The right hand side depends only on $\varphi$, the left hand side only on $r$,
so, again, both sides need to be constant.
We call the constant $\mu^2$.
This completes the separation of the variables:
\begin{align}
\Phi''(\varphi)+\mu^2\Phi(\varphi)&=0\label{phigleichung}\\
r^2R''(r)+rR'(r)+(\lambda^2 r^2-\mu^2)R(r)&=0\label{rgleichung}
\end{align}

\subsection{Solution of the separated equations}
The general solution of the equation \eqref{phigleichung} is
\[
\Phi(\varphi)=A\cos\mu\varphi +B\sin\mu\varphi.
\]
Because $\varphi$ and $\varphi+2\pi$ describe the same points in the domain,
$\Phi$ must be $2\pi$-periodic, which only happens if
$\mu$ is an integer multiple of $2\pi$.
So we write $\mu_k = 2\pi k$ with $k\in\mathbb Z$.

The equation \eqref{rgleichung} for $R(r)$ now becomes the form
\[
r^2R''(r)+rR'(r)+(\lambda^2 r^2-k^2)R(r)=0,
\]
it is related to Bessel's equation in the following way.
The function
$P(\varrho)=R(\varrho/\lambda)=R(r)$ has solutions
\begin{align*}
\varrho P'(\varrho)&=\frac{\varrho}{\lambda}R'(\varrho/\lambda)=rR'(r)\\
\varrho^2 P''(\varrho)&=\frac{\varrho^2}{\lambda^2}R'(\varrho/\lambda)=r^2R''(r)
\end{align*}
and solves Bessel's equation
\[
\varrho^2P''(\varrho)+\varrho P'(\varrho)+(\varrho^2-k^2)P(\varrho).
\]
The solutions of Bessel's equation are the Bessel functions
\[
P(\varrho)=J_{\pm k}(\lambda r)=R(r)
\]
As for a rectangular membrane, the general solution must now be
linearly combined from partial solutions.

