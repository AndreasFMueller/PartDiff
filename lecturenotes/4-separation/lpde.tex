%
% lpde.tex -- why linear pdes?
%
% (c) 2019 Prof Dr Andreas Mueller
%
\section{Separation for linear partial differential equations}
The base idea of the separation method produces a family of functions
that depend an some integration and separation constants.
On the boundary we are usually given some arbitrary functions.
In general it will be impossible to tune these few constants to
values that reproduce the boundary functions.
This basic version of the separation method thus is incapable of
solving a general partial differential equation due to the lack
of flexibility in the set of solutions found.

This problem changes if different solutions can be combined into new ones.
Linear combinations of solutions introduce a large set of additional
parameters to tune the solution.
In particular, Fourier theory shows that linear combinations of
basic functions can be tuned to approximate just about any periodic
function.
However, linear combinations of solutions are in general no longer
solutions the equation, except for linear partial differential
equations.
This is expressed in the following theorem.

\begin{satz}
If $u_1$ and $u_2$
are solutions of a homogeneous linear partial differential equation,
then $u_1+u_2$ and $\lambda u_1$ with $\lambda\in\mathbb R$ are also
solutions.
\end{satz}

\begin{proof}
The differential equation
\[
F(x_1,\dots,x_2,u,\frac{\partial u}{\partial x_1},\dots)=0
\]
is linear in $u$ and its derivatives, so we can expand sums and
pull factors from inside $F$:
\begin{align*}
F(x_1,\dots,x_2,u_1+u_2,\frac{\partial u_1}{\partial x_1}+\frac{\partial u_2}{\partial x_1},\dots)
&=
F(x_1,\dots,x_2,u_1,\frac{\partial u_1}{\partial x_1},\dots)
\\
&+
F(x_1,\dots,x_2,u_2,\frac{\partial u_2}{\partial x_1},\dots)=0
\\
F(x_1,\dots,x_2,\lambda u_1,\frac{\partial \lambda u_1}{\partial x_1},\dots)
&=
\lambda
F(x_1,\dots,x_2,u_1,\frac{\partial u_1}{\partial x_1},\dots)
=0
\end{align*}
Thus linear combinations of $u_1$ and $u_2$ are solutions too.
\end{proof}

Linear cominations allow us to first find as many different solutions
$u_1$, $u_2$, $u_3,\dots$ and then to use suitable coefficients $a_k$
to combine them into a solution
\[
u(x)=\sum_{i=1}^\infty a_ku_k
\]
that also satisfies the boundary conditions.

Inhomogeneous linear partial differential equations can be solved using
this method too.
As pointed out in chapter~\ref{chapter:terminology-and-notation},
we first have to find a particular solution $u_p$ which solves the
inhomogeneous partial differential equation independently
of any boundary conditions.
The separation method can be helpful for this too.
Then the problem is reduced to finding a solutions $u_h$ to the homogeneous
equations with boundary condition $g-u_p$ on $\partial\Omega$.
Using the separation method, we can build up $u_h$ as a linear combination
of solutions.

The coefficients $a_k$ for the linear combination often lead us into
Fourier theory or some generalization of it.
Such partial solutions often have immediate physical significance.
In mechanical or electrical engineering they appear as vibration modes,
in quantum mechanics as energy states.


