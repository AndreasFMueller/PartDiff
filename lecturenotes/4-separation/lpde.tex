%
% lpde.tex -- XXX
%
% (c) 2019 Prof Dr Andreas Mueller
%
\section{Separation für lineare partielle Differentialgleichungen}
Die Grundidee des Separationsverfahrens liefert eine Familie von Funktionen,
von ein paar Integrationskonstanten abhängen. Vorgegeben sind auf dem
Rand aber beliebige Funktionen, es ist im Allgemeinen nicht möglich,
durch richtige Wahl von wenigen Parametern beliebige Funktionen zu erhalten.
Des Separationsverfahren in der bisherigen Form kann also ein
beliebiges Randwertproblem noch nicht lösen.

Nehmen wir an, es müsse die Differentialgleichung $Lu=f$ auf dem
Gebiet $\Omega$ mit Randwerte $g(x)$ für $x\in\partial \Omega$
gelöst werden.
Wir haben nur dann eine Chance, aus den bisherigen Resultaten eine 
vollständige Lösung des Problems zu erhalten, wenn wir in der
Lage sind, solche Teillösungen zu einer vollständigen Lösung
zu kombinieren. Dazu sind aber im allgemeinen zusätzliche
Bedingungen an die Differentialgleichung nötig. Linearität der
Differentialgleichung ist, was wir hier verwenden wollen:

\begin{satz}Sind $u_1$ und $u_2$ Lösungen einer homogenen linearen partiellen
Differentialgleichung, dann sind auch $u_1+u_2$ und $\lambda u_1$ für
$\lambda\in\mathbb R$ Lösungen
\end{satz}

\begin{proof}[Beweis]
Die Differentialgleichung
\[
F(x_1,\dots,x_2,u,\frac{\partial u}{\partial x_1},\dots)=0
\]
ist linear in $u$ und den Ableitungen, man darf also Summen und Vielfache
in $u$ und den Ableitungen aus der Funktion herausziehen:
\begin{align*}
F(x_1,\dots,x_2,u_1+u_2,\frac{\partial u_1}{\partial x_1}+\frac{\partial u_2}{\partial x_1},\dots)
&=
F(x_1,\dots,x_2,u_1,\frac{\partial u_1}{\partial x_1},\dots)
\\
&+
F(x_1,\dots,x_2,u_2,\frac{\partial u_2}{\partial x_1},\dots)=0
\\
F(x_1,\dots,x_2,\lambda u_1,\frac{\partial \lambda u_1}{\partial x_1},\dots)
&=
\lambda
F(x_1,\dots,x_2,u_1,\frac{\partial u_1}{\partial x_1},\dots)
=0
\end{align*}
Die Linearkombinationen von $u_1$ und $u_2$ sind also auch Lösungen.
\end{proof}

Um die Differentialgleichung zu lösen, kann man also versuchen, mit
dem bisherigen Separationsverfahren möglichst viele Lösungen zu
$u_1$, $u_2$, $u_3,\dots$ zu finden, und diese dann zu einer Gesamtlösung
zu kombinieren
\[
u(x)=\sum_{i=1}^\infty a_ku_k
\]
mit geeigneten Koeffizienten $a_k$, die so zu wählen sind, dass die
Randbedingungen erfüllt sind.

Inhomogene lineare partielle Differentialgleichungen kann man mit diesem
Verfahren ebenfalls lösen. Dazu findet man zunächst eine partikuläre
Lösung $u_p$, welche die inhomogenen Differentialgleichung löst, ohne
allerdings korrekte Randwerte zu liefern.
Das ursprüngliche Separationsverfahren kann hierbei hilfreich sein.
Dann verwendet man das skizzierte Verfahren für homogene lineare
partielle Differentialgleichungen für modifizierte Randwerte $g-u_p$ auf
$\partial\Omega$ um eine Lösung der homogenen Gleichung $u_h$ zu finden.
Die Summe $u=u_p+u_h$ ist dann eine Lösung der inhomogenen Gleichung
mit Randwerten $u_p + (g-u_p)=g$, also ein Lösung des ursprünglichen
Problems.

Die Bestimmung der Koeffizienten $a_k$ für die gefundene Familie
von Teillösungen führt oft auf die Fourier-Theorie, oder allgemeiner
auf orthogonale Funktionenfamilien. Solche Teillösungen haben oft eine
unmittelbare physikalische Bedeutung, zum Beispiel treten sie auf als
Schiwngungsmoden (in der Mechanik oder Elektrotechnik) oder als
Elektronenzustände in der Quantenmechanik.

