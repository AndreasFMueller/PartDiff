%
% idea.tex -- idea of the method
%
% (c) 2008 Prof Dr Andreas Mueller
%

\section{Idea of the method}
In applications one often has some indications from the application
domain what the solution function will most probably look like, or one
is looking for a very particular type of solution.
In partuclar, it may be known how the solution depends on one of the
variables up to a factor depending only on the other variable.
In these situations one can try to write the solution as a product
or sum of functions that depend on only one variable.

Even if one knows nothing about the solution, one can still try such
an {\em ansatz}, as the following example tries to illustrate.
Let's attempt to solve the partial differetnial equation
\begin{equation}
\frac1x
\frac{\partial u}{\partial x}
+
\frac1y
\frac{\partial u}{\partial y}
=\frac1{y^2}
,
\qquad x>1, y>1,
\label{separation:beispiel1}
\end{equation}
ignoring the boundary conditions for the time being.
We try to represent the solution as a sum of two functions which depend
on one of $x$ and $y$ only.
\begin{equation}
u(x,y)=X(x)+Y(y)
\quad\Rightarrow\quad
\begin{cases}
\quad{\displaystyle \frac{\partial u}{\partial x}}&=X'(x)\\
\\
\quad{\displaystyle \frac{\partial u}{\partial y}}&=Y'(y)\\
\end{cases}
\label{separation:beispiel1:ansatz}
\end{equation}
Substituting this into the differential equation
(\ref{separation:beispiel1})
gives the new equation
\[
\frac{X'(x)}{x}+\frac{Y'(y)}{y}=\frac1{y^2}
\]
or
\begin{equation}
\frac{X'(x)}{x}
=\frac1{y^2}
-\frac{Y'(y)}{y}.
\label{separation:beispiel1:separiert}
\end{equation}
The Form (\ref{separation:beispiel1:separiert}) has a distinct property:
the variable $x$ only appears on the left side, the variable $y$ only on
the right.
If we fix some value $y$, the right hand side cannot change, so the left
hand side must not depend on $x$.
Conversely, if we fix $x$, then the left side cannot change any more,
and thus the right hand side cannot change either.
We conclude that both sides must be the same constant.
Calling this constant $k$ we find the two ordinary differential equations
\begin{align}
\frac{X'(x)}{x}&=k
&
k&=\frac1{y^2}-\frac{Y'(y)}{y}
\label{separation:beispiel1:separiertedgl}
\end{align}
for $X(x)$ and $Y(y)$.

The differential equation for $X$ is easy to solve:
\begin{align*}
X'(x)&=kx\quad\Rightarrow\quad X(x)=
\frac12kx^2+C_x.
\end{align*}
The right equation is only slightly more complicated:
\begin{align*}
Y'(y)=\frac1y-ky
\quad\Rightarrow\quad
Y(y)=\int\frac1y-ky\,dy=
\log y-\frac12ky^2+C_y.
\end{align*}
We can now combine these functions into a solution of the initial
differential equation:
\begin{equation}
u(x,y)=
\frac12kx^2+
\log y-\frac12ky^2+C.
\label{separation:beispiel1:loesung}
\end{equation}
By varying the parameters $k$ and $C$, formula
(\ref{separation:beispiel1:loesung}) gives an infinite family of
solutions of the partial differential equation.

The values of the constants need to be determined by boundary conditions
in a manner to be studied later.

Let's summarize the method so far:
\begin{enumerate}
\item
Choose an {\em ansatz} from functions that depend from disjoint
sets of variables.
\item
Substitute into the partial differential equation and separate terms
involving the separate sets of variables.
The two sides of the equation depend on disjoint sets of variables
und must therefore be constant.
\item 
Split the equation into two coupled equations, each with a
different set of independent variables.
\item
Solve each equation individually.
\item
Put solutions together using the boundary conditions.
\end{enumerate}
As may suspected, the problem most of the time is not the solution
of the individual equations but rather the last step.
In the following sections we want to illustrate how this can be
done in a variety of examples.

