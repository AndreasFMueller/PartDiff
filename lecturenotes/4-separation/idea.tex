%
% idea.tex -- XXX
%
% (c) 2008 Prof Dr Andreas Mueller
% $Id: c02-separation.tex,v 1.3 2008/09/13 23:01:45 afm Exp $
%

\section{Idee des Verfahrens}
Oft hat man aus dem Anwendungsgebiet, aus dem eine partielle
Differentialgleichung stammt, Hinweise darauf, wie die Lösungsfunktion
von {\em einer} der unabhängigen Variablen abhängt.
Dann kann man Versuchen, die Lösung als Produkt oder Summe von
solchen ``vermuteten'' Funktionen darzustellen.

\index{Ansatz}
\index{Separationsansatz}
Aber selbst wenn man nichts über die Lösung weiss, kann man 
versuchen, die Lösung als Summe oder Produkt von Funktionen darzustellen,
die nur von jeweils einer Variablen abhängen. Als Beispiel versuchen
wir die lineare Differentialgleichung 
\begin{equation}
\frac1x
\frac{\partial u}{\partial x}
+
\frac1y
\frac{\partial u}{\partial y}
=\frac1{y^2}
,
\qquad x>1, y>1
\label{separation:beispiel1}
\end{equation}
zu lösen, wobei wir die Randbedingungen für den Moment ignorieren.
Wir versuchen, die Lösung als Summe von zwei Funktionen darzustellen,
welche nur von $x$ bzw.~$y$ abhängen, also
\begin{equation}
u(x,y)=X(x)+Y(y)
\quad\Rightarrow\quad
\begin{cases}
\quad{\displaystyle \frac{\partial u}{\partial x}}&=X'(x)\\
\\
\quad{\displaystyle \frac{\partial u}{\partial y}}&=Y'(y)\\
\end{cases}
\label{separation:beispiel1:ansatz}
\end{equation}
Setzt man dies in die Differentialgleichung (\ref{separation:beispiel1})
ein, erhalten wir die Gleichung
\[
\frac{X'(x)}{x}+\frac{Y'(y)}{y}=\frac1{y^2}
\]
oder
\begin{equation}
\frac{X'(x)}{x}
=\frac1{y^2}
-\frac{Y'(y)}{y}.
\label{separation:beispiel1:separiert}
\end{equation}
Die Form (\ref{separation:beispiel1:separiert}) hat eine besondere
Eigenschaft: die Variable $x$ kommt nur auf der linken Seite vor,
die Variable $y$ nur auf der rechten. Setzt man für $y$ irgend
einen Wert ein, verändert sich die rechte Seite nicht mehr,
die linke Seite muss also für alle $x$ den gleichen Wert geben.
Lässt man jetzt wieder das $y$ varieren, muss sich auch auf der rechten
Seite immer der gleiche Wert ergeben. Wir nennen den gemeinsamen Wert
$k$, und bekommen zwei gewöhnliche Differentialgleichungen
für $X(x)$ und $Y(y)$:
\begin{align}
\frac{X'(x)}{x}&=k
&
k&=\frac1{y^2}-\frac{Y'(y)}{y}
\label{separation:beispiel1:separiertedgl}
\end{align}
Die linke Differentialgleichung ist einfach zu lösen:
\begin{align*}
X'(x)&=kx\quad\Rightarrow\quad X(x)=
\frac12kx^2+C_x.
\end{align*}
Die rechte Differentialgleichung ist nur leicht komplizierter:
\begin{align*}
Y'(y)=\frac1y-ky
\quad\Rightarrow\quad
Y(y)=\int\frac1y-ky\,dy=
\log y-\frac12ky^2+C_y.
\end{align*}
Diese beiden Lösungen können wir jetzt wieder zu einer Lösung der
ursprünglichen Differentialgleichung zusammensetzen:
\begin{equation}
u(x,y)=
\frac12kx^2+
\log y-\frac12ky^2+C.
\label{separation:beispiel1:loesung}
\end{equation}
Wir haben damit eine unendliche Familie von Lösungen der
Differentialgleichung gefunden, für jedes Paar $(k,C)$ von
Parametern liefert die Formel (\ref{separation:beispiel1:loesung})
eine Lösung.
Dazu müssen wir in irgend einer Weise die Randbedingungen verwenden.
Damit haben wir eine Skizze für das Lösungsverfahren mit Separation:
\begin{enumerate}
\item Finde einen Lösungsansatz aus Funktionen, die nur von einer
Variablen abhängen.
\item Setze in die Differentialgleichung ein und separiere die Terme
so, dass zwei Variablen jeweils nur auf einer Seite vorkommen. Dann
hängen beide Seiten nicht mehr von dieser Variablen ab, jede Seite
ist eine Differentialgleichung mit weniger Variablen.
\item Löse die Teildifferentialgleichungen, und setze daraus 
Lösungen zusammen. 
\item Verwende die Randbedingungen, um eine Lösung zu finden.
\end{enumerate}
Wie man unschwer erkennen kann, ist weniger die Lösung der
einzelnen Teildifferentialgleichungen das Problem, sondern der letzte
Schritt. In den folgenden Abschnitten zeigen wir an Beispielen, wie
dies möglich ist.

