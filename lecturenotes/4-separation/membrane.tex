%
% membrane.tex -- XXX
%
% (c) 2019 Prof Dr Andreas Mueller
%
\section{Vibrating rectangular membrane}
\rhead{Rectangular Membrane}
Let's consider the vibration of a rectangulare membrane fixed on
the boundary of the domain
\[
R=\{(x,y)\,|\,0\le x\le a,0\le y\le b\} =(0,a)\times(0,b).
\]
At time $t=0$, the shape of the membrane is given by a function
$f(x,y)$ defined on $R$.
For arbitrary times $t\ge 0$ it is described by a function $u(x,y,t)$ 
with differential equation
\[
\frac1{c^2}\frac{\partial^2u}{\partial t^2}=\frac{\partial^2u}{\partial x^2}+\frac{\partial^2u}{\partial y^2}
\]
and satisfies the boundary conditions
\begin{align*}
u(x,y,0)&=f(x,y)\quad\forall 0\le x\le a,0\le y\le b,
\\
\frac{\partial}{\partial t}u(x,y,0)&=g(x,y)\quad\forall 0\le x\le a,0\le y\le b
\end{align*}
at time $t=0$ and
\begin{align*}
u(0,y,t)&=0&u(a,y,t)&=0&\forall t\ge 0,0\le y\le b,\\
u(x,0,t)&=0&u(x,b,t)&=0&\forall t\ge 0,0\le x\le a.
\end{align*}
around the boundary of $R$.

\subsection{Separating Time}
We want to separate variable $t$ from the rest of the variables.
According to the ideas expanded in the introduction to this chapter,
we try to stuff the time dependence into a separate function $T(t)$,
and the dependence on $x$ an $y$ in $\varphi(x,y)$.
We thus use the ansatz
\[
u(x,y,t)=T(t)\cdot\varphi(x,y).
\]
It will obviously not be possible to describe every vibrating membrane
with just one such function.
We thus expect to find a sufficiently large set of partial solutions
that will allow us to tune a linear combination to also satisfy the 
initial conditions.


In any event, the boundary conditions along $\partial R$ do not depend
on time, so we need that the function $\varphi$ satisfies the following
boundary conditions:
\begin{align*}
\varphi(0,y)&=0&\varphi(a,y)&=0&0\le y\le b\\
\varphi(x,0)&=0&\varphi(x,b)&=0&0\le x\le a
\end{align*}

Now let's substitute the ansatz for $u$ into the wave equation.
We obtain:
\[
\frac1{c^2}T''(t)\varphi(x,y)=T(t)\left(
\frac{\partial^2\varphi}{\partial x^2}
+
\frac{\partial^2\varphi}{\partial y^2}
\right).
\]
We are looking for a function $u$ that does not vanish identically,
thus points where $T$ vanishes are isolated.
Also the points where $\varphi(x,y)$ vanishes form a set of area $0$,
so in most points, it is perfectly OK to devide by $T(t)$ and
$\varphi(x,y)$.
This turns the equation into
\begin{equation}
\frac1{c^2}\frac{T''(t)}{T(t)}
= \frac1{\varphi(x,y)}\left( \frac{\partial^2\varphi}{\partial x^2}
+ \frac{\partial^2\varphi}{\partial y^2} \right)
\label{separiertMembran}
\end{equation}
The right hand side depends only on $(x,y)$, the left hand side only on $t$.
According to the basic principle of the separation method, both sides
must be constant.
There is therefore a $k$ with the property
\[
\frac1{c^2}\frac{T''(t)}{T(t)}=k
\qquad\Leftrightarrow\qquad
T''(t)=k T(t).
\]
This is an ordinary differential equation of second order with solutions
in the form
$e^{\pm\sqrt{k}t}$ for positive $k$.
For negative $k$, $\sin\sqrt{k}t$ and $\cos\sqrt{k}t$ are solutions.
From the point of view of physics only solutions with oscillating
character are meaningful, we thus may assume that $k<0$, so we can
write $k=-\lambda^2$ for some $\lambda$.
The differential equation can thus also be written as
\[
\frac1{c^2}\frac{T''(t)}{T(t)}=-\lambda^2
\]
or
\[
T''(t)=-c^2\lambda^2 T(t).
\]
The solutions are linear combinations 
\[
A\cos c\lambda t+B\sin c\lambda t.
\]

\subsection{Reduction to an eigenvalue problem}
The right hand side of \eqref{separiertMembran}
also must be the constant $-\lambda^2$:
\begin{align*}
\frac1{\varphi(x,y)}\left(
\frac{\partial^2\varphi}{\partial x^2}
+
\frac{\partial^2\varphi}{\partial y^2}
\right)&=-\lambda^2\\
\frac{\partial^2\varphi}{\partial x^2}
+
\frac{\partial^2\varphi}{\partial y^2}
=
\Delta\varphi
&=-\lambda^2
\varphi(x,y)
\end{align*}
The function $\varphi$ is thus an eigenvector of the linear 
oeprator $\Delta$ with eigenvalue $-\lambda^2$.
This limits the possible frequencies $\lambda$ to the eigenvalues
of the operator $\Delta$.

\subsection{Separation of $x$ and $y$}
% XXX
We once more try the separation ansatz
\[
\varphi(x,y)=X(x)\cdot Y(y)
\]
for the eigenvalue problem.
Substituted into the differential equation it becomes
\begin{align*}
X''(x)Y(x)+X(x)Y''(y)&=-\lambda^2 X(x)Y(y)
\\
\frac{X''(x)}{X(x)}+\frac{Y''(y)}{Y(y)}&=-\lambda^2.
\intertext{%
Each of the fractions only depends on one variable, which allows us to separate
the variables as in
}
\frac{X''(x)}{X(x)}&=-\frac{Y''(y)}{Y(y)}-\lambda^2
\end{align*}
We conclude that both sides must be constant.
We write these constants separately as
\begin{align*}
X''(x)&=-\lambda_1^2X(x)\\
Y''(y)&=-\lambda_2^2Y(y)\\
\lambda_1^2+\lambda_2^2&=\lambda^2.
\end{align*}
This is justified by the fact that we are looking for solutions
that vanish at the boundary, where exponential functions would
not vanish.
The boundary conditions even force us to reject the cosine solutions
of these equations, so we are left with the sine functions
\begin{align*}
X(x)&=A\sin \lambda_1x\\
Y(y)&=B\sin \lambda_2y
\end{align*}
which satisify the boundary condition for $\varphi$ at the left and
bottom boundaries of the rectangle $R$.

The boundary conditions at the other boundaries $x=a$ and $y=b$ can
only be satisfied if $\lambda_1a$ and $\lambda_2b$ are multiples of
$\pi$, so there must be integers $k$ and $l$ such that
\[
\lambda_1=\frac{k\pi}a
\qquad
\text{and}
\qquad
\lambda_2=\frac{l\pi}b
\]
The possible values for $\lambda$ thus are
\[
\lambda_{kl}^2=\left(\frac{k^2}{a^2} + \frac{l^2}{b^2}\right)\pi^2,\qquad k,l\in\mathbb Z
\]
The general solution must no be linearly combined from the partial
solutions
\[
\varphi_{kl}(x,y)=\sin \frac{k\pi}{a}x\sin\frac{l\pi}{b}y
\]
of the eigenvalue problem.
Furthermore, the general time dependent solution must be linearly
combined from the these solutions multiplied by solutions for $T$
with $\lambda^2=\lambda_1^2 + \lambda_2^2$, i.~e.
\[
u_{kl}(x,y,t)
=
(A_{kl}\cos c\lambda_{kl} t+
B_{kl}\sin c\lambda_{kl} t)
\sin \frac{k\pi}{a}x\sin\frac{l\pi}{b}y
\]
The most general solution has the form
\begin{equation}
u(x,y,t)=\sum_{k,l}
(A_{kl}\cos c\lambda_{kl} t+
B_{kl}\sin c\lambda_{kl} t)
\sin \frac{k\pi}{a}x\sin\frac{l\pi}{b}y.
\label{allgemeineloesung}
\end{equation}

\subsection{Initial conditions}
The solution constructed so far satifies the boundary conditions along
the boundary of $R$, but not yet at $t=0$
This initial conditions can now also be expressed in terms of the
coefficients $A_{kl}$ and $B_{kl}$:
\begin{align*}
\sum_{k,l}A_{kl}
\sin \frac{k\pi}{a}x\sin\frac{l\pi}{b}y&=f(x,y)\\
\sum_{k,l}B_{kl}c\lambda_{kl}
\sin \frac{k\pi}{a}x\sin\frac{l\pi}{b}y&=g(x,y)\\
\end{align*}
In some cases, one can explicitly find the coefficients using Fourier
theory.

