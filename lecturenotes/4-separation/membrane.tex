%
% membrane.tex -- XXX
%
% (c) 2019 Prof Dr Andreas Mueller
%
\section{Schwingende rechteckige Membran}
\rhead{Rechteckige Membran}
\index{Membran!rechteckig}
Wir betrachten die Schwingung einer rechteckigen Membran, die am Rande
des Gebietes
\[
R=\{(x,y)\,|\,0\le x\le a,0\le y\le b\} =(0,a)\times(0,b)
\]
eingespannt ist. Zur Zeit $t=0$ sei die Form der Membran durch die
Funktion $f(x,y)$ gegeben.
Für beliebige Zeit $t\ge 0$ wird sie beschrieben durch eine Funktion $u(x,y,t)$,
welche der Differentialgleichung
\[
\frac1{c^2}\frac{\partial^2u}{\partial t^2}=\frac{\partial^2u}{\partial x^2}+\frac{\partial^2u}{\partial y^2}
\]
genügt mit den Anfangsbedingungen
\begin{align*}
u(x,y,0)&=f(x,y)\quad\forall 0\le x\le a,0\le y\le b,
\\
\frac{\partial}{\partial t}u(x,y,0)&=g(x,y)\quad\forall 0\le x\le a,0\le y\le b
\end{align*}
und den Randbedingungen
\begin{align*}
u(0,y,t)&=0&u(a,y,t)&=0&\forall t\ge 0,0\le y\le b,\\
u(x,0,t)&=0&u(x,b,t)&=0&\forall t\ge 0,0\le x\le a.
\end{align*}

\subsection{Separation der Zeit}
\index{Separation}
Nach der in der Einleitung motivierten Idee suchen wir Lösungen also
Produkt einer Funktion $T(t)$, die nur von der Zeit abhängt, und einer Funktion
$\varphi(x,y)$, welche nur vom Ort abhängt, also
\[
u(x,y,t)=T(t)\cdot\varphi(x,y).
\]
Leider kann ein einzelnes solches Produkt nicht alle Anfangsbedingungen
erfüllen. Wäre dies nämlich möglich, müsste $\varphi(x,y)\sim f(x,y)$
sein und alle Teile der Membran würden im Gleichtakt hin und her schwingen.
Simulationen oder physikalische Experimente zeigen aber, dass es
Anfangsbedingungen gibt, bei denen die Teile der Membran gegenläufig
schwingen.

Anderseits, muss die Lösung auf jeden Fall die Randbedingung erfüllen,
es muss also gelten
\begin{align*}
\varphi(0,y)&=0&\varphi(a,y)&=0&0\le y\le b\\
\varphi(x,0)&=0&\varphi(x,b)&=0&0\le x\le a
\end{align*}
Setzen wir diesen Ansatz für $u$ in der Wellengleichung ein,
erhalten wir
\[
\frac1{c^2}T''(t)\varphi(x,y)=T(t)\left(
\frac{\partial^2\varphi}{\partial x^2}
+
\frac{\partial^2\varphi}{\partial y^2}
\right)
\]
Wir suchen eine Funktion $u$, die nicht identisch verschwindet,
es gibt also einige Zeitpunkte $t$ und Orte $(x,y)$, an denen $T(t)$
und $\varphi(x,y)$ nicht verschwinden. An diesen Stellen kann man die
Gleichung umformen in
\begin{equation}
\frac1{c^2}\frac{T''(t)}{T(t)}
= \frac1{\varphi(x,y)}\left( \frac{\partial^2\varphi}{\partial x^2}
+ \frac{\partial^2\varphi}{\partial y^2} \right)
\label{separiertMembran}
\end{equation}
Die rechte Seite hängt nur
vom Ort ab, darf sich also nicht ändern, wenn man die Zeit $t$ variert.
Als Funktion der Zeit muss die linke Seite eine Konstante sein,
es gibt also ein $k$ mit der Eigenschaft
\[
\frac1{c^2}\frac{T''(t)}{T(t)}=k
\qquad\Leftrightarrow\qquad
T''(t)=k T(t).
\]
Diese gewöhnliche Differentialgleichung hat Lösungen der Form 
$e^{\pm\sqrt{k}t}$ für positives $k$. Für negatives $k$ sind $\sin\sqrt{k}t$ 
und $\cos\sqrt{k}t$ Lösungen.
Aus physikalischer Sicht sind nur Lösungen mit Schwingungscharakter sinnvoll,
wir können daher annehmen, dass $k<0$.
Ein solches $k$ lässt sich in der Form $k=-\lambda^2$ schreiben.
Es gibt also ein $\lambda$ mit der Eigenschaft
\[
\frac1{c^2}\frac{T''(t)}{T(t)}=-\lambda^2
\]
oder
\[
T''(t)=-c^2\lambda^2 T(t).
\]
Dies ist eine gewöhnliche Differentialgleichung zweiter Ordnung, welche mit
bekannten Methoden gelöst werden kann.
Die allgemeine Lösung dieser Gleichung ist von der Form
\[
A\cos c\lambda t+B\sin c\lambda t.
\]

\subsection{Reduktion auf ein Eigenwertproblem}
\index{Eigenwertproblem}
Die linke Seite von (\ref{separiertMembran}) hängt nur von der Zeit ab, darf sich
also nicht ändern, wenn man $x$ oder $y$ variert. Als Funktion des Ortes
muss die rechte Seite also ebenfalls konstant sein:
\begin{align*}
\frac1{\varphi(x,y)}\left(
\frac{\partial^2\varphi}{\partial x^2}
+
\frac{\partial^2\varphi}{\partial y^2}
\right)&=-\lambda^2\\
\frac{\partial^2\varphi}{\partial x^2}
+
\frac{\partial^2\varphi}{\partial y^2}
=\Delta\varphi
&=-\lambda^2
\varphi(x,y)
\end{align*}
Die gesuchte Funktion $\varphi$ ist also ein Eigenvektor des linearen
Operators $\Delta$ zum Eigenwert $-\lambda^2$.
Nur die Eigenwerte des Operator $\Delta$ kommen also für die
Zeitabhängigkeitsgleichung in Frage.

\subsection{Separation von $x$ und $y$}
\index{Separation}
Für das Eigenwertproblem können wir erneut den Separationsansatz
\[
\varphi(x,y)=X(x)\cdot Y(y)
\]
versuchen.
Einsetzen in die Differentialgleichung ergibt
\begin{align*}
X''(x)Y(x)+X(x)Y''(y)&=-\lambda^2 X(x)Y(y)
\\
\frac{X''(x)}{X(x)}+\frac{Y''(y)}{Y(y)}&=-\lambda^2
\end{align*}
Jeder der Brüche hängt nur von jeweils einer Variable ab, was nur
möglich ist, wenn beide Terme konstant sind. Damit ist das Problem
reduziert auf zwei Gleichungen
\begin{align*}
X''(x)&=-\lambda_1^2X(x)\\
Y''(y)&=-\lambda_2^2Y(y)\\
\lambda_1^2+\lambda_2^2&=\lambda^2
\end{align*}
Die allgemeinen Lösungen dieser Gleichungen, die auch die Randbedingung
bei $x=0$ bzw.~$y=0$ erfüllt, sind
\begin{align*}
X(x)&=A\sin \lambda_1x\\
Y(y)&=B\sin \lambda_2y
\end{align*}
Die Randbedingungen für $x=a$ und $y=b$ können nur erfüllt werden,
wenn $\lambda_1a$ und $\lambda_2b$ Vielfache von $\pi$ sind, also
\[
\lambda_1=\frac{k\pi}a
\qquad
\text{und}
\qquad
\lambda_2=\frac{l\pi}b
\]
Die möglichen Werte von $\lambda$ sind also
\[
\lambda_{kl}^2=\left(\frac{k^2}{a^2} + \frac{l^2}{b^2}\right)\pi^2,\qquad k,l\in\mathbb Z
\]
Damit kann man jetzt die allgemeine Lösung des Schwingungsproblems aus den
Teillösungen
\[
\varphi_{kl}(x,y)=\sin \frac{k\pi}{a}x\sin\frac{l\pi}{b}y
\]
für das Eigenwertproblem
und den Teillösungen
\[
u_{kl}(x,y,t)
=
(A_{kl}\cos c\lambda_{kl} t+
B_{kl}\sin c\lambda_{kl} t)
\sin \frac{k\pi}{a}x\sin\frac{l\pi}{b}y
\]
für das zeitabhängige Problem
zu einer allgemeinen Lösung
\begin{equation}
u(x,y,t)=\sum_{k,l}
(A_{kl}\cos c\lambda_{kl} t+
B_{kl}\sin c\lambda_{kl} t)
\sin \frac{k\pi}{a}x\sin\frac{l\pi}{b}y
\label{allgemeineloesung}
\end{equation}
zusammensetzen.

\subsection{Anfangsbedingungen}
\index{Anfangsbedingungen}
Die allgemeine Lösung muss jetzt auch noch die Anfangsbedingung erfüllen:
\begin{align*}
\sum_{k,l}A_{kl}
\sin \frac{k\pi}{a}x\sin\frac{l\pi}{b}y&=f(x,y)\\
\sum_{k,l}B_{kl}c\lambda_{kl}
\sin \frac{k\pi}{a}x\sin\frac{l\pi}{b}y&=g(x,y)\\
\end{align*}
Die Koeffizienten $A_{kl}$ und $B_{kl}$ können in einfachen Fällen mit
Koeffizientenvergleich und im Allgemeinen mit Hilfe der Theorie
der Fourierreihen berechnet werden.
\index{Fourierreihe}

