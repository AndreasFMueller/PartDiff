%
% initial.tex -- XXX
%
% (c) 2019 Prof Dr Andreas Mueller
%
\rhead{Anfangsbedingungen}
\section{Anfangsbedingungen}
In den bisherigen Beispielen haben wir Lösungen einer partiellen
Differentialgleichung gesucht und gefunden, welche bestenfalls einen
Teil der Randbedingungen erfüllt haben.
So haben wir zwar sichergestellt, dass die schwingende Membran eingespannt
bleibt, aber die Auslenkung der Membran zu Beginn haben wir ignoriert.

Um zu verstehen, wie die Anfangsbedingungen ebenfalls berücksichtig
werden können, betrachten wir die Wellengleichung
\[
\frac{\partial^2 u}{\partial t^2}=\frac{\partial^2 u}{\partial x^2}
\]
auf dem Gebiet $(t,x)\in\mathbb R\times [0,\pi]$
mit den Randbedingungen
\[
u(t,0)=u(t,\pi)=0.
\]
Wir verwenden den Separationsansatz
$u(t,x)=T(t)\cdot X(t)$, welcher uns wie früher dargestellt auf eine
Gleichung
\[
\frac{T''(t)}{T(t)}=\frac{X''(x)}{X(x)}=-\lambda^2
\]
führt.
Die Gleichung 
\[
X''(x)=-\lambda^2 X(x)
\]
hat als Lösung Linearkombinationen von Sinus- und Kosinusfunktionen
\[
X(x)=A\cos\lambda x+B\sin\lambda x.
\]
Damit die Anfangsbedingung am linken Rand erfüllt ist, muss $A=0$
sein. Am rechten Rand bleibt daher nur $B\sin\lambda \pi$, und wir
müssen $B\ne 0$ annehmen, da sonst die ganze Lösung verschwinden
würde. $\sin\lambda \pi$ wird aber nur dann verschwinden, wenn
$\lambda$ eine ganze Zahl ist, also
\[
X_k(x)=B\sin kx, \quad 0<k\in\mathbb Z.
\]
Die dazu passende Lösung von $T''(t)=-k^2T(t)$ hat genau die
gleiche Form, so dass die allgemeine Lösung zum Wert $\lambda=k$
\[
u_k(t,x)=\sin kx\left(A_k\cos kt+B_k\sin kt\right)
\]
ist.

Diese Teillösungen $u_k(t,x)$ erfüllen bereits die Differentialgleichung
und die Randbedingungen. Noch nicht erfüllt werden die Anfangsbedingungen
zur Zeit $t=0$. Wir geben sie in der Form
\begin{align*}
u(0,x)&=f(x)\quad x\in[0,\pi]\\
\frac{\partial u}{\partial t}(0,x)&=g(x)\quad x\in[0,\pi]
\end{align*}
vor.

Wir suchen jetzt also eine Lösung in der Form
\[
u(t,x)=\sum_{k=1}^{\infty}
\left(A_k\cos kt+B_k\sin kt\right)
\sin kx,
\]
welche die Anfangsbedingung erfüllt. Durch Einsetzen erhält
man
\begin{align*}
\sum_{k=1}^{\infty}
A_k \sin kx
&=f(x)
\\
\sum_{k=1}^{\infty}
B_kk\sin kx
&=g(x)
\end{align*}
für $x\in[0,\pi]$.
Die Lösung $u(t,x)$ kann also vollständig bestimmt werden, indem man
die Anfangsbedingungen in eine Fourier-$\sin$-Reihe entwickelt. Sind
$\hat f(k)$ und $\hat g(k)$ die Fourier-Koeffizienten, wird die
vollständige Lösung
\[
u(t,x)
=
\sum_{k=1}^{\infty}(\hat f(k)\cos kt+\hat g(k)k\sin kt)\sin kx.
\]
Mit geeigneten Voraussetzungen an die Funktionen $f$ und $g$ werden
diese Reihen konvergieren.

