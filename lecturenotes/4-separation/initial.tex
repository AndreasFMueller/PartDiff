%
% initial.tex -- how to handle initial conditions with Fourier theory
%
% (c) 2019 Prof Dr Andreas Mueller
%
\rhead{Initial conditions}
\section{Initial conditions}
In the previous examples we have found solutions that only followed
some of the boundary conditions.
We had ensured that the membrane excursion $u(x,y,t)$ is $0$ along the
boundary, but we have neglected the conditions for time $t=0$ so far.
This is about to change.

We again illustrate the principle with the wave equation
for a vibrating string
\[
\frac{\partial^2 u}{\partial t^2}=\frac{\partial^2 u}{\partial x^2}
\]
on the domain
$(t,x)\in\mathbb R\times [0,\pi] $
with boundary conditions
\[
u(t,0)=u(t,\pi)=0.
\]
We use separation in the form
$u(t,x)=T(t)\cdot X(t)$,
which leads us as before to the equation
\[
\frac{T''(t)}{T(t)}=\frac{X''(x)}{X(x)}=-\lambda^2.
\]
The equation for $X(x)$
\[
X''(x)=-\lambda^2 X(x)
\]
has linear combinations of sine and cosine functions a solutions,
we write it as
\[
X(x)=A\cos\lambda x+B\sin\lambda x.
\]
In order for it to satisfy the boundary condition on the left boundary,
the cosine term may must vanish, we conclude $A=0$.
On the right boundary, the remaining term $B\sin\lambda \pi$ with $B\ne 0$
(otherwise we would have a trivial solution) must also vanish, 
which happens only if $\lambda\pi$ is a multiple of $\pi$, or only if
$\lambda$ is an integer.
Thus
\[
X_k(x)=B\sin kx, \quad 0<k\in\mathbb Z.
\]
The matching solution for $T$ now has to satisfy 
\[
T''(t)=-k^2T(t).
\]
We again find sine and cosine functions as solutions, but this time
we meet no restriction from the boundary conditions.
Combining the partial solutions 
\[
u_k(t,x)=\sin kx\left(A_k\cos kt+B_k\sin kt\right)
\]
will give the complete solution.

The functions $u_k(t,x)$ solve the differential equation and satisfy
the boundary conditions at $x=0$ and $x=\pi$, but the don't satisfy 
the boundary conditions at $t=0$.
Let's assume they boundary conditions are given as
\begin{align*}
u(0,x)&=f(x)\quad x\in[0,\pi]\\
\frac{\partial u}{\partial t}(0,x)&=g(x)\quad x\in[0,\pi].
\end{align*}

We now construct a linear combination
\[
u(t,x)=\sum_{k=1}^{\infty}
\left(A_k\cos kt+B_k\sin kt\right)
\sin kx,
\]
and attempt to determine the coefficients $A_k$ and $B_k$.
Substituting $u(t,x)$ in the boundary condition gives
\begin{align*}
\sum_{k=1}^{\infty}
A_k \sin kx
&=f(x)
\\
\sum_{k=1}^{\infty}
B_kk\sin kx
&=g(x)
\end{align*}
for $x\in[0,\pi]$.
So both functions must be expressible as function series using only 
sine functions.

To complete the solution, we have to develop the functions $f(x)$ and
$g(x)$ into Fourier sine series.
Let's call their respective Fourier coefficients
$\hat f(k)$ and $\hat g(k)$, then the complete solution becomes
\[
u(t,x)
=
\sum_{k=1}^{\infty}(\hat f(k)\cos kt+\hat g(k)k\sin kt)\sin kx.
\]
Under suitable conditions on $f$ and $g$, e.~g.~if both functions are
continuous, these series will converge.

