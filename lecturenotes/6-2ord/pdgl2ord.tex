%
% pdgl2ord.tex -- Klassifikation linearer PDGL zweiter Ordnung
%
% (c) 2016 Prof Dr Andreas Mueller, Hochschule Rapperswil
%
\chapter{Partial differential equations of second order\label{chapter-2ordnung}}
\lhead{Second Order partial differential equations}
Linear partial differential equations of second order have the form
\begin{equation}
\sum_{i,j=1}^na_{ij}\frac{\partial^2}{\partial x_i\,\partial x_j} u
+
\sum_{i=1}^nb_i\frac{\partial}{\partial x_i} u+cu=f.
\label{operator2ordnung}
\end{equation}
All the examples in chapter
\ref{chapter:examples}
match this template.
In spite of the superficial similarities, the solutions to these problems
exhibited completely different behaviour.
der Beispielgleichungen grundsätzlich verschiedenes Verhalten.
\begin{itemize}
\item The wave equation has solutions that propagate with finite 
velocity.
A change in initial conditions is noticed in more distant points of
the domain only after some delay.
\item
Changes in the boundary conditions of the heat equation affect the solution
for later times instantaneously, they propagate with infinite speed through
the domain, but they don't affect the past.
\item
Changing the initial conditions of the poisson problem
$\Delta \varphi=f$,
changes the solution immediately everyhwere in the domain.
\end{itemize}
In this chapter we will classify partial differential equations of
second order.
The goal is to show that any linear partial differential differential
equation of second order falls into one of those three categories.

%
% standard.tex -- 
%
% (c) 2019 Prof Dr Andreas Mueller, Hochschule Rapperswil
%
\section{Standard form}
\rhead{Standard form}
In the form (\ref{operator2ordnung})
the differential equation is defined by the choice of coefficients
$(a_{ij})$, $b_i$ and $c$.
They can also be functions of the independen variables $x_i$.

However, not all the coefficients are equally imporant in influencing
the character of the solutions.
A change of coordinates will also change the coefficients.
We can expect that suitable transformations will be able to turn
most coefficients to zero, so that only very few coefficients remain
that then determine the character of the solutions.
So the first step taken in this section is to find out how a change
of coordinates changes the coefficients.
The nexst step then is to find out what properties are invariant
under coordinate transformation.

\subsection{Linear transformation of the coordinates
\label{lineare-transformation}}
A coordinate transformation
\[
x_i=\sum_{j}t_{ij}x_j'
\]
transforms the derivates according to the chain rule
\[
\frac{\partial u}{\partial x'_j}
=
\sum_{i=1}^n
\frac{\partial x_i}{\partial x'_j} \frac{\partial u}{\partial x_i}
=
\sum_{i=1}^nt_{ij}\frac{\partial u}{\partial x_i}.
\]
We can derive from this how the coefficients $b_i$ will be transformed.
We denote the coefficients of the first derivatives in
(\ref{operator2ordnung}) expressed in $x_i'$-coordinate by $b_i'$.
In both coordinate systems, the first derivative terms have to agree, so
we have
\[
\sum_{i=1}^n b_i\frac{\partial u}{\partial x_i}
=
\sum_{i=1}^n b_i'\frac{\partial u}{\partial x_i'}
=
\sum_{i,j=1}^n b_i't_{ji}\frac{\partial u}{\partial x_j}
=
\sum_{j,i=1}^n b_j't_{ij}\frac{\partial u}{\partial x_i}.
\]
But this only works if
\[
b_i = \sum_{j=1}^n t_{ij}b_j'.
\]
If we denote the matrix inverse to the matrix $(t_{ij})$ by $(t'_{ij})$,
then rule that expresses the $b_i'$ by $b_i$ becomes
\[
b_j'=\sum_{i=1}^n t'_{ji}b_i.
\]

Let $T$ denote the matrix with coefficients $t_{ij}$ and let $b$
denote the vector of coefficients $b_i$.
Similarly for $t_{ij}'$ and $b_i'$.
Then we can write the coordinate transformation in matrix form as
\[
b=Tb'
\qquad\Rightarrow\qquad
b'=T^{-1}b=T'b.
\]

In a similar fashion we can treat the second derivatives.
First we compute the derivatives using the chain rule:
\[
\frac{\partial}{\partial x_i'}\frac{\partial}{\partial x_j'} u
=
\frac{\partial}{\partial x_i'}
\sum_{k=1}^nt_{kj}\frac{\partial}{\partial x_k}u
=
\sum_{k,l=1}^nt_{li}t_{kj}\frac{\partial}{\partial x_l}\frac{\partial}{\partial x_k}u.
\]
In transformed coordinates, we must get the analogous expression
with coefficients $a'_{ij}$, so we have
\begin{align*}
\sum_{i,j=1}^n
a'_{ij}\frac{\partial}{\partial x_i'}\frac{\partial}{\partial x_j'} u
&=
\sum_{i,j=1}^n
a_{ij}'
\sum_{k,l=1}^n
t_{li}t_{kj}\frac{\partial}{\partial x_l}\frac{\partial}{\partial x_k}u.
=
\sum_{k,l=1}^n
\biggl(
\sum_{i,j=1}^n
a_{ij}'
t_{li}t_{kj}
\biggr)
\frac{\partial}{\partial x_l}\frac{\partial}{\partial x_k}u
\\
&=
\sum_{i,j=1}^n
\underbrace{
\biggl(
\sum_{k,l=1}^n
a_{lk}'
t_{il}t_{jk}
\biggr)}_{\textstyle a_{ij}}
\frac{\partial}{\partial x_i}\frac{\partial}{\partial x_j}u.
\end{align*}
Thus the coefficients
$a_{ij}$ are transformed according to
\[
a_{ij}=\sum_{k,l=1}^n t_{il}a_{lk}'t_{jk},
\]
or in matrix form
\[
A=TA'T^t
\qquad\Rightarrow\qquad
A'=T'AT'^t.
\]
These rules are of course only true if 
$a_{ij}$ and $b_i$ are constants.
Otherwise we get additional terms from the derivatives of the coefficients.

\subsection{Influence of first order terms
\label{einfluss-terme-erster-ordnung}}
We want to show that the coefficients $b_i$ are of no importance to
the question, where boundary values need to be specified in order
for a particular linear partial differential equation problem
of second order to be well posed.

For an ordinary differential equation of second order, the first
order terms describe damping, which leads to exponentially decaying
amplitude.
By enhancing the solution with the help of an exponential factor,
we can counteract the damping and thus make clear whether we see
exponentiall decaying solutions or oscillatory solutions.

A similar statement holds true for the solution $u(x)$
of the partial differential equation
\eqref{operator2ordnung}.
To find this statement, let's consider the function
\[
v(x)=e^{k\cdot x} u(x),
\qquad
k\cdot x = \sum_{i=1}^n k_ix_i \quad(\text{scalar product})
\]
and look for a partial differential equation for $v$.

To this effect we compute the action of the part $M$ of the differential
operator consisting only of the second order terms of $L$:
\[
M=\sum_{i,j=1}^n a_{ij}\frac{\partial^2}{\partial x_i\,\partial x_j}.
\]
It's action on $v$ is
\begin{align*}
Mv
&=
\sum_{i,j=1}^na_{ij}
\frac{\partial}{\partial x_i}
\frac{\partial}{\partial x_j} e^{k\cdot x}u
\\
&=
e^{k\cdot x}
\biggl(
\sum_{i,j=1}^na_{ij}
\frac{\partial^2 u}{\partial x_i\,\partial x_j}
+
\sum_{i,j=1}^n a_{ij}k_i\frac{\partial u}{\partial x_j}
+
\sum_{i,j=1}^n a_{ij}k_j\frac{\partial u}{\partial x_i}
+
\sum_{i,j=1}^n a_{ij}k_ik_ju
\biggr)
\\
&=
e^{k\cdot x}
\biggl(
\sum_{i,j=1}^na_{ij}
\frac{\partial^2 u}{\partial x_i\,\partial x_j}
+
2\sum_{i,j=1}^n a_{ij}k_i\frac{\partial u}{\partial x_j}
+
\sum_{i,j=1}^n a_{ij}k_ik_ju
\biggr)
\end{align*}
In the last step we have used that we can rename the index of summation
however we want.

Now we use the freedom to choose the coefficients $k_i$.
We want to ensure that the middle term exactly matches the
first order derivatives in \eqref{operator2ordnung}.
For this wee need
\begin{equation}
b_j=2\sum_{i=1}^n a_{ij}k_i
\label{bequation}
\end{equation}
If the matrix $A$ with coefficients $a_{ij}$ is regular, we can
solve this linear system of equations.
We obtain
\[
k=
{A^t}^{-1}
\frac{b}2.
\]
If we write
\[
\alpha = \sum_{i,j=1}^n a_{ij}k_ik_j,
\]
it follows that the equation
$Lu=f$ holds precisely when
\begin{align}
Mv
&=
e^{k\cdot x}\biggl(
\sum_{i,j=1}^n a_{ij}\frac{\partial^2}{\partial x_i\,\partial x_j}u
+\sum_{i=1}^n b_i\frac{\partial u}{\partial x_i}+\alpha u
\biggr)
\\
&=
e^{k\cdot x}
(L+\alpha -c)u
=
e^{k\cdot x}Lu + (\alpha -c)v
\notag
\\
(M-(\alpha - c))v&=e^{k\cdot x}Lu=e^{k\cdot x}f
\label{reducedoperator}
\end{align}
holds.
The equation \eqref{operator2ordnung} has a solution $u$ precisely
when function $v(x)=e^{k\cdot x}u(x)$  is a solution of the equation
\eqref{reducedoperator}.
The solution $u$ of the equation \eqref{operator2ordnung} in the point $x$
is changed by a change in the boundary condition precisely when the solution
$v$ of the equation \eqref{reducedoperator} also changes.

In summary, if the matrix $a_{ij}$ is regular, or more generally if the
system of equations \eqref{bequation} has a solution, then we can
ignore the first order terms as far as the questions at the beginning
of this chapter are concerned.

If $a_{ij}$ is not regular, then the first order terms may not all
be removed.

\subsection{Symbol matrix}
Because the second order derivatives do not depend on the order
of differentiation,
\[
\frac{\partial^2}{\partial x_i\,\partial x_j}
=
\frac{\partial^2}{\partial x_j\,\partial x_i},
\]
the coeficients are 
$a_{ij}$ and $a_{ji}$ are interdependent.
In particular, we can additionally require that $a_{ij}=a_{ji}$,
which means that the matrix $A$ be symmetric.

\begin{definition}
If $L$ is the linear partial differential operator of second order
defined by
\[
Lu=\sum_{i,j=1}^na_{ij}\frac{\partial u}{\partial x_i\,\partial x_j},
\]
then the matrix $(a_{ij})$ is called the symbol of the operator
\end{definition}

In section~\ref{einfluss-terme-erster-ordnung} we have seen how to
transform the equation to make the first order derivatives to
disappear if the equation
\[
Ak=\frac{b}2
\]
has a solution.
If $A$ is not regular, then $b$ can be removed if and only if $b$ is in 
the image of the matrix $A$.
Vectors orthognal to the image can be transformed away.
If $b\perp \operatorname{im} A$, then 
\[
b\cdot Ax=(A^tb)\cdot x\;\forall x
\qquad\Rightarrow\qquad
Ab=0,
\]
a vector $b$ in the kernel of $A$ cannot be removed.

\begin{beispiel}
As an example consider the operator
\[
L
=
\sum_{i=2}^n\frac{\partial^2}{\partial x_i^2}
+
\sum_{i=1}^nb_i\frac{\partial}{\partial x_i}
\]
with the symbol matrix
\[
A=\begin{pmatrix}
      0&      0& \dots&0\\
      0&      1& \dots&0\\
\vdots &\vdots &\ddots&\vdots\\
      0&      0&\dots &1
\end{pmatrix}
\]
The vector $e_1$ is in the kernel of $A$, so the first order derivative
with respect to $x_1$ cannot be transformed away.
The simplest form for the operator that we can achieve is thus
\[
\sum_{i=2}^n\frac{\partial^2}{\partial x_i^2}
+
\frac{\partial}{\partial x_1}.
\]
This is the operator we have seen in the heat equation.
\end{beispiel}


%
% types.tex
%
% (c) 2008 Prof Dr Andreas Mueller
%
\section{Spezielle Typen von partiellen Differentialgleichungen\label{klassifikation:spezielletypen}}
\rhead{Spezielle PDGL}
Für einige spezielle Typen von Differentialgleichungen werden wir
die Frage nach Existenz und Eindeutigkeit einer Lösung beantworten
können, und manchmal sogar eine explizite Lösung angeben können.
Dazu gehören die linearen und die quasilinearen partiellen
Differentialgleichungen, die in diesem Abschnitt beschrieben werden.

\subsection{Lineare partielle Differentialgleichungen\label{klassifikation:linear}}
Die bisher vorgestellten Differentialgleichungen sind also alle lineare
Ausdrücke in den ersten und zweiten Ableitungen, man könnte sie in der
Form
\begin{align*}
\sum_{i,j=1}^n a_{ij}(x)\frac{\partial}{\partial x_i} \frac{\partial}{\partial x_j}\psi(x)
+\sum_{i=1}^na_i(x)\frac{\partial}{\partial x_i}\psi(x)&=f(x)
\\
\sum_{i,j=1}^n a_{ij}(x)\partial_i \partial_j\psi(x)
+\sum_{i=1}^na_i(x)\partial_i\psi(x)&=f(x)
\end{align*}
Die Gleichung heisst homogen, wenn $f=0$ ist.

Damit diese Probleme überhaupt eine Lösung haben, müssen noch 
Randbedingungen hinzugefügt werden.
Auch diese lassen sich in der Form linearer Gleichungen zwischen den 
Funktionswerten und den Normalableitungen von $\psi$ auf dem Rand des
Gebietes gegeben:
\begin{align*}
a\psi(x)+
b\frac{\partial}{\partial n}\psi(x)
&=g(x)\quad\forall x\in\gamma\\
a\psi(x)+b\partial_n\psi(x)&=g(x)\quad\forall x\in\gamma
\end{align*}
Die Randbedingungen heissen homogen, wenn $g=0$ ist.

Sind $\psi_1$ und $\psi_2$ zwei Lösungen der homogenen Gleichungen und der
homogenen Randbedingungen, dann ist auch $\alpha_1\psi_1+\alpha_2\psi_2$
Lösungen der homogenen Gleichung under homogenen Randbedingungen:
\begin{align*}
&\sum_{i,j=1}^n a_{ij}(x)\partial_i \partial_j
(\alpha_1\psi_1(x)+\alpha_2\psi_2(x))
+\sum_{i=1}^na_i(x)\partial_i
(\alpha_1\psi_1(x)+\alpha_2\psi_2(x))
\\
+
\alpha_1
&\sum_{i,j=1}^n a_{ij}(x)\partial_i \partial_j
\psi_1(x)
+
\alpha_1
\sum_{i=1}^na_i(x)\partial_i
\psi_1(x)
\\
+
\alpha_2
&\sum_{i,j=1}^n a_{ij}(x)\partial_i \partial_j
\psi_2(x)
+
\alpha_2
\sum_{i=1}^na_i(x)\partial_i
\psi_2(x)
=0
\end{align*}
oder für die Randbedingung
\begin{align*}
a(\alpha_1\psi_1(x) +\alpha_2\psi_2(x))
\;+&b\partial_n
(\alpha_1\psi_1(x) +\alpha_2\psi_2(x))\\
=\alpha_1(a\psi_1(x)
\;+&b\partial_n
\psi_1(x))\\
+\;\alpha_2(a\psi_2(x)
\;+&b\partial_n
\psi_2(x))
=0\quad\forall x\in\gamma
\end{align*}
Die Lösungen einer homogenen partiellen Differentialgleichung
bilden also einen Vektorraum. Insbesondere lassen sich beliebige
Lösungen der inhomogenen Differentialgleichung dadurch finden, dass
man eine partikuläre Lösung $\psi_p$ der homogenen Differentialgleichung
findet, und dazu eine beliebige Lösung der homogenen Differentialgleichung
$\psi_h$ hinzuaddiert.

\subsection{Quasilineare partielle Differentialgleichungen erster Ordnung\label{klassifikation:quasilinear}}
Einen interessanten Spezialfall bilden die quasilinearen PDGL erster
Ordnung. Sie sind nicht unbedingt linear, aber die partiellen
Ableitungen erster Ordnung kommen nur linear vor. Die Funktion
$
F(x,y,u,p,q)
$
ist also linear in $p$ und $q$. Die Variablen $x$ und $y$ sowie die
gesuchte Funktion können daher nur in den Koeffizienten der
Variablen $p$ und $q$ vorkommen, $F$ muss von der Form
\[
F(x,y,u,p,q)=a(x,y,u)p+b(x,y,u)q+c(x,y,u)
\]
sein. Eine quasilineare Differentialgleichung in zwei Variablen
hat also die Form
\[
a(x,y,u(x,y))\frac{\partial u}{\partial x}+b(x,y,u(x,y))\frac{\partial u}{\partial y}
=c(x,y,u(x,y)).
\]
Für mehr Variablen $x_1,\dots,x_n$ gilt analog, dass eine quasilineare
Differentialgleichung die Form
\[
a_1(x_1,\dots,x_n,u)\frac{\partial u}{\partial x_1}
+
a_2(x_1,\dots,x_n,u)\frac{\partial u}{\partial x_2}
+\dots
+
a_n(x_1,\dots,x_n,u)\frac{\partial u}{\partial x_n}
=c(x_1,\dots,x_n,u)
\]
hat.

Natürlich lassen sich auch quasilineare partielle Differentialgleichungen
höherer Ordnung definieren, in einer solchen Differentialgleichung
kommen zwar höheren partielle Ableitungen vor, aber immer nur linear,
man kann sie also immer in der Form schreiben:
\[
\sum_{\bf k} a_{\bf k}(x,y,u)\partial_{\bf k} u(x,y) = 0,
\]
worin die Funktionen $a_{\bf k}(x,y,u)$ nur für endlich viele
Multiindizes ${\bf k}$ von $0$ verschieden sind.

\subsection{Nichtlineare Gleichungen\label{klassifikation:nichtlinear}}
Die bisher vorgestellten Beispiele von partiellen Differentialgleichungen
sind alle linear. Viele Gleichungen der Physik sind jedoch
nicht linear.  Berühmtestes Beispiel sind die Gleichungen, die
die Strömung eines Gases beschreiben. Bereits Leonhard Euler hat 
für die Strömung eines idealen Gases ein System von partiellen
\index{Dichte}
\index{Druck}
Differentialgleichungen gefunden für die Dichte $\varrho$, den Druck
$p$ und die Strömungsgeschwindigkeit $\vec v$, alle drei sind Funktionen
\index{Stromungsgeschwindigkeit@Str\ömungsgeschwindigkeit}
von allen drei Raumkoordinaten und der Zeit. Die wichtigste davon
ist die Eulersche Gleichung:
\index{Eulersche Gleichung}
\begin{align*}
\frac{\partial \vec v}{\partial t}
+(\vec v\cdot \nabla)\vec v
=-\frac1{\varrho}\operatorname{grad}p
\\
\frac{\partial v_i}{\partial t}
+\sum_{j=1}^3v_j\frac{\partial v_i}{\partial x_j}
=
-\frac1{\varrho}\frac{\partial p}{\partial x_i}
\end{align*}
Diese Gleichung kann nicht linear sein, weil im zweiten Term
Produkte von $v_j$ mit Ableitungen von $v_i$ vorkommen.
Berücksichtigt man auch noch die Zähigkeit, wird die Gleichung
noch komplizierter:
\[
\varrho\left(
\frac{\partial\vec v}{\partial t}
+
(\vec v\cdot\nabla)\vec v
\right)
=
-\operatorname{grad}p+\eta\Delta \vec v+\left(\zeta+\frac{\eta}3\right)
\operatorname{grad}\operatorname{div}\vec v
\]

In einer Dimension bleibt von dieser Gleichung nur noch eine Komponente
$u(t,x)$ übrig, für die eine Gleichung der ungefähren Form
\[
\frac{\partial u}{\partial t}+u\frac{\partial u}{\partial x}
=\eta\frac{\partial^2u}{\partial x^2}
\]
gelten muss (wir haben den Druckgradienten vernachlässigt und
$\varrho = 1$ angenommen). Diese Gleichung wurde von
Johannes Martinus Burgers ausgiebig studiert, und heisst
daher Gleichung von Burgers.
Eine Lösung des Anfangswertproblems der Gleichung von Burgers kann 
in expliziter Form gefunden werden, wir beschreiben diese in Kapitel
\ref{chapter-nichtlinear}.



%
% canonical.tex -- 
%
% (c) 2019 Prof Dr Andreas Mueller, Hochschule Rapperswil
%
\section{Canonical Form}
\rhead{Canonical Form}
For linear partial differential equations with constant coefficients,
i.~e.~equations where the $a_{ij}$ do not depend on the location or time,
a rotation of the coordinate system can be found that turns the coefficent
matrix into a diagonal matrix.
Rescaling the axis even allows to get the diagonal elements in $\{0,\pm1\}$.
This raises the question whether something similar is also possible with
more general, i.~e.~location dependen coefficients.

In two dimensions, we can find a suitable coordinate system using the
following idea.
In every point $(x,y)$ there are two vectors $\vec e_+$ und $\vec e_-$
which are eigenvectors for the larger and the smaller eigenvalue of the
coefficent matrix $A$ respectively.
At least if the eigenvalues are different, these vector fields can be
constructed in a continuous manner.
And because $A$ is symmetric, the two vectors are orthogonal everywhere.

Now construct solution curves for each vector field.
Each set of curves is parametrized by a parameter, call them $\xi$ and $\eta$.
Each point of the plane is identified by a curve from each set that
run through the point and thus by the coordinates $\xi$ und $\eta$.
In these coordinates, the matrix $A$ gets diagonal form.

This idea does not generalize to three dimensions.
Additional conditions on the vector fields of eigenvectors are needed.

But it is still possible to achieve diagonal form in any point of the
domain.
By continuity, the differential equation will be in approximate diagonal
form in any neighborhood of the point.
This is sufficient to study the local behavior of the solutions.
Consequently it is sufficient to study the behavior of the solutions
of the following standard differential equations:
\begin{align*}
\Delta u&=0&&\text{elliptic}\\
\partial_tu&=k\Delta u&&\text{parabolic}\\
\partial_t^2u&=a^2\Delta u&&\text{hyperbolic}\\
\end{align*}
which is precisely what we are going to do in the following three chapters.
\index{elliptic}
\index{parabolic}
\index{hyperbolic}



%
% splitting.tex --
%
% (c) 2019 Prof Dr Andreas Mueller, Hochschule Rapperswil
%
\section{Splitting the solution}
\rhead{Splitting the solution}
We would like to better understand the solution space of a linear partial
differential equation of second order with operator $L$.
We start with the formulation
\begin{equation}
\begin{aligned}
Lu&=f&&\qquad\text{in $\Omega$}\\
 u&=g&&\qquad\text{on $\partial\Omega$,}
\end{aligned}
\label{pdgl2ord:allg}
\end{equation}
of the problem and are looking for solutions.
Because the equation is linear, we can attempt to build the solution
as a linear combination of functions that each partially solve the
problem.

In a firs step, we try to solve the equation without regard to the
boundary conditions.
We call such a solution $u_p$, it solves
\begin{equation}
Lu_p=f\qquad\text{in $\Omega$.}
\label{pdgl2ord:up}
\end{equation}
We don't require anything about the values of $u_p$ on the boundary
$\partial\Omega$.

To solve the original problem, we now need an additional summand $u_r$
that takes care of the boundary values.
In order for $u_p+u_r$ to be a better solution of the original problem,
it still has to solve the partial differential equation:
\[
L(u_p+u_r)=f+Lu_r=f\qquad\Rightarrow\qquad Lu_r=0.
\qquad\text{in $\Omega$.}
\]
In addition, we want to make sure that the boundary conditions
\[
u_p+u_r=g\qquad\Rightarrow\qquad u_r=g-u_p\qquad\text{auf $\partial\Omega$}
\]
are satisfied.
So the function $u_r$ has to be a solution of the following partial
differential equation problem:
\begin{equation}
\begin{aligned}
Lu_r&=0&&\qquad\text{in $\Omega$}\\
 u_r&=g-u_p&&\qquad\text{auf $\partial\Omega$}
\end{aligned}
\label{pdgl2ord:ur}
\end{equation}
Note that $u_r$ satisfies a simpler partial differential equation,
simpler mainly because it is homogenous.
The complication has been shifted to the boundary values.

It could still be possible that the original problem has more solutions
than the one just found.
We should be able to find such solutions from $u_p+u_r$ by adding
an additional term $u_h$.
This term must not destroy what we have accomplished with the first
two steps,
it thereforem ust satisfy the equations
\begin{equation}
\begin{aligned}
L(u_p+u_r+u_h)&=f+Lu_h&&\Rightarrow&Lu_h&=0&&\text{in $\Omega$}\\
  u_p+u_r+u_h &=g +u_h&&\Rightarrow& u_h&=0&&\text{auf $\partial\Omega$}
\end{aligned}
\label{pdgl2ord:uh}
\end{equation}
A solution of the problem~(\ref{pdgl2ord:allg}) thus consists of three
terms $u_p$, $u_r$ and $u_h$:
\begin{enumerate}
\item
$u_p$ solves the differential equation~(\ref{pdgl2ord:up}) disregarding
the boundary conditions.
\item
$u_r$ solves the differential equation~(\ref{pdgl2ord:ur}),
$u_p+u_r$ solves the original problem.
\item
If the homogeneous problem~(\ref{pdgl2ord:uh}) with homogeneous
boundary conditions has nontrivial solution, then the solution
of the original problem is not unique.
Any solution $u_h$ of the homogeneous problem gives an additional
solution.
\end{enumerate}




