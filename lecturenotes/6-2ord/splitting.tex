%
% splitting.tex --
%
% (c) 2019 Prof Dr Andreas Mueller, Hochschule Rapperswil
%
\section{Splitting the solution}
\rhead{Splitting the solution}
We would like to better understand the solution space of a linear partial
differential equation of second order with operator $L$.
We start with the formulation
\begin{equation}
\begin{aligned}
Lu&=f&&\qquad\text{in $\Omega$}\\
 u&=g&&\qquad\text{on $\partial\Omega$,}
\end{aligned}
\label{pdgl2ord:allg}
\end{equation}
of the problem and are looking for solutions.
Because the equation is linear, we can attempt to build the solution
as a linear combination of functions that each partially solve the
problem.

In a firs step, we try to solve the equation without regard to the
boundary conditions.
We call such a solution $u_p$, it solves
\begin{equation}
Lu_p=f\qquad\text{in $\Omega$.}
\label{pdgl2ord:up}
\end{equation}
We don't require anything about the values of $u_p$ on the boundary
$\partial\Omega$.

To solve the original problem, we now need an additional summand $u_r$
that takes care of the boundary values.
In order for $u_p+u_r$ to be a better solution of the original problem,
it still has to solve the partial differential equation:
\[
L(u_p+u_r)=f+Lu_r=f\qquad\Rightarrow\qquad Lu_r=0.
\qquad\text{in $\Omega$.}
\]
In addition, we want to make sure that the boundary conditions
\[
u_p+u_r=g\qquad\Rightarrow\qquad u_r=g-u_p\qquad\text{auf $\partial\Omega$}
\]
are satisfied.
So the function $u_r$ has to be a solution of the following partial
differential equation problem:
\begin{equation}
\begin{aligned}
Lu_r&=0&&\qquad\text{in $\Omega$}\\
 u_r&=g-u_p&&\qquad\text{auf $\partial\Omega$}
\end{aligned}
\label{pdgl2ord:ur}
\end{equation}
Note that $u_r$ satisfies a simpler partial differential equation,
simpler mainly because it is homogenous.
The complication has been shifted to the boundary values.

It could still be possible that the original problem has more solutions
than the one just found.
We should be able to find such solutions from $u_p+u_r$ by adding
an additional term $u_h$.
This term must not destroy what we have accomplished with the first
two steps,
it thereforem ust satisfy the equations
\begin{equation}
\begin{aligned}
L(u_p+u_r+u_h)&=f+Lu_h&&\Rightarrow&Lu_h&=0&&\text{in $\Omega$}\\
  u_p+u_r+u_h &=g +u_h&&\Rightarrow& u_h&=0&&\text{auf $\partial\Omega$}
\end{aligned}
\label{pdgl2ord:uh}
\end{equation}
A solution of the problem~(\ref{pdgl2ord:allg}) thus consists of three
terms $u_p$, $u_r$ and $u_h$:
\begin{enumerate}
\item
$u_p$ solves the differential equation~(\ref{pdgl2ord:up}) disregarding
the boundary conditions.
\item
$u_r$ solves the differential equation~(\ref{pdgl2ord:ur}),
$u_p+u_r$ solves the original problem.
\item
If the homogeneous problem~(\ref{pdgl2ord:uh}) with homogeneous
boundary conditions has nontrivial solution, then the solution
of the original problem is not unique.
Any solution $u_h$ of the homogeneous problem gives an additional
solution.
\end{enumerate}


