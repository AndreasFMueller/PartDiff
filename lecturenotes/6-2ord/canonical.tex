%
% canonical.tex -- 
%
% (c) 2019 Prof Dr Andreas Mueller, Hochschule Rapperswil
%
\section{Canonical Form}
\rhead{Canonical Form}
For linear partial differential equations with constant coefficients,
i.~e.~equations where the $a_{ij}$ do not depend on the location or time,
a rotation of the coordinate system can be found that turns the coefficent
matrix into a diagonal matrix.
Rescaling the axis even allows to get the diagonal elements in $\{0,\pm1\}$.
This raises the question whether something similar is also possible with
more general, i.~e.~location dependen coefficients.

In two dimensions, we can find a suitable coordinate system using the
following idea.
In every point $(x,y)$ there are two vectors $\vec e_+$ und $\vec e_-$
which are eigenvectors for the larger and the smaller eigenvalue of the
coefficent matrix $A$ respectively.
At least if the eigenvalues are different, these vector fields can be
constructed in a continuous manner.
And because $A$ is symmetric, the two vectors are orthogonal everywhere.

Now construct solution curves for each vector field.
Each set of curves is parametrized by a parameter, call them $\xi$ and $\eta$.
Each point of the plane is identified by a curve from each set that
run through the point and thus by the coordinates $\xi$ und $\eta$.
In these coordinates, the matrix $A$ gets diagonal form.

This idea does not generalize to three dimensions.
Additional conditions on the vector fields of eigenvectors are needed.

But it is still possible to achieve diagonal form in any point of the
domain.
By continuity, the differential equation will be in approximate diagonal
form in any neighborhood of the point.
This is sufficient to study the local behavior of the solutions.
Consequently it is sufficient to study the behavior of the solutions
of the following standard differential equations:
\begin{align*}
\Delta u&=0&&\text{elliptic}\\
\partial_tu&=k\Delta u&&\text{parabolic}\\
\partial_t^2u&=a^2\Delta u&&\text{hyperbolic}\\
\end{align*}
which is precisely what we are going to do in the following three chapters.
\index{elliptic}
\index{parabolic}
\index{hyperbolic}


