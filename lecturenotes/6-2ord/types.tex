%
% types.tex -- 
%
% (c) 2019 Prof Dr Andreas Mueller, Hochschule Rapperswil
%
\section{Types of partial differential equations}
\rhead{Classification}
In section~\ref{lineare-transformation}
it was shown that a coordinate transform with matrix $T$ transforms
the symbol matrix $A$ according to
\[
A'=TAT^t .
\]
Linear algebra teaches that for any symmetric matrix $A$ it is always possible 
to find an orthogonal matrix $T$ such that $A'=TAT^t$ is diagonal with
the eigenvalues of $A$ on the the diagonal:
\[
TAT^t
=
\begin{pmatrix}\lambda_1&\dots&0\\
\vdots&\ddots&\vdots\\
0&\dots&\lambda_n
\end{pmatrix}.
\]
Using a suitable coordinate system it is thus always possible to 
bring a partial differential operator of second order into the form
\[
\sum_{i=1}\lambda_i\frac{\partial^2}{\partial x_i^2}u.
\]

By stretching the coordinate axis $x_i$ by 
$\sqrt{|\lambda_i|}$
we can furthermore achieve that the differential operator
has the form
\[
\sum_{i=1}\varepsilon_i\frac{\partial^2}{\partial x_i^2}u,
\]
where the numbers $\varepsilon_i$ have values $0$, $1$ or $-1$.

If needed, we can also change the sign of the differential equation.
This changes the sign of the symbol matrix and the eigenvalues, which
allows us to have the majority of eigenvalues positive.

This means that partial differential operators of second order are
characterised by the signs of the eigenvalues of the symbol matrix.
We denote the number of positive eigenvalues $P$, the number of negative
eigenvalues $N$ and the number of zero eigenvalues $Z$.
The examples from chapter \ref{chapter-beispiele}
can be catalogued as follows:
\begin{center}
\begin{tabular}{l|ccc}
Differential operator&P&N&Z
\\
\hline
Laplace&
$n$&$0$&$0$
\\
waves&
$n-1$&$1$&$0$
\\
heat&
$n-1$&$0$&$1$
\end{tabular}
\end{center}
We derive the following classification:

\begin{definition} The partial differential equation
(\ref{operator2ordnung})
with numbers $P$, $N$ and $Z$ are called
\begin{center}
\begin{tabular}{lcl}
hyperbolic&provided&$Z=0$ and $P=1$ or $P=n-1$\\
parabolic&provided&$Z>0$\\
elliptic&provided&$Z=0$ and $P=n$ or $P=0$\\
ultrahyperbolic&provided&$Z=0$ and $1<P<n-1$
\end{tabular}
\end{center}
\end{definition}
In particular, the equations in chapter \ref{chapter-beispiele}
fall into these categories as follows:
\begin{itemize}
\item {\bf elliptisch:} Potential problem
\item {\bf hyperbolisch:} wave equation, linearized supersonic flow
\item {\bf parabolisch:} heat equation, diffusion
\end{itemize}

