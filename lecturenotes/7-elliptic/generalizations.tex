%
% generalizations.tex
%
% (c) 2019 Prof Dr Andreas Mueller
%
\section{Verallgemeinerungen}
\rhead{Verallgemeinerungen}
Die in diesem Kapitel gefundenen Methoden können in verschiedene Richtungen
verallgemeinert werden.
\subsection{Neumann-Randbedingungen}
\index{Neumann-Randbedingungen}
Das Dirichlet-Problem wurde mit Hilfe einer Greenschen Funktion $G(x,\xi)$
gelöst, welche als Funktion von $x$ die Randbedingung
\[
G(x,\xi)=0\quad\forall x\in\partial\Omega
\]
erfüllt.
Verwendet man stattdessen eine Greensche Funktion, welche die
%\marginpar{\tiny Greensche Funktion für Neumann-Randbedingungen}
Neumann-Randbedingungen
\[
\frac{\partial}{\partial n}G(x,\xi)=0\quad\forall x\in\partial\Omega
\]
erfüllt, bleibt im Beweis von \ref{dirichletloesung}
die Formel
\begin{align*}
u(x)&=\int_{\Omega}G(x,\xi)f(\xi)\,d\mu(\xi)+\int_{\partial\Omega}G(x,\xi)\operatorname{grad}u(\xi)\cdot dn
\end{align*}
stehen. Für Neumann-Randbedingungen $\partial_n u=g$ kann man dies ersetzen
durch
\begin{align*}
u(x)&=\int_{\Omega}G(x,\xi)f(\xi)\,d\mu(\xi)+\int_{\partial\Omega}G(x,\xi)g(\xi)\,d\mu(\xi)
\end{align*}
Auch für Neumann-Randbedingungen gibt es also eine Greensche Funktion, welche
das Problem für beliebige Randwerte löst.

\subsection{Allgemeinere Operatoren}
Die Greensche Funktion kann unter gewissen Voraussetzungen
an die Koeffizienten auch für beliebige elliptische Operatoren
\[
Lu=\biggl(\sum_{i,j}a_{ij}\frac{\partial^2}{\partial x_i\partial x_j}
+\sum_ib_i\frac{\partial}{\partial x_i} +c\biggr)u
\]
zweiter Ordnung konstruiert werden.

\subsection{Abstrakte Formulierung}
Das in diesem Kapitel erreichte kann auch wie folgt formuliert werden.
Gegeben war elliptischen Operator $L$ und ein weiterer Operator $B$,
der aus der Funktion $u$ die relevanten Randwerte ermittelte. $Bu$
ist ein Funktion auf dem Rand $\partial \Omega$ des Gebietes. 
Das Problem
\begin{align*}
Lu&=f(x)&&x\in\Omega
\\
Bu&=g(x)&&x\in\partial\Omega
\end{align*}
konnte mit Hilfe einer Integralformel
\[
u(x)=\int_\Omega G(x,\xi)f(\xi)\,d\mu(\xi)+\int_{\partial \Omega}K(x,\xi)g(\xi)\,d\mu(\xi)
\]
gelöst werden.
Die Funktionen $G$ und $K$ ermöglichen also, die Operatoren $L$ und $B$
zu invertieren.

