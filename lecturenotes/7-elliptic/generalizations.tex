%
% generalizations.tex
%
% (c) 2019 Prof Dr Andreas Mueller
%
\section{Generalizations}
\rhead{Generalizations}
The methods presented in this chapter can be generalized in various
directions.

\subsection{Neumann boundary conditions}
\index{Neumann boundary conditions}
The Poisson problem with Dirichlet boundary conditions was solved
using Green's function
$G(x,\xi)$ which, as a function of $x$ satisfies homogeneous
Dirichlet boundary conditions
\begin{align*}
G(x,\xi)
&=
0\quad\forall x\in\partial\Omega.
\\
\intertext{If we construct a Green's function which satisfies homogenous 
Neumann boundary conditions}
\frac{\partial}{\partial n}G(x,\xi)
&=
0\quad\forall x\in\partial\Omega,
\end{align*}
then what remains from the proof of theorem \ref{dirichletloesung}
is the formula
\begin{align*}
u(x)&=\int_{\Omega}G(x,\xi)f(\xi)\,d\mu(\xi)+\int_{\partial\Omega}G(x,\xi)\operatorname{grad}u(\xi)\cdot dn
\end{align*}
For homgeneous Neumann boundary conditions
$\partial_n u=g$, one can replace that by
\begin{align*}
u(x)&=\int_{\Omega}G(x,\xi)f(\xi)\,d\mu(\xi)+\int_{\partial\Omega}G(x,\xi)g(\xi)\,d\mu(\xi)
\end{align*}
So for Neumann boundary conditions there also is a Green's function
which solves the Neumann boundary value problem for arbitrary 
boundary values.

\subsection{More general operators}
Green's function can be constructed under suitable conditions
for the coefficients of the operator also for arbitrary
elliptic operators
\[
Lu=\biggl(\sum_{i,j}a_{ij}\frac{\partial^2}{\partial x_i\partial x_j}
+\sum_ib_i\frac{\partial}{\partial x_i} +c\biggr)u
\]
of second order.

\subsection{Abstract solution}
The principles in this chapter can be summarized and generalized as 
follows.
Given an elliptic operator $L$ and an additional operator $B$ which
extracts the boundary values from $u$.
For any function $u$ on $\bar\Omega$ the function $Bu$ is defined on
$\partial\Omega$.
The boundary value problem then can be stated as 
\[
\begin{aligned}
Lu&=f(x)&&x\in\Omega
\\
Bu&=g(x)&&x\in\partial\Omega
\end{aligned}
\]
and can be solved by some integral formular of the form
\[
u(x)
=
\int_\Omega G(x,\xi)f(\xi)\,d\mu(\xi)
+
\int_{\partial \Omega}K(x,\xi)g(\xi)\,d\mu(\xi).
\]
The functions $G$ and $K$ allow the operators $L$ and $B$ to be inverted.

