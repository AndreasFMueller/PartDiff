%
% inverse.tex -- 
%
% (c) 2019 Prof Dr Andreas Mueller
%
\section{Does there exist an operator inverse to $\Delta$?}
\rhead{Does there exist $\Delta^{-1}$?}
We want to solve the problem $\Delta u=f$.
If we had an inverse operator $\Delta^{-1}$, we could use it to
find the solution $u=\Delta^{-1}f$.
This dream is too good to be true, but the examples below 
illustrate that for elliptic operators there is at least a partial
realization of it.

\subsection{Derivative and Integration}
The equation 
\[
u''(x)=f(x)
\]
in one dimension can be solved by simple integration.
To this effect we need to find an antiderivative $F$ of $f$, i.~e.~$F'=f$
and then an antiderivative $u$ of $F$.
In spite of $u''(x)=F'(x)=f(x)$, $u$ need no be a solution because
it does not satisfy the boundary conditions.

\subsection{Linear equations}
By discretization the differential equation $u''=f$ on the interval
$[a,b]$ can be turned into a linear system of equations of
the form $Au=f$.
The components of $u$ and $f$ are values of the functions of $u$ and $f$
in selected points of the domain.
And since we have seen that the solution is unique for elliptic operators,
there should be an inverse matrix $G$ that can be used to find the
solution in the form $u=Gf$.

The matrix product is
\[
\sum_{j}g_{ij}f_j = u_i,
\]
and the $g_{ij}$ can also be considred to be values of a function $G(x,y)$
defined in particular points of $\Omega\times\Omega$.

If the points used for discretization come closer together, the sum
can ultimately better be approximated by an integral of the form
\[
u(x)=\int_a^b G(x,\xi)f(\xi)\,d\xi.
\]
So we are left with the question whether there actually exists such
a function $G(x,\xi)$.
Still, there is no guarantee that this will be a solution, as
the boundary conditions may not be satisfied.

\subsection{Boundary conditions}
The boundary conditions also enter linearly into the problem.
In the one dimensional problem we have two points on the boundary
so we expect a contribution of the form
\[
K(x,a) g(a) + K(x,b)g(b),
\]
where $K(x,\xi)$ is another function defined on $x\in [a,b]$ and
$\xi \in\{a,b\}=\partial[a,b]$.
If the boundary is more complicated and consists of multiple points,
the contribution generalizes to a sum
\[
\sum_{i}K(x,a_i)g(a_i),
\]
where all the points $a_i$ are on the boundary.
Again, with the boundary becoming a curve or surface, the
contribution of the boundary values must be computed with
the help of an integral
\[
\int_{\partial\Omega}K(x,\xi)g(\xi)\,d\xi
\]
over the boundary of the domain.
So we expect a solution in the form
\begin{equation}
u(x)=\int_{\Omega} G(x,\xi)f(\xi)\,d\xi + \int_{\partial\Omega} K(x,\xi)g(\xi)\,d\xi.
\label{greenformula}
\end{equation}
The goal of the subsequent sections is to show that in some cases,
such functions actually exists.

