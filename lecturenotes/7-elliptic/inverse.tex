%
% inverse.tex -- XXX
%
% (c) 2019 Prof Dr Andreas Mueller
%
\section{Gibt es einen zu $\Delta$ inversen Operator?}
Um das Problem $\Delta u=f$ zu lösen, müsste man einfach einen
inversen Operator $\Delta^{-1}$ kennen, dann wäre die gesuchte
Lösung $u=\Delta^{-1}f$. Dies ist natürlich zu schön, um wahr
zu sein, aber die folgenden zwei Beispiele sollen illustrieren,
dass für elliptische Operatoren
mindestens etwas Ähnliches durchaus im Bereich des Möglichen
liegen könnte.

\subsection{Ableitung und Integration}
Die Gleichung in einer Dimension
\[
u''(x)=f(x)
\]
kann direkt mit Hilfe von Integrationen gelöst werden. Dazu muss
man nur eine Stammfunktion $F$ von $f$ finden, also $F' = f$ und
weiter eine Stammfunktion $u$ von $F$, dann gilt
$u''(x)=F'(x)=f(x)$. Trotzdem ist $u$ noch keine Lösung, denn 
sie erfüllt die Randbedingungen noch nicht.

\subsection{Lineare Gleichungen}
Man kann sich vorstellen, dass durch Diskretisation aus der 
Differentialgleichung $u''=f$ auf dem Interval $[a,b]$ ein lineares Gleichungssystem
$Au=f$ entsteht. Die Komponenten der Vektoren $u$ und $f$ sind dabei
Werte von $u$ und $f$ an ausgewählten Punkten. Und es ist nicht
abwegig, dass es eine inverse Matrix $G$ geben könnte, mit der man
die Lösung $u=Gf$ finden könnte.

Das Matrizenprodukt ist
\[
\sum_{j}g_{ij}f_j = u_i,
\]
und auch die $g_{ij}$ könnte man als Werte einer Funktion $G(x,y)$
betrachten. Die Summe ist dann nur der diskrete Fall eines Integrals,
man könnte also auf die Idee kommen, dass
die Lösung $u$ mit einer Formel der Form
\[
u(x)=\int_a^b G(x,\xi)f(\xi)\,d\xi
\]
gefunden werden kann. Auch dies wird noch nicht funktionieren,
weil wiederum die Randbedingung nicht berücksichtigt wurde.

\subsection{Randbedingungen}
Die Randbedingungen sollten ebenfalls linear in das Problem eingehen.
Im eindimensionalen Problem haben wir nur zwei Punkte auf dem Rand,
also müssten sie einen Beitrag der Form
\[
K(x,a) g(a) + K(x,b)g(b)
\]
beitragen. Besteht der Rand dagegen aus vielen Punkten, würde dies
wieder zu einer grossen Summe
\[
\sum_{i}K(x,a_i)g(a_i),
\]
wobei die Punkte $a_i$ auf dem Rand liegen. In der Grenze erwartet man
als für die Randwerte einen zweiten Beitrag
\[
\int_{\partial\Omega}K(x,\xi)g(\xi)\,d\xi
\]
in Form eines Kurven- oder Oberflächenintegrals über den Rand des
Gebietes.
Wir suchen daher eine allgmeine Lösung des Problems in der Form
\begin{equation}
u(x)=\int_{\Omega} G(x,\xi)f(\xi)\,d\xi + \int_{\partial\Omega} K(x,\xi)g(\xi)\,d\xi.
\label{greenformula}
\end{equation}

