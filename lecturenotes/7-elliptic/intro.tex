%
% intro.tex -- 
%
% (c) 2019 Prof Dr Andreas Mueller
%
\section{Introduction}
\rhead{Introduction}
In previous chapters, the differential equation were defined an
very specific domains, e.~g.~rectangles, disk, strips, half planes or
the whole of $\mathbb R^n$.
However, the applications demand that elliptic partial differential
equations can be solved on arbitrary domains.
This will no longer be possible manually or in closed for, but at
least we can decide whether the solution is uniquely determined.
Without this assurance, we can not expect a numerical method to
return a stable solution.

\subsection{Laplace equation}
To better understand the properties of solutions of elliptic linear
partial differetnial equations, we study the Laplace equation
on a domain $\Omega\subset\mathbb R^n$
\begin{align}
\Delta u&=f &&\text{in $\Omega$}
\label{elliptisch:laplaceequation}
\\
\intertext{with Dirichlet boundary conditions}
u&=g && \text{on $\partial\Omega$},
\label{dirichletrandbedingung}
\\
\intertext{Neumann boundary conditions}
\frac{\partial u}{\partial n}&=g && \text{auf $\partial\Omega$}
\\
\intertext{or mixed boundary conditions}
\alpha u + \beta\frac{\partial u}{\partial n}&=g && \text{on $\partial\Omega$.}
\label{elliptisch:gemischt}
\end{align}
A soluton of the is a twice continuously differentiable function
on $\Omega$ which can be continuously extended to $\bar\Omega$
and satisfies the boundary conditions on the boundary $\partial\Omega$.
For simplicity we consider only Dirichlet boundary conditions
\eqref{dirichletrandbedingung}.

From the general theory of linear partial differential equations
we derive that we have to proceed as follows to get a solution:
\begin{enumerate}
\item
First we need a particular solution $u_p$ of the Poisson equation
\eqref{elliptisch:laplaceequation}.
\item
Given $u_p$ the problem is reduced to find a solution
Mit $u_p$ reduziert sich das Problem darauf, eine Lösung
$u_r=u-u_p$ of the homogeneous equation $\Delta u_r=0$ which
satisfies the boundary condition
\[
u_r=g-u_p
\]
on $\partial\Omega$.
\item
Additional soluions of the problem differ from the solution
just found by a function $u_h$ which satisfies
$\Delta u_h=0$ in $\Omega$ and $u_h=0$ on $\partial\Omega$.
\end{enumerate}
We will show in this chapter that the solutions of such problems
are unique and that the only possible solution is $v=0$.
In addition we will present formulae for particular solutions as well
als for solutions of the homogeneous problem.

\subsection{Harmonic functions}
\index{harmonic functions}
Considering the importance of the solutions of the homogeneous problem
we give them a special name:

\begin{definition}
A twice constinuously differentiable function $u$ on the domain $\Omega$
is called harmonic if $\Delta u=0$.
\end{definition}

