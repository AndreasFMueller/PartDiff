%
% intro.tex -- XXX
%
% (c) 2019 Prof Dr Andreas Mueller
%
\section{Einführung}
\rhead{Einführung}
In den bisherigen Kapiteln waren die Differentialgleichungen auf sehr
speziellen Gebieten definiert gewesen, zum Beispiel Rechtecken, Kreissscheiben,
unendlichen Streifen, Halbebenen oder auf ganz $\mathbb R^n$. 
Die Anwendungen verlangen aber, dass elliptische partielle
Differentialgleichungen, auf fast beliebigen Gebieten gelöst werden können.
Insbesondere wollen wir erstehen, unter welchen Bedingungen eine
Lösung existiert und eindeutig bestimmt ist.

\subsection{Laplace-Gleichung}
\index{Laplace-Gleichung}
Als Beispiel für die Eigenschaften der Lösungen
%\marginpar{\tiny Formulierung des Beispielproblems}
elliptischer linearer partieller Differentialgleichungen
betrachten wir in diesem Kapitel die Laplace-Gleichung
auf einem Gebiet $\Omega\subset\mathbb R^n$
\begin{align}
\Delta u&=f &&\text{in $\Omega$}
\label{elliptisch:laplaceequation}
\\
\intertext{mit Dirichlet Randbedingungen}
u&=g && \text{auf $\partial\Omega$},
\label{dirichletrandbedingung}
\\
\intertext{Neumann Randbedingungen}
\frac{\partial u}{\partial n}&=g && \text{auf $\partial\Omega$}
\\
\intertext{oder der gemischten Randbedingungen}
\alpha u + \beta\frac{\partial u}{\partial n}&=g && \text{auf $\partial\Omega$.}
\label{elliptisch:gemischt}
\end{align}
Eine Lösung dieses Problems ist eine auf $\Omega$
zweimal stetig differenzierbare
Funktion $u$, welche stetig auf den Rand ausgedehnt werden kann,
und dort die Randbedingungen erfüllt. Zur Vereinfachung der
Diskussion betrachten wir in diesem Abschnitt nur die
Dirichlet-Randbedingungen (\ref{dirichletrandbedingung}).

Aus der allgemeinen Theorie linearer partieller Differentialgleichungen
%\marginpar{\tiny Allgemeine Lösungsstrategie}
kann man ableiten, dass wir für
die Lösung des gestellten Problems wie folgt vorgehen müssen
\begin{enumerate}
\item Zunächst brauchen wir eine partikuläre
Lösung $u_p$ der Laplace-Gleichung (\ref{elliptisch:laplaceequation}).
\item Mit $u_p$ reduziert sich das Problem darauf, eine Lösung
$u_r=u-u_p$ der homogenen Gleichung $\Delta u_r=0$
zu finden, welche die Randbedingung
\[
u_r=g-u_p
\]
erfüllt.
\item
Weitere Lösungen des Problems unterscheiden sich von dem
bereits gefundenen Problem nur durch eine Funktion $u_h$, welche
$\Delta u_h=0$ in $\Omega$ und $u_h=0$ auf $\partial\Omega$ erfüllt.
\end{enumerate}
Wir werden in diesem Kapitel zeigen, dass die Lösung diese Problems
eindeutig ist, dass also das einzige mögliche $v=0$ ist.
Ausserdem werden wir Formeln für partikuläre Lösungen wie auch für
das homogene Problem aufstellen.

\subsection{Harmonische Funktionen}
\index{harmonische Funktion}
Angesichts der Bedeutung der Lösungen des homogenen Problems
geben wir diesen einen speziellen Namen:

\begin{definition}
Eine zweimal stetig differenzierbare Funktion $u$ auf dem Gebiet $\Omega$
heisst harmonisch, wenn $\Delta u=0$.
\end{definition}

