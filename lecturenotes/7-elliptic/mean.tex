%
% mean.tex -- XXX
%
% (c) 2019 Prof Dr Andreas Mueller
%
\section{Mittelwerteigenschaft harmonischer Funktionen}
\rhead{Mittelwerteigenschaft}
Harmonische Funktionen haben eine Eigenschaft, die über das Maximumprinzip
für elliptische Operatoren hinausgeht.
Das Maximumprinzip sagt, dass die Werte der Lösung einer homogenen
elliptischen partiellen Differentialgleichung zwischen den
Extremwerten auf dem Rand liegen.
Die Mittelwerteigenschaft sagt darüber hinaus, dass die Funktionswerte
(geeignete) Mittelwerte der Randwerte sind.

\subsection{Mittelwerteigenschaft}
In diesem Abschnitt wollen wir illustrieren, dass
Werte von harmonischen Funktionen in einem Punkt
Mittelwerte der Funktionswerte auf einer Kugel um den gegebenen Punkt sind.

\begin{satz}[Mittelwerteigenschaft]
Sei $u$ eine harmonische Funktion im Gebiet $\Omega$ und $x_0\in\Omega$.
%\marginpar{\tiny Mittelwerteigenschaft harmonischer Funktionen}
Sei $r$ so klein, dass die Kugel mit Radius $r$ um den Punkt $x_0$
ebenfalls in $\Omega$ enthalten ist. Dann gilt
\[
u(0)=\frac1{\mu(S^{n-1}_r)}\int_{S^{n-1}_r}u(x)\,d\mu(x).
\]
\end{satz}
\begin{proof}[Beweis]
Wir können ohne Beschränkung der Allgemeinheit annehmen, dass $x_0=0$.
Wegen 
\[
\Delta u=\operatorname{div}\operatorname{grad}u
\]
folgt aus dem Satz von Gauss
\begin{align*}
0&=\int_{B_r^n}\Delta u\,d\mu(x)
\\
&=\int_{B_r^n}\operatorname{div}\operatorname{grad}u\,d\mu(x)
\\
&=\int_{S_r^{n-1}} \operatorname{grad}u\cdot dn
\end{align*}
Andererseits ist 
\begin{align*}
\frac{d}{dr}\frac{1}{\mu(S_r^{n-1})}\int_{S_r^{n-1}} u(x)\,d\mu(x)
&=
\frac1{\mu(S_1^{n-1})}\frac{d}{dr}\int_{S_1^{n-1}}u(xr)\,d\mu(x)
\\
&=
\frac1{\mu(S_1^{n-1})}\int_{S_1^{n-1}}\operatorname{grad}u(xr)\cdot x
\,d\mu(x)
\\
&=
\frac1{\mu(S_1^{n-1})}\int_{S_1^{n-1}}\operatorname{grad}u(xr)\cdot dn=0
\end{align*}
Der Mittelwert hängt also nicht vom Radius ab. Da aber wegen
der Stetigkeit von $u$ im Punkt $0$ auch
\[
\lim_{r\to 0}\frac1{\mu(S_r^{n-1})}\int_{S_r^{n-1}}u(x)\,d\mu(x)=u(0)
\]
folgt die Behauptung.
\end{proof}

Man kann die Mittelwerteigenschaft auch dafür verwenden, das Dirichlet-Problem
auf einem Ball zu lösen:

\begin{satz}[Poisson-Formel]
\index{Poisson-Formel}
Sei $g$ eine stetige Funktion auf dem Rand der $n$-dimensio\-nalen
%\marginpar{\tiny Poisson-Formel}
Einheitskugel, dann ist
\[
u(x)=\begin{cases}
\displaystyle \frac{1-|x|^2}{\mu(S^{n-1})}
\int_{S^{n-1}}\frac{g(\xi)}{|x-\xi|^n}\,d\mu(\xi)&\qquad |x|<1\\
g(x)&\qquad |x|=1
\end{cases}
\]
eine harmonische Funktion mit Randwerten $u_{|S_1^{n-1}}=g$.
\end{satz}

\subsection{Maximumprinzip und Mittelwerteigeschaft}
\rhead{Maximumprinzip}
\index{Maximumprinzip}
\begin{satz}[Maximumprinzip]Ist die Funktion $u$ auf dem Gebiet
%\marginpar{\tiny Maximumprinzip für harmonische Funktionen}
$\Omega$ harmonisch, und nimmt sie in einem inneren Punkt $x\in\Omega$
ein Maximum an, dann ist $u$ konstant.
\end{satz}

\begin{proof}[Beweis]
Sei $x_0\in\Omega$ der innere Punkt, in dem die Funktion $u$
ihr Maximum annimmt. Dann ist auch eine kleine Kugel mit Radius
$r$ in $\Omega$ enthalten, deren Oberfläche wir mit $K_r$
bezeichen. Falls $u$ nicht konstant ist, gibt
es eine Zahl $\varepsilon > 0$ und 
ein Teilgebiet $S\subset K_r$ nicht verschwindenden Volumnes $\mu(S)$,
auf welchem die Funktionswerte
mindestens $\varepsilon$ kleiner sind, also
\[
u(x)<u(x_0)-\varepsilon\quad\text{für $x\in S$}
\]
Der Funktionswert in $x_0$ ist aber der Mittelwert der Funktionswerte
über die Kugeloberfläche, also
\begin{align*}
u(x_0)
&=
\frac1{\mu(K_r)}\int_{K_r}u(x)\,d\mu(x)
\\
&=\frac1{\mu(K_r)}\int_Su(x)\,d\mu(x)+\frac1{\mu(K_r)}\int_{K_r\setminus S}u(x)\,d\mu(x)
\\
&\le \frac1{\mu(K_r)}\int_S u(x_0)\,d\mu(x)+\frac1{\mu(K_r)}\int_{K_r\setminus S}u(x_0)-\varepsilon\,d\mu(x)
\\
&\le \frac1{\mu(K_r)}u(x_0)\mu(S)+\frac1{\mu(K_r)}(u(x_0)-\varepsilon)\mu(K_r\setminus S)
\\
&=
u(x_0)-\varepsilon\frac{\mu(S)}{\mu(K_r)}<u(x_0)
\end{align*}
Ein solches Teilgebiet $S$ kann es daher nicht geben, der Funktionswert
von $u$ muss auf der ganzen Kugel identisch sein.
\end{proof}

Aus dem Maximum-Prinzip folgt auch, dass eine beliebige Funktion, die
%\marginpar{\tiny Funktionen mit Mittelwerteigenschaft sind harmonisch}
die Mittelwerteigenschaft hat, auch harmonisch ist. Ist $u$ ein
Funktion, die die Mittelwerteigenschaft hat, dann kann man mit der
Poisson-Formel eine Funktion $v$ innerhalb eines Balles konstruieren,
welche auf der Oberfläche des Balles mit $u$ übereinstimmt.
Die Differenz ist dann eine Funktion, die ebenfalls die Mittelwerteigenschaft
hat, aber auch auf dem Rand des Balles verschwindet. Da die Differenz
nach dem Maximumprinzip
im Inneren das Balles nicht grösser sein kann als auf dem Rand,
muss $u=v$ sein. Da $v$ harmonisch ist, ist auch $u$ harmonisch.

Die Analogie zum eindimensionalen Fall ist ebenfalls durchführbar.
%\marginpar{\tiny Maximumprinzip gilt auch im eindimensionalen Musterproblem}
Ist $u$ eine Funktion von einer Variablen mit $u''=0$, dann ist
$u$ linear, also $u(x)=ax+b$. Hat $u(x)$ ein Maximum im Inneren
des Definitionsbereichs, folgt $u'(x)=a=0$, also ist $u(x)=b$ konstant.

