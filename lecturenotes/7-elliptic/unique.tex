%
% unique.tex -- 
%
% (c) 2019 Prof Dr Andreas Mueller
%
\section{Uniqueness of a solution}
\rhead{Uniqueness of a solution}
We now study the question whether the solution of the Dirichlet problem
\eqref{elliptisch:laplaceequation} and \eqref{dirichletrandbedingung},
if it exist, is unique.

Let $u_1$ and $u_2$ be two solutions of the problem.
The difference satisfies
\[
\begin{aligned}
\Delta u&=\Delta u_1-\Delta u_2=f-f=0&&\text{in $\Omega$,}\\
u&=u_1-u_2=g-g=0&&\text{on $\partial\Omega$,}
\end{aligned}
\]
so $u$ is a solution of the homogeneous problem with homogeneous 
boundary conditions.

This observation holds true for any other linear partial differential
operator of the form
\[
L=\sum_{i,j}a_{ij}\partial_i\partial_j+\sum_i b_i\partial_i.
\]
Uniqueness of solutions is thus equivalent to uniqueness of solutions
of the homogeneous problem with homogeneous boundary conditions.
This follows from the following theorem:

\begin{satz}[Maximum principle for elliptic operators]
\label{maximumprinzip}
If $L$ is an elliptic differential operator on a connected and
bounded domain $\Omega$, and $u$ is a solution $Lu=0$, then $u$ takes
its maximum and minimum on the boundary of $\Omega$.
\end{satz}

The condition that the domain $\Omega$ needs to be bounded is essential.
The function $u(x,y)=x$ is harmonic, $\Delta x=0$, on the domain
$\Omega=\{(x,y)\,|\,x>0\}$, and it vanishes on the boundary, which is
the $y$-axis, where we have $u(0,y)=0$.
So this $u$ is a solution of the Poisson problem with homogeneous
Dirichlet boundary conditions,
but it does not take a maximum anywhere.

The connectedness of the domain is essential too.
Define the function
\[
u(x)=n\qquad\text{für $2^{-n} < |x| < \frac{3}{2}2^{-n}$},
\]
then $\Delta u=0$, it is constant on rings around the origin and hence
$u$ is harmonic.
The values of $u$ are not bounded as there are infinitely many disjoint
rings.
In particular, there is no maximum.

\begin{proof}[Proof idea]
We prove the theorem with the help of a contradiction.
We assume that $u$ takes a maximum in some interior point $x$
in $\Omega$.

We may assume that the matrix $a_{ij}$ is diagonal and that, as $L$
is elliptic, $a_{ii}=\lambda_i$ are all positive.
As $u$ has a maximum in $x$, all first order derivatives vanish
in $x$ and at least one of the second derivatives
must be negative:
\[
\frac{\partial^2u}{\partial x_i^2}
\le 0
\quad
\Rightarrow
\quad
\sum_{i}\lambda_i \frac{\partial^2u}{\partial x_i^2} \le 0
\quad
\Rightarrow
\quad
Lu<0
\]
This contradiction shows that there cannot be a maximum in the
interior of the domain.

However, this is unfortunately not a proof.
There is no reason why not all second derivatives of $\partial^2_{x_i} u$ 
could be zero.
We give a complete proof at the end of this section.
\end{proof}

\begin{satz}
If $L$ is an elliptic differential operator on a connected and
bounded domain $\Omega$, then there is at most one solution of
$Lu=f$ 
with boundary conditions
$u_{|\partial\Omega}=g$.
\end{satz}

\begin{proof}
Let $u_1$ und $u_2$ be solutions, then $u=u_1-u_2$ is a solution of
$Lu=0$ with boundary conditions $u_{|\partial\Omega}=0$.
According to theorem \ref{maximumprinzip}, $u$ takes its maximum and
minimum on the boundary.
Since $u$ vanishes on the boundary, the maximum as well as the minimum
are both $0$, so $0\le u\le 0$ in $\Omega$.
Consequently $u=0$ or $u_1=u-2$, it is therefore impossible that
there are two different solutions.
\end{proof}

{\small
\subsubsection{A proof of the maximum principle}
In the following we give a complete proof promised for the maximum principle.
The essential point of the proof idea was that $Lu=0$ but in an internal
maximum we would conclude $Lu<0$.
However, the method we employed was only really capable of concluding that
$Lu\le 0$.
Our argument so far only proofs the following statement:
\begin{satz}
\label{maximumgt}
If $L$ is an elliptic differential operator on a connected and
bounded domain $\Omega$ and $u$ a function with the property $Lu>0$, then
$u$ takes a maximum on the boundary.
\end{satz}

This allows us to conclude:
\begin{satz}
\label{semiharmonicth}
If $L$ is an elliptic differential operator on a conneted and
bounded domain $\Omega$ and $u$ is a function with $Lu\ge  0$,
then $u$ takes its maximum on the boundary.
\end{satz}

\begin{proof}
As in the proof sketch, we may assume that in the point $x$ the 
coefficient matrix $(a_{ij})$ is diagonal with all diagonal elements
positive.
Let $u_{\varepsilon}$ be the function
$u_{\varepsilon}=u-\varepsilon e^{\lambda x_1}$.
We get
\begin{align*}
Lu_{\varepsilon}&=Lu-\varepsilon L(e^{\lambda x_1})\\
&=Lu-\varepsilon(a_{11}\lambda^2+b_1\lambda)e^{\lambda x_1}
\end{align*}
Because $a_{11}>0$,
the quadratic polynomial $a_{11}\lambda^2+b_1\lambda$ will be positive
for sufficnently large $\lambda$.
Thus
\[
Lu_{\varepsilon}= Lu-\varepsilon(a_{11}\lambda^2+b_1\lambda)e^{\lambda x_1}<0
\]
for every $\varepsilon$.
According to theorem \ref{maximumgt},
the maximum of $u_\varepsilon$ is reached on the boundary.

In the limit $\varepsilon\to 0$ we have $u^\varepsilon\to u$, so
$u$ too must reach a maximum on the boundary.
\end{proof}

If $u$ satisfies $Lu \le 0$, then $L(-u)\ge 0$, so
theorem~\ref{semiharmonicth} applies to $-u$ and lets us conclude, that
$-u$ has its maximum on the boundary, or equivalently, that $u$ has its
minimum on the boundary.
The theorem below follows.

\begin{satz}
If $L$ is an elliptic differential operator on a connected and
bounded domain $\Omega$ and $u$ a function with $Lu\le 0$, then
$u$ takes a minimum on the boundary.
\end{satz}

For solutions of $Lu=0$ the theorem \ref{maximumprinzip} follows.
}

