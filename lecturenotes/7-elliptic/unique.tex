%
% unique.tex -- XXX
%
% (c) 2019 Prof Dr Andreas Mueller
%
\section{Eindeutigkeit der Lösung}
Wir untersuchen jetzt die Frage, ob die Lösung des 
Dirichlet-Problems (\ref{elliptisch:laplaceequation}) und
(\ref{dirichletrandbedingung}), sollte sie existieren, eindeutig ist.
Seien also $u_1$ und $u_2$ zwei Lösungen des Problems. Die Differenz
$u=u_1-u_2$ erfüllt dann
\[
\begin{aligned}
\Delta u&=\Delta u_1-\Delta u_2=f-f=0&&\text{in $\Omega$,}\\
u&=u_1-u_2=g-g=0&&\text{auf $\partial\Omega$.}
\end{aligned}
\]
$u$ ist also eine Lösung des homogenen Problems mit homogenen
Randbedingungen.

Diese Beobachtung trifft natürlich auch zu für jeden anderen
Differentialoperator der Form
\[
L=\sum_{i,j}a_{ij}\partial_i\partial_j+\sum_i b_i\partial_i.
\]
Um die Eindeutigkeit der Lösung zu verstehen, muss man also verstehen,
ob das homogene Problem mit homogenen Randbedingungen eine
eindeutige Lösung hat. Dies wiederum folgt aus dem folgenden Satz.

\begin{satz}[Maximum-Prinzip für elliptische Operatoren]
\label{maximumprinzip}
Ist $L$ ein elliptischer Differentialoperator auf einem zusammenhängenden
und beschränkten Gebiet $\Omega$,
und $u$ eine Lösung von $Lu=0$,
dann nimmt $u$ sein Maximum und Minimum auf dem Rand an.
\end{satz}

Die Bedingung, dass das Gebiet $\Omega$ beschränkt sein muss, ist
wesentlich.
Die Funktion $u(x,y)=x$ ist harmonisch, $\Delta x=0$, und auf
dem Rand des Gebietes $\Omega=\{(x,y)\,|\,x>0\}$, nämlich
der $y$-Achse nimmt, gilt $u(0,y)=0$.
Es gibt also mehr als eine Lösung des Dirichletproblems mit
homogenen Randbedingungen für dieses Gebiet.

Auch der Zusammenhang des Gebietes ist wesentlich.  Definieren wir
die Funktion 
\[
u(x)=n\qquad\text{für $2^{-n} < x < \frac{3}{2}2^{-n}$},
\]
dann ist $\Delta u=0$, da aber die Werte von $u$ unbeschränkt sind,
wird auch kein Maximum angenommen.

\begin{proof}[Beweisidee]
Wir führen den Beweis mit Hilfe eines Widerspruchs. Wir nehmen
an, $u$ nehme in einem inneren Punkt $x$ des Gebietes $\Omega$ ein
Maximum an.

Wir dürfen annehmen, dass die Matrix $a_{ij}$ diagonal ist,
und dass,
da $L$ elliptisch ist, die Diagonalelement $a_{ii}=\lambda_i$
alle positiv sind. Da das Maximum in $x$ angenommen wird, muss
mindestens eine zweite Ableitungen negativ sein:
\[
\frac{\partial^2u}{\partial x_i^2}
\le 0
\quad
\Rightarrow
\quad
\sum_{i}\lambda_i \frac{\partial^2u}{\partial x_i^2} \le 0
\quad
\Rightarrow
\quad
Lu<0
\]
Der Widerspruch zeigt, dass es kein inneres Maximum geben kann.

Dies ist jedoch kein Beweis, weil es keinen Grund gibt, warum
nicht alle zweiten Ableitungen von $u$ im Punkt $x$ verschwinden
könnten. Wir geben einen vollständigen Beweis am Ende dieses
Abschnittes.
\end{proof}

\begin{satz}
Es gibt höchstens eine Lösung der Gleichung
$Lu=f$ 
mit Randbedingung
$u_{|\partial\Omega}=g$.
\end{satz}

\begin{proof}[Beweis]
Seien $u_1$ und $u_2$ Lösungen, dann ist $u=u_1-u_2$ eine Lösung 
von $Lu=0$ mit Randbedingungen $u_{|\partial\Omega}=0$. Nach Satz
\ref{maximumprinzip} nimmt $u$ sein Maximum und sein Minimum auf dem Rand an,
es gilt
also $0\le u\le 0$ in ganz $\Omega$.
Daher ist $u=0$ oder $u_1=u-2$,
zwei verschiedene Lösungen kann es also nicht geben.
\end{proof}

{\small
\subsubsection{Ein Beweis des Maximum-Prinzips}
Im Folgenden wollen wir wie versprochen einen vollständigen Beweis für das
Maximumprinzip angeben. Der wesentliche Punkt in der Beweisidee
war, dass $Lu=0$, wir aber bei einem Maximum (inkorrekterweise)
auf $Lu<0$ schliessen wollten. Dabei erlaubt unsere Method nur zu
schliessen, dass $Lu\le 0$.
Unser Argument beweist also nur die folgende Aussage:
\begin{satz}
\label{maximumgt}
Ist $L$ ein elliptischer Differentialoperator auf einem zusammenhängenden
und beschränkten Gebiet $\Omega$,
und $u$ eine Funktion mit der Eigenschaft $Lu>0$,
dann nimmt $u$ sein Maximum auf dem Rand an.
\end{satz}

Daraus kann man jetzt aber auch folgern:
\begin{satz}
Ist $L$ ein elliptischer Differentialoperator auf einem zusammenhängenden
und beschränkten Gebiet $\Omega$,
und $u$ eine Funktion mit der Eigenschaft $Lu\ge0$,
dann nimmt $u$ sein Maximum auf dem Rand an.
\end{satz}

\begin{proof}[Beweis]
Wir dürfen wie in der bereits oben skizzierten Beweisidee annehmen,
dass im betrachteten Punkt $x$ die Koeffizientenmatrix $(a_{ij})$
diagonal ist und alle $a_{ii}$ positiv sind.

Wir betrachten die Funktion $u_{\varepsilon}=u-\varepsilon e^{\lambda x_1}$.
Es gilt
\begin{align*}
Lu_{\varepsilon}&=Lu-\varepsilon L(e^{\lambda x_1})\\
&=Lu-\varepsilon(a_{11}\lambda^2+b_1\lambda)e^{\lambda x_1}
\end{align*}
Da $a_{11}>0$ gilt, wird das quadratische Polynom $a_{11}\lambda^2+b_1\lambda$
für genügend grosses $\lambda$ positiv, und damit 
\[
Lu_{\varepsilon}= Lu-\varepsilon(a_{11}\lambda^2+b_1\lambda)e^{\lambda x_1}<0
\]
für jedes beliebige $\varepsilon$. Nach Satz \ref{maximumgt} wird das Maximum
von $u_\varepsilon$ also auf dem Rand angenommen.
Im Grenzübergang $\varepsilon\to 0$ ist $u^\varepsilon\to u$, also
muss auch $u$ das Maximum auf dem Rand annehmen.
\end{proof}

Setzt man $-u$ in den eben bewiesenen Satz, folgt das Minimumprinzip:
\index{Minimumprinzip}
\begin{satz}
Ist $L$ ein elliptischer Differentialoperator auf einem zusammenhängenden
und beschränkten Gebiet $\Omega$,
und $u$ eine Funktion mit der Eigenschaft $Lu\le0$,
dann nimmt $u$ sein Maximum auf dem Rand an.
\end{satz}

Für Lösungen von $Lu=0$ folgt dann der Satz \ref{maximumprinzip}.
}

