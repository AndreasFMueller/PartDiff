%
% elliptic.tex -- 
%
% (c) 2019 Prof Dr Andreas Mueller
%
\chapter{Elliptic partial differential equations
\label{chapter:elliptic}}
In this chapter we study the solutions of elliptic partial differential
equations and discover the maximum-minimum principle as well as the
the mean value property of harmonic functions.
At the end of the chapter we describe Green's function, which allows
to give a formula for the solution of a boundary value problem.

%
% intro.tex -- XXX
%
% (c) 2019 Prof Dr Andreas Mueller
%
\section{Einführung}
\rhead{Einführung}
In den bisherigen Kapiteln waren die Differentialgleichungen auf sehr
speziellen Gebieten definiert gewesen, zum Beispiel Rechtecken, Kreissscheiben,
unendlichen Streifen, Halbebenen oder auf ganz $\mathbb R^n$. 
Die Anwendungen verlangen aber, dass elliptische partielle
Differentialgleichungen, auf fast beliebigen Gebieten gelöst werden können.
Insbesondere wollen wir erstehen, unter welchen Bedingungen eine
Lösung existiert und eindeutig bestimmt ist.

\subsection{Laplace-Gleichung}
\index{Laplace-Gleichung}
Als Beispiel für die Eigenschaften der Lösungen
%\marginpar{\tiny Formulierung des Beispielproblems}
elliptischer linearer partieller Differentialgleichungen
betrachten wir in diesem Kapitel die Laplace-Gleichung
auf einem Gebiet $\Omega\subset\mathbb R^n$
\begin{align}
\Delta u&=f &&\text{in $\Omega$}
\label{elliptisch:laplaceequation}
\\
\intertext{mit Dirichlet Randbedingungen}
u&=g && \text{auf $\partial\Omega$},
\label{dirichletrandbedingung}
\\
\intertext{Neumann Randbedingungen}
\frac{\partial u}{\partial n}&=g && \text{auf $\partial\Omega$}
\\
\intertext{oder der gemischten Randbedingungen}
\alpha u + \beta\frac{\partial u}{\partial n}&=g && \text{auf $\partial\Omega$.}
\label{elliptisch:gemischt}
\end{align}
Eine Lösung dieses Problems ist eine auf $\Omega$
zweimal stetig differenzierbare
Funktion $u$, welche stetig auf den Rand ausgedehnt werden kann,
und dort die Randbedingungen erfüllt. Zur Vereinfachung der
Diskussion betrachten wir in diesem Abschnitt nur die
Dirichlet-Randbedingungen (\ref{dirichletrandbedingung}).

Aus der allgemeinen Theorie linearer partieller Differentialgleichungen
%\marginpar{\tiny Allgemeine Lösungsstrategie}
kann man ableiten, dass wir für
die Lösung des gestellten Problems wie folgt vorgehen müssen
\begin{enumerate}
\item Zunächst brauchen wir eine partikuläre
Lösung $u_p$ der Laplace-Gleichung (\ref{elliptisch:laplaceequation}).
\item Mit $u_p$ reduziert sich das Problem darauf, eine Lösung
$u_r=u-u_p$ der homogenen Gleichung $\Delta u_r=0$
zu finden, welche die Randbedingung
\[
u_r=g-u_p
\]
erfüllt.
\item
Weitere Lösungen des Problems unterscheiden sich von dem
bereits gefundenen Problem nur durch eine Funktion $u_h$, welche
$\Delta u_h=0$ in $\Omega$ und $u_h=0$ auf $\partial\Omega$ erfüllt.
\end{enumerate}
Wir werden in diesem Kapitel zeigen, dass die Lösung diese Problems
eindeutig ist, dass also das einzige mögliche $v=0$ ist.
Ausserdem werden wir Formeln für partikuläre Lösungen wie auch für
das homogene Problem aufstellen.

\subsection{Harmonische Funktionen}
\index{harmonische Funktion}
Angesichts der Bedeutung der Lösungen des homogenen Problems
geben wir diesen einen speziellen Namen:

\begin{definition}
Eine zweimal stetig differenzierbare Funktion $u$ auf dem Gebiet $\Omega$
heisst harmonisch, wenn $\Delta u=0$.
\end{definition}


%
% unique.tex -- 
%
% (c) 2019 Prof Dr Andreas Mueller
%
\section{Uniqueness of a solution}
We now study the question whether the solution of the Dirichlet problem
\eqref{elliptisch:laplaceequation} and \eqref{dirichletrandbedingung},
if they exist, is unique.

Let $u_1$ and $u_2$ be two solutions of the problem.
The difference satisfies
\[
\begin{aligned}
\Delta u&=\Delta u_1-\Delta u_2=f-f=0&&\text{in $\Omega$,}\\
u&=u_1-u_2=g-g=0&&\text{on $\partial\Omega$.}
\end{aligned}
\]
$u$ is a solution of the homogeneous problem with homogeneous 
boundary conditions.

This observation holds true for any other linear partial differential
operator of the form
\[
L=\sum_{i,j}a_{ij}\partial_i\partial_j+\sum_i b_i\partial_i.
\]
Uniqueness of solutions is thus equivalent to uniqueness of solutions
of the homogeneous problem with homogeneous boundary conditions.
This follows from the following theorem:
eindeutige Lösung hat. Dies wiederum folgt aus dem folgenden Satz.

\begin{satz}[Maximum principle for elliptic operators]
\label{maximumprinzip}
If $L$ is an elliptic differential operator on a connected and
bounded domain $\Omega$, and $u$ is a solution $Lu=0$, then $u$ takes
its maximum and minimum on the boundary of $\Omega$.
\end{satz}

The condition that the domain $\Omega$ needs to be bounded is essential.
The function $u(x,y)=x$ is harmonic, $\Delta x=0$, on the domain
$\Omega=\{(x,y)\,|\,x>0\}$, and it vanishes on the boundary, which is
the $y$-axis, where we have $u(0,y)=0$.
So this $u$ is a solution of the Poisson problem with homogeneous
Dirichlet boundary conditions with homogeneous boundary conditions, 
but it does not take a maximum anywhere.

The connectedness of the domain es essential too.
Define the function
\[
u(x)=n\qquad\text{für $2^{-n} < |x| < \frac{3}{2}2^{-n}$},
\]
then $\Delta u=0$, it is constant on rings around the origin and hence
$u$ is harmonic.
The values of $u$ are not bounded as there are infinitely many disjoint
rings.
In particular, there is no maximum.

\begin{proof}[Proof idea]
We prove the theorem with the help of a constradiction.
We assume that $u$ takes a maximum in some interior point $x$
in $\Omega$.

We may assume that the matrix $a_{ij}$ is diagonal that, as $L$
is elliptic, that $a_{ii}=\lambda_i$ are all positive.
As $u$ has a maximum in $x$, all first order derivatives vanish
in $x$ and at least one of the second derivatives
must be nagative:
\[
\frac{\partial^2u}{\partial x_i^2}
\le 0
\quad
\Rightarrow
\quad
\sum_{i}\lambda_i \frac{\partial^2u}{\partial x_i^2} \le 0
\quad
\Rightarrow
\quad
Lu<0
\]
This contradiction shows that there cannot be a maximum in the
interior of the domain.

However, this is unfortunately not a proof.
There is no reason why not all second derivatives of $\partial^2_{x_i} u$ 
could be zero.
We give a complete proof at the end of this section.
\end{proof}

\begin{satz}
There is at most one solution of
$Lu=f$ 
with boundary conditions
$u_{|\partial\Omega}=g$.
\end{satz}

\begin{proof}
Let $u_1$ und $u_2$ be solutions, then $u=u_1-u_2$ is a solution of
$Lu=0$ with boundary conditions $u_{|\partial\Omega}=0$.
According to theorem \ref{maximumprinzip} $u$ takes its maximum and
minimum on the boundary, so $0\le u\le 0$ in $\Omega$.
Consequently $u=0$ or $u_1=u-2$, it is therefore impossible that
there are two different solutions.
\end{proof}

{\small
\subsubsection{A proof of the maximum principle}
In the following we give a complete proof promised for the maximum principle.
The essential point of the proof idea was that $Lu=0$ but in an internal
maximum we could conclude $Lu<0$.
But the method we employed was only real capable of concluding that
$Lu\le 0$.
Our argument so far only proofs the following statement:
\begin{satz}
\label{maximumgt}
If $L$ is an elliptic differential operator on a connected and
bounded domain $\Omega$ and $u$ a function with the property $Lu>0$, then
$u$ takes a maximum on the boundary.
\end{satz}

This allows us to conclude:
\begin{satz}
If $L$ is an elliptic differential operator on a conneted and
bounded domain $\Omega$ and $u$ is a function with $Lu\ge > 0$,
then $u$ takes its maximum on the boundary.
\end{satz}

\begin{proof}
As in the proof sketch, we may assume that in the point $x$ the 
coefficient matrix $(a_{ij})$ is diagonal with all diagonal elements
positive.
Let $u_{\varepsilon}$ be the function
$u_{\varepsilon}=u-\varepsilon e^{\lambda x_1}$.
We get
\begin{align*}
Lu_{\varepsilon}&=Lu-\varepsilon L(e^{\lambda x_1})\\
&=Lu-\varepsilon(a_{11}\lambda^2+b_1\lambda)e^{\lambda x_1}
\end{align*}
Because $a_{11}>0$,
the quadratic polynomial $a_{11}\lambda^2+b_1\lambda$ will be positive
for sufficnently large $\lambda$.
Thus
\[
Lu_{\varepsilon}= Lu-\varepsilon(a_{11}\lambda^2+b_1\lambda)e^{\lambda x_1}<0
\]
for every $\varepsilon$.
According to theorem \ref{maximumgt},
the maximum of $u_\varepsilon$ is reached on the boundary.

In the limit $\varepsilon\to 0$ we have $u^\varepsilon\to u$, so
$u$ too must reach a maximum on the boundary.
\end{proof}

Applying this theorem to the function $-u$, the minimum principle follows:
\begin{satz}
If $L$ is an elliptic differential operator on a connected and
bounded domain $\Omega$ and $u$ a function with $Lu\le 0$, then
$u$ takes a minimum on the boundary.
\end{satz}

For solutions of $Lu=0$ the theorem \ref{maximumprinzip} follows.
}


%
% inverse.tex -- 
%
% (c) 2019 Prof Dr Andreas Mueller
%
\section{Does there exist an operator inverse to $\Delta$?}
\rhead{Does there exist $\Delta^{-1}$?}
We want to solve the problem $\Delta u=f$.
If we had an inverse operator $\Delta^{-1}$, we could use it to
find the solution $u=\Delta^{-1}f$.
This dream is too good to be true, but the examples below 
illustrate that for elliptic operators there is at least a partial
realization of it.

\subsection{Derivative and Integration}
The equation 
\[
u''(x)=f(x)
\]
in one dimension can be solved by simple integration.
To this effect we need to find an antiderivative $F$ of $f$, i.~e.~$F'=f$
and then an antiderivative $u$ of $F$.
In spite of $u''(x)=F'(x)=f(x)$, $u$ need no be a solution because
it does not satisfy the boundary conditions.

\subsection{Linear equations}
By discretization the differential equation $u''=f$ on the interval
$[a,b]$ can be turned into a linear system of equations of
the form $Au=f$.
The components of $u$ and $f$ are values of the functions of $u$ and $f$
in selected points of the domain.
And since we have seen that the solution is unique for elliptic operators,
there should be an inverse matrix $G$ that can be used to find the
solution in the form $u=Gf$.

The matrix product is
\[
\sum_{j}g_{ij}f_j = u_i,
\]
and the $g_{ij}$ can also be considred to be values of a function $G(x,y)$
defined in particular points of $\Omega\times\Omega$.

If the points used for discretization come closer together, the sum
can ultimately better be approximated by an integral of the form
\[
u(x)=\int_a^b G(x,\xi)f(\xi)\,d\xi.
\]
So we are left with the question whether there actually exists such
a function $G(x,\xi)$.
Still, there is no guarantee that this will be a solution, as
the boundary conditions may not be satisfied.

\subsection{Boundary conditions}
The boundary conditions also enter linearly into the problem.
In the one dimensional problem we have two points on the boundary
so we expect a contribution of the form
\[
K(x,a) g(a) + K(x,b)g(b),
\]
where $K(x,\xi)$ is another function defined on $x\in [a,b]$ and
$\xi \in\{a,b\}=\partial[a,b]$.
If the boundary is more complicated and consists of multiple points,
the contribution generalizes to a sum
\[
\sum_{i}K(x,a_i)g(a_i),
\]
where all the points $a_i$ are on the boundary.
Again, with the boundary becoming a curve or surface, the
contribution of the boundary values must be computed with
the help of an integral
\[
\int_{\partial\Omega}K(x,\xi)g(\xi)\,d\xi
\]
over the boundary of the domain.
So we expect a solution in the form
\begin{equation}
u(x)=\int_{\Omega} G(x,\xi)f(\xi)\,d\xi + \int_{\partial\Omega} K(x,\xi)g(\xi)\,d\xi.
\label{greenformula}
\end{equation}
The goal of the subsequent sections is to show that in some cases,
such functions actually exists.


%
% 1dim.tex -- XXX
%
% (c) 2019 Prof Dr Andreas Mueller
%
\section{Das eindimensionale Problem}
\rhead{Der eindimensionale Fall}
Zum Beweis, dass das im vorangegangenen Abschnitt angedeutete Programm 
tatsächlich durchführbar ist,
betrachten wird das eindimensionale Problem
\[
u''(x)=f(x)
\]
mit der Randbedingung
\[
u(0)=a,\quad u(1)=b.
\]
Ziel ist, die Funktion $G$ zu finden.

Die Lösung dieses Problems wird in der Theorie der gewöhnlichen
%\marginpar{\tiny Lösung des gewöhnlichen Randwertproblems als Leitlinie
%für den mehrdimensionalen Fall}
Differential\-gleichungen behandelt, wir wollen hier die allgemeine
Lösung jedoch so darstellen, dass sie als Leitlinie für die
Entwicklung einer Lösung für das mehrdimensionale Problem dienen
kann.
\subsection{Partikuläre Lösung}
Ein Lösung der Gleichung kann sofort gefunden werden, die Stammfunktion
liefert die Lösung
\begin{align*}
u_p'(x)&=\int_0^xf(\xi)\,d\xi,
\\
u_p(x)&=\int_0^xu_p'(\eta)\,d\eta=\int_0^x\int_0^\eta f(\xi)\,d\xi\,d\eta
\end{align*}
diese erfüllt aber die Randbedingung nicht.
Die gesuchte Lösung muss aber eine Summe der partikulären Lösung 
und einer Lösung der homogenen Gleichung
\[
u_p''(x) + u_h''(x)=f(x)\quad\Rightarrow\quad u_h''(x)=0
\]
sein, mit der Randbedingungen
\begin{align*}
u_h(0)&=a-u_p(0)=a\\
u_h(1)&=b-u_p(1)=b-\int_0^1\int_0^\eta f(\xi)\,d\xi\,d\eta
\end{align*}

\subsection{Das homogene Problem}
Das homogene Problem
\[
u''(x)=0
\]
kann mit den Standardverfahren gelöst werden. Die allgemeine
Lösung hat die Form $Ax+B$, die Konstanten müssen so gewählt werden,
dass die Randbedingungen erfüllt sind, also
\begin{align*}
A\cdot 0+B&=a\qquad\Rightarrow&B&=a\\
A\cdot 1+B&=b\qquad\Rightarrow&A&=b-a
\end{align*}
also $u(x)=(b-a)x+a=(1-x)a+xb$.

\subsection{Allgemeine Lösung}
Im vorliegenden Fall sollte $u_h$ den rechten Randwert $b-\int_0^1\int_0^\eta f(\xi)\,d\xi\,d\eta$
annehmen, also
\[
u_r(x)=(1-x)a+x\left(b-\int_0^1\int_0^\eta f(\xi)\,d\xi\,d\eta\right).
\]
Damit wird die vollständige Lösung
\begin{align}
u(x)&=u_p(x)+u_r(x)\notag
\\
&=\int_0^x\int_0^\eta f(\xi)\,d\xi\,d\eta+(1-x)a+x\left(b-\int_0^1\int_0^\eta f(\xi)\,d\xi\,d\eta\right)\notag
\\
&=
(1-x)a+xb
+\int_0^x\int_0^\eta f(\xi)\,d\xi\,d\eta
-x\int_0^1\int_0^\eta f(\xi)\,d\xi\,d\eta\notag
\\
&=
(1-x)a+xb
+\int_0^1\vartheta(x-\eta)\int_0^1 \vartheta(\eta - \xi)f(\xi)\,d\xi\,d\eta
-x\int_0^1\int_0^1\vartheta(\eta-\xi) f(\xi)\,d\xi\,d\eta\notag
\\
&=
(1-x)a+xb
+\int_0^1\int_0^1
(\vartheta(x-\eta)-x)\vartheta(\eta -\xi)
f(\xi)\,d\xi\,d\eta
\notag
\\
&=
(1-x)a+xb+\int_0^1\int_0^1
(\vartheta(x-\eta)-x)\vartheta(\eta -\xi)
\,d\eta\,
f(\xi)\,d\xi
\label{1dimgreen}
\end{align}
Darin ist $\vartheta(t)$ die Stufenfunktion
\[
\vartheta(t)=\begin{cases}
0&\qquad t<0\\
1&\qquad t\ge 0
\end{cases}
\]
Die Funktion $\vartheta(x-\xi)$ verschwindet, sobald $\xi$ grösser wird
als $x$, da dann $x-\xi<0$ gilt.

Das innere Integral in (\ref{1dimgreen}) kann vollständig berechnet
werden:
\begin{align*}
\int_0^1(\vartheta(x-\eta)-x)\vartheta(\eta-\xi)\,d\eta
&=
\int_\xi^1\vartheta(x-\eta)-x\,d\eta
\\
&=
\int_\xi^1\vartheta(x-\eta)\,d\eta-\int_\xi^1x\,d\eta
\\
&=
\int_\xi^1\vartheta(x-\eta)\,d\eta-(1-\xi)x
\end{align*}
Das erste Integral muss mit einer Fallunterscheidung berechnet werden:
\begin{align*}
\int_\xi^1\vartheta(x-\eta)\,d\eta
&=
\begin{cases}
\int_\xi^x\,d\eta=x-\xi&\qquad\eta>\xi\\
0&\qquad x\le \xi
\end{cases}
\\
&=(x-\xi)\vartheta(x-\xi)
\end{align*}
Insgesamt finden wir also für das innere
Integral die Funktion
\[
G(x,\xi)=(x-\xi)\vartheta(x-\xi)-x(1-\xi).
\]
Man kann diese Funktion auch mit Hilfe der Betragsfunktion
schreiben.
Die Funktion $x\mapsto h(x,\xi)$ ist an der Stelle $\xi$ nicht 
differenzierbar, links und rechts davon ist $h$ jedoch linear
\[
G(x,\xi)=\begin{cases}
(x-\xi)-x(1-\xi)=\xi(x-1)&\qquad x>\xi\\
x(\xi-1)&\qquad x<\xi
\end{cases}
\]
Also sollte in der Form $a|x-\xi|+bx+c$ geschrieben
werden können, nach etwas Rechnung findet man.
\[
G(x,\xi)={\textstyle \frac12}|x-\xi|-({\textstyle \frac12}-\xi)x-{\textstyle\frac12}\xi.
\]
Der zweite und dritte Term auf der rechten Seite bilden eine lineare
Funktion, die zur zweiten Ableitung nichts beiträgt. Die Ableitung
des ersten Teils ist eine Stufenfunktion mit Stufenhöhe $1$:
\[
\frac{\partial}{\partial x}({\textstyle \frac12}|x-\xi|)
=
\vartheta(x-\xi)
-
{\textstyle\frac12},
\]
und die Ableitung davon ist eine $\delta$-Funktion
\[
\frac{\partial^2}{\partial x^2}({\textstyle \frac12}|x-\xi|)
=
\delta(x-\xi).
\]
Der wesentliche Teil ist also die Funktion
\begin{equation}
\sigma(x,\xi)=\frac12|x-\xi|.\label{n1sigma}
\end{equation}

\subsection{Greensche Funktion}
\begin{figure}
\begin{center}
\includegraphics[width=\hsize]{../common/3d/green.jpg}
\end{center}
\caption{Darstellung der Greenschen Funktion $G(x,\xi)$
für das Problem $u''=f$ auf
dem Interval $[0,1]$ als Fläche über dem Quadrat $(x,\xi)\in[0,1]^2$.
Für jeden Wert von $\xi$ ist die partielle Funktion $x\mapsto G(x,\xi)$
eine Lösung der Differentialgleichung $u''=\delta_\xi$ zu den Randbedingungen
$u(0)=u(1)=0$.
\label{elliptisch:green3dflaeche}}
\end{figure}
\begin{figure}
\begin{center}
\includegraphics{../common/images/green-1.pdf}
\end{center}
\caption{Partielle Funktionen $x\mapsto G(x,\xi)$ der Greensche Funktion
des Problems $u''=f$ mit Randbedingungen $u(0)=u(1)=0$ für
verschiedene Werte von $\xi$.
\label{elliptisch:green1schar}}
\end{figure}

Die bis jetzt gefundenen Formeln für die Lösung $u(x)$
in Form eines Doppelintegrals haben noch
nicht die gewünschte Form eines einfachen Integrals.
Dies kann mit der Funktion $h$ korrigiert werden.
Die Funktion $x\mapsto h(x,\xi)=(x-\xi)\vartheta(x-\xi)$ hat die Randwerte
$0$ und $1-\xi$, also hat die Funktion
\[
G(x,\xi)
=
(x-\xi)\vartheta(x-\xi)-x(1-\xi)
=\begin{cases}
(x-\xi)+x(\xi-1)&\qquad x\le \xi \\
x(\xi-1)&\qquad x<\xi
\end{cases}
\]
die Randwerte $0$.
$G$ erfüllt
\[
\frac{\partial^2}{\partial x^2}G(x,\xi)=\delta(x-\xi)
\]
und
\[
G(0,\xi)=G(1,\xi)=0.
\]
Eine Lösung der Differentialgleichung lässt sich damit
als 
\[
u(x)=\int_0^1G(x,\xi)f(\xi)\,d\xi+a(1-x)+bx
\]
finden.

Der eindimensionale Fall zeigt also, dass man man zu dem Operator $D^2u=f$
mit der Funktion $G$ eine Inverse konstruieren kann.

Die Abbildungen \ref{elliptisch:green3dflaeche} und
\ref{elliptisch:green1schar} zeigen die Greensche Funktion $G(x,\xi)$.
Die partiellen Funktionen $x\mapsto G(x,\xi)$ sind jeweils Lösungen
des Problems $u''=\delta_\xi$ mit Randwerten $u(0)=u(1)=0$.
In den Abbildungen sind die partiellen Funktionen für eine Anzahl
von Werten $\xi$ hervorgehoben.

\begin{beispiel}
Die Differentialgleichung
\[
y''=\cos 3\pi x.
\]
hat die Lösung
\[
y(x)=-\frac1{9\pi^2}(\cos 3\pi x + 2x - 1),
\]
die man auch mit der Greenschen Funktion berechnen kann:
\[
y(x)=\int_0^1 G(x,\xi)\cos 3\pi\xi\,d\xi.
\]
In Abbildung
\ref{elliptisch:green-beispiele}
ist $f(x)=\cos 3\pi x$ durch eine Summe von einigen $\delta$-Distribution
approximiert worden (blau).
Die zugehörige Lösung ist jeweils rot eingezeichnet, sie hat
Knicke bei den $\delta$-Funktionen. 
Bei grösser werdender Zahl von $\delta$-Funktionen unterscheidet sich
das Integral $\int G(x,\xi)f(\xi)\,d\xi$ kaum mehr von der Lösung.
\begin{figure}
\begin{center}
\includegraphics[width=0.7\hsize]{../common/graphics/green-1.pdf}\\
\includegraphics[width=0.7\hsize]{../common/graphics/green-324.pdf}\\
\includegraphics[width=0.7\hsize]{../common/graphics/green-1082.pdf}
\end{center}
\caption{Lösung (rot) von $y''=f$ für die Approximation der Funktion $f$
durch eine Summe von $\delta$-Distributionen (blau).
\label{elliptisch:green-beispiele}}
\end{figure}

Eine Animation der Berechnung der Lösung $y(x)$ mit der Greenschen
Funktion ist auch auf Youtube zu finden: \url{http://www.youtube.com/watch?v=Wpi7Gf7V2HY}
\end{beispiel}


%
% general.tex -- 
%
% (c) 2019 Prof Dr Andreas Mueller
%
\section{The general case}
\rhead{The general case}
We now want to solve the general case of the problem
\[
\begin{aligned}
\Delta u&=f&&\text{in $\Omega$}\\
u&=g&&\text{auf $\partial\Omega$}
\end{aligned}
\]
using an integral formula of the kind
\eqref{greenformula}.
Following the one dimensional case, we are looking for a particular
solution $G(x,\xi)$ for the problem with a $\delta$-distribution on
the right hand side.
Once such a solution is found, we can find solutions for the general
problem using an integral formular.

\subsection{A particular solution}
\rhead{Particular solution}
The one-dimensional sample problem suggests that there is a function
$\sigma(x,\xi)$, which allows to find a particular solution of the
Laplace equation using the integral
\begin{equation}
u_p(x)=\int_\Omega \sigma(x,\xi)  f(\xi)\,d\xi.
\label{singulaereloesunglaplace}
\end{equation}
This particular solution does not need to satisfy the boundary conditions,
we just require that $\Delta u=f$ holds.

The solution $\sigma$ does not have to depend on the shape of the domain,
as it does not have to respect any boundary conditions,
but it will change with the dimension $n$.
If we choose the right hand side $f$ to be a $\delta$ distribution in the
point $y$, then
\[
u(x)
=
\int_\Omega \sigma(x,\xi)\delta(\xi - y)\,d\xi
=
\sigma(x,y).
\]
Thus the partial function $x\mapsto \sigma(x,y)$ satisfies the equation
$\Delta \sigma(x,y)=0$ for all points in ${\mathbb R}^n\setminus\{y\}$.
The function $\sigma$ thus is a harmonic function in the set
$\mathbb R^n\setminus\{y\}$.

Determining the function $\sigma(x,y)$ is a bit tedious, we just give the
result.
We find\footnote{%
This solution can be found using the following argument.
First the solution should only depend on the distance $|x-\xi|$, which
means we can assume without losing any generality that $\xi=0$.
As $\Delta u=\operatorname{div}\operatorname{grad}u$ we  conclude using
Gauss' theorem that
\begin{align*}
1&=\int_{B_r^n} \Delta u(x)\,d\mu(x)
=
\int_{B_r^n} \operatorname{div}\operatorname{grad} u(x)\,d\mu(x)
=\int_{S_r^{n-1}}\operatorname{grad}u(x)\cdot dn
\\
&=\int_{S_r^{n-1}}|\operatorname{grad}u(x)|\,d\mu(x)
=\mu(S_r^{n-1})\left|\frac{d}{dr}u(r)\right|
\\
\Rightarrow\qquad\frac{d}{dr}u(r)
&=\frac1{\mu(S_r^{n-1})}
=\frac1{\mu(S_1^{n-1})r^{n-1}}.
\end{align*}
For $n=2$ we have
$\mu(S_r^1)=2\pi r$, which implies
\[
u'(r)=\frac1{2\pi r}\quad\Rightarrow\quad u(r)=\frac1{2\pi}\log|r|.
\]
For $n\ge 3$ we get
\[
u'(r)=\frac1{\mu(S^{n-1})}r^{1-n}\quad\Rightarrow\quad u(r)=\frac1{\mu(S^{n-1})}\cdot \frac1{2-n}r^{2-n}.
\]
}
\begin{equation}
\sigma(x,\xi)=
\begin{cases}
\displaystyle \frac12|x-\xi|
&\qquad \text{for $n=1$}
\\
\\
\displaystyle \frac1{2\pi}\log|x-\xi|
&\qquad \text{for $n=2$}
\\
\\
\displaystyle -\frac1{4\pi}\frac1{|x-\xi|}
&\qquad \text{for $n= 3$}
\\
\\
\displaystyle \frac1{(2-n)\mu(S^{n-1})}|x-\xi|^{2-n}
&\qquad \text{for $n\ge 3$}
\end{cases}
\end{equation}
For $n=1$ we already have determined this function in \eqref{n1sigma}.
By carefully performing all these derivatives we can verify
that these functions are in fact solutions and that
\eqref{singulaereloesunglaplace} is particular solution of
Laplace equation.

\subsection{Green's function}
The propsed $\sigma$ is of course not the only function for which
the formula \eqref{singulaereloesunglaplace} will give a particular solution.
By adding any function $h(x,\xi)$ which is harmonic as a function of $x$
the integral
\[
u(x)=\int_{\Omega}\sigma(x,\xi)f(\xi)\,d\xi-\int_{\Omega}h(x,\xi)f(\xi)\,d\xi
\]
will also be a solution of
\eqref{elliptisch:laplaceequation}.
In particular we can choose $h(x,\xi)$ in such a way that
the partial function
$x\mapsto h(x,\xi)$
solves the boundary value problem
\[
\begin{aligned}
\Delta_x h(x,\xi)&=0&&x\in\Omega\\
h(x,\xi)&=\sigma(x,\xi)&&x\in\partial\Omega.
\end{aligned}
\]
The function $h$ then has the same boundary values as $\sigma$.
Consequently the difference $\sigma(x,\xi)-h(x,\xi)$ is a function
that vanishes on the boundary $\partial\Omega$.
The solution $u(x)$ found using \eqref{singulaereloesunglaplace}
then also vanishes on the boundary.
We set
\[
G(x,\xi)=\sigma(x,\xi)-h(x,\xi)
\]
and call this the Green's function for the Poisson problem on the
domain $\Omega$.

\begin{satz}If $\Omega$ is a domain on which the Poisson problem
has unique solutions, then there is a function $G(x,\xi)$,
which, as a function $x$, solves the equation
\[
\Delta G(x,\xi)=\delta(x-\xi)
\]
with homogenous boundary conditions.
\end{satz}

Note that this theorem guarantees the existence of Green's function 
but does not provide a method to obtain it.
Green's function deends only on the shape of the domain, not on
the actual boundary condtions.
Since there are domains where the Poisson problem with Dirichlet
boundary conditions cannot be solved in a unique way, the theorem
needs to require this as a precondition.

Green's function for the Poisson problem solves the problem
not only for homogeneous ($g=0$) boundary conditions, sondern
for any boundary values $g\ne 0$:

\begin{satz}[Solution of Poisson problem with Dirichlet boundary conditions]
\label{dirichletloesung}
If $G$ is Green's function for the Poisson problem with Dirichlet
boundary values on the domain $\Omega$, then
\[
u(x)
=
\int_{\Omega}G(x,\xi)f(\xi)\,d\mu(\xi)
+
\int_{\partial\Omega}g(\xi)\operatorname{grad}_\xi G(x,\xi)\cdot dn
\]
is the solution for the same problem with arbitrary boundary values
\eqref{dirichletrandbedingung}.
\end{satz}

Note that is very much in line with what we have found for the one
dimensional case.
The terms of \eqref{elliptic:greendiff} containing the derivatives
of $G$ can be considered as the integral of the gradient of $G$
over the bounary of the interval, which would be computed as a sum
of the values in the boundary points.

\begin{proof}
We already know that the integral over $\Omega$ is a solution of the
inhomogeneous partial differential equation with homogeneous boundary
conditions.
What remains is to show that the second term is a harmonic function,
i.~e.~it solves the homogeneous differential equation,
with boundary values $g$.

For this proof we need the following identity
\begin{align*}
\operatorname{div}(u\operatorname{grad}v)
&=
\sum_i\partial_i(u\partial_iv)
\\
&=\sum_i(\partial_iu)(\partial_iv)+\sum_iu\partial_i^2v
\\
&=\operatorname{grad}u\cdot\operatorname{grad}v+u\Delta v,
\end{align*}
from which we can derive
\[
u\Delta v-v\Delta u
=
\operatorname{div}(u\operatorname{grad}v-v\operatorname{grad}u).
\]

We apply these identities to the function $u(\xi)$ and 
to Green's function $\xi\mapsto v(\xi)=G(x,\xi)$:
\[
u(\xi)\Delta G(x,\xi)-G(x,\xi)\Delta u
=
\operatorname{div}(u\operatorname{grad}G(x,\xi)-G(x,\xi)\operatorname{grad}u)
\]
Integrating both sides over $\Omega$ and applying Gauss' theorem to
the right hand side gives
\begin{align*}
\int_{\Omega}u(\xi)\delta(\xi - x)-G(x,\xi)f(\xi)\,d\mu(\xi)
&=
\int_{\Omega}\operatorname{div}(u(\xi)\operatorname{grad}G(x,\xi)-G(x,\xi)\operatorname{grad}u)\,d\mu(\xi)
\\
u(x)-\int_{\Omega}G(x,\xi)f(\xi)\,d\mu(\xi)
&=\int_{\partial \Omega}u(\xi)\operatorname{grad}G(x,\xi)-G(x,\xi)\operatorname{grad}u(\xi)\cdot dn
\end{align*}
The second term on the right vanishes becuase $G(x,\xi)$ was chosen so
that the boundary values vanish.
In the left term we can replace $u(\xi)$ by the boundary values $g(\xi)$.
What remains is
\[
u(x)
=
\int_{\Omega}G(x,\xi)f(\xi)\,d\mu(\xi)
+
\int_{\partial\Omega}g(\xi)\operatorname{grad}G(x,\xi)\cdot dn,
\]
as claimed.
\end{proof}

\subsection{Application}
\rhead{Application}
\begin{figure}
\begin{center}
\includegraphics[width=0.8\hsize]{../common/graphics/neilcontour}
\end{center}
\caption{Level lines\label{neilcontour}}
\end{figure}
\begin{figure}
\begin{center}
\includegraphics[width=0.8\hsize]{../common/graphics/neilloesung}
\end{center}
\caption{Potential surface\label{neilloesung}}
\end{figure}
This example illustrates how the ideas in this section can be used
to solve some practical problems at least using some numerical tool.

{\parindent 0pt
\medskip
{\bf Problem:}
The boundary of an electricaly conducting rectangular plate is connected
to the negative terminal of a battery and at the point
$x_0\in\mathbb R^2$ in the interior of the plate, the other terminal
of the battery is connected.
Compute the potential at any point $x$ of the plate with respect to
the boundary.

\medskip
}
The potential
$u(x,y)$
we are looking for is the solution of the equation
\[
\Delta u=\delta(x-x_0)
\]
with boundary values
\[
u(x)=0.
\]
Numerical computation of the solution in the vicinity of the point $x_0$
is very imprecise.
This motivates the following procedure.
The function $\sigma(x,x_0)$ is a harmonic function outside $x_0$,
\[
\Delta\sigma(x,x_0)=\delta(x-x_0)
\]
So appart from the boundary values that are not correct it could be
a solution.
So the next step is to fix the boundary values of this partial solution.
For this we solve the boundary value problem
\begin{align*}
\Delta h(x)&=0&&x\in\Omega\\
h(x)&=\sigma(x,x_0)&&x\in\partial\Omega,
\end{align*}
which can be done efficiently with existing software.
E.~g.~we could use the mean value property to be discussed later in this
chater to numerically compute such a solution.
Then the function
\[
u(x)=\sigma(x,x_0)-h(x)
\]
is a solution of the original problem.
The level lines of this solutions and the potential surface of $u(x)$ is
shown in figures~\ref{neilloesung} and \ref{neilcontour}.


%
% mean.tex -- 
%
% (c) 2019 Prof Dr Andreas Mueller
%
\section{The mean value property of harmonic functions}
\rhead{Mean value property}
Harmonic functions have a property that is even stronger than
the maximum principle for elliptic operators.
The maximum principle says that the solution of a homogeneous elliptic
partial differential equation is bounded above and below by the
extreme values on the boundary.
The mean value property in addition says that the value in some
point is actually the mean of all the values in small circle
around the point.

\subsection{Mean value property}
In this section we are going to illustrate that the values of harmonic
functions are means of the values of the function on a sphere around 
the point.
\begin{figure}
\centering
\includegraphics{7-elliptic/images/meanvalue.pdf}
\caption{Mean value property of a harmonic function.
The surface represents a harmonic function. 
The mean value along each of the red circles gives the same value,
namely the function value in the yellow center point $C$.
\label{elliptic:meanvalue:image}}
\end{figure}

\begin{satz}[Mean value property]
Let $u$ be a harmonic function on $\Omega$ and $x_0\in\Omega$.
If $r$ is small enough so that the sphere with radius $r$ in $x_0$
fits inside the domain $\Omega$,
then
\[
u(0)=\frac1{\mu(S^{n-1}_r)}\int_{S^{n-1}_r}u(x)\,d\mu(x).
\]
\end{satz}

The proof of this statement requires some techniques from multivariate
analysis, in particular Gauss' divergence theorem.

\begin{proof}
Without loss of generality, we can assume that $x_0=0$.
Then
\[
\Delta u=\operatorname{div}\operatorname{grad}u
\]
follows from Gauss' theorem
\begin{align*}
0&=\int_{B_r^n}\Delta u\,d\mu(x)
\\
&=\int_{B_r^n}\operatorname{div}\operatorname{grad}u\,d\mu(x)
\\
&=\int_{S_r^{n-1}} \operatorname{grad}u\cdot dn.
\end{align*}
On the other hand
\begin{align*}
\frac{d}{dr}\frac{1}{\mu(S_r^{n-1})}\int_{S_r^{n-1}} u(x)\,d\mu(x)
&=
\frac1{\mu(S_1^{n-1})}\frac{d}{dr}\int_{S_1^{n-1}}u(xr)\,d\mu(x)
\\
&=
\frac1{\mu(S_1^{n-1})}\int_{S_1^{n-1}}\operatorname{grad}u(xr)\cdot x
\,d\mu(x)
\\
&=
\frac1{\mu(S_1^{n-1})}\int_{S_1^{n-1}}\operatorname{grad}u(xr)\cdot dn=0.
\end{align*}
The mean value does not depend on the radius.
By continuity
\[
\lim_{r\to 0}\frac1{\mu(S_r^{n-1})}\int_{S_r^{n-1}}u(x)\,d\mu(x)=u(0),
\]
so the claim follows.
\end{proof}

We can use the mean value property to solve the Poisson problem with
Dirichlet boundary values for a ball.

\begin{satz}[Poisson-Formula]
\index{Poisson-Formula}
Let $g$ be a continuous function on the boundary of an $n$-dimensional
unit sphere.
Then
\[
u(x)=\begin{cases}
\displaystyle \frac{1-|x|^2}{\mu(S^{n-1})}
\int_{S^{n-1}}\frac{g(\xi)}{|x-\xi|^n}\,d\mu(\xi)&\qquad |x|<1\\
g(x)&\qquad |x|=1
\end{cases}
\]
is a harmonic function with boundary values $u_{|S_1^{n-1}}=g$.
\end{satz}

\subsection{Maximum principle and mean value proeprty}
\begin{satz}[Maximum principle]
If the function $u$ on the domain
$\Omega$
is harmonic and assumes a maximum in an interior point,
then $u$ is constant.
\end{satz}

\begin{proof}
Let $x_0\in\Omega$ be the interior point where $u$ takes the maximum.
If $u$ is not constant, the set of points $x$ where $u(x)=u(x_0)$ is
closed and at same time it is not the whole of $\Omega$.
So there is a point $x$ with $u(x)=u(x_0)$ and at the same time
in every neighborhood of $x$ we will find points where the value of
$u$ is strictly smaller.
Since $\Omega$ is open, there is some small sphere of radius $r$
around $x$ that is completely contained in $\Omega$.
We denote the surface of this sphere by $K_r$.
We know that there are points on $K_r$ where $u$ is strictly smaller
than $u(x_0)$.
This implies that the mean value of $u$ over $K_r$ will also be
strictly smaller than $u(x_0)$, contradicting the mean value property.
This contradiction shows that $u$ must be constant.
\end{proof}

The maximum principle imples that any function that has the
mean value property is also harmonic.
For if $u$ has the mean value property and $v$ is harmonic with
the same boundary values as $u$, then the difference $u-v$ is 
function that satisfies the maximum principle with boundary
values $0$.
Consequently $u-v=0$, so $u$ and $v$ are the same und $u$ is also
harmonic.

This works in one dimension too.
A ``harmonic'' function in one dimension is just a linear function
$u(x)=ax+b$.
If $u(x)$ has a maximum in the interior, then $u(x)$ the derivative
has to be $0$ in that interior point, which implies $a=0$ and $u(x)=b$
constant.


%
% generalizations.tex
%
% (c) 2019 Prof Dr Andreas Mueller
%
\section{Generalizations}
\rhead{Generalizations}
The methods presented in this chapter can be generalized in various
directions.

\subsection{Neumann boundary conditions}
\index{Neumann boundary conditions}
The Poisson problem with Dirichlet boundary conditions was solved
using Green's function
$G(x,\xi)$ which, as a function of $x$ satisfies homogeneous
Dirichlet boundary conditions
\begin{align*}
G(x,\xi)
&=
0\quad\forall x\in\partial\Omega.
\\
\intertext{If we construct a Green's function which satisfies homogenous 
Neumann boundary conditions}
\frac{\partial}{\partial n}G(x,\xi)
&=
0\quad\forall x\in\partial\Omega,
\end{align*}
then what remains from the proof of theorem \ref{dirichletloesung}
is the formula
\begin{align*}
u(x)&=\int_{\Omega}G(x,\xi)f(\xi)\,d\mu(\xi)+\int_{\partial\Omega}G(x,\xi)\operatorname{grad}u(\xi)\cdot dn
\end{align*}
For homgeneous Neumann boundary conditions
$\partial_n u=g$, one can replace that by
\begin{align*}
u(x)&=\int_{\Omega}G(x,\xi)f(\xi)\,d\mu(\xi)+\int_{\partial\Omega}G(x,\xi)g(\xi)\,d\mu(\xi)
\end{align*}
So for Neumann boundary conditions there also is a Green's function
which solves the Neumann boundary value problem for arbitrary 
boundary values.

\subsection{More general operators}
Green's function can be constructed under suitable conditions
for the coefficients of the operator also for arbitrary
elliptic operators
\[
Lu=\biggl(\sum_{i,j}a_{ij}\frac{\partial^2}{\partial x_i\partial x_j}
+\sum_ib_i\frac{\partial}{\partial x_i} +c\biggr)u
\]
of second order.

\subsection{Abstract solution}
The principles in this chapter can be summarized and generalized as 
follows.
Given an elliptic operator $L$ and an additional operator $B$ which
extracts the boundary values from $u$.
For any function $u$ on $\bar\Omega$ the function $Bu$ is defined on
$\partial\Omega$.
The boundary value problem then can be stated as 
\[
\begin{aligned}
Lu&=f(x)&&x\in\Omega
\\
Bu&=g(x)&&x\in\partial\Omega
\end{aligned}
\]
and can be solved by some integral formular of the form
\[
u(x)
=
\int_\Omega G(x,\xi)f(\xi)\,d\mu(\xi)
+
\int_{\partial \Omega}K(x,\xi)g(\xi)\,d\mu(\xi).
\]
The functions $G$ and $K$ allow the operators $L$ and $B$ to be inverted.



\section{Summary}
\rhead{Summary}
\begin{enumerate}
\item
The differential equation $\Delta u=f$ is the prototypical elliptic
partial differential equation.
\item
Solutions of a homogeneous elliptic partial differential equation,
in particular the equation $\Delta u=0$, satisfy the maximum principle:
minimum and maximum of $u$ lie on the boundary of the domain.
\item
The solutions of elliptic partial differential equation with boundary
values along the boundary of bounded domain are unique.
\item
A well posed elliptic differential equation has a Green's function,
i.~e.~a solution of the differential equation
$\Delta G(x,\xi)=\delta_\xi$ for every point in $\xi\in\Omega$
with homogeneous boundary conditions.
\item
The Green's function can be used to solve the inhomogeneous equation
by means of an integral over the domain.
\item
Harmonic functions are solutions of the equation $\Delta u=0$.
\item
Harmonic functions have the mean value property: the value of a harmonic
function in a point is the mean of the function values on a sphere around
the point.
\end{enumerate}


