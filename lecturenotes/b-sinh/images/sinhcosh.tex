%
% sinhcosh.tex -- Template für standalone TIKZ Bilder
%
% (c) 2019 Prof Dr Andreas Müller, Hochschule Rapperswil
%
\documentclass[tikz,12pt]{standalone}
\usepackage{amsmath}
\usepackage{times}
\usepackage{txfonts}
\usepackage{pgfplots}
\usepackage{csvsimple}
\usetikzlibrary{arrows,intersections,math}
\begin{document}
\begin{tikzpicture}[>=latex,scale=2]

\draw[->,line width=0.7pt] (-2.15,0)--(2.5,0) coordinate[label={$x$}];
\draw[->,line width=0.7pt] (0,-3.8)--(0,3.9) coordinate[label={right:$y$}];

\draw[color=red,line width=1pt]
	plot[domain=-2:2,samples=100] ({\x},{0.5*(exp(\x)-exp(-\x))});
\draw[color=blue,line width=1pt]
	plot[domain=-2:2,samples=100] ({\x},{0.5*(exp(\x)+exp(-\x))});

\foreach \y in {1,...,3}{
	\draw[line width=0.7pt] (-0.05,{\y})--(0.05,{\y});
	\draw[line width=0.7pt] (-0.05,{-\y})--(0.05,{-\y});
	\node at (-0.05,{-\y}) [left] {$-\y$};
}
\foreach \x in {1,...,2}{
	\draw[line width=0.7pt] ({\x},-0.05)--({\x},0.05);
	\draw[line width=0.7pt] ({-\x},-0.05)--({-\x},0.05);
	\node at ({\x},-0.05) [below] {$\x$};
	\node at ({-\x},-0.05) [below] {$-\x$};
}

\node at (0,0) [below right] {$0$};
\node at (-0.05,1) [below left] {$1$};
\node at (-0.05,2) [left] {$2$};
\node at (-0.05,3) [left] {$3$};

\end{tikzpicture}
\end{document}

