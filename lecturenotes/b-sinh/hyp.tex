%
% hyp.tex -- 
%
% (c) 2015 Prof Dr Andreas Müller, Hochschule Rapperswil
%
\section{Hyperbolic functions}
\lhead{Hyperbolic functions}
\begin{figure}
\centering
\includegraphics{../common/images/hf-1.pdf}
\caption{Graphs of the functions 
$x\mapsto\cosh x$ (blue)
and $x\mapsto\sinh x$ (red)
\label{hyp:graphen}}
\end{figure}
The hyperbolic function are defined as
\[
\sinh x =\frac{e^x-e^{-x}}2
\qquad\text{and}\qquad
\cosh x = \frac{e^x+e^{-x}}2.
\]
Figure~\ref{hyp:graphen} shows graphs of those two functions.
At first sight these definitions have nothing to do with trigonometric
functions, the names {\em sinus hyperbolicus} for $\sinh x$ and
{\em cosinus hyperbolicus} for $\cosh x$ seem unjustified.

\subsection{Complex definition of the trigonometric functions}
From the Euler-formula
\[
e^{it}=\cos t+i\sin t
\]
a complex definition for the trigonometric functions can be derived
as follows.
Using the Euler-formula for $t$ and $-t$ we get
\begin{align*}
\cos t+i\sin t&=e^{it}\\
\cos t-i\sin t&=e^{-it}.
\end{align*}
This is a system of linear equations which can be solved
for the functions $\cos t$ and $\sin t$ using the addition method.
We find
\begin{align*}
\cos t
&=
\frac{e^{it}+e^{-it}}2
&
\sin t
&=
\frac{e^{it}-e^{-it}}{2i}.
\end{align*}
The trigonometric functions can thus be defined in a way very similar
to the hyperbolic functions.
The only difference is a factor $i$ here and there.
Thus it is reasonable to expect that the hyperbolic functions
have many properties completely analogous to properties of the
trigonometric functions, with may be a sign change here and there.

\subsection{Geometry}
The trigonometric functions stem from the desire to compute right
triangles.
More precisely, the trigonometric functions compute right triangles
where one side has length $1$.
This is usually illustrated by identifying the values of the
trigonometric functions as line segments in a unit circle.
This is possible thanks to the well known identity
\[
\sin^2t+\cos^2t=1
\]
which implies that the function
\[
t\mapsto (\cos t,\sin t)
\]
parametrizes the unit circle.

\begin{figure}
\centering
\includegraphics{../common/images/hf-2.pdf}
\caption{The curve with the parametrization
$t\mapsto (\cosh t, \sinh t)$ is a ``unit hyperbola'' (red) with 
asymptotes
$y=\pm x$ (blue).
\label{anhang:hyperbel}}
\end{figure}
The hyperbolic functions define a map
\[
t\mapsto (\cosh t, \sinh t)
\]
which is of course in the plane.
To find out what type of curve this could be, we compute the
squares of these functions:
\begin{align}
\sinh^2 t&=\frac14(e^{2t}-2+e^{-2t}),
\\
\cosh^2 t&=\frac14(e^{2t}+2+e^{-2t}).
\label{hyp:definition}
\end{align}
Up to the middle term, the are identical.
Their difference is
\[
\cosh^2t - \sinh^2t=1,
\]
so the hyperbolic functions describe a curve with the equation
\[
x^2-y^2=1.
\]
This is a hyperbola with asymptotes $y=\pm x$
(figure~\ref{anhang:hyperbel}).
This justifies that we call these functions hyperbolic functions.

\subsection{Addition theorems}
Is there an addition formular for hyperbolic functions?
If the analogy to trigonometric functions works, we would expect 
formulae like
\begin{align*}
\sinh(a+b)&=\sinh a\cosh b + \cosh a\sinh b,\\
\cosh(a+b)&=\cosh a\cosh b - \sinh a\sinh b,
\end{align*}
with some different signs here and there.
To verify this, we use the definition and compute
\begin{align*}
\sinh a\cosh b + \cosh b\sinh b
&=
\frac14(e^a-e^{-a})(e^b+e^{-b})
+
\frac14(e^a+e^{-a})(e^b-e^{-b})
\\
&=\frac14(e^{a+b}+e^{a-b}-e^{-a+b}-e^{-a-b} + e^{a+b}-e^{a-b}+e^{-a+b}-e^{-a-b})
\\
&=
\frac14(2e^{a+b}-2e^{-a-b})
=
\frac12(e^{a+b}-e^{-a-b})=\sinh(a+b),
\\
\cosh a\cosh b-\sinh a\sinh b
&=
\frac14(e^a+e^{-a})(e^b+e^{-b})
-
\frac14(e^a-e^{-a})(e^b-e^{-b})
\\
&=
\frac14(e^{a+b}+e^{a-b}+e^{-a+b}+e^{-a-b})
-
\frac14(e^{a+b}-e^{a-b}-e^{-a+b}+e^{-a-b})
\\
&=
\frac14(2e^{a-b}+2e^{-a+b})
=
\frac12(e^{a-b}+e^{-a+b})=\cosh(a-b).
\end{align*}
The conjecture regarding the addition theorems was correct in the cause
of $\sinh(a+b)$, but for $\cos(a+b)$ we had to change a sign.
The correct form of the addition formulae is
\begin{align*}
\sinh(a\pm b)&=\sinh a\cosh b \pm \cosh a\sinh b,\\
\cosh(a\pm b)&=\cosh a\cosh b \pm \sinh a\sinh b.
\end{align*}

\subsection{Derivatives}
In analysis one learns to derive formulae for the derivatives of the
trigonomtric functions from the addition theorems.
Since the addition theorems for hyperbolic functions are so similar
up the a sign, the the rules for the derivatives should look just the
same, up to a sign.

We can also find the derivatives directly from the definition
\eqref{hyp:definition} by applying the derivation rules for the
exponential function:
\begin{align*}
\frac{d}{dx}\cosh x
&=
\frac12\frac{d}{dx}(e^x+e^{-x})
=
\frac12(e^x-e^{-x})=\sinh x
\\
\frac{d}{dx}\sinh x
&=
\frac12\frac{d}{dx}(e^x-e^{-x})
=
\frac12(e^x+e^{-x})=\cosh x
\end{align*}
Thus the derivatives are even simpler, as they start to repeat already
at the second derivative.
The second derivative already is the original function
\begin{align*}
\sinh''x&=\sinh x
&
\cosh''x&=\cosh x.
\end{align*}

\subsection{Values for argument $0$}
The values of the trigonometric functions and their first derivatives
at the origin and the respective values for the hyperbolic
functions also are very similar:
\begin{center}
\begin{tabular}{|l|>{$}c<{$}>{$}c<{$}|>{$}c<{$}>{$}c<{$}|}
\hline
Wert für $x=0$ von&\sin x&\cos x&\sinh x&\cosh x\\
\hline
Funktion           &  0   &  1   &   0   &   1   \\
Ableitung          &  1   &  0   &   1   &   0   \\
\hline
\end{tabular}
\end{center}

\subsection{Solving differential equations}
The values for argument $0$ are precisely the properties
that make satisfying the initial values of a differential 
equation so simple.
The differential equation
\[
y''-k^2y=0
\]
with initial conditions
\[
y(0)=y_0\qquad\text{and}\qquad y'(0)=v_0
\]
has solution
\[
y(x)=y_0\cosh kx +\frac{v_0}{k}\sinh kx,
\]
in complete analogy to
\eqref{hyp:loesung}.

