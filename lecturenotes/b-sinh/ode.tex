%
% ode.tex -- 
%
% (c) 2019 Prof Dr Andreas Müller, Hochschule Rapperswil
%
\section{Differential equations of second order}
\lhead{Differential equations of second order}
The standard method to solve ordinary differential equations
with constant coefficients like
\[
a_2y''+ a_1y'+a_0y=0
\]
prescribes that first the zeros $\lambda_{1,2}$ of the characteristic
polynomial
\[
p(\lambda)=a_2\lambda^2+a_1\lambda+a_0
\]
need to be found.
The general solution of the differential equation then is a linear
combination
\[
y(x)=
A_1 e^{\lambda_1t}
+
A_2 e^{\lambda_2t},
\]
the constants $A_1$ and $A_2$ are to be determined from the initial
conditions.
If $y_0$ and $v_0$ are the initial value and the initial derivative,
then the linear system of equations
\[
\begin{linsys}{3}
         A_1&+&         A_2&=&y_0\\
\lambda_1A_1&+&\lambda_2A_2&=&v_0
\end{linsys}
\]
can be used to find the coefficients $A_1$ and $A_2$.
This becomes particularly easy to solve for the differential equation
\[
y''+k^2 y=0.
\]
Then the characteristic values are $\lambda_{1,2}=\pm\sqrt{-k^2}$ imaginary,
and instead of the exponential functions we can use the trigonometric
functions
\[
y(x)=A\cos kx+B\sin kx.
\]
The value of $\sin kx$ vanishes at $x=0$ just as the derivative
of $\cos kx$.
The remaining equations give the constants $A$ and $B$ immediately as
\[
A=y_0
\qquad\text{und}\qquad
B=\frac1kv_0,
\]
or
\begin{equation}
y(x)=y_0\cos kx + \frac{v_0}{k}\sin kx
\label{hyp:loesung}
\end{equation}

This does not work for the analogous differential equation
$y''-k^2y=0$, which has a different sign.
The zeros of the characteristic polynomial are $\lambda_{1,2}=\pm k$,
which leads to the linear system of equations
\[
\begin{linsys}{3}
 A_1&+& A_2&=&y_0\phantom{.}\\
kA_1&-&kA_2&=&v_0.
\end{linsys}
\]
The solution can be found using the determinant method, which is
capable of giving formulae for the solutions:
\begin{align*}
A_1
&=
\frac{\left|\begin{matrix}y_0&1\\v_0&-k\end{matrix}\right|}{\left|\begin{matrix}1&1\\k&-k\end{matrix}\right|}
=
\frac{-ky_0+v_0}{-2k},
&
A_2
&=
\frac{\left|\begin{matrix}1&y_0\\k&v_0\end{matrix}\right|}{\left|\begin{matrix}1&1\\k&-k\end{matrix}\right|}
=\frac{v_0-ky_0}{-2k}.
\end{align*}
This now allows to write down the solutions in this case:
\begin{align}
y(x)
&=
A_1e^{kx}+A_2e^{-kx}
=
\frac12\biggl(
\frac{ky_0-v_0}k e^{kx}
+
\frac{-v_0+ky_0}k e^{-kx}
\biggr)
\notag
\\
&=
y_0\frac{e^{kx}+e^{-kx}}2
+\frac{v_0}{k}\frac{e^{kx}-e^{-kx}}2.
\label{hyp:hyperbelfunktionen}
\end{align}
It turns out that satisfying the initial conditions becomes simple
even in this case provided we do not use the functions $e^{\pm kx}$
but instead the factors of $y_0$ and $v_0/k$ in~\eqref{hyp:hyperbelfunktionen}.

