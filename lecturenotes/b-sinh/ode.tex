%
% ode.tex -- XXX
%
% (c) 2019 Prof Dr Andreas Müller, Hochschule Rapperswil
%
\section{Differentialgleichungen zweiter Ordnung}
Das Standardverfahren für die Lösung einer gewöhnlichen linearen
Differentialgleichung mit konstanten Koeffizienten
\[
a_2y''+ a_1y'+a_0y=0
\]
schreibt vor, dass man erst die Nullstellen $\lambda_{1,2}$
des charakteristischen Polynoms
\[
p(\lambda)=a_2\lambda^2+a_1\lambda+a_0
\]
finden muss.
Die allgemeine Lösung der Differentialgleichung ist dann
eine Linearkombination
\[
y(x)=
A_1 e^{\lambda_1t}
+
A_2 e^{\lambda_2t},
\]
die Konstanten $A_1$ und $A_2$ sind aus den Anfangswerten zu bestimmen.
Sinde $y_0$ und $v_0$ der Anfangswert und die Anfangsableitung, dann
findet man das lineare Gleichungssystem
%\newcolumntype{\linsysR}{>{$}r<{$}}
%\newcolumntype{\linsysL}{>{$}l<{$}}
%\newcolumntype{\linsysC}{>{$}c<{$}}
%\newenvironment{linsys}[1]{%
%\begin{tabular}{*{#1}{\linsysR@{\;}\linsysC}@{\;}\linsysR}}%
%{\end{tabular}}
\[
\begin{linsys}{3}
         A_1&+&         A_2&=&y_0\phantom{.}\\
\lambda_1A_1&+&\lambda_2A_2&=&v_0.
\end{linsys}
\]
Besonders einfach wird die Bestimmung jedoch für die Differentialgleichung
\[
y''+k^2 y=0.
\]
Dann sind die $\lambda_{1,2}=\pm\sqrt{-k^2}$ imaginär, und man kann statt
der Exponentiallösungen auch den Ansatz
\[
y(x)=A\cos kx+B\sin kx
\]
verwenden.
Da der Wert von $\sin kx$ bei $x=0$ verschwindet, und die Ableitung von
$\cos kx$ ebenfalls, ist die Bestimmung der Konstanten viel einfacher:
\[
A=y_0
\qquad\text{und}\qquad
B=\frac1kv_0,
\]
oder
\begin{equation}
y(x)=y_0\cos kx + \frac{v_0}{k}\sin kx
\label{hyp:loesung}
\end{equation}

Für die analoge Differentialgleichung $y''-k^2y=0$ geht dies nicht.
Die Nullstellen des charakteristischen Polynoms sind hier 
$\lambda_{1,2}=\pm k$, und es führt nichts an dem linearen Gleichungssystem
\[
\begin{linsys}{3}
 A_1&+& A_2&=&y_0\\
kA_1&-&kA_2&=&v_0
\end{linsys}
\]
vorbei.
Die Lösung kann allerdings zum Beispiel mit dem Determinantenverfahren
ziemlich direkt gefunden werden:
\begin{align*}
A_1
&=
\frac{\left|\begin{matrix}y_0&1\\v_0&-k\end{matrix}\right|}{\left|\begin{matrix}1&1\\k&-k\end{matrix}\right|}
=
\frac{-ky_0+v_0}{-2k},
&
A_2
&=
\frac{\left|\begin{matrix}1&y_0\\k&v_0\end{matrix}\right|}{\left|\begin{matrix}1&1\\k&-k\end{matrix}\right|}
=\frac{v_0-ky_0}{-2k}.
\end{align*}
Damit kann man jetzt die Lösung auch in diesem Fall hinschreiben:
\begin{align}
y(x)
&=
A_1e^{kx}+A_2e^{-kx}
=
\frac12\biggl(
\frac{ky_0-v_0}k e^{kx}
+
\frac{-v_0+ky_0}k e^{-kx}
\biggr)
\notag
\\
&=
y_0\frac{e^{kx}+e^{-kx}}2
+\frac{v_0}{k}\frac{e^{kx}-e^{-kx}}2.
\label{hyp:hyperbelfunktionen}
\end{align}
Die Erfüllung der Anfangsbedingung könnte also auch in diesem Falle
sehr einfach sein, wenn man nicht die Funktionen $e^{\pm kx}$ verwenden
würde, sondern deren Linearkombinationen wie in (\ref{hyp:hyperbelfunktionen}).

