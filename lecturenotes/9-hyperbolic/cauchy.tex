%
% cauchy.tex -- 
%
% (c) 2019 Prof Dr Andreas Mueller
%
\section{The Cauchy problem in higher dimensions}
\rhead{The Cauchy problem}
In the last section we have given initial values on the straight line
$t=0$ which also fixed the initial values for $\partial_x u(0,x)$.
In addition, by specifying $v_0$, the normal derivative 
$\partial_t u(0,x)$ is given.
The graph of the solution function is a surface, called an integral
surface of the differential equation.
The initial conditions define a curve $x\mapsto(x,0,u_0(x))$,
the solution surface must contain this curve.
By specifying the normal derivative, in each point of the initial
curve we also specify a tangent plane.

As the example shows, the solution is determined uniquely by specifying
the initial curve and an initial sheaf of tangent planes.
More generally, the Cauchy problem consists in finding an integral
surface that contains a given curve and also, at each point of the curve,
has a given tangent plane.
This type of initial data is called an initial ``strip''
(see figure~\ref{skript:streifen}).

The solution of a partial differential equation for the function $u(t,x,y)$
of three variables cannot be uniquely determined by specifying
the initial values $u(t,x,y)=u_0(x,y)$ at time $t=0$.
This already fixes the partial derivatives
$\partial_x u(0,x,y)=\partial_xu_0(x,y)$
and $\partial_y u(0,x,y)=\partial_y u_0(x,y)$
in the $x$ and $y$ direction.
The solution will only be determined if we specify in addition the normal
derivative, e.~g.~in the form
$\partial_t(0,x,y)=v_0(x,y)$.
This means that the initial data implies initial values in all points
with $t=0$ 
and all derivatives in each point of $t=0$.

More generally, the Cauchy problem is to find a solution that assumes
given values on a surface in $(t,x,y)$-space and 
given normal derivatives on the same surface.

