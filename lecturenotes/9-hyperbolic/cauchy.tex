%
% cauchy.tex -- XXX
%
% (c) 2019 Prof Dr Andreas Mueller
%
\section{Das Cauchy-Problem in höheren Dimensionen}
\rhead{Das Cauchy-Problem}
Im letzten Abschnitt haben wir Anfangswerte auf der Geraden $t=0$
vorgegeben, damit waren auch gleichzeitig die Ableitungen 
$\partial_x u(0,x)$ festgelegt. Ausserdem hatten wir mit der Funktion $v_0$
die Ableitungen $\partial_t u(0,x)$ vorgegeben.
Der Graph der Lösungsfunktion ist ein Fläche, eine sogenante
Integralfläche der Differentialgleichung. Die Anfangsbedingung
definiert eine Kurve $x\mapsto(x,0,u_0(x))$, die Integralfläche muss
durch diese Kurve gehen.
Durch die zwei Ableitungen ist zudem in jedem Punkt der Kurve
eine Tangentialebene an die Integralfläche vorgegeben.

Wie das Beispiel zeigt, ist die Lösungsfläche durch Vorgabe einer Kurve und
und der Tangentialebenen in jedem Punkt der Kurve bestimmt ist.
Etwas allgemeiner besteht das Cauchy-Problem darin, eine Integralfläche
zu finden, die durch eine beliebige Kurve geht, und ausserdem in jedem Punkt
der Kurve eine bestimmte Tangentialebene hat. Diese Vorgaben nennt man
einen ``Streifen'' (Abbildung~\ref{skript:streifen}).
Eine partielle Differentialgleichung für eine Funktion $u(t,x,y)$
von drei Variablen kann zum Beispiel dadurch festgelegen werden,
dass man Funktionswerte $u(t,x,y)=u_0(x,y)$ zur Zeit $t=0$ festlegt.
Dadurch sind auch die Ableitungen $\partial_x u(0,x,y)=\partial_xu_0(x,y)$
und $\partial_y u(0,x,y)=\partial_y u_0(x,y)$ bestimmt. Die Lösung wird
aber erst eindeutig bestimmt sein, wenn auch die Ableitung in $t$-Richtung
vorgegeben ist, zum Beispiel in der Form $\partial_t(0,x,y)=v_0(x,y)$.

Allgemeiner besteht das Cauchy-Problem darin, eine Lösung zu finden
die entlang einer beliebigen Fläche im $(t,x,y)$-Raum vorgegebene
Werte annimmt. Ausserdem muss die Richtungsableitungen in eine Richtung
senkrecht auf die Fläche (Normalableitung) ebenfalls vorgegebene Werte annehmen.


