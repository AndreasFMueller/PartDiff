%
% waves.tex -- XXX
%
% (c) 2019 Prof Dr Andreas Mueller
%
\section{Wellengleichung in einer Dimension}
\rhead{Eindimensionale Wellengleichung}
Die Wellengleichung in der Ebene ist
\[
\partial_t^2u-a^2\partial_x^2u=0,
\]
welche wir auf dem Gebiet
\[
\Omega = \{(x,t) \,|\, t > 0\}
\]
lösen möchten.
$a$ hat die Dimension einer Geschwindigkeit, $a$ ist die
Ausbreitungsgeschwindigkeit der Wellen entlang der $x$-Achse.

\subsection{Konstante Geschwindigkeit}
Wir nehmen an, dass $a$ eine Konstante ist. Dann lässt sich die Gleichung
auch als
\begin{align*}
(\partial_t -a\partial_x)(\partial_t+a\partial_x)u&=0
\\
\text{oder}&
\\
(\partial_t +a\partial_x)(\partial_t-a\partial_x)u&=0
\end{align*}
schreiben.
Offenbar sind Lösungen der folgenden partiellen Differentialgleichungen
erster Ordnung
\begin{align}
\partial_t u-a\partial_x u&=0
\label{wellelinks}
\\
\partial_t u+a\partial_x u&=0
\label{wellerechts}
\end{align}
automatisch auch Lösungen der Wellengleichung.

\subsection{Lösung der partiellen Differentialgleichung erster Ordnung}
Wir möchten die Wellengleichung für eine Anfangsbedingung der Art
\begin{equation}
u(x,t)=u_0(x),\qquad x\in\mathbb R
\label{welleanfang}
\end{equation}
lösen, und suchen daher zunächst Lösungen der beiden
PDGL erster Ordnung (\ref{wellelinks}) und (\ref{wellerechts})
mit genau derselben Anfangsbedingung.

Beide Differentialgleichungen sind quasilineare Differentialgleichungen
erster Ordnung, das Verfahren aus Kapitel~\ref{chapter-geometrie}
liefert dafür eine Lösung. Wir haben dazu zunächst die Lösung der
Gleichung der Charakteristiken zu finden. Diese lautet
\begin{align*}
\frac{dx}{ds}&=-a
\\
\frac{dt}{ds}&=1
\\
\frac{du}{ds}&=0
\end{align*}
Die dritte Gleichung sagt, dass $u$ nicht von $s$ abhängt. Die
zweite Gleichung sagt, dass bis auf eine additive Konstante $s$
und $t$ übereinstimmen, wir können also ohne weiteres $t=s$
wählen. Damit bleibt nur noch die erste Gleichung, welche ebenfalls
einfach zu lösen ist, wir erhalten als Lösung
\begin{equation}
\begin{aligned}
x(s)&=-as+x_0\\
t(s)&=s\\
z(s)&=z_0
\end{aligned}
\label{hyperbolisch:quasi1}
\end{equation}

Der zweite Parameter in der Lösung des Cauchy-Problems ist der
Parameter entlang der Anfangskurve, in unserem Fall ist dies $x_0$,
denn die Anfangskurve wird durch die Werte entlang der $x$-Achse
gegeben. Wir haben also genauer die Gleichungen
\begin{equation}
\begin{aligned}
x(s,x_0)&=-as+x_0\\
t(s,x_0)&=s\\
z(s,x_0)&=u_0(x_0)
\end{aligned}
\label{hyperbolisch:quasi2}
\end{equation}

Der zweite Schritt des Lösungsverfahrens von Kapitel~\ref{chapter-geometrie}
besagt, dass die Variablen $s$ und $x_0$ aus den Gleichungen
(\ref{hyperbolisch:quasi2})
\begin{equation}
u(x(s,x_0), t(s,x_0))=z(s,x_0)
\label{hyperbolisch:quasi3}
\end{equation}
zu eliminieren seien.
Aber aus der zweiten Gleichung 
(\ref{hyperbolisch:quasi2})
folgt $s=t$, und aus
ersten Gleichung von
$x_0=at+x$. Setzt man dies zusammen mit der dritten Gleichung von
(\ref{hyperbolisch:quasi2}) in 
(\ref{hyperbolisch:quasi3}) ein, erhält man
\[
u(x, t)=u_0(x_0)=u_0(at+x).
\]
Als Funktion von $x$ ist
$u(x,t)$ als eine um $at$ nach links verschobene Kopie von $u_0$.
Die Lösung von (\ref{wellelinks})
ist also eine mit Geschwindigkeit $a$ nach links
laufende Welle.

Analog liefert die Gleichung (\ref{wellerechts}) eine mit Geschwindigkeit
$a$ nach rechts laufende Kopie von $u_0$. Da die Wellengleichung linear ist,
ist auch jede Linearkombination dieser beiden Lösungen eine Lösung der
Differentialgleichung. Sind $u_+(x)$ und $u_-(x)$ zwei beliebige Funktionen
derart, dass $u_+(x)+u_-(x)=u_0(x)$, dann ist
\begin{equation}
u(x,t)=u_+(x+at)+u_-(x-at)
\label{dalembertloesung}
\end{equation}
eine Lösung der Wellengleichung mit der Anfangsbedingung (\ref{welleanfang}).

\subsection{Anfangsgeschwindigkeit}
Die Lösung der Wellengleichung ist aber erst durch eine weitere
Anfangsbedingung der Form
\begin{equation}
\partial_tu(x,0)=v_0(x)\label{welleanfangdt}
\end{equation}
vollständig bestimmt.
Wenn sich die Lösung in der Form (\ref{dalembertloesung}) schreiben lassen
soll, muss gelten
\begin{align*}
u_+(x)+u_-(x)&=u_0(x)\\
au_+'(x)-au_-'(x)&=v_0(x)
\end{align*}
Ist $V_0$ eine Stammfunktion von $\frac1av_0$, also $V_0'=\frac1av_0$,
dann folgt aus der zweiten Gleichung, dass 
\[
u_+(x)-u_-(x)=V_0(x)+c.
\]
Diese Gleichungen kann man nach $u_+$ und $u_-$ auflösen:
\begin{align*}
u_+(x)&=\frac12(u_0(x)+V_0(x)+c)\\
u_-(x)&=\frac12(u_0(x)-V_0(x)-c)
\end{align*}
und damit
\begin{align}
u(x,t)
&=
\frac12\bigl(u_0(x+at)+V_0(x+at)+c\bigr)+\frac12\bigl(u_0(x-at)-V_0(x-at)-c\bigr)
\notag
\\
&=
\frac12\bigl(u_0(x+at)+V_0(x+at)\bigr)+\frac12\bigl(u_0(x-at)-V_0(x-at)\bigr)
\label{hyperbolisch:dalembert}
\end{align}
Die Lösung (\ref{hyperbolisch:dalembert})
heisst die d'Alembert-Lösung der Wellengleichung.
\index{d'Alembert}
\index{d'Alembert-L\ösung}

Natürlich sind die einzelnen Summanden Lösungen der Wellengleichung, aber auch
die Anfangsbedingungen sind so erfüllt:
\begin{align*}
u(x,0)&=u_0(x)\\
\partial_tu(x,0)&=\frac12\bigl(au_0'(x+at)+v_0(x)-au_0'(x)+v_0(x)\bigr) =v_0(x)
\end{align*}
Somit lässt sich die Wellengleichung in der Ebene einfach durch Finden einer
Stammfunktion für eine Funktion entlang der $x$-Achse konstruieren.
Das anfangs des Abschnittes
gestellte Problem entspricht dem Fall $v_0(x)=0$.

