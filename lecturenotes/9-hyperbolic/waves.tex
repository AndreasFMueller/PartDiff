%
% waves.tex -- 
%
% (c) 2019 Prof Dr Andreas Mueller
%
\section{Wave equation one dimension}
\rhead{One dimensional wave equation}
The wave equation in the plane is
\[
\partial_t^2u-a^2\partial_x^2u=0,
\]
which we want to solve in the domain
\[
\Omega = \{(x,t) \,|\, t > 0\}.
\]
The parameter $a$ has the dimension of velocity, it is the
speed of the wave along the $x$ axis.

\subsection{Constant velocity}
Assume for the time being that $a$ is a constant.
then the equation can also be written as
\begin{align*}
(\partial_t -a\partial_x)(\partial_t+a\partial_x)u&=0
\\
\text{oder}&
\\
(\partial_t +a\partial_x)(\partial_t-a\partial_x)u&=0.
\end{align*}
Thus solutions of the first order equations
\begin{align}
\partial_t u-a\partial_x u&=0
\label{wellelinks}
\\
\partial_t u+a\partial_x u&=0
\label{wellerechts}
\end{align}
are automatically solutions of the wave equation.

\subsection{Solutions of the first order equations}
We want to solve the wave equation for initial conditions of the kind
\begin{equation}
u(x,t)=u_0(x),\qquad x\in\mathbb R.
\label{welleanfang}
\end{equation}
To this end we solve the two first order equations
\eqref{wellelinks} and \eqref{wellerechts} with
those some initial conditions.

Both differential equations are quasilinear equations of first order
for which the method of characteristics can give a solution.
We first determine the characteristics.
Their differential equation is
\begin{align*}
\frac{dx}{ds}&=-a
\\
\frac{dt}{ds}&=1
\\
\frac{du}{ds}&=0.
\end{align*}
The third equation sais that $u$ does not depend on the parameter $s$.
The second equation then says that $s$ is equal to $t$ up to an additive
constant.
Finally, we can solve the first equation by integrating with respect to $s$.
In total, we get the solution
\begin{equation}
\begin{aligned}
x(s)&=-as+x_0\\
t(s)&=s\\
z(s)&=z_0
\end{aligned}
\label{hyperbolisch:quasi1}
\end{equation}
The parameter $x_0$ is the parameter along the characteristic curve
and $z_0$ is the value of $u$ at the initial point $(x_0,0)$.
So we get the solution
\begin{equation}
\begin{aligned}
x(s,x_0)&=-as+x_0\\
t(s,x_0)&=s\\
z(s,x_0)&=u_0(x_0)
\end{aligned}
\label{hyperbolisch:quasi2}
\end{equation}
The second step in the solution algorithm from chapter~\ref{chapter-geometrie}
says that the variables $s$ and $x_0$ have to be eliminated
from~\eqref{hyperbolisch:quasi2}.
We already have identified $s$ as $t$, so we only need to identify
$x_0$ as $x_0=x  at$ and substitute that into the last equation to get
\begin{equation}
u(x,t) = u_0(x+at)
\label{hyperbolisch:quasi3}
\end{equation}
This solution describes a solution of \eqref{wellelinks}
is a wave that travels with velocity $a$ to the left.

Analogously the equation \eqref{wellerechts} gives a wave that
travels to the right.
Combining the two waves allows us to write the most general solution
of the wave equation as
\begin{equation}
u(x,t)=u_+(x+at)+u_-(x-at)
\label{dalembertloesung}
\end{equation}
The functions $u_+$ and $u_-$ have to be chosen in such a way as to
match the initial conditions.

\subsection{Initial velocity}
Initial values alone do not uniquely determine the solutions of a wave
equation, an additional initial condition of the form
\begin{equation}
\partial_tu(x,0)=v_0(x)\label{welleanfangdt}
\end{equation}
is needed.
Using the solution in the form \eqref{dalembertloesung} then leads
to the following equations for $u_+(x)$ and $u_-(x)$
\begin{align*}
u_+(x)+u_-(x)&=u_0(x)\\
au_+'(x)-au_-'(x)&=v_0(x)
\end{align*}
If $V_0$ is an antiderivative of $\frac12av_0$, or in other
words $V_0'=\frac1av_0$, then from the second equation we derive that
\[
u_+(x)-u_-(x)=V_0(x)+c.
\]
With this equation we can no solve for $u_+$ and $u_-$ and find:
\begin{align*}
u_+(x)&=\frac12(u_0(x)+V_0(x)+c)\\
u_-(x)&=\frac12(u_0(x)-V_0(x)-c)
\end{align*}
Thus the solution of the wave equation is
\begin{align}
u(x,t)
&=
\frac12\bigl(u_0(x+at)+V_0(x+at)+c\bigr)+\frac12\bigl(u_0(x-at)-V_0(x-at)-c\bigr)
\notag
\\
&=
\frac12\bigl(u_0(x+at)+V_0(x+at)\bigr)+\frac12\bigl(u_0(x-at)-V_0(x-at)\bigr)
\label{hyperbolisch:dalembert}
\end{align}
The solution \eqref{hyperbolisch:dalembert} is calledt the
{\em d'Alembert solution} of the wave equation.

Both terms in the d'Alembert solution solve the wave equation, but it
is also easy to verify that initial conditions are satisfied:
\begin{align*}
u(x,0)&=u_0(x)\\
\partial_tu(x,0)&=\frac12\bigl(au_0'(x+at)+v_0(x)-au_0'(x)+v_0(x)\bigr) =v_0(x)
\end{align*}
In particular, finding a solution for the wave equation is as easy as
finding an antiderivative of a function along the $x$ axis.
The problem posed initially in this section is the case
$v_0(x)=0$.


