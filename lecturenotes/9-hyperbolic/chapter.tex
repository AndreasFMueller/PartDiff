%
% chapter.tex -- 
%
% (c) 2008 Prof Dr Andreas Mueller
%
\chapter{Hyperbolic partial differential equations
\label{chapter-hyperbolisch}}
\lhead{Hyperbolic equations}
\rhead{}
In this chapter we discuss the wave equation as a very prominent
example of a hyperbolic equation.
Just like in the parabolic case, the time coordinate has special
significance.
The wave equation also is a kind of equation of motion, but of second
order, and thus more similar to the equations we are used to from
mechanics.

While a change in initial conditions or boundary conditions 
influences the solution everywhere, the wave equation behaves
very differently in that such changes propagate with finite speed.
For each point in the domain there are points that can be affected
by a change of the value in the point and others that don't notice
the change.
The interfaces between these domains are an important aspect that
helps understand the solutions.

To get closer to a solution, we start with the one dimensional case
and we will find a solution as a superposition of waves that travel in
both directions.
We then embark on a study how initial conditions influence the solutions.
Asking which initial conditions lead to a well posed problem then
leads us to the concept of characteristics.

%
% separation.tex -- XXX
%
% (c) 2019 Prof Dr Andreas Mueller
%
\lhead{Separation der Variablen}
\rhead{}
\chapter{Separation der Variablen\label{chapter-separation}}
\index{Stroboskop}
\index{Eigenschwingung}
\index{stehende Welle}
\begin{figure}
\begin{center}
\includegraphics[width=0.8\hsize]{../common/graphics/stringvibrlarge-10-06-06.jpg}
\end{center}
\caption{Schwingende Saite (Bild von A.~Davidhazy, http://people.rit.edu/andpph/)
\label{separation:schwingendesaite}}
\end{figure}
Beleuchtet man eine schwingende Saite mit einem Stroboskop mit der
Frequenz der Eigenschwingung, scheint die Saite stillzustehen. 
In periodischen Zeit\-ab\-stän\-den sieht die Lösung der Wellengleichung
also gleich aus.
Misst man andererseits die Auslenkung der Saite
an einer Stelle in Abhängigkeit von der Zeit, beobachtet man
eine harmonische Schwingung, die sich mit Hilfe von $\sin$- und
$\cos$-Funktionen beschreiben lässt. Man kann also vermuten,
dass die Lösung der Wellengleichung der schwingenden Saite
ein Produkt
\[
u(x,t)=X(x)\cdot\sin\omega t\quad\text{oder}\quad X(x)\cdot\cos\omega t.
\]
ist. Ziel dieses Kapitels ist, diese Idee zu einem Lösungsverfahren
weiterzuentwickeln und auf einige Differentialgleichungen anzuwenden.

%
% ode.tex -- ordinary differential equations
%
% (c) 2019 Prof Dr Andreas Mueller
%
\section{Separation of Variables for ordinary differential equations}
The differential equation
\begin{equation}
y'-xy=0
\label{separation:ode}
\end{equation}
can be solved using separation of variables:
\begin{align*}
\frac{dy}{dx}&=xy\\
\frac1y\,dy&=x\,dx\\
\int\frac1y\,dy&=\int x\,dx\\
\log|y|&=\frac12x^2+C\\
y&=y_0e^{\frac12x^2}.
\end{align*}
The method is based on the idea to have only a single variable
on either side of the equation.
Even the derivative is formally decomposed as a fraction $dy/dx$, which
isn't really meaningful but justified by the success of the method.
This reduces the solution of the differential equation to the computation
of two integrals, i.~e.~the solution of the particularly simple
differential equation $y'=f(x)$ with solution $y(x)=\int f(x)\,dx$..
This solution also tells us what kind of initial conditions we need
for the solution to be uniquely determined.
If $y_0$ is the value of the solution at $x=x_0$, then the solution is
\[
y(x)=\int_{x_0}^xf(\xi)\,d\xi + y_0.
\]

Of course the process will not be quite as simple when we transition
to partial differential equations.
In particular the suspicious operation to separate the differentials
of $dy/dx$ has no counterpart for more than one independent variable.
However, depending on the shape of the domain, it might still be possible
to separate the variables and to transform the equation into the form
\[
\text{functions/derivatives only involving $x$}
=
\text{functions/derivatives only involving $y$}.
\]
Since the left hand side only depends on $x$ while the right hand side
depends only on $y$, both sides must be constant.
So we can write
\begin{align*}
\text{functions/derivatives only involving $x$} &= \mu
\\
\text{functions/derivatives only involving $y$} &= \mu
\end{align*}
with a new constant $\mu$ that has to be determined later.
This means that we have succeeded to reduce the partial differential
equation to two ordinary differential equations.
Since we know ``all'' about ordinary differential equations, we should
now be in a position to solve the partial differential equation and
to determine what kinds of boundary values are needed to make the
solution unique.


%
% idea.tex -- idea of the method
%
% (c) 2008 Prof Dr Andreas Mueller
%

\section{Idea of the method}
In applications one often has some indications from the application
domain what the solution function will most probably look like, or one
is looking for a very particular type of solution.
In partuclar, it may be known how the solution depends on one of the
variables up to a factor depending only on the other variable.
In these situations one can try to write the solution as a product
or sum of functions that depend on only one variable.

Even if one knows nothing about the solution, one can still try such
an {\em ansatz}, as the following example tries to illustrate.
Let's attempt to solve the partial differetnial equation
\begin{equation}
\frac1x
\frac{\partial u}{\partial x}
+
\frac1y
\frac{\partial u}{\partial y}
=\frac1{y^2}
,
\qquad x>1, y>1,
\label{separation:beispiel1}
\end{equation}
ignoring the boundary conditions for the time being.
We try to represent the solution as a sum of two functions which depend
on one of $x$ and $y$ only.
\begin{equation}
u(x,y)=X(x)+Y(y)
\quad\Rightarrow\quad
\begin{cases}
\quad{\displaystyle \frac{\partial u}{\partial x}}&=X'(x)\\
\\
\quad{\displaystyle \frac{\partial u}{\partial y}}&=Y'(y)\\
\end{cases}
\label{separation:beispiel1:ansatz}
\end{equation}
Substituting this into the differential equation
(\ref{separation:beispiel1})
gives the new equation
\[
\frac{X'(x)}{x}+\frac{Y'(y)}{y}=\frac1{y^2}
\]
or
\begin{equation}
\frac{X'(x)}{x}
=\frac1{y^2}
-\frac{Y'(y)}{y}.
\label{separation:beispiel1:separiert}
\end{equation}
The Form (\ref{separation:beispiel1:separiert}) has a distinct property:
the variable $x$ only appears on the left side, the variable $y$ only on
the right.
If we fix some value $y$, the right hand side cannot change, so the left
hand side must not depend on $x$.
Conversely, if we fix $x$, then the left side cannot change any more,
and thus the right hand side cannot change either.
We conclude that both sides must be the same constant.
Calling this constant $k$ we find the two ordinary differential equations
\begin{align}
\frac{X'(x)}{x}&=k
&
k&=\frac1{y^2}-\frac{Y'(y)}{y}
\label{separation:beispiel1:separiertedgl}
\end{align}
for $X(x)$ and $Y(y)$.

The differential equation for $X$ is easy to solve:
\begin{align*}
X'(x)&=kx\quad\Rightarrow\quad X(x)=
\frac12kx^2+C_x.
\end{align*}
The right equation is only slightly more complicated:
\begin{align*}
Y'(y)=\frac1y-ky
\quad\Rightarrow\quad
Y(y)=\int\frac1y-ky\,dy=
\log y-\frac12ky^2+C_y.
\end{align*}
We can now combine these functions into a solution of the initial
differential equation:
\begin{equation}
u(x,y)=
\frac12kx^2+
\log y-\frac12ky^2+C.
\label{separation:beispiel1:loesung}
\end{equation}
By varying the parameters $k$ and $C$, formula
(\ref{separation:beispiel1:loesung}) gives an infinite family of
solutions of the partial differential equation.

The values of the constants need to be determined by boundary conditions
in a manner to be studied later.

Let's summarize the method so far:
\begin{enumerate}
\item
Choose an {\em ansatz} from functions that depend from disjoint
sets of variables.
\item
Substitute into the partial differential equation and separate terms
involving the separate sets of variables.
The two sides of the equation depend on disjoint sets of variables
und must therefore be constant.
\item 
Split the equation into two coupled equations, each with a
different set of independent variables.
\item
Solve each equation individually.
\item
Put solutions together using the boundary conditions.
\end{enumerate}
As may suspected, the problem most of the time is not the solution
of the individual equations but rather the last step.
In the following sections we want to illustrate how this can be
done in a variety of examples.


%
% lpde.tex -- why linear pdes?
%
% (c) 2019 Prof Dr Andreas Mueller
%
\section{Separation for linear partial differential equations}
The base idea of the separation method produces a family of functions
that depend an some integration and separation constants.
On the boundary we are usually given some arbitrary functions.
In general it will be impossible to tune these few constants to
values that reproduce the boundary functions.
This basic version of the separation method thus is incapable of
solving a general partial differential equation due to the lack
of flexibility in the set of solutions found.

This problem changes if different solutions can be combined into new ones.
Linear combinations of solutions introduce a large set of additional
parameters to tune the solution.
In particular, Fourier theory shows that linear combinations of
basic functions can be tuned to approximate just about any periodic
function.
However, linear combinations of solutions are in general no longer
solutions the equation, except for linear partial differential
equations.
This is expressed in the following theorem.

\begin{satz}
If $u_1$ and $u_2$
are solutions of a homogeneous linear partial differential equation,
then $u_1+u_2$ and $\lambda u_1$ with $\lambda\in\mathbb R$ are also
solutions.
\end{satz}

\begin{proof}
The differential equation
\[
F(x_1,\dots,x_2,u,\frac{\partial u}{\partial x_1},\dots)=0
\]
is linear in $u$ and its derivatives, so we can expand sums and
pull factors from inside $F$:
\begin{align*}
F(x_1,\dots,x_2,u_1+u_2,\frac{\partial u_1}{\partial x_1}+\frac{\partial u_2}{\partial x_1},\dots)
&=
F(x_1,\dots,x_2,u_1,\frac{\partial u_1}{\partial x_1},\dots)
\\
&+
F(x_1,\dots,x_2,u_2,\frac{\partial u_2}{\partial x_1},\dots)=0
\\
F(x_1,\dots,x_2,\lambda u_1,\frac{\partial \lambda u_1}{\partial x_1},\dots)
&=
\lambda
F(x_1,\dots,x_2,u_1,\frac{\partial u_1}{\partial x_1},\dots)
=0
\end{align*}
Thus linear combinations of $u_1$ and $u_2$ are solutions too.
\end{proof}

Linear cominations allow us to first find as many different solutions
$u_1$, $u_2$, $u_3,\dots$ and then to use suitable coefficients $a_k$
to combine them into a solution
\[
u(x)=\sum_{i=1}^\infty a_ku_k
\]
that also satisfies the boundary conditions.

Inhomogeneous linear partial differential equations can be solved using
this method too.
As pointed out in chapter~\ref{chapter:terminology-and-notation},
we first have to find a particular solution $u_p$ which solves the
inhomogeneous partial differential equation independently
of any boundary conditions.
The separation method can be helpful for this too.
Then the problem is reduced to finding a solutions $u_h$ to the homogeneous
equations with boundary condition $g-u_p$ on $\partial\Omega$.
Using the separation method, we can build up $u_h$ as a linear combination
of solutions.

The coefficients $a_k$ for the linear combination often lead us into
Fourier theory or some generalization of it.
Such partial solutions often have immediate physical significance.
In mechanical or electrical engineering they appear as vibration modes,
in quantum mechanics as energy states.



%
% membrane.tex -- XXX
%
% (c) 2019 Prof Dr Andreas Mueller
%
\section{Schwingende rechteckige Membran}
\rhead{Rechteckige Membran}
\index{Membran!rechteckig}
Wir betrachten die Schwingung einer rechteckigen Membran, die am Rande
des Gebietes
\[
R=\{(x,y)\,|\,0\le x\le a,0\le y\le b\} =(0,a)\times(0,b)
\]
eingespannt ist. Zur Zeit $t=0$ sei die Form der Membran durch die
Funktion $f(x,y)$ gegeben.
Für beliebige Zeit $t\ge 0$ wird sie beschrieben durch eine Funktion $u(x,y,t)$,
welche der Differentialgleichung
\[
\frac1{c^2}\frac{\partial^2u}{\partial t^2}=\frac{\partial^2u}{\partial x^2}+\frac{\partial^2u}{\partial y^2}
\]
genügt mit den Anfangsbedingungen
\begin{align*}
u(x,y,0)&=f(x,y)\quad\forall 0\le x\le a,0\le y\le b,
\\
\frac{\partial}{\partial t}u(x,y,0)&=g(x,y)\quad\forall 0\le x\le a,0\le y\le b
\end{align*}
und den Randbedingungen
\begin{align*}
u(0,y,t)&=0&u(a,y,t)&=0&\forall t\ge 0,0\le y\le b,\\
u(x,0,t)&=0&u(x,b,t)&=0&\forall t\ge 0,0\le x\le a.
\end{align*}

\subsection{Separation der Zeit}
\index{Separation}
Nach der in der Einleitung motivierten Idee suchen wir Lösungen also
Produkt einer Funktion $T(t)$, die nur von der Zeit abhängt, und einer Funktion
$\varphi(x,y)$, welche nur vom Ort abhängt, also
\[
u(x,y,t)=T(t)\cdot\varphi(x,y).
\]
Leider kann ein einzelnes solches Produkt nicht alle Anfangsbedingungen
erfüllen. Wäre dies nämlich möglich, müsste $\varphi(x,y)\sim f(x,y)$
sein und alle Teile der Membran würden im Gleichtakt hin und her schwingen.
Simulationen oder physikalische Experimente zeigen aber, dass es
Anfangsbedingungen gibt, bei denen die Teile der Membran gegenläufig
schwingen.

Anderseits, muss die Lösung auf jeden Fall die Randbedingung erfüllen,
es muss also gelten
\begin{align*}
\varphi(0,y)&=0&\varphi(a,y)&=0&0\le y\le b\\
\varphi(x,0)&=0&\varphi(x,b)&=0&0\le x\le a
\end{align*}
Setzen wir diesen Ansatz für $u$ in der Wellengleichung ein,
erhalten wir
\[
\frac1{c^2}T''(t)\varphi(x,y)=T(t)\left(
\frac{\partial^2\varphi}{\partial x^2}
+
\frac{\partial^2\varphi}{\partial y^2}
\right)
\]
Wir suchen eine Funktion $u$, die nicht identisch verschwindet,
es gibt also einige Zeitpunkte $t$ und Orte $(x,y)$, an denen $T(t)$
und $\varphi(x,y)$ nicht verschwinden. An diesen Stellen kann man die
Gleichung umformen in
\begin{equation}
\frac1{c^2}\frac{T''(t)}{T(t)}
= \frac1{\varphi(x,y)}\left( \frac{\partial^2\varphi}{\partial x^2}
+ \frac{\partial^2\varphi}{\partial y^2} \right)
\label{separiertMembran}
\end{equation}
Die rechte Seite hängt nur
vom Ort ab, darf sich also nicht ändern, wenn man die Zeit $t$ variert.
Als Funktion der Zeit muss die linke Seite eine Konstante sein,
es gibt also ein $k$ mit der Eigenschaft
\[
\frac1{c^2}\frac{T''(t)}{T(t)}=k
\qquad\Leftrightarrow\qquad
T''(t)=k T(t).
\]
Diese gewöhnliche Differentialgleichung hat Lösungen der Form 
$e^{\pm\sqrt{k}t}$ für positives $k$. Für negatives $k$ sind $\sin\sqrt{k}t$ 
und $\cos\sqrt{k}t$ Lösungen.
Aus physikalischer Sicht sind nur Lösungen mit Schwingungscharakter sinnvoll,
wir können daher annehmen, dass $k<0$.
Ein solches $k$ lässt sich in der Form $k=-\lambda^2$ schreiben.
Es gibt also ein $\lambda$ mit der Eigenschaft
\[
\frac1{c^2}\frac{T''(t)}{T(t)}=-\lambda^2
\]
oder
\[
T''(t)=-c^2\lambda^2 T(t).
\]
Dies ist eine gewöhnliche Differentialgleichung zweiter Ordnung, welche mit
bekannten Methoden gelöst werden kann.
Die allgemeine Lösung dieser Gleichung ist von der Form
\[
A\cos c\lambda t+B\sin c\lambda t.
\]

\subsection{Reduktion auf ein Eigenwertproblem}
\index{Eigenwertproblem}
Die linke Seite von (\ref{separiertMembran}) hängt nur von der Zeit ab, darf sich
also nicht ändern, wenn man $x$ oder $y$ variert. Als Funktion des Ortes
muss die rechte Seite also ebenfalls konstant sein:
\begin{align*}
\frac1{\varphi(x,y)}\left(
\frac{\partial^2\varphi}{\partial x^2}
+
\frac{\partial^2\varphi}{\partial y^2}
\right)&=-\lambda^2\\
\frac{\partial^2\varphi}{\partial x^2}
+
\frac{\partial^2\varphi}{\partial y^2}
=\Delta\varphi
&=-\lambda^2
\varphi(x,y)
\end{align*}
Die gesuchte Funktion $\varphi$ ist also ein Eigenvektor des linearen
Operators $\Delta$ zum Eigenwert $-\lambda^2$.
Nur die Eigenwerte des Operator $\Delta$ kommen also für die
Zeitabhängigkeitsgleichung in Frage.

\subsection{Separation von $x$ und $y$}
\index{Separation}
Für das Eigenwertproblem können wir erneut den Separationsansatz
\[
\varphi(x,y)=X(x)\cdot Y(y)
\]
versuchen.
Einsetzen in die Differentialgleichung ergibt
\begin{align*}
X''(x)Y(x)+X(x)Y''(y)&=-\lambda^2 X(x)Y(y)
\\
\frac{X''(x)}{X(x)}+\frac{Y''(y)}{Y(y)}&=-\lambda^2
\end{align*}
Jeder der Brüche hängt nur von jeweils einer Variable ab, was nur
möglich ist, wenn beide Terme konstant sind. Damit ist das Problem
reduziert auf zwei Gleichungen
\begin{align*}
X''(x)&=-\lambda_1^2X(x)\\
Y''(y)&=-\lambda_2^2Y(y)\\
\lambda_1^2+\lambda_2^2&=\lambda^2
\end{align*}
Die allgemeinen Lösungen dieser Gleichungen, die auch die Randbedingung
bei $x=0$ bzw.~$y=0$ erfüllt, sind
\begin{align*}
X(x)&=A\sin \lambda_1x\\
Y(y)&=B\sin \lambda_2y
\end{align*}
Die Randbedingungen für $x=a$ und $y=b$ können nur erfüllt werden,
wenn $\lambda_1a$ und $\lambda_2b$ Vielfache von $\pi$ sind, also
\[
\lambda_1=\frac{k\pi}a
\qquad
\text{und}
\qquad
\lambda_2=\frac{l\pi}b
\]
Die möglichen Werte von $\lambda$ sind also
\[
\lambda_{kl}^2=\left(\frac{k^2}{a^2} + \frac{l^2}{b^2}\right)\pi^2,\qquad k,l\in\mathbb Z
\]
Damit kann man jetzt die allgemeine Lösung des Schwingungsproblems aus den
Teillösungen
\[
\varphi_{kl}(x,y)=\sin \frac{k\pi}{a}x\sin\frac{l\pi}{b}y
\]
für das Eigenwertproblem
und den Teillösungen
\[
u_{kl}(x,y,t)
=
(A_{kl}\cos c\lambda_{kl} t+
B_{kl}\sin c\lambda_{kl} t)
\sin \frac{k\pi}{a}x\sin\frac{l\pi}{b}y
\]
für das zeitabhängige Problem
zu einer allgemeinen Lösung
\begin{equation}
u(x,y,t)=\sum_{k,l}
(A_{kl}\cos c\lambda_{kl} t+
B_{kl}\sin c\lambda_{kl} t)
\sin \frac{k\pi}{a}x\sin\frac{l\pi}{b}y
\label{allgemeineloesung}
\end{equation}
zusammensetzen.

\subsection{Anfangsbedingungen}
\index{Anfangsbedingungen}
Die allgemeine Lösung muss jetzt auch noch die Anfangsbedingung erfüllen:
\begin{align*}
\sum_{k,l}A_{kl}
\sin \frac{k\pi}{a}x\sin\frac{l\pi}{b}y&=f(x,y)\\
\sum_{k,l}B_{kl}c\lambda_{kl}
\sin \frac{k\pi}{a}x\sin\frac{l\pi}{b}y&=g(x,y)\\
\end{align*}
Die Koeffizienten $A_{kl}$ und $B_{kl}$ können in einfachen Fällen mit
Koeffizientenvergleich und im Allgemeinen mit Hilfe der Theorie
der Fourierreihen berechnet werden.
\index{Fourierreihe}


%
% disk.tex -- XXX
%
% (c) 2019 Prof Dr Andreas Mueller
%
\section{Kreisgebiet}
\rhead{Kreisgebiet}
\index{Kreisgebiet}
\index{Kreisscheibe}
In diesem Abschnitt betrachten wir eine Kreisscheibe
\[
G=\{(x,y)\in\mathbb R^2|x^2+y^2 < R\}
\]
mit Radius $R$ als Definitionsbereich. Da sich dieses Gebiet durch
eine Streckung um den Faktor $\frac1R$ immer auf einen Einheitskreis
abbilden lässt, können wir ohne Verlust an Allgemeinheit vorausetzen,
dass $R=1$ ist.

Ein Kreisgebiet tritt zum Beispiel beim Problem auf, die Schwingungen
einer kreisförmigen Membran zu berechnen, wie sie bei einer Kesselpauke
vorkommen. Nach den Ergebnissen des ersten Kapitels suchen wir nach einer
Funktion $u$, welche auf $G$ die Gleichung
\[
\frac1{a^2}\frac{\partial^2 u}{\partial t^2}=\frac{\partial^2 u}{\partial x^2}+\frac{\partial^2 u}{\partial y^2}
\]
erfüllt. Wie bei der Schwingung der einer rechteckigen Platte
wird daraus mit dem Ansatz $ u(x,y,t)=u(x,y)\cdot T(t)$ ein
Eigenwertproblem:
\begin{align*}
T''(t)&=-a^2\lambda^2 T(t)\\
\Delta u(x,y)&=-\lambda^2u(x,y)
\end{align*}
Das Poissonproblem ist der Spezialfall $\lambda=0$.
\index{Poissonproblem}

\subsection{Polarkoordinaten}
\index{Polarkoordinaten}
Offenbar sind Polarkoordinaten speziell gut an das Problem angepasst, 
eine Randbedingung lässt sich zum Beispiel durch eine Funktion beschreiben,
welche nur vom Polarwinkel abhängt.
Eine schwingende kreisförmite Membran führt also auf die partielle
Differentialgleichung
\[
\frac{\partial^2u(r,\varphi)}{\partial t^2}=\Delta u(r,\varphi)
\]
mit der Randbedingung
\[
u(R,\varphi)=0,\qquad\varphi\in[0,2\pi],
\]
wobei wie oben $R$ der Radius der Membran ist.

Damit das Problem auf einem Kreisgebiet in Polarkoordinaten behandelt
werden kann,
brauchen wir einen Ausdruck für $\Delta u$ in Polarkoordinaten.
\begin{align}
x&=r\cos\varphi\\
y&=r\sin\varphi
\label{polarkoordinaten}
\end{align}
Um die Ableitungen nach $x$ und $y$ durch Ableitungen $\varphi$ und $r$ zu
ersetzen, leiten wir (\ref{polarkoordinaten}) nach $x$ und $y$ ab:
\begin{align*}
1&=
\frac{\partial r}{\partial x}\cos\varphi
-r\sin\varphi \frac{\partial\varphi}{\partial x}
&
0&=
\frac{\partial r}{\partial y}\cos\varphi
-r\sin\varphi \frac{\partial\varphi}{\partial y}
\\
0&=
\frac{\partial r}{\partial x}\sin\varphi
+r\cos\varphi \frac{\partial\varphi}{\partial x}
&
1&=
\frac{\partial r}{\partial y}\sin\varphi
+r\cos\varphi \frac{\partial\varphi}{\partial y}
\end{align*}
In Matrixschreibweise ist dies
\begin{align*}
\begin{pmatrix}1\\0\end{pmatrix}
&=
\begin{pmatrix}
\cos\varphi&-\sin\varphi\\
\sin\varphi&\cos\varphi
\end{pmatrix}
\begin{pmatrix}
\frac{\partial r}{\partial x}\\
r\frac{\partial \varphi}{\partial x}
\end{pmatrix}
&
\begin{pmatrix}0\\1\end{pmatrix}
&=
\begin{pmatrix}
\cos\varphi&-\sin\varphi\\
\sin\varphi&\cos\varphi
\end{pmatrix}
\begin{pmatrix}
\frac{\partial r}{\partial y}\\
r\frac{\partial \varphi}{\partial y}
\end{pmatrix}
\end{align*}
Die $2\times2$ Matrix ist eine Drehmatrix, die Inverse findet man, indem man
$\varphi$ durch $-\varphi$ ersetzt. Die Multiplikation auf der linken Seite
ergibt jeweils die erste bzw. zweite Spalte der Drehmatrix zum
Winkel $\varphi$:
\begin{align*}
\cos\varphi
&=\frac{\partial r}{\partial x}
&&
&
\sin\varphi
&=
\frac{\partial r}{\partial y}
&&
\\
-\sin\varphi
&=r\frac{\partial \varphi}{\partial x}
&\Rightarrow\quad
\frac{\partial\varphi}{\partial x}&=-\frac1r\sin\varphi
&
\cos\varphi
&=
r\frac{\partial\varphi}{\partial y}
&\Rightarrow\quad
\frac{\partial\varphi}{\partial y}&=\frac1r\cos\varphi
\end{align*}
Mit diesen Formeln können wir jetzt die höheren Ableitungen
von $u$ nach  $x$ und $y$ durch Ableitungen nach $r$ und $\varphi$
ersetzen.

Die partiellen Ableitungen von $\varphi$ nach $x$ und $y$ sind
\begin{align*}
\frac{\partial u}{\partial x}
&=
\frac{\partial u}{\partial r}
\frac{\partial r}{\partial x}
+
\frac{\partial u}{\partial\varphi}
\frac{\partial \varphi}{\partial x}
=
\frac{\partial u}{\partial r}
\cos\varphi
-
\frac{\partial u}{\partial\varphi}
\frac1r\sin\varphi
\\
\frac{\partial u}{\partial y}
&=
\frac{\partial u}{\partial r}
\frac{\partial r}{\partial y}
+
\frac{\partial u}{\partial\varphi}
\frac{\partial \varphi}{\partial y}
=
\frac{\partial u}{\partial r}
\sin\varphi
+
\frac{\partial u}{\partial\varphi}
\frac1r\cos\varphi
\end{align*}
Die zweiten Ableitungen sind
\begin{align*}
\frac{\partial^2u}{\partial x^2}
&=
\frac{\partial}{\partial r}
\left(
\frac{\partial u}{\partial r}
\cos\varphi
-
\frac{\partial u}{\partial\varphi}
\frac1r\sin\varphi
\right)
\frac{\partial r}{\partial x}
+
\frac{\partial }{\partial \varphi}
\left(
\frac{\partial u}{\partial r}
\cos\varphi
-
\frac{\partial u}{\partial\varphi}
\frac1r\sin\varphi
\right)
\frac{\partial\varphi}{\partial x}
\\
&=
\frac{\partial}{\partial r}
\left(
\frac{\partial u}{\partial r}
\cos\varphi
-
\frac{\partial u}{\partial\varphi}
\frac1r\sin\varphi
\right)
\cos\varphi
-
\frac{\partial }{\partial \varphi}
\left(
\frac{\partial u}{\partial r}
\cos\varphi
-
\frac{\partial u}{\partial\varphi}
\frac1r\sin\varphi
\right)
\frac1r\sin\varphi
\\
&=
\frac{\partial^2u}{\partial r^2} \cos^2\varphi
-
\frac{\partial^2u}{\partial r\partial\varphi} \frac1r\sin\varphi \cos\varphi
+
\frac{\partial u}{\partial\varphi} \frac1{r^2}\sin\varphi\cos\varphi
\\
&\quad
-
\frac{\partial^2u}{\partial\varphi\partial r}\frac1r \cos\varphi\sin\varphi
+\frac{\partial u}{\partial r}\frac1r\sin^2\varphi
+\frac{\partial^2u}{\partial\varphi^2}
\frac1{r^2}\sin^2\varphi
+\frac{\partial u}{\partial\varphi}\frac1{r^2}\cos\varphi\sin\varphi
\\
\frac{\partial^2u}{\partial y^2}
&=
\frac{\partial}{\partial r}
\left(
\frac{\partial u}{\partial r}
\sin\varphi
+
\frac{\partial u}{\partial\varphi}
\frac1r\cos\varphi
\right)
\frac{\partial r}{\partial y}
+
\frac{\partial}{\partial \varphi}
\left(
\frac{\partial u}{\partial r}
\sin\varphi
+
\frac{\partial u}{\partial\varphi}
\frac1r\cos\varphi
\right)
\frac{\partial \varphi}{\partial y}
\\
&=
\frac{\partial}{\partial r}
\left(
\frac{\partial u}{\partial r}
\sin\varphi
+
\frac{\partial u}{\partial\varphi}
\frac1r\cos\varphi
\right)
\sin\varphi
+
\frac{\partial}{\partial \varphi}
\left(
\frac{\partial u}{\partial r}
\sin\varphi
+
\frac{\partial u}{\partial\varphi}
\frac1r\cos\varphi
\right)
\frac1r\cos\varphi
\\
&=
\frac{\partial^2u}{\partial r^2}\sin^2\varphi
+\frac{\partial^2u}{\partial r\partial\varphi}\frac1r\cos\varphi\sin\varphi
-\frac{\partial u}{\partial\varphi}\frac1{r^2}\cos\varphi\sin\varphi
\\
&\quad
+
\frac{\partial^2u}{\partial\varphi\partial r}\frac1r\sin\varphi\cos\varphi
+\frac{\partial u}{\partial r}\frac1r\cos^2\varphi
+\frac{\partial^2u}{\partial \varphi^2}\frac1{r^2}\cos^2\varphi
-\frac{\partial u}{\partial \varphi}\frac1{r^2}\sin\varphi\cos\varphi
\end{align*}
\index{Laplace-Operator!in Polarkoordinaten}
Die Summe dieser zwei Terme ist die gesucht Darstellung des Laplace-Operators
in Polarkoordinaten:
\begin{align*}
\frac{\partial^2u}{\partial x^2}+\frac{\partial^2u}{\partial y^2}
&=
\frac{\partial^2u}{\partial r^2}
+\frac{\partial u}{\partial r}\frac1r
+\frac{\partial^2u}{\partial\varphi^2}\frac1{r^2}
\\
&=
\left(\frac1r\frac{\partial}{\partial r}r\frac{\partial}{\partial r}+\frac1{r^2}\frac{\partial^2}{\partial \varphi^2}\right)u
\end{align*}
Darstellungen des Laplace-Operators in weiteren Koordinatensystemen können
in jeder einigermassen vollständigen Formelsammlung gefunden werden.

\subsection{Separation der Ortsvariablen}
Die Lösung $u(r,\varphi)$ des Eigenwertproblems setzen wir wieder
als Produkt einer Funktion
$R(r)$
nur von  $r$ und einer Funktion $\Phi(\varphi)$ nur von $\varphi$ an.
Mit der im vorangegangenen Abschnitt gefundenen Formel für den Laplace-Operator
in Polarkoordinaten erhalten wir jetzt die Gleichungen
\begin{align*}
\Delta u=
\biggl(R''(r) + \frac1rR'(r)\biggr)\Phi(\varphi)
+\frac1{r^2}R(r)\Phi''(\varphi)&=-\lambda^2 R(r)\cdot\Phi(\varphi)\\
\frac{r^2R''(r)+rR'(r)}{R(r)}+\frac{\Phi''(\varphi)}{\Phi(\varphi)}
&=-\lambda^2 r^2
\\
\frac{r^2R''(r)+rR'(r)}{R(r)}+\lambda^2 r^2&=-\frac{\Phi''(\varphi)}{\Phi(\varphi)}
\end{align*}
Da die rechte Seite nur von $\varphi$ abhängt, die linke Seite aber nur von $r$,
müssen beide Seiten konstant sein, wir nennen diese Konstante $\mu^2$.
Damit sind die Variablen separiert:
\begin{align}
\Phi''(\varphi)+\mu^2\Phi(\varphi)&=0\label{phigleichung}\\
r^2R''(r)+rR'(r)+(\lambda^2 r^2-\mu^2)R(r)&=0\label{rgleichung}
\end{align}

\subsection{Lösung der separierten Differentialgleichungen}
Die allgemeine Lösung der Gleichung (\ref{phigleichung}) ist
\[
\Phi(\varphi)=A\cos\mu\varphi +B\sin\mu\varphi.
\]
Dies ist nur dann $2\pi$-periodisch, wenn $\mu$ eine ganze
Zahl ist, also $\mu=k$ mit $k\in\mathbb Z$.

Die Gleichung (\ref{rgleichung}) für $R$ bekommt damit die Form
\[
r^2R''(r)+rR'(r)+(\lambda^2 r^2-k^2)R(r)=0,
\]
sie ist verwandt mit der Besselschen Differentialgleichung.
Die Funktion $P(\varrho)=R(\varrho/\lambda)=R(r)$ hat die Ableitungen
\begin{align*}
\varrho P'(\varrho)&=\frac{\varrho}{\lambda}R'(\varrho/\lambda)=rR'(r)\\
\varrho^2 P''(\varrho)&=\frac{\varrho^2}{\lambda^2}R'(\varrho/\lambda)=r^2R''(r)
\end{align*}
und erfüllt somit die Besselsche Differentialgleichung
\[
\varrho^2P''(\varrho)+\varrho P'(\varrho)+(\varrho^2-k^2)P(\varrho).
\]
Lösungen der Besselschen Differentialgleichungen sind die Besselfunktionen
\[
P(\varrho)=J_{\pm k}(\lambda r)=R(r)
\]
Wie bei der rechteckigen Membran kann die allgemeine Lösung jetzt aus
den Teillösungen zusammengesetzt werden.


%
% initial.tex -- XXX
%
% (c) 2019 Prof Dr Andreas Mueller
%
\rhead{Anfangsbedingungen}
\section{Anfangsbedingungen}
In den bisherigen Beispielen haben wir Lösungen einer partiellen
Differentialgleichung gesucht und gefunden, welche bestenfalls einen
Teil der Randbedingungen erfüllt haben.
So haben wir zwar sichergestellt, dass die schwingende Membran eingespannt
bleibt, aber die Auslenkung der Membran zu Beginn haben wir ignoriert.

Um zu verstehen, wie die Anfangsbedingungen ebenfalls berücksichtig
werden können, betrachten wir die Wellengleichung
\[
\frac{\partial^2 u}{\partial t^2}=\frac{\partial^2 u}{\partial x^2}
\]
auf dem Gebiet $(t,x)\in\mathbb R\times [0,\pi]$
mit den Randbedingungen
\[
u(t,0)=u(t,\pi)=0.
\]
Wir verwenden den Separationsansatz
$u(t,x)=T(t)\cdot X(t)$, welcher uns wie früher dargestellt auf eine
Gleichung
\[
\frac{T''(t)}{T(t)}=\frac{X''(x)}{X(x)}=-\lambda^2
\]
führt.
Die Gleichung 
\[
X''(x)=-\lambda^2 X(x)
\]
hat als Lösung Linearkombinationen von Sinus- und Kosinusfunktionen
\[
X(x)=A\cos\lambda x+B\sin\lambda x.
\]
Damit die Anfangsbedingung am linken Rand erfüllt ist, muss $A=0$
sein. Am rechten Rand bleibt daher nur $B\sin\lambda \pi$, und wir
müssen $B\ne 0$ annehmen, da sonst die ganze Lösung verschwinden
würde. $\sin\lambda \pi$ wird aber nur dann verschwinden, wenn
$\lambda$ eine ganze Zahl ist, also
\[
X_k(x)=B\sin kx, \quad 0<k\in\mathbb Z.
\]
Die dazu passende Lösung von $T''(t)=-k^2T(t)$ hat genau die
gleiche Form, so dass die allgemeine Lösung zum Wert $\lambda=k$
\[
u_k(t,x)=\sin kx\left(A_k\cos kt+B_k\sin kt\right)
\]
ist.

Diese Teillösungen $u_k(t,x)$ erfüllen bereits die Differentialgleichung
und die Randbedingungen. Noch nicht erfüllt werden die Anfangsbedingungen
zur Zeit $t=0$. Wir geben sie in der Form
\begin{align*}
u(0,x)&=f(x)\quad x\in[0,\pi]\\
\frac{\partial u}{\partial t}(0,x)&=g(x)\quad x\in[0,\pi]
\end{align*}
vor.

Wir suchen jetzt also eine Lösung in der Form
\[
u(t,x)=\sum_{k=1}^{\infty}
\left(A_k\cos kt+B_k\sin kt\right)
\sin kx,
\]
welche die Anfangsbedingung erfüllt. Durch Einsetzen erhält
man
\begin{align*}
\sum_{k=1}^{\infty}
A_k \sin kx
&=f(x)
\\
\sum_{k=1}^{\infty}
B_kk\sin kx
&=g(x)
\end{align*}
für $x\in[0,\pi]$.
Die Lösung $u(t,x)$ kann also vollständig bestimmt werden, indem man
die Anfangsbedingungen in eine Fourier-$\sin$-Reihe entwickelt. Sind
$\hat f(k)$ und $\hat g(k)$ die Fourier-Koeffizienten, wird die
vollständige Lösung
\[
u(t,x)
=
\sum_{k=1}^{\infty}(\hat f(k)\cos kt+\hat g(k)k\sin kt)\sin kx.
\]
Mit geeigneten Voraussetzungen an die Funktionen $f$ und $g$ werden
diese Reihen konvergieren.


%
% summary.tex -- XXX
%
% (c) 2019 Prof Dr Andreas Mueller
%
\section{Zusammenfassung: Separationsverfahren}
Aus diesen Beispielen lässt sich jetzt das allgemeine Prinzip 
ableiten. Gegeben ist eine partielle Differentialgleichung
beliebiger Ordnung mit unabhängigen Variablen $x_1,\dots,x_n$.
Ziel ist, die Differentialgleichung auf eine solche mit weniger
unabhängigen Variablen zu reduzieren. Sobald man die Reduktion
bis auf eine Variable geschafft hat, hat man die partielle
Differentialgleichung in gewöhnliche Differentialgleichungen
umgewandelt, typischerweise in Randwertprobleme,
die man mit gekannten Techniken lösen kann.

Da man am Schluss die Lösung aus den Teillösungen zusammensetzen
muss, die die separierten Gleichungen liefern, ist dieses Vorgehen
nur bei linearen PDGL sinnvoll. Wir gehen also im folgenden von
einer linearen PDGL aus.

Wir gehen also von einer Differentialgleichung für die Funktion
$u(x_1,\dots,x_n)$ aus, und wollen die Variable $x_1$ separieren.
Dazu geht man wie folgt vor.
\begin{enumerate}
\item Setzt die Lösung $u$ der Differentialgleichung in der
Form eines Produktes an:
\[
u(x_1,\dots,x_n)=X_1(x_1)u_1(x_2,\dots,x_n).
\]
\item Einsetzen des Ansatzes in die Differentialgleichung.
\item
Mit etwas Glück lassen sich die Terme, die
$X_1$ und $u_1$ enthalten trennen und auf verschiedene Seiten
des Gleichheitszeichens bringen.
Da die Lösung $u\equiv 0$ nicht interessant ist, kann man
zu diesem Zweck durch $u$ dividieren, die Gleichung muss
ausserhalb der Nullstellen von $u$ immer noch erfüllt sein.
Die Gleichung hat jetzt also die Form
\[
F(x_1,X_1,X_1',\dots,X_1^{(n)})
=
G(x_2,\dots,x_n,u_1,\partial_2u_1,\dots\partial_nu_n,\dots)
\]
\item
Da die linke Seite nur von $x_1$, die rechte nur von $x_2,\dots,x_n$
abhängt, müssen beide Konstant sein, wir haben also die ursprüngliche
PDGL in zwei Differentialgleichungen zerlegt:
\begin{equation}
\begin{aligned}
F(x_1, X_1,X_1',\dots, X_1^{(n)})&=k\\
G(x_2,\dots,x_n,u_1,\partial_2u_1,\dots\partial_nu_n,\dots)&=k
\end{aligned}
\label{separiert}
\end{equation}
wobei $k$ eine Konstante ist.
Dies sind zwei Differentialgleichungen, die erste ist eine
gewöhnliche Differntialgleichung, und falls $n>2$ ist die zweite
eine partielle Differentialgleichung, die unter Umständen noch
einmal mit dem gleichen Verfahren behandelt werden muss.
Gesucht werden alle Konstanten,
für welche beide Gleichungen eine Lösung haben.
\item Sind $X_1(k,x_1)$ und $u_1(k,x_2,\dots,x_n)$ Lösungen der
Gleichungen (\ref{separiert}), dann sind 
\[
u_k(x_1,\dots,x_n)=X_1(k,x_1)u_1(k,x_2,\dots,x_n)
\]
Lösungen der ursprünglichen PDGL. Die allgmeine Lösung ist daher
eine Summe
\[
u(x_1,\dots,x_n)=
\sum_{k}
a_k
u_k(x_1,\dots,x_n)=X_1(k,x_1)u_1(k,x_2,\dots,x_n),
\]
wobei die Summe über die möglichen $k$ zu erstrecken ist.
\item
Zur Erfüllung von Randbedingungen müssen jetzt die Koeffizienten
$a_k$ bestimmt werden, für die die Randtterme korrekt werden.
\end{enumerate}
Das Verfahren kann an zwei Stellen zusammenbrechen:
\begin{itemize}
\item In Schritt 3 wird vorausgesetzt, dass die Trennung in 
Terme, die $x_1$ enthalten  und solche, die $x_1$ nicht enthalten
möglich ist. Dies ist nicht automatisch der Fall, kann aber in
vielen praktisch wichtigen Fällen durch Wahl eines geeigneten
Koordinatensystems erreicht werden.
\item In Schritt 6 wird vorausgesetzt, dass die Randbedingungen
mit Hilfe der Randwerte der Teillösungen $u_k$ erfüllt werden
können. In den Beispielen in diesem Kapitel wurde dafür jeweils
die nicht triviale Fourier-Theorie benötigt. 
\end{itemize}



\section{Zusammenfassung: das Wichtigste in Kürze}
\begin{enumerate}
\item
Durch einen geeigneten Ansatz lassen sich einige partielle
Differentialgleichung in über Konstanten gekoppelte gewöhnliche
Differentialgleichungen zerlegen.
\item
Die Wahl des Lösungsansatzes wird durch die Geometrie des Gebietes
(Koordinatensystem) und die Art der Differentialgleichung bestimmt.
\item
Für lineare Differentialgleichung lassen sich aus den durch Separation
gefundenen Teil\-lö\-sungen Lösungen der ursprünglichen Differentialgleichung
linear kombinieren.
\item
Die zentrale Idee des Verfahrens ist, dass in einer Gleichung,
in der die eine Seite nur von $x$, die ander aber nicht von $x$
abhängt, beide Seiten konstant sein müssen.
\item
Im Falle von partiellen Differentialgleichungen zweiter Ordnung, die
sich häufig mit einem Produktansatz behandeln lassen, führt die
Separation das ursprüngliche Problem auf ein Eigenwertproblem mit
weniger Variablen.
\end{enumerate}

%
% tsunami.tex
%
% (c) 2011 Prof Dr Andreas Mueller, Hochschule Rapperswil
%

\section{Anwendung: Wellenausbreitung auf einer Kugel oder der Tsunami von 2011}
\index{Tsunami}
\index{Wellenausbreitung!auf der Kugeloberfl\"ache}
Am 11.~M"arz 2011 l"oste ein Erdbeben der St"arke 9 im japanischen
Meer, das Sendai Erdbeben, einen Tsunami aus, der grosse K"ustengebiete
\index{Sendai Erdbeben}
\index{Fukushima}
Japans verw"ustete, "uber
15000 Tote forderte und Unf"alle in mehreren Kernkraftwerken
ausl"oste, wovon der Unfall in Fukushima-Daichi mit einer
teilweisen Kernschmelze der schwerwiegendste war.
Tsunamis sind von Erdbeben ausgel"oste Wellen, die im offenen
Meer unscheinbar sind, aber wegen der mit kleiner werdender Wassertiefe
geringeren Ausbreitungsgeschwindigkeit in K"ustenn"ahe grosse
Amplituden erreichen k"onnen. Die Ausbreitung solcher Wellen
kann nat"urlich mit partiellen Differentialgleichungen modelliert
und berechnet werden. Die Abbildungen \ref{tsunamiausbreitung}
und \ref{tsunamienergie}
zeigt die mit einem Computer berechnete Ausbreitung des vom
Sendai-Erdbeben erzeugten Tsunami durch den Pazifik.
Dieses Modell ber"ucksichtigt offenbar die Topographie des
Meeresbodens.

Eine direkte Berechnung der Wellenausbreitung mit der bisher
gelernten Theorie ist nat"urlich nicht m"oglich, dazu m"usste
Topographie und K"ustenlinie des Pazifik im Detail bekannt
sein. Als vereinfachtes
Modell k"onnen wir jedoch versuchen, die Wellenausbreitung auf
einer Kugeloberfl"ache zu verstehen, dies entspricht einer 
kugelf"ormigen Erde, die mit einem Meer konstanter Tiefe bedeckt
ist.
\index{Meer}

\begin{figure}
\begin{center}
\includegraphics[width=\hsize]{graphics/sendainoaa}
\end{center}
\caption{Ausbreitung des vom Sendai-Erdbeben vom 11.~M"arz 2011 
ausgel"osten Tsunami durch den Pazifik nach einer Simulation der NOAA.
\index{Pazifik}
\index{NOAA}
Hawai und andere Inseln reduzieren die Wassertiefe und damit die
Ausbreitungsgeschwindigkeit und verz"ogern damit die Ausbreitung
der Welle. Ebenfalls deutlich beobachtbar ist die Abschattung 
der Welle durch grosse Hindernisse wie Neuseeland.
\index{Neuseeland}
\label{tsunamiausbreitung}}
\end{figure}

\begin{figure}
\begin{center}
\includegraphics[width=\hsize]{graphics/sendaienergy}
\end{center}
\caption{Amplitude des Tsunami vom 11.~M"arz 2011.
Man beachte, dass durch die Wahl der Kartenprojektion 
die Grosskreise, entlang derer sich die Wellen ausbreiten,
zu S-Kurven gebogen werden. In K"ustenn"ahe nimmt die
Amplitude wegen der abnehmenden Wassertiefe und der damit
reduzierten Ausbreitungsgeschwindigkeit zu.
\label{tsunamienergie}}
\end{figure}


\subsection{Koordinaten und Randbedingungen}
Es interessieren uns nur die von einem Punkt aus erzeugte Wellen,
so wie dies bei einem Erdbeben der Fall ist. Die L"osung muss
notwendigerweise rotationssymmetrisch sein um eine Achse, die
durch den Ausgangspunkt verl"auft. 

Als Koordinatensystem auf einer Kugel verwenden wir Kugelkoordinaten
$(r,\vartheta,\varphi)$. $\vartheta$ ist die geographische Breite
vom Nordpol gemessen, der auch gleich der Ausgangspunkt der
Welle sein soll. $\varphi$ ist die geographische L"ange, wir
suchen jedoch eine L"osung, die von der geographischen Breite
unabh"angig ist. Ebenso interessiert uns der Radius $r$ nicht,
da wir uns auf die Kugeloberfl"ache beschr"anken wollen, wir
setzen daher $r=1$.

Gesucht ist also eine Funktion $u(t,\vartheta)$, welche die
Anfangsbedingungen
\begin{align*}
u(0,\vartheta)&=F(\vartheta)\\
\frac{\partial}{\partial t}u(0,\vartheta)&=G(\vartheta)
\end{align*}
erf"ullen m"ussen.

\subsection{Wellengleichung auf der Kugeloberfl"ache}
Die Wellengleichung auf der Kugeloberfl"ache entsteht als 
Einschr"ankung der dreidimensionalen Wellengleichung
\[
\frac1{c^2} \frac{\partial^2}{\partial t^2}u =\Delta u.
\]
Wir nehmen an, dass die Einheiten so gew"ahlt worden sind,
dass $c=1$ in diesen Einheiten gilt (dies erreicht man zum
Beispiel, wenn man als L"angeneinheit die in einer Zeiteinheit
zur"uckgelegte Strecke verwendet).
Der Laplace-Operator muss in Kugelkoordinaten ausgedr"uckt werden,
\index{Laplace-Operator!in Kugelkoordinaten}
\[
\Delta u
=
\frac1{r^2} \frac{\partial}{\partial r}r^2\frac{\partial}{\partial r}u
+
\frac1{r^2\sin\vartheta}
\frac{\partial}{\partial\vartheta}
\sin\vartheta
\frac{\partial}{\partial\vartheta}
u
+
\frac1{r^2\sin^2\vartheta}\frac{\partial^2}{\partial\varphi^2}u
=
\frac1{\sin\vartheta}
\frac{\partial}{\partial\vartheta}
\sin\vartheta
\frac{\partial}{\partial\vartheta}
u
\]
Die Wellengleichung lautet jetzt also noch
\begin{equation}
\frac{\partial^2u}{\partial t^2}=
\frac1{\sin\vartheta}
\frac{\partial}{\partial\vartheta}
\sin\vartheta
\frac{\partial}{\partial\vartheta}
u=0.
\label{tsunami-gleichung}
\end{equation}

\subsection{Separation}
F"ur die L"osung der Wellengleichung (\ref{tsunami-gleichung}) machen
wir jetzt den "ublichen Separationsansatz:
\[
u(t,\vartheta)=T(t)\Theta(\vartheta),
\]
und setzen ihn in die Differentialgleichung ein:
\[
T''(t)\Theta(\vartheta)=
T(t)
\frac1{\sin\vartheta}
\frac{d}{d\vartheta}
\sin\vartheta
\frac{d}{d\vartheta}\Theta(\vartheta)
\]
Da wir eine L"osung suchen, die nicht "uberall verschwindet,
d"urfen wir annehmen, dass $T$ und $\Theta$ ausser an einzelnen
Punkten nicht verschwinden, dass wir also ``meistens'' durch
$T(t)\Theta(\vartheta)$ teilen d"urfen. Damit erreichen wir
die gew"unschte Trennung der Variablen:
\begin{equation}
\frac{T''(t)}{T(t)}
=
\frac1{\Theta(\vartheta)}
\frac1{\sin\vartheta}
\frac{d}{d\vartheta}
\sin\vartheta
\frac{d}{d\vartheta}\Theta(\vartheta)
\label{tsunami-separiert}
\end{equation}
Die linke Seite ist nur von $t$ abh"angig, die rechte nur von $\vartheta$,
diese Gleichung kann also nur erf"ullt sein, wenn beide seiten konstant
sind.  Wir erhalten also zwei Gleichungen
\begin{align}
T''(t)&=mT(t)
\label{tsunami:zeitabh}
\\
\frac1{\sin\vartheta}
\frac{d}{d\vartheta}
\sin\vartheta
\frac{d}{d\vartheta}\Theta(\vartheta)
&=m\Theta(\vartheta).
\label{tsunami:winkelabh}
\end{align}
Es ist jetzt also zu ermitteln, f"ur welche Werte von $m$ die beiden
Gleichungen L"osungen haben. Dann k"onnen f"ur diese Werte von $m$ 
L"osungen der partiellen Differentialgleichung zusammengebaut werden,
mit denen sich dann beliebige L"osungen durch "Uberlangerungen
erf"ullen lassen m"ussen.

\subsection{Zeitabh"angigkeit}
Die Zeitabh"angigkeit (\ref{tsunami:zeitabh}) ist eine gew"ohnliche
Schwingungsdifferentialgleichung.
Die L"osungen sollten Schwingungscharakter haben, was nur zutrifft, wenn
$m<0$ ist. Die allgemeine L"osung ist dann
\[
T_m(t)=a_m\cos\sqrt{-m}t+b_m\sin\sqrt{-m}t
\]

\subsection{Winkelabh"angigkeit}
Die Differentialgleichung (\ref{tsunami:winkelabh}) f"ur $\Theta$
ist in dieser Form etwas unhandlich.
Daher ersetzen wir $\Theta(\vartheta)$ durch eine
Funktion $y(x)$ mit Hilfe der Substitution $x=\cos\vartheta$.
Die Ableitung nach $\vartheta$ kann mit Hilfe der Kettenregel
in eine Ableitung nach $x$ umgewandelt werden:
\[
\frac{d}{d\vartheta}\Theta(\vartheta)
=\frac{dy(x)}{dx}\frac{dx}{d\vartheta}
=-\sin\vartheta \frac{d}{dx} y(x)
\]
Setzt man dies in die Differentialgleichung ein, wird sie zu
\begin{align*}
\frac1{\sin\vartheta}
(-\sin{\vartheta})\frac{d}{dx}\sin\vartheta (-\sin\vartheta)
\frac{d}{dx}y(x)
&=
\frac{d}{dx}\sin^2\vartheta\frac{d}{dx}y(x)\\
&=
\frac{d}{dx}(1-\cos^2\vartheta)\frac{d}{dx}y(x)\\
&=
\frac{d}{dx}(1-x^2)\frac{d}{dx}y(x).
\end{align*}
Die gesuchten Funktionen sind also L"osungen der Differentialgleichung
\begin{equation}
\frac{d}{dx}(1-x^2)\frac{d}{dx}y(x)
=
my(x)
\label{tsunamieigenwertproblem}
\end{equation}
Die Funktionen $y(x)$ m"ussen im ganzen Interval $[-1,1]$ definiert
sein. Dies ist nicht unbedingt selbstverst"andlich, wie schon der Fall
$m=0$ zeigt. In diesem Fall kann man die Differntialgleichung
durch zweimaliges Integrieren l"osen:
\begin{align*}
\frac{d}{dx}(1-x^2)\frac{d}{dx}y(x)&=0\\
(1-x^2)\frac{d}{dx}y(x)&=C\\
\frac{d}{dx}y(x)&=\frac{C}{1-x^2}\\
y(x)&=C\int\frac{dx}{1-x^2}\\
&=\frac{C}2\int\frac{dx}{1-x}+\frac{C}2\int\frac{dx}{1+x}\\
&=-\frac{C}2\log(1-x)+\frac{C}2\log(1+x) +D\\
&=\frac{C}2\log\frac{1+x}{1-x} + D
\end{align*}
An beiden Intervallenden w"achst die Funktion "uber alle Grenzen,
es sei denn es sei $C=0$.

Mit Sicherheit auf dem ganzen Interval definiert w"aren Polynome,
wir k"onnten also einen Ansatz
\[
y(x)=a_0+a_1+a_2x^2+\dots a_nx^n
\]
probieren. Setzt man dies in die Differentialgleichung ein und
beh"alt nur die Terme vom Grad $x^n$, bekommt man auf der rechten
Seite von (\ref{tsunamieigenwertproblem}) $ma_nx^n$, auf
der linken Seite
\[
-\frac{d}{dx}x^2\frac{d}{dx}a_nx^n
=
-\frac{d}{dx}x^2na_nx^{n-1}
=
-\frac{d}{dx}na_nx^{n+1}
=
-n(n+1)a_nx^n
\]
Damit folgt: $m=-n(n+1)$, nur f"ur solche Werte kann
(\ref{tsunamieigenwertproblem}) ein Polynom vom Grad $n$ als L"osung
haben. Die Differentialgleichung wird jetzt zu
\begin{equation}
\frac{d}{dx}(1-x^2)\frac{d}{dx}y(x)+n(n+1)y(x)=0
\label{legendredgl}
\end{equation}

\subsection{Legendre-Polynome}
\index{Legendre-Polynom}
Die Differentialgleichung (\ref{legendredgl}) ist die Differentialgleichung
der Legendre-Polynome.
Das Legendre-Polynom $P_n(x)$ ist eine polynomiale L"osung von
(\ref{legendredgl}) mit $P_n(1)=1$.
Dies legt aber die Funktion nicht fest, es sind weitere Bedingungen
n"otig. Daher wird verlangt, dass die Polynome auch orthogonal
sein sollen, also die Bedingung
\[
\int_{-1}^1 P_k(x)P_l(x)\,dx=0\quad\text{f"ur $k\ne l$}
\]
erf"ullen. Damit werden die Polynome eindeutig.
Die ersten sechs Legendre-Polynome sind
\begin{align*}
P_0(x)&=1\\
P_1(x)&=x\\
P_2(x)&=\frac12(3x^2-1)\\
P_3(x)&=\frac12(5x^3-3x)\\
P_4(x)&=\frac18(35x^4-30x^2+3)\\
P_5(x)&=\frac18(63x^5-70x^3+15x)
\end{align*}
Ausserdem gilt
\[
\int_{-1}^1 P_k(x)^2\,dx = \frac{2}{2k+1}.
\]
Da man 
jede Funktion auf dem Interval $[-1,1]$ mit Polynomen approximieren kann,
kann man auch jede Funktion durch Linearkombinationen von Legendre-Polynomen
$P_n(x)$ schreiben. 

Die Koeffizienten kann man mit Hilfe eines Integrals finden. Setzt man
\[
f(x)=\sum_{k\ge 0} c_k P_k(x)
\]
und berechnet man das Integral
\[
\int_{-1}^1 f(x)P_l(x)\,dx
=
\sum_{k\ge 0} c_k \int_{-1}^1 P_k(x)P_l(x)\,dx
=
\frac{2c_k}{2k+1}
\]
folgt
\[
c_k=\frac{2k+1}{2}\int_{-1}^1P_k(x)f(x)\,dx.
\]
Die Koeffizienten $c_k$ sind sozusagen die ``Legendre-Koeffizienten''
der Entwicklung der Funktion $f(x)$ nach Legendre-Polynomen,
analog zu den Fourier-Koeffizienten auf einem Interval.

\subsection{Anfangsbedingungen}
\index{Anfangsbedingungen}
Unter Verwendung der Legendre-Polynome kann man jetzt die Wellengleichung
zu beliebigen Anfangsbedingungen l"osen.
Die L"osung der Differentialgleichung muss von der Form sein
\[
u(t, x)=\sum_{k=0}^{\infty}(a_k\cos \lambda_k t+b_k\sin\lambda_k t)P_k(x),
\]
wobei $\lambda_k=\sqrt{k(k+1)}$.
Die Koeffizienten m"ussen aus der Anfangsbedingung, also aus den 
Funktionen $F(\vartheta)=f(x)$ und $G(\vartheta)=g(x)$ bestimmt werden.
Die Anfangsbedingung f"ur $u(t,x)$ ergibt
\begin{align*}
u(0,x)
&=\sum_{k=0}^{\infty} a_kP_k(x)=f(x)
\end{align*}
F"ur $\partial_tu(t,x)$ ergibt sich entsprechend
\begin{align*}
\frac{\partial}{\partial t}u(0,x)
&=\sum_{k=0}^{\infty} \lambda_k b_kP_k(x)=g(x)
\end{align*}
Die Koeffizienten $a_k$ und $b_k$ kann man mit
\begin{align*}
a_k&=
\frac{2k+1}{2}\int_{-1}^1 P_k(x)f(x)\,dx
\\
b_k&=
\frac{2k+1}{2\lambda_k}\int_{-1}^1P_k(x)f(x)\,dx
\end{align*}
berechnen.

\subsection{Punktquelle}
\begin{figure}
\begin{center}
\includegraphics[width=\hsize]{graphics/tsunami0}
\end{center}
\caption{N"aherungsl"osung f"ur $N=25$ und $t=0$\label{tsunami0}}
\end{figure}
\begin{figure}
\begin{center}
\includegraphics[width=\hsize]{graphics/tsunami50}
\end{center}
\caption{N"aherungsl"osung f"ur $N=25$ und $t=1$\label{tsunami50}}
\end{figure}

Wir w"ahlen jetzt eine spezielle Anfangsbedingung:
\begin{align*}
f_\varepsilon(x)&=\begin{cases}
\frac1{\varepsilon}&\qquad 1-\varepsilon<x\le 1\\
0&\qquad\text{sonst}
\end{cases}
\\
g(x)&=0
\end{align*}
In einer kleinen Umgebung des Nordpoles ist der Wert 
$\frac1{\varepsilon}$, also sehr gross, in allen anderen Punkten $0$.
Offenbar sind die $b_k=0$, und es bleiben nur die 
$a_k$ zu berechnen. Dazu gilt:
\begin{align*}
a_k(\varepsilon)&=\frac{2k+1}{2}\int_{-1}^1P_k(x)f_\varepsilon(x)\,dx
\\
&=\frac{2k+1}{2}\int_{1-\varepsilon}^1P_k(x)\frac1{\varepsilon}\,dx
\end{align*}
Da uns nur der Grenzwert $\varepsilon\to 0$ interessiert, gehen wir
zur Grenze "uber
\begin{align*}
\lim_{\varepsilon\to 0} a_k(\varepsilon)
&=
\frac{2k+1}{2}\lim_{\varepsilon\to 0}\frac1{\varepsilon}\int_{1-\varepsilon}^1P_k(x)\,dx
\end{align*}
Mit einer Stammfunktion $I_k(x)$ von $P_k(x)$ wird dies zu
\begin{align*}
\lim_{\varepsilon\to 0} a_k(\varepsilon)
&=
\frac{2k+1}{2}\lim_{\varepsilon\to 0}\frac{I_k(1)-I_k(1-\varepsilon)}{\varepsilon}
\\
&=\frac{2k+1}{2}I_k'(1)=\frac{2k+1}{2}P_k(1)=\frac{2k+1}{2}
\end{align*}
Als L"osung bekommt man damit formal
\begin{equation}
u(t,x)
=
\sum_{k=0}^\infty \frac{2k+1}{2}P_k(x) \cos \sqrt{k(k+1)}t.
\end{equation}
Leider ist diese Reihe nicht konvergent, was angesichts der sehr
speziellen Anfangsbedingungen auch nicht zu erwarten war.
Wenn man sie aber nach $N$ Termen abbricht, und mit $\frac1{N^2}$ 
normiert, erh"alt man eine L"osungsfunktion die ein ungef"ahres
Bild f"ur die Wellenausbreitung ergibt.
In den Abbildungen \ref{tsunami0} und \ref{tsunami50} wurde die Reihe nach 25 Termen
abgebrochen.

%
% jacobi.tex -- Anwendung: Hamilton-Jacobi-Formulierung der Mechanik
%
% (c) 2012 Prof Dr Andreas Mueller, Hochschule Rapperswil
% $Id$
%
\section{Anwendung: Hamiltonsche Mechanik\label{hamilton-mechanik}}
In den bisherigen Bespielen wurde jeweils ein Separationsansatz mit
einem Produkt von Teilfunktion gew"ahlt.
Dieser Abschnitt soll illustrieren, dass in einigen F"allen auch
ein Separationsansatz mit einer Summe von Teilfunktionen
zum Ziel f"uhren kann.
Dieser Fall ist f"ur die Anwendungen recht wichtig, denn die
dabei entstehende partielle Differentialgleichungen hat eine
gen"ugend einfach Struktur, dass der erste Separationsschritt immer
durchgef"uhrt werden kann.

\subsection{Motivation}
\index{Newtonsche Gesetze}
\index{Planeten}
\index{Satelliten}
Die Newtonschen Gesetze reichen vollst"andig, um die Bewegung von
Planeten und Satelliten vorherzusagen.
Unterliegt
ein K"orper der Masse $m$ mit zeitabh"angigen Koordinaten $\vec x(t)$
einer ortsabh"angigen Kraft $\vec F(\vec x)$, dann muss die Bahnkurve 
$\vec x(t)$ die gew"ohnliche Differentialgleichung
\begin{equation}
m\frac{d^2}{dt^2}\vec x(t)=\vec F(\vec x(t))
\label{jacobi:newton}
\end{equation}
erf"ullen. Ausgehend von einem bekanten Anfangspunkt $\vec x_0$ und
der Anfangsgeschwindigkeit $\vec v_0$ l"asst sich durch
l"osen der Differentialgleichung die Bahnkurve bestimmen.

Diese Beschreibung hat jedoch ein paar praktisch bedeutsame Nachteile.
Oft ist das Kraftgesetz nicht exakt bekannt. Zum Beispiel werden
Satelliten in niedrigem Erdorbit\footnote{Low earth orbit, wenige 100km}
\index{Erdorbit}
von der zwar sehr d"unnen, aber nicht vernachl"assigbaren
Erdatmosph"are abgebremst.
\index{Erdatmosphare@Erdatmosph\"are}
\index{Merkur}
Oder der Planet Merkur ver"andert laufend seine Bahn um einen winzigen
Betrag, ein Effekt, den erst Albert Einstein mit seiner speziellen
Relativit"atstheorie erkl"aren konnte.
Man sieht sich also oft mit der Aufgabe konfrontiert, dass man zwar die
Bahn berechnen k"onnte, wenn das Kraftgesetz exakt zutreffen w"urde,
dass man aber die Bahnver"anderungen unter dem Einfluss kleiner
Abweichungen vom exakten Kraftgesetz bestimmen sollte.
\index{Luftwiderstand}

\begin{beispiel}
\index{Billardtisch}
Als Beispiel betrachten wir ein Kugel auf einem ideal horizontal angenommenen
Billardtisch. Die Bewegung dieser Kugel wird offenbar genau beschrieben
durch den Anfangspunkt und die Geschwindigkeit $\vec x_0$ und $\vec v_0$.
Die Kugel wird in Richtung $\vec x_0$ weiterrollen, dabei
aber langsamer werden und schliesslich zum Stillstand kommen.
Es gibt also eine Kurve $t\mapsto \vec x(t, \vec x_0, \vec v_0))$,
die beiden Vektoren $\vec x_0$ und $\vec v_0$ bestimmen die
Bahn vollst"andig.

Was passiert, wenn der Billiardtisch nicht exakt horizontal ist?
Offenbar wirkt dann eine kleine zus"atzlich Kraft auf die Billard-Kugel,
und zwar in Richtung der gr"ossten Neigung der Billardtischplatte.
Wenn die Neigung sehr klein ist, ist die Abweichung von 
$\vec x(t,\vec x_0,\vec v_0))$ sehr gering.
Man k"onnte die Beschreibung diese Modifikation dadurch beschreiben,
dass man $\vec x_0$ und $\vec v_0$ leicht anpasst.
Die tats"achliche Position der Kugel zur Zeit $t$ ist die Position,
die eine Billiardkugel auf einem horizontalen Tisch ausgehend von
einem etwas anderen Ausgangspunkt $\vec x_0(t)$ mit einer
etwas anderen Ausgangsgeschwindigkeit $\vec v_0(t)$
nach Zeit $t$ erreicht h"atte.
\end{beispiel}

Der Vorteil dieser Beschreibung besteht darin, dass sich die
Abweichung vom exakten Kraftgesetz direkt in der Zeitabh"angigkeit
von $x_0(t)$ und $v_0(t)$ ausdr"uckt. Ist das Kraftgesetz exakt
erf"ullt, bleiben $x_0(t)$ und $v_0(t)$ konstant. Ausserdem ist die
Berechnung der Bahn ganz einfach: zu $\vec x_0$ kommt einfach
ein Vielfaches von $\vec v_0$ hinzu.

Die Bahnen der Satelliten sind offenbar viel komplizierter. 
Ort und Geschwindigkeit sind nicht geeignet, die
Bahn auf diese Art und Weise zu charakterisieren, sie "andern laufend
dramatisch ihre Gr"osse. Niemand kann mit Leichtigkeit sagen, ob
die Bahn vom Punkt  $\vec x_0$ mit Anfangsgeschwindigkeit $\vec v_0$
irgendwie "ahnlich aussieht wie die Bahn vom Punkt $\vec x_1$
mit Anfangsgeschwindigkeit $\vec v_1$.

\begin{aufgabe}
\label{jacobi:aufgabe}
Zu einem beliebigen mechanischen System finde man einen Satz von 
Parametern 
$Q_i$ und $P_i$ so, dass sich {\em jede}
Bahn mit Hilfe dieser Parameter in der Form
\begin{equation}
x_i(t) = x_i(t,Q_1,\dots,Q_n,P_1,\dots,P_n)
\label{jacobi:aufgabekurve}
\end{equation}
beschreiben l"asst.
\end{aufgabe}

\begin{beispiel}
Betrachten wir wieder das Problem der Billardkugel, aber nehmen wir
diesmal an, dass die Kugel ohne Reibung rollt. Dann gilt
\begin{equation}
\begin{aligned}
x(t)&=x_0 + tv_{x0},\\
y(t)&=y_0 + tv_{y0}.
\end{aligned}
\label{jacobi:linear}
\end{equation}
In dieser Form ist das bereits eine L"osung der gestellten Aufgabe,
denn jede Bahnkurve l"asst such durch geeignete Wahl der
Parameter $\vec x_0$ und $\vec v_0$ charakterisieren.

Man kann aber noch mehr erreichen: man kann die Koordinaten $Q_i$ und $P_i$
so w"ahlen, dass die $P_i$ alle die Bedeutung einer Energie,
und die $Q_i$ die Bedeutung einer Zeit haben.
Die Funktion der $P_i$ k"onnen die Terme
\begin{equation*}
\begin{aligned}
P_x&=\frac12mv_x^2,&
P_y&=\frac12mv_y^2
\end{aligned}
\end{equation*}
"ubernehmen, also die Beitr"age der Bewegungskomponenten in $x$-
bzw.~$y$-Richtung zur Energie. Da die Kugel reibungsfrei rollt, sind
diese Gr"ossen konstant.

Die Gr"ossen $Q_x$ und $Q_y$ m"ussen so gew"ahlt werden, dass 
die Gleichungen (\ref{jacobi:linear}) die tats"achliche Bahn beschreiben.
Durch Aufl"osen von (\ref{jacobi:aufgabekurve}) nach $Q_i$ findet man
\begin{equation*}
\begin{aligned}
Q_x&=-\frac{x_0}{v_{0x}},\\
Q_y&=-\frac{y_0}{v_{0y}}.
\end{aligned}
\end{equation*}
Die Zahlen $-Q_x$ und $-Q_y$ sind also die Zeiten, zu denen die
Billardkugel die $x$- bzw.~$y$-Achse kreuzt.
\end{beispiel}

Die schwierigere Frage ist, ob jedes mechanische System eine
L"osung der Aufgabe \ref{jacobi:aufgabe} zul"asst.
Die Antwort passt ins Thema: ja, man kann eine solche Beschreibung
finden, aber man muss dazu die L"osung einer speziellen
partiellen Differentialgleichung finden. Oft l"asst sich die
Differentialgleichung mit einem Separationsansatz l"osen.

\subsection{Hamilton-Jacobi-Formalismus}
\index{Hamilton-Jacobi-Formalismus}
In der Hamiltonschen Beschreibung der Mechanik geht man aus von
der Gesamtenergie $H(x_i, p_i)$ ausgedr"uckt durch Ort und Impuls.
F"ur das Beispielproblem ist
\index{Hamilton-Funktion}
\begin{equation}
H(x_i,p_i)=\frac1{2m}(p_x^2+p_y^2).
\label{jacobi:hamilton:funktion}
\end{equation}
Der Zusammenhang zwischen den Koordinaten und den Impulsen ist dann
durch die Hamiltonschen Differentialgleichungen gegeben:
\begin{align}
\frac{d}{dt}x_i&=\frac{\partial H}{\partial p_i}
&\Rightarrow
\qquad \dot x_i&=\frac{p_i}{m}=v_i
\label{jacobi:hamilton:geschwindigkeit}
\\
\frac{d}{dt}p_i&=-\frac{\partial H}{\partial x_i}
&\Rightarrow
\qquad
\dot p_i&=m\ddot x_i=0
\label{jacobi:hamilton:newton}
\end{align}
\index{Hamilton-Gleichungen}
Die erste Gleichung (\ref{jacobi:hamilton:geschwindigkeit}) stellt
nur den Zusammenhang zwischen der Geschwindigkeit und dem Impuls
her. Die zweite Gleichung (\ref{jacobi:hamilton:newton}) besagt in
diesem Fall, dass keine Kraft wirkt. N"ahme man in 
(\ref{jacobi:hamilton:funktion}) noch ein Potential $V(x)$ hinzu,
w"urde die zweite Gleichung zu
\[
m\ddot x_i=-\frac{\partial V}{\partial x_i} = F_i(x),
\]
also genau dem Newtonschen Gesetz.

Nach Hamiliton ist jeder andere Satz von Koordinaten $Q_i$ und $P_i$
genauso geeignet zur Beschreibung des mechanischen Systems, solange die
Gleichungen 
(\ref{jacobi:hamilton:geschwindigkeit}) und (\ref{jacobi:hamilton:newton})
weiterhin G"ultigkeit haben. Jacobi und Hamilton haben eine Methode
angegeben, mit der man eine Koordinatentransformation finden kann.

\begin{satz}
\label{jacobi:satz}
Eine L"osung der Aufgabe (\ref{jacobi:aufgabe}) wird gefunden mit
Hilfe einer Funktion $S(x_i, t)$, die L"osung der partiellen
Differentialgleichung
\begin{equation}
\frac{\partial S}{\partial t}
=
H\biggl(
x_i,
\frac{\partial S}{\partial x_i}
\biggr)
\label{jacobi:hamilton:dgl}
\end{equation}
erf"ullt.
Eine L"osungsfunktion $S$ der Differentialgleichung enth"alt notwendigerweise
ein Anzahl von Integrationskonstanten $P_i$.
Die partiellen Ableitungen von $S$ nach diesen $P_i$
sind die neuen konstanten Bahnparameter $Q_i$
\begin{equation}
Q_i=\frac{\partial S}{\partial P_i}
\label{jacobi:hamilton:impuls}
\end{equation}
Die Gleichungen (\ref{jacobi:hamilton:impuls}) enthalten ausser den
Gr"ossen $P_i$ und $Q_i$ auch die Koordinaten $x_i$ und die Zeit $t$.
Durch Aufl"osen nach den Variablen $x_i$ l"asst sich die Bahnkurve
als Funktion
\[
x_i(t,Q_i,P_i)
\]
ausdr"ucken.
\end{satz}
Der Parametersatz $(Q_i,P_i)$ beschreibt also alle m"oglichen 
Bahnen. Zwei Bahnen k"onnen sehr einfach verglichen werden, wenn die
Paramter $Q_i$ und $P_i$ nahe beeinander sind, dann liegen auch
die Bahnen nahe beeinander.

Eine Begr"undung f"ur diesen Satz liegt ausserhalb der Ziele
dieser Vorlesung, wir wollen aber mit zwei Beispielen zeigen,
dass dieser Formulismus funktioniert und die L"osung der
Bewegungsdifferentialgleichungen des Systems erm"oglicht.

\subsection{Beispiele}
\subsubsection{Bewegung ohne "ausseren Krafeinfluss in zwei Dimensionen}
\index{Bewegung ohne Krafteinfluss}
Wir beginnen wieder bei der Hamilton-Funktion 
\[
H(p_x, p_y)=\frac1{2m}(p_x^2+p_y^2).
\]
Nach (\ref{jacobi:hamilton:dgl}) m"ussen wir jetzt also die
Differentialgleichung
\[
\frac1{2m}\biggl(
\biggl(\frac{\partial S}{\partial x}\biggr)^2
+
\biggl(\frac{\partial S}{\partial y}\biggr)^2
\biggr)=\frac{\partial S}{\partial t}
\]
f"ur die Funktion $S(t,x,y)$ l"osen. Wir verwenden dazu einen 
Separationsansatz der Form
\[
S(t,x,y)=S_1(x)+S_2(y) + S_3(t).
\]
Einsetzen in die Differentialgleichung liefert
\begin{equation}
\frac1{2m}( S_1'(x)^2+S_2'(y)^2)=S_3'(t).
\label{jacobi:kraeftefrei:sep1}
\end{equation}
Die linke Seite h"angt nicht von $t$ ab, die rechte h"angt
aber nur von $t$ ab, also sind beide Seiten konstant.
Wir nennen die Konstanten $P_1$. Damit l"asst sich
jetzt $S_3(t)$ bestimmen, es muss eine geeignete Integrationskonstante
geben, so dass
\[
S_3(t)=P_1t
\]
damit ist die Aufgabe \ref{jacobi:aufgabe} mindestens f"ur die Variable
$t$ bereits gel"ost.

Wir untersuchen jetzt die linke Seite von (\ref{jacobi:kraeftefrei:sep1}).
Auch diese Gleichung kann man unter Verwendung der Konstanten $P_3$
separieren:
\[
\frac1{2m} S_1'(x)^2
=
P_1-\frac1{2m}S_2'(y)^2.
\]
Die linke Seite h"angt nur von $x$ ab, die rechte nur von $y$, also
sind beide konstant. 
Wir nennen die Konstante $P_1$ und finden die L"osung
\[
S_1(x)
=
\sqrt{2mP_2}x.
\]
Jetzt kann man aber auch noch $S_2$ bestimmen. Es ist n"amlich 
\[
\frac1{2m} S_2'(y)^2
=P_1-P_2
\]
woraus sich wie vorhin die L"osung
\[
S_2(y)
=
\sqrt{2m(P_1-P_2)}y
\]
ergibt. Damit haben wir jetzt eine L"osung der Differentialgleichung:
\[
S(x,y,t)=
P_1t
+
\sqrt{2mP_2}x
+
\sqrt{2m(P_1-P_2)}y
\]
Nach den Regeln des Satzes \ref{jacobi:satz}, sind die zu verwendenden
Koordinaten die partiellen Ableitungen:
\begin{align*}
Q_1&=\frac{\partial S}{\partial P_1}
=
t + \sqrt{\frac{m}{2(P_1-P_2)}}y,\\
Q_2&=\frac{\partial S}{\partial P_2}
=
\sqrt{\frac{m}{2P_2}}x
-
\sqrt{\frac{m}{2(P_1-P_2)}}y
\end{align*}
Jetzt kann man nach $x$ und $y$ aufl"osen und damit die
Bahnkurve durch die Konstanten $Q_i$ und $P_i$ und die Zeit $t$
ausdr"ucken:
\begin{align*}
x&=\sqrt{\frac{2P_2}{m}}(Q_1+Q_2-t)\\
y&=\sqrt{\frac{2(P_1-P_2)}{m}}(Q_1-t).
\end{align*}
Wie erwartet beschreiben diese Gleichungen eine gleichf"ormige
Bewegung. Die Geschwindigkeit ist die Ableitung nach der Zeit,
also
\begin{align*}
v_x
&=
\sqrt{\frac{2P_2}{m}}
&\Rightarrow&&
P_2&=\frac{mv_x^2}2
\\
v_y
&=
\sqrt{\frac{2(P_1-P_2)}{m}}
&\Rightarrow&&
P_1&=P_2+\frac{mv_y^2}2=\frac{m}2(v_x^2+v_y^2)=\frac12mv^2
\end{align*}
Die physikalische Bedeutung von $P_1$ ist also die kinetische
Energie des
Gesamtsystems, w"ahrend $P_2$ die kinetische Energie darstellt, die in
der Bewegungskomponent in $x$-Richtung steckt.

\subsubsection{Schiefer Wurf}
\index{schiefer Wurf}
Wir betrachten den schiefen Wurf, der sich vom vorangegangenen
Beispiel durch ein zus"atzliches Gravitationspotential
unterscheidet. Die Gesamtenergie ist jetzt
\[
H(x,y,t)=\frac1{2m}(p_x^2+p_y^2)+mgy.
\]
Die Hamilton-Jacobi-Differentialgleichung lautet jetzt
\[
\frac1{2m}\biggl(
\biggl(\frac{\partial S}{\partial x}\biggr)^2
+
\biggl(\frac{\partial S}{\partial y}\biggr)^2
\biggr)
+mgy=\frac{\partial S}{\partial t} 
\]
Der selbe Separationsansatz wie vorhin f"uhrt auch wieder zu
\[
\frac1{2m}(S_1'(x)^2+S_2'(y)^2)+mgy=S_3'(t),
\]
woraus wieder folgt, dass $S_3'(t)=P_1$ konstant ist.

Wir k"onnen aber auch $x$ und $y$ separieren:
\begin{align*}
\frac1{2m}S_1'(x)^2&=P_1-\frac1{2m}S_2'(y)^2-mgy
%\\
%\frac1{\sqrt{2m}}S_1'(x)&=\sqrt{P_3-\frac1{2m}S_2'(y)^2+mgy}
\end{align*}
Die linke Seite h"angt nicht von $y$ ab, die rechte nicht von $x$, also sind
beide konstant, wir nennen die Konstante $P_2$, und bekommen
die L"osung $S_1=\sqrt{2mP_2} x$.

Die rechte Seite k"onnen wir jetzt auch l"osen:
\begin{align*}
P_2&=
P_1-\frac1{2m}S_2'(y)^2-mgy
\\
\frac1{2m}S_2'(y)^2
&=
P_1-P_2-mgy
\\
S_2'(y)&=\sqrt{
2m(P_1-P_2-mgy)
}
\\
S_2(y)
&=
\frac1{3m^2g}\bigl(2m(P_1-P_2-mgy)\bigr)^{\frac32}
\end{align*}
Damit ist jetzt eine L"osungsfunktion $S(x,y,t)$ vollst"andig
bekannt:
\[
S(x,y,t)=P_1t+\sqrt{2mP_2}x +
\frac1{3m^2g}\bigl(2m(P_1-P_2-mgy)\bigr)^{\frac32}
\]
Die zu verwendenden Koordinaten bekommt man jetzt wie vorhin durch
partielle Ableitung nach den $P_i$. Man bekommt nacheinander:
\begin{equation}
\begin{aligned}
Q_1=\frac{\partial S}{\partial P_1}
&=
t+
\frac1{g}\sqrt{\frac{2(P_1-P_2-mgy)}{m}}
\\
Q_2=\frac{\partial S}{\partial P_2}
&=
\sqrt{\frac{m}{2P_1}}x
-
\frac1{g}\sqrt{\frac{2(P_1-P_2-mgy)}{m}}
\end{aligned}
\label{jacobi:aufloesung}
\end{equation}
Die erste Gleichung kann man nach $y$ aufl"osen:
\begin{equation}
y=
\frac{P_1-P_2}{mg}-\frac{g}{2}(Q_1-t)^2
\label{jacobi:quadratisch}
\end{equation}
Die H"ohe h"angt quadratisch von der Zeit ab, $Q_1$ ist die Zeit
der gr"ossten H"ohe, die Scheitelzeit.

Die Aufl"osung nach $x$ wird einfacher, wenn man erst die Summe der beiden
Gleichungen (\ref{jacobi:aufloesung}) bildet:
\begin{align*}
Q_1+Q_2&=t+\sqrt{\frac{m}{2P_1}}x\\
x&=\sqrt{\frac{2P_1}{m}}(Q_1+Q_2-t)
\end{align*}
Die $x$-Koordinate nimmt offenbar linear mit der Zeit zu. Die
Geschwindigkeit ist
\[
v_x=\sqrt{\frac{2P_1}{m}}\quad\Rightarrow\quad P_1=\frac{mv_x^2}2,
\]
$P_1$ ist also die kinetische Energie der Horizontalbewegung.
$Q_2$ gibt an, wie viel Zeit nach dem Scheiteldurchgang die $x$-Koordinate
verschwindet.

Zur Zeit $t=Q_1$ verschwindet der quadratische Term in
(\ref{jacobi:quadratisch}), und man kann die Gleichung vereinfachen
zu 
\[
mgy + P_2=P_1.
\]
Da $P_2$ die kinetische Energie der Horizontalbewegung ist, und $mgy$
die potentielle Energie, ist $P_1$ die Gesamtenergie.



%
% waves.tex -- 
%
% (c) 2019 Prof Dr Andreas Mueller
%
\section{Wave equation one dimension}
\rhead{One dimensional wave equation}
The wave equation in the plane is
\[
\partial_t^2u-a^2\partial_x^2u=0,
\]
which we want to solve in the domain
\[
\Omega = \{(x,t) \,|\, t > 0\}.
\]
The parameter $a$ has the dimension of velocity, it is the
speed of the wave along the $x$ axis.

\subsection{Constant velocity}
Assume for the time being that $a$ is a constant.
then the equation can also be written as
\begin{align*}
(\partial_t -a\partial_x)(\partial_t+a\partial_x)u&=0
\\
\text{oder}&
\\
(\partial_t +a\partial_x)(\partial_t-a\partial_x)u&=0.
\end{align*}
Thus solutions of the first order equations
\begin{align}
\partial_t u-a\partial_x u&=0
\label{wellelinks}
\\
\partial_t u+a\partial_x u&=0
\label{wellerechts}
\end{align}
are automatically solutions of the wave equation.

\subsection{Solutions of the first order equations}
We want to solve the wave equation for initial conditions of the kind
\begin{equation}
u(x,t)=u_0(x),\qquad x\in\mathbb R.
\label{welleanfang}
\end{equation}
To this end we solve the two first order equations
\eqref{wellelinks} and \eqref{wellerechts} with
those some initial conditions.

Both differential equations are quasilinear equations of first order
for which the method of characteristics can give a solution.
We first determine the characteristics.
Their differential equation is
\begin{align*}
\frac{dx}{ds}&=-a
\\
\frac{dt}{ds}&=1
\\
\frac{du}{ds}&=0.
\end{align*}
The third equation sais that $u$ does not depend on the parameter $s$.
The second equation then says that $s$ is equal to $t$ up to an additive
constant.
Finally, we can solve the first equation by integrating with respect to $s$.
In total, we get the solution
\begin{equation}
\begin{aligned}
x(s)&=-as+x_0\\
t(s)&=s\\
z(s)&=z_0
\end{aligned}
\label{hyperbolisch:quasi1}
\end{equation}
The parameter $x_0$ is the parameter along the characteristic curve
and $z_0$ is the value of $u$ at the initial point $(x_0,0)$.
So we get the solution
\begin{equation}
\begin{aligned}
x(s,x_0)&=-as+x_0\\
t(s,x_0)&=s\\
z(s,x_0)&=u_0(x_0)
\end{aligned}
\label{hyperbolisch:quasi2}
\end{equation}
The second step in the solution algorithm from chapter~\ref{chapter-geometrie}
says that the variables $s$ and $x_0$ have to be eliminated
from~\eqref{hyperbolisch:quasi2}.
We already have identified $s$ as $t$, so we only need to identify
$x_0$ as $x_0=x  at$ and substitute that into the last equation to get
\begin{equation}
u(x,t) = u_0(x+at)
\label{hyperbolisch:quasi3}
\end{equation}
This solution describes a solution of \eqref{wellelinks}
is a wave that travels with velocity $a$ to the left.

Analogously the equation \eqref{wellerechts} gives a wave that
travels to the right.
Combining the two waves allows us to write the most general solution
of the wave equation as
\begin{equation}
u(x,t)=u_+(x+at)+u_-(x-at)
\label{dalembertloesung}
\end{equation}
The functions $u_+$ and $u_-$ have to be chosen in such a way as to
match the initial conditions.

\subsection{Initial velocity}
Initial values alone do not uniquely determine the solutions of a wave
equation, an additional initial condition of the form
\begin{equation}
\partial_tu(x,0)=v_0(x)\label{welleanfangdt}
\end{equation}
is needed.
Using the solution in the form \eqref{dalembertloesung} then leads
to the following equations for $u_+(x)$ and $u_-(x)$
\begin{align*}
u_+(x)+u_-(x)&=u_0(x)\\
au_+'(x)-au_-'(x)&=v_0(x)
\end{align*}
If $V_0$ is an antiderivative of $\frac12av_0$, or in other
words $V_0'=\frac1av_0$, then from the second equation we derive that
\[
u_+(x)-u_-(x)=V_0(x)+c.
\]
With this equation we can no solve for $u_+$ and $u_-$ and find:
\begin{align*}
u_+(x)&=\frac12(u_0(x)+V_0(x)+c)\\
u_-(x)&=\frac12(u_0(x)-V_0(x)-c)
\end{align*}
Thus the solution of the wave equation is
\begin{align}
u(x,t)
&=
\frac12\bigl(u_0(x+at)+V_0(x+at)+c\bigr)+\frac12\bigl(u_0(x-at)-V_0(x-at)-c\bigr)
\notag
\\
&=
\frac12\bigl(u_0(x+at)+V_0(x+at)\bigr)+\frac12\bigl(u_0(x-at)-V_0(x-at)\bigr)
\label{hyperbolisch:dalembert}
\end{align}
The solution \eqref{hyperbolisch:dalembert} is calledt the
{\em d'Alembert solution} of the wave equation.

Both terms in the d'Alembert solution solve the wave equation, but it
is also easy to verify that initial conditions are satisfied:
\begin{align*}
u(x,0)&=u_0(x)\\
\partial_tu(x,0)&=\frac12\bigl(au_0'(x+at)+v_0(x)-au_0'(x)+v_0(x)\bigr) =v_0(x)
\end{align*}
In particular, finding a solution for the wave equation is as easy as
finding an antiderivative of a function along the $x$ axis.
The problem posed initially in this section is the case
$v_0(x)=0$.



%
% cauchy.tex -- XXX
%
% (c) 2019 Prof Dr Andreas Mueller
%
\section{Das Cauchy-Problem in höheren Dimensionen}
\rhead{Das Cauchy-Problem}
Im letzten Abschnitt haben wir Anfangswerte auf der Geraden $t=0$
vorgegeben, damit waren auch gleichzeitig die Ableitungen 
$\partial_x u(0,x)$ festgelegt. Ausserdem hatten wir mit der Funktion $v_0$
die Ableitungen $\partial_t u(0,x)$ vorgegeben.
Der Graph der Lösungsfunktion ist ein Fläche, eine sogenante
Integralfläche der Differentialgleichung. Die Anfangsbedingung
definiert eine Kurve $x\mapsto(x,0,u_0(x))$, die Integralfläche muss
durch diese Kurve gehen.
Durch die zwei Ableitungen ist zudem in jedem Punkt der Kurve
eine Tangentialebene an die Integralfläche vorgegeben.

Wie das Beispiel zeigt, ist die Lösungsfläche durch Vorgabe einer Kurve und
und der Tangentialebenen in jedem Punkt der Kurve bestimmt ist.
Etwas allgemeiner besteht das Cauchy-Problem darin, eine Integralfläche
zu finden, die durch eine beliebige Kurve geht, und ausserdem in jedem Punkt
der Kurve eine bestimmte Tangentialebene hat. Diese Vorgaben nennt man
einen ``Streifen'' (Abbildung~\ref{skript:streifen}).
Eine partielle Differentialgleichung für eine Funktion $u(t,x,y)$
von drei Variablen kann zum Beispiel dadurch festgelegen werden,
dass man Funktionswerte $u(t,x,y)=u_0(x,y)$ zur Zeit $t=0$ festlegt.
Dadurch sind auch die Ableitungen $\partial_x u(0,x,y)=\partial_xu_0(x,y)$
und $\partial_y u(0,x,y)=\partial_y u_0(x,y)$ bestimmt. Die Lösung wird
aber erst eindeutig bestimmt sein, wenn auch die Ableitung in $t$-Richtung
vorgegeben ist, zum Beispiel in der Form $\partial_t(0,x,y)=v_0(x,y)$.

Allgemeiner besteht das Cauchy-Problem darin, eine Lösung zu finden
die entlang einer beliebigen Fläche im $(t,x,y)$-Raum vorgegebene
Werte annimmt. Ausserdem muss die Richtungsableitungen in eine Richtung
senkrecht auf die Fläche (Normalableitung) ebenfalls vorgegebene Werte annehmen.



%
% characteristics.tex -- 
%
% (c) 2019 Prof Dr Andreas Mueller
%
\section{Characteristics}
\rhead{Characteristics}
The method of characteristics in chapter~\ref{chapter-geometrie} 
was particularly successful in deciding whether the Cauchy initial
data was sufficient to determine the solution of a partial differential
equation.
It relied heavily on the geometry of characteristic curves, but the
derivation for their differential equation relied heavily on the fact
that the equation was of first order.

However the d'Alembert-solution suggests that there is an intimate
link between hyperbolic partial differential equations and characteristics.
However, since the equation is now of second order, we have to deal
with the fact that simple curves will not suffice, prompting an extension
of the concept to that of a characteristic strip in
section~\ref{subsection:characteristic-strip}.
We also expect that the differential equation for the characteristic
strip, to be derived in section~\ref{subsection:characteristics-curves},
will be more complicated.
Nevertheless, the definition of the characteristic curve as one not
suitable as initial curve will remain unchanged.
This is the idea we start from in the next section.

\subsection{An unsolvable Cauchy problem}
We want to study the circumstances under which the Cauchy problem
for a hyperbolic partial differential equation can be solved
uniquely.
To better understand what can go wrong, we study the hyperbolic
partial differential equation
\[
\partial_x\partial_y u=0,
\]
and specify the initial conditions
\begin{align*}
u(0,y)&=u_0(y)
\\
\partial_xu(0,y)&=v_0(y).
\end{align*}
From the differential equation we conclude that $\partial_y u$
does not depend on $x$.
This means that $u$ cannot depend on $x$ either, so the boundary
condition for $\partial_xu(0,y)$ is redundant, $v_0$ must vanish
and the solution is $u(x,y)=u_0(y)$.

But because the partial derivatives commute,
$\partial_x\partial_yu=\partial_y\partial_xu$, by the same
argument we also find that $u$ does not depend on $y$, so both functions
$u_0$ and $v_0$ must be constants.
In particular, the Cauchy problem cannot be solved if these
functions are not constants.
The reason for this pathology is that not all the second derivatives can be
determined from the initial data, the second derivative with respect
to $x$ is undefined.

\subsection{Characteristic strip\label{subsection:characteristic-strip}}
The Cauchy problem can only be solved if the values and the
first partial derivatives on the initial curve uniquely determine 
all the second order derivatives.
We try to find those curves where this is not possible.
We start with the differential equation
\begin{equation}
a\frac{\partial^2 u}{\partial x^2}
+
2b\frac{\partial^2 u}{\partial x\partial y}
+
c\frac{\partial^2 u}{\partial y^2}
+
d\frac{\partial u}{\partial x}
+
e\frac{\partial u}{\partial y}
+
fu
=g,
\label{charequation}
\end{equation}
Because we are interested in the second order derivatives,
we bring all the other terms to the right hand side and
abbreviate the new right hand side as $h$:
\begin{equation}
a\frac{\partial^2 u}{\partial x^2}
+
2b\frac{\partial^2 u}{\partial x\partial y}
+
c\frac{\partial^2 u}{\partial y^2}
=
g
-
d\frac{\partial u}{\partial x}
+
e\frac{\partial u}{\partial y}
+
fu
=h.
\notag
\end{equation}
We consider this along the curver
$t\mapsto(x(t),y(t))$.
We assume that we have initial values and first partial derivatives
\begin{equation}
\left.
\begin{aligned}
u(x(t),y(t))&=u(t)\\
\frac{\partial u}{\partial x}(x(t),y(t)) &= p(t)\\
\frac{\partial u}{\partial y}(x(t),y(t)) &= q(t)
\end{aligned}
\qquad
\right\}
\label{charanfangs}
\end{equation}
We call this type of data a {\em strip}
(Figure~\ref{skript:streifen}).

\begin{figure}
\centering
\includegraphics[width=\hsize]{../common/3d/streifen0.png}
\caption{A strip consists of the values and tangent planes
along a curve (red).
The green surface is a solution of the wave equation with the
data of the strip as initial data.
\label{skript:streifen}}
\end{figure}

\begin{figure}
\centering
\includegraphics[width=\hsize]{../common/3d/streifen2.png}
\caption{Along the common red curve both solution surfaces have
the same initial values, but different tangent planes, so they
have different strips.
\label{skript:streifen:eindeutig}}
\end{figure}

\begin{figure}
\centering
\includegraphics[width=\hsize]{../common/3d/streifen1.png}
\caption{Characteristics are those curves that cannot be used to
define a strip that would uniquely determine the solution surface.
The graph shows two solution of the wave equation with the same
strip: they go through the same curve and they have the same
tangent planes, but the differ in the curvature perpendicular to the
initial curve.
This means that the second order derivatives are not determined by
the equation and the strip data.
\label{skript:streifen:zweideutig}}
\end{figure}

The functions $u(t)$, $p(t)$ and $q(t)$ are not completely arbitrary.
By deriving the first equation \eqref{charanfangs} with respect to $t$,
we find
\[
\dot{u}(t)=\frac{d}{dt}u(t)
=
\frac{d}{dt}u(x(t),y(t))
=
\frac{\partial u}{\partial x}(x(t), y(t))\frac{d}{dt}x(t)
+
\frac{\partial u}{\partial y}(x(t), y(t))\frac{d}{dt}y(t).
\]
Since the partial derivatives are also given in \eqref{charanfangs},
we must also have the relation
\begin{equation}
\dot{u}(t)= p(t)\dot{x}(t) + q(t)\dot{y}(t).
\label{cauchydatarestriction}
\end{equation}

\subsection{Characteristic curves\label{subsection:characteristics-curves}}
Specifying the data of a strip often determines the hyperbolic
partial differential equation uniquely.
In figure~\ref{skript:streifen:eindeutig} two different solutions
are shown that have the same initial curve but different tangent
planes, so the have different strips along this curve.

Deriving the last two equations of \eqref{charanfangs} with respect to $t$,
gives the system of linear equations
\[
\begin{linsys}{3}
\dot p(t)
&=&
\partial_x\partial_xu(x(t),y(t))\,\dot x(t)
&+&
\partial_x\partial_yu(x(t),y(t))\,\dot y(t)
& &
\\
\dot q(t)
&=&
& &
\partial_x\partial_yu(x(t),y(t))\,\dot x(t)
&+&
\partial_y\partial_yu(x(t),y(t))\,\dot y(t)
\end{linsys}
\]
Together with the differential equation we have now three linear equations
to determine the second partial derivatives (colored red)
\[
\begin{linsys}{4}
a{\color{red}\displaystyle\frac{\partial^2 u}{\partial x^2}}
&+&
2b{\color{red}\displaystyle\frac{\partial^2 u}{\partial x\partial y}}
&+&
c{\color{red}\displaystyle\frac{\partial^2 u}{\partial y^2}}
&=&
h(t)\phantom{.}
&=&
g-dp(t)-eq(t)-fu\\
\dot x(t)
{\color{red}\displaystyle\frac{\partial^2 u}{\partial x^2}}
&+&
\dot y(t)
{\color{red}\displaystyle\frac{\partial^2 u}{\partial x\partial y}}
& &
&=&
\dot p(t)\phantom{.}
& &
\\
& &
\dot x(t)
{\color{red}\displaystyle\frac{\partial^2 u}{\partial x\partial y}}
&+&
\dot y(t)
{\color{red}\displaystyle\frac{\partial^2 u}{\partial y^2}}
&=&
\dot q(t).
& &
\end{linsys}
\]
This linear system of equations for the second order derivatives
has the coefficient matrix
\[
\begin{pmatrix}
a&2b&c\\
\dot x(t)&\dot y(t)&0\\
0&\dot x(t)&\dot y(t)
\end{pmatrix}.
\]
The system of equation is not or not uniquely solvable if the
determinant vanishes, i.~e.
\begin{align*}
0&=\left|\begin{matrix}
a&2b&c\\
\dot x(t)&\dot y(t)&0\\
0&\dot x(t)&\dot y(t)
\end{matrix}\right|
\\
&=a\dot y(t)^2-2b\dot x(t)\dot y(t)+c\dot x(t)^2
\end{align*}

\begin{definition}
The characteristics of a differential equation of the form
\eqref{charequation}
are the curves
$t\mapsto(x(t),y(t))$, for which the initial data 
\eqref{charanfangs} does not determine the second partial
derivatives uniquely.
\end{definition}

\begin{satz}
\label{charakteristikendgl}
The characteristics of a partial differential equation
\eqref{charequation}
solve the differential equation
\[
a\dot y(t)^2-2b\dot x(t)\dot y(t)+c\dot x(t)^2=0.
\]
\end{satz}

\subsection{Characteristic strip}
We shoose a characteristic $t\mapsto(x(t),y(t))$.
We are only interested in the case where there are infinitely
many possible values for the second derivative.
This case happens when the determinants
\[
\left|
\begin{matrix}
h&2b&c\\
\dot p(t)&\dot y(t)&0\\
\dot q(t)&\dot x(t)&\dot y(t)
\end{matrix}
\right|
,
\quad
\left|
\begin{matrix}
a&h&c\\
\dot x(t)&\dot p(t)&0\\
0&\dot q(t)&\dot y(t)
\end{matrix}
\right|
,
\quad
\left|
\begin{matrix}
a&2b&h\\
\dot x(t)&\dot y(t)&\dot p(t)\\
0&\dot x(t)&\dot q(t)
\end{matrix}
\right|
\]
all vanish, where $h=g-dp(t)-eq(t)-fu(x(t),y(t))$.
It suffices the select a single one of those determinants,
we choose the second:
\begin{align*}
a\dot p(t)\dot y(t)-h\dot x(t)\dot y(t)+c\dot x(t)\dot q(t)&=0
\end{align*}

Together with the condition~\eqref{cauchydatarestriction}
we now have three equations that the functions
$x$, $y$, $u$, $p$ and $q$ must satisfy in order for the second
derivatives to not be determined uniquely by the initial data.

\begin{definition}
A strip along a characteristic which satisfies
\[
a\dot p(t)\dot y(t)-h\dot x(t)\dot y(t)+c\dot x(t)\dot q(t)=0
\]
is called a {\em characteristic strip}.
\end{definition}

Thus it is also possible that the integral surfaces of a partial
differential equation touch along a curve, but are different
nevertheless.
By necessity, the tangent planes along the intersection curve form
a characteristic strip.

\begin{satz}
\label{skript:satz:charakteristiken}
If two different integral surfaces touch along a curve,
then this curve together with the tangent planes form a characteristic strip.
\end{satz}

Figure~\ref{skript:streifen:zweideutig} shows two solutions of the wave
equation that touch along a characteristic.
From theorem~\ref{skript:satz:charakteristiken} the red strip is a 
characteristic strip.

\begin{proof}
Apparently there are at least two different solutions of the
partial differential equation that contain the curve which in
addition have the same tangent planes.
The curve and the tangent planes do not determine the solution
uniquely, so they form a characteristic strip.
\end{proof}

\subsection{Examples}
\subsubsection{Wave equation}
The characteristics of the wave equation
\begin{equation}
\partial_t^2u-a^2\partial_x^2u=0
\label{hyperbolisch:wellengleichung}
\end{equation}
are the curves $s\mapsto(t(s),x(s))$, that satisify the equation
\begin{align*}
\left(
\frac{dx(s)}{ds}\right)^2-a^2\left(\frac{dt(s)}{ds}\right)^2&=0
\\
\frac{dx(s)}{ds}
&=
\pm a\frac{dt(s)}{ds}
\\
\Rightarrow
\frac{dx}{dt}=\pm a.
\end{align*}
These are straight lines with slope $\pm a$.
\begin{figure}
\begin{center}
\includegraphics[width=0.8\hsize]{../common/images/char-2.pdf}
\end{center}
\caption{Characteristics of the wave
equation~(\ref{hyperbolisch:wellengleichung})
\label{hyp:wellen}}
\end{figure}
Figure~\ref{hyp:wellen} shows the characteristics.

\subsubsection{The equation $\partial_x\partial_yu=0$}
The condition for the characteristics in this case is
\[
-\dot x(t)\dot y(t)=0
\]
One of the derivatives must disappear, which is only possible for
curves that are parallel to the $x$- or the $y$-axis.
Figure~\ref{hyp:dxdy} shows these characteristics
\begin{figure}
\begin{center}
\includegraphics[width=0.8\hsize]{../common/images/char-3.pdf}
\end{center}
\caption{characteristics of the hyperbolic partial differential equation
$\partial_x\partial_yu=0$.
\label{hyp:dxdy}}
\end{figure}

\subsubsection{Curved characteristics}
The partial differential equation
\begin{equation}
\partial_t^2u-x^2\partial_x^2u=0
\label{hyperbolisch:gekruemmt}
\end{equation}
is hyperbolic for $x\ne 0$.
The characteristics satisfy the equation
\begin{align*}
x'(s)^2-x^2t'(s)^2&=0
\\
xt'&=\pm  x'
\\
\frac{d}{ds}t&=\pm\frac{d}{ds}\log x
\\
t&=\pm\log x+C
\\
x&=x_0e^{\pm t}
\end{align*}
The characteristics are exponential curves.
In figure \ref{hyp:exp}
the characteristics for the positive sign are drawn in red,
the characteristics for the negative sign green.
\begin{figure}
\begin{center}
\includegraphics[width=0.8\hsize]{../common/images/hypexp-1.pdf}
\end{center}
\caption{Characteristics for $x\ne 0$ for the hyperbolic partial
differential equation~\eqref{hyperbolisch:gekruemmt}.
\label{hyp:exp}}
\end{figure}

This equation describes a wave equation in a medium in which
the wave velocity increases with increasing $x$.
The exponential curves suggest that the wave becomes faster when moving ``out''.

\subsection{Characteristics of elliptic or parabolic partial differential
equations}
The theory of the characteristics developed above can also be
applied to elliptic or hyperbolic partial differential equations.

In the elliptic case, the differential equation of the characteristics
\[
a\dot y(t)^2-2b\dot x(t)\dot y(t)+c\dot x(t)^2=0
\]
does not have any solutions, as the expression only vanishes only for
$\dot x(t)=0$ and $\dot y(t)=0$.

For parabolic partial differential equations the characteristic
equation becomes in a suitable coordinate system
\[
-\kappa t'(s)^2=0.
\]
This can only hold true if $t$ is constant.
The characteristics in this case are straight lines parallel
to the $x$-axis.
In fact it is not possible to determine the second derivative
with respect to $t$ for initial data along a line parallel
to the $x$-axis.

We will summarize the information about causality or which points on
the boundery can influence the solution in which points of the domain
in section~\ref{section:which-boundary-points}

\subsection{Some interesting theorems}

\begin{satz}
Every integral surface can be covered with a set of characteristic
strips.
\end{satz}

\begin{proof}
Let $u$ be the solution of a partial differential equation.
The differential equation in theorem~\ref{charakteristikendgl}
describes two curves $t\mapsto(x(t),y(t))$ in every point of the domain.
Obviously it is possible to cover the solution surface by such curves.
By substituting these curves into $u(x,y)$, $\partial_xu(x,y)$
and $\partial_yu(x,y)$ we get a set of characteristic strips
as claimed.
\end{proof}

\begin{satz}
If a set of characteristic strips covers a surface $S$ defined
by the function $u(x,y)$ and this function has continuous 
second derivatives then $u$ is a solution of the differential equation.
\end{satz}

\begin{proof}
The characteristic strips satisfy the equations
\begin{equation}
\begin{gathered}
a\dot y(t)^2-2b\dot x(t)\dot y(t)+c\dot x(t)^2=0,
\\
a\dot p(t)\dot y(t)-h\dot x(t)\dot y(t)+c\dot x(t)\dot q(t)=0,
\\
\dot u(t)=p(t)\dot x(t)+q(t)\dot y(t).
\end{gathered}
\label{alle}
\end{equation}
Let's call the second derivatives of $u$ along a cahracteristic curve
\begin{align*}
R&=\partial_x^2u(x(t),y(t)),
\\
S&=\partial_x\partial_yu(x(t),y(t)),
\\
T&=\partial_y^2u(x(t),y(t)).
\end{align*}
We can write
\begin{align*}
\dot p(t)&=R(t)\dot x(t)+S(t)\dot y(t)\qquad\text{and}\\
\dot q(t)&=S(t)\dot x(t)+T(t)\dot y(t).
\end{align*}
If we substitute this in the second equation of \eqref{alle}, we get
\begin{align*}
a(R\dot x+S\dot y)\dot y-h\dot x\dot y+c\dot x(S\dot x+T\dot y)&=0
\\
\Rightarrow \qquad(aR-h+cT)\dot x\dot y+aS\dot y^2 +cS \dot x^2&=0.
\end{align*}
Multiplying the first equation of
\eqref{alle} by $t$ and subtracting it we get
\[
(aR+2bS+cT-h)\dot x\dot y=0.
\]
Writing out the parenthesis leads to the equation
\[
a\partial_x^2u+2b\partial_x\partial_yu+c\partial_y^2u-h=0
\]
which is the original differential equation.
\end{proof}
This theorem teaches that the hyperbolic partial differential equation
can be solved by looking for characteristic strips.
for this it suffices the solve ordinary differential equations for the
functions $x$, $y$, $p$, $q$, $R$, $S$ and $T$.



%
% discontinuities.tex -- 
%
% (c) 2019 Prof Dr Andreas Mueller
%
\section{Propagation of discontinuities}
\rhead{Discontinuities}
The study of the wave equation at the beginning of this chapter
allowed us to find solutions as translates of an initial function
$u_0$ along the characteristics, which were straight lines
$x\pm at=\operatorname{const}$.
If $u_0$ is not everywhere differentiable, then the solution will
not be differentiable along the characteristic.

Assume that $u$ is continuously differentiable everywhere
and twice continuously differentiable everywhere
except along a curve.
This curve separates the domain into subdomains.
In each subdomain, $u$ is a solution of the differential equation
with the same boundary values and tangent planes on the curve.
This implies that the curve and the tangent planes must
specify a characteristic strip.

\begin{satz}
If $u$ is continuously differentiable everywhere and twice
continuously differentiable everywhere except along a curve,
and $u$ is a solution everywhere except on that curve,
then that curve is a characteristic.
\end{satz}

This means that discontinuities of the second derivative
can only propagate along the characteristic curves.

The wave equation can be used to approximately compute the
flow around a supersonic plane.
The plane produces shock waves which are discontinuities in the
solution of the wave equation.
According to the theorem above, these shock waves follow
characteristic curves.
The can be made visible using the Schlieren technique
(figure~\ref{ueberschall2d}).

\begin{figure}
\begin{center}
\includegraphics[width=0.8\hsize]{../common/graphics/i-5-1}
\end{center}
\caption{Flow around a supersonic plane\label{ueberschall2d}}
\end{figure}


%
% higherdim.tex -- XXX
%
% (c) 2019 Prof Dr Andreas Mueller
%
\section{Characteristic in higher dimensions}
\rhead{Characteristic in higher dimensions}
% XXX
Wir wollen jetzt die Frage nach den Charakteristiken in höheren Dimensionen
stellen. Um die Diskussion überschaubar zu halten, beschränken wir uns auf
das dreidimensionale Problem. Die gefundenen Schlussfolgerungen 
lassen sich sofort auf beliebig viele Dimensionen verallgemeinern.

\subsection{Problemstellung}
Das Cauchy-Problem für eine partielle Differentialgleichung der
Form
\[
\sum_{i,j=1}^3a_{ij}\partial_i\partial_ju+\sum_{i=1}^3b_i\partial_iu+cu=f
\]
besteht darin, dass auf einer Fläche, die zum Beispiel durch
eine Gleichung
\[
\omega(x_1,x_2,x_3)=0
\]
gegeben werden kann,
die Funktionswerte von $u$ vorgegeben werden, sowie die ersten
Ableitungen in eine Richtung senkrecht auf die Fläche.
Diese Ableitung kann mit Hilfe des Gradienten berechnet werden:
\[
\frac{\partial u}{\partial n}=\operatorname{grad}u\cdot \frac{\operatorname{grad}\omega}{|\operatorname{grad}\omega|}
\]
Die Lösung hängt wieder davon ab, ob die Differentialgleichung
und die Anfangsbedingungen die zweiten Ableitungen von $u$ bereits
eindeutig bestimmen.

\subsection{Ein Spezialfall}
Wir betrachten wieder den Spezialfall, in dem die Anfangswerte auf der
Ebene $x_1=0$ vorgegeben sind, also
\begin{align*}
u(0,x_2,x_3)&=u_0(x_2,x_3),
\\
\partial_1u(0,x_2,x_3)&=h(x_2,x_3)
\end{align*}
Dies entspricht dem Fall $\omega(x_1,x_2,x_3)=x_1$.

Durch die Vorgabe der Anfangswerte in Form der Funktion $u_0$ sind die Ableitungen
$\partial_iu$ für $i=2,3$ ebenfalls festgelegt.
Durch Ableiten der Anfangsbedingungen nach $x_2,\dots,x_3$
sind auch die zweiten Ableitungen 
\[
\partial_i\partial_ju(0,x_2,x_3)\qquad 2\le i\le 3,\;1\le j\le 3
\]
bekannt. Es ist also nur noch $\partial_1^2u$ zu bestimmen.
Die Differentialgleichung kann nach $\partial_1^2u$ aufgelöst
werden, wenn $a_{11}\ne 0$. Dies ist gleichbedeutend damit, dass
\[
a_{11}=\begin{pmatrix}
1&0&\dots&0
\end{pmatrix}
A
\begin{pmatrix}1\\0\\\vdots\\0\end{pmatrix}
=0.
\]

\subsection{Der allgemeine Fall}
Wir betrachten die Funktion $\omega(x_1,x_2,x_3)$ als die erste
Koordinaten eines neuen Koordinatensystems. Die Koordinaten in diesem
neuen System nennen wir $\omega_1,\dots,\omega_3$, sie sind Funktionen
der alten Koordinaten
\[
\omega_1(x_1,x_2,x_3)=\omega(x_1,x_2,x_3)
,\qquad
\omega_2(x_1,x_2,x_3)
,\qquad
\omega_3(x_1,x_2,x_3)
\]
Die Funktion $u$ kann natürlich auch in den neuen Koordinaten geschrieben
werden: $\tilde u(\omega_1,\omega_2,\omega_3)$ hat die Eigenschaft
\[
\tilde u(
\omega_1(x_1,x_2,x_3),
\omega_2(x_1,x_2,x_3),
\omega_3(x_1,x_2,x_3)) = u(x_1,x_2,x_3).
\]
Setzt man dies in die Differentialgleichung ein, ergibt sich
\begin{align*}
\partial_iu
&=
\sum_{k=1}^3
\frac{\partial\tilde u}{\partial \omega_k}
\frac{\partial\omega_k}{\partial x_i}
\\
\partial_i\partial_ju
&=
\sum_{k,l=1}^3
\frac{\partial^2\tilde u}{\partial \omega_k\partial\omega_l}
\frac{\partial\omega_k}{\partial x_i}
\frac{\partial\omega_l}{\partial x_j}
\\
\sum_{i,j=1}^3a_{ij}\partial_i\partial_ju
&=
\sum_{i,j,k,l=1}^3a_{ij}
\frac{\partial^2\tilde u}{\partial \omega_k\partial\omega_l}
\frac{\partial\omega_k}{\partial x_i}
\frac{\partial\omega_l}{\partial x_j}
\\
\sum_{i=1}^3b_i\partial_iu
&=
\sum_{i,k=1}^3b_i
\frac{\partial\tilde u}{\partial \omega_k}
\frac{\partial\omega_k}{\partial x_i},
\end{align*}
die Differentialgleichung für $\tilde u$ in den neuen 
Koordinaten lautet also
\[
\sum_{k,l=1}^3a_{ij}
\biggl(
\sum_{i,j=1}^3a_{ij}
\frac{\partial\omega_k}{\partial x_i}
\frac{\partial\omega_l}{\partial x_j}
\biggr)
\frac{\partial^2\tilde u}{\partial \omega_k\partial\omega_l}
+
\sum_{k=1}^3
\biggl(
\sum_{i=1}^3
b_i
\frac{\partial\omega_k}{\partial x_i}
\biggr)
\frac{\partial\tilde u}{\partial \omega_k}
+c\tilde u
=f.
\]
Die neuen Koeffizienten der zweiten Ableitungen sind also
\[
\tilde a_{kl}=
\sum_{i,j=1}^3a_{ij}
\frac{\partial\omega_k}{\partial x_i}
\frac{\partial\omega_l}{\partial x_j}
\]
Die Fläche entspricht in den $\omega$-Koordinaten dem Spezialfall $\omega_1=0$,
die zweiten Ableitungen sind also genau dann eindeutig bestimmt, wenn
der Koeffizient $\tilde a_{11}$ nicht verschwindet. Entlang der durch $\omega$
definierten Fläche sind also genau dann die zweiten Ableitungen nicht eindeutig
bestimmt, wenn 
\[
\sum_{i,j=1}^3
a_{ij}
\frac{\partial\omega}{\partial x_i}
\frac{\partial\omega}{\partial x_j}
=
\operatorname{grad}\omega
\cdot
A
\operatorname{grad}\omega
=0
\]
Da der Gradient senkrecht auf der Fläche steht, also alle Flächen
problematisch, deren Normalen $\vec n$ die Vektorgleichung
\[
\vec n\cdot A\vec n=0
\]
erfüllen. Diese Vektoren heissen {\em charakteristische Normalen}.

\begin{definition}
Eine durch $\omega(x_1,\dots,x_n)=0$ definierte Fläche heisst
charakteristische Fläche, wenn auf der Fläche
\[
\sum_{i,j=1}^na_{ij}\partial_i\omega\partial_j\omega=0
\]
gilt.
\end{definition}

\subsection{Charakteristische Flächen der Wellengleichung}
\begin{figure}
\begin{center}
\includegraphics[width=0.8\hsize]{../common/graphics/shock}
\end{center}
\caption{Schockwelle eines Überschallflugzeugs als Beispiel einer
charakteristischen Fläche von $\partial_t^2u-a^2\Delta u=0$.\label{ueberschallkegel}}
\end{figure}
Wir bestimmen die charakteristischen Flächen der
Wellengleichung
\[
\partial_t^2u-a^2\Delta u=0.
\]
Die Koeffizientenmatrix ist
\[
\begin{pmatrix}
1&0&0\\
0&-a^2&0\\
0&0&-a^2
\end{pmatrix},
\]
die charakteristischen Normalen sind also Vektoren $\vec v$, welche die
Gleichung
\[
v_1^2-a^2v_2^2-a^2v_3^2=0
\]
erfüllen. Diese beschreibt einen Doppelkegel, alle Vektoren, welche mit
der $x_1$-Achse einen festen Winkel einschliessen, sind charakteristische
Normalen. Der Winkel $\alpha$ muss der Bedingung
\[
\cos^2\alpha-a^2\sin^2\alpha=0
\]
genügen, also
\[
\tan\alpha=\pm\frac1a.
\]

Die Winkelbedingung ist die einzige Einschränkung an
die charakteristischen Flächen,  entsprechend gibt es eine
grosse Vielfalt:
\begin{enumerate}
\item
Jeder Kegel mit halbem Öffnungswinkel $\frac{\pi}2-\alpha$
und Achse parallel zur $x_1$-Achse ist eine charakteristische Fläche.
Der Kegel schneidet cie $x_2$-$x_3$-Ebene in einem Kreis, dessen Radius mit
grösser werdendem $x_1$ mit der Geschwindigkeit $a$ grösser wird.

Abbildung \ref{ueberschallkegel}
zeigt die Schockwelle eines Überschallflugzeuges. Schockwellen
als Unstetigkeiten müssen sich entlang der charakteristischen Flächen ausbreiten,
also entlang eines Kegels.
\item 
Jede Ebene, die mit der $x_1$-Achse einen Winkel von $\frac\pi2-\alpha$
einschliesst, ist charakteristische Fläche.
Die Ebene schneidet die $x_2$-$x_3$-Ebene in einer Geraden, die sich
mit der Geschwindigkeit $a$ senkrecht zur Geraden fortbewegt.
Diese charakteristische Fläche beschreibt die Fortpflanzung einer
ebenen Welle.
\end{enumerate}



%
% boundry.tex  -- 
%
% (c) 2019 Prof Dr Andreas Mueller
%
\section{Boundary conditions}
\index{boundary conditions!for quasilinear partial differential equations of first order}
The method of characteristics allows us to find out on which parts of the
boundary we have to specify boundary conditions to make the solution
unique.
The solution surface corresponding to $u(x,y)$ consists of characteristics
so it is uniquely determined if exactly one characteristic goes through
each point.

For the differential equation
\begin{equation}
\frac{\partial u}{\partial x}+2\frac{\partial u}{\partial y}=3
\label{geometrie:knickbeispiel}
\end{equation}
we have found the characteristics
\[
t\mapsto\begin{pmatrix}x_0\\y_0\\z_0\end{pmatrix}+t\begin{pmatrix}1\\2\\3\end{pmatrix}.
\]
The graph of the solution function $u(x,y)$ must be covered by characteristics.
The projections of these curves into the $x$-$y$-plane are straight lines
with slope $2$.
The boundary of the domain $\Omega$, in which the equation needs to be
solved, thus must intersect each straight line with slope $2$ exactly once.

The domain
$\{(x,y)\in\mathbb R^2\,|\, x >0\}$  from \ref{konstantekoeff}
has the $x$-axis as boundary which intersects every straight line with
slope exactly once.

The domain $\Omega=\{(x,y)\,|\,0<x,y<1\}$ is more interesting.
Figures \ref{geometrie:charrand1}
to \ref{geometrie:charrand3}
show various possibilities:
\begin{enumerate}
\item
In figure~\ref{geometrie:charrand1}
boundary values are prescribed on the left and right boundary.
These boundary values are not sufficient to determine the solution
everywhere in the domain.
There is a part not covered by characteristics where the solution
is not fixed.
\item
In figure~\ref{geometrie:charrand2}
boundary values are given on the top and bottom boundaries of the
square.
A solution is only possible if the values solution emanating from
the left half of the bottom boundary coincides with the solution
emanating from the right half of the top boundary.
If the boundary values on these parts are not compatible, no solution
exists.
\item
In figure~\ref{geometrie:charrand3}
the boundary values are specified on the left and bottom side
of the square.
This fixes the solution function uniquely.
Nevertheless it is not entirely clear that we have found a solution,
because it is still possible that the combined function is not
differentiable on the characteristic through the lower left
corner of the square.
\end{enumerate}
The last situation is very common, the solution we find here is
differentiable outside of a set of measure zero.
This is called a {\em weak} solution of the differential equation.
\index{weak solution}

\begin{figure}
\begin{center}
\includegraphics{3-geometry/images/underdetermined.pdf}
\end{center}
\caption{Boundary values on the left and right side: solution
not determined in the light green portion of the domain.
\label{geometrie:charrand1}}
\end{figure}

\begin{figure}
\begin{center}
\includegraphics{3-geometry/images/overdetermined.pdf}
\end{center}
\caption{Boundary values on the top and bottom side of the square:
solution overdetermined in the light green portioin of the domain
\label{geometrie:charrand2}}
\end{figure}

\begin{figure}
\begin{center}
\includegraphics{3-geometry/images/nondifferentiable.pdf}
\end{center}
\caption{Boundary values on the left and bottom sides of the square:
solution well defined but differentiability on the light green
characteristic emanating from $(0,0)$ is not guaranteed.
\label{geometrie:charrand3}}
\end{figure}

\begin{figure}
\centering
\begin{tikzpicture}[>=latex]
\node at (0,0) {%
\includegraphics[width=0.9\hsize]{../common/3d/knick.jpg}%
};
\node at (7,-2.4) {$x$};
\node at (-7.2,-1) {$y$};
\node at (-3.7,3) {$z$};
\end{tikzpicture}
\caption{The solution of the differential
equation~(\ref{geometrie:knickbeispiel}),
with boundary values along the left and bottom sides of the square
(see also figure~\ref{geometrie:charrand3})
is not differentiable along the characteristic through $(0,0)$
(vertical axis scaled by a factor $0.4$).
\label{geometrie:knick}}
\end{figure}

As an illustration of the last case consider the boundary values
\begin{align*}
u(x_0,0)&=0,\\
u(0,y_0)&=y_0.
\end{align*}
Each side fixes part of the solution in the unit square, we can get 
explicit formulas using the method of characteristics.
The characteristics starting from the bottom side are
\[
\left.
\begin{aligned}
x&=x_0+t\\
y&=2t\\
u&=3t
\end{aligned}
\right\}
\qquad\Rightarrow\qquad
\left\{
\begin{aligned}
t&=\frac12y\\
u&=\frac32y
\end{aligned}
\right.
\]
The characteristics from the left side of the square are
\[
\left.
\begin{aligned}
x&=t\\
y&=y_0+2t\\
u&=y_0+3t
\end{aligned}
\right\}
\qquad\Rightarrow\qquad
\left\{
\begin{aligned}
y_0&=y-2x\\
u&=y+x
\end{aligned}
\right.
\]
Thus the solution function is
\[
u(x,y)=\begin{cases}
\frac32y&\qquad y<2x\\
x+y&\qquad y>2x,
\end{cases}
\]
as displayed in figure~\ref{geometrie:knick}.
For $y=2x$, both terms coincide, so the solution function is continuous
in all of $\Omega$.
However, the partial function $x\mapsto u(x,y)$ has slope $1$
for $y>2x$ and slope $0$ for $y<2x$, so it is not differentiable at $y=2x$.



\section{Summary}
\begin{enumerate}
\item
The solution surface of a hyperbolic partial differential equation can be
covered by characteristic strips.
\item
Discontinuities or other singularities propagate along characteristics.
\item
As for quasilinear partial differential equations of first order,
characteristics can be used to determine which boundary values or
normal derivatives need to be specified for a hyperbolic problem to
be well posed.
\end{enumerate}

