%
% boundary.tex -- XXX
%
% (c) 2019 Prof Dr Andreas Mueller
%
\section{Welche Randwerte beeinflussen die Lösung in einem Punkt?}
\begin{figure}
\centering
\includegraphics{../common/images/kausal-1.pdf}\qquad\qquad%
\includegraphics{../common/images/kausal-2.pdf}
\caption{Randwerte und Punkte des Gebietes, die Einfluss auf einen
Funktionswert der Lösung haben.
Bei einer elliptischen Differentialgleichung $\Delta u=f$
(links)
hängt jeder Funktionswert von jedem Randwert und von jedem Wert der
Funktion $f$ ab.
Die Lösung einer quasilinearen partiellen Differentialgleichung  (rechts)
im Punkt $x$ hängt vom Randwert im Anfangspunkt $x_0$ einer Charaketeristik
ab, die durch $x$ verläuft.
\label{einflussmenge1}}
\end{figure}
\begin{figure}
\centering
\includegraphics{../common/images/kausal-3.pdf}\qquad\qquad%
\includegraphics{../common/images/kausal-4.pdf}
\caption{Randwerte und Punkte des Gebietes, die Einfluss auf einen
Funktionswert der Lösung haben.
Die Lösung einer parabolischen partiellen Differentialgleichung
zur Zeit $t$ hängt von allen Werten $t'<t$ ab.
Bei einer hyperbolischen partiellen Differentialgleichung beinflussen 
nur die Funktionswerte innerhalb eines vom Punkt $x$ ausgehenden Kegels
von Charakteristiken den Wert der Lösung im Punkt $x$.
\label{einflussmenge2}}
\end{figure}
Mit der entwickelten Theorie ist es jetzt möglich, einen Antwort auf die
Frage zu geben, welche Punkte auf dem Rand oder auch im Inneren des Gebietes
einen Einfluss auf die Lösung einer partiellen Differentialgleichung
der Form $Lu=f$ haben.

Im Kapitel~\ref{chapter-elliptisch} haben wir für die Lösung einer
elliptischen partiellen Differentialgleichung eine Formel angegeben,
die $u(x)$ aus allen Randwerten $g$ und der Funktion $f$ in allen 
inneren Punkten berechnet:
\[
u(x)
=
\int_\omega G(x,\xi)f(\xi)\,d\xi
+
\int_{\partial\Omega} \operatorname{grad}_\xi G(x,\xi)g(\xi)\,d\xi.
\]
Eine Änderung von $g$ in irgend einem Punkt des Randes wird im allgemeinen
Die Lösung $u(x)$ verändern, ebenso eine beliebige Änderung von $f$
(Abbildung~\ref{einflussmenge1}, links).

Im Gegensatz dazu wurde im Kapitel~\ref{chapter-geometrie} gezeigt, 
dass die Lösung einer quasilinearen partiellen Differentialgleichung
erster Ordnung im Punkt $x$ nur vom Wert der Anfangsbedingung im Anfangspunkt
derjenigen Charakteristik abhängt, die durch $x$ verläuft
(Abbildung~\ref{einflussmenge1}, rechts).

Die Lösung einer parabolischen partiellen Differentialgleichungen 
kann ebenfalls mit Hilfe einer Greenschen Funktion ausgedrückt werden.
Die Formel~(\ref{green-parabolisch}) zeigt, dass die Werte $u(t,x)$
im Allgemeinen von allen Randwerten $g(t',\xi)$ und $f(t',\xi)$
mit $t'<t$ abhängt (Abbildung~\ref{einflussmenge2}, links).

Die Charakteristischen Flächen (Kurven) beantworten die entsprechende
Frage für eine hyperbolische partielle Differentialgleichung.
Die charakteristischen Flächen (Kurven), die durch den Punkte $x$ verlaufen,
definieren einen Kegel.
Der Wert $u(x)$ der Lösung im Punkt $x$ hängt ab von den
Funktionswerten der rechten Seite der Gleichung und den Randwerten
in diesem Kegel
(Abbildung~\ref{einflussmenge2}, rechts).

