%
% separation.tex -- 
%
% (c) 2019 Prof Dr Andreas Mueller
%
\section{Separation}
We apply the wave equation to a domain of the form
$\Omega = G\times(0,\infty)$ where $G$ is a spacial domain.
By way of example the twodimensional shape of the membrane of a drum
does not change while it is being played.
In this case, it is possible to use separation.
The hyperbolic partial differential equation
\[
\frac1{a^2}\frac{\partial^2 u}{\partial t^2}=\Delta u
\]
can be attacked with the ansatz $u(x,t)=\varphi(x)\cdot T(t)$ and be 
separated into two differential equations
\begin{gather*}
\begin{aligned}
\frac1{a^2}T''(t)\varphi(x)&=T(t)\Delta \varphi(x)\\
\frac1{a^2}\frac{T''(t)}{T(t)}&=\frac{\Delta\varphi(x)}{\varphi(x)}=\lambda\\
\end{aligned}
\\
\Rightarrow\qquad
T''(t)=\lambda a^2T(t)\qquad\text{und}\qquad\Delta \varphi=\lambda\varphi.
\end{gather*}
The equation for $T$ is an ordinary oscillator equation and does
not pose any problems regarding existence and uniqueness of solutions.
If $T(0)$ and $T'(0)$ are given, the solution can be written down
for arbitrary $t$.

The second equation $\Delta \varphi-\lambda\varphi=0$ is an elliptic
eigenvalue problem that we have discussed in chapter~\ref{chapter:elliptic}.
For example we have seen that elliptic partial differential equations
on connected and bounded domains have unique solutions if boundary
values are specified along the complete boundary.

We can derive from this that hyperbolic problems have a solution if
we know $u(x,0)$ and $\partial_tu(x,0)$ for $x\in G$.
This matches the intuition gained from the example of the drum.

This method fails in cases where the shape of the spatial domain
changes over time.
An example would be a trombone which works by changing the oscillating
gas volume, which is the ``domain'' for a wave equation describing
the dynamics of the gas inside the trombone. 
Of course, nature apparently has no problem solving that more 
difficult wave equation, as the trombone player manages to play
complicated tunes without having to study the theory of the
wave equation first.

This simple situation changes radically if the shape of the domain
changes faster than the speed of sound, which of course never happens
for a trombone.
In this case, we will get shock waves which will be studied later.

