%
% separation.tex -- XXX
%
% (c) 2019 Prof Dr Andreas Mueller
%
\section{Separation}
Oft wird die Wellengleichung auf ein Gebiet der Form
$\Omega = G\times(0,\infty)$ angewandt, also ein 
festes Raumgebiet $G$.
Eine Pauke ändert zum Beispiel die Form des Felles während des
Spielens nicht. In diesen Fällen ist ein Separationsansatz möglich.
Die hyperbolische partielle Differentialgleichung 
\[
\frac1{a^2}\frac{\partial^2 u}{\partial t^2}=\Delta u
\]
kann mit dem Ansatz $u(x,t)=\varphi(x)\cdot T(t)$ in zwei
Differentialgleichungen zerlegt werden:
\begin{gather*}
\begin{aligned}
\frac1{a^2}T''(t)\varphi(x)&=T(t)\Delta \varphi(x)\\
\frac1{a^2}\frac{T''(t)}{T(t)}&=\frac{\Delta\varphi(x)}{\varphi(x)}=\lambda\\
\end{aligned}
\\
\Rightarrow\qquad
T''(t)=\lambda a^2T(t)\qquad\text{und}\qquad\Delta \varphi=\lambda\varphi.
\end{gather*}
Die Differentialgleichung für $T$ ist eine Schwingungsdifferentialgleichung
und stellt damit bezüglich Existenz und Eindeutigkeit der Lösung
kein grosses Problem dar. Wenn $T(0)$ und $T'(0)$ bekannt sind, dann
können wir die Lösung für beliebiges $t$ angeben.

Die zweite Gleichung $\Delta \varphi-\lambda\varphi=0$ ist eine
elliptische partielle Differentialgleichung, über welche wir
in Kapitel~\ref{chapter-elliptisch} einiges an Theorie entwickelt haben.
Zum Beispiel wurde dort gesagt, dass elliptische Probleme auf
zusammenhängenden und beschränkten Gebieten eine eindeutige
Lösung haben, wenn man die Randwerte entlang des gesamten Randes
kennt.

Wir können daraus schliessen, dass das hyperbolische Problem eine
Lösung hat, wenn wir $u(x,0)$ und $\partial_tu(x,0)$ für
$x\in G$ kennen, und ausserdem die Werte $u(x,t)$ für $x\in\partial G$.
Dies passt zu der am Paukenfell gewonnenen Intuition: dessen Schwingung
ist bekannt, wenn Anfangsauslenkung und -geschwindigkeit bekannt sind,
und ausserdem die Randwerte des Felles während der ganzen Zeit
bekannt sind.

Dieses Verfahren versagt aber in Fällen, wo sich das Gebiet verändert,
zum Beispiel bei der Expansion eines Gases.
Die langsame Expansion zum Beispiel in einer Posaune, deren ``Gebiet''
sich ja während des Spielens dauernd verändert, scheint dabei kein
Problem zu sein, jedenfalls kann ein Posaunist auch schwierige Passagen
ohne Studium der Wellengleichung meistern.
Was passiert aber, wenn der
Kolben, der eine Expansionsgefäss verschliesst, mit einer Geschwindigkeit
grösser als die Schallgeschwindigkeit bewegt wird? 

