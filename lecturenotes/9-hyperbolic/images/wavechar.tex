%
% wavechar.tex -- characteristics of the wave equation
%
% (c) 2019 Prof Dr Andreas Müller, Hochschule Rapperswil
%
\documentclass[tikz,12pt]{standalone}
\usepackage{amsmath}
\usepackage{times}
\usepackage{txfonts}
\usepackage{pgfplots}
\usepackage{csvsimple}
\usetikzlibrary{arrows,intersections,math}
\begin{document}
\definecolor{darkgreen}{rgb}{0,0.6,0}
\begin{tikzpicture}[>=latex]

\begin{scope}
\clip (-6.5,-0.5) rectangle (6.5,8.5);

\foreach \x in {-10,...,10}{
\draw[color=red,line width=1pt] ({\x-5},{-10})--({\x+5},{+10});
\draw[color=darkgreen,line width=1pt] ({\x-5},{+10})--({\x+5},{-10});
}

\end{scope}

\draw[->,line width=0.7pt] (-6.7,0)--(6.9,0) coordinate[label={$x$}];
\draw[->,line width=0.7pt] (0,-0.7)--(0,8.7) coordinate[label={right:$y$}];

\end{tikzpicture}
\end{document}
