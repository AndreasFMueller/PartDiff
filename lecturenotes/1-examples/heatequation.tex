%
% heatequation.tex -- heat equation
%
% (c) 2008 Prof Dr Andreas Mueller
%
\rhead{Heat equation}
\section{Heat equation}
\index{heat equation}
The heat equation is a prototype for a so called parabolic differential
equation.
\index{parabolic}
Such equations are also used to describe diffusion processes.
The Schrödinger equation, the basis of quantum mechanics, is also of this type.
\index{Schrödinger equation}
\index{quantum mechanics}

We derive the heat equation for the one dimensional case.
We examine a rod of length $l$ between $x$-coordinates $0$ and $l$
and strive to compute the temperature distribution $T(x,t)$ for 
$0<x<l$ and for all times $t>0$.
We need the initial temperature distribution, which we call
$T(x,0) = f(x)$.
We expect to solve this problem using a partial differential equation.

The heat that flows in a time interval $\Delta t$ through the rod at
coordinate $x$ is proportional to the temperature gradient at that point.
The amount of heat flowing into the section  of the rod between
$x$ and $x+\Delta x$ is therefore proportional to
\[
\underbrace{
\frac{\partial T}{\partial x}(x+\Delta x)}
_{\parbox{4cm}{\raggedright\centering heat flow into segment through $x+\Delta x$}}
-
\underbrace{\frac{\partial T}{\partial x}(x).}_{\parbox{3.2cm}{\raggedright
\centering
heat flow out of segment through $x$}}
\]
This additional heat lets the temperature of the segment rise, depending
on the heat capacity of the material.
The volume of the segment is $\Delta x$, the temperature change is smaller
when the volume is larger.
It becomes
\[
\Delta T
=
\kappa
\frac{1}{\Delta x}
\biggl(
\frac{\partial T}{\partial x}(x+\Delta x)-\frac{\partial T}{\partial x}(x).
\biggr) \Delta t,
\]
Where $\kappa$ combines the various physical constants involved into
a single proportionality factor.
Dividing by $\Delta t$ and going to the limits $\Delta t\to 0$
and $\Delta x\to 0$ leads to
\begin{equation}
\lim_{\Delta t\to 0}\frac{\Delta T}{\Delta t}
=
\frac{\partial T}{\partial t}
=
\kappa
\lim_{\Delta x\to 0}\frac1{\Delta x}\left(\frac{\partial T}{\partial x}(x+\Delta x)-\frac{\partial T}{\partial x}(x)\right)
=\kappa\frac{\partial^2T}{\partial x^2}.
\label{examples:heat-equation}
\end{equation}

Again we have to specify suitable initial and boundary conditions.
As initial condition we already have found
\[
T(x,0)=f(x)\qquad \forall x\in[0,l].
\]
As a boundary condition at the end of the rod we could prescribe the
temperature.
In physics terms this means that both ends of the rod are in contact
with a heat reservoir at constant temperature.
If we prescribe the derivatives of $T$ with respect to $x$ at $x=0$ and
$x=l$, we fix the flow of heat into the rod.
In particular, requiring
\[
\frac{\partial T}{\partial x} = 0
\]
means that no heat flows through the ends of the rod, the rod is
thermally isolated from its environment.

A particularly interesting case arises when we ask for a stationary
temperature distribution.
This is the equilibrium distribution a system reaches after sufficiently
long time.
Since in this case the partial derivative with respect to time 
vanishes, the temperature distribution satisfies
$\Delta T=0$, so we are back to the Poisson problem.

So far, we have not considered heat sources in the interior.
This situation models what happens inside the food in a microwave oven: the
microwave radiation is absorbed by the food and heats it up locally.
This can be modelled by adding an additional term on the right hand
side of the heat equation \eqref{examples:heat-equation}.
For the stationary distribution, we are back to the inhomgeneous Poisson
problem.



