%
% beamequation.tex -- beam equation
%
% (c) 2019 Prof Dr Andreas Mueller
%
\rhead{Beam equation}
\section{Beam equation}
\index{beam}
In the derivation of the differential equation of a string we
have used that fact that the string does not resist to being
bent.
Even large curvature of the string does not result in force trying
to straighten it out again.
A beam behaves completely differently.
Bending the beam creates internal stresses in the beam that
try to bring the beam back to its original shape.
We don't try to explain these forces, but it is natural to expect that
they will be proportional to the curvature and thus to the second
derivatives of the beam.
In addition, the are proportional to some material constants
(the elastic modulus $E$)
and to a property derived from the cross section of the beam,
namely the second moment $I$%
\footnote{The definition of $I$ is not relevant for the present
discussion.}.
\index{second moment}

A segment of length $\Delta x$ of a beam with linear mass density $m$
therefore experiences the following force components:
\begin{enumerate}
\item
Restoring forces caused by the stresses in the beam:
\index{stress}
$\displaystyle -EI\frac{\partial^4}{\partial^4 x}w(t,x)\Delta x$
\item
Damping
\index{damping}
$\displaystyle -b\frac{\partial}{\partial t}w(t,x)\Delta x$
\item
Exterior forces, e.~g.~loads on the beam:
$q(x,t)\Delta x$
\end{enumerate}
By Newton's law, these forces must sum up to mass times acceleration or
\[
m\Delta x\cdot \frac{\partial^2}{\partial^2 t}w(t,x).
\]
By going to the limit $\Delta x\to 0$ once more we get
\begin{align}
m\frac{\partial^2}{\partial^2t}w(t,x)
&=-EI\frac{\partial^4}{\partial^4x}w(t,x)-b\frac{\partial}{\partial t}w(t,x)+q(t,x)
\notag
\\
\Leftrightarrow\quad
EI\frac{\partial^4}{\partial^4x}w(t,x)
+b\frac{\partial}{\partial t}w(t,x)
+m\frac{\partial^2}{\partial^2t}w(t,x)
&=q(t,x)
\label{examples:beam-equation}
\end{align}
The motion of a beam therefore follows a partial differential equation.


