%
% supersonic.tex -- the equation for a supersonic flow
%
% (c) 2008 Prof Dr Andreas Mueller
%
\rhead{Supersonic flow}
\section{Supersonic flow}
In the year 1928, Jakob Ackeret habilitated at ETH Zprich with a 
paper with the title
``Über Luft-Kräfte bei sehr grossen
Geschwindigkeiten insbesondere bei ebenen Strömungen''.
He showed how to compute the aerodynamic forces on an object
in a supersonic flow using a linear approximation.
The velocity field of the gas that enters the region of interest with
velocity $v_1$ in $x$-direction turns out to be the gradient of
a function $\varphi(x,y,z)$ that satisfies the equation
\[
(1-Ma_1)\frac{\partial^2\varphi}{\partial x^2}
+
\frac{\partial^2\varphi}{\partial y^2}
+
\frac{\partial^2\varphi}{\partial z^2}=0.
\]
In the expression
$Ma_1=\frac{v_1}{c_1}$, $c_1$ is the speed of sound.

For small velocities, we have $(1-Ma_1)>0$, and the equation
becomes similar to the Poisson problem.
This is called potential flow.

For supersonic flow, however, the first term
$(1-Ma_1) < 0$
changes sign, and the equation behaves like a wave equation.
In fact, the flow shows show waves propagating away from the object
and thus carrying away it's energy
(figure~\ref{ueberschall2d}).
By carefully analyzing this solution, Ackeret was able to compute
aredynamic drag due to these shock waves.

