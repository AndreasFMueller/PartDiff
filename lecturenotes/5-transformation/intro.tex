%
% intro.tex -- XXX
%
% (c) 2019 Prof Dr Andreas Mueller, Hochschule Rapperswil
%
\section{Einführungsbeispiel}
\subsection{Wellengleichung für die schwingende Saite}
\rhead{Schwingende Saite}
Zur Einführung betrachten wir erneut das Beispiel der
schwingenden Saite mit der Gleichung
\[
\partial_t^2u=\partial_x^2u.
\]
Durch einen Separationsansatz hatten wir dieses Problem
darauf zurückgeführt, Gleichungen $X''=-k^2X$ und $T''=-k^2T$
zu lösen.
Dabei haben wir die harmonischen Funktionen wieder gefunden,
mit Hilfe der Fouriertheorie ist sodann die Bestimmung einer Lösung
gelungen.

Andererseits könnten wir auch mit der Fouriertheorie beginnen und
verwenden, dass sich eine Funktion auf dem Interval $[0,\pi]$
durch Spiegelung zu einer periodischen Funktion auf $[-\pi,0]$
erweitern lässt. Indem wir zu jedem Zeitpunkt $t$ eine Fourieranalyse
durchführen, lässt sich die Funktion $x\mapsto u(t,x)$ als
Fourierreihe 
\[
u(t,x)=\frac{a_0(t)}2+\sum_{k=1}^\infty a_k(t)\cos kx+b_k(t)\sin kx
\]
mit zeitabhängigen Koeffizienten schreiben.
Setzen wir diesen Ansatz in die Differentialgleichung ein, ergibt sich
\begin{align*}
\partial_t^2u(t,x)&=\frac{a_0''(t)}2
+\sum_{k=1}^\infty a_k''(t)\cos kx+b_k''(t)\sin kx\\
\partial_x^2u(t,x)&=
-\sum_{k=1}^\infty a_k(t)k^2\cos kx+b_k(t)k^2\sin kx
\end{align*}
Wenn diese Funktionen gleich sein sollen, muss
\[
\frac{a_0''(t)}2
+\sum_{k=1}^\infty (a_k''(t)+k^2a_k(t))\cos kx+(b_k''(t)+k^2b_k(t))\sin kx=0
\]
gelten. Nach der Fourier-Theorie ist dies nur möglich, wenn die
Fourier-Koeffizienten alle verschwinden:
\begin{align*}
a_0''(t)&=0\\
a_k''(t)&=-k^2a_k(t)\\
b_k''(t)&=-k^2b_k(t)
\end{align*}
mit $k>0$.
Wir haben also durch Übergang zu Fourierkoeffizienten ein System
von gewöhnlichen Differentialgleichungen für die Fourier-Koeffizienten
gefunden. Die Lösungen sind wohl bekannt:
\begin{align*}
a_0(t)&=m_0t+c_0\\
a_k(t)&=A^a_k\cos kt+B^a_k\sin kt\\
b_k(t)&=A^b_k\cos kt+B^b_k\sin kt\\
\end{align*}
Dies entspricht den Lösungen, die wir bereits im vorangegangenen
Kapitel gefunden haben.

\subsection{Anfangsbedingungen}
Die Differentialgleichungen für die Koeffizienten $a_k(t)$ und $b_k(t)$
können erst dann vollständig gelöst werden, wenn Anfangs oder Randbedingungen
gegeben sind. Sei also zusätzlich zur Wellengleichung die Anfangsbedingung
\begin{align*}
u(0,x)&=f(x)\\
\frac{\partial u}{\partial t}&=g(x)
\end{align*}
gegeben. auch die Funktionen $f$ und $g$ können duch eine Fourierreihe
dargestellt werden, werden wir schreiben
\begin{align*}
f(x)&=\frac{a_0^f}2+\sum_{k=1}^\infty a_k^f\cos kx+b_k^f\sin kx\\
g(x)&=\frac{a_0^g}2+\sum_{k=1}^\infty a_k^g\cos kx+b_k^g\sin kx
\end{align*}
Zusammen mit dem Ansatz für $u(t,x)$ als Fourierreihe finden wir jetzt
die Bedingungen für $t=0$
\begin{align*}
\frac{a_0(0)}2+\sum_{k=1}^\infty a_k(0)\cos kx +b_k(0)\sin kx
&=
\frac{a_0^f}2+\sum_{k=1}^\infty a_k^f\cos kx+b_k^f\sin kx\\
\\
\frac{a_0'(0)}2+\sum_{k=1}^\infty a_k'(0)\cos kx+b_k'(0)\sin kx
&=
\frac{a_0^g}2+\sum_{k=1}^\infty a_k^g\cos kx+b_k^g\sin kx
\end{align*}
Koeffizientenvergleich liefert jetzt die Anfangsbedingungen für die
Funktionen $a_k$ und $b_k$:
\begin{align*}
a_k(0)&=a_k^f&b_k(0)&=b_k^f\\
a_k'(0)&=a_k^g&b_k'(0)&=b_k^g\\
\end{align*}
Zusammen mit den früher gefundenen Lösungen gilt also
\begin{align*}
c_0&=a_0^f&&&A_k^a&=a_k^f&&&&&A_k^b&=b_k^f\\
m_0&=a_0^g&&&kB_k^a&=a_k^g&\Rightarrow B_k^a&=\frac1ka_k^g&&&kB_k^b&=b_k^g&\Rightarrow B_k^b&=\frac1kb_k^g
\end{align*}
Die vollständige Lösung ist damit
\begin{align*}
u(t,x)=\frac{a_0^gt+a_0^f}2
+\sum_{k=1}^\infty
\biggl(a_k^f\cos kt+\frac1ka_k^g\sin kt\biggr)\cos kx
+
\biggl(b_k^f\cos kt+\frac1kb_k^g\sin kt\biggr)\sin kx.
\end{align*}

\subsection{Inhomogene Wellengleichung}
Das Verfähren lässt sich auch die inhomogene Wellengleichung
\[
\partial_t^2u-\partial_x^2u=f
\]
verallgemeinern. Die Funktion $f(t,x)$ lässt sich natürlich ebenso wie
$u$ als Fourierreihe entwickeln:
\[
f(t,x)=\frac{a_0^f(t)}2+\sum_{k=1}^\infty a_k^f(t)\cos kx+b_k^f(t)\sin kx.
\]
Setzt man dies zusammen mit der Entwicklung für $u(t,x)$ in die
Differentialgleichung ein, findet man die Gleichungen
\begin{align*}
a''_k(t)+k^2a_k(t)&=a_k^f(t)\\
b''_k(t)+k^2b_k(t)&=b_k^f(t),
\end{align*}
also ein System von gewöhnlichen inhomogenen linearen Differentialgleichungen.

\subsection{Lehren aus dem Einführungsbeispiel}
Durch die Transformation auf die Fourier-Koeffizienten verschwindet die
Ableitung nach $x$,  stattdessen bleibt nur noch eine gewöhnliche
Differentialgleichung, die zweifache Ableitung nach $x$ wurde zur
algebraischen Operation der Multiplikation mit $-k^2$. Diese Transformation
war möglich, weil der $x$-Definitionsbereich ein Interval war.
Für andere Definitionsbereiche wird die Fourier-Transformation nicht
geeignet sein.

