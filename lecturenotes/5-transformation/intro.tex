%
% intro.tex -- 
%
% (c) 2019 Prof Dr Andreas Mueller, Hochschule Rapperswil
%
\section{Introductory example}
\subsection{Wave equation for a vibrating string}
\rhead{Vibrating string}
We introduce the ideas to the transformation method by applying it
once more to the wave equation of the vibrating string
\[
\partial_t^2u=\partial_x^2u.
\]
The separation method gave us two ordinary differential equations
$X''=-k^2X$ and $T''=-k^2T$ with the trigonometric functions as
solutions.
Combining the linearly lead us to using the Fourier series of the
initial conditions to find the solution.

However, we could begin right away with Fourier theory and
use that any function on the interval $[0,\pi]$ can be extended
in some way to a function on $[-\pi,0]$ and further to a periodic
function on $\mathbb R$.
By using Fourier analysis at each point in time, we get a
Fourier series for the functions $x\mapsto u(t,x)$
\[
u(t,x)
=
\frac{a_0(t)}2+\sum_{k=1}^\infty \bigl( a_k(t)\cos kx+b_k(t)\sin kx \bigr)
\]
with time dependent Fourier coefficients.
Substituting this into the wave equation we get
\begin{align*}
\partial_t^2u(t,x)&=\frac{a_0''(t)}2
+\sum_{k=1}^\infty\bigl( a_k''(t)\cos kx+b_k''(t)\sin kx\bigr)\\
\partial_x^2u(t,x)&=
-\sum_{k=1}^\infty \bigl(a_k(t)k^2\cos kx+b_k(t)k^2\sin kx \bigr)
\end{align*}
If these functions are the same the equations
the equation
\[
\frac{a_0''(t)}2
+\sum_{k=1}^\infty (a_k''(t)+k^2a_k(t))\cos kx+(b_k''(t)+k^2b_k(t))\sin kx=0
\]
must hold.
According to Fourier theory, this is only possible of all the coefficients
are $0$:
\begin{align*}
a_0''(t)&=0\\
a_k''(t)&=-k^2a_k(t)\\
b_k''(t)&=-k^2b_k(t)
\end{align*}
with $k>0$.
Thus by going to Fourier coefficients, a system of ordinary differential
equations for the Fourier coefficents was found.
The solution of these ordinary differential equations are well known:
\begin{align*}
a_0(t)&=m_0t+c_0\\
a_k(t)&=A^a_k\cos kt+B^a_k\sin kt\\
b_k(t)&=A^b_k\cos kt+B^b_k\sin kt
\end{align*}
These are the solutions that we have found in the previous chapter.
The parameters $A$ and $B$ need to be found from initial conditions.

\subsection{Initial conditions}
The differential equations for the coefficients $a_k(t)$ and $b_k(t)$
require some initial conditions in order for us to solve them uniquely.
Let's assume that the initial conditions are given in the form
\begin{align*}
u(0,x)&=f(x)\\
\frac{\partial u}{\partial t}&=g(x).
\end{align*}
The functions $f$ and $g$ can also be represented as Fourier series,
we write them as
\begin{align*}
f(x)&=\frac{a_0^f}2+\sum_{k=1}^\infty \bigl( a_k^f\cos kx+b_k^f\sin kx \bigr)\\
g(x)&=\frac{a_0^g}2+\sum_{k=1}^\infty \bigl( a_k^g\cos kx+b_k^g\sin kx \bigr).
\end{align*}
Together with the solution $u(t,x)$ written as a Fourier series 
we can now identify the coefficients by substiting $u(0,x)$ in
the initial conditions
\begin{align*}
\frac{a_0(0)}2+\sum_{k=1}^\infty \bigl( a_k(0)\cos kx +b_k(0)\sin kx \bigr)
&=
\frac{a_0^f}2+\sum_{k=1}^\infty \bigl( a_k^f\cos kx+b_k^f\sin kx \bigr)\\
\\
\frac{a_0'(0)}2+\sum_{k=1}^\infty \bigl( a_k'(0)\cos kx+b_k'(0)\sin kx \bigr)
&=
\frac{a_0^g}2+\sum_{k=1}^\infty \bigl( a_k^g\cos kx+b_k^g\sin kx \bigr).
\end{align*}
Comparing coefficients not gives us the initial conditions
for the functions $a_k(t)$ and $b_k(t)$:
\begin{align*}
a_k(0)&=a_k^f&b_k(0)&=b_k^f\\
a_k'(0)&=a_k^g&b_k'(0)&=b_k^g
\end{align*}
Using the previously found solutions we get
\begin{align*}
c_0&=a_0^f&&&A_k^a&=a_k^f&&&&&A_k^b&=b_k^f\\
m_0&=a_0^g&&&kB_k^a&=a_k^g&\Rightarrow B_k^a&=\frac1ka_k^g&&&kB_k^b&=b_k^g&\Rightarrow B_k^b&=\frac1kb_k^g.
\end{align*}
The complete solution thus is
\begin{align*}
u(t,x)=\frac{a_0^gt+a_0^f}2
+\sum_{k=1}^\infty \biggl(
\bigl(a_k^f\cos kt+{\textstyle\frac1k}a_k^g\sin kt\bigr)\cos kx
+
\bigl(b_k^f\cos kt+{\textstyle\frac1k}b_k^g\sin kt\bigr)\sin kx
\biggr).
\end{align*}

\subsection{Inhomogeneous wave equation}
The method can be generalized to the inhomogeneous wave equation
\[
\partial_t^2u-\partial_x^2u=f.
\]
The partial function $x\mapsto f(t,x)$ can also be written 
as a Fourier series in the variable $x$ just as we did for $u$:
\[
f(t,x)
=
\frac{a_0^f(t)}2
+
\sum_{k=1}^\infty \bigl( a_k^f(t)\cos kx+b_k^f(t)\sin kx \bigr).
\]
Substituting this into the inhomegeneous wave equation gives
\begin{align*}
a''_k(t)+k^2a_k(t)&=a_k^f(t)\\
b''_k(t)+k^2b_k(t)&=b_k^f(t),
\end{align*}
again a system of ordinary differential equations.
While the previous set of ordinary differential equations was linear
and homogeneous, the new set of equations is inhomogeneous.

\subsection{Lessons learned from the introductory example}
The transform to Fourier coefficients makes derivatives with respect
to $x$ disappear, instead what remains is a family of ordinary
differential equations.
The second derivative with respect to $x$ becomes the algebraic
operation of multipication by $-k^2$.
This transform was possible because the $x$-domain was an interval.
For other $x$-domains, this method will not work.

