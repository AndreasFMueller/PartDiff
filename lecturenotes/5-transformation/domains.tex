%
% domains.tex -- 
%
% (c) 2019 Prof Dr Andreas Mueller, Hochschule Rapperswil
%
\section{Transforms with more general types of domains}
Integral transforms are useful to help partial differential equations,
but only if the transform matches the domain:
\begin{center}
\begin{tabular}{cl}
Domain&Transform\\
\hline
$[0,\infty)$&Laplace transform\\
$\mathbb R$&Fourier transform\\
$[-\pi,\pi]$&Fourier series
\end{tabular}
\end{center}
The separation method has shown how this technique can be generalized.
We illustrated the basic principle for the heat equation on a domain
$G\subset \mathbb R^n$.
The differential equation is
\[
\partial_t u(t,x)=\Delta u(t,x)
\]
for $(t,x)\in [0,\infty)\times G$ with initial conditions
\[
u(0,x)=u_0(x) \quad x\in G
\]
and boundary conditions
\[
au(t,x)+b\partial_nu(t,x)=0\quad (t,x)\in[0,\infty)\times \partial G.
\]
Chapter~\ref{chapter:separation} suggests that we can find a family
of functions
$u_k(x)$ which are eigenfunctions of the Laplace operator
\[
\Delta u_k=\lambda_k u_k
\]
and satisfy homogeneous boundary conditions.
It also turns out that these functions can be used to expand any
function on $G$ into a series
\[
u_0(x)=\sum_{k=1}^\infty a_ku_k(x).
\]
No we can immediately write down the solution to the heat equation:
\[
u(t,x)=\sum_{k=1}e^{-\lambda_kt}u_k(x)
\]
Apparently, the functions $u_k$ take over the role the functions
$e^{ikx}$ had in the Fourier theory.

Modern mathematics knows a theory of harmonic analysis on so called
Lie groups, which generalize and unify Laplace transform and Fourier
transform and which allow to handle more general domains.
As an example, expansions on $n$-dimensional spheres are possible
using so called spherical functions.

