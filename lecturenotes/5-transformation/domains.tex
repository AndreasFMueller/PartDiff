%
% domains.tex -- XXX
%
% (c) 2019 Prof Dr Andreas Mueller, Hochschule Rapperswil
%
\section{Transformation bei beliebigen Gebieten}
Mit Hilfe von Integraltransformationen können partielle
Differentialgleichungen  gelöst werden, wenn die Transformation
an das Definitionsgebiet angepasst ist:
\begin{center}
\begin{tabular}{cl}
Definitionsgebiet&Transformation\\
\hline
$[0,\infty)$&Laplacetransformation\\
$\mathbb R$&Fouriertransformation\\
$[-\pi,\pi]$&Fourierreihen
\end{tabular}
\end{center}
Die Separationsmethode zeigt uns aber auch, wie dieses 
Verfahren verallgemeinert werden kann. Die Wärmeleitungsgleichung
soll auf einem Gebiet $G\subset \mathbb R^n$ gelöst werden, d.~h.~es
soll eine Lösung der Differentialgleichung
\[
\partial_t u(t,x)=\Delta u(t,x)
\]
gefunden werden für $(t,x)\in [0,\infty)\times G$ mit der Anfangsbedingung
\[
u(0,x)=u_0(x) \quad x\in G
\]
und der Randbedingung
\[
au(t,x)+b\partial_nu(t,x)=0\quad (t,x)\in[0,\infty)\times \partial G
\]
gefunden werden. Kapitel \ref{chapter-separation} suggeriert, dass
es eine Familie von Funktionen $u_k(x)$ gibt, die der Gleichung
\[
\Delta u_k=\lambda_k u_k
\]
und der Randbedingung genügen, und mit denen sich die 
Anfangsbedingung als Reihe schreiben lässt:
\[
u_0(x)=\sum_{k=1}^\infty a_ku_k(x).
\]
Dann kann man die Lösung der Wärmeleitungsgleichung sofort
hinschreiben:
\[
u(t,x)=\sum_{k=1}e^{-\lambda_kt}u_k(x)
\]
Die Funktionen spielen die Rolle von $e^{ikx}$ in der Fouriertheorie.

Die moderne Mathematik kennt eine allgemeine Theorie der harmonischen
Analyse auf sogenannten Lie-Gruppen, welche die Fourier- und 
Laplace-Theorie verallgemeinert und vereinheitlicht, und welche
weitere Gebiete zu behandeln erlaubt. Zum Beispiel ermöglicht
die Entwicklung von Funktionen auf einer ($n$-dimensionalen)
Kugel mit sogenannten Kugelfunktionen die Idee der Fourierreihen
auf einem Interval auf höherdimensionale Gebiete zu verallgemeinern.

