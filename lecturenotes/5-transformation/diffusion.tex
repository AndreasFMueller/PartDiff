%
% diffusion.tex -- XXX
%
% (c) 2019 Prof Dr Andreas Mueller, Hochschule Rapperswil
%
\section{Diffusion}
\rhead{Diffusion}
Wir betrachten jetzt die Diffusionsgleichung auf der Halbebene
$(t,x)\in[0,\infty)\times \mathbb R$
\[
\partial_tu-\partial_x^2u=f(t,x)
\]
mit den Anfangsbedingungen
\[
u(0,x)=g(x)\quad \forall x\in\mathbb R.
\]
Wir wenden jetzt die jeweils an den Definitionsbereich angepasste
Transformation durch.

\subsection{Fouriertransformation}
Die Fouriertransformation für eine Funktion $g(x)$ von $x$ ist
\[
\hat g(k)=\frac1{\sqrt{2\pi}}\int_{\mathbb R}g(x)e^{-ikx}\,dx.
\]
Die Fouriertransformation macht aus den gegebenen Gleichungen
\begin{align*}
\partial_t \hat u(t,k)+k^2\hat u(t,k)&=\hat f(t,k)\\
\hat u(0,k)&=\hat g(k)
\end{align*}

\subsection{Laplace-Transformation}
In $t$-Richtung wenden wir eine Laplace-Transformation an
\[
-\hat u(0,k)+(s+k^2){\cal L}\hat u(s,k)={\cal L}\hat f(s,k)
\]
Diese Gleichung kann nach ${\cal L}\hat u(s,k)$ auflösen:
\[
{\cal L}\hat u(s,k)=\frac{\hat u(0,k)+{\cal L}\hat f(s,k)}{s+k^2}
\]

\subsection{Lösung des Anfangswertproblems}
Durch Rücktransformation können jetzt auch die Lösungen wieder
gewonnen werden.
\begin{align*}
\hat u(t,k)&={\cal L}^{-1}
\frac{\hat u(0,k)+{\cal L}\hat f(s,k)}{s+k^2}
\\
u(t,x)&={\cal F}^{-1}{\cal L}^{-1}
\left(\frac{\hat u(0,k)+{\cal L}\hat f(s,k)}{s+k^2}\right)(t,x)
\end{align*}

