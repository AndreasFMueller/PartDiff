%
% transformation.tex -- XXX
%
% (c) 2019 Prof Dr Andreas Mueller, Hochschule Rapperswil
%
\chapter{Transformation}
\lhead{Transformation}
Bei der Lösung gewöhnlicher linearer Differentialgleichungen lieferte die
Laplacetransformation eine hervorragende Methode, mit der sehr
viele Anfangswertprobleme gelöst werden konnten.
In diesem Kapitel wird gezeigt, wie diese Methode wie auch
Fouriertransformation oder Fourierreihen 
für lineare partielle Differentialgleichungen nutzbar gemacht
werden können.

%
% intro.tex -- 
%
% (c) 2019 Prof Dr Andreas Mueller, Hochschule Rapperswil
%
\section{Introductory example}
\subsection{Wave equation for a vibrating string}
\rhead{Vibrating string}
We introduce the ideas to the transformation method by applying it
once more to the wave equation of the vibrating string
\[
\partial_t^2u=\partial_x^2u.
\]
The separation method gave us two ordinary differential equations
$X''=-k^2X$ and $T''=-k^2T$ with the trigonometric functions as
solutions.
Combining the linearly lead us to using the Fourier series of the
initial conditions to find the solution.

However, we could begin right away with Fourier theory and
use that any function on the interval $[0,\pi]$ can be extended
in some way to a function on $[-\pi,0]$ and further to a periodic
function on $\mathbb R$.
By using Fourier analysis at each point in time, we get a
Fourier series for the functions $x\mapsto u(t,x)$
\[
u(t,x)
=
\frac{a_0(t)}2+\sum_{k=1}^\infty \bigl( a_k(t)\cos kx+b_k(t)\sin kx \bigr)
\]
with time dependent Fourier coefficients.
Substituting this into the wave equation we get
\begin{align*}
\partial_t^2u(t,x)&=\frac{a_0''(t)}2
+\sum_{k=1}^\infty\bigl( a_k''(t)\cos kx+b_k''(t)\sin kx\bigr)\\
\partial_x^2u(t,x)&=
-\sum_{k=1}^\infty \bigl(a_k(t)k^2\cos kx+b_k(t)k^2\sin kx \bigr)
\end{align*}
If these functions are the same the equations
the equation
\[
\frac{a_0''(t)}2
+\sum_{k=1}^\infty (a_k''(t)+k^2a_k(t))\cos kx+(b_k''(t)+k^2b_k(t))\sin kx=0
\]
must hold.
According to Fourier theory, this is only possible of all the coefficients
are $0$:
\begin{align*}
a_0''(t)&=0\\
a_k''(t)&=-k^2a_k(t)\\
b_k''(t)&=-k^2b_k(t)
\end{align*}
with $k>0$.
Thus by going to Fourier coefficients, a system of ordinary differential
equations for the Fourier coefficents was found.
The solution of these ordinary differential equations are well known:
\begin{align*}
a_0(t)&=m_0t+c_0\\
a_k(t)&=A^a_k\cos kt+B^a_k\sin kt\\
b_k(t)&=A^b_k\cos kt+B^b_k\sin kt
\end{align*}
These are the solutions that we have found in the previous chapter.
The parameters $A$ and $B$ need to be found from initial conditions.

\subsection{Initial conditions}
The differential equations for the coefficients $a_k(t)$ and $b_k(t)$
require some initial conditions in order for us to solve them uniquely.
Let's assume that the initial conditions are given in the form
\begin{align*}
u(0,x)&=f(x)\\
\frac{\partial u}{\partial t}&=g(x).
\end{align*}
The functions $f$ and $g$ can also be represented as Fourier series,
we write them as
\begin{align*}
f(x)&=\frac{a_0^f}2+\sum_{k=1}^\infty \bigl( a_k^f\cos kx+b_k^f\sin kx \bigr)\\
g(x)&=\frac{a_0^g}2+\sum_{k=1}^\infty \bigl( a_k^g\cos kx+b_k^g\sin kx \bigr).
\end{align*}
Together with the solution $u(t,x)$ written as a Fourier series 
we can now identify the coefficients by substiting $u(0,x)$ in
the initial conditions
\begin{align*}
\frac{a_0(0)}2+\sum_{k=1}^\infty \bigl( a_k(0)\cos kx +b_k(0)\sin kx \bigr)
&=
\frac{a_0^f}2+\sum_{k=1}^\infty \bigl( a_k^f\cos kx+b_k^f\sin kx \bigr)\\
\\
\frac{a_0'(0)}2+\sum_{k=1}^\infty \bigl( a_k'(0)\cos kx+b_k'(0)\sin kx \bigr)
&=
\frac{a_0^g}2+\sum_{k=1}^\infty \bigl( a_k^g\cos kx+b_k^g\sin kx \bigr).
\end{align*}
Comparing coefficients not gives us the initial conditions
for the functions $a_k(t)$ and $b_k(t)$:
\begin{align*}
a_k(0)&=a_k^f&b_k(0)&=b_k^f\\
a_k'(0)&=a_k^g&b_k'(0)&=b_k^g
\end{align*}
Using the previously found solutions we get
\begin{align*}
c_0&=a_0^f&&&A_k^a&=a_k^f&&&&&A_k^b&=b_k^f\\
m_0&=a_0^g&&&kB_k^a&=a_k^g&\Rightarrow B_k^a&=\frac1ka_k^g&&&kB_k^b&=b_k^g&\Rightarrow B_k^b&=\frac1kb_k^g.
\end{align*}
The complete solution thus is
\begin{align*}
u(t,x)=\frac{a_0^gt+a_0^f}2
+\sum_{k=1}^\infty \biggl(
\bigl(a_k^f\cos kt+{\textstyle\frac1k}a_k^g\sin kt\bigr)\cos kx
+
\bigl(b_k^f\cos kt+{\textstyle\frac1k}b_k^g\sin kt\bigr)\sin kx
\biggr).
\end{align*}

\subsection{Inhomogeneous wave equation}
The method can be generalized to the inhomogeneous wave equation
\[
\partial_t^2u-\partial_x^2u=f.
\]
The partial function $x\mapsto f(t,x)$ can also be written 
as a Fourier series in the variable $x$ just as we did for $u$:
\[
f(t,x)
=
\frac{a_0^f(t)}2
+
\sum_{k=1}^\infty \bigl( a_k^f(t)\cos kx+b_k^f(t)\sin kx \bigr).
\]
Substituting this into the inhomegeneous wave equation gives
\begin{align*}
a''_k(t)+k^2a_k(t)&=a_k^f(t)\\
b''_k(t)+k^2b_k(t)&=b_k^f(t),
\end{align*}
again a system of ordinary differential equations.
While the previous set of ordinary differential equations was linear
and homogeneous, the new set of equations is inhomogeneous.

\subsection{Lessons learned from the introductory example}
The transform to Fourier coefficients makes derivatives with respect
to $x$ disappear, instead what remains is a family of ordinary
differential equations.
The second derivative with respect to $x$ becomes the algebraic
operation of multipication by $-k^2$.
This transform was possible because the $x$-domain was an interval.
For other $x$-domains, this method will not work.


%
% laplace.tex -- 
%
% (c) 2019 Prof Dr Andreas Mueller, Hochschule Rapperswil
%
\section{Laplace transformation crash course}
For the domain of definition $[0,\infty]$, the Laplace transform
replaces the Fourier transform.
This section contains a short crash course to present the most important
ideas of the Laplace transform and its application to the solution
of ordinary differential equation.
The key here is that the Laplace transform converts derivatives into
much more easily solved algebraic equations.
Their solution can then be transformed back to give a solution of the
original differential equation.

\subsection{Definition}
\begin{definition}
The Laplace transform ${\cal L}f$ of a function
$f\colon[0,\infty)\to\mathbb R$ is the function
\[
({\cal L}f)(s)=\int_0^\infty f(t)e^{-st}\,dt.
\]
\end{definition}
Apparently ${\cal L}f$ is well defined for any bounded function $f$.
It is also a linear operation by definition.

\subsection{Examples}
The following three examples of Laplace transforms will turn out to
be useful.
\begin{enumerate}
\item
Let $f(t)=c$ be the constant function.
Its Lapace transform is
\[
({\cal L}f)(s)
=
\int_0^\infty ce^{-st}\,dt
=
\left[
-\frac{c}{s}e^{-st}
\right]_0^\infty=\frac{c}{s}.
\]

\item
Let $f(t)=e^{-kt}$ with $k>0$ be a bounded exponential function.
Its Laplace transform is
\[
({\cal L}f)(s)=\int_0^\infty e^{-kt}e^{-st}\,dt
=\int_0^{\infty}e^{-(s+k)t}\,dt
=
\left[
- \frac1{s+k}e^{-(s+k)t}
\right]_0^\infty
=\frac1{s+k}.
\]

\item
We compute the Laplace transform of the derivative $f'(t)$ of a function
$f(t)$, namely
$({\cal L}f')(s)$:
\begin{align*}
({\cal L}f')(s)
&=
\int_0^\infty f'(t)e^{-st}\,dt
\\
&=
\left[f(t)e^{-ts}\right]_0^\infty
+
\int_0^\infty sf(t)e^{-st}\,dt
\\
&=
-f(0)+s\int_0^\infty f(t)e^{-st}\,dt
\\
&=s({\cal L}f)(s)-f(0).
\end{align*}
As promised, the Laplace transform converts a derivative into an
algebraic operation.
\end{enumerate}

\subsection{Solving ordinary differential equations}
We want to solve the ordinary differential equation
\[
\dot x(t)+px(t)=f(t)
\]
with initial condition $x(0)=0$.

The Lapalace transform of the differential equation is
\begin{equation}
s({\cal L}x)(s)-x(0)+p({\cal L}x)(s)=({\cal L}f)(s).
\label{laplace:transformeddgl}
\end{equation}
We have to determine $x$, so we first solve the
equation~\eqref{laplace:transformeddgl} for $mathcal{L}x$ and
then try to transform back.

Solving form $(\mathcal{L}x)(x)$ gives
\[
({\cal L}x)(s)
=
\frac{({\cal L}f)(s)}{s+p}
\]
To retrieve the solution $x(t)$, we have to transform it back.
We do this for $f(t)=q$, using the transform
\[
({\cal L}f)(s)=\frac{q}{s}
\]
and thus 
\[
({\cal L}x)(s)
=
\frac{q}{s(s+p)}=\frac{q}{p}\frac{1}{s}-\frac{q}{p}\frac{1}{s+p}
\]
Both terms on the right can be dealt with by the examples
above.
The reverse transform then becomes
\[
x(t)=\frac{q}{p}-\frac{q}{p}e^{-pt}=\frac{q}{p}(1-e^{-pt}).
\]

\subsection{Solving linear partial differential equations}
The Laplace transform method also works for linear partial differential
equations.
As an example, let's solve the first order equation
\[
\frac{\partial u}{\partial t}+x\frac{\partial u}{\partial x}=x
\]
where $u$ is defined for $t\ge 0$, $x\ge 0$ and we require the
boundary conditions
\begin{align*}
u(x,0)&=0\qquad x>0\\
u(0,t)&=0\qquad t>0\\
\end{align*}
The Laplace transform of the differential equation with respect to
the variable $t$ is
\begin{equation}
s({\cal L}u)(s,x)-u(0,x)+x\frac{\partial}{\partial x}({\cal L}u)(s,x)
=
\frac{x}{s}.
\label{laplace:transformedpde}
\end{equation}
The transform has made the time derivative disappear, instead it has
turned into multiplication by $s$.
For each $s$, the \eqref{laplace:transformedpde} is an ordinary
differential equationfor the function $x\mapsto ({\cal L}u)(s,x)$
with initial condition
$({\cal L}u)(s,0)=0$.
The solution can be found with any method and is
\[
({\cal L}u)(s,x)=\frac{x}{s(s+1)}=\frac{x}{s}-\frac{x}{s+1}.
\]
Both terms can be transformed back, the solution becomes
\[
u(t,x)=x-xe^{-t}=x(1-e^{-t}).
\]
To check this result, we compute the derivatives
\begin{align*}
\frac{\partial u}{\partial t}
&=
xe^{-t}
\\
\frac{\partial u}{\partial x}
&=
1-e^{-t}.
\end{align*}
Substituting in the differential equation gives
\[
xe^{-t}+x(1-e^{-t})=x,
\]
the differential equation holds, and thanks to
\begin{align*}
u(x,0)
&=
0
\\
u(0,t)
&=
0
\end{align*}
the boundary condition holds, too.


%
% heat.tex -- XXX
%
% (c) 2019 Prof Dr Andreas Mueller, Hochschule Rapperswil
%
\section{Wärmeleitung auf einem Kreis}
\rhead{Wärmeleitung}
Wir betrachten jetzt die Wärmeleitungsgleichung auf einem Drahtring
mit gegebenen Anfangswerten. Die Temperaturverteilung auf dem
Kreis beschreiben wir als eine periodische Funktion auf dem Interval
$[-\pi,\pi]$. Gesucht ist daher eine Lösung der Wärmelteitungsgleichung
mit den Anfangsbedingungen
\begin{align*}
\partial_t u(t,x)&=\partial_x^2 u(t,x) &&\forall(t,x)\in\mathbb R\times[-\pi,\pi]\\
u(0,x)&=f(x)&& \forall x\in[-\pi,\pi]\\
\end{align*}

\subsection{Fouriertransformation auf dem Interval}
Wie im einführenden Beispiel gehen wir jetzt zu Fourierkoeffizienten
über.
Der Einfachheit halber verwenden wir komplexe Fourierkoeffizienten
\[
c_k=\hat f(k)=\frac1{2\pi}\int_{-\pi}^{\pi}e^{-ikx}f(x)\,dx
\]
Wenden wir die Transformation auf die ursprüngliche Differentialgleichung
an, erhalten wir
\begin{align}
\frac1{2\pi}\int_{-\pi}^{\pi}e^{-ikx}\partial_t u(t,x)\,dx
&=\frac1{2\pi}\int_{-\pi}^{\pi}e^{-ikx}\partial_x^2 u(t,x)\,dx
\notag
\\
\partial_t\frac1{2\pi}\int_{-\pi}^{\pi}e^{-ikx} u(t,x)\,dx
&=
\frac1{2\pi}
\underbrace{
\left[
e^{-ikx}\partial_x u(t,x)
\right]_{-\pi}^{\pi}}_{=0}
+
ik\frac1{2\pi}\int_{-\pi}^{\pi}e^{-ikx}\partial_x u(t,x)\,dx
\notag
\\
&=
0 + 
ik\frac1{2\pi}
\underbrace{
\left[
e^{-ikx}u(t,x)
\right]_{-\pi}^{\pi}}_{=0}
-
k^2\frac1{2\pi}\int_{-\pi}^{\pi}e^{-ikx} u(t,x)\,dx
\notag
\\
&=
-
k^2\frac1{2\pi}\int_{-\pi}^{\pi}e^{-ikx} u(t,x)\,dx
\notag
\\
\partial_t\hat u(t,k)&=-k^2 \hat u(t,k)
\label{fouriertransformiert}
\end{align}
Aus der partiellen Differentialgleichung ist ein System von gewöhnlichen
Differentialgleichungen für $t\ge 0$ geworden.

\subsection{Laplacetransformation}
Durch Laplace-Transformation der Gleichung (\ref{fouriertransformiert})
erhalten wir die Gleichung
\begin{align*}
\int_0^{\infty} \partial_t \hat u(t,k)e^{-st}\,dt
&=
-k^2\int_0^{\infty}\hat u(t,k)e^{-st}\,dt
\\
\left[\hat u(t,k)e^{-st}\right]_0^{\infty}
+
s\int_0^{\infty}\hat u(t,k)e^{-st}\,dt
&=
-k^2{\cal L}\hat u(s,k)
\\
\hat u(0,k)-s{\cal L}\hat u(s,k)&=k^2{\cal L}\hat u(s,k).
\\
{\cal L}\hat u(s,k)&=\frac{\hat u(0,k)}{s+k^2}
\end{align*}

\subsection{Rücktransformation}
Um die allgemeine Lösung der Differentialgleichung zu finden,
muss man jetzt rücktransformieren, zunächst nach Laplace:
\begin{align*}
\hat u(t,k)&={\cal L}^{-1}\left(\frac1{s+k^2}\right)\hat u(0,k)
\\
&=e^{-k^2t}\hat u(0,k)
\end{align*}
und dann auch noch nach Fourier, also durch Bilden der Fourierreihe
\begin{align*}
u(t,x)&=\sum_{k\in\mathbb Z}e^{-k^2t}\hat u(0,k)e^{ikx}
\\
&=\sum_{k\in\mathbb Z}\hat u(0,k)e^{ikx-k^2t}
\end{align*}


%
% diffusion.tex -- 
%
% (c) 2019 Prof Dr Andreas Mueller, Hochschule Rapperswil
%
\section{Diffusion}
\rhead{Diffusion}
We now consider the diffusion equation in the half plane
$(t,x)\in[0,\infty)\times \mathbb R$
\[
\partial_tu-\partial_x^2u=f(t,x)
\]
with initial condition
\[
u(0,x)=g(x)\quad \forall x\in\mathbb R.
\]
We apply the transforms suited to the particular domain of definition.

\subsection{Fourier transform}
The Fourier transform or a function $g(x)$ defined on $\mathbb R$ is
defined as
\[
\hat g(k)=\frac1{\sqrt{2\pi}}\int_{\mathbb R}g(x)e^{-ikx}\,dx.
\]
The Fourier transform turns the diffusion equation and the initial
condition into
\begin{align*}
\partial_t \hat u(t,k)+k^2\hat u(t,k)&=\hat f(t,k)\\
\hat u(0,k)&=\hat g(k)
\end{align*}

\subsection{Laplace transform}
In $t$-direction, we apply the Laplace transform to get
\[
-\hat u(0,k)+(s+k^2){\cal L}\hat u(s,k)={\cal L}\hat f(s,k).
\]
This equation can be solved for ${\cal L}\hat u(s,k)$:
\[
{\cal L}\hat u(s,k)=\frac{\hat u(0,k)+{\cal L}\hat f(s,k)}{s+k^2}.
\]

\subsection{Solution of the initial value problem}
The reverse transforms can now be used to recover the solution
\begin{align*}
\hat u(t,k)&={\cal L}^{-1}
\frac{\hat u(0,k)+{\cal L}\hat f(s,k)}{s+k^2}
\\
u(t,x)&={\cal F}^{-1}{\cal L}^{-1}
\left(\frac{\hat u(0,k)+{\cal L}\hat f(s,k)}{s+k^2}\right)(t,x).
\end{align*}


%
% domains.tex -- 
%
% (c) 2019 Prof Dr Andreas Mueller, Hochschule Rapperswil
%
\section{Transforms with more general types of domains}
Integral transforms are useful to help partial differential equations,
but only if the transform matches the domain:
\begin{center}
\begin{tabular}{cl}
Domain&Transform\\
\hline
$[0,\infty)$&Laplace transform\\
$\mathbb R$&Fourier transform\\
$[-\pi,\pi]$&Fourier series
\end{tabular}
\end{center}
The separation method has shown how this technique can be generalized.
We illustrated the basic principle for the heat equation on a domain
$G\subset \mathbb R^n$.
The differential equation is
\[
\partial_t u(t,x)=\Delta u(t,x)
\]
for $(t,x)\in [0,\infty)\times G$ with initial conditions
\[
u(0,x)=u_0(x) \quad x\in G
\]
and boundary conditions
\[
au(t,x)+b\partial_nu(t,x)=0\quad (t,x)\in[0,\infty)\times \partial G.
\]
Chapter~\ref{chapter:separation} suggests that we can find a family
of functions
$u_k(x)$ which are eigenfunctions of the Laplace operator
\[
\Delta u_k=\lambda_k u_k
\]
and satisfy homogeneous boundary conditions.
It also turns out that these functions can be used to expand any
function on $G$ into a series
\[
u_0(x)=\sum_{k=1}^\infty a_ku_k(x).
\]
No we can immediately write down the solution to the heat equation:
\[
u(t,x)=\sum_{k=1}e^{-\lambda_kt}u_k(x)
\]
Apparently, the functions $u_k$ take over the role the functions
$e^{ikx}$ had in the Fourier theory.

Modern mathematics knows a theory of harmonic analysis on so called
Lie groups, which generalize and unify Laplace transform and Fourier
transform and which allow to handle more general domains.
As an example, expansions on $n$-dimensional spheres are possible
using so called spherical functions.



\section{Zusammenfassung: das Wichtigste in Kürze}
\begin{enumerate}
\item Der Übergang von Funktionen zu Fourierreihen verwandelt
eine partielle Differentialgleichung in eine Famile gewöhnlicher
Differentialgleichung für die einzelnen Fourier-Koeffizienten.
\item Integraltransformationen können ein partielle Differentialgleichung
in eine Familie partieller Differentialgleichungen mit weniger Variablen
oder sogar gewöhnlicher Differentialgleichungen verwandeln.
\item Integraltransformationen und die Rücktransformationen können
Formeln für die Lösungen gewisser partieller Differentialgleichungen
liefern, und damit die Frage beantworten, für welche Randwertvorgaben
die Gleichungen gut gestellt sind.
\end{enumerate}
