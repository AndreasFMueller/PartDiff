%
% transformation.tex -- XXX
%
% (c) 2019 Prof Dr Andreas Mueller, Hochschule Rapperswil
%
\chapter{Transformation}
\lhead{Transformation}
Bei der Lösung gewöhnlicher linearer Differentialgleichungen lieferte die
Laplacetransformation eine hervorragende Methode, mit der sehr
viele Anfangswertprobleme gelöst werden konnten.
In diesem Kapitel wird gezeigt, wie diese Methode wie auch
Fouriertransformation oder Fourierreihen 
für lineare partielle Differentialgleichungen nutzbar gemacht
werden können.

%
% intro.tex -- XXX
%
% (c) 2019 Prof Dr Andreas Mueller
%
\section{Einführung}
\rhead{Einführung}
In den bisherigen Kapiteln waren die Differentialgleichungen auf sehr
speziellen Gebieten definiert gewesen, zum Beispiel Rechtecken, Kreissscheiben,
unendlichen Streifen, Halbebenen oder auf ganz $\mathbb R^n$. 
Die Anwendungen verlangen aber, dass elliptische partielle
Differentialgleichungen, auf fast beliebigen Gebieten gelöst werden können.
Insbesondere wollen wir erstehen, unter welchen Bedingungen eine
Lösung existiert und eindeutig bestimmt ist.

\subsection{Laplace-Gleichung}
\index{Laplace-Gleichung}
Als Beispiel für die Eigenschaften der Lösungen
%\marginpar{\tiny Formulierung des Beispielproblems}
elliptischer linearer partieller Differentialgleichungen
betrachten wir in diesem Kapitel die Laplace-Gleichung
auf einem Gebiet $\Omega\subset\mathbb R^n$
\begin{align}
\Delta u&=f &&\text{in $\Omega$}
\label{elliptisch:laplaceequation}
\\
\intertext{mit Dirichlet Randbedingungen}
u&=g && \text{auf $\partial\Omega$},
\label{dirichletrandbedingung}
\\
\intertext{Neumann Randbedingungen}
\frac{\partial u}{\partial n}&=g && \text{auf $\partial\Omega$}
\\
\intertext{oder der gemischten Randbedingungen}
\alpha u + \beta\frac{\partial u}{\partial n}&=g && \text{auf $\partial\Omega$.}
\label{elliptisch:gemischt}
\end{align}
Eine Lösung dieses Problems ist eine auf $\Omega$
zweimal stetig differenzierbare
Funktion $u$, welche stetig auf den Rand ausgedehnt werden kann,
und dort die Randbedingungen erfüllt. Zur Vereinfachung der
Diskussion betrachten wir in diesem Abschnitt nur die
Dirichlet-Randbedingungen (\ref{dirichletrandbedingung}).

Aus der allgemeinen Theorie linearer partieller Differentialgleichungen
%\marginpar{\tiny Allgemeine Lösungsstrategie}
kann man ableiten, dass wir für
die Lösung des gestellten Problems wie folgt vorgehen müssen
\begin{enumerate}
\item Zunächst brauchen wir eine partikuläre
Lösung $u_p$ der Laplace-Gleichung (\ref{elliptisch:laplaceequation}).
\item Mit $u_p$ reduziert sich das Problem darauf, eine Lösung
$u_r=u-u_p$ der homogenen Gleichung $\Delta u_r=0$
zu finden, welche die Randbedingung
\[
u_r=g-u_p
\]
erfüllt.
\item
Weitere Lösungen des Problems unterscheiden sich von dem
bereits gefundenen Problem nur durch eine Funktion $u_h$, welche
$\Delta u_h=0$ in $\Omega$ und $u_h=0$ auf $\partial\Omega$ erfüllt.
\end{enumerate}
Wir werden in diesem Kapitel zeigen, dass die Lösung diese Problems
eindeutig ist, dass also das einzige mögliche $v=0$ ist.
Ausserdem werden wir Formeln für partikuläre Lösungen wie auch für
das homogene Problem aufstellen.

\subsection{Harmonische Funktionen}
\index{harmonische Funktion}
Angesichts der Bedeutung der Lösungen des homogenen Problems
geben wir diesen einen speziellen Namen:

\begin{definition}
Eine zweimal stetig differenzierbare Funktion $u$ auf dem Gebiet $\Omega$
heisst harmonisch, wenn $\Delta u=0$.
\end{definition}


%
% laplace.tex -- XXX
%
% (c) 2019 Prof Dr Andreas Mueller, Hochschule Rapperswil
%
\section{Laplace-Transformation Crash-Course}
Für $[0,\infty[$ als Definitionsbereich ist die Laplace-Transformation
die geeignete Transformation. Dieser Abschnitt enthält einen Kurzabriss
der wesentlichen Ideen der Theorie der Laplace-Transformation.
Für uns wesentlich ist dabei, dass die Transformation die Ableitungen
in algebraische Operationen umwandelt. Eine algebraische Gleichung ist
einfacher zu lösen, die Rücktransformation liefert daraus die Lösung.

\subsection{Definition}
\begin{definition}
Die Laplace-Transformierte ${\cal L}f$ einer Funktion
$f\colon[0,\infty)\to\mathbb R$ ist die Funktion
\[
({\cal L}f)(s)=\int_0^\infty f(t)e^{-st}\,dt.
\]
\end{definition}
Offenbar ist ${\cal L}f$ für beliebige beschränkte Funktionen $f$ definiert.

\subsection{Beispiele}
Drei Beispiele sollen das Rechnen mit der Laplace-Transformation
illustrieren.
\begin{enumerate}
\item
Sei $f(t)=c$ eine konstante Funktion. Die Laplace-Transformation
\[
({\cal L}f)(s)
=
\int_0^\infty ce^{-st}\,dt
=
\left[
-\frac{c}{s}e^{-st}
\right]_0^\infty=\frac{c}{s}.
\]

\item
Sei jetzt $f(t)=e^{-kt}$, die Laplace-Transformation ist
\[
({\cal L}f)(s)=\int_0^\infty e^{-kt}e^{-st}\,dt
=\int_0^{\infty}e^{-(s+k)t}\,dt
=
\left[
- \frac1{s+k}e^{-(s+k)t}
\right]_0^\infty
=\frac1{s+k}.
\]

\item
Wir berechnen jetzt die Laplace-Transformierte einer Ableitung, also
$({\cal L}f')(s)$:
\begin{align*}
({\cal L}f')(s)
&=
\int_0^\infty f'(t)e^{-st}\,dt
\\
&=
\left[f(t)e^{-ts}\right]_0^\infty
+
\int_0^\infty sf(t)e^{-st}\,dt
\\
&=
-f(0)+s\int_0^\infty f(t)e^{-st}\,dt
\\
&=s({\cal L}f)(s)-f(0).
\end{align*}
Wie versprochen wandelt die Laplace-Transformation die Ableitung
ein eine algebraische Operation um.
\end{enumerate}

\subsection{Lösung einer gewöhnlichen Differentialgleichung}
Wir lösen die Differentialgleichung
\[
\dot x(t)+px(t)=f(t)
\]
mit der Anfangsbedingung $x(0)=0$.
Die Laplace-Transformation der Differentialgleichung ist
\[
s({\cal L}x)(s)-x(0)+p({\cal L}x)(s)=({\cal L}f)(s).
\]
Zu bestimmen ist $x$, diese Gleichung ist eine algebraische Gleichung
für ${\cal L}x$, die man auch nach ${\cal L}x$ auflösen kann:
\[
({\cal L}x)(s)
=
\frac{({\cal L}f)(s)}{s+p}
\]
Um zu zeigen, wie mit dieser Methode die vollständige Lösung gefunden
werden kann, führen wir die Rücktransformation für $f(t)=q$ durch.
Zunächst ist in diesem Fall
\[
({\cal L}f)(s)=\frac{q}{s}
\]
und damit 
\[
({\cal L}x)(s)
=
\frac{q}{s(s+p)}=\frac{q}{p}\frac{1}{s}-\frac{q}{p}\frac{1}{s+p}
\]
Beide Termen auf der rechten Seite können mit den Beispielen im vorangegangenen
Abschnitt rücktransformiert werden:
\[
x(t)=\frac{q}{p}-\frac{q}{p}e^{-pt}=\frac{q}{p}(1-e^{-pt}).
\]

\subsection{Lösung einer partiellen Differentialgleichung}
Dieses Verfahren funktioniert auch für partielle Differentialgleichungen.
Als Beispiel lösen wir die quaslineare Differentialgleichung
\[
\frac{\partial u}{\partial t}+x\frac{\partial u}{\partial x}=x
\]
$u$ ist definiert für $t\ge0$, $x\ge 0$ und wir verlangen die Randbedingungen
\begin{align*}
u(x,0)&=0\qquad x>0\\
u(0,t)&=0\qquad t>0\\
\end{align*}
Die Laplacetransformation der Differentialgleichung ist
\[
s({\cal L}u)(s,x)-u(0,x)+x\frac{\partial}{\partial x}({\cal L}u)(s,x)=\frac{x}{s},
\]
sie hat die Ableitung nach der Zeit zum Verschwinden gebracht. Für jedes
$s$ kann man dies als eine gewöhnliche Differentialgleichung für die
Funktion $x\mapsto ({\cal L}u)(s,x)$ mit der Anfangsbedingung
$({\cal L}u)(s,0)=0$ betrachten.
Die Lösung mit den Standardmethoden 
ergibt
\[
({\cal L}u)(s,x)=\frac{x}{s(s+1)}=\frac{x}{s}-\frac{x}{s+1}.
\]
Beide Terme können rücktransformiert werden, die Lösung ist
\[
u(t,x)=x-xe^{-t}=x(1-e^{-t}).
\]
Wir berechnen zur Kontrolle die Ableitungen
\begin{align*}
\frac{\partial u}{\partial t}
&=
xe^{-t}
\\
\frac{\partial u}{\partial x}
&=
1-e^{-t},
\end{align*}
damit 
\[
xe^{-t}+x(1-e^{-t})=x,
\]
die Differentialgleichung ist also erfüllt, und wegen
\begin{align*}
u(x,0)
&=
0
\\
u(0,t)
&=
0
\end{align*}
sind auch die Randbedingungen erfüllt.


%
% heat.tex -- 
%
% (c) 2019 Prof Dr Andreas Mueller, Hochschule Rapperswil
%
\section{Heat equation on a circle}
\rhead{Heat equation}
We now consider the heat equation on a piece of wire formed into
a circle with given initial temperatures.
In polar coordinates, the temperature distribution can be considered
a function on the interval $[-\pi,\pi]$, and can also be considered
a $2\pi$-periodic function on $\mathbb R$.
Thus we are to solve the heat equation
\begin{align*}
\partial_t u(t,x)&=\partial_x^2 u(t,x) &&\forall(t,x)\in\mathbb R\times[-\pi,\pi]\\
\intertext{with initial conditions}
u(0,x)&=f(x)&& \forall x\in[-\pi,\pi]\\
\end{align*}

\subsection{Fourier transform on an interval}
Just like in the introductory example, we transform to Fourier coefficients.
For simplicity, we are using complex Fourier coefficients defined by
\[
c_k=\hat f(k)=\frac1{2\pi}\int_{-\pi}^{\pi}e^{-ikx}f(x)\,dx.
\]
Applying the transform to the original differential equation gives
\begin{align}
\frac1{2\pi}\int_{-\pi}^{\pi}e^{-ikx}\partial_t u(t,x)\,dx
&=\frac1{2\pi}\int_{-\pi}^{\pi}e^{-ikx}\partial_x^2 u(t,x)\,dx
\notag
\\
\partial_x\frac1{2\pi}\int_{-\pi}^{\pi}e^{-ikx} u(t,x)\,dx
&=
\frac1{2\pi}
\underbrace{
\left[
e^{-ikx}\partial_x u(t,x)
\right]_{-\pi}^{\pi}}_{=0}
+
ik\frac1{2\pi}\int_{-\pi}^{\pi}e^{-ikx}\partial_x u(t,x)\,dx
\notag
\\
&=
0 + 
ik\frac1{2\pi}
\underbrace{
\left[
e^{-ikx}u(t,x)
\right]_{-\pi}^{\pi}}_{=0}
-
k^2\frac1{2\pi}\int_{-\pi}^{\pi}e^{-ikx} u(t,x)\,dx
\notag
\\
&=
-
k^2\frac1{2\pi}\int_{-\pi}^{\pi}e^{-ikx} u(t,x)\,dx
\notag
\\
\partial_x\hat u(t,k)&=-k^2 \hat u(t,k)
\label{fouriertransformiert}
\end{align}
The partial differential equation transforms into a family of ordinary
differential equation parametrized by $t\ge 0$.

\subsection{Laplace transform}
We now apply the Laplace transform to the equation
\eqref{fouriertransformiert}
to get
\begin{align*}
\int_0^{\infty} \partial_t \hat u(t,k)e^{-st}\,dt
&=
-k^2\int_0^{\infty}\hat u(t,k)e^{-st}\,dt
\\
\left[\hat u(t,k)e^{-st}\right]_0^{\infty}
+
s\int_0^{\infty}\hat u(t,k)e^{-st}\,dt
&=
-k^2{\cal L}\hat u(s,k)
\\
\hat u(0,k)-s{\cal L}\hat u(s,k)&=k^2{\cal L}\hat u(s,k)
\\
{\cal L}\hat u(s,k)&=\frac{\hat u(0,k)}{s+k^2}.
\end{align*}

\subsection{Reverse transform}
To find the general solution of the differential equation, 
we now have to find the reverse transform.
In the first step we reverse the Laplace transform to
formally get
\begin{align*}
\hat u(t,k)&={\cal L}^{-1}\left(\frac1{s+k^2}\right)\hat u(0,k)
\\
&=e^{-k^2t}\hat u(0,k).
\end{align*}
The second step is reversing the Fourier transform by summing the
Fourier series
\begin{align*}
u(t,x)&=\sum_{k\in\mathbb Z}e^{-k^2t}\hat u(0,k)e^{ikx}
\\
&=\sum_{k\in\mathbb Z}\hat u(0,k)e^{ikx-k^2t}.
\end{align*}


%
% diffusion.tex -- XXX
%
% (c) 2019 Prof Dr Andreas Mueller, Hochschule Rapperswil
%
\section{Diffusion}
\rhead{Diffusion}
Wir betrachten jetzt die Diffusionsgleichung auf der Halbebene
$(t,x)\in[0,\infty)\times \mathbb R$
\[
\partial_tu-\partial_x^2u=f(t,x)
\]
mit den Anfangsbedingungen
\[
u(0,x)=g(x)\quad \forall x\in\mathbb R.
\]
Wir wenden jetzt die jeweils an den Definitionsbereich angepasste
Transformation durch.

\subsection{Fouriertransformation}
Die Fouriertransformation für eine Funktion $g(x)$ von $x$ ist
\[
\hat g(k)=\frac1{\sqrt{2\pi}}\int_{\mathbb R}g(x)e^{-ikx}\,dx.
\]
Die Fouriertransformation macht aus den gegebenen Gleichungen
\begin{align*}
\partial_t \hat u(t,k)+k^2\hat u(t,k)&=\hat f(t,k)\\
\hat u(0,k)&=\hat g(k)
\end{align*}

\subsection{Laplace-Transformation}
In $t$-Richtung wenden wir eine Laplace-Transformation an
\[
-\hat u(0,k)+(s+k^2){\cal L}\hat u(s,k)={\cal L}\hat f(s,k)
\]
Diese Gleichung kann nach ${\cal L}\hat u(s,k)$ auflösen:
\[
{\cal L}\hat u(s,k)=\frac{\hat u(0,k)+{\cal L}\hat f(s,k)}{s+k^2}
\]

\subsection{Lösung des Anfangswertproblems}
Durch Rücktransformation können jetzt auch die Lösungen wieder
gewonnen werden.
\begin{align*}
\hat u(t,k)&={\cal L}^{-1}
\frac{\hat u(0,k)+{\cal L}\hat f(s,k)}{s+k^2}
\\
u(t,x)&={\cal F}^{-1}{\cal L}^{-1}
\left(\frac{\hat u(0,k)+{\cal L}\hat f(s,k)}{s+k^2}\right)(t,x)
\end{align*}


%
% domains.tex -- 
%
% (c) 2019 Prof Dr Andreas Mueller, Hochschule Rapperswil
%
\section{Transforms with more general types of domains}
Integral transforms are useful to help partial differential equations,
but only if the transform matches the domain:
\begin{center}
\begin{tabular}{cl}
Domain&Transform\\
\hline
$[0,\infty)$&Laplace transform\\
$\mathbb R$&Fourier transform\\
$[-\pi,\pi]$&Fourier series
\end{tabular}
\end{center}
The separation method has shown how this technique can be generalized.
We illustrated the basic principle for the heat equation on a domain
$G\subset \mathbb R^n$.
The differential equation is
\[
\partial_t u(t,x)=\Delta u(t,x)
\]
for $(t,x)\in [0,\infty)\times G$ with initial conditions
\[
u(0,x)=u_0(x) \quad x\in G
\]
and boundary conditions
\[
au(t,x)+b\partial_nu(t,x)=0\quad (t,x)\in[0,\infty)\times \partial G.
\]
Chapter~\ref{chapter-separation} suggests that we can find a family
of functions
$u_k(x)$ which are eigenfunctions of the Laplace operator
\[
\Delta u_k=\lambda_k u_k
\]
and satisfy homogeneous boundary conditions.
It also turns out that these functions can be used to expand any
function on $G$ into a series
\[
u_0(x)=\sum_{k=1}^\infty a_ku_k(x).
\]
No we can immediately write down the solution to the heat equation:
\[
u(t,x)=\sum_{k=1}e^{-\lambda_kt}u_k(x)
\]
Apparently, the functions $u_k$ take over the role the functions
$e^{ikx}$ had in the Fourier theory.

Modern mathematics knows a theory of harmonic analysis on so called
Lie groups, which generalize and unify Laplace transform and Fourier
transform and which allow to handle more general domains.
As an example, expansions on $n$-dimensional spheres are possible
using so called spherical functions.



\section{Zusammenfassung: das Wichtigste in Kürze}
\begin{enumerate}
\item Der Übergang von Funktionen zu Fourierreihen verwandelt
eine partielle Differentialgleichung in eine Famile gewöhnlicher
Differentialgleichung für die einzelnen Fourier-Koeffizienten.
\item Integraltransformationen können ein partielle Differentialgleichung
in eine Familie partieller Differentialgleichungen mit weniger Variablen
oder sogar gewöhnlicher Differentialgleichungen verwandeln.
\item Integraltransformationen und die Rücktransformationen können
Formeln für die Lösungen gewisser partieller Differentialgleichungen
liefern, und damit die Frage beantworten, für welche Randwertvorgaben
die Gleichungen gut gestellt sind.
\end{enumerate}
