%
% laplace.tex -- XXX
%
% (c) 2019 Prof Dr Andreas Mueller, Hochschule Rapperswil
%
\section{Laplace-Transformation Crash-Course}
Für $[0,\infty[$ als Definitionsbereich ist die Laplace-Transformation
die geeignete Transformation. Dieser Abschnitt enthält einen Kurzabriss
der wesentlichen Ideen der Theorie der Laplace-Transformation.
Für uns wesentlich ist dabei, dass die Transformation die Ableitungen
in algebraische Operationen umwandelt. Eine algebraische Gleichung ist
einfacher zu lösen, die Rücktransformation liefert daraus die Lösung.

\subsection{Definition}
\begin{definition}
Die Laplace-Transformierte ${\cal L}f$ einer Funktion
$f\colon[0,\infty)\to\mathbb R$ ist die Funktion
\[
({\cal L}f)(s)=\int_0^\infty f(t)e^{-st}\,dt.
\]
\end{definition}
Offenbar ist ${\cal L}f$ für beliebige beschränkte Funktionen $f$ definiert.

\subsection{Beispiele}
Drei Beispiele sollen das Rechnen mit der Laplace-Transformation
illustrieren.
\begin{enumerate}
\item
Sei $f(t)=c$ eine konstante Funktion. Die Laplace-Transformation
\[
({\cal L}f)(s)
=
\int_0^\infty ce^{-st}\,dt
=
\left[
-\frac{c}{s}e^{-st}
\right]_0^\infty=\frac{c}{s}.
\]

\item
Sei jetzt $f(t)=e^{-kt}$, die Laplace-Transformation ist
\[
({\cal L}f)(s)=\int_0^\infty e^{-kt}e^{-st}\,dt
=\int_0^{\infty}e^{-(s+k)t}\,dt
=
\left[
- \frac1{s+k}e^{-(s+k)t}
\right]_0^\infty
=\frac1{s+k}.
\]

\item
Wir berechnen jetzt die Laplace-Transformierte einer Ableitung, also
$({\cal L}f')(s)$:
\begin{align*}
({\cal L}f')(s)
&=
\int_0^\infty f'(t)e^{-st}\,dt
\\
&=
\left[f(t)e^{-ts}\right]_0^\infty
+
\int_0^\infty sf(t)e^{-st}\,dt
\\
&=
-f(0)+s\int_0^\infty f(t)e^{-st}\,dt
\\
&=s({\cal L}f)(s)-f(0).
\end{align*}
Wie versprochen wandelt die Laplace-Transformation die Ableitung
ein eine algebraische Operation um.
\end{enumerate}

\subsection{Lösung einer gewöhnlichen Differentialgleichung}
Wir lösen die Differentialgleichung
\[
\dot x(t)+px(t)=f(t)
\]
mit der Anfangsbedingung $x(0)=0$.
Die Laplace-Transformation der Differentialgleichung ist
\[
s({\cal L}x)(s)-x(0)+p({\cal L}x)(s)=({\cal L}f)(s).
\]
Zu bestimmen ist $x$, diese Gleichung ist eine algebraische Gleichung
für ${\cal L}x$, die man auch nach ${\cal L}x$ auflösen kann:
\[
({\cal L}x)(s)
=
\frac{({\cal L}f)(s)}{s+p}
\]
Um zu zeigen, wie mit dieser Methode die vollständige Lösung gefunden
werden kann, führen wir die Rücktransformation für $f(t)=q$ durch.
Zunächst ist in diesem Fall
\[
({\cal L}f)(s)=\frac{q}{s}
\]
und damit 
\[
({\cal L}x)(s)
=
\frac{q}{s(s+p)}=\frac{q}{p}\frac{1}{s}-\frac{q}{p}\frac{1}{s+p}
\]
Beide Termen auf der rechten Seite können mit den Beispielen im vorangegangenen
Abschnitt rücktransformiert werden:
\[
x(t)=\frac{q}{p}-\frac{q}{p}e^{-pt}=\frac{q}{p}(1-e^{-pt}).
\]

\subsection{Lösung einer partiellen Differentialgleichung}
Dieses Verfahren funktioniert auch für partielle Differentialgleichungen.
Als Beispiel lösen wir die quaslineare Differentialgleichung
\[
\frac{\partial u}{\partial t}+x\frac{\partial u}{\partial x}=x
\]
$u$ ist definiert für $t\ge0$, $x\ge 0$ und wir verlangen die Randbedingungen
\begin{align*}
u(x,0)&=0\qquad x>0\\
u(0,t)&=0\qquad t>0\\
\end{align*}
Die Laplacetransformation der Differentialgleichung ist
\[
s({\cal L}u)(s,x)-u(0,x)+x\frac{\partial}{\partial x}({\cal L}u)(s,x)=\frac{x}{s},
\]
sie hat die Ableitung nach der Zeit zum Verschwinden gebracht. Für jedes
$s$ kann man dies als eine gewöhnliche Differentialgleichung für die
Funktion $x\mapsto ({\cal L}u)(s,x)$ mit der Anfangsbedingung
$({\cal L}u)(s,0)=0$ betrachten.
Die Lösung mit den Standardmethoden 
ergibt
\[
({\cal L}u)(s,x)=\frac{x}{s(s+1)}=\frac{x}{s}-\frac{x}{s+1}.
\]
Beide Terme können rücktransformiert werden, die Lösung ist
\[
u(t,x)=x-xe^{-t}=x(1-e^{-t}).
\]
Wir berechnen zur Kontrolle die Ableitungen
\begin{align*}
\frac{\partial u}{\partial t}
&=
xe^{-t}
\\
\frac{\partial u}{\partial x}
&=
1-e^{-t},
\end{align*}
damit 
\[
xe^{-t}+x(1-e^{-t})=x,
\]
die Differentialgleichung ist also erfüllt, und wegen
\begin{align*}
u(x,0)
&=
0
\\
u(0,t)
&=
0
\end{align*}
sind auch die Randbedingungen erfüllt.

