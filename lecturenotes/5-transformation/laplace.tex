%
% laplace.tex -- 
%
% (c) 2019 Prof Dr Andreas Mueller, Hochschule Rapperswil
%
\section{Laplace transformation crash course}
For the domain of definition $[0,\infty]$, the Laplace transform
replaces the Fourier transform.
This section contains a short crash course to present the most important
ideas of the Laplace transform and its application to the solution
of ordinary differential equation.
The key here is that the Laplace transform converts derivatives into
much more easily solved algebraic operations.
This turns partial differential equations into partial differential equations
with fewer derivatives or even ordinary differential equations.
Their solution can then be transformed back to give a solution of the
original differential equation.

\subsection{Definition}
\begin{definition}
The Laplace transform $\mathscr{L}f$ of a function
$f\colon[0,\infty)\to\mathbb R$ is the function
\[
(\mathscr{L}f)(s)=\int_0^\infty f(t)e^{-st}\,dt.
\]
\end{definition}
Apparently $\mathscr{L}f$ is well defined for any bounded function $f$.
It is also a linear operation by definition.

\subsection{Examples}
The following three examples of Laplace transforms will turn out to
be useful.
\begin{enumerate}
\item
Let $f(t)=c$ be the constant function.
Its Lapace transform is
\[
(\mathscr{L}f)(s)
=
\int_0^\infty ce^{-st}\,dt
=
\left[
-\frac{c}{s}e^{-st}
\right]_0^\infty=\frac{c}{s}.
\]

\item
Let $f(t)=e^{-kt}$ with $k>0$ be a bounded exponential function.
Its Laplace transform is
\[
(\mathscr{L}f)(s)=\int_0^\infty e^{-kt}e^{-st}\,dt
=\int_0^{\infty}e^{-(s+k)t}\,dt
=
\left[
- \frac1{s+k}e^{-(s+k)t}
\right]_0^\infty
=\frac1{s+k}.
\]

\item
We compute the Laplace transform of the derivative $f'(t)$ of a function
$f(t)$, namely
$(\mathscr{L}f')(s)$:
\begin{align*}
(\mathscr{L}f')(s)
&=
\int_0^\infty f'(t)e^{-st}\,dt
\\
&=
\left[f(t)e^{-ts}\right]_0^\infty
+
\int_0^\infty sf(t)e^{-st}\,dt
\\
&=
-f(0)+s\int_0^\infty f(t)e^{-st}\,dt
\\
&=s(\mathscr{L}f)(s)-f(0).
\end{align*}
As promised, the Laplace transform converts a derivative into an
algebraic operation.
\item The last formula can be iterated to give the Laplace transform of
a second or higher order derivative.
First by writing $f'=g$ we can derive
\begin{align*}
(\mathscr{L}f'')(s)
&=
(\mathscr{L}g')(s)
=
s(\mathscr{L}g)(s)-g(0)
\\
&=
s(\mathscr{L}f')(s)-g'(0)
=
s\biggl( s\mathscr{L}f(s)-f(0)\biggr) - f'(0)
\\
&=
s^2(\mathscr{L}f)(s) - sf(0) -f'(0).
\intertext{Using the same process, we can get a recursion
formula for the Laplace transform of an $n$-th order derivative.
Set $g=f^{(n-1)}$ and use $g'=f^{n}$ to simplify}
(\mathscr{L}f^{(n)})(s)
&=
(\mathscr{L}g')(s)
=
s(\mathscr{L}g)(s)
-g(0)
=
s(\mathscr{L}f^{(n-1)})(s) - f^{n-1}(0).
\intertext{Applying this formula repeatedly, we get the general formula
for the Laplace transform of the $n$-th order derivative}
(\mathscr{L}f^{(n)})(s)
&=
s^n(\mathscr{L}f)(s)
-
s^{n-1}f(0)
-
s^{n-2}f'(0)
-
\dots
-
f^{(n-1)}(0).
\end{align*}
\end{enumerate}

\subsection{Solving ordinary differential equations}
We want to solve the ordinary differential equation
\[
\dot x(t)+px(t)=f(t)
\]
with initial condition $x(0)=0$.

The Lapalace transform of the differential equation is
\begin{equation}
s(\mathscr{L}x)(s)-x(0)+p(\mathscr{L}x)(s)=(\mathscr{L}f)(s).
\label{laplace:transformeddgl}
\end{equation}
We have to determine $x$, so we first solve the
equation~\eqref{laplace:transformeddgl} for $\mathscr{L}x$ and
then try to transform back.

Solving for $(\mathscr{L}x)(s)$ gives
\[
(\mathscr{L}x)(s)
=
\frac{(\mathscr{L}f)(s)}{s+p}
\]
To retrieve the solution $x(t)$, we have to transform it back.
We do this for $f(t)=q$, using the transform
\[
(\mathscr{L}f)(s)=\frac{q}{s}.
\]
We get
\begin{align*}
(\mathscr{L}x)(s)
&=
\frac{q}{s(s+p)}
\end{align*}
and try to decompose these fractions using the partial fraction
decomposition:
\begin{align*}
(\mathscr{L}x)(s)
&=
\frac{A}{s} + \frac{B}{s+p}
=
\frac{A(s+p) + Bs}{s(s+p)}
\quad\Rightarrow\quad
(A+B)s + Ap = q
\quad\Rightarrow\quad
A=-B\;\wedge\; A=\frac{q}{p}.
\end{align*}
Substituting $A$ and $B$ into the partial fraction decomposition gives
\begin{align*}
(\mathscr{L}x)(s)
&=
\frac{q}{p}\frac{1}{s}-\frac{q}{p}\frac{1}{s+p}.
\end{align*}
Both terms on the right can be dealt with by the examples
above.
The reverse transform then becomes
\[
x(t)=\frac{q}{p}-\frac{q}{p}e^{-pt}=\frac{q}{p}(1-e^{-pt}).
\]

\subsection{Solving linear partial differential equations}
The Laplace transform method also works for linear partial differential
equations.
As an example, let's solve the first order equation
\[
\frac{\partial u}{\partial t}+x\frac{\partial u}{\partial x}=x
\]
where $u$ is defined for $t\ge 0$, $x\ge 0$ and we require the
boundary conditions
\begin{align*}
u(x,0)&=0\qquad x>0\\
u(0,t)&=0\qquad t>0.
\end{align*}
The Laplace transform of the differential equation with respect to
the variable $t$ is
\begin{equation}
s(\mathscr{L}u)(s,x)-u(0,x)+x\frac{\partial}{\partial x}(\mathscr{L}u)(s,x)
=
\frac{x}{s}.
\label{laplace:transformedpde}
\end{equation}
The transform has made the time derivative disappear, it has
turned into multiplication by $s$.
For each $s$, the \eqref{laplace:transformedpde} is an ordinary
differential equationfor the function $x\mapsto (\mathscr{L}u)(s,x)$
with initial condition
$(\mathscr{L}u)(s,0)=0$.
The solution can be found with any method and is
\[
(\mathscr{L}u)(s,x)=\frac{x}{s(s+1)}=\frac{x}{s}-\frac{x}{s+1}.
\]
Both terms can be transformed back, the solution becomes
\[
u(t,x)=x-xe^{-t}=x(1-e^{-t}).
\]
To check this result, we compute the derivatives
\begin{align*}
\frac{\partial u}{\partial t}
&=
xe^{-t}
\\
\frac{\partial u}{\partial x}
&=
1-e^{-t}.
\end{align*}
Substituting in the differential equation gives
\[
xe^{-t}+x(1-e^{-t})=x,
\]
the differential equation holds, and thanks to
\begin{align*}
u(x,0)
&=
0
\\
u(0,t)
&=
0
\end{align*}
the boundary condition holds, too.

