%
% heat.tex -- 
%
% (c) 2019 Prof Dr Andreas Mueller, Hochschule Rapperswil
%
\section{Heat equation on a circle}
\rhead{Heat equation}
We now consider the heat equation on a piece of wire formed into
a circle with given initial temperatures.
In polar coordinates, the temperature distribution can be considered
a function on the interval $[-\pi,\pi]$, and can also be considered
a $2\pi$-periodic function on $\mathbb R$.
Thus we are to solve the heat equation
\begin{align*}
\partial_t u(t,x)&=\partial_x^2 u(t,x) &&\forall(t,x)\in\mathbb R\times[-\pi,\pi]\\
\intertext{with initial conditions}
u(0,x)&=f(x)&& \forall x\in[-\pi,\pi]\\
\end{align*}

\subsection{Fourier transform on an interval}
Just like in the introductory example, we transform to Fourier coefficients.
For simplicity, we are using complex Fourier coefficients defined by
\[
c_k=\hat f(k)=\frac1{2\pi}\int_{-\pi}^{\pi}e^{-ikx}f(x)\,dx.
\]
Applying the transform to the original differential equation gives
\begin{align}
\frac1{2\pi}\int_{-\pi}^{\pi}e^{-ikx}\partial_t u(t,x)\,dx
&=\frac1{2\pi}\int_{-\pi}^{\pi}e^{-ikx}\partial_x^2 u(t,x)\,dx
\notag
\\
\partial_t\frac1{2\pi}\int_{-\pi}^{\pi}e^{-ikx} u(t,x)\,dx
&=
\frac1{2\pi}
\underbrace{
\left[
e^{-ikx}\partial_x u(t,x)
\right]_{-\pi}^{\pi}}_{=0}
+
ik\frac1{2\pi}\int_{-\pi}^{\pi}e^{-ikx}\partial_x u(t,x)\,dx
\notag
\\
&=
0 + 
ik\frac1{2\pi}
\underbrace{
\left[
e^{-ikx}u(t,x)
\right]_{-\pi}^{\pi}}_{=0}
-
k^2\frac1{2\pi}\int_{-\pi}^{\pi}e^{-ikx} u(t,x)\,dx
\notag
\\
&=
-
k^2\frac1{2\pi}\int_{-\pi}^{\pi}e^{-ikx} u(t,x)\,dx
\notag
\\
\partial_t\hat u(t,k)&=-k^2 \hat u(t,k)
\label{fouriertransformiert}
\end{align}
The partial differential equation transforms into a family of ordinary
differential equation parametrized by $t\ge 0$.

\subsection{Laplace transform}
We now apply the Laplace transform to the equation
\eqref{fouriertransformiert}
to get
\begin{align*}
\int_0^{\infty} \partial_t \hat u(t,k)e^{-st}\,dt
&=
-k^2\int_0^{\infty}\hat u(t,k)e^{-st}\,dt
\\
\left[\hat u(t,k)e^{-st}\right]_0^{\infty}
+
s\int_0^{\infty}\hat u(t,k)e^{-st}\,dt
&=
-k^2{\cal L}\hat u(s,k)
\\
\hat u(0,k)-s{\cal L}\hat u(s,k)&=k^2{\cal L}\hat u(s,k)
\\
{\cal L}\hat u(s,k)&=\frac{\hat u(0,k)}{s+k^2}.
\end{align*}

\subsection{Reverse transform}
To find the general solution of the differential equation, 
we now have to find the reverse transform.
In the first step we reverse the Laplace transform to
formally get
\begin{align*}
\hat u(t,k)&={\cal L}^{-1}\left(\frac1{s+k^2}\right)\hat u(0,k)
\\
&=e^{-k^2t}\hat u(0,k).
\end{align*}
The second step is reversing the Fourier transform by summing the
Fourier series
\begin{align*}
u(t,x)&=\sum_{k\in\mathbb Z}e^{-k^2t}\hat u(0,k)e^{ikx}
\\
&=\sum_{k\in\mathbb Z}\hat u(0,k)e^{ikx-k^2t}.
\end{align*}

