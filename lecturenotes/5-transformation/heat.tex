%
% heat.tex -- XXX
%
% (c) 2019 Prof Dr Andreas Mueller, Hochschule Rapperswil
%
\section{Wärmeleitung auf einem Kreis}
\rhead{Wärmeleitung}
Wir betrachten jetzt die Wärmeleitungsgleichung auf einem Drahtring
mit gegebenen Anfangswerten. Die Temperaturverteilung auf dem
Kreis beschreiben wir als eine periodische Funktion auf dem Interval
$[-\pi,\pi]$. Gesucht ist daher eine Lösung der Wärmelteitungsgleichung
mit den Anfangsbedingungen
\begin{align*}
\partial_t u(t,x)&=\partial_x^2 u(t,x) &&\forall(t,x)\in\mathbb R\times[-\pi,\pi]\\
u(0,x)&=f(x)&& \forall x\in[-\pi,\pi]\\
\end{align*}

\subsection{Fouriertransformation auf dem Interval}
Wie im einführenden Beispiel gehen wir jetzt zu Fourierkoeffizienten
über.
Der Einfachheit halber verwenden wir komplexe Fourierkoeffizienten
\[
c_k=\hat f(k)=\frac1{2\pi}\int_{-\pi}^{\pi}e^{-ikx}f(x)\,dx
\]
Wenden wir die Transformation auf die ursprüngliche Differentialgleichung
an, erhalten wir
\begin{align}
\frac1{2\pi}\int_{-\pi}^{\pi}e^{-ikx}\partial_t u(t,x)\,dx
&=\frac1{2\pi}\int_{-\pi}^{\pi}e^{-ikx}\partial_x^2 u(t,x)\,dx
\notag
\\
\partial_t\frac1{2\pi}\int_{-\pi}^{\pi}e^{-ikx} u(t,x)\,dx
&=
\frac1{2\pi}
\underbrace{
\left[
e^{-ikx}\partial_x u(t,x)
\right]_{-\pi}^{\pi}}_{=0}
+
ik\frac1{2\pi}\int_{-\pi}^{\pi}e^{-ikx}\partial_x u(t,x)\,dx
\notag
\\
&=
0 + 
ik\frac1{2\pi}
\underbrace{
\left[
e^{-ikx}u(t,x)
\right]_{-\pi}^{\pi}}_{=0}
-
k^2\frac1{2\pi}\int_{-\pi}^{\pi}e^{-ikx} u(t,x)\,dx
\notag
\\
&=
-
k^2\frac1{2\pi}\int_{-\pi}^{\pi}e^{-ikx} u(t,x)\,dx
\notag
\\
\partial_t\hat u(t,k)&=-k^2 \hat u(t,k)
\label{fouriertransformiert}
\end{align}
Aus der partiellen Differentialgleichung ist ein System von gewöhnlichen
Differentialgleichungen für $t\ge 0$ geworden.

\subsection{Laplacetransformation}
Durch Laplace-Transformation der Gleichung (\ref{fouriertransformiert})
erhalten wir die Gleichung
\begin{align*}
\int_0^{\infty} \partial_t \hat u(t,k)e^{-st}\,dt
&=
-k^2\int_0^{\infty}\hat u(t,k)e^{-st}\,dt
\\
\left[\hat u(t,k)e^{-st}\right]_0^{\infty}
+
s\int_0^{\infty}\hat u(t,k)e^{-st}\,dt
&=
-k^2{\cal L}\hat u(s,k)
\\
\hat u(0,k)-s{\cal L}\hat u(s,k)&=k^2{\cal L}\hat u(s,k).
\\
{\cal L}\hat u(s,k)&=\frac{\hat u(0,k)}{s+k^2}
\end{align*}

\subsection{Rücktransformation}
Um die allgemeine Lösung der Differentialgleichung zu finden,
muss man jetzt rücktransformieren, zunächst nach Laplace:
\begin{align*}
\hat u(t,k)&={\cal L}^{-1}\left(\frac1{s+k^2}\right)\hat u(0,k)
\\
&=e^{-k^2t}\hat u(0,k)
\end{align*}
und dann auch noch nach Fourier, also durch Bilden der Fourierreihe
\begin{align*}
u(t,x)&=\sum_{k\in\mathbb Z}e^{-k^2t}\hat u(0,k)e^{ikx}
\\
&=\sum_{k\in\mathbb Z}\hat u(0,k)e^{ikx-k^2t}
\end{align*}

