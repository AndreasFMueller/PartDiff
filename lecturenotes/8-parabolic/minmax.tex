%
% minmax.tex -- XXX
%
% (c) 2019 Prof Dr Andreas Mueller, Hochschule Rapperswil
%
\section{Maximum-Minimum-Prinzip}
\index{Maximumprinzip}
\rhead{Maximum-Prinzip}
In der Theorie der elliptischen PDGL war die Mittelwerteigenschaft
der Ausgangspunkt für den Beweis des Maximumprinzips. Obwohl
diese Eigenschaft bei den 
Lösungen der Wärmeleitungsgleichung nicht mehr erfüllt ist,
gilt für sie ein Maximum-Minimum-Prinzip.
\begin{satz}[Maximum-Minimum-Prinzip für die Wärmeleitungsgleichung]
Sei $\Omega$ ein beschränktes Gebiet und
$u$ eine Funktion auf $[0,T]\times\Omega$.
Weiter sei
\begin{align*}
M_{\partial \Omega}&:=\max\{u(t,x)\,|\,x\in\partial\Omega, t\in[0,T]\}
&
m_{\partial \Omega}&:=\min\{u(t,x)\,|\,x\in\partial\Omega, t\in[0,T]\}
\\
M_0&:=
\max\{u(0,x)\,|\,x\in\Omega\}
&
m_0&:=
\min\{u(0,x)\,|\,x\in\Omega\}
\\
M&:=\max\{M_0,M_{\partial\Omega}\}
&
m&:=\min\{M_0,M_{\partial\Omega}\}
\end{align*}
Falls $u$ die homogene
Wärmeleitungsgleichung erfüllt, also $\partial_tu-\kappa\Delta u=0$,
gilt
\[
m\le u(t,x)\le M\quad\forall(t,x)\in[0,T]\times\bar\Omega.
\]
\end{satz}
Wie bei den elliptischen partiellen Differentialgleichungen folgt, dass
die das homogene Anfangs- und Randwert-Problem der Wärmeleitungsgleichung
nur eine Lösung hat.

