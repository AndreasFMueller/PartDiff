%
% minmax.tex -- 
%
% (c) 2019 Prof Dr Andreas Mueller, Hochschule Rapperswil
%
\section{The maximum principle}
\rhead{maximum principle}
In the theory of elliptic partial differential equations, the mean
value property was one of the starting points for proving the
maximum principle.
The mean value property no longer holds for the heat equation,
but there is still a maximum principle.

\begin{satz}[Maximum principle for the heat equation]
Let $\Omega$ be a bounded domain and 
$u$ a function on $[0,T]\times\bar{\Omega}$.
Further let
\begin{align*}
M_{\partial \Omega}&:=\max\{u(t,x)\,|\,x\in\partial\Omega, t\in[0,T]\}
&
m_{\partial \Omega}&:=\min\{u(t,x)\,|\,x\in\partial\Omega, t\in[0,T]\}
\\
M_0&:=
\max\{u(0,x)\,|\,x\in\Omega\}
&
m_0&:=
\min\{u(0,x)\,|\,x\in\Omega\}
\\
M&:=\max\{M_0,M_{\partial\Omega}\}
&
m&:=\min\{M_0,M_{\partial\Omega}\}
\end{align*}
If $u$ solves the homogeneous heat equation 
$\partial_tu-\kappa\Delta u=0$,
then
\[
m\le u(t,x)\le M\quad\forall(t,x)\in[0,T]\times\bar\Omega
\]
holds.
\end{satz}
As for elliptic partial differential equations it follows that
the boundary value problem for the heat equation has a unique solution.

