%
% eigenfunctions.tex -- XXX
%
% (c) 2019 Prof Dr Andreas Mueller, Hochschule Rapperswil
%
\section{Eigenfunktionen und Lösungen der Wärmeleitungsgleichung}
\index{Eigenfunktion}
\rhead{Eigenfunktionen}
Im vorangegangenen Kapitel haben wir das Problem
\begin{align*}
\Delta u&=f&&\text{in $\Omega$}\\
u&=g&&\text{auf $\partial\Omega$}
\end{align*}
studiert und dabei festgestellt, dass es ähnlich wie bei Matrixgleichungen
$Ax=b$ eine Art Inverse gibt, welche durch die Greensche Funktion
bereitgestellt wird. Sogar formal war die Lösung analog einer Matrixgleichung,
die Rolle der Matrix mit zwei Indizes übernahm eine Funktion mit zwei
Variablen, statt einer Summe war ein Integral zu bilden.

Für eine parabolische PDGL tritt die Zeitentwicklung hinzu. In
ein Matrixproblem übersetzt geht es darum, einen zeitlich
veränderlichen Vektor $x(t)$ zu finden, dessen Ableitung mit einer
Matrixgleichung $\dot x=Ax$ berechnet werden kann. Wir versuchen daher,
die Lösung dieser Art von Matrix-Differentialgleichung auf das
Wärmeleitungsproblem anzuwenden.

\subsection{Systeme gewöhnlicher DGL erster Ordnung}
Das einer homogenen parabolischen linearen PDGL äquivalente Problem
ist ein homogenes System
von linearen gewöhnlichen Differentialgleichungen für eine vektorwertige Funktion
$t\mapsto x(t)\in\mathbb R^n$.
Ein solches System ist gegeben durch eine Matrix $A$ und die Gleichung
\[
\frac{d}{dt}x(t)=Ax(t).
\]
\index{Diagonalform}
Hätte die Matrix $A$ Diagonalform, könnte die allgemeine Lösung 
sofort angegeben werden:
\[
A=\begin{pmatrix}
\lambda_1&\dots&0\\
\vdots&\ddots&\vdots\\
0&\dots&\lambda_n
\end{pmatrix}
\qquad\Rightarrow\qquad
x(t)=\begin{pmatrix}
x_1(0)e^{\lambda_1t}
\\
\vdots
\\
x_n(0)e^{\lambda_nt}
\end{pmatrix}.
\]
Unter gewissen Voraussetzungen an die Matrix $A$ kann man $n$
orthonormierte Eigenvektoren
\index{Eigenvektoren!orthonormiert}
$e_1,\dots,e_n$ finden. Nimmt man einen solchen Vektor als
Anfangsbedingung, ist die Lösung $e_ie^{\lambda_it}$. Schreibt man die
Anfangsbedingung in den Eigenvektoren, z.~B.
\[
x(0)=\sum_{i=1}^n(e_i \cdot x(0))e_i,
\]
kann man die Lösung der Differentialgleichung sofort hinschreiben:
\begin{equation}
x(t)=\sum_{i=1}^n
e^{\lambda_i t}
(e_i\cdot x(0))e_i
.
\label{development}
\end{equation}
Tatsächlich ergibt Einsetzen in die beiden Seiten der Differentialgleichung
\begin{align*}
\frac{d}{dt}x(t)&=\sum_{i=1}^n\lambda_ie^{\lambda_i t}(e_i\cdot x(0))e_i
\\
Ax(t)
&=\sum_{i=1}e^{\lambda_it}(e_i\cdot x(0))Ae_i
=\sum_{i=1}e^{\lambda_it}(e_i\cdot x(0))\lambda_i e_i
\end{align*}
Damit haben wir die Lösungen des homogenen Differentialgleichungssystems
gefunden.

\subsection{Variation der Konstanten}
\index{Variation der Konstanten}
Das Verfahren der Variation der Konstanten erlaubt, eine inhomogene
Differentialgleichung erster Ordnung zu lösen, also
\[
\frac{d}{dt}x(t)-Ax(t)=f(t),
\]
wobei $f$ eine vektorwertige Funktion ist.
Die allgemeine Lösung der homogenen Gleichung war
\[
\sum_{i=0}^nc_ie^{\lambda_i t}e_i,
\]
wobei $c_i=(e_i\cdot x(0))$ war.
Das Verfahren der Variation der Konstanten schreibt vor, die Konstanten
$c_i$ durch Funktionen $c_i(t)$ zu ersetzen,
\[
x(t)=\sum_{i=1}^nc_i(t)e^{\lambda_it}e_i,
\]
und in die Differentialgleichung
einzusetzen, also
\begin{align*}
\sum_{i=1}^n\dot c_i(t)e^{\lambda_it}e_i
+
\sum_{i=1}^n\lambda_i c_i(t)e^{\lambda_it}e_i
-\sum_{i=1}^n\lambda_i c_i(t)e^{\lambda_it}e_i
&=
f(t)
\\
\sum_{i=1}^n\dot c_i(t)e^{\lambda_it}e_i
&=
f(t)
\end{align*}
Bildet man das Skalarprodukt mit $e_i$, ergibt sich 
\[
\dot c_i(t)e^{\lambda_i t}=(e_i\cdot f(t)).
\]
Wir kürzen die rechte Seite mit $f_i(t)=e_i\cdot f(t)$ ab.
Dann ist
\[
c_i(t)=c_i(0)+\int_0^te^{-\lambda_i \tau}f_i(\tau)\,d\tau
\]
und für die Lösung
\begin{align*}
x(t)&=
\sum_{i=1}^n
(e_i\cdot x(0))e_i+
\sum_{i=1}^ne^{\lambda_i t}\int_0^te^{-\lambda_i \tau}(e_i\cdot f(\tau))e_i\,d\tau
\\
&=
x(0)
+
\int_0^t
\biggl(
\sum_{i=1}^n
e^{\lambda_i(t- \tau)}(e_i\cdot f(\tau))\biggr)e_i\,d\tau
\end{align*}
Man kann dies durch Einsetzen nachprüfen:
\begin{align*}
\frac{d}{dt}x(t)
&=
\sum_{i=1}^n\left.e^{-\lambda_i (t-\tau)}(e_i\cdot f(\tau))e_i \right|_{\tau=t}
\\
&=
\sum_{i=1}^n(e_i\cdot f(t))e_i=f(t)
\end{align*}

\subsection{Eigenwerte und Eigenvektoren}
\index{Eigenwert}
\index{Eigenvektor}
Ist das Gebiet $\Omega$ beschränkt mit nicht zu ``kompliziertem'' Rand,
kann man für zeigen, dass eine Folge $u_i(x)$
von Eigenfunktionen zu jedem Eigenwert $\lambda_i$ existiert,
\index{Eigenfunktion}
die das Dirichlet-Problem mit homogenen
Randbedingungen löst, also
\[
\Delta u_i=\lambda_iu_i,\qquad u_{i|\partial\Omega} = 0.
\]
Ausserdem kann man diese Lösungen so skalieren, dass sie ``Betrag'' 1 haben:
\[
\int_{\Omega}|u_i(\xi)|^2\,d\xi=1
\]
und ausserdem orthogonal aufeinander stehen:
\[
\int_{\Omega}u_i(\xi)u_j(\xi)\,d\xi=0\qquad\forall i\ne j.
\]
\index{orthogonale Funktionen}

Verwendet man $u_i$ als ein Faktor in einem Separationsansatz für
die parabolische Gleichung
\[
\partial_tu=\kappa\Delta u,
\]
erhält man für die Gleichung
\begin{align*}
\partial_t (T(t)u_i(t))-\kappa\Delta(T(t)u_i(t))&=0
\\
T'(t)u_i(t)-\kappa T(t)\Delta u_i(t)&=0
\\
T'(t)u_i(t)-\kappa T(t)\lambda_i u_i(t)&=0
\\
\frac{T'(t)}{T(t)}&=\kappa\lambda_i
\\
\Rightarrow\qquad T(t)=Ce^{\kappa\lambda_it}
\end{align*}
Für solche Anfangsbedingungen kann also die Lösung sofort angegeben
werden.
\subsection{Die inhomogene Gleichung}
Das Verfahren der Variation der Konstanten kann auch für partielle
Differentialgleichungen durchgeführt werden. Wir setzen die Lösung
an als
\[
u(t,x)=\sum_{i=0}^\infty c_i(t) e^{\kappa\lambda_i t}u_i(x)
\]
Einsetzen in die Differentialgleichung ergibt
\begin{align*}
\sum_{i=0}^\infty \dot c_i(t)e^{\kappa\lambda_it}u_i(x)
+\kappa\sum_{i=0}^\infty c_i(t)\kappa\lambda_i e^{\lambda_it}u_i(x)
-\kappa\sum_{i=0}^\infty c_i(t)e^{\kappa\lambda_it}\lambda_iu_i(x)
&=f(x)
\\
\sum_{i=0}^\infty \dot c_i(t)e^{\kappa\lambda_it}u_i(x)
&=f(t,x)
\end{align*}
Skalarprodukt mit $u_i$ ergibt
\begin{align*}
\dot c_i(t)&= e^{-\kappa\lambda_it}\int_{\Omega}u_i(\xi)f(t,\xi)\,d\xi
\\
c_i(t)&=c_i(0)+\int_0^te^{-\kappa\lambda_i\tau}\int_{\Omega}u_i(\xi)f(\tau,\xi)\,d\xi\,d\tau
\end{align*}
und für die partikuläre Lösung
\begin{align*}
u(t,x)&=
\sum_{i=0}^\infty
u_i(x)
\int_0^t
e^{\kappa\lambda_i(t-\tau)}\int_{\Omega}u_i(\xi)f(\tau,\xi)\,d\xi\,d\tau
\end{align*}
Die Greensche Funktion für das Wärmeleitungsproblem mit
Dirichlet-Randbedinungen kann also auch als
\[
G(t,x,\tau,\xi)
=
\sum_{i=0}^\infty
e^{\kappa\lambda_i (t-\tau)}
u_i(x)
u_i(\xi)
\]
geschrieben werden. Die Lösung des parabolischen Problems ist damit
auf die Lösung des elliptischen Eigenwertproblems reduziert worden.


