%
% eigenfunctions.tex -- 
%
% (c) 2019 Prof Dr Andreas Mueller, Hochschule Rapperswil
%
\section{Eigenfunctions and solutions of the heat equation}
\rhead{Eigenfunctions}
In the previous chapter we have studied the problem
\begin{align*}
\Delta u&=f&&\text{in $\Omega$}\\
u&=g&&\text{on $\partial\Omega$}
\end{align*}
and found that similarly to matrix equations $Ax=b$ Green's function is a kind
of inverse.
Formally, the solution is similar to a matrix equation in the sense
that we just had to replace the sums used in matrix operations by
integrals.

For a parabolic differential equation we have to add time development.
In the framework of a matrix problem, we would have to solve the
equation $\dot x = Ax$, which can be solved by the matrix
exponential function.
The latter is most elegantly computed using eigenvectors, so the
natural questions arises whether we can solve the heat equation
using the eigenvectors of the associated elliptic problem.

This section intends to substantiate this plan.

\subsection{Systems of ordinary differential equation}
The problem analogous to a parabolic partial differential equation
is a ordinary homogeneous linear differential equation for a vector valued
function $t\mapsto x(t)\in\mathbb R^n$.
Such a system can be stated as
\[
\frac{d}{dt}x(t)=Ax(t)
\]
with some constant Matrix $A$.

If tthe matrix $A$ was diagonal, the general solution could be given
immediately as
\[
A=\begin{pmatrix}
\lambda_1&\dots&0\\
\vdots&\ddots&\vdots\\
0&\dots&\lambda_n
\end{pmatrix}
\qquad\Rightarrow\qquad
x(t)=\begin{pmatrix}
x_1(0)e^{\lambda_1t}
\\
\vdots
\\
x_n(0)e^{\lambda_nt}
\end{pmatrix}.
\]
Under certain conditions on the matrix $A$ we can find a set of
orthonormal eigenvectors $e_1,\dots,e_n$.
By using such a vector as initial condition, we get the solution
$e_ie^{\lambda_it}$.
Writing the initial condition as a linear combination
\[
x(0)=\sum_{i=1}^n(e_i \cdot x(0))e_i
\]
of these eigenvectors,
We can immediately write the solution of the differential equation
as a linear combination
\begin{equation}
x(t)=\sum_{i=1}^n
e^{\lambda_i t}
(e_i\cdot x(0))e_i
.
\label{development}
\end{equation}
of the basic solutions.
In fact, substituting this into the differential equation gives
\begin{align*}
\frac{d}{dt}x(t)&=\sum_{i=1}^n\lambda_ie^{\lambda_i t}(e_i\cdot x(0))e_i
\\
Ax(t)
&=\sum_{i=1}e^{\lambda_it}(e_i\cdot x(0))Ae_i
=\sum_{i=1}e^{\lambda_it}(e_i\cdot x(0))\lambda_i e_i
\end{align*}
So for a homogeneous differential equation, eigenvectors immediately
immediately give us the solution.

\subsection{Variation of constants}
The method of variation of constant allows to solve a inhomogeneous
differential equation of first order
\[
\frac{d}{dt}x(t)-Ax(t)=f(t),
\]
for a vector valued function $f$.
For this the constants in the general solution for the homogeneous 
equation are replaced by functions that depend on $t$:
\[
x(t)=\sum_{i=1}^nc_i(t)e^{\lambda_it}e_i.
\]
Substituting this into the differential equation gives
\begin{align*}
\sum_{i=1}^n\dot c_i(t)e^{\lambda_it}e_i
+
\sum_{i=1}^n\lambda_i c_i(t)e^{\lambda_it}e_i
-\sum_{i=1}^n\lambda_i c_i(t)e^{\lambda_it}e_i
&=
f(t)
\\
\sum_{i=1}^n\dot c_i(t)e^{\lambda_it}e_i
&=
f(t)
\end{align*}
By taking the scalar product with the eigenvectors
$e_i$, we get.
\[
\dot c_i(t)e^{\lambda_i t}=(e_i\cdot f(t)).
\]
Abbreviating the right hand side using
$f_i(t)=e_i\cdot f(t)$, turns this into
\[
c_i(t)=c_i(0)+\int_0^te^{-\lambda_i \tau}f_i(\tau)\,d\tau
\]
and the solution becomes
\begin{align*}
x(t)&=
\sum_{i=1}^n
(e_i\cdot x(0))e_i+
\sum_{i=1}^ne^{\lambda_i t}\int_0^te^{-\lambda_i \tau}(e_i\cdot f(\tau))e_i\,d\tau
\\
&=
x(0)
+
\int_0^t
\biggl(
\sum_{i=1}^n
e^{\lambda_i(t- \tau)}(e_i\cdot f(\tau))\biggr)e_i\,d\tau
\end{align*}
One can verify this directly:
\begin{align*}
\frac{d}{dt}x(t)
&=
\sum_{i=1}^n\left.e^{-\lambda_i (t-\tau)}(e_i\cdot f(\tau))e_i \right|_{\tau=t}
\\
&=
\sum_{i=1}^n(e_i\cdot f(t))e_i=f(t)
\end{align*}

\subsection{Eigenvalues and eigenvectors}
Assume now that the domain $\Omega$ is bounded and does not have a 
boundary too complicated, so that there is a sequence of eigenfunctions
$u_i(x)$ of the Laplace operator with eigenvalue $\lambda_i$
with homogeneous boundary conditions, i.~e.
\[
\Delta u_i=\lambda_iu_i,\qquad u_{i|\partial\Omega} = 0.
\]
In addition, we can scale these solutions so that the have 
$L^2$-norm $1$:
\[
\int_{\Omega}|u_i(\xi)|^2\,d\xi=1
\]
The general theory even guarantees that the are orthogonal:
\[
\int_{\Omega}u_i(\xi)u_j(\xi)\,d\xi=0\qquad\forall i\ne j.
\]

We now use these functions in a separation ansatz
$u(x,t)=u_i(x)\cdot T(t)$ for the parabolic equation
\[
\partial_tu=\kappa\Delta u,
\]
and get the equation
\begin{align*}
\partial_t (T(t)u_i(t))-\kappa\Delta(T(t)u_i(t))&=0
\\
T'(t)u_i(t)-\kappa T(t)\Delta u_i(t)&=0
\\
T'(t)u_i(t)-\kappa T(t)\lambda_i u_i(t)&=0
\\
\frac{T'(t)}{T(t)}&=\kappa\lambda_i
\\
\Rightarrow\qquad T(t)=Ce^{\kappa\lambda_it}
\end{align*}
In particular, for initial conditions that are eigenvalues we
immediately get a solution for the parabolic problem.

\subsection{The inhomogeous equation}
The process of variation of constants cann also be used for partial
differential equations.
We write the solution as
\[
u(t,x)=\sum_{i=0}^\infty c_i(t) e^{\kappa\lambda_i t}u_i(x)
\]
and substitute into the differential equation
\begin{align*}
\sum_{i=0}^\infty \dot c_i(t)e^{\kappa\lambda_it}u_i(x)
+\kappa\sum_{i=0}^\infty c_i(t)\kappa\lambda_i e^{\lambda_it}u_i(x)
-\kappa\sum_{i=0}^\infty c_i(t)e^{\kappa\lambda_it}\lambda_iu_i(x)
&=f(x)
\\
\sum_{i=0}^\infty \dot c_i(t)e^{\kappa\lambda_it}u_i(x)
&=f(t,x)
\end{align*}
The scalar product with $u_i$ then gives
\begin{align*}
\dot c_i(t)&= e^{-\kappa\lambda_it}\int_{\Omega}u_i(\xi)f(t,\xi)\,d\xi
\\
c_i(t)&=c_i(0)+\int_0^te^{-\kappa\lambda_i\tau}\int_{\Omega}u_i(\xi)f(\tau,\xi)\,d\xi\,d\tau
\end{align*}
and the particular solution
\begin{align*}
u(t,x)&=
\sum_{i=0}^\infty
u_i(x)
\int_0^t
e^{\kappa\lambda_i(t-\tau)}\int_{\Omega}u_i(\xi)f(\tau,\xi)\,d\xi\,d\tau
\end{align*}
Green's function for the heat equation with Dirichlet boundary conditions
can thus be written as
\[
G(t,x,\tau,\xi)
=
\sum_{i=0}^\infty
e^{\kappa\lambda_i (t-\tau)}
u_i(x)
u_i(\xi).
\]
So we have reduced the solution of the parabolic problem to the
elliptic problem.

