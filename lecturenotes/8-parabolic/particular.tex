%
% particular.tex -- 
%
% (c) 2019 Prof Dr Andreas Mueller, Hochschule Rapperswil
%
\section{Particular solution}
\rhead{Particular solution}
To solve the problem posed at the beginning of this chapter, we can
proceed in a similar way as for the elliptic problem.
First we need special solutions for the case that solve the heat
equation for $\delta$-functions as right hand sides, but without
necessarily respecting boundary values.

By some tedious computation one can show that for $0<\tau<t$
the expression
\[
\frac1{(4\pi\kappa(t-\tau))^{\frac{n}2}}
\exp\biggl(-\frac{|x-\xi|^2}{4\kappa(t-\tau)}\biggr)
\]
solves the heat equation.
The function
\begin{equation}
K(t-\tau, x-\xi)
=
\vartheta(t-\tau)
\frac1{(4\pi\kappa(t-\tau))^{\frac{n}2}}
\exp\biggl(-\frac{|x-\xi|^2}{4\kappa(t-\tau)}\biggr)
\label{parabolischsingulaer}
\end{equation}
thus has the desired properties:

\begin{satz}
The funcion $K$ satisifes the heat equation
\begin{align*}
\partial_tK-\kappa\Delta_xK&=0&&(\tau, \xi)\ne(t,x)
\\
\lim_{t\to\tau^+}K(t-\tau, x-\xi)&=\delta(x-\xi),
\end{align*}
where the limit is taken in the weak sense.

With repsect to the coordinates $(\tau,\xi)$ we have
\begin{align*}
-\partial_{\tau} K-\kappa\Delta_{\xi}K&=0\qquad(\tau,\xi)\ne(t,x)
\\
\lim_{\tau\to t^-}K(t-\tau, x-\xi)&=\delta(x-\xi)
\end{align*}
\end{satz}

A particular solution of th eproblem can be found using the integrals
\begin{align*}
u(t,x)
&=
\int_\Omega\int_0^\infty
K(t-\tau,x-\xi)f(\tau,\xi)
\,d\tau\,d\xi
\\
&=
\int_\Omega\int_0^\infty
\vartheta(t-\tau)\frac1{(4\pi\kappa(t-\tau))^{\frac{n}2}}
\exp\biggl(-\frac{|x-\xi|^2}{4\kappa(t-\tau)}\biggr)
f(\tau,\xi)
\,d\tau\,d\xi
\\
&=
\int_\Omega\int_0^t
\frac1{(4\pi\kappa(t-\tau))^{\frac{n}2}}
\exp\biggl(-\frac{|x-\xi|^2}{4\kappa(t-\tau)}\biggr)
f(\tau,\xi)
\,d\tau\,d\xi
\end{align*}
Substituting this into the heat equation gives
\begin{align*}
(\partial_t-\kappa\Delta)u
&=
\int_\Omega\int_0^t
(\partial_t-\kappa\Delta)\biggl[
\frac1{(4\pi\kappa(t-\tau))^{\frac{n}2}}
\exp\biggl(-\frac{|x-\xi|^2}{4\kappa(t-\tau)}\biggr)\biggr]
f(\tau,\xi)
\,d\tau\,d\xi
\\
&\quad+
\lim_{\tau\to t-}
\int_\Omega
\frac1{(4\pi\kappa(t-\tau))^{\frac{n}2}}
\exp\biggl(-\frac{|x-\xi|^2}{4\kappa(t-\tau)}\biggr)
f(t,\xi)
\,d\xi
\end{align*}
The first integral vanishes because the expression in brackets
satisfies the heat equation for $0<\tau<t$, which means that it
is annihilated by $\partial_t-\kappa\Delta$.
The part of the integrand in front of $f(t,\xi)$ in the second integral
is an approximation of the $\delta$-function at $x$, so its limit 
is $f(x,t)$.
Combining this tells us
\[
(\partial_t-\kappa\Delta)u=f,
\]
as claimed.

