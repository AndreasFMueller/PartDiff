%
% particular.tex
%
% (c) 2019 Prof Dr Andreas Mueller, Hochschule Rapperswil
%
\section{Partikuläre Lösung}
\index{partikul\äre L\ösung}
\rhead{Partikuläre Lösung}
Um das anfangs des Kapitels gestellt Problem zu lösen, kann man jetzt
gleich vorgehen wie im Fall elliptischer Randwertprobleme. Zunächst
braucht man singuläre Lösungen, also Lösungen, welche die
Wärmeleitungsgleichung mit einer $\delta$-Funktion als rechte Seite
lösen.
Man kann durch etwas mühsame Rechnung nachprüfen, dass für $0<\tau<t$
der Ausdruck
\[
\frac1{(4\pi\kappa(t-\tau))^{\frac{n}2}}
\exp\biggl(-\frac{|x-\xi|^2}{4\kappa(t-\tau)}\biggr)
\]
die Wärmeleitungsgleichung erfüllt. Daraus lässt sich
dann die Funktion
\begin{equation}
K(t-\tau, x-\xi)
=
\vartheta(t-\tau)
\frac1{(4\pi\kappa(t-\tau))^{\frac{n}2}}
\exp\biggl(-\frac{|x-\xi|^2}{4\kappa(t-\tau)}\biggr)
\label{parabolischsingulaer}
\end{equation}
konstruieren, welche die gewünschten Eigenschaften hat:

\begin{satz}
Die Funktion $K$ erfüllt die Wärmeleitungsgleichung
\begin{align*}
\partial_tK-\kappa\Delta_xK&=0&&(\tau, \xi)\ne(t,x)
\\
\lim_{t\to\tau^+}K(t-\tau, x-\xi)&=\delta(x-\xi).
\end{align*}
Bezüglich der Koordinaten $(\tau,\xi)$ gilt
\begin{align*}
-\partial_{\tau} K-\kappa\Delta_{\xi}K&=0\qquad(\tau,\xi)\ne(t,x)
\\
\lim_{\tau\to t^-}K(t-\tau, x-\xi)&=\delta(x-\xi)
\end{align*}
\end{satz}

Eine partikuläre Lösung des Problems kann wieder mit Hilfe eines Integrals
gefunden werden:
\begin{align*}
u(t,x)
&=
\int_\Omega\int_0^\infty
K(t-\tau,x-\xi)f(\tau,\xi)
\,d\tau\,d\xi
\\
&=
\int_\Omega\int_0^\infty
\vartheta(t-\tau)\frac1{(4\pi\kappa(t-\tau))^{\frac{n}2}}
\exp\biggl(-\frac{|x-\xi|^2}{4\kappa(t-\tau)}\biggr)
f(\tau,\xi)
\,d\tau\,d\xi
\\
&=
\int_\Omega\int_0^t
\frac1{(4\pi\kappa(t-\tau))^{\frac{n}2}}
\exp\biggl(-\frac{|x-\xi|^2}{4\kappa(t-\tau)}\biggr)
f(\tau,\xi)
\,d\tau\,d\xi
\end{align*}
Setzt man dies in die Wärmeleitungsgleichung ein, ergibt sich
\begin{align*}
(\partial_t-\kappa\Delta)u
&=
\int_\Omega\int_0^t
(\partial_t-\kappa\Delta)\biggl[
\frac1{(4\pi\kappa(t-\tau))^{\frac{n}2}}
\exp\biggl(-\frac{|x-\xi|^2}{4\kappa(t-\tau)}\biggr)\biggr]
f(\tau,\xi)
\,d\tau\,d\xi
\\
&\quad+
\lim_{\tau\to t-}
\int_\Omega
\frac1{(4\pi\kappa(t-\tau))^{\frac{n}2}}
\exp\biggl(-\frac{|x-\xi|^2}{4\kappa(t-\tau)}\biggr)
f(t,\xi)
\,d\xi
\end{align*}
Das erste Integral verschwindet, weil der Ausdruck in der eckigen Klammer
für alle $0<\tau<t$ die Wärmeleitungsgleichung erfüllt, also von
$\partial_t-\kappa\Delta$ zu $0$ gemacht wird.
Der Teil des Integranden vor $f(t,\xi)$ im zweiten Integral ist eine
Approximation der $\delta$-Funktion an der Stelle $x$, also ist der Grenzwert
$f(t,x)$. Insgesamt ist also
\[
(\partial_t-\kappa\Delta)u=f,
\]
wie behauptet.

