%
% parabolic.tex -- 
%
% (c) 2019 Prof Dr Andreas Mueller, Hochschule Rapperswil
%
\chapter{Parabolic partial differeential equations
\label{chapter-parabolisch}}
\lhead{Parabolic partial differeential equations}
\rhead{}
In the previous chapter we were able to develop a formular for the
solution of an elliptic boundary value problem using Green's 
function.
This method dependedessentially on the maximum principle.
It guarantees uniqueness of solutions and thus allowed teh
construction of Green's function in the first place.

For parabolic differential equations a similar technique is
available.
However, the time coordinate, i.~e.~the coordinate with 
the zero eigenvalue of the symbol matrix, will always have to be
treated specially.
We just summarize the main ideas in the forst three sections
without proofs or deeper analysis.

In contrast to elliptic equations, the heat equation can also be
considered as a ``equation of motion'' for the temperature distribution
just like the ordinary differential equation $\dot x=f(x,t)$ is an
equation of motion for the state vector $x$.
The domain of the heat equation is typically of the form
$\Omega\times[0,\infty[$, where $\Omega$ is a domain in space.
The solution of the heat equation is function in $\Omega$ with 
an additional time dependency, which we can express using a
separation ansatz.
It turns out that this allows the solution of the heat equation
by using the already established solution of the elliptic problem.

%
% problem.tex -- 
%
% (c) 2019 Prof Dr Andreas Mueller, Hochschule Rapperswil
%
\section{The problem}
\rhead{The problem}
Let $\Omega$ be a domain in $\mathbb R^n$.
The operator
\[
\partial_t-\kappa\Delta
\]
is a parabolic operator on $\mathbb R_+\times\Omega$.
A solution of the heat problem is a function
$u\colon\mathbb R_+\times\Omega\to\mathbb R,$
that satisfies the following conditions
\begin{align*}
\partial_tu(t,x)-\kappa\Delta u(t,x)&=f(t,x)&&(t,x)\in\mathbb R_+\times\Omega
\\
u(0,x)&=u_0(x)&&x\in\Omega
\\
\alpha u+\beta\frac{\partial u}{\partial n}&=g(t, x)&&x\in\partial\Omega, t>0.
\end{align*}
If the domain $\Omega$ is unbounded, as in the the example $\Omega=\mathbb R^n$,
it will be necessary to add some conditions for the behavior of the
solution for $t\to\infty$ to ensure that the solution will be 
uniquely determined.


%
% minmax.tex -- 
%
% (c) 2019 Prof Dr Andreas Mueller, Hochschule Rapperswil
%
\section{The maximum principle}
\rhead{maximum principle}
In the theory of elliptic partial differential equations, the mean
value property was one of the starting points for proving the
maximum principle.
The mean value property no longer holds for the heat equation,
but there is still a maximum principle.

\begin{satz}[Maximum principle for the heat equation]
Let $\Omega$ be a bounded domain and 
$u$ a function on $[0,T]\times\Omega$.
Further let
\begin{align*}
M_{\partial \Omega}&:=\max\{u(t,x)\,|\,x\in\partial\Omega, t\in[0,T]\}
&
m_{\partial \Omega}&:=\min\{u(t,x)\,|\,x\in\partial\Omega, t\in[0,T]\}
\\
M_0&:=
\max\{u(0,x)\,|\,x\in\Omega\}
&
m_0&:=
\min\{u(0,x)\,|\,x\in\Omega\}
\\
M&:=\max\{M_0,M_{\partial\Omega}\}
&
m&:=\min\{M_0,M_{\partial\Omega}\}
\end{align*}
If $u$ solves the homogeneous heat equation 
$\partial_tu-\kappa\Delta u=0$,
then
\[
m\le u(t,x)\le M\quad\forall(t,x)\in[0,T]\times\bar\Omega
\]
holds.
\end{satz}
As for elliptic partial differential equations it follows that
the boundary value problem for the heat equation has a unique solution.


%
% particular.tex -- 
%
% (c) 2019 Prof Dr Andreas Mueller, Hochschule Rapperswil
%
\section{Particular solution}
\rhead{Particular solution}
To solve the problem posed at the beginning of this chapter, we can
proceed in a similar way as for the elliptic problem.
First we need special solutions that solve the heat equation for
the special case of $\delta$-functions as right hand sides, but without
necessarily respecting boundary values.

By some tedious computation one can show that for $0<\tau<t$
the expression
\[
\frac1{(4\pi\kappa(t-\tau))^{\frac{n}2}}
\exp\biggl(-\frac{|x-\xi|^2}{4\kappa(t-\tau)}\biggr)
\]
solves the heat equation.
The function
\begin{equation}
K(t-\tau, x-\xi)
=
\vartheta(t-\tau)
\frac1{(4\pi\kappa(t-\tau))^{\frac{n}2}}
\exp\biggl(-\frac{|x-\xi|^2}{4\kappa(t-\tau)}\biggr)
\label{parabolischsingulaer}
\end{equation}
thus has the desired properties:

\begin{satz}
The funcion $K$ satisifes the heat equation
\begin{align*}
\partial_tK-\kappa\Delta_xK&=0&&(\tau, \xi)\ne(t,x)
\\
\lim_{t\to\tau^+}K(t-\tau, x-\xi)&=\delta(x-\xi),
\end{align*}
where the limit is taken in the weak sense.

With repsect to the coordinates $(\tau,\xi)$ we have
\begin{align*}
-\partial_{\tau} K-\kappa\Delta_{\xi}K&=0\qquad(\tau,\xi)\ne(t,x)
\\
\lim_{\tau\to t^-}K(t-\tau, x-\xi)&=\delta(x-\xi)
\end{align*}
\end{satz}

A particular solution of th eproblem can be found using the integrals
\begin{align*}
u(t,x)
&=
\int_\Omega\int_0^\infty
K(t-\tau,x-\xi)f(\tau,\xi)
\,d\tau\,d\xi
\\
&=
\int_\Omega\int_0^\infty
\vartheta(t-\tau)\frac1{(4\pi\kappa(t-\tau))^{\frac{n}2}}
\exp\biggl(-\frac{|x-\xi|^2}{4\kappa(t-\tau)}\biggr)
f(\tau,\xi)
\,d\tau\,d\xi
\\
&=
\int_\Omega\int_0^t
\frac1{(4\pi\kappa(t-\tau))^{\frac{n}2}}
\exp\biggl(-\frac{|x-\xi|^2}{4\kappa(t-\tau)}\biggr)
f(\tau,\xi)
\,d\tau\,d\xi
\end{align*}
Substituting this into the heat equation gives
\begin{align*}
(\partial_t-\kappa\Delta)u
&=
\int_\Omega\int_0^t
(\partial_t-\kappa\Delta)\biggl[
\frac1{(4\pi\kappa(t-\tau))^{\frac{n}2}}
\exp\biggl(-\frac{|x-\xi|^2}{4\kappa(t-\tau)}\biggr)\biggr]
f(\tau,\xi)
\,d\tau\,d\xi
\\
&\quad+
\lim_{\tau\to t-}
\int_\Omega
\frac1{(4\pi\kappa(t-\tau))^{\frac{n}2}}
\exp\biggl(-\frac{|x-\xi|^2}{4\kappa(t-\tau)}\biggr)
f(t,\xi)
\,d\xi
\end{align*}
The first integral vanishes because the expression in brackets
satisfies the heat equation for $0<\tau<t$, which means that it
is annihilated by $\partial_t-\kappa\Delta$.
The part of the integrand in front of $f(t,\xi)$ in the second integral
is an approximation of the $\delta$-function at $x$, so its limit 
is $f(x,t)$.
Combining this tells us
\[
(\partial_t-\kappa\Delta)u=f,
\]
as claimed.


\def\gruen{\draw[color=darkgreen] (0,0)
--(0.0100,-0.0041)
--(0.0200,-0.0083)
--(0.0300,-0.0125)
--(0.0400,-0.0167)
--(0.0500,-0.0209)
--(0.0600,-0.0251)
--(0.0700,-0.0294)
--(0.0800,-0.0336)
--(0.0900,-0.0379)
--(0.1000,-0.0422)
--(0.1100,-0.0465)
--(0.1200,-0.0508)
--(0.1300,-0.0551)
--(0.1400,-0.0595)
--(0.1500,-0.0638)
--(0.1600,-0.0682)
--(0.1700,-0.0726)
--(0.1800,-0.0770)
--(0.1900,-0.0815)
--(0.2000,-0.0859)
--(0.2100,-0.0904)
--(0.2200,-0.0949)
--(0.2300,-0.0994)
--(0.2400,-0.1039)
--(0.2500,-0.1084)
--(0.2600,-0.1129)
--(0.2700,-0.1175)
--(0.2800,-0.1221)
--(0.2900,-0.1267)
--(0.3000,-0.1313)
--(0.3100,-0.1360)
--(0.3200,-0.1406)
--(0.3300,-0.1453)
--(0.3400,-0.1500)
--(0.3500,-0.1547)
--(0.3600,-0.1594)
--(0.3700,-0.1642)
--(0.3800,-0.1690)
--(0.3900,-0.1737)
--(0.4000,-0.1785)
--(0.4100,-0.1834)
--(0.4200,-0.1882)
--(0.4300,-0.1931)
--(0.4400,-0.1980)
--(0.4500,-0.2029)
--(0.4600,-0.2078)
--(0.4700,-0.2128)
--(0.4800,-0.2177)
--(0.4900,-0.2227)
--(0.5000,-0.2277)
--(0.5100,-0.2327)
--(0.5200,-0.2378)
--(0.5300,-0.2429)
--(0.5400,-0.2480)
--(0.5500,-0.2531)
--(0.5600,-0.2582)
--(0.5700,-0.2634)
--(0.5800,-0.2686)
--(0.5900,-0.2738)
--(0.6000,-0.2790)
--(0.6100,-0.2842)
--(0.6200,-0.2895)
--(0.6300,-0.2948)
--(0.6400,-0.3001)
--(0.6500,-0.3055)
--(0.6600,-0.3108)
--(0.6700,-0.3162)
--(0.6800,-0.3216)
--(0.6900,-0.3271)
--(0.7000,-0.3325)
--(0.7100,-0.3380)
--(0.7200,-0.3435)
--(0.7300,-0.3491)
--(0.7400,-0.3546)
--(0.7500,-0.3602)
--(0.7600,-0.3658)
--(0.7700,-0.3715)
--(0.7800,-0.3771)
--(0.7900,-0.3828)
--(0.8000,-0.3885)
--(0.8100,-0.3943)
--(0.8200,-0.4001)
--(0.8300,-0.4059)
--(0.8400,-0.4117)
--(0.8500,-0.4175)
--(0.8600,-0.4234)
--(0.8700,-0.4293)
--(0.8800,-0.4353)
--(0.8900,-0.4412)
--(0.9000,-0.4472)
--(0.9100,-0.4533)
--(0.9200,-0.4593)
--(0.9300,-0.4654)
--(0.9400,-0.4715)
--(0.9500,-0.4777)
--(0.9600,-0.4839)
--(0.9700,-0.4901)
--(0.9800,-0.4963)
--(0.9900,-0.5026)
--(1.0000,-0.5089)
--(1.0100,-0.5152)
--(1.0200,-0.5216)
--(1.0300,-0.5280)
--(1.0400,-0.5344)
--(1.0500,-0.5409)
;}
\def\gruenundef{\fill[color=darkgreen!20] (0,1)
--(0.0100,0.9959)
--(0.0200,0.9917)
--(0.0300,0.9875)
--(0.0400,0.9833)
--(0.0500,0.9791)
--(0.0600,0.9749)
--(0.0700,0.9706)
--(0.0800,0.9664)
--(0.0900,0.9621)
--(0.1000,0.9578)
--(0.1100,0.9535)
--(0.1200,0.9492)
--(0.1300,0.9449)
--(0.1400,0.9405)
--(0.1500,0.9362)
--(0.1600,0.9318)
--(0.1700,0.9274)
--(0.1800,0.9230)
--(0.1900,0.9185)
--(0.2000,0.9141)
--(0.2100,0.9096)
--(0.2200,0.9051)
--(0.2300,0.9006)
--(0.2400,0.8961)
--(0.2500,0.8916)
--(0.2600,0.8871)
--(0.2700,0.8825)
--(0.2800,0.8779)
--(0.2900,0.8733)
--(0.3000,0.8687)
--(0.3100,0.8640)
--(0.3200,0.8594)
--(0.3300,0.8547)
--(0.3400,0.8500)
--(0.3500,0.8453)
--(0.3600,0.8406)
--(0.3700,0.8358)
--(0.3800,0.8310)
--(0.3900,0.8263)
--(0.4000,0.8215)
--(0.4100,0.8166)
--(0.4200,0.8118)
--(0.4300,0.8069)
--(0.4400,0.8020)
--(0.4500,0.7971)
--(0.4600,0.7922)
--(0.4700,0.7872)
--(0.4800,0.7823)
--(0.4900,0.7773)
--(0.5000,0.7723)
--(0.5100,0.7673)
--(0.5200,0.7622)
--(0.5300,0.7571)
--(0.5400,0.7520)
--(0.5500,0.7469)
--(0.5600,0.7418)
--(0.5700,0.7366)
--(0.5800,0.7314)
--(0.5900,0.7262)
--(0.6000,0.7210)
--(0.6100,0.7158)
--(0.6200,0.7105)
--(0.6300,0.7052)
--(0.6400,0.6999)
--(0.6500,0.6945)
--(0.6600,0.6892)
--(0.6700,0.6838)
--(0.6800,0.6784)
--(0.6900,0.6729)
--(0.7000,0.6675)
--(0.7100,0.6620)
--(0.7200,0.6565)
--(0.7300,0.6509)
--(0.7400,0.6454)
--(0.7500,0.6398)
--(0.7600,0.6342)
--(0.7700,0.6285)
--(0.7800,0.6229)
--(0.7900,0.6172)
--(0.8000,0.6115)
--(0.8100,0.6057)
--(0.8200,0.5999)
--(0.8300,0.5941)
--(0.8400,0.5883)
--(0.8500,0.5825)
--(0.8600,0.5766)
--(0.8700,0.5707)
--(0.8800,0.5647)
--(0.8900,0.5588)
--(0.9000,0.5528)
--(0.9100,0.5467)
--(0.9200,0.5407)
--(0.9300,0.5346)
--(0.9400,0.5285)
--(0.9500,0.5223)
--(0.9600,0.5161)
--(0.9700,0.5099)
--(0.9800,0.5037)
--(0.9900,0.4974)
--(1.0000,0.4911)
--(1,1)--cycle;
}

%
% causality.tex -- 
%
% (c) 2019 Prof Dr Andreas Mueller, Hochschule Rapperswil
%
\section{Causality}
The singular solution $K$ demonstrates that a change in $f$ at
$(t_0,x_0)$ affects all values of the solution $u(t,x)$ for $t>t_0$.
A disturbance at time $t_0$ propagates instantaneously throughout
the the $\Omega$.


%
% eigenfunctions.tex -- 
%
% (c) 2019 Prof Dr Andreas Mueller, Hochschule Rapperswil
%
\section{Eigenfunctions and solutions of the heat equation}
\rhead{Eigenfunctions}
In the previous chapter we have studied the problem
\begin{align*}
\Delta u&=f&&\text{in $\Omega$}\\
u&=g&&\text{on $\partial\Omega$}
\end{align*}
and found that similarly to matrix equations $Ax=b$ Green's function is a kind
of inverse.
Formally, the solution is similar to a matrix equation in the sense
that we just had to replace the sums used in matrix operations by
integrals.

For a parabolic differential equation we have to add time development.
In the framework of a matrix problem, we would have to solve the
equation $\dot x = Ax$, which can be solved by the matrix
exponential function.
The latter is most elegantly computed using eigenvectors, so the
natural questions arises whether we can solve the heat equation
using the eigenvectors of the associated elliptic problem.

This section intends to substantiate this plan.

\subsection{Systems of ordinary differential equation}
The problem analogous to a parabolic partial differential equation
is a ordinary homogeneous linear differential equation for a vector valued
function $t\mapsto x(t)\in\mathbb R^n$.
Such a system can be stated as
\[
\frac{d}{dt}x(t)=Ax(t)
\]
with some constant Matrix $A$.

If tthe matrix $A$ was diagonal, the general solution could be given
immediately as
\[
A=\begin{pmatrix}
\lambda_1&\dots&0\\
\vdots&\ddots&\vdots\\
0&\dots&\lambda_n
\end{pmatrix}
\qquad\Rightarrow\qquad
x(t)=\begin{pmatrix}
x_1(0)e^{\lambda_1t}
\\
\vdots
\\
x_n(0)e^{\lambda_nt}
\end{pmatrix}.
\]
Under certain conditions on the matrix $A$ we can find a set of
orthonormal eigenvectors $e_1,\dots,e_n$.
By using such a vector as initial condition, we get the solution
$e_ie^{\lambda_it}$.
Writing the initial condition as a linear combination
\[
x(0)=\sum_{i=1}^n(e_i \cdot x(0))e_i
\]
of these eigenvectors,
We can immediately write the solution of the differential equation
as a linear combination
\begin{equation}
x(t)=\sum_{i=1}^n
e^{\lambda_i t}
(e_i\cdot x(0))e_i
.
\label{development}
\end{equation}
of the basic solutions.
In fact, substituting this into the differential equation gives
\begin{align*}
\frac{d}{dt}x(t)&=\sum_{i=1}^n\lambda_ie^{\lambda_i t}(e_i\cdot x(0))e_i
\\
Ax(t)
&=\sum_{i=1}e^{\lambda_it}(e_i\cdot x(0))Ae_i
=\sum_{i=1}e^{\lambda_it}(e_i\cdot x(0))\lambda_i e_i
\end{align*}
So for a homogeneous differential equation, eigenvectors immediately
immediately give us the solution.

\subsection{Variation of constants}
The method of variation of constant allows to solve a inhomogeneous
differential equation of first order
\[
\frac{d}{dt}x(t)-Ax(t)=f(t),
\]
for a vector valued function $f$.
For this the constants in the general solution for the homogeneous 
equation are replaced by functions that depend on $t$:
\[
x(t)=\sum_{i=1}^nc_i(t)e^{\lambda_it}e_i.
\]
Substituting this into the differential equation gives
\begin{align*}
\sum_{i=1}^n\dot c_i(t)e^{\lambda_it}e_i
+
\sum_{i=1}^n\lambda_i c_i(t)e^{\lambda_it}e_i
-\sum_{i=1}^n\lambda_i c_i(t)e^{\lambda_it}e_i
&=
f(t)
\\
\sum_{i=1}^n\dot c_i(t)e^{\lambda_it}e_i
&=
f(t)
\end{align*}
By taking the scalar product with the eigenvectors
$e_i$, we get.
\[
\dot c_i(t)e^{\lambda_i t}=(e_i\cdot f(t)).
\]
Abbreviating the right hand side using
$f_i(t)=e_i\cdot f(t)$, turns this into
\[
c_i(t)=c_i(0)+\int_0^te^{-\lambda_i \tau}f_i(\tau)\,d\tau
\]
and the solution becomes
\begin{align*}
x(t)&=
\sum_{i=1}^n
(e_i\cdot x(0))e_i+
\sum_{i=1}^ne^{\lambda_i t}\int_0^te^{-\lambda_i \tau}(e_i\cdot f(\tau))e_i\,d\tau
\\
&=
x(0)
+
\int_0^t
\biggl(
\sum_{i=1}^n
e^{\lambda_i(t- \tau)}(e_i\cdot f(\tau))\biggr)e_i\,d\tau
\end{align*}
One can verify this directly:
\begin{align*}
\frac{d}{dt}x(t)
&=
\sum_{i=1}^n\left.e^{-\lambda_i (t-\tau)}(e_i\cdot f(\tau))e_i \right|_{\tau=t}
\\
&=
\sum_{i=1}^n(e_i\cdot f(t))e_i=f(t)
\end{align*}

\subsection{Eigenvalues and eigenvectors}
Assume now that the domain $\Omega$ is bounded and does not have a 
boundary too complicated, so that there is a sequence of eigenfunctions
$u_i(x)$ of the Laplace operator with eigenvalue $\lambda_i$
with homogeneous boundary conditions, i.~e.
\[
\Delta u_i=\lambda_iu_i,\qquad u_{i|\partial\Omega} = 0.
\]
In addition, we can scale these solutions so that the have 
$L^2$-norm $1$:
\[
\int_{\Omega}|u_i(\xi)|^2\,d\xi=1
\]
The general theory even guarantees that the are orthogonal:
\[
\int_{\Omega}u_i(\xi)u_j(\xi)\,d\xi=0\qquad\forall i\ne j.
\]

We now use these functions in a separation ansatz
$u(x,t)=u_i(x)\cdot T(t)$ for the parabolic equation
\[
\partial_tu=\kappa\Delta u,
\]
and get the equation
\begin{align*}
\partial_t (T(t)u_i(t))-\kappa\Delta(T(t)u_i(t))&=0
\\
T'(t)u_i(t)-\kappa T(t)\Delta u_i(t)&=0
\\
T'(t)u_i(t)-\kappa T(t)\lambda_i u_i(t)&=0
\\
\frac{T'(t)}{T(t)}&=\kappa\lambda_i
\\
\Rightarrow\qquad T(t)=Ce^{\kappa\lambda_it}
\end{align*}
In particular, for initial conditions that are eigenvalues we
immediately get a solution for the parabolic problem.

\subsection{The inhomogeous equation}
The process of variation of constants cann also be used for partial
differential equations.
We write the solution as
\[
u(t,x)=\sum_{i=0}^\infty c_i(t) e^{\kappa\lambda_i t}u_i(x)
\]
and substitute into the differential equation
\begin{align*}
\sum_{i=0}^\infty \dot c_i(t)e^{\kappa\lambda_it}u_i(x)
+\kappa\sum_{i=0}^\infty c_i(t)\kappa\lambda_i e^{\lambda_it}u_i(x)
-\kappa\sum_{i=0}^\infty c_i(t)e^{\kappa\lambda_it}\lambda_iu_i(x)
&=f(x)
\\
\sum_{i=0}^\infty \dot c_i(t)e^{\kappa\lambda_it}u_i(x)
&=f(t,x)
\end{align*}
The scalar product with $u_i$ then gives
\begin{align*}
\dot c_i(t)&= e^{-\kappa\lambda_it}\int_{\Omega}u_i(\xi)f(t,\xi)\,d\xi
\\
c_i(t)&=c_i(0)+\int_0^te^{-\kappa\lambda_i\tau}\int_{\Omega}u_i(\xi)f(\tau,\xi)\,d\xi\,d\tau
\end{align*}
and the particular solution
\begin{align*}
u(t,x)&=
\sum_{i=0}^\infty
u_i(x)
\int_0^t
e^{\kappa\lambda_i(t-\tau)}\int_{\Omega}u_i(\xi)f(\tau,\xi)\,d\xi\,d\tau
\end{align*}
Green's function for the heat equation with Dirichlet boundary conditions
can thus be written as
\[
G(t,x,\tau,\xi)
=
\sum_{i=0}^\infty
e^{\kappa\lambda_i (t-\tau)}
u_i(x)
u_i(\xi).
\]
So we have reduced the solution of the parabolic problem to the
elliptic problem.



