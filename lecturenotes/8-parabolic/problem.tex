%
% problem.tex -- XXX
%
% (c) 2019 Prof Dr Andreas Mueller, Hochschule Rapperswil
%
\section{Problemstellung}
\rhead{Problemstellung}
Sei $\Omega$ ein Gebiet in $\mathbb R^n$. Der Operator 
\[
\partial_t-\kappa\Delta
\]
ist ein parabolischer Operator auf $\mathbb R_+\times \Omega$.
Eine Lösung des Wärmeleitungsproblems ist eine Funktion
$u\colon\mathbb R_+\times\Omega\to\mathbb R,$
welche folgende Bedingungen erfüllt
\begin{align*}
\partial_tu(t,x)-\kappa\Delta u(t,x)&=f(t,x)&&(t,x)\in\mathbb R_+\times\Omega
\\
u(0,x)&=u_0(x)&&x\in\Omega
\\
\alpha u+\beta\frac{\partial u}{\partial n}&=g(t, x)&&x\in\partial\Omega, t>0
\end{align*}
Falls das Gebiet $\Omega$ unbeschränkt ist, zum Beispiel $\Omega=\mathbb R^n$,
wird es notwendig sein, zusätzliche Bedingungen für das Verhalten
der Lösung für $t\to\infty$ hinzuzufügen, damit die Lösungen weiterhin
wohlbestimmt sind.

