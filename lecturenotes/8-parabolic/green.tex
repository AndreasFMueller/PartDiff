%
% green.tex -- 
%
% (c) 2019 Prof Dr Andreas Mueller, Hochschule Rapperswil
%
\section{Green's function}
\rhead{Green's function}
We want to extend the singular solutions $K$ to a complete solution 
of the problem with homogeneous boundary conditions.
To this end we are looking for a function
$J(t,x,\tau,\xi)$ that solves the same differential equation as $K$, i.~e.
\begin{align*}
\partial_t J-\kappa\Delta_xJ&=0
&
-\partial_\tau J-\kappa\Delta_\xi J&=0
&t>\tau
\\
J&=0&J&=0&t<\tau
\end{align*}
but has boundary conditions
\begin{align*}
J(t,x,\tau,\xi)&=-K(t,x,\tau,\xi)&&x\in\partial\Omega
\end{align*}
If this problem has a solution, we call the sum
\[
G(t,x,\tau,\xi)=K(t,x,\tau,\xi)+J(t,x,\tau,\xi)
\]
the Green's function for the heat equation with Dirichlet boundary
conditions.
In analogy with  elliptic partial differential equations it is possible
to construct a solution formula
\begin{align}
u(t,x)&=
\int_0^t\int_{\Omega}G(t,x,\tau,\xi)f(\tau,\xi)\,d\xi\,d\tau
\notag
\\
&\quad+\int_{\Omega}G(t,x,0,\xi)u_0(\xi)\,d\xi
\notag
\\
&\quad +\kappa\int_0^t\int_{\partial \Omega}
G(t,x,\tau,\xi)\operatorname{grad}_\xi u
-u\operatorname{grad}_\xi G(t,x,\tau,\xi)\cdot dn\,d\tau
\label{green-parabolisch}
\end{align}
It is also possible to extend the theory to Neumann boundary conditions.

