%
% green.tex -- XXX
%
% (c) 2019 Prof Dr Andreas Mueller, Hochschule Rapperswil
%
\section{Greensche Funktion}
\index{Greensche Funktion}
\rhead{Greensche Funktion}
Wir wollen jetzt die singulären Lösungen $K$ zu einer Lösung des Problems
mit homogenen Randbedingungen erweitern.
Wie beim Laplace-Problem suchen wir jetzt eine Funktion $J(t,x,\tau,\xi)$,
welche wie $K$ die Differentialgleichungen erfüllt, also
\begin{align*}
\partial_t J-\kappa\Delta_xJ&=0
&
-\partial_\tau J-\kappa\Delta_\xi J&=0
&t>\tau
\\
J&=0&J&=0&t<\tau
\end{align*}
und ausserdem die Randbedingung 
\begin{align*}
J(t,x,\tau,\xi)&=-K(t,x,\tau,\xi)&&x\in\partial\Omega
\end{align*}
Falls dieses Problem immer eine Lösung hat, nennen wir die
Summe
\[
G(t,x,\tau,\xi)=K(t,x,\tau,\xi)+J(t,x,\tau,\xi)
\]
wieder die Greensche Funktion für das Wärmeleitungsproblem
mit Dirichlet-Rand\-bedingungen. Und ebenfalls analog wie bei elliptischen
partiellen Differentialgleichungen kann man eine Lösungsformel
angeben
\begin{align}
u(t,x)&=
\int_0^t\int_{\Omega}G(t,x,\tau,\xi)f(\tau,\xi)\,d\xi\,d\tau
\notag
\\
&\quad+\int_{\Omega}G(t,x,0,\xi)u_0(\xi)\,d\xi
\notag
\\
&\quad +\kappa\int_0^t\int_{\partial \Omega}
G(t,x,\tau,\xi)\operatorname{grad}_\xi u
-u\operatorname{grad}_\xi G(t,x,\tau,\xi)\cdot dn\,d\tau
\label{green-parabolisch}
\end{align}

Selbstverständlich kann die Theorie wie bei elliptischen Gleichungen
auch für Neumann-Randbedingungen verfeinert werden.

