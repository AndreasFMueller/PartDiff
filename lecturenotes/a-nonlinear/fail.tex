%
% fail.tex
%
% (c) 2019 Prof Dr Andreas Mueller, Hochschule Rapperswil 
%
\section{What will stop working?}
\rhead{Problems with nonlinear partial differential equations}
In den Lösungsverfahren linearer PDGL wurde die Linearität an verschiedenen
Stellen entscheidend benutzt:
\begin{enumerate}
\item
Das Überlagerungsprinzip ermöglicht, lokale Lösungen, die zum Beispiel
mit Hilfe eines Separationsansatzes gewonnen wurden, so zu kombinieren, dass
Anfangsbedingungen erfüllt werden können.
\item
Partikuläre Lösung und Lösung des homogenen Systems.
Das Überlagerungsprinzip ermöglicht die Lösung des inhomogenen Systems
in zwei Schritten. Einerseits wird eine beliebige Lösung der inhomogenen
Gleichung ermittelt, andererseits werden eventuell geforderte Anfangs-
oder Randbedingungen mit Hilfe der allgemeinen Lösungen des homogenen
Problems befriedigt.
\item
Konstruktion der Greenschen Funktion. In der Konstruktion der
Greenschen Funktion war verwendet worden, dass Singularitäts-Lösungen 
der PDGL mit harmonischen Funktionen kombiniert werden können, so 
dass sie auch die Randbedingungen erfüllen.
\end{enumerate}

