%
% burgers.tex -- XXX
%
% (c) 2019 Prof Dr Andreas Mueller, Hochschule Rapperswil 
%
\section{Burgers' Equation\label{burgers}}
\rhead{Burgers' equation}
Als Beispiel einer nichtlinearen Gleichung betrachten wir die Gleichung
von Burgers in der Form
\[
\partial_t u=\partial_x^2u+u\partial_xu
\]
und zeigen ein paar Möglichkeiten, wie solche Gleichungen
gelöst werden können.

\subsection{Coordinate transforms}
Koordinatentransformationen können helfen, die Eigenschaften der
Lösungen einer partiellen Differentialgleichung zu ergründen.

Sie $u(t,x)$ eine Lösung der Burgers Gleichung. Dann sind auch
zeitlich und örtlich verschobenen Kopien
\[
w(t,x)=u(t+C_4, x+C_3)
\]
der Funktion $u$ Lösungen.
Streckt man hingegen die $x$-Achse mit dem Faktor $C_1$, ersetzt
also $x$ durch $C_1x$, dann
muss man auch die $t$-Achse entsprechend korrigieren, also $t$
durch $C_1^2t$ ersetzen. Setzt man
\[
w(t,x)=u(C_1^2t,C_1x)
\]
in die Burgers Gleichung ergibt
\[
C_1^2\partial_t w(t,x)=C_1^2\partial_xw(t,x)+C_1w\partial_xw(t,x)
\]
was offenbar die Gleichung nicht löst. Erst die Funktion $C_1w(t,x)$
erfüllt die Gleichung, denn damit wird die Differentialgleichung zu
\[
C_1^3\partial_t w(t,x)=C_1^3\partial_xw(t,x)+C_1^3w\partial_xw(t,x)
\]
Aus einer zeitabhängigen Verschiebung von $u$ in der Form
\[
w(t,x)=u(t,x+C_2t)
\]
kann ebenfalls eine Lösung gewonnen werden:
\begin{align*}
\partial_t w(t,x)&=\partial_t u(t,x+C_2t)+C_2\partial_x u(t,x+C_2t)
\\
\partial_x w(t,x)&=\partial_x u(t,x+C_2t)
\\
\partial_x^2 w(t,x)&=\partial_x^2 u(t,x+C_2t)
\end{align*}
eingesetzt in die Differentialgleichung ergibt die Gleichung
\[
\partial_t u(t,x+C_2t)+C_2\partial u(t,x+C_2t)
=
u(t,x+C_2t)\partial_xu(t,x+C_2t)
+
\partial_x^2 u(t,x+C_2t)
\]
die jedoch wegen des zweiten Terms auf der linken Seite nicht erfüllt
sein kann. Addiert man aber zu $w$ noch die Konstante $C_2$, ergibt sich
aus dem nichtlinearen Term auf der rechten Seite zusätzlich
\[
C_2\partial_xu(t,x+C_2t),
\]
so dass also
\[
w(t,x)=u(t,x+C_2t)+C_2
\]
eine Lösung der Burgers Gleichung ist.

\subsection{Stationary solutions}
Hat die Burgers Gleichung stationäre Lösungen? Eine stationäre Lösung ist
eine Lösung, die nicht von der Zeit abhängt, also
$u(t,x)=u(x)$.
Eine stationäre Lösung muss die gewöhnliche
Differentialgleichung
\[
u''(x)+u(x)u'(x)=0
\]
erfüllen. Der zweite Term ist bis auf einen Faktor die Ableitung
des Quadrates $u(x)^2$. Durch die Substitution $u(x)=y(x/2)$ kann man
die Differentialgleichung in die Form
\begin{align*}
\frac14y''(x)+\frac12y(x)y'(x)&=0
\\
y''(x)+2y(x)y'(x)&=0&\Rightarrow&\qquad y''(x)+\frac{d}{dx}(y(x)^2)=0
\end{align*}
bringen.
Dies ist die Ableitung der Gleichung
\[
y'(x)+y^2(x)=B,
\]
die mit Separation gelöst werden kann:
\begin{align*}
\frac{dy}{dx}&=B-y^2\\
\int\frac{dy}{B-y^2}&=x+A
\end{align*}
Für $B>0$  findet man das Integral in Formelsammlungen als
\[
\frac1{\sqrt{B}}\operatorname{ar}\tanh \frac{y}{\sqrt{B}}=x+A
\]
Nach $y$ aufgelöst findet man also
\begin{align*}
y(x)&=\sqrt{B}\tanh(\sqrt{B}(x+A))
\end{align*}
Durch Einsetzen findet man jetzt auch die Lösung
\begin{align*}
u(x)&=
\sqrt{B}\tanh\left(\sqrt{B}\left(\frac{x}2+A\right)\right)
\end{align*}
Da die genauen Werte der Integrationskonstanten bedeutungslos sind, können
wir die Lösung auch als
\[
u(x)= 2A \tanh (Ax+B) 
\]
schreiben.

Mit Hilfe der Koordinatentransformation findet man jetzt weitere Lösungen
der Gleichung
\[
u(t,x)=
2A \tanh (A(x+\lambda t)+B) +\lambda.
\]

\subsection{Special Solutions}
Aus der physikalischen Motivation für die Gleichung lassen sich auch
einige spezielle Lösungen ableiten.
Nimmt die Geschwindigkeit linear zu, dann bleibt dies über die Zeit auch
so, aber die Steigung der Geschwindigkeitszunahme wird sich ändern.
Die Funktion
\[
u(t,x)=\frac{A-x}{B+t}
\]
sollte daher
eine Lösung sein. Tatsächlich findet man durch Einsetzen
\begin{align*}
\partial_t u(t,x)&=-\frac{A-x}{(B+t)^2}
\\
\partial_x u(t,x)&=-\frac{1}{B+t}
\\
\partial_x^2 u(t,x)&=0
\\
u(t,x)
\partial_xu(t,x)&=-\frac{A-x}{(B+t)^2}
\end{align*}
Da die zweiten Ableitungen nach $x$ verschwinden,
zeigen die erste und letzte Gleichung, dass die Burgers-Gleichung
erfüllt ist.

