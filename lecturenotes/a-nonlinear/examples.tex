%
% examples.tex
%
% (c) 2019 Prof Dr Andreas Mueller, Hochschule Rapperswil 
%
\section{Bespiele nichtlinearer PDGL}
\rhead{Beispiele}
Wir stellen ein paar Beispiel nichtlinearer Differentialgleichungen
zusammen, die sich in Anwendungsproblemen stellen. Einzelne Beispiele
kamen bereits im ersten Kapitel zur Sprache.

\subsection{Eulersche Gleichung}
Die Eulersche Gleichung
\begin{equation}
\frac{\partial \vec v}{\partial t}+(\vec v\cdot\nabla)\vec v=-\frac1\varrho\operatorname{grad}p
\label{euler}
\end{equation}
drückt das Newtonsche Gesetz für eine Flüssigkeit der Dichte $\varrho$ aus,
welche sich mit Geschwindigkeit $\vec v$ bewegt.
Zusammen mit der Kontinuitätsgleichung
\[
\frac{\partial\varrho}{\partial t}+\operatorname{div}(\varrho \vec v)=0
\]
ergeben sich vier Gleichungen für die fünf unbekannten Funktionen
$\vec v$ (drei Komponenten) und $\varrho$ und $p$. Für eine vollständige
Lösung des Problems sind also noch weitere Gleichungen notwendig, auf die
wir jedoch nicht eingehen wollen.

Als Gleichung für $\vec v$ ist (\ref{euler}) nicht linear wegen des Termes
$(\vec v\cdot \nabla)\vec v$. Ausgeschrieben ist seine $i$-Komponente
\[
\biggl(\sum_{k=1}^3v_k\partial_k\biggr)v_i.
\]
In nur einer Dimension wird die Gleichung zu
\[
\partial_tv+v\partial_xv=-\frac1\varrho\operatorname{grad}p
\]
Ist $v$ eine Lösung der homogenen Gleichung, also
\[
\partial_t v+v\partial_x v=0,
\]
dann erfüllt $\lambda v$ die Differentialgleichung nicht mehr. Würde
$\lambda v$ die Differentialgleichung erfüllen, wurde folgen
\begin{align*}
&&\lambda(\partial_t v+\lambda v\partial_xv)&=0
\\
\Rightarrow
&&
\partial_t v+\lambda v\partial_xv&=0
\\
\Rightarrow
&&
(\lambda -1)v\partial_xv&=0
\end{align*}
Dies ist jedoch nur möglich, wenn $\partial_xv=0$, also für eine
konstante Strömung.

\subsection{Navier-Stokes Gleichung}
Die wohl berühmteste nichtlineare partielle Differentialgleichung ist
die Navier-Stokes Gleichung, welche ein strömendes zähes Medium beschreibt.
\[
\varrho\biggl(
\frac{\partial \vec v}{\partial t}+(\vec v\operatorname \nabla)\vec v
\biggr)
=
-\operatorname{grad}p
+\eta\Delta \vec v+\biggl(\zeta+\frac{\eta}3\biggr)\operatorname{grad}\operatorname{div}\vec v
\]
Auch diese Gleichung für $\vec v$ ist wegen des Ausdrucks $(\vec v\cdot\nabla)\vec v$
nicht linear.

\subsection{Gleichung von Burgers}
In einer Dimension  wird die Navier-Stokes Gleichung für eine inkompressibles
Medium zu
\[
\varrho(\partial_t v+v\partial_x v)-a\partial_x^2v=0
\]
($a$ ist ein Ausdruck in $\zeta$ und $\eta$).
Durch die Substitution
\[
u(t,x)=v(t,-x)
\]
erhält man daraus 
\begin{align*}
\partial_tu(t,x)&=\partial_tv(t,-x)\\
\partial_xu(t,x)&=-\partial_xv(t,-x)\\
\partial_x^2u(t,x)&=\partial_x^2v(t,-x)\\
\varrho
\partial_t v+v\partial_x v-\partial_x^2v
&=
\varrho(\partial_tu(t,x)-u\partial_x(t,x))-a\partial_x^2u
\end{align*}
Für $\varrho=1$ und $a=1$ wird die Gleichung zu
\[
\partial_tu=u\partial_xu+\partial_x^2u,
\]
in dieser Form heisst sie die Gleichung von Burgers,
und wird weiter unten im Abschnitt
\ref{burgers} genauer studiert.

Aus der Euler Gleichung lässt sich analog die Burgers Gleichung
für die ideale Flüssigkeit ableiten:
\[
\frac{\partial}{\partial t}u+\frac{\partial}{\partial x}\left(\frac{u^2}2\right)
=0
\]
Wir verwenden diese in Abschnitt \ref{burgersunstetig} um zu illustrieren,
wie nichtlineare Gleichungen spontan Unstetigkeiten entwickeln können.

\subsection{Korteweg-deVries Gleichung}
Die Wellenausbreitung in einem Kanal kann mit einer PDGL beschrieben werden,
welche äquivalent ist zu
\[
\partial_tu+6u\partial_xu+\partial_x^3u=0.
\]
Wegen des Terms $u\partial_xu$ ist auch diese Gleichung nicht linear.
Sie ist berühmt für die sogenannten Solitonen, Wellen, die über
lange Zeit ihre Form erhalten.

