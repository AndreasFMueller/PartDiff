%
% characteristics.tex -- XXX
%
% (c) 2019 Prof Dr Andreas Mueller, Hochschule Rapperswil 
%
\section{Charakteristiken bei nichtlinearen PDGL erster Ordnung}
Im Kapitel~\ref{chapter-geometrie} war die Idee erfolgreich, die Lösung,
die man sich als Fläche im Raum mit den Koordinaten $(x,y,u)$
vorstellen konnte, aus Kurven zusammenzusetzen, die sich auf der
Fläche befinden. Das hat vor allem deshalb funktioniert, weil es
einfach war, einen Vektor zu finden, der tangential an die Fläche
war. Im wesentlichen war die quasilineare Differentialgleichung ein
Skalarprodukt mit der Flächennormalen als einem Faktor, so dass
der andere Faktor eine Tangente sein musste. Da alle möglichen
normalen in einer Ebene lagen, gab es nur eine einzige gemeinsme
Tangente, welche wir für die Differentialgleichung der
Charakteristiken verwenden konnten.

Im nichtlinearen Fall funktioniert dies nicht mehr so einfach.
Die möglichen Normalen liegen nicht mehr alle in einer Ebene,
weil die Beziehung zwischen den partiellen Ableitungen nicht
mehr linear ist. Es gibt also auch nicht mehr nur eine einzige
Tangente, die Wahl der geeigneten Tangente hängt jetzt auch
von den Werten der partiellen Ableitungen in einem Punkt ab.
Es genügt also nicht mehr nach einer Kurve
\[
s\mapsto
\begin{pmatrix}
x(s)\\
y(s)\\
u(s)
\end{pmatrix}
\]
zu finden, stattdessen müssen wir eine Kurve 
\[
s\mapsto
\begin{pmatrix}
x(s)\\
y(s)\\
u(s)\\
p(s)\\
q(s)
\end{pmatrix}.
\]
suchen, wobei $p$ und $q$ wieder für die partiellen Ableitungen
steht. Ziel ist jetzt also, ein System von Differentialgleichungen
für diese fünf Funktionen zu finden.

\subsection{Cauchy-Methode im nichtlinearen Fall}
Gegeben sei eine nichtlineare Differentialgleichung von zwei
Variablen $x$ und $y$, die wir in der Form
\[
F(x,y,u,p,q)=0
\]
schreiben, wobei $p$ und $q$ wieder für die partiellen Ableitungen
von $u$ nach $x$ bzw.~$y$ stehen.

Wir suchen Kurven auf der Lösungsfläche, wir verwenden $s$ als
Kurvenparameter.
Nehmen wir weiter an, wir hätten bereits eine Lösungsfunktion $u(x,y)$.
Weil es jetzt keinen
einfachen linearen Zusammenhang zwischen den ersten Ableitungen
mehr gibt, wie im quasilinearen Fall, müssen wir allgemeiner
auch $p(s)$ und $p(s)$ mitschleppen, wir suchen also fünf
Funktionen
\begin{equation}
x(s),\;
y(s),\; 
u(s),\; 
p(s),\; 
\text{und}\;
q(s),
\label{nichtlinear:kurve}
\end{equation}
so dass die Kurve $(x(s),y(s),z(s))$ auf der Lösungsfläche
verläuft und ausserdem die partiellen Ableitungen 
von $u(x,y)$ mit den Werten von $p$ und $q$ übereinstimmen.
Es muss also gelten
\begin{align*}
\frac{\partial u}{\partial x}(x(s), y(s))&=p(s)\\
\frac{\partial u}{\partial y}(x(s), y(s))&=q(s)
\end{align*}
Setzt man die Funktionen (\ref{nichtlinear:kurve}) in die Funktion
$F$ ein, ist die Gleichung immer erfüllt. Als muss auch die Ableitung
verschwinden:
\begin{align}
0
&=
\frac{d}{ds}F(x(s),y(s),u(s),p(s),q(s))=0
\notag
\\
&=
\frac{\partial F}{\partial x}x'(s)
+
\frac{\partial F}{\partial y}y'(s)
+
\frac{\partial F}{\partial u}u'(s)
+
\frac{\partial F}{\partial p}p'(s)
+
\frac{\partial F}{\partial q}q'(s)
\label{nichtlinear:totale}
\end{align}
Die Ableitung der Bedingung, dass die Kurve in der Fläche bleiben muss,
liefert eine weitere Gleichung für die unbekannten Funktionen:
\begin{align}
\frac{d}{ds}u(x(s),y(s))&=\frac{d}{ds}u(s)
\notag
\\
\frac{\partial u}{\partial x}x'(s)
+
\frac{\partial u}{\partial y}y'(s)
&=
u'(s)
\notag
\\
p(s)x'(s)+q(s)y'(s)
&=
u'(s).
\label{nichtlinear:flaeche}
\end{align}
%Schliesslich muss auch gelten, dass die gemischten zweiten Ableitungen
%nicht von der Reihenfolge der Ableitungen abhängen:
%\[
%\frac{\partial^2u}{\partial x\partial y}
%=
%\frac{\partial^2u}{\partial y\partial x}.
%\]
Insgesamt hat man jetzt also die Gleichungen
\begin{equation}
\begin{pmatrix}
\partial_x F+p\partial_uF& \partial_y F+q\partial_uF&0&\partial_pF&\partial_qF\\
p&q&-1&0&0
\end{pmatrix}
\begin{pmatrix}
x'(s)\\
y'(s)\\
u'(s)\\
p'(s)\\
q'(s)
\end{pmatrix}=0.
\label{nichtlinear:gleichungen}
\end{equation}
Sowenig wie die Wahl der Tangenten bei den quasilinearen partiellen
Differentialgleichungen eindeutig war, kann man erwarten dass im
allgemeinen nichtlinearen Fall ein Richtungsvektor eindeutig zu
bestimmen ist. Aber der Vektor 
\[
\begin{pmatrix}
\partial_pF\\
\partial_qF\\
p\partial_pF+q\partial_qF\\
-\partial_xF-p\partial_uF\\
-\partial_yF-q\partial_uF
\end{pmatrix}
\]
löst das Gleichungssystem (\ref{nichtlinear:gleichungen}).
Man kann also versuchen, die Kurve als Lösungskurve des
Differentialgleichungssystems
\begin{align*}
x'(s)
&=
\partial_pF
\\
y'(s)
&=
\partial_qF
\\
u'(s)
&=
p\partial_pF+q\partial_qF
\\
p'(s)
&=
-\partial_xF-p\partial_uF
\\
q'(s)
&=
-\partial_yF-q\partial_uF
\end{align*}
Diese Kurven heissen die Charakteristiken der nichtlinearen
Differentialgleichung.
Sie können auf die genau gleiche Art zur Lösung der Gleichung
verwendet werden wie im quasilinearen Fall.

\subsection{Beispiel: Eikonal-Gleichung}
Als Beispiel betrachten wir die Eikonal-Gleichung 
\[
\biggl(\frac{\partial u}{\partial x}\biggr)^2
+
\biggl(\frac{\partial u}{\partial y}\biggr)^2
=
n(x,y)^2.
\]
Sie spielt in der Optik eine wichtige Rolle, sie beschreibt die
Phase einer Welle, die sich in einem Medium mit Brechungsindex $n(x,y)$
ausbreitet. Wir wollen die Gleichung im Gebiet $\Omega=\{(x,y)\,|\,y>0\}$
lösen mit der Randbedingung $u(0,y)=0$. Weiter unten werden wir auch 
die Funktion $n(x,y)$ noch weiter einschränken.

\subsubsection{Differentialgleichungen der Charakteristiken}
Die Funktion $F$ ist offenbar
\[
F(x,y,u,p,q)=p^2+q^2-n(x,y)^2
\]
Die partiellen Ableitungen sind
\begin{align*}
\partial_xF&=-2n(x,y)\partial_x n(x,y)\\
\partial_yF&=-2n(x,y)\partial_y n(x,y)\\
\partial_uF&=0\\
\partial_pF&=2p\\
\partial_qF&=2q.
\end{align*}
Wir spezialisieren jetzt auf den Fall $n(x,y)=y$, dann wird das 
Differentialgleichungssystem für die Charakteristiken
\begin{align*}
x'&=\partial_pF=2p\\
y'&=\partial_qF=2q\\
u'&=p\partial_pF+q\partial_qF=2p^2+2q^2\\
p'&=-\partial_xF-p\partial_uF=0\\
q'&=-\partial_yF-q\partial_uF=2y
\end{align*}
Wir suchen eine Familie von Kurven, welche für $s=0$ jeweils
im Punkt $(0,y_0,0)$ beginnen.
Ausserdem muss auch $\partial_yu(0,y_0)=q(0,y_0)=0$ gelten.

\subsubsection{Lösung der Charakteristikengleichung}
Aus der Gleichung für $p'$ folgt, dass $p$ entlang einer Charakteristik
konstant ist, $p=p_0$. Aus der Gleichung für $x'$ folgt, dann
dass $x=p_0s+p_1$ ist. Weil aber $x(0)=0$ sein muss, folgt $p_1=0$.
Offenbar spielt es auch keine Rolle, welchen Wert wir für $p_0$
wählen, eine andere Wahl ändert nur die Geschwindigkeit, mit der
die Charakteristik durchlaufen wird. Wir setzen also $s=x$.

Die beiden Funktionen sind über die zweite und fünfte Gleichung
aneinander gekoppelt, leitet man die zweite einmal ab und setzt die
fünfte ein, erhält man die Gleichung
\[
y''=4y.
\]
Diese hat die bekannten Lösungen $\cosh 2s$ und $\sinh 2s$.
Mit geeigneten Konstanten $A$ und $B$ ist die allgemeine
Lösung also
\[
y(x)=A\cosh 2x + B\sinh 2x.
\]
Da $y(0)=y_0$ gelten muss, folgt $A=y_0$. Durch Ableiten finden wir
$y'(x)=2q(x)=2y_0\sinh 2x + 2B\cosh 2x$, also ist $q(x)=y_0\sinh 2x+B\cosh 2x$. 
Weil aber für $x=0$ gelten muss $q(0)=0$, ist $B=0$.
Somit haben wir die Funktionen $y$ und $q$ vollständig bestimmt:
\begin{align}
y(x)&=y_0\cosh 2x,
\label{nichtlinear:y}
\\
q(x)&=y_0\sinh 2x.\notag
\end{align}

Jetzt bleibt nur noch $u$ zu bestimmen, wozu wir die dritte Gleichung
verwenden:
\begin{align}
u'&=2p^2+2q^2=
2 + 2y_0^2\cosh^22x
\notag
\\
\Rightarrow\qquad
u&=\int
2 + 2y_0^2\cosh^22x
\,dx
\notag
\\
&=2x +2y_0^2\biggl(\frac18\sinh 4x+\frac12x\biggr)
\label{nichtlinear:l2}
\end{align}
Quadriert man (\ref{nichtlinear:y}), erhält man
\begin{align*}
y^2
&=
y_0^2\biggl(\frac12\cosh 4x+\frac12\biggr)
\\
y_0^2&=\frac{2y^2}{\cosh4x + 1}.
\end{align*}
Dies kann man in (\ref{nichtlinear:l2}) einsetzen, was die Lösung
liefert:
\begin{equation}
u(x,y)
=
2x+\frac{y^2(\sinh4x+4x)}{4(\cosh4x+1)}
\label{nichtlinear:loesung}
\end{equation}

