%
% linearization.tex -- XXX
%
% (c) 2019 Prof Dr Andreas Mueller, Hochschule Rapperswil 
%
\section{Linearisierung}
\rhead{Linearisierung}
In einigen Fällen sind Lösungen einer nichtlinearen PDGL gesucht, die 
nur wenig von einer bekannten Lösung abweichen. Beispielsweise ändert
ein stromlinienförmig gebautes Flugzeug bei hoher Geschwindigkeit die
Strömung nur vergleichsweise wenig. Man kann daher versuchen, aus der
nichtlinearen Gleichung eine lineare Gleichung für die Abweichung
von der bekannten Lösung abzuleiten. Dieses Verfahren wird oft auch
Störungstheorie genannt.

\subsection{Das allgemeine Vorgehen}
Der Einfachheit halber führen wir das Verfahren nur für Gleichungen
erster Ordnung in zwei Variablen durch. Eine solche PDGL kann mit Hilfe
einer Funktion $F(x,y,u,p,q)$ von fünf Variablen geschrieben werden als
\begin{equation}
F(x,y,u(x,y), \partial_xu(x,y),\partial_yu(x,y))=0.
\label{nichtlinear}
\end{equation}
Sei $u(x,y)$ eine Lösung der Gleichung (\ref{nichtlinear}). Wir suchen
jetzt weitere Lösungen der Gleichung, die sich jedoch nur wenig von
$u$ unterscheiden dürfen. Wir setzen diese Lösungen in der Form
\begin{equation}
u(x,y)+av(x,y)
\label{linearisierungansatz}
\end{equation}
an, wobei der Parameter $a$ dazu dienen soll, die den
zweiten Term beliebig klein machen zu können. Wir möchten eine Gleichung
für $v(x,y)$ aufstellen.

Wir setzen den Ansatz (\ref{linearisierungansatz}) in die Gleichung
ein, und erhalten
\[
F(x,y,u(x,y)+av(x,y),\partial_xu(x,y)+a\partial_xv(x,y),
\partial_yu(x,y)+a\partial_yv(x,y))=0
\]
Für $a=0$ ist die Gleichung erfüllt, wir suchen ein $v$ so dass die Gleichung
für kleine $a$ näherungsweise auch erfüllt ist. Dies erreichen wir,
indem wir nach $a$ ableiten:
\begin{align*}
0&=
\left.\frac{d}{da}
F(x,y,u(x,y)+av(x,y),\partial_xu(x,y)+a\partial_xv(x,y),
\partial_yu(x,y)+a\partial_yv(x,y))\right|_{a=0}
\\
&=F(x,y,u(x,y),\partial_xu(x,y),\partial_yu(x,y))
\\
&\qquad
+
\partial_uF(x,y,u(x,y),\partial_xu(x,y),\partial_yu(x,y))\cdot v(x,y)
\\
&\qquad
+
\partial_pF(x,y,u(x,y),\partial_xu(x,y),\partial_yu(x,y))\cdot \partial_xv(x,y)
\\
&\qquad
+
\partial_qF(x,y,u(x,y),\partial_xu(x,y),\partial_yu(x,y))\cdot \partial_yv(x,y)
\end{align*}
Der erste Term fällt weg, weil $u$ bereits eine Lösung ist,
es bleibt eine lineare PDGL für $v$:
\begin{align*}
0&=
\partial_uF(x,y,u(x,y),\partial_xu(x,y),\partial_yu(x,y))\cdot v(x,y)
\\
&\qquad
+
\partial_pF(x,y,u(x,y),\partial_xu(x,y),\partial_yu(x,y))\cdot \partial_xv(x,y)
\\
&\qquad
+
\partial_qF(x,y,u(x,y),\partial_xu(x,y),\partial_yu(x,y))\cdot \partial_yv(x,y)
\end{align*}
Etwas allgemeiner könnte auch noch die Funktion $F$ von $a$ abhängen,
also $F(a,x,y,u,p,q)$. In diesem Fall wird die Ableitung nach $a$ an
der Stelle $a=0$ zu
\begin{align*}
0&=
\left.\frac{d}{da}
F(x,y,u(x,y)+av(x,y),\partial_xu(x,y)+a\partial_xv(x,y),
\partial_yu(x,y)+a\partial_yv(x,y))\right|_{a=0}
\\
&=
F(0,x,y,u(x,y),\partial_xu(x,y),\partial_yu(x,y))
\\
&\qquad
+\partial_aF(0,x,y,u(x,y),\partial_xu(x,y),\partial_yu(x,y))
\\
&\qquad
+
\partial_uF(a,x,y,u(x,y),\partial_xu(x,y),\partial_yu(x,y))\cdot v(x,y)
\\
&\qquad
+
\partial_pF(a,x,y,u(x,y),\partial_xu(x,y),\partial_yu(x,y))\cdot \partial_xv(x,y)
\\
&\qquad
+
\partial_qF(a,x,y,u(x,y),\partial_xu(x,y),\partial_yu(x,y))\cdot \partial_yv(x,y)
\end{align*}
Die lineare PDGL ist in diesem Fall
\begin{align*}
0
&=
\partial_aF(0,x,y,u(x,y),\partial_xu(x,y),\partial_yu(x,y))
\\
&\qquad
+
\partial_uF(x,y,u(x,y),\partial_xu(x,y),\partial_yu(x,y))\cdot v(x,y)
\\
&\qquad
+
\partial_pF(x,y,u(x,y),\partial_xu(x,y),\partial_yu(x,y))\cdot \partial_xv(x,y)
\\
&\qquad
+
\partial_qF(x,y,u(x,y),\partial_xu(x,y),\partial_yu(x,y))\cdot \partial_yv(x,y)
\end{align*}
Jede Lösung der nichtlinearen Gleichung gibt also Anlass zu Lösungen
in ``unmittelbarer'' Nähe, welche aus der linearisierten Gleichung
gefunden werden können.

\subsection{Linearisierung von PDGL zweiter Ordnung}
Eine PDGL zweiter Ordnung ist gegeben durch eine Funktion von neun
Variablen
\[
F(x,y,u,p,q,r,s,t),
\]
in die man die Funktionswerte und die Ableitungen einsetzt:
\[
F(x,y,u(x,y), \partial_xu(x,y),\partial_yu(x,y),
\partial_x^2u(x,y),
\partial_x\partial_yu(x,y),
\partial_y^2u(x,y))
=0
\]
Um die linearisierte PDGL zu finden, geht man wieder von einer
Lösung $u(x,y)$ aus, und sucht eine ``Nachbarlösung'' in der
Form $u(x,y)+av(x,y)$. Diesen Ansatz setzt man in die 
Differentialgleichung ein und leitet an der Stelle $a=0$
nach $a$ ab. Wie im Falle der Gleichung erster Ordnung kann auch
hier der Parameter $a$ auch in $F$ vorkommen, wir führen gleich
von Anfang an diesen allgemeineren Fall  durch:
\begin{align*}
0&=
\frac{d}{da}
F(a,x,y,u(x,y), \partial_xu(x,y)+a\partial_xv(x,y),\partial_yu(x,y)+a\partial_yv(x,y),
\\
&\qquad
\partial_x^2u(x,y)+a\partial_x^2v(x,y),
\partial_x\partial_yu(x,y)+a\partial_x\partial_yv(x,y),
\partial_y^2u(x,y)+a\partial_y^2v(x,y))\bigg|_{a=0}
\\
&=
F(x,y,u(x,y), \partial_xu(x,y),\partial_yu(x,y),
\partial_x^2u(x,y),
\partial_x\partial_yu(x,y),
\partial_y^2u(x,y))
\\
&\qquad+
\partial_a
F(0,x,y,u(x,y), \partial_xu(x,y),\partial_yu(x,y),
\partial_x^2u(x,y),
\partial_x\partial_yu(x,y),
\partial_y^2u(x,y))
\\
&\qquad+
\partial_u
F(0,x,y,u(x,y), \partial_xu(x,y),\partial_yu(x,y),
\partial_x^2u(x,y),
\partial_x\partial_yu(x,y),
\partial_y^2u(x,y))\cdot v(x,y)
\\
&\qquad+
\partial_p
F(0,x,y,u(x,y), \partial_xu(x,y),\partial_yu(x,y),
\partial_x^2u(x,y),
\partial_x\partial_yu(x,y),
\partial_y^2u(x,y))\cdot \partial_xv(x,y)
\\
&\qquad+
\partial_q
F(0,x,y,u(x,y), \partial_xu(x,y),\partial_yu(x,y),
\partial_x^2u(x,y),
\partial_x\partial_yu(x,y),
\partial_y^2u(x,y))\cdot \partial_yv(x,y)
\\
&\qquad+
\partial_q
F(0,x,y,u(x,y), \partial_xu(x,y),\partial_yu(x,y),
\partial_x^2u(x,y),
\partial_x\partial_yu(x,y),
\partial_y^2u(x,y))\cdot \partial_x^2v(x,y)
\\
&\qquad+
\partial_r
F(0,x,y,u(x,y), \partial_xu(x,y),\partial_yu(x,y),
\partial_x^2u(x,y),
\partial_x\partial_yu(x,y),
\partial_y^2u(x,y))\cdot \partial_x\partial_yv(x,y)
\\
&\qquad+
\partial_s
F(0,x,y,u(x,y), \partial_xu(x,y),\partial_yu(x,y),
\partial_x^2u(x,y),
\partial_x\partial_yu(x,y),
\partial_y^2u(x,y))\cdot \partial_y^2v(x,y)
\end{align*}

\subsection{Linearisierung der Burgers Gleichung}
Wir wenden das Linearisierungsverfahren auf die Gleichung von Burgers an.
Wir schreiben sie zur leichteren Übertragbarkeit der Formeln
mit $x$ und $y$ als unabhängige Variablen anstellen von $x$ und $t$,
wir suche also Lösungen der Differentialgleichung
\[
\partial_yu-u\partial_xu-\partial_x^2u=0
\]
Die Funktion $F$ ist in diesem Fall
\[
F(x,y,u,p,q,r,s,t)=q-r-up.
\]
Die Linearisierungsformeln sagen, dass wir die Ableitungen von $F$ nach
den Variablen verwenden müssen als Koeffizienten der Ableitungen,
für die die Variablen stehen.  Dabei müssen wir in die partiellen Ableitungen
von $F$ jeweils die Lösung $u$ bzw.~ihre partiellen Ableitungen einsetzen.
Die Koeffizienten sind
\begin{align*}
\partial_uF&=-p=-\partial_xu(x,y)
\\
\partial_pF
&=-u=-u(x,y)
\\
\partial_qF
&=1
\\
\partial_rF
&=-1
\end{align*}
alle anderen partiellen Ableitungen von $F$ verschwinden. Die linearisierte
Gleichung lautet also
\begin{align*}
-\partial_xu(x,y)\cdot v(x,y)
-
u(x,y)\cdot\partial_xv(x,y)
+\partial_yv(x,y)
-\partial_x^2v(x,y)=0,
\end{align*}
eine parabolische PDGL. In den ursprünglichen Koordinaten und der
üblichen Reihenfolge der Ableitungen geschrieben
lautet sie
\[
-\partial_x^2v(t,x)
+\partial_tv(t,x)
+ u(t,x)\cdot\partial_xv(t,x)
+\partial_xu(t,x)\cdot v(t,x)
=0,
\]

\subsubsection{Konstante Geschwindigkeit}
Die konstante Funktion $u(t,x)=c$ ist eine Lösung der nichtlinearen
Gleichung. Die linearisierte Gleichung wird damit zu
\[
\partial_tv
-\partial_x^2v
-c\partial_xv=0.
\]
Leider ist diese Gleichung auch noch nicht direkt in einer Form,
in der wir sie lösen können. Schreiben wir die gesuchte Funktion
in der Form
\[
v(t,x)=w(t,x+ct)
\]
und bezeichnen die partiellen Ableitungen von $w$ nach der ersten
und zweiten Variablen mit $\partial_1w$ bzw.~$\partial_2w$, dann
erhalten wir zunächst die partiellen Ableitungen von $v$
in der Form
\begin{align*}
\partial_t v(t,x)&=\partial_1w(t,x+ct)+c\partial_2w(t,x+ct)
\\
\partial_x v(t,x)&=\partial_2w(t,x+ct)
\\
\partial_x^2v(t,x)&=\partial_2^2w(t,x+ct)
\end{align*}
In die PDGL eingesetzt erhalten wir
\begin{align*}
0&=
\partial_t v(t,x)
-\partial_x^2v(t,x)
-c\partial_x v(t,x)
\\
&=
\partial_1w(t,x+ct)+c\partial_2w(t,x+ct)
-\partial_2^2w(t,x+ct)
-c\partial_2w(t,x+ct)
\\
&=\partial_1w(t,x+ct)-\partial_2w(t,x+ct)
\end{align*}
Die Funktion $w$ ist also eine Lösung der Wärmeleitungsgleichung.

\subsubsection{Lösungen der linearisierten Gleichung}
Die Wärmeleitungsgleichung hat die Standardlösungen
\[
w(t,x)=\frac1{\sqrt{t}}e^{-\frac{(x-\xi)^2}{4t}},
\]
die man durch Nachrechnen unmittelbar bestätigen kann:
\begin{align*}
\partial_t w(t,x)
&=
\left(
-\frac1{2t^{\frac32}}
+\frac{(x-\xi)^2}{4t^{\frac52}}
\right)e^{-\frac{x^2}{4t}}
\\
\partial_x w(t,x)
&=
-\frac{(x-\xi)}{2t^{\frac32}}
e^{-\frac{x^2}{4t}}
\\
\partial_x^2w(t,x)
&=
\left(
\frac{(x-\xi)^2}{4t^{\frac52}}
-\frac{1}{2t^{\frac32}}
\right)e^{-\frac{x^2}{4t}}
\\
\partial_tw(t,x)-\partial_x^2w(t,x)
&=
\left(
-\frac1{2t^{\frac32}}
+\frac{(x-\xi)^2}{4t^{\frac52}}
-\frac{(x-\xi)^2}{4t^{\frac52}}
+\frac{1}{2t^{\frac32}}
\right)e^{-\frac{(x-\xi)^2}{4t}}=0
\end{align*}
Als Lösungen der linearisierten Gleichung kommen also Funktionen der
Form
\[
v(t,x)=\frac1{\sqrt{t}}e^{-\frac{(x-\xi+ct)^2}{4t}}
\]
oder Überlagerungen derselben in Frage.

\subsubsection{Stabilität der Strömung}
Die im vorangegangenen Abschnitt abgeleiteten Lösung der linearisierten
Gleichung lässt sich auch so interpretieren: eine kleine Störung zur Zeit
$t=0$ wird Anlass zu einem mit Geschwindigkeit $c$ nach links
laufenden gaussschen Wellenbuckel Anlass geben. Dieser Buckel wird mit
der Zeit immer flacher und ausgedehnter werden. Insbesondere verschwinden
kleine Störungen mit der Zeit wieder, die Strömung ist also stabil.

Tritt jedoch eine Störung auf, die so gross ist, dass die Linearisierung
nicht mehr anwendbar ist, dann werden neue Phänomene auftreten, insbesondere kann
die Strömung instabil sein. Störungen können sich aufschaukeln bis die
Lösung nicht mehr stetig ist (Geschwindigkeitssprünge, Schockwellen)
oder die Geschwindigkeit könnte unvorhersagbar zu schwanken beginnen
(Turbulenz).

Alternativ könnte man auch mit dem Maximumprinzip argumentieren: Extremwerte
sind in der Anfangsbedingung zu finden, eine Störung muss also mit der
Zeit immer schwächer werden.

\subsubsection{Ansteigende Geschwindigkeit}
Wir untersuchen diese Gleichung noch für den Fall der bereits 
früher untersuchten speziellen Lösung 
\[
u(t,x)=\frac{A-x}{B+t}
\]
explizit aufschreiben. Dazu benötigen wir die partielle Ableitung
nach $x$, die wir ebenfalls bereits früher berechnet haben:
\begin{align*}
\partial_x u(t,x)&=-\frac{1}{B+t}
\end{align*}
Die linearisierte Gleichung wird damit zu
\begin{align*}
-\partial_x^2v(t,x)
+\partial_tv(t,x)
+ \frac{A-x}{B+t}\partial_xv(t,x)
-\frac{1}{B+t} v(t,x)
&=0
\\
-(B+t)\partial_x^2v(t,x)
+(B+t)\partial_tv(t,x)
+ (A-x)\partial_xv(t,x)
- v(t,x)
&=0
\end{align*}
Leider lässt sich in diesem Fall nicht so offensichtlich eine Lösung finden.

