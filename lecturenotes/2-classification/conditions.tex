%
% conditions.tex -- XXX
%
% (c) 2019 Prof Dr Andreas Mueller
%
\section{Boundary conditions\label{klassifikation:randbedingungen}}
\rhead{Boundary conditions}
As with ordinary differential equations it is not enough to 
specify the partial differential equation to determine the
solution.
This section shows how the concept of initial conditions and boundary
conditions needs to be extended for partial differential equations.

\subsection{Initial and boundary conditions for ordinary differential equations\label{klassifkation:anfangswerte-ode}}
For ordinary differential equations, we have to specify initial or boundary
conditions.
To fix the differential equation
\[
y''+p(x)y'+q(x)y=0
\]
on the interval
$[0,1]$,
we need two additional conditions, e.~g.
\begin{itemize}
\item the values $y(0)$ and $y'(0)$ (initial conditions)
\item the values $y(0)$ und $y(1)$ (boundary conditions)
\item two linear equations combining these
\begin{align*}
\alpha_0y(0)+\beta_0y'(0)&=\gamma_0\\
\alpha_1y(1)+\beta_1y'(1)&=\gamma_1.
\end{align*}
%wobei die Koeffizientenmatrix auf der linken Seite regulär sein muss.
\end{itemize}

\begin{beispiel}
Even for an linear ordinary differential equation of second order
not every combiation of boundary values will allow a solution.
The equation
\[
y''=0
\]
means that the solution curve does not have curvature.
Consequently the solution is a straight line with equation $y=Ax+B$.
In particular, the slope must be equal at both ends of the interval.
The boundary values
\[
y'(0)=0\qquad y'(1)=1,
\]
thus lead to a problem that has no solution.
\end{beispiel}

\subsection{Boundary values for partial differential equations\label{klassifikation:randwerte-pde}}
Specifying the boundary values for partial differential equations
becomes much more complicated.
To develope some intuition for this task, let's look again at the
vibrating string.
\begin{figure}
\begin{center}
\includegraphics{../common/images/randwerte-1.pdf}
\end{center}
\caption{boundary values for the wave equation for a vibrating string.
The domain is colored light red, the boundary is red, the boundary
values are specified along the boundary.
\label{klassifikation:randwertesaite}}
\end{figure}
In figure \ref{klassifikation:randwertesaite}
we have drawn the domain in light red.
Boundary values can be given on the left, right and bottom portions
of the boundary.

The theory of ordinary differential equations teaches that a 
differential equation of second order needs precisly two independent
boundary values.
We can either give values at each end of an interval or the value and the
first derivative at one end.

We do the same for the wave equation.
If we forget the $t$ dependence for a moment, we are reduced to a
differential equation of second order in $x$ on the interval $[0,l]$,
so we expect that we can give boundary values at each end
as in
\[
u(0,t)=0\qquad u(l,t)=0.
\]
If we disregard the $x$-dependence, we are left with a differential equation
in $t$ for $t>0$, i.~e.~the only choice for boundary conditions is to
specify the value and the first derivative at $t=0$.

In principle there are multiple first derivatives.
However, if we already know the values of $u(x,0)=f(x)$ along the bottom
boundary, we
also know the values of the first derivative with respect to $x$, it is
\[
\frac{\partial }{\partial x}u(x,0) = f'(x)
\]
We no longer have the option to specify this derivative.
So the only derivative is the one with respect to $t$, or in the direction
orthogonal to the boundary.

It turns out that these boundary conditions do in fact uniquely
determine the solution.
Unfortunately this relatively straight forward discussion becomes
much more complicated for general domains.

\subsubsection{General Discussion}
Right from the start we don't even know on which part of the boundary
we should try to specify boundary conditions.
In principle we can use any part of the boundary.

In addition, we have to decide whether to involve derivatives of the
unknown function, which we have already seen cannot be specified
arbitrarily but are subject to certain restrictions, that we would like
to highlight with the following example.
Consider the half plane
\[
\Omega=\{(x,y)\in\mathbb R^2\,|\,x>0\}.
\]
Let $u$ be a function defined on $\bar\Omega$.
If the values of $u$ are given along the boundary $\partial\Omega$
then we also know the derivatives in direction tangential to the boundary:
\[
\text{
$u(0,y)$ known
}\qquad\Rightarrow\qquad
\text{$\frac{\partial u}{\partial y}$ known.}
\]
This we can only reasonably expect to specify derivatives orthogonally
to the boundary.

Consider now the general case: the function $u$ ist defined in a point 
$x=(x_1,\dots,x_k)\in\mathbb R^k$ of the boundary $\partial\Omega$.
We assume that $\partial \Omega$ has a well defined tangential plane
in $x$.
If we know the function values along the boundary, then we know all the
partial derivatives of $u$ in directions tangent to the boundary.
They are computed using the directional derivative
\[
D_{\vec v}u(x_1,\dots,x_k)=\vec v\cdot\operatorname{grad}u
\]
for vectors $v$ in the tangent plane.
The only derivative not known yet is the derivative in the direction
orthogonal to the tangent plane.
If we call $\vec{n}$ the normal vector of the tangent plane, then the
derivative in this direction is
\[
\frac{\partial u}{\partial n}
=
\frac{\partial u}{\partial \vec{n}}
=
\vec{n}\cdot \operatorname{grad}u .
\]
This is called the normal derivative of $u$ in the point $x$.

\begin{beispiel}
Let $\Omega=\{(x,y)\in\mathbb R^2\,|\,x^2+y^2<1\}$ be the unit disk.
Find the normal derivative of the function
$u(x,y)=x^3y$
in each point on the unit circle.

The outside normal in the point $(x,y)$ is
\[
\vec n=\begin{pmatrix}x\\y\end{pmatrix}.
\]
The derivative of a function $u(x,y)$
in this direction then is
\[
\frac{\partial u}{\partial n}=
x\frac{\partial u}{\partial x}
+
y\frac{\partial u}{\partial y}.
\]
For the function given in the problem $u(x,y)=x^3y$:
\[
\frac{\partial u}{\partial n}
=
x\cdot 3x^2y+y\cdot x^3=
4x^3y.
\qedhere
\]
\end{beispiel}

\begin{beispiel}
Find the normal derivative of the function
$x^2-y^2$
on the boundary of the domain
$\Omega=\{(x,y)\in\mathbb R^2\,|\,y+x>0\}$.

The domain has the straight line
$y=-x$ as boundary, which has the vector
\[
\vec n=\begin{pmatrix}1\\1\end{pmatrix}
\]
as normal.
The directional derivative of $u$ in this direction in direction $\vec{n}$
in the point $(x,y)$ is
\[
D_{\vec n}u(x,y)=\vec n\cdot\operatorname{grad}u
=\begin{pmatrix}1\\1\end{pmatrix}\cdot\begin{pmatrix}2x\\-2y\end{pmatrix}
=2x-2y
\]
In the point $(x,-x)$ on the boundary the normal derivative becomes
\[
\frac{\partial u}{\partial n}
=
2x+2x=4x.
\qedhere
\]
\end{beispiel}

\subsection{Particular Cases\label{klassifikation:randwerte-speziell}}
\subsubsection{The Cauchy problem\label{klassifikation:cauchy-problem}}
The boundary of a domain usually consists of many more than just a few
points, usualy of curves and surfaces.
To better understand the problem of specifying boundary conditions
we look more closely at the case of a function $u(x,y)$ of two independent
variables.
The solution function $u(x,y)$ can be visualized as a surface over the
$x$-$y$-plane also called the graph of $u$.

The boundary of a domain in $\mathbb R^2$ ist a curve.
If we prescribe values of the function $u(x,y)$ on this curve, we
essentially prescribe a curve contained in the graph of $u$.
This is called the {\em Cauchy problem}: given a curve in $\mathbb R^3$, 
find a function $u$ such that the graph of $u$ is a surface containing
the curve.

\subsubsection{Partial differential equations of first order}
An ordinary differential equation of first order only needs a single
initial value.
By analogy, we expect that should be able to solve a partial differential
equation of first order on the domain $\{x>0\}$ with only boundary values
on the $y$-axis.
Thus we assume that 
\[
u(0,y)=g(y)
\]
uniquely determines the solution.

We assume that $g$ is continuously differentiable.
Then
\[
\frac{\partial u}{\partial y}(0, y)=g'(y),
\]
the partial derivative with respect to $y$ along the $y$-axis is already
fixed.

A partial differential equation of first order defines a relation
of the form
\[
F(x,y,u(x,y), \partial_x u, \partial_y u)=0
\]
between function and partial derivatives.
For not too complicated functions $F(x,y,u,p,q)$ we can solve for
$p$.

%
% XXX
% XXX
%

nach $p$ auflösbar sein, und zwar sogar auf dem Rand, also für
$x\to 0$. Gibt es eine Lösung $u(x,y)$ des Cauchy-Problems, dann
ist 
\begin{align*}
0&=
F(x,y,u(x,y), \partial_x u(x,y), \partial_y u(x,y)
\\
\Rightarrow
0&=
\lim_{x\to 0}
F(x,y,u(x,y), \partial_x u(x,y), \partial_y u(x,y)\\
&=F(0,y,u(x,y),\partial_x u(0,y), \partial_y u(0,y))
\\
&=
F(0,y,g(y),\partial_x u(0,y), g'(y))
\end{align*}
Da wir von der Funktion $F$ angenommen haben, dass sie sich
nach $p$ auflösen lässt, ist also $\partial_x u(0,y)$ durch
die Anfangsbedingungen und die Gleichung ebenfalls festgelegt.

Das Cauchy-Problem für eine Differentialgleichung erster Ordnung
muss als nur Anfangswerte entlang einer Geraden vorgeben, Ableitungen
sind nicht nötig, da sie durch die Funktionswerte und die Gleichung
bereits festgelegt sind.

Statt der Werte entlang der $y$-Achse könnten wir auch nur den
Wert in einem Punkt sowie die Ableitungen
in diese Richtung vorgeben, zum Beispiel in der Form
\begin{align*}
u(0,0)&=u_0\\
\frac{\partial u}{\partial y}(0,y)&=h(y).
\end{align*}
Dies liefert jedoch nichts neues, denn die Funktion $g(y)=u(0,y)$
wird durch die gewöhnliche Differentialgleichung
\[
g'(y)=\frac{\partial u}{\partial y}(0,y)=h(y)
\]
für die unbekannte Funktion $g(y)$
mit der Anfangsbedinung $g(0)=u_0$ bestimmt.
Indem man diese Gleichung löst, führt man Anfangsbedingungen
mit Ableitungen entlang der $y$-Achse auf gewöhnliche
Anfangswerte zurück.

\subsubsection{Partielle Differentialgleichungen zweiter Ordnung}
Für eine gewöhnliche Differentialgleichung zweiter Ordnung müssen
zusätzliche Anfangsbedingungen spezifiziert werden, typischerweise
die erste Ableitung.

Für eine partielle Differentialgleichung zweiter
Ordnung wird man entsprechend zu den Anfangswerten $g(y)$ die Werte
der ersten Ableitungen hinzunehmen wollen.
Wie wir im vorangegangenen Abschnitt gesehen haben, sind die partiellen
Ableitungen in $y$-Richtung durch die Anfangsbedingung $g(y)$ bereits
festgelegt. Nur die erste Ableitung in $x$-Richtung kann noch
vorgegeben werden. Die Anfangsbedingung kann also so formuliert
werden:
\begin{align}
&\text{Dirichlet-Randbedingung:}&
u(0,y)&=g(y)&
\label{klassifikation:dirichlet-randbedingung}
\\
&\text{Neumann-Randbedingung:}&
\frac{\partial u}{\partial x}(0,y)&=h(y)
\label{klassifikation:neumann-randbedingung}
\end{align}

\subsubsection{Beliebige Anfangskurve}
Im allgemeinen Fall sind die Randwerte nicht auf einer Koordinatenachse
vorgegeben. Es gibt sogar Fälle, wo dies gar nicht möglich ist,
ein Beispiel wird in \ref{unloesbar} gezeigt.
Die Festlegung von Werten entlang der der $y$-Achse entspricht
der Forderung, dass die Lösungsfläche durche eine bestimmte Kurve
verlaufen muss, die als Parameterdarstellung $y\mapsto (0,y,g(y))$
hat. Die Verallgemeinerung das allgemeine Cauchy-Problem:

\begin{problem}[Cauchy-Problem] Sei $\gamma$ eine Kurve im Raum und
$F(x,y,u,\partial_xu,\partial_yu)=0$ eine PDGL. Eine Funktion $u(x,y)$
heisst eine Lösung des Cauchy-Problems mit Anfangskurve $\gamma$, wenn
$\gamma$ im Graphen von $u$ enthalten ist.
\end{problem}

Durch die Wahl eines geeigneten Koordinatensystems kann aber immer
erreicht werden, dass in der Umgebung eines einzelnen Randpunktes
dieses Situation vorliegt.
Die Richtung der $x$-Achse ist dann die Normale auf die Randkurve oder
Randfläche.
Die Ableitung entlang der $x$-Achse ist daher eigentlich eine Ableitung
in Normalenrichtung, wir schreiben dafür
\[
\frac{\partial u}{\partial n}
\]
und nennen sie die Normalableitung.
Bei der eben beschriebenen Wahl des Koordinatensystems kann man
die Normalableitung mit Hilfe der Formel
\[
\frac{\partial u}{\partial n}
=\frac{\partial u}{\partial x}
\]
berechnen.

Ist $n$ der Normalenvektor auf die Kurve (oder Fläche für $n=3$), auf
der die Randwerte vorgegeben sind, dann kann die Normalableitung mit 
Hilfe der Richtungsableitung berechnet werden, es ist
\[
\frac{\partial u}{\partial n}=D_nu = n\cdot \operatorname{grad} u.
\]

\subsection{Allgemeines Randwertproblem\label{klassifikation:allgemeines-randwertproblem}}
Bei partiellen Differentialgleichungen hat der Definitionsbereich einen
weit grösseren Einfluss auf die Lösung als bei gewöhnlichen
Differentialgleichungen. Das in den voranstehenden Abschnitten
diskutierte Cauchy-Problem erlaubt zunächst zu beurteilen, welche
Art von Randbedingungen sinnvoll ist. 
So haben Wir gesehen, dass Randwerte in folgenden Formen vorgegeben
werden können: 
\begin{itemize}
\item Als Werte entlang des Randes des Definitionsgebietes der
Differentialgleichung, also in der Form
\[
u(x)=g(x)\quad \forall x\in\partial G\subset \mathbb R^n.
\]
Diese Randbedingungen heissen Dirichlet-Randbedingungen.
\item Als Werte der Normalableitung auf dem
Rand, geschrieben
\[
\frac{\partial u}{\partial n}(x)=h(x)\quad\forall x\in\partial G\subset \mathbb R^n.
\]
Dieser Randbedingungen heissen Neumann-Randbedingungen.
\item Als Kombination von Funktionswerten und Ableitungen: es ist
möglich, auf verschiedenen Teilen des Randes Randwerte oder Normalableitungen
vorzugeben, es ist aber auch möglich, eine Linearkombination
von Randwerten und Normalableitungen vorzugeben.
\end{itemize}
Dies genügt jedoch noch nicht, zu entscheiden, ob durch diese
Bedingungen die Lösungen eindeutig festgelegt sind.
Dazu ist ein vertieftere Theorie nötig. Während bei den gewöhnlichen
Differentialgleichungen eine relativ einfache Methode (Picard-Iteration)
unter relativ milden Voraussetzungen zu beweisen  erlaubt, dass
gewöhnliche Differentialgleichungen zu gegebenen Anfangswerten eine
eindeutig bestimmte Lösung haben, wenigstens für eine gewisses Zeitinterval,
ist dieser Beweis für partielle Differentialgleichungen ungleich schwieriger.
Für gewisse Typen von partiellen Differentialgleichungen ist dies jedoch
möglich, einige davon werden in den kommenden Kapiteln behandelt.

