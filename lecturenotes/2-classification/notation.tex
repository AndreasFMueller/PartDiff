%
% klassifikation.tex
% 
%
% (c) 2008 Prof Dr Andreas Mueller
%
\section{Appendix: Notation}
\rhead{Notation}
The traditional way to write derivatives isn't very compact, which is why
we will use the following equivalent notation
\begin{align*}
\frac{\partial f(x,y)}{\partial x}
&=\frac{\partial}{\partial x}f(x,y)
=\partial_x f(x,y)\\
\frac{\partial f}{\partial x_i}(x,y)
&=\partial_{x_i}f(x,y)=\partial_if(x,y)
\end{align*}
We will also use the following operator notation for derivatives:
\begin{align*}
\frac{\partial}{\partial x}
&=
\partial_x\\
\frac{\partial}{\partial x_i}
&=
\partial_{x_i}
=\partial_i
\end{align*}
In this notation, the well known opeators can be written as:
\begin{align*}
\Delta &=\partial_x^2+\partial_y^2+\dots=\sum_{i=0}^n\partial_i^2\\
\operatorname{grad}&=\nabla=\begin{pmatrix}\partial_1\\\vdots\\\partial_n\end{pmatrix}
\end{align*}
The normal derivative which is used in boundary conditions for partial
differential equations is written as
\[
\frac{\partial u}{\partial n}=\partial_nu.
\]

