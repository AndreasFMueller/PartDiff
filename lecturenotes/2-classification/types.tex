%
% types.tex
%
% (c) 2008 Prof Dr Andreas Mueller
%
\section{Spezielle Typen von partiellen Differentialgleichungen\label{klassifikation:spezielletypen}}
\rhead{Spezielle PDGL}
Für einige spezielle Typen von Differentialgleichungen werden wir
die Frage nach Existenz und Eindeutigkeit einer Lösung beantworten
können, und manchmal sogar eine explizite Lösung angeben können.
Dazu gehören die linearen und die quasilinearen partiellen
Differentialgleichungen, die in diesem Abschnitt beschrieben werden.

\subsection{Lineare partielle Differentialgleichungen\label{klassifikation:linear}}
Die bisher vorgestellten Differentialgleichungen sind also alle lineare
Ausdrücke in den ersten und zweiten Ableitungen, man könnte sie in der
Form
\begin{align*}
\sum_{i,j=1}^n a_{ij}(x)\frac{\partial}{\partial x_i} \frac{\partial}{\partial x_j}\psi(x)
+\sum_{i=1}^na_i(x)\frac{\partial}{\partial x_i}\psi(x)&=f(x)
\\
\sum_{i,j=1}^n a_{ij}(x)\partial_i \partial_j\psi(x)
+\sum_{i=1}^na_i(x)\partial_i\psi(x)&=f(x)
\end{align*}
Die Gleichung heisst homogen, wenn $f=0$ ist.

Damit diese Probleme überhaupt eine Lösung haben, müssen noch 
Randbedingungen hinzugefügt werden.
Auch diese lassen sich in der Form linearer Gleichungen zwischen den 
Funktionswerten und den Normalableitungen von $\psi$ auf dem Rand des
Gebietes gegeben:
\begin{align*}
a\psi(x)+
b\frac{\partial}{\partial n}\psi(x)
&=g(x)\quad\forall x\in\gamma\\
a\psi(x)+b\partial_n\psi(x)&=g(x)\quad\forall x\in\gamma
\end{align*}
Die Randbedingungen heissen homogen, wenn $g=0$ ist.

Sind $\psi_1$ und $\psi_2$ zwei Lösungen der homogenen Gleichungen und der
homogenen Randbedingungen, dann ist auch $\alpha_1\psi_1+\alpha_2\psi_2$
Lösungen der homogenen Gleichung under homogenen Randbedingungen:
\begin{align*}
&\sum_{i,j=1}^n a_{ij}(x)\partial_i \partial_j
(\alpha_1\psi_1(x)+\alpha_2\psi_2(x))
+\sum_{i=1}^na_i(x)\partial_i
(\alpha_1\psi_1(x)+\alpha_2\psi_2(x))
\\
+
\alpha_1
&\sum_{i,j=1}^n a_{ij}(x)\partial_i \partial_j
\psi_1(x)
+
\alpha_1
\sum_{i=1}^na_i(x)\partial_i
\psi_1(x)
\\
+
\alpha_2
&\sum_{i,j=1}^n a_{ij}(x)\partial_i \partial_j
\psi_2(x)
+
\alpha_2
\sum_{i=1}^na_i(x)\partial_i
\psi_2(x)
=0
\end{align*}
oder für die Randbedingung
\begin{align*}
a(\alpha_1\psi_1(x) +\alpha_2\psi_2(x))
\;+&b\partial_n
(\alpha_1\psi_1(x) +\alpha_2\psi_2(x))\\
=\alpha_1(a\psi_1(x)
\;+&b\partial_n
\psi_1(x))\\
+\;\alpha_2(a\psi_2(x)
\;+&b\partial_n
\psi_2(x))
=0\quad\forall x\in\gamma
\end{align*}
Die Lösungen einer homogenen partiellen Differentialgleichung
bilden also einen Vektorraum. Insbesondere lassen sich beliebige
Lösungen der inhomogenen Differentialgleichung dadurch finden, dass
man eine partikuläre Lösung $\psi_p$ der homogenen Differentialgleichung
findet, und dazu eine beliebige Lösung der homogenen Differentialgleichung
$\psi_h$ hinzuaddiert.

\subsection{Quasilineare partielle Differentialgleichungen erster Ordnung\label{klassifikation:quasilinear}}
Einen interessanten Spezialfall bilden die quasilinearen PDGL erster
Ordnung. Sie sind nicht unbedingt linear, aber die partiellen
Ableitungen erster Ordnung kommen nur linear vor. Die Funktion
$
F(x,y,u,p,q)
$
ist also linear in $p$ und $q$. Die Variablen $x$ und $y$ sowie die
gesuchte Funktion können daher nur in den Koeffizienten der
Variablen $p$ und $q$ vorkommen, $F$ muss von der Form
\[
F(x,y,u,p,q)=a(x,y,u)p+b(x,y,u)q+c(x,y,u)
\]
sein. Eine quasilineare Differentialgleichung in zwei Variablen
hat also die Form
\[
a(x,y,u(x,y))\frac{\partial u}{\partial x}+b(x,y,u(x,y))\frac{\partial u}{\partial y}
=c(x,y,u(x,y)).
\]
Für mehr Variablen $x_1,\dots,x_n$ gilt analog, dass eine quasilineare
Differentialgleichung die Form
\[
a_1(x_1,\dots,x_n,u)\frac{\partial u}{\partial x_1}
+
a_2(x_1,\dots,x_n,u)\frac{\partial u}{\partial x_2}
+\dots
+
a_n(x_1,\dots,x_n,u)\frac{\partial u}{\partial x_n}
=c(x_1,\dots,x_n,u)
\]
hat.

Natürlich lassen sich auch quasilineare partielle Differentialgleichungen
höherer Ordnung definieren, in einer solchen Differentialgleichung
kommen zwar höheren partielle Ableitungen vor, aber immer nur linear,
man kann sie also immer in der Form schreiben:
\[
\sum_{\bf k} a_{\bf k}(x,y,u)\partial_{\bf k} u(x,y) = 0,
\]
worin die Funktionen $a_{\bf k}(x,y,u)$ nur für endlich viele
Multiindizes ${\bf k}$ von $0$ verschieden sind.

\subsection{Nichtlineare Gleichungen\label{klassifikation:nichtlinear}}
Die bisher vorgestellten Beispiele von partiellen Differentialgleichungen
sind alle linear. Viele Gleichungen der Physik sind jedoch
nicht linear.  Berühmtestes Beispiel sind die Gleichungen, die
die Strömung eines Gases beschreiben. Bereits Leonhard Euler hat 
für die Strömung eines idealen Gases ein System von partiellen
\index{Dichte}
\index{Druck}
Differentialgleichungen gefunden für die Dichte $\varrho$, den Druck
$p$ und die Strömungsgeschwindigkeit $\vec v$, alle drei sind Funktionen
\index{Stromungsgeschwindigkeit@Str\ömungsgeschwindigkeit}
von allen drei Raumkoordinaten und der Zeit. Die wichtigste davon
ist die Eulersche Gleichung:
\index{Eulersche Gleichung}
\begin{align*}
\frac{\partial \vec v}{\partial t}
+(\vec v\cdot \nabla)\vec v
=-\frac1{\varrho}\operatorname{grad}p
\\
\frac{\partial v_i}{\partial t}
+\sum_{j=1}^3v_j\frac{\partial v_i}{\partial x_j}
=
-\frac1{\varrho}\frac{\partial p}{\partial x_i}
\end{align*}
Diese Gleichung kann nicht linear sein, weil im zweiten Term
Produkte von $v_j$ mit Ableitungen von $v_i$ vorkommen.
Berücksichtigt man auch noch die Zähigkeit, wird die Gleichung
noch komplizierter:
\[
\varrho\left(
\frac{\partial\vec v}{\partial t}
+
(\vec v\cdot\nabla)\vec v
\right)
=
-\operatorname{grad}p+\eta\Delta \vec v+\left(\zeta+\frac{\eta}3\right)
\operatorname{grad}\operatorname{div}\vec v
\]

In einer Dimension bleibt von dieser Gleichung nur noch eine Komponente
$u(t,x)$ übrig, für die eine Gleichung der ungefähren Form
\[
\frac{\partial u}{\partial t}+u\frac{\partial u}{\partial x}
=\eta\frac{\partial^2u}{\partial x^2}
\]
gelten muss (wir haben den Druckgradienten vernachlässigt und
$\varrho = 1$ angenommen). Diese Gleichung wurde von
Johannes Martinus Burgers ausgiebig studiert, und heisst
daher Gleichung von Burgers.
Eine Lösung des Anfangswertproblems der Gleichung von Burgers kann 
in expliziter Form gefunden werden, wir beschreiben diese in Kapitel
\ref{chapter-nichtlinear}.


