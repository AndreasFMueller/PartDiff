%
% types.tex --
%
% (c) 2019 Prof Dr Andreas Mueller
%
\section{Special types of partial differential equations
\label{klassifikation:spezielletypen}}
\rhead{Special PDE}
For a couple of special types of partial differential equations we will
be able to answer the question regarding existence and uniqueness of 
solutions.
In some cases we will even be able to give explicit formulas for the solutions.
Linear and quasilinear partial differential equations are part of this
very limited set of equations.
This section is devoted to describing them.

\subsection{Linear partial differential equations\label{klassifikation:linear}}
The differential equations proposed so far were all linear expressions in
the first and second derivatives.
In the most general form, we can write them as
Form
\begin{align*}
\sum_{i,j=1}^n a_{ij}(x)\frac{\partial}{\partial x_i} \frac{\partial}{\partial x_j}\psi(x)
+\sum_{i=1}^na_i(x)\frac{\partial}{\partial x_i}\psi(x)&=f(x)
\\
\sum_{i,j=1}^n a_{ij}(x)\partial_i \partial_j\psi(x)
+\sum_{i=1}^na_i(x)\partial_i\psi(x)&=f(x).
\end{align*}
We call such an equation {\em homogeneous} if $f=0$.
To every linear partial differential equation of this form we can construct
the associated homogeneous equation by setting $f=0$.

In order for this problem to have a solution, we have to add boundary
conditions.
As we have learned, these too can be written as linear combinations
of function values and normal derivatives of $\psi$ on the boundary
of the domain:
\begin{align*}
a\psi(x)+
b\frac{\partial}{\partial n}\psi(x)
&=g(x)\quad\forall x\in\gamma\\
a\psi(x)+b\partial_n\psi(x)&=g(x)\quad\forall x\in\gamma
\end{align*}
The boundary conditions are called {\em homogeneous} if $g=0$.
To every set of boundary conditions we can construct the associated
set of homogeneour boundary conditions by setting $g=0$.

If $\psi_1$ and $\psi_2$ are two solutions of a homogeneous linear partial
differential equation with homogeneous boundary conditions, then
a linear combination $\alpha_1\psi_1+\alpha_2\psi_2$ is a solution
as well:
\begin{align*}
&\sum_{i,j=1}^n a_{ij}(x)\partial_i \partial_j
(\alpha_1\psi_1(x)+\alpha_2\psi_2(x))
+\sum_{i=1}^na_i(x)\partial_i
(\alpha_1\psi_1(x)+\alpha_2\psi_2(x))
\\
+
\alpha_1
&\sum_{i,j=1}^n a_{ij}(x)\partial_i \partial_j
\psi_1(x)
+
\alpha_1
\sum_{i=1}^na_i(x)\partial_i
\psi_1(x)
\\
+
\alpha_2
&\sum_{i,j=1}^n a_{ij}(x)\partial_i \partial_j
\psi_2(x)
+
\alpha_2
\sum_{i=1}^na_i(x)\partial_i
\psi_2(x)
=0
\end{align*}
oder für die Randbedingung
\begin{align*}
a(\alpha_1\psi_1(x) +\alpha_2\psi_2(x))
\;+&b\partial_n
(\alpha_1\psi_1(x) +\alpha_2\psi_2(x))\\
=\alpha_1(a\psi_1(x)
\;+&b\partial_n
\psi_1(x))\\
+\;\alpha_2(a\psi_2(x)
\;+&b\partial_n
\psi_2(x))
=0\quad\forall x\in\gamma
\end{align*}
This means that the solutions of a homogeneous linear
partial differential equation with homogeneous boundary conditions
form a vector space.
The well known properties of solutions of linear systems of equations
apply to the solution space as well.
In particular, to find any solution of the inhomogeneous partial
differential equation we simply have to produce a single solution
$\psi_p$ called a particular solution of the inhomogeneous equation
and add solutions of the homogeneous equation to it.
Knowledge of the solution space of the homogeneous equation transfers
to knowledge of the structure of the solution set of any inhomogeneous
version of the equation.

\subsection{Quasilinear partial differential equations of first order
\label{klassifikation:quasilinear}}
A very interesting special case are the quasilinear partial differential
equations of first order.
They are not necessarily linear, but the first order derivatives appear only
linearly.
The function
$
F(x,y,u,p,q)
$
is therefore linear in $p$ und $q$.
The variables $x$ and $y$ and the placeholder $u$ for the unknown
function can only appear in coefficients of the variables $p$ and $q$.
Thus $F$ must have the form
\[
F(x,y,u,p,q)=a(x,y,u)p+b(x,y,u)q-c(x,y,u).
\]
The negative sign in front of $c$ is chosen for consistency with the
theory to be developed in the next chapter.
A quasilinear partial differential equation of first order therefore
has the form
\[
a(x,y,u(x,y))\frac{\partial u}{\partial x}+b(x,y,u(x,y))\frac{\partial u}{\partial y}
=c(x,y,u(x,y)).
\]
For more than $2$ variables, we can similarly conclude that equation must
be of the form
\[
a_1(x_1,\dots,x_n,u)\frac{\partial u}{\partial x_1}
+
a_2(x_1,\dots,x_n,u)\frac{\partial u}{\partial x_2}
+\dots
+
a_n(x_1,\dots,x_n,u)\frac{\partial u}{\partial x_n}
=c(x_1,\dots,x_n,u).
\]

Quasilinear partial differential equations of higher order can also be
defined.
In such an equation, higher derivatives may show up, but they can only
show up linearly.
This means that the equation can be written in the form
\[
\sum_{\bf k} a_{\bf k}(x,y,u)\partial_{\bf k} u(x,y) = 0,
\]
where the functions $a_{\bf k}(x,y,u)$ are nonzero for only finitely
many multiindices ${\bf k}$.

\subsection{Nonlinear equations
\label{klassifikation:nichtlinear}}
The partial differential equations from applications presented so far
were linear.
However, many problems from physics are not linear.
Most notably the equations that govern the flow of a fluid.
Already Leonhard Euler has formulated a system of partial differential
equations for the flow of an incompressible gas.
It connects density $\varrho$, pressure $p$ and the velocity vector
$\vec{v}$, all three being functions of position and time, with the
each other and their derivatives at every point in space.
The most important of these equations is Euler's equation:
\index{Euler's equation}
\begin{align*}
\frac{\partial \vec v}{\partial t}
+(\vec v\cdot \nabla)\vec v
=-\frac1{\varrho}\operatorname{grad}p
\\
\frac{\partial v_i}{\partial t}
+\sum_{j=1}^3v_j\frac{\partial v_i}{\partial x_j}
=
-\frac1{\varrho}\frac{\partial p}{\partial x_i}.
\end{align*}
This equation cannot be linear, products of $v_j$ and derivatives of $v_i$
appear in the second term.

Adding viscosity makes the equations even more complicated:
\[
\varrho\left(
\frac{\partial\vec v}{\partial t}
+
(\vec v\cdot\nabla)\vec v
\right)
=
-\operatorname{grad}p+\eta\Delta \vec v+\left(\zeta+\frac{\eta}3\right)
\operatorname{grad}\operatorname{div}\vec v
\]

In one dimension, there is not much left, just a single velocity component
$u(t,x)$, for which the equation essentially has the Form
\[
\frac{\partial u}{\partial t}+u\frac{\partial u}{\partial x}
=\eta\frac{\partial^2u}{\partial x^2}
\]
(we have neglected the pressure gradient and assumed $\varrho=1$).
This equation has been studied extensively by Johannes Martinus Burgers
and is thus called Burgers' equation.
It exhibits many common features of nonlinear partial differential equations.
Solutions can be found in explicit form, as we will indicate in the next
chapter.
