%
% solutions.tex
%
% (c) 2019 Prof Dr Andreas Mueller
%
\section{Solutions of partial differential equations\label{klassifikation:loesung}}
To solve a partial differential equation, the following data has to be
specified:
\begin{enumerate}
\item
A differential equation, e.~g.~by specifying the function $F$
\item
A domain $\Omega$
\item
Boundary values on part of the boundary $\partial\Omega$.
\end{enumerate}
\begin{definition}
A solution of a partial differential equation is a function $u$
defined on $\bar \Omega$ (not only on $\Omega$!), which is
differentiable in $\Omega$, satisfies the differential equation in
$\Omega$, and satisfies the boundary conditions on $\partial\Omega$.
\end{definition}

A problem is called {\em well posed} if the data determines the solution
uniquely.
The theory to be developed in subsequent chapters has to be able to
determine whether a problem is well posed.
Only then does it make sense to try to use a numerical algorithm
and to expect a well behaved solution.

%
% XXX weak solution
%
