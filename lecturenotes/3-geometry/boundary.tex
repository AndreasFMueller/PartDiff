%
% boundry.tex  -- XXX
%
% (c) 2019 Prof Dr Andreas Mueller
%
\section{Randbedingungen}
Die Methode der Charakterisitiken erlaubt uns herauszufinden, auf welchen
Teilen des Randes die Werte einer Lösunge einer partiellen
Differentialgleichung vorgegeben werden müssen.
Die Lösungsfläche $u(x,y)$ besteht ja aus Charatkeristiken, sie ist also
genau dann vollständig bestimmt, wenn jede Charakteristik durch mindestens 
einen vorgegebenen Randpunkt verläuft.

Für die Differentialgleichung
\begin{equation}
\frac{\partial u}{\partial x}+2\frac{\partial u}{\partial y}=3
\label{geometrie:knickbeispiel}
\end{equation}
hatten wir die Charakterisiken 
\[
t\mapsto\begin{pmatrix}x_0\\y_0\\z_0\end{pmatrix}+t\begin{pmatrix}1\\2\\3\end{pmatrix}
\]
gefunden. Eine Lösungsfunktion $u(x,y)$ muss von Charakteristiken
überdeckt werden. 
Die Projektionen dieser Kurven in die $x$-$y$-Ebene sind Geraden
mit Steigung $2$. Der Rand des Gebietes $\Omega$, in dem die Gleichung gelöst
werden soll, muss also jede Gerade mit Steigung $2$ höchstens einmal
schneiden.

Das Gebiet $\{(x,y)\in\mathbb R^2\,|\, x >0\}$  aus \ref{konstantekoeff}
hat die $x$-Achse als Rand, und diese schneidet jede Gerade mit Steigung
$2$ genau einmal.

Interessanter wird die Diskussion jedoch für das Gebiet
$\Omega=\{(x,y)\,|\,0<x,y<1\}$. Die Abbildungen \ref{geometrie:charrand1}
bis \ref{geometrie:charrand3} zeigen verschiedene Möglichkeiten:
\begin{enumerate}
\item
In Abbildung~\ref{geometrie:charrand1} sind Randwerte auf dem linken
und rechten Rand vorgegeben. Offenbar reichen diese Randwerte nicht, um
die Funktionswerte überall im Gebiet eindeutig zu bestimmen, es bleibt
ein Teilgebiet, in dem die Lösungsfunktion nicht festgelegt ist.
\item
In Abbildung~\ref{geometrie:charrand2}
sind die Randwerte am oberen und unteren Rand vorgegeben.
In diesem Fall sind die Randwerte überbestimmt, eine Lösung ist
nur möglich, wenn die Werte auf der linken Hälfte des unteren
Randes mit den Werten auf der rechten Hälfte des oberen Randes
kompatibel sind.
\item
In Abbildung~\ref{geometrie:charrand3}
sind die Randewerte am linken und unteren Rand vorgegeben.
Damit ist die Funktion eindeutig bestimmt, trotzdem
muss damit noch nicht eine Lösung der Differentialgleichung
gefunden sein.
Es ist nämlich nicht unbedingt garantiert, dass die zusammengestetze
Funktion 
auf der von $(0,0)$ ausgehenden Charakteristik differenzierbar ist.
\end{enumerate}

\begin{figure}
\begin{center}
\includegraphics{../common/images/randwerte-2.pdf}
\end{center}
\caption{Randwerte am linken und rechten Rand vorgegeben: im hellgrünen
Bereich sind die Funktionswerte nicht bestimmt \label{geometrie:charrand1}}
\end{figure}

\begin{figure}
\begin{center}
\includegraphics{../common/images/randwerte-3.pdf}
\end{center}
\caption{Randwerte am unteren und oberen Rand vorgegeben: im hellgrünen
Bereich sind die Funktionswerte überbestimmt \label{geometrie:charrand2}}
\end{figure}

\begin{figure}
\begin{center}
\includegraphics{../common/images/randwerte-4.pdf}
\end{center}
\caption{Randwerte am linken und unteren Rand vorgegeben: die Funktion
ist zwar eindeutig bestimmt, aber die Differenzierbarkeit auf
der von $(0,0)$ ausgehenden Charakteristik (hellgrün) ist nicht garantiert.
\label{geometrie:charrand3}}
\end{figure}

\begin{figure}
\centering
\includegraphics[width=\hsize]{../common/3d/knick.jpg}
\caption{Lösung der Differentialgleichung~(\ref{geometrie:knickbeispiel}),
mit Randwerten entlang des linken und unteren Randes (vergleiche
Abbildung~\ref{geometrie:charrand3}) ist entlang einer Charakteristik
nicht differenzierbar (vertikale Achse mit dem Faktor $0.4$ skaliert).
\label{geometrie:knick}}
\end{figure}

Als Illustration für den letzten Fall betrachte man die Randwertvorgabe
\begin{align*}
u(x_0,0)&=0,\\
u(0,y_0)&=y_0.
\end{align*}
Jedes Randstück legt die Lösung in einem Teil des Einheitsquadrates
fest, die man durch Anwendung der Methode der Charakteristiken erhalten
kann. Die vom unteren Rand ausgehenden Charakteristiken sind
\[
\left.
\begin{aligned}
x&=x_0+t\\
y&=2t\\
u&=3t
\end{aligned}
\right\}
\qquad\Rightarrow\qquad
\left\{
\begin{aligned}
t&=\frac12y\\
u&=\frac32y
\end{aligned}
\right.
\]
Die vom linken Rand ausgehenden Charakteristiken sind 
\[
\left.
\begin{aligned}
x&=t\\
y&=y_0+2t\\
u&=y_0+3t
\end{aligned}
\right\}
\qquad\Rightarrow\qquad
\left\{
\begin{aligned}
y_0&=y-2x\\
u&=y+x
\end{aligned}
\right.
\]
Die Lösungsfunktion ist also
\[
u(x,y)=\begin{cases}
\frac32y&\qquad y<2x\\
x+y&\qquad y>2x,
\end{cases}
\]
wie in Abbildung~\ref{geometrie:knick} dargestellt.
Für $y=2x$ stimmen die beiden Terme zwar überein, die Lösungsfunktion
ist also sicher in ganz $\Omega$ stetig.
Aber die
Funktion $x\mapsto u(x,y)$ ist hat Steigung $1$ für $y>2x$ und
Steigung 0 für $y<2x$, ist also für $y=2x$ nicht differenzierbar.

