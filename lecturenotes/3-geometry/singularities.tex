%
% singularities.tex -- 
%
% (c) 2019 Prof Dr Andreas Mueller
%
\section{Development of singularities}
\index{singularity}
The solution of a partial differential equation depends crucially
on the geometry of the characteristics.
If the projections onto the $x$-$y$-plane of two characteristics 
meet, the solution can no longer be unique.
The solution is only unique if the projections of the characteristics
to the $x$-$y$-plane don't intersect.
Otherwise singularities will develop.

As an example we consider the nonlinear equation of Burgers
\index{Burgers' equation}
\begin{equation}
\frac{\partial u}{\partial t}+u\frac{\partial u}{\partial x}=0.
\label{wellenichtlinear}
\end{equation}
with initial conditions
\[
u(0,x_0)=g(x_0)=e^{-4(x_0-\frac12)^2}.
\]
According to the method of characteristics, we need solution curves for the
vector field
\[
\begin{pmatrix}
1\\
u\\
0
\end{pmatrix}
\]
through the point
$(0,x_0, z_0)$,
i.~e. a solution onf the system of differential equations
\[
\left.
\begin{aligned}
\frac{d}{d\tau} t(\tau)&=1\\
\frac{d}{d\tau} x(\tau)&=z(\tau)\\
\frac{d}{d\tau} z(\tau)&=0
\end{aligned}
\quad
\right\}
\qquad
\Rightarrow
\qquad
\left\{
\quad
\begin{aligned}
t(\tau)&=\tau\\
z(\tau)&=z(0)\\
x(\tau)&=z(0)\tau +x(0)
\end{aligned}
\right.
\]
We read off that the parameter $\tau$ 
is identical to $t$.
The solution surface thus has the parametrization
\[
(t,x_0)\mapsto
\begin{pmatrix}
t\\
g(x_0)t+x_0\\
g(x_0)
\end{pmatrix}.
\]
Projections of characteristics into the $x$-$t$-plane are straight
lines with slope depending on $g(x_0)$.
The larger $g(x_0)$, the more the integral curve veers off to the right.
Large values of the initial condition will thus overtake the smaller
values, creating jump singularities.

At time $t$, the solution has parametrization
\[
x_0\mapsto (g(x_0)t+x_0,g(x_0)).
\]
The graphs in figure~\ref{g} show how the solution develops over time.
It becomes obvious that for sufficiently large time $t$ the solution
surface can not longer be the graph of a function $u(t,x)$.

\begin{figure}
\centering
\begin{tikzpicture}[>=latex]
\node at (0,0) {%
\includegraphics[width=0.97\hsize]{../common/graphics/welle.jpg}%
};
\node at (1.1,-2.9) {$x$};
\node at (1.3,2.8) {$t$};
\node at (-7.6,2.4) [right] {$u(t,x)$};
\end{tikzpicture}
\caption{Solution of the partial differential equation
\eqref{wellenichtlinear} developing a singularity \label{g}}
\end{figure}

