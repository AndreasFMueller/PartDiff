%
% higher.tex - QLPDE in higher dimensions
%
% (c) 2021 Prof Dr Andreas Müller, OST Ostschweizer Fachhochschule
%
\section{Higher dimensions
\label{geometry:section:higher-dimensions}}
\rhead{Higher dimensions}
The very geometric principles used to solve a quasilinear
partial differential equation of first order also works in higher
dimensions.

\subsection{Differential equation in dimension $n$}
A quasilinear partial differential equation of $n$ variables $x_1,\dots,x_n$
is of the form
\begin{equation}
a_1(x_1,\dots,x_n,u) \frac{\partial u}{\partial x_1}
+
a_2(x_1,\dots,x_n,u) \frac{\partial u}{\partial x_2}
+
\dots
+
a_n(x_1,\dots,x_n,u) \frac{\partial u}{\partial x_n}
=
c(x_1,\dots,x_n,u)
\label{qlpden:eqn:equation}
\end{equation}
where the coefficients $a_1$ to $a_n$ and $c$ are functions
depending on the variables $x_1,\dots,x_n$ and the function $u$.
If they do not depend on $u$, then the equation is actually linear.

The equation~\eqref{qlpden:eqn:equation} is defined on the domain
$\Omega\subset\mathbb{R}^n$.
For brevity, we will write the points $(x_1,\dots,x_n)\in \Omega$
as just $x\in\Omega$.

In addition, we need some boundary conditions in the form
$
u(x) = f(x) 
$
for points $x$ in some part of the boundary $\partial\Omega$.
To compute the solution using the method of characteristics, we need to
parametrize the part of the boundary, an $n-1$-dimensional surface, using 
a function
\begin{equation}
s=(s_1,\dots,s_{n-1})
\mapsto
\xi(s_1,\dots,s_{n-1}) = \xi(s) \in \partial\Omega.
\end{equation}
The parameters $s=(s_1,\dots,s_{n-1})$ come from a subset
$S\subset\mathbb{R}^{h-1}$.
The boundary conditions for the function $u$ can then be written as
\begin{equation}
u(\xi(s)) = f(\xi(s))
\end{equation}
for all $s\in S$.
The boundary surface is an $n-1$-dimensional hypersurface of the
$n$-dimensional space of the variables $x_1,\dots,x_n$.

\subsection{Solution as a graph}
The solution of the partial differential equation~\eqref{qlpden:eqn:equation}
is a function  $u(x_1,\dots,x_n)$.
This function describes an $n$-dimensional hypersurface in the
$n+1$-dimensional space with
variables $(x_1,\dots,x_n,z)\in\mathbb{R}^{n+1}$ consisting of all the
points of the form $(x_1,\dots,x_n,u(x_1,\dots,x_n))$.

The boundary conditions also have a geometric description.
For points $x$ on the boundary part for which we have boundary conditions,
$u(x)=f(x)$.
This means that the $n$-dimensional surface specified by
$(x_1,\dots,x_n,u(x))$ has to pass through the points of the $n-1$-dimensional
surface of points $(x_1,\dots,x_n,f(x))$ where $x$ is in the relevant
part of the boundary.
Using the parametrization, the $n-1$-dimensional boundary surface is
parametrized by $S$ via the function
\[
S\to \mathbb{R}^n
:
s\mapsto (\xi_1(s), \dots,\xi_n(s), f(\xi(s))).
\]
This is the generalized formulation of the Cauchy-Problem.

\subsection{Characteristics}
As in the two-dimensional situation for $n=2$, the differential
equation \eqref{qlpden:eqn:equation} can be interpreted as the
vector equation
\[
\underbrace{
{\color{red}
\begin{pmatrix}
a_1(x_1,\dots,x_n,u)\\
\vdots \\
a_1(x_1,\dots,x_n,u)\\
c(x_1,\dots,x_n,u)
\end{pmatrix}
}}_{\displaystyle = \vec{v}}
\cdot
\begin{pmatrix}
\frac{\partial u}{\partial x_1}\\
\vdots \\
\frac{\partial u}{\partial x_n}\\
-1
\end{pmatrix}
=
0.
\]
The vector on the left was previously called $\vec{v}$ and was usually
shown in {\color{red}red}.
The vector on the right is again the normal vector on the $n$-dimensional
hypersurface defined by $u$.
To understand this, we have to look at the the function
$
g(x,z) = u(x) - z.
$
The graph of $u$ is the set of points such that $g(x,z) = 0$.
The normal to this surface is obtained by taking the gradient of $g$ in
this $n+1$-dimensional space, i.~e.
\begin{equation}
\vec{n}
=
\operatorname{grad} g
=
\begin{pmatrix}
\frac{\partial g}{\partial x_1}\\
\vdots \\
\frac{\partial g}{\partial x_n}\\
\frac{\partial g}{\partial z}
\end{pmatrix}
=
\begin{pmatrix}
\frac{\partial u}{\partial x_1}\\
\vdots \\
\frac{\partial u}{\partial x_n}\\
-\frac{\partial z}{\partial z}
\end{pmatrix}
=
\begin{pmatrix}
\frac{\partial u}{\partial x_1}\\
\vdots \\
\frac{\partial u}{\partial x_n}\\
-1
\end{pmatrix}.
\end{equation}
The geometric content of the partial differential equation thus is that
$\vec{v}(x_1,\dots,x_n,z)\perp \vec{n}(x_1,\dots,x_n,u)$ for all points
on the solution surface.

Following the ideas of the two-dimensional case we conclude that a
curve that always has $\vec{v}$ as a tangent vector will be contained
in the solution surface.
These curves can be parametrized using some parameter $t$ and written
as functions
\begin{equation}
x(t)
=
\begin{pmatrix}
x_1(t)\\
\vdots\\
x_n(t)
\end{pmatrix}.
\label{qlpden:charparam}
\end{equation}
They are solutions of the ordinary differential equations
\begin{equation}
\frac{d}{dt}
\begin{pmatrix}
x_1(t)\\
x_2(t)\\
\vdots\\
x_n(t)\\
z(t)
\end{pmatrix}
=
\begin{pmatrix}
a_1(x_1(t),\dots,x_n(t),z(t))\\
a_2(x_1(t),\dots,x_n(t),z(t))\\
\vdots \\
a_n(x_1(t),\dots,x_n(t),z(t))\\
c(x_1(t),\cdots,x_n(t),z(t))
\end{pmatrix}.
\label{qlpden:chareqn}
\end{equation}
The curves should all begin on the $n-1$-dimensional boundary surface,
so the initial values of the functions $x_i(t)$ and $z(t)$ für $t=0$ 
must be given by $f$ in the form
\begin{equation}
\begin{aligned}
x_1(0) &= \xi_1(s)
\\
x_2(0) &= \xi_2(s)
\\
&\phantom{i}\vdots
\\
x_n(0) &= \xi_n(s)
\\
z(0) &= f(\xi(s)),
\end{aligned}
\label{qlpden:initial}
\end{equation}
where $s\in S$ is the $n-1$-dimensional parameter that parametrizes the
boundary.

\subsection{Solution}
To solve the partial differential equation, we first have to solve
the ordinary differential equations~\eqref{qlpden:chareqn} with the
initial conditions~\eqref{qlpden:initial}.
The solution curves will depend on the parameters $s_1,\dots,s_{n-1}$ 
of the boundary, so we will get a curve parametrization $x(t,s)$
with coordinates
$x_i(t,s_1,\dots,s_{n-1})$
satisfying the differential equations
\begin{align*}
\frac{d}{dt}
x_i(t,s_1,\dots,s_{n-1})
&=
a_i(x(t,s_1,\dots,s_{n-1}), z(t))
\\
\frac{d}{dt}
z(t,s_1,\dots,s_{n-1})
&=
c(x(t,s_1,\dots,s_{n-1}),z(t))
\end{align*}
with the initial conditions
\begin{align*}
x_i(0,s_1,\dots,s_{n-1})
&=
\xi(s_1,\dots,s_{n-1})
\\
z(0,s_1,\dots,s_{n-1})
&=
f(\xi(s_1,\dots,s_{n-1}))
\end{align*}
at $t=0$.

Now we have to turn the functions $x_i$ and $z$ into a function
$u(x_1,\dots,x_n)$.
The equations
\[
\begin{aligned}
x_i &= x_i(t,s_1,\dots,s_{n-1}) &\qquad 1\le i \le n
\\
  u(x_1,\dots,x_n) &= z(t,s_1,\dots,s_{n-1})
\end{aligned}
\]
are $n+1$ equations with variables $x_i$, $t$, $s_j$ and $u$,
i.~e.~$n+1+(n-1)+1=2n+1$ variables.
By inverting the first $n$ equations and expressing the parameters
$t,s_1,\dots,s_n$ in terms of $x_1,\dots,x_n$, we can eliminate the
parameters by substiting them into the function $z$.
This gives the solution function in the form $u(x_1,\dots,x_n)$.

\subsection{Well-posedness}
The method of characteristics was also used in the two-dimensional 
case to decide whether the specified boundary conditions uniquely
determine the solution.
The problem is well posed if the characteristics that emanate from
the Cauchy initial curve cover all of the domain exactly once.
Similarly, the problem in $n$ dimensions is well posed if the
for each point $x\in\Omega$ there is precisely one boundary point
and boundary value such that the characteristic starting at this
boundary value passes over the point $x$, and the value of the
solution is the $z$-coordinate of the characteristic when it passes
over $x$.

\subsection{Examples}
\subsubsection{An equation on a a half space}
Find the solution of the equation
\[
\frac{\partial u}{\partial x_1}
+
2x_1
\frac{\partial u}{\partial x_2}
+
3
\frac{\partial u}{\partial x_3}
=
0
\]
in the half space
\[
\Omega
=
\{
(x_1,x_2,x_3)
\,|\,
x_3>0
\}
\]
with boundary values
\begin{align*}
u(x_1,x_2,0) = f(x_1,x_2)
\end{align*}
for $x_2,x_3\in\mathbb{R}$.

The differential equations for the characteristics are
\begin{align*}
\dot{x}_1 &= 1    &&\Rightarrow&  x_1(t) &= t + s_1                 \\
\dot{x}_2 &= 2x_1 &&\Rightarrow&  x_2(t) &= t^2 + 2s_1t + s_2        \\
\dot{x}_3 &= 3    &&\Rightarrow&  x_3(t) &= 3t                      \\
\dot{u}   &= 0    &&\Rightarrow&  u(t)   &= f(s_1,s_2)
\end{align*}
From these equations, we have to eliminate $t$, $s_1$ and $s_2$.
The third equation tells us that $x_3=3t$, which eliminates $t$:
\begin{align*}
x_1 &= \frac13x_3 + s_1                   \\
x_2 &= \frac1{9}x_3^2 + \frac{2s_1x_3}3 + s_2 \\
u   &= f(s_1,s_2).
\end{align*}
The first equation is equivalent to $s_1=x_1-\frac13x_3$,
which eliminates $s_1$:
\begin{align*}
x_2
&=
\frac1{9}x_3^2 + \biggl(x_1-\frac{x_3}3\biggr)\frac{2x_3}3 + s_2
=
-\frac{x_3^2}{9}
+
\frac{2x_1x_3}{3} + s_2\\
u   &= f\biggl(x_1-\frac13x_3,s_2\biggr).
\end{align*}
The equation for $x_2$ can be solved for
\[
s_2
=
x_2+\frac{1}{9}x_3^2-\frac{2x_1x_3}3
\]
and substituted into the expression for $u$ which gives the solution
\begin{equation}
u(x_1,x_2,x_3)
=
f\biggl(x_1-\frac13x_3, x_2+\frac{1}{9}x_3^2-\frac{2x_1x_3}{3}\biggr).
\label{qlpden:ex1solution}
\end{equation}
To verify this solution, we compute the partial derivatives
\begin{align*}
\frac{\partial u}{\partial x_1}
&=
\frac{\partial f}{\partial s_1}
-\frac{2x_3}{3}
\frac{\partial f}{\partial s_2}
\\
\frac{\partial u}{\partial x_2}
&=
\frac{\partial f}{\partial s_2}
\\
\frac{\partial u}{\partial x_3}
&=
-
\frac{1}{3}
\frac{\partial f}{\partial s_1}
+
\biggl(
\frac{2}{9}x_3-\frac{2x_1}{3}
\biggr)
\frac{\partial f}{\partial s_2}
\end{align*}
and substitute them into the partial differential equation
\begin{align*}
\frac{\partial u}{\partial x_1}
+
2x_1\frac{\partial u}{\partial x_2}
+
3
\frac{\partial u}{\partial x_3}
&=
\biggl(
\frac{\partial f}{\partial s_1}
-\frac{2x_3}3
\frac{\partial f}{\partial s_2}
\biggr)
+2x_1
\frac{\partial f}{\partial s_2}
+
3
\biggl(
-
\frac{1}{3}
\frac{\partial f}{\partial s_1}
+
\biggl(
\frac{2x_3}{9}
-\frac{2x_1}{3}
\biggr)
\frac{\partial f}{\partial s_2}
\biggr)
\\
&=
\biggl(
1-3\cdot\frac13
\biggr)
\frac{\partial f}{\partial s_1}
+
\biggl(
-\frac{2x_3}{3}
+
2x_1
+
3\cdot \frac{2}{9}x_3
-3\cdot\frac{2x_1}{3}
\biggr)
\frac{\partial f}{\partial s_2}
=0.
\end{align*}
So \eqref{qlpden:ex1solution} is in fact the solution of the equation.

\subsubsection{An analog to Burgers equation in three dimensions}
We want to solve the equation
\begin{equation}
u\biggl(
\frac{\partial u}{\partial x_1}
+
\frac{\partial u}{\partial x_2}
+
\frac{\partial u}{\partial x_3}
\biggr)
=
0
\end{equation}
which is clearly quasilinear and of first order.
We want tos olve it in the domain
\[
\Omega
=
\{ (x_1,x_2,x_3) \,| x_3>0\},
\]
the boundary of which is the $x_1$-$x_2$-plane which we can parametrize
using the $s_1=x_1$ and $s_2=x_2$ with with the function
\[
\xi(s)
=
\begin{pmatrix}
s_1\\
s_2\\
0
\end{pmatrix}.
\]
The boundary values are given by a function $f(x_1,x_2)$.

The system of ordinary differential equations for the characteristics is
\begin{align*}
\dot{x}_1 &= z &&\Rightarrow& x_1(t) = zt + s_1
\\
\dot{x}_2 &= z &&\Rightarrow& x_2(t) = zt + s_2
\\
\dot{x}_3 &= z &&\Rightarrow& x_3(t) = zt
\\
\dot{z} &= 0 &&\rightarrow& z = f(s_1,s_2)
\end{align*}
where we have already substituted the boundary conditions.
We now have to eliminate the variables $t$, $s_1$ and $s_2$ from the
equations
\[
\begin{aligned}
x_1 &= f(s_1,s_2)t + s_1 \\
x_2 &= f(s_1,s_2)t + s_2 \\
x_3 &= f(s_1,s_2)t       \\
u(x_1,x_2,x_3) &= f(s_1,s_2).
\end{aligned}
\]
The first three equations describe a straight line
in the domain
through the point
\[
\vec{p}
=\begin{pmatrix}
s_1\\s_2\\0
\end{pmatrix}
\]
with direction vector
\[
\vec{r}
=
f(s_1,s_2)
\begin{pmatrix} 
1\\1\\1
\end{pmatrix}.
\]
which establishes the similarity to the solution of Burgers equation
we have investigated previously.
In particular, the $3$-dimensional solution surface is covered by straight
lines
of the form
\[
\begin{pmatrix}
x_1\\
x_2\\
0\\
f(x_1,x_2)
\end{pmatrix}
+
f(x_1,x_2)\begin{pmatrix}1\\1\\1\\0\end{pmatrix}.
\]
