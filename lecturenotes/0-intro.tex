%
% a-einleitung.tex
%
% (c) 2008 Prof Dr Andreas Mueller, Hochschule Rapperswil
%
\lhead{Introduction}
\chapter*{Introduction}
\index{mechanics!classical}
\index{electrodynamics}
\index{population dynamics}
Ordinary differential equations are used in classical mechanics,
in electrodynamics, in population dynamics and more generally in any
application where the rate of change of a variable depends on the current
values of the same variable.
\index{point mass}
\index{Newton's law}
A point mass at coordinate $x$ changes its position according to 
Newton's law
\[
ma=m\ddot x=F.
\]
If the force $F$ depends on the position of the mass, as e.~g.~in the
case of a weight suspended from a spring, we get a differential equation
for the time dependent position $x(t)$:
\[
m\frac{d^2}{dt^2}x(t)=F(x(t)).
\]
Classical analysis proves theorems that show that we can expect such
a differential equation to have a unique solution for each
initial condition under relatively mild assumptions on the function $F$.
Only slight additional work is required for differential equations 
for a vector valued function.

\index{fields!electrical}
\index{temperature distribution}
\index{pressure}
\index{gas}
\index{flow}
\index{fluid}
To describe electrical fields, temperature distributions, pressure
in a gas or flow velocity in a fluid, we need to determine functions 
that do not only depend on a single time variable, but also on additional
variables like position and velocity.
In addition, the rate of change over time of these quantities may also depend
on the rate of change with respect to spacial displacements.

The deflection of a vibrating string is a function $f(x,t)$, where $x$ is
the position along the string and $t$ is time.
\index{string}
For each point $t_0$ in time, the shape of the string is different and
described by the function $x\mapsto f(x,t_0)$.
Any hope that it might be possible to derive a differential equation
for the motion of point on the string in isolation is quickly dashed
by the simple observation, that the force acting on such point
increases if neightboring points are deviate more from a straight line.
This means that the acceleration of a point on the string depends on 
the curvature.
We therefore expect an equation that links the second time derivative of $f$
to the second spacial derivative.
The equation we will later derive is the wave equation, it has the form
\index{wave equation}
\begin{align*}
a^2\frac{\partial^2}{\partial x^2}f&= \frac{\partial^2}{\partial t^2}f.
\end{align*}
If we are able to solve such a partial differential equation,
we will be able to predict the shape of the string for every point in
time.
Equivalently, we will be able to predict the movement $t\mapsto f(x_0,t)$
for every point $x_0$ on the string.
\index{differential equation!partial}

\index{Differentialgleichung!gew\"ohnliche}
The theory of ordinary differential equations shows that the equation
by itself does not determine the solution.
Additional data in the form of initial or boundary conditions has to be
specified.
Partial differential equations will not be different in that regard.
The theory thus has to provide answers to the following questions:
\begin{enumerate}
\item
How do we have to formulate the differential equation in order to ensure
that a solution will be well defined?
\item
What properties will the solution have?
What does the solution look like close to the boundary, does it remain
continuous or differentiable, does it increase beyond any limit?
\item
How can we compute solutions numerically?
\end{enumerate}
This course tries to give some basics answers to these questions.
It has two parts.
The first part deals with the theoretical basics.
It teaches how to set up a problem involving partial differential equations
so that conclusions about existence and uniqueness of solutions are possible.
In some cases, we will even be able to find a solution in closed form.
This, however, will be rather the exception than the rule.
The second part then strives to actually compute these solutions using
a computer algorithm.

The present lecture notes cover the first part of the course.
They are structured as follows:
\begin{enumerate}
\item
Examples of partial differential equations.
The most important partial differential equations from mechanics,
thermodynamics and electrodynamics are presented.
They serve as prototypical examples in later chapters to illustrate
and apply the theoretical insights.
\item
Method of separation of variables.
\item
Solutions using integral transforms like the Laplace transform or the 
Fourier transform.
\index{Laplace transform}
\index{Fourier transform}
\item
Classification of partial differential equations of second order.
These equations turn out to be of utmost practical importance, but
they arise in three distinctly different types with very different properties
and which require vastly different solution methods.
\item
Elliptic partial differential equations.
\item
Parabolic partial differential equations with the heat equation as the
prime example.
\index{parabolic}
\index{heat equation}
\item
Hyperbolic partial differential equations with the wave equation as the
prime example.
\index{hyperbolic}
\index{wave equation}
\item
Some nonlinear partial differential equations.
\end{enumerate}


