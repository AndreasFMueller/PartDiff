%\documentclass[handout]{beamer}
\documentclass{beamer}
\usepackage{array}
\usepackage{graphicx}
\usepackage{german}
%\usepackage{txfonts}

\mode<beamer>{%
\usetheme[hideothersubsections,hidetitle]{Hannover}
}
\title[]{Partielle Differentialgleichungen}
\subtitle{7. Sitzung: Hyperbolische PDGL}
\date[1.~April 2015]{1.~April 2015}
\author{Prof.~Dr.~Andreas M"uller}
\begin{document}

\begin{frame}
\section{Hyperbolische PDGL}
\titlepage
\end{frame}

\begin{frame}
\frametitle{Klassifikation}

\begin{center}
\begin{tabular}{llll}
Klasse&Bedingung&Beispiel&Anwendung\\
\hline
elliptisch &$\begin{aligned}P&=n\mathstrut\end{aligned}$
	&$\displaystyle \Delta u=f                                $
		&Potential\\
&	&	&Eigenwertproblem\\
\hline
parabolisch&%$P=n-1, Z=1$
$\begin{aligned}P&=n-1\mathstrut\\Z&=1\mathstrut\end{aligned}$
	&$\displaystyle \frac{\partial u}{\partial t}=\Delta u    $
		&W"armeleitung\\
\hline
hyperbolisch&%$P=n-1, N=1$
$\begin{aligned}P&=n-1\mathstrut\\N&=1\mathstrut\end{aligned}$
	&$\displaystyle \frac{\partial^2 u}{\partial t^2}=\Delta u$
		&Wellen\\
\hline
\end{tabular}
\end{center}

Klasse verr"at:
\begin{itemize}
\item Charakter der L"osung: Schwingungen, exponentielles Abklingen
\item Welches L"osungsverfahren geeignet ist
\end{itemize}

\end{frame}

\begin{frame}
\frametitle{Maximumprinzip}
\begin{definition}
$u\colon\Omega\to\mathbb R$ heisst harmonisch wenn $\Delta u=0$.
\end{definition}

\begin{theorem}
Wenn $\Omega$ beschr"ankt und zusammenh"anged ist, dann nimmt eine
harmonische Funktion ihr Maximum und Minimum auf dem Rand $\partial\Omega$ an.
\end{theorem}

\begin{theorem}
Wenn $\Omega$ beschr"ankt und zusammenh"angend ist, dann ist die 
L"osung des Randwertproblems eindeutig, falls sie existiert.
\end{theorem}

\end{frame}

\begin{frame}
\frametitle{Greensche Funktion}
{\bf Problem:}
\[
\Delta u=f\quad\text{in $\Omega$},\qquad u_{|\partial\Omega}=g
\]
\bigskip

{\bf L"osung:}
\[
u(x)
=
\int_{\Omega} G(x,\xi)f(\xi)\,d\xi
+
\int_{\partial\Omega}  \operatorname{grad}_\xi G(x,\xi) g(\xi)\,d\xi.
\]
\bigskip
\pause

{\bf Konsequenz:} Alle Randwerte haben Einfluss auf $u(x)$.

\bigskip
\pause

{\bf Kontrast:} Quasilineare PDGL 1. Ordnung: $u(x)$ h"angt nur von den
Randwerten am Anfang der Charakteristik durch $x$ ab.

\end{frame}

\begin{frame}
\frametitle{Elliptische PDGL}
\begin{center}
\includegraphics{../../skript/images/kausal-1.pdf}
\end{center}
\end{frame}

\begin{frame}
\frametitle{Quasilineare PDGL 1. Ordnung}
\begin{center}
\includegraphics{../../skript/images/kausal-2.pdf}
\end{center}
\end{frame}

\begin{frame}
\frametitle{Parabolische PDGL}
\begin{center}
\includegraphics{../../skript/images/kausal-3.pdf}
\end{center}
\end{frame}

\begin{frame}
\frametitle{Hyperbolische PDGL}

\begin{itemize}[<+->]
\item
Wellengleichung
\item
endliche Ausbreitungsgeschwindigkeit
\item
Jeder Randwert hat ein durch die Ausbreitungsgeschwindigkeit
definiertes Einflussgebiet
\end{itemize}

\bigskip

{\bf Aufgabe:}
Finde die Einflussgebiete f"ur eine beliebige hyperbolische PDGL

\end{frame}

\begin{frame}
\frametitle{Charakteristiken}

Anfangskurven f"ur das Cauchy-Problem:
\[
t\mapsto \left\{\begin{aligned}
x(t)\\
y(t)\\
u(t)
\end{aligned}\right.
\]

\begin{definition}
Charakteristiken sind Kurven, die nicht als Anfangskurven f"ur ein
Cauchy-Problem geeignet sind.
\end{definition}

\bigskip

{\bf Hauptresultat:}
\medskip

\begin{theorem}
Die L"osungsfl"achen wird von Charakteristiken erzeugt.
\end{theorem}

\end{frame}

\begin{frame}
\frametitle{Hyperbolische PDGL}
Partielle Differentialgleichung zweiter Ordnung
\[
Lu
=
a\frac{\partial^2 u}{\partial x^2}
+
2b \frac{\partial^2 u}{\partial x\,\partial y}
+
c\frac{\partial^2 u}{\partial y^2}
+
d\frac{\partial u}{\partial x}
+
e\frac{\partial u}{\partial y}
+
fu=g
\]
mit
\[
\left|\begin{matrix}
a&b\\b&c
\end{matrix}\right|
=ac-b^2 < 0
\quad\Leftrightarrow\quad
\text{$L$ hyperbolisch}
\]
im Gebiet $\Omega$.
\end{frame}

\begin{frame}
\frametitle{Streifen}

Anfangsdaten f"ur eine PDGL 2.~Ordnung:
\[
t\mapsto
\left\{
\begin{aligned}
x(t)\\
y(t)\\
u(t)\\
p(t)\\
q(t)
\end{aligned}
\right.
\]

\begin{definition}
Cauchy-Problem f"ur die PDGL $Lu = g$: finde Funktion $u(x,y)$ mit $Lu=g$ so,
dass
\begin{align*}
u(x(t),y(t))&=u(t), &\frac{\partial u}{\partial x}(x(t),y(t))&=p(t)\\
            &       &\frac{\partial u}{\partial y}(x(t),y(t))&=q(t)
\end{align*}
\end{definition}

\end{frame}

\begin{frame}
\frametitle{Charakteristiken}

Streifen und PDGL m"ussen zusammen die zweiten Ableitungen festlegen.
\begin{align*}
p(t)&=\frac{\partial u}{\partial x}(x(t), y(t))
&
\dot p(t)
&={\color{red} \frac{\partial^2 u}{\partial x^2}}\dot x(t)
+
{\color{red}\frac{\partial^2u}{\partial x\partial y}}\dot y(t)
\\
q(t)&=\frac{\partial u}{\partial y}(x(t), y(t))
&
\dot q(t)
&=
{\color{red}\frac{\partial^2 u}{\partial y\partial x}}\dot x(t)
+
{\color{red}\frac{\partial^2u}{\partial y^2}}\dot y(t)
\\
&&g(x(t),y(t))&=Lu (x(t),y(t))
\end{align*}
3 lineare Gleichungen f"ur drei 2.~Ableitungen
$\Rightarrow$
Bedingung f"ur Charakteristik:
\[
a\,\dot y(t)^2-2b\,\dot x(t)\,\dot y(t)+c\,\dot x(t)^2=0
\]
\end{frame}

\begin{frame}
\frametitle{Charakteristische Streifen}
Streifen so, dass
\begin{enumerate}
\item
$x(t), y(t), u(t)$ eine Charakteristik bilden.
\item
Es gibt unendliche viele L"osungen f"ur die zweiten Ableitungen entlang
dieser Charakteristik
\end{enumerate}

Zusatzbedingungen:
\begin{gather*}
a\dot p(t)\dot y(t)
-
h \dot x(t) \dot y(t)
+
c \dot x(t)\dot q(t)=0
\\
\text{mit\;}
h=g-dp(t)-eq(t)-fu(x(t),y(t))
\end{gather*}

\bigskip

{\bf Hauptresultat:}
\medskip

\begin{theorem}
Die L"osungsfl"ache wird "uberdeckt von charakteristischen Streifen.
\end{theorem}

\end{frame}

\begin{frame}
\frametitle{Hyperbolische PDGL}
\begin{center}
\includegraphics{../../skript/images/kausal-4.pdf}
\end{center}
\end{frame}

\begin{frame}
\frametitle{"Uberschallstr"omung}
\begin{center}
\includegraphics[width=\hsize]{../../skript/graphics/i-5-1.jpg}
\end{center}
\end{frame}

\begin{frame}
\frametitle{Mach-Kegel}
\begin{center}
\includegraphics[width=0.8\hsize]{../../skript/graphics/shock.png}
\end{center}
\end{frame}

\end{document}
