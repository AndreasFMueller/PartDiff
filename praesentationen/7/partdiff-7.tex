%\documentclass[handout]{beamer}
\documentclass{beamer}
\usepackage{array}
\usepackage{graphicx}
\usepackage{german}
%\usepackage{txfonts}

\mode<beamer>{%
\usetheme[hideothersubsections,hidetitle]{Hannover}
}
\title[]{Partielle Differentialgleichungen}
\subtitle{7. Sitzung: Hyperbolische PDGL}
\date[1.~April 2015]{1.~April 2015}
\author{Prof.~Dr.~Andreas M"uller}
\begin{document}

\begin{frame}
\section{Hyperbolische PDGL}
\titlepage
\end{frame}

\begin{frame}
\frametitle{Klassifikation}

\begin{center}
\begin{tabular}{llll}
Klasse&Bedingung&Beispiel&Anwendung\\
\hline
elliptisch &$\begin{aligned}P&=n\mathstrut\end{aligned}$
	&$\displaystyle \Delta u=f                                $
		&Potential\\
&	&	&Eigenwertproblem\\
\hline
parabolisch&%$P=n-1, Z=1$
$\begin{aligned}P&=n-1\mathstrut\\Z&=1\mathstrut\end{aligned}$
	&$\displaystyle \frac{\partial u}{\partial t}=\Delta u    $
		&W"armeleitung\\
\hline
hyperbolisch&%$P=n-1, N=1$
$\begin{aligned}P&=n-1\mathstrut\\N&=1\mathstrut\end{aligned}$
	&$\displaystyle \frac{\partial^2 u}{\partial t^2}=\Delta u$
		&Wellen\\
\hline
\end{tabular}
\end{center}

Klasse verr"at:
\begin{itemize}
\item Charakter der L"osung: Schwingungen, exponentielles Abklingen
\item Welches L"osungsverfahren geeignet ist
\end{itemize}

\end{frame}

\begin{frame}
\frametitle{Maximumprinzip}
\begin{definition}
$u\colon\Omega\to\mathbb R$ heisst harmonisch wenn $\Delta u=0$.
\end{definition}

\begin{theorem}
Wenn $\Omega$ beschr"ankt und zusammenh"anged ist, dann nimmt eine
harmonische Funktion ihr Maximum und Minimum auf dem Rand $\partial\Omega$ an.
\end{theorem}

\begin{theorem}
Wenn $\Omega$ beschr"ankt und zusammenh"angend ist, dann ist die 
L"osung des Randwertproblems eindeutig, falls sie existiert.
\end{theorem}

\end{frame}

\begin{frame}
\frametitle{Greensche Funktion}
{\bf Problem:}
\[
\Delta u=f\quad\text{in $\Omega$},\qquad u_{|\partial\Omega}=g
\]
\bigskip

{\bf L"osung:}
\[
u(x)
=
\int_{\Omega} G(x,\xi)f(\xi)\,d\xi
+
\int_{\partial\Omega}  \operatorname{grad}_\xi G(x,\xi) g(\xi)\,d\xi.
\]
\bigskip
\pause

{\bf Konsequenz:} Alle Randwerte haben Einfluss auf $u(x)$.

\bigskip
\pause

{\bf Kontrast:} Quasilineare PDGL 1. Ordnung: $u(x)$ h"angt nur von den
Randwerten am Anfang der Charakteristik durch $x$ ab.

\end{frame}

\begin{frame}
\frametitle{Elliptische PDGL}
\begin{center}
\includegraphics{../../skript/images/kausal-1.pdf}
\end{center}
\end{frame}

\begin{frame}
\frametitle{Quasilineare PDGL 1. Ordnung}
\begin{center}
\includegraphics{../../skript/images/kausal-2.pdf}
\end{center}
\end{frame}

\begin{frame}
\frametitle{Parabolische PDGL}
\begin{center}
\includegraphics{../../skript/images/kausal-3.pdf}
\end{center}
\end{frame}

\begin{frame}
\frametitle{Charakteristiken}

\begin{definition}
Kurven, die nicht als Anfangskurven f"ur ein Cauchy-Problem geeignet sind.
\end{definition}

\bigskip

Hauptresultat:
\medskip

\begin{theorem}
Die L"osungsfl"achen wird von Charakteristiken erzeugt.
\end{theorem}

\end{frame}

\begin{frame}
\frametitle{Streifen}

\begin{definition}
Drei Funktionen $u(s)$, $p(s)$ und $q(s)$, wobei $p$ und $q$ die Bedeutung
der Steigung $\partial u/\partial x$ bzw.~$\partial u/\partial y$ haben.
\end{definition}

\end{frame}

\begin{frame}
\frametitle{Hyperbolische PDGL}
\begin{center}
\includegraphics{../../skript/images/kausal-4.pdf}
\end{center}
\end{frame}

\end{document}
