%\documentclass[handout]{beamer}
\documentclass{beamer}
\usepackage{array}
\usepackage{graphicx}
\usepackage{german}
%\usepackage{txfonts}

\mode<beamer>{%
\usetheme[hideothersubsections,hidetitle]{Hannover}
}
\title[]{Partielle Differentialgleichungen}
\subtitle{5. Sitzung: Transformation II, Klassifikation}
\date[18.~M"arz 2015]{18.~M"arz 2015}
\author{Prof.~Dr.~Andreas M"uller}
\begin{document}

\begin{frame}
\section{Transformation}
\titlepage
\end{frame}

\begin{frame}
\frametitle{Differentialgleichung}

Wellengleichung:
\[
\frac{1}{c^2}
\frac{\partial^2 u}{\partial t^2}
=
\frac{{\color{red}\partial^2} u}{{\color{red}\partial x^2}},
\]
Gebiet:
\[
\Omega = \mathbb R^+ \times {\color{red}(0,2\pi)} \to \mathbb R
\]
Randbedingungen:
\begin{align*}
u(t,    0)&= 0&u(0,{\color{red}x})&=f({\color{red}x})\\
u(t, 2\pi)&= 0&\frac{\partial}{\partial t}u(0,{\color{red}x})&=g({\color{red}x})
\end{align*}
Fourierkoeffizienten von $f$ und $g$: $b_k^{(f)}$ und $b^{(g)}_k$

\end{frame}

\begin{frame}
\frametitle{Funktion von $x$ und $t$}

%$u$ eine Funktion von $x$ und $t$:
%\[
%u\colon (0,2\pi)\times \mathbb R^+\to \mathbb R
%\]
Fourierkoeffizienten h"angen von $t$ ab:
\[
u(t,x)=\frac{a_0(t)}2+\sum_{k=1}^\infty \bigl(a_k(t)\cos kx+b_k(t)\sin kx\bigr)
\]
Fourier-Transformation:
\[
{\cal F}\colon
u\mapsto
a_0(t), a_k(t), b_k(t)
\]
Implizite Separation:
\begin{enumerate}
\item Ortsabh"angigkeit in $\cos kx$ und $\sin kx$
\item Zeitabh"angigkeit in $a_k(t)$ und $b_k(t)$
\end{enumerate}

\end{frame}

\begin{frame}
\frametitle{Transformation}
Randbedingungen:
\[
a_0=0\qquad
a_k=0\;\forall k > 0
\]
Differentialgleichung:
\begin{align*}
\frac{1}{c^2}\ddot b_k(t)&= {\color{red}-k^2}b_k(t)
\end{align*}
Anfangsbedingung:
\begin{align*}
b_k(0)&= b_k^{(f)}& \dot b_k(0)&=b_k^{(g)}
\end{align*}
{\bf Gew"ohnliche} Differentialgleichungen f"ur die Fourier-Koeffizienten
\end{frame}

\begin{frame}
\frametitle{Idee}

{\bf Lineare} Transformation $\cal T$ mit folgenden Eigenschaften

\begin{align*}
u({\color{red}x},y)&&&\mapsto &&{\cal T}u(k,y)
\\
\frac{\partial u}{\partial y}&&&\mapsto&&\frac{\partial{\cal T}u}{\partial y}(k,y)
\\
\frac{{\color{red}\partial} u}{\color{red}\partial x}&&&\mapsto&&\text{{\color{red}algebraischer Ausdruck} mit $\displaystyle {\cal T}u(k,y)$}
\\
\text{PDGL}&&&\mapsto&&\text{gew"ohnliche DGL f"ur $y\mapsto {\cal T}u(k,y)$}
\\
           &&&       &&\text{algebraische Gleichung in $k$}
\\
\text{L"osung:}&&&   &&{\cal T}^{-1}({\cal T}u)(x,y)
\end{align*}
D.~h.~{\color{red}$x$-Ableitung} wird in eine {\color{red}algebraische Operation} transformiert

\end{frame}

\begin{frame}
\frametitle{Laplace-Transformation}
\begin{definition}
$f\colon \mathbb R^+\to\mathbb R$ hat Laplace-Transformierte
\[
{\cal L}f(s)=\int_0^\infty f(t)e^{-st}\,dt
\]
\end{definition}
\pause
\begin{center}
\begin{tabular}{>{$}c<{$}>{$\displaystyle}c<{$}}
f(t)&({\cal L}f)(s)\\
\hline
c&\frac{c\mathstrut}{s\mathstrut}\\
t^n&\frac{n!\mathstrut}{s^{n+1}\mathstrut}\\
e^{at}&\frac{1\mathstrut}{s-a\mathstrut}\\
\sin(at)&\frac{a\mathstrut}{s^2+a^2\mathstrut}\\
\cos(at)&\frac{s\mathstrut}{s^2+a^2\mathstrut}\\
\hline
\end{tabular}
\end{center}

\end{frame}

\begin{frame}
\frametitle{Ableitung}

\begin{align*}
({\cal L}f')(s)
&=
\int_0^\infty \underbrace{\mathstrut f'(t)}_{\textstyle\uparrow}\underbrace{\mathstrut e^{-st}}_{\textstyle\downarrow}\,dt
=
\underbrace{\biggl[f(t)e^{-st}\biggr]_0^\infty}_{\textstyle-f(0)}
+s\underbrace{\int_0^\infty f(t)e^{-st}\,dt}_{\textstyle({\cal L}f)(s)}
\end{align*}

\pause

\begin{center}
\begin{tabular}{>{$}l<{$}>{$\displaystyle}l<{$}}
f(t)&({\cal L}f)(s)\mathstrut\\
\hline
f'(t)&s({\cal L}f)(s)-f(0)\\
f''(t)&s^2({\cal L}f)(s)-sf(0)-f'(0)\\
\dots&\dots\\
f^{(n)}(t)&s^n({\cal L}f)(s)-s^{n-1}f(0)-s^{n-2}f'(0)-\dots - f^{(n-1)}(0)\\
\hline
\end{tabular}
\end{center}

Ziel erreicht:
\[
\text{$n$-te Ableitung}\rightarrow \text{Multiplikation mit $s^n$}
\]
\end{frame}

\begin{frame}
\frametitle{R"ucktransformation}
\begin{enumerate}
\item ${\cal L}$ linear $\Rightarrow$ ${\cal L}^{-1}$ linear
\item Tabelle der Laplace-Transformierten
\item Partialbruchzerlegung
\end{enumerate}
\end{frame}

\begin{frame}
\frametitle{Transformationsmethode}
\begin{enumerate}[<+->]
\item PDGL transformieren
\begin{align*}
\frac{{\color{red}\partial} u}{\color{red}\partial t}&=\frac{\partial^2 u}{\partial x^2}&
&\mapsto&
{\color{red}s} ({\cal L}u)(s,x){\color{red}-u(0,x)}&=\frac{\partial^2 ({\cal L}u)(s,x)}{\partial x^2}
\end{align*}
\item Randwerte einsetzen
\[
{\color{red}s}({\cal L}u)(s,x)-f(x)=\frac{\partial^2 ({\cal L}u)(s,x)}{\partial x^2}
\]
%\item Randwerte f"ur $x\mapsto ({\cal L}u)(s,x)$ 
\item  F"ur jedes $s$ gew"ohnliche DGL in $x$ l"osen mit
Randbedingungen f"ur $({\cal L})(s,x)$:
\begin{align*}
u(t,-l)&=g(t)&&\mapsto&({\cal L}u)(s,-l)&=({\cal L}g)(s)\\
u(t,l)&=h(t)&&\mapsto&({\cal L}u)(s,l)&=({\cal L}h)(s)
\end{align*}
\item Aufl"osen nach $({\cal L}u)(s,x)$
\item R"ucktransformation
\end{enumerate}
\end{frame}

\begin{frame}
\section{Klassifikation}
\frametitle{Symbol}
Zu jedem Differentialoperator 2. Ordnung gibt es eine Symbol-Matrix
\begin{align*}
a\frac{\partial^2}{\partial x^2}
+2b\frac{\partial^2}{\partial x\,\partial y}
+c\frac{\partial^2}{\partial y^2}
&\mapsto
\begin{pmatrix}
a&b\\
b&c
\end{pmatrix}
\\
\sum_{i,j=1}^na_{ij}\frac{\partial^2}{\partial x_i\,\partial x_j}
&\mapsto
A=
\begin{pmatrix}
a_{11}&\dots &a_{1n}\\
\vdots&\ddots&\vdots\\
a_{n1}&\dots &a_{nn}
\end{pmatrix}
\end{align*}
$A$ ist symmetrisch
\end{frame}

\begin{frame}
\frametitle{Eigenwerte}
$A$ symmetrisch $\Rightarrow$ $A$ ist diagonalisierbar
\[
A\sim
\begin{pmatrix}
\lambda_1&         &      &         \\
         &\lambda_2&      &         \\
         &         &\ddots&         \\
         &         &      &\lambda_n
\end{pmatrix}
\]
Interessant sind die Vorzeichen der $\lambda_i$:
\begin{center}
\begin{tabular}{lc}\\
Eigenwerte&Anzahl\\
\hline
positiv&$P$\\
negativ&$N$\\
$0$    &$Z$\\
\hline
\end{tabular}
\end{center}
\end{frame}

\begin{frame}
\frametitle{Klassifikation}

\begin{center}
\begin{tabular}{llll}
Klasse&Bedingung&Beispiel&Anwendung\\
\hline
elliptisch &$\begin{aligned}P&=n\mathstrut\end{aligned}$
	&$\displaystyle \Delta u=f                                $
		&Potential\\
&	&	&Eigenwertproblem\\
\hline
parabolisch&%$P=n-1, Z=1$
$\begin{aligned}P&=n-1\mathstrut\\Z&=1\mathstrut\end{aligned}$
	&$\displaystyle \frac{\partial u}{\partial t}=\Delta u    $
		&W"armeleitung\\
\hline
hyperbolisch&%$P=n-1, N=1$
$\begin{aligned}P&=n-1\mathstrut\\N&=1\mathstrut\end{aligned}$
	&$\displaystyle \frac{\partial^2 u}{\partial t^2}=\Delta u$
		&Wellen\\
\hline
\end{tabular}
\end{center}

Klasse verr"at:
\begin{itemize}
\item Charakter der L"osung: Schwingungen, exponentielles Abklingen
\item Welches L"osungsverfahren geeignet ist
\end{itemize}

\end{frame}

\end{document}
