
\section{2. Ordnung}

\begin{frame}
\frametitle{Linearer Differentialoperator 2. Ordnung}
Operator:
\[
L
=
\sum_{i,j=1}^n a_{ij}(x)\frac{\partial^2}{\partial x_i\partial x_j}
+
\sum_{i=1}^n b_i(x)\frac{\partial}{\partial x_i}
+
c(x)
\]
Homogene Gleichung:
\[
Lu=0\qquad\text{in $\Omega$}
\]
Inhomogene Gleichung
\[
Lu=f\qquad\text{in $\Omega$}
\]
\end{frame}

\begin{frame}
\frametitle{L"osungsmenge}
{\bf Homogene Gleichung:}

\[
{\mathbb L}
=\{
u\colon \Omega\to \mathbb R\,|\, Lu=f
\}
\]

\bigskip

{\bf Inhomogene Gleichung:}

\[
{\mathbb L}_h
=\{
u\colon \Omega\to \mathbb R\,|\, Lu=0
\}
\]

{\bf Allgemein:}
\[
{\mathbb L} = u_p + {\mathbb L}_h
=\{u_p+u_h\,|\,u_h\in\mathbb L_h\}
\]
$u_p$ heisst partikul"are L"osung.

\end{frame}

\begin{frame}
\frametitle{L"osungsalgorithmus}
Problemstellung:
\begin{align*}
Lu&=f\quad\text{in $\Omega$,}
&
 u&=g\quad\text{auf $\partial\Omega$}
\end{align*}

\begin{enumerate}
\item Partikul"are L"osung $u_p$ (ohne Randwerte)
\begin{align*}
Lu_p&=f\quad\text{in $\Omega$}
\end{align*}
\item
L"osung des Randwertproblems $u_r\in\mathbb L_h$
\begin{align*}
Lu_r&=0    \quad\text{in $\Omega$,}&
 u_r&=g-u_p\quad\text{auf $\partial\Omega$}
\end{align*}
\item
L"osung $u_h$ mit
\begin{align*}
Lu_h&=0\quad\text{in $\Omega$,}
&
 u_h&=0\quad\text{auf $\partial\Omega$}
\end{align*}
\end{enumerate}
Zusammen:
\[
u = u_p+u_r+u_h
\]
\end{frame}

\begin{frame}
\section{Klassifikation}
\frametitle{Symbol}
Zu jedem Differentialoperator 2. Ordnung gibt es eine Symbol-Matrix
\begin{align*}
a\frac{\partial^2}{\partial x^2}
+2b\frac{\partial^2}{\partial x\,\partial y}
+c\frac{\partial^2}{\partial y^2}
&\mapsto
\begin{pmatrix}
a&b\\
b&c
\end{pmatrix}
\\
\sum_{i,j=1}^na_{ij}\frac{\partial^2}{\partial x_i\,\partial x_j}
&\mapsto
A=
\begin{pmatrix}
a_{11}&\dots &a_{1n}\\
\vdots&\ddots&\vdots\\
a_{n1}&\dots &a_{nn}
\end{pmatrix}
\end{align*}
$A$ ist symmetrisch
\end{frame}

\begin{frame}
\frametitle{Eigenwerte}
$A$ symmetrisch $\Rightarrow$ $A$ ist diagonalisierbar
\[
A\sim
\begin{pmatrix}
\lambda_1&         &      &         \\
         &\lambda_2&      &         \\
         &         &\ddots&         \\
         &         &      &\lambda_n
\end{pmatrix}
\]
Interessant sind die Vorzeichen der $\lambda_i$:
\begin{center}
\begin{tabular}{lc}\\
Eigenwerte&Anzahl\\
\hline
positiv&$P$\\
negativ&$N$\\
$0$    &$Z$\\
\hline
\end{tabular}
\end{center}
\end{frame}

\begin{frame}
\frametitle{Klassifikation}

\begin{center}
\begin{tabular}{llll}
Klasse&Bedingung&Beispiel&Anwendung\\
\hline
elliptisch &$\begin{aligned}P&=n\mathstrut\end{aligned}$
	&$\displaystyle \Delta u=f                                $
		&Potential\\
&	&	&Eigenwertproblem\\
\hline
parabolisch&%$P=n-1, Z=1$
$\begin{aligned}P&=n-1\mathstrut\\Z&=1\mathstrut\end{aligned}$
	&$\displaystyle \frac{\partial u}{\partial t}=\Delta u    $
		&W"armeleitung\\
\hline
hyperbolisch&%$P=n-1, N=1$
$\begin{aligned}P&=n-1\mathstrut\\N&=1\mathstrut\end{aligned}$
	&$\displaystyle \frac{\partial^2 u}{\partial t^2}=\Delta u$
		&Wellen\\
\hline
\end{tabular}
\end{center}

Klasse verr"at:
\begin{itemize}
\item Charakter der L"osung: Schwingungen, exponentielles Abklingen
\item Welches L"osungsverfahren geeignet ist
\end{itemize}

\end{frame}

\section{Elliptisch}

\begin{frame}
\frametitle{Elliptische Differentialgleichungen}
Standardproblem:
\[
\Delta u = f
\qquad\text{mit}\qquad
\Delta 
=
\frac{\partial^2}{\partial x^2}
+
\frac{\partial^2}{\partial y^2}
+
\frac{\partial^2}{\partial z^2}
\]
hat Symbol
\[
A=\begin{pmatrix}1&0&0\\0&1&0\\0&0&1\end{pmatrix}
\qquad
\Rightarrow
\;
\text{$\Delta$ ist elliptisch}
\]
L"osungen des homogenen Systems:
\[
\Delta u=0
\qquad\Rightarrow\;
\text{$u$ harmonisch}
\]
\end{frame}
