\documentclass{beamer}
\usepackage{array}
\usepackage{german}
\usepackage{graphicx}
\mode<beamer>{%
\usetheme[hideothersubsections,hidetitle]{Hannover}
}
\begin{document}

\begin{frame}
\section{Separation}
\frametitle{Separation}
Zwei wesentliche Ideen:
\begin{enumerate}
\item Ansatz:
\begin{align*}
u(x,y)&=X(x)\cdot Y(y)\\
S(x,y,z,t)&=S_1(x)+S_2(y)+S_3(z)+S_4(t)
\end{align*}
\item Beide Seiten der Gleichung sind konstant:
\[
F(x)=G(y)\quad\Rightarrow\quad\left\{
\begin{aligned}
F(x)&=\lambda\\
G(y)&=\lambda
\end{aligned}
\right.
\]
\end{enumerate}
Separation zerlegt eine PDGL in zwei gew"ohnliche DGL, welche "uber
die Konstante $\lambda$ miteinander verbunden sind.
\end{frame}

\begin{frame}
\frametitle{Vorgehen}
\begin{enumerate}
\item Separationsansatz in DGL einsetzen
\item Variablen trennen $\Rightarrow$ zwei gew"ohnliche DGL, verbunden
mit einer Konstanten $\lambda$
\item Gew"ohnliche DGL f"ur jedes $\lambda$ l"osen: f"ur jedes $\lambda$ eine L"osung
$u_\lambda$
\item Mit Randbedingungen soweit m"oglich L"osungen reduzieren $\Rightarrow$
m"ogliche $\lambda$ bestimmen
\item Allgemeine L"osung zusammensetzen: $u = \sum_{\lambda}a_\lambda u_\lambda$
\item Verbleibende Randbedingungen verwenden, um $a_\lambda$ zu bestimmen
\end{enumerate}
\end{frame}

\begin{frame}
\frametitle{Beobachtungen}
\begin{enumerate}
\item Auf kompakten Gebieten gibt es eine diskrete Familie von Teill"osungen
$u_{\lambda}$.
\item Vollst"andige L"osung entsteht durch "Uberlagerung von Teill"osungen
$\Rightarrow$ dazu wird {\bf Linearit"at ben"otigt}
\item Randbedingungen k"onnen durch Fourier-Reihen erf"ullt werden
\item $\Rightarrow$ zu jedem kompakten Gebiet gibt es eine ``verallgemeinerte
Fourier-Theorie'' ($\Rightarrow$ Hilbert-R"aume)
\end{enumerate}
\end{frame}

\begin{frame}
\section{Anwendungen}
\frametitle{Anwendungen}
\begin{enumerate}
\item Eletrodynamik
\item W"armelehre
\item Wahrscheinlichkeitsrechnung
\item Fluiddynamik
\item Quantenphysik
\item Finanzmathematik
\item Mechanik
\end{enumerate}
\end{frame}

\begin{frame}
\frametitle{Elektrodynamik}
\begin{enumerate}
\item Signalausbreitung entlang eines Kabels (Telegraphengleichung)
\item Stromverteilung in Leitern mit einfachem Querschnitt:
Skin-Effekt
\item Feld von Wellenleitern mit einfacher Geometrie: 
\begin{itemize}
\item Koaxialkabel (Zylinderkoordinaten)
\item Hohlleiter mit rechteckigem Querschnitt
\item Lecherleitung (elliptische Zylinderkoordinaten)
\end{itemize}
\item Abstrahlung einfacher Antennen
\begin{itemize}
\item idealisierter Dipol (Kugelkoordinaten)
\item Kugel (Kugelkoordinaten)
\end{itemize}
\end{enumerate}
\end{frame}

\begin{frame}
\frametitle{Elektrodynamik: Dipolstrahlung}
Berechnung der Dipolstrahlung durch Heinrich Hertz 1889
\begin{center}
\includegraphics[width=\hsize]{nfld-fig2re.pdf}
\end{center}
\end{frame}

\begin{frame}
\frametitle{W"armelehre}
\begin{enumerate}
\item W"armeleitung in Gebieten mit einfacher
Geometrie (Rechtecke, Quader, Kreise, Kugeln, St"abe)
\item Diffusion
\item Transport (Fokker-Planck-Gleichung) in einfachen Geometrien
\item Transformation auf verschiedene thermodynamische Potentiale
\end{enumerate}
\end{frame}

\begin{frame}
\frametitle{Fluiddynamik}

Anwendungsm"oglichkeiten beschr"ankt, da Navier-Stokes nicht linear!

\medskip

Anwendung auf linearisierte Probleme oder laminare Str"omung:
\begin{enumerate}
\item Instabilit"at der laminaren Str"omung in Rohren mit einfachen
Querschnitten (Kreis, Rechteck)
\item Str"omung zwischen zwei sich drehenden Zylindern (Lager, Str"omung
des Schmiermittels)
\item "Uberschallstr"omung
\end{enumerate}
\end{frame}

\begin{frame}
\frametitle{"Uberschallstr"omung}
\begin{center}
\includegraphics[width=\hsize]{schuss.pdf}
\end{center}
\end{frame}

\begin{frame}
\frametitle{Quantenphysik}
\begin{enumerate}
\item Teilchen im Vakuum
\item Teilchen in einem Potential (Elektron im homogenen elektrischen Feld)
\item Energieniveaus eines Wasserstoff-Atoms, He-Ions oder approximative
Berechnung der Energieniveaus in schwereren Atomen
(Kugelkoordinaten) $\Rightarrow$ ``Orbitale''
\item Energieniveaus eines ionisierten Wasserstoff-Molek"uls (elliptische Koordinaten)
\end{enumerate}
\end{frame}

\begin{frame}
\frametitle{Wasserstoff-Spektrum}
\begin{center}
\includegraphics[width=\hsize]{spectrum.pdf}
\end{center}
Die Quantentheorie erkl"art z.~B.~die Rydberg Formel:
\[
\frac1{\lambda}=R_H\biggl(\frac1{n_1^2}-\frac1{n_2^2}\biggr),\qquad n_2>n_1
\]
$\lambda=$ Wellenl"ange der Spektrallinie zum einem "Ubergang zwischen Quantenzust"anden
mit den Nummern $n_1$ und $n_2$.
\end{frame}

\begin{frame}
\frametitle{Balmer-Serie}
Sichtbar sind nur die Linien der Balmer-Serie, f"ur die 
Johann Jakob Balmer (1825-1898) die Formel
\[
\lambda=\frac{hm^2}{m^2-4},\qquad h=3.6456\cdot10^{-7}\text{m}
\]
f"ur die Wellenl"angen empirisch gefunden hat (1885).
\begin{itemize}
\item
Balmer konnte eine Spektrallinie f"ur $m=7$ mit Wellenl"ange 396nm
vorhersagen, die, wie sich herausstellte, bereits von \AA ngstr"om beobachtet
worden war.
\item 
Intensivstes Wasserstoff-Licht: H-$\alpha$ Linie f"ur $m=3$, 656nm, rot,
wird oft f"ur astronomische Beobachtungen herausgefiltert (``man sieht nur den
Wasserstoff'')
\end{itemize}

\medskip
Balmer war zeit seines Lebens Mathematiklehrer an einem M"adchengymnasium
in Basel.
\end{frame}

\begin{frame}
\frametitle{Sonne in H-$\alpha$}
\begin{center}
\includegraphics[width=0.7\hsize]{sunhalpha.pdf}
\end{center}
\end{frame}

\begin{frame}
\frametitle{Milchstrasse in H-$\alpha$}
\begin{center}
\includegraphics[width=\hsize]{skysurvey.jpg}
\end{center}
\end{frame}

\begin{frame}
\frametitle{Finanzmathematik}
Black-Scholes-Gleichung f"ur den Preis $V$ einer Option zu einer Aktie mit Preis
$S$
\[
\frac{\partial V}{\partial t}
+
rS\frac{\partial V}{\partial S}
+
\frac12\sigma^2S^2\frac{\partial^2 V}{\partial S^2}=rV
\]
$\sigma$ zuk"unftige Volatilit"at des Basiswertes, $r$ Zinssatz

Anfangswerte: Geldfl"usse zum F"alligkeitszeitpunkt der Option.

\end{frame}

\begin{frame}
\frametitle{Mechanik}
\begin{enumerate}
\item Ausbreitung von Wellen in einfachen Geometrien
\item Elastizit"at einfacher K"orper: Balken, rechteckige, kreisf"ormige
oder elliptische Platten, Kugelf"ormige Schalen. $\Rightarrow$ Eigenfrequenzen
\item Hamilton-Jacobi-Theorie
\end{enumerate}
\end{frame}

\begin{frame}
\section{Hamilton-Jacobi}
\frametitle{Hamilton-Jacobi-Theorie I}
{\bf Newton:} Ein K"orper bleibt im Zustand der Ruhe oder der gleichf"ormigen
Bewegung, solange keine "ausseren Kr"afte auf ihn wirken. 

{\bf In Formeln:}
\begin{align*}
x&=\frac{p}{m}\cdot t\\
p&=\operatorname{const}
\end{align*}

{\bf Idee:} Kann man das vielleicht noch besser machen? Zum Beispiel alles konstant!?
\begin{align*}
Q_i&=\operatorname{const}\\
P_i&=\operatorname{const}
\end{align*}
Kann man jede beliebige Wurfparabel durch 6 Konstanten beschreiben?
\end{frame}

\begin{frame}
\frametitle{Hamilton-Jacobi-Theorie II}

{\bf Ausgangspunkt:}
Hamilton-Funktion $H=$ Gesamtenergie, ausgedr"uckt in Koordinaten
und Impulsen. 
\[
H(x,y,z,p_x,p_y,p_z)=\frac1{2m}(p_x^2+p_y^2+p_z^2)+mgz
\]
(Galileo-Experiment: Massepunkt im Schwerefeld auf der Erdoberfl"ache)

{\bf Bewegungsgleichungen:} nach Hamilton
\begin{align*}
\frac{dx}{dt}&=\frac{\partial H}{\partial p_x}=\frac{p_x}{m}&
\frac{dp_x}{dt}&=-\frac{\partial H}{\partial x}=0\\
\frac{dy}{dt}&=\frac{\partial H}{\partial p_y}=\frac{p_y}{m}&
\frac{dp_y}{dt}&=-\frac{\partial H}{\partial y}=0\\
\frac{dy}{dt}&=\frac{\partial H}{\partial p_z}=\frac{p_z}{m}&
\frac{dp_z}{dt}&=-\frac{\partial H}{\partial z}=-mg
\end{align*}

\end{frame}

\begin{frame}
\frametitle{Hamilton-Jacobi-Theorie III}
Neue Koordinaten $Q_i$ und $P_i$ sind Konstanten: 
\begin{align*}
\frac{dQ_i}{dt}&=0&
\frac{dP_i}{dt}&=0
\end{align*}
Aus den Hamilton-Gleichungen folgt, dass die neue Hamilton-Funktion $K$ nur
eine Konstante ist:
\begin{align*}
\frac{\partial K}{\partial Q_i}&=-\frac{dP_i}{dt}=0&&\text{$K$ h"angt nicht von $Q_i$ ab}\\
\frac{\partial K}{\partial P_i}&=\frac{dQ_i}{dt}=0&&\text{$K$ h"angt nicht von $P_i$ ab}
\end{align*}
$\Rightarrow$
\[
K=\operatorname{const}
\]
\end{frame}

\begin{frame}
\frametitle{Hamilton-Jacobi-Theorie IV}

{\bf Transformation:} Transformation auf neue Koordinaten ist mit einer Funktion
\[
S(x,y,z,P_1,P_2,P_3)
\]
m"oglich.  $P_i$ sind die neuen, konstanten Impulse:

{\bf Koordinatentransformationsformeln:}
\begin{align*}
p_x&=\frac{\partial S}{\partial x}&
Q_1&=\frac{\partial S}{\partial P_1}\\
p_y&=\frac{\partial S}{\partial y}&
Q_2&=\frac{\partial S}{\partial P_2}\\
p_z&=\frac{\partial S}{\partial z}&
Q_3&=\frac{\partial S}{\partial P_3}
\end{align*}
$Q_i$ sind die neuen Koordinaten.
Neuen Hamilton-Funktion
\[
K(Q,P)=H+\frac{\partial S}{\partial t}
\]
\end{frame}

\begin{frame}
\frametitle{Hamilton-Jacobi-Theorie V}

{\bf Hamilton-Jacobi-Gleichung:}
\[
H\biggl(x,y,z,\frac{\partial S}{\partial x},\frac{\partial S}{\partial y},\frac{\partial S}{\partial z}\biggr)+\frac{\partial S}{\partial t}=0
\]
Nicht lineare PDGL:
\[
\frac1{2m}\biggl(
\biggl(\frac{\partial S}{\partial x}\biggr)^2
+
\biggl(\frac{\partial S}{\partial y}\biggr)^2
+
\biggl(\frac{\partial S}{\partial z}\biggr)^2
\biggr)
+mgz+\frac{\partial S}{\partial t}
=0
\]
Nach Separation:
\[
\frac1{2m}\biggl(
\biggl(\frac{dS_1}{dx}\biggr)^2
+
\biggl(\frac{dS_2}{dy}\biggr)^2
+
\biggl(\frac{dS_3}{dz}\biggr)^2
\biggr)+mgz=-\frac{dS_4}{dt}
\]
Linke Seite h"angt nicht von $t$ ab:
\[
\frac{dS_4}{dt}=\frac{\partial S}{\partial t}=\operatorname{const}=-P_1
\quad\Rightarrow\quad
S_4(t)=-P_1t
\]
\end{frame}

\begin{frame}
\frametitle{Hamilton-Jacobi-Theorie VI}

Trennung der $x$-Funktion $S_1$:
\[
\frac1{2m}
\biggl(\frac{dS_1}{dx}\biggr)^2
=-\frac1{2m}\biggl(
\biggl(\frac{dS_2}{dy}\biggr)^2
+
\biggl(\frac{dS_3}{dz}\biggr)^2
\biggr)
-mgz
+P_1
\]
Rechte Seite h"angt nicht von $x$ ab:
\[
\frac1{2m}\biggl(\frac{dS_1}{dx}\biggr)^2=\operatorname{const}=P_2
\quad\Rightarrow\quad
S_1=\sqrt{2mP_2}\, x
\]
Neue Differentialgleichung
\[
P_2+\frac1{2m}\biggl(\frac{dS_2}{dy}\biggr)^2
=
-\frac1{2m}\biggl(\frac{dS_3}{dz}\biggr)^2-mgz
+P_1
\]
Linke Seite h"angt nur von $y$ ab, rechte nur von $z$

\end{frame}

\begin{frame}
\frametitle{Hamilton-Jacobi-Theorie VII}
\[
\frac1{2m}\biggl(\frac{dS_2}{dy}\biggr)^2=\operatorname{const}=P_3
\quad \Rightarrow\quad
S_2(y)=\sqrt{2mP_3}\,y
\]
Verbleibende Gleichung f"ur $S_3$:
\[
P_2+P_3+
\frac1{2m}\biggl(\frac{dS_3}{dz}\biggr)^2+mgz=P_1
\]
oder
\begin{align*}
\frac1{2m}\biggl(\frac{dS_3}{dz}\biggr)^2+mgz&=P_1 -P_2-P_3
\\
\frac{dS_3}{dz}&=\sqrt{2m(P_1-P_2-P_3-mgz)}
\\
S_3(z)&=-\frac1{3m^2g}(2m(P_1-P_2-P_3-gz))^{\frac32}
\end{align*}

\end{frame}

\begin{frame}
\frametitle{Hamilton-Jacobi-Theorie VIII}

Zusammenstellung der Resultate bis jetzt:
\begin{align*}
S_1(x)&=\sqrt{2mP_2}\, x
\\
S_2(y)&=\sqrt{2mP_3}\, y
\\
S_3(z)&=-\frac1{3m^2g}(2m(P_1-P_2-P_3-gz))^{\frac32}
\\
S_4(t)&=-P_1t
\end{align*}
Zusammenhang mit den alten Impulsen:
\begin{align*}
\frac{\partial S}{\partial x}&=p_x=\sqrt{2mP_2}
\\
\frac{\partial S}{\partial y}&=p_y=\sqrt{2mP_3}
\\
\frac{\partial S}{\partial z}&=p_z=\sqrt{2m(P_1-P_2-P_3-mgz)}
\end{align*}

\end{frame}

\begin{frame}
\frametitle{Hamilton-Jacobi-Theorie IX}
Neue Koordinaten
\begin{align*}
Q_1=\frac{\partial S}{\partial P_1}&=
-t-\frac1{mg}\sqrt{2m(P_1-P_2-P_3-mgz)}
\\
Q_2=\frac{\partial S}{\partial P_2}&=
\sqrt{\frac{m}{2P_2}}x+\frac1{mg}\sqrt{2m(P_1-P_2-P_3-mgz)}
\\
Q_3=\frac{\partial S}{\partial P_3}&=
\sqrt{\frac{m}{2P_3}}y+\frac1{mg}\sqrt{2m(P_1-P_2-P_3-mgz)}
\end{align*}
\end{frame}

\begin{frame}
\frametitle{Hamilton-Jacobi-Theorie X}

Gleichungen nach $x$, $y$ und $z$ aufl"osen:
\begin{align}
z&=-\frac12g(t+Q_1)^2+\frac1{mg}(P_1-P_2-P_3)
\label{zfunktion}
\\
x&=\sqrt{\frac{2P_2}m}(Q_1+Q_2+t)
\quad\Rightarrow\quad
\dot x=\sqrt{\frac{2P_2}m}
\notag
\\
y&=\sqrt{\frac{2P_3}m}(Q_1+Q_3+t)
\quad\Rightarrow\quad
\dot y=\sqrt{\frac{2P_3}m}
\notag
\end{align}
Man stellt fest:
\begin{itemize}
\item
Gleichf"ormige Bewegung in $x$- und $y$-Richtung
\item
Quadratische Abh"angigkeit der H"ohe von der Zeit: Wurfparabel (Galilei)
\item Zusammenhang zwischen $P_2$, $P_3$ und den urspr"unglichen Koordinaten:
\[
P_2=\frac12mv_x^2,
\qquad
P_3=\frac12mv_y^2
\]
\end{itemize}
\end{frame}

\begin{frame}
\frametitle{Hamilton-Jacobi-Theorie XI}

Physikalische Bedeutung der neuen verallgemeinerten Koordinaten:
\begin{center}
\begin{tabular}{|cl|}
\hline
Koordinate&Bedeutung\\
\hline
$Q_1$&Scheitelzeit\\
$Q_2$&$x$-Achsen-"Uberquerungszeit\\
$Q_3$&$y$-Achsen-"Uberquerungszeit\\
$P_1$&Gesamtenergie\\
$P_2$&Beitrag der $x$-Bewegung zur Energie\\
$P_3$&Beitrag der $y$-Bewegung zur Energie\\
\hline
\end{tabular}
\end{center}
$Q_2$ und $Q_3$ sind bezogen auf die Scheitelzeit: wieviel vor oder nach 
dem Scheiteldurchgang werden $x$- bzw.~$y$-Achse "uberquert?

$P_1-P_2-P_3$ in (\ref{zfunktion}) ist der Energieinhalt der Vertikalbewegung.


\end{frame}


\begin{frame}
\frametitle{Hamilton-Jacobi-Theorie XII}

{\bf Wozu das alles?}
\begin{itemize}
\item $Q_i$, $P_i$ sind die physikalisch nat"urlichsten Bahn-Elemente
\item Bahnberechnung braucht keine numerische Integration $\Rightarrow$
keine Rundungsfehler $\Rightarrow$ Bahnberechnung "uber sehr lange Zeit
(Jahrmilliarden) m"oglich
\item St"orungen der Bahn "aussern sich in langsamen "Anderungen der
Elemente $Q_i$ und $P_i$  $\Rightarrow$ Stabilit"at des Sonnensystems
"uber Jahrmilliarden berechenbar
\item Anwendung: Satellitenbahnen
\begin{itemize}
\item Bahn wird durch 6 Elemente beschrieben
\item St"orung z.~B.~Mond oder Sonne "aussern sich in kleinen "Anderungen
von $Q_i$ und $P_i$.
\item Korrekturen sind kleine "Anderungen von $Q_i$ und $P_i$
\end{itemize}
\end{itemize}
\end{frame}

\end{document}
