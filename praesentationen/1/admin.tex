\begin{frame}
\frametitle{Administratives}
\begin{enumerate}
\item Unterrichtsunterlagen $\to$ Moodle
\begin{itemize}
\item Skript
\item Aufgabensammlung
\end{itemize}
\item "Ubungsbetrieb $\to$ 2h, Aufgaben im Moodle
\begin{itemize}
\item Musterlösung
\item Fragen klären, Zusammenfassung aufbauen
\end{itemize}
\item Prüfung:
\begin{itemize}
\item für jeden Teil 4 Aufgaben
\item Hilfsmittel $\to$ $2\times 10\times \text{A4}$, Taschenrechner
\end{itemize}
\item Prüfungsvorbereitung
\begin{itemize}
\item Aufgabensammlung
\item Moodle-Forum: Fragen und Antworten
\item Prüfungen früherer Jahre
\end{itemize}
\end{enumerate}
\end{frame}

\begin{frame}
\frametitle{Inhalt}
\begin{itemize}[<+->]
\item Teil 1: Grundlagen.
\begin{enumerate}[<+->]
\item Definitionen und Notation
\item Quasilineare Differentialgleichungen 1. Ordnung, Charakteristiken
\item Separationsmethode
\item Transformationsmethode
\item PDGL 2. Ordnung: elliptisch
\item PDGL 2. Ordnung: hyperbolisch, Charakteristiken
\end{enumerate}
\item Teil 2: Numerik
\end{itemize}
\end{frame}

\begin{frame}
\frametitle{Teil 1: Grundlagen}

\begin{itemize}[<+->]
\item
Wie muss man ein Problem partieller Differntialgleichungen stellen?
\item
Unter welchen Voraussetzungen können wir überhaupt mit einer Lösung 
rechnen?
\item
Unter welchen Voraussetzungen können wir damit rechnen, dass es genau
eine Lösung gibt?
\end{itemize}

\end{frame}

\begin{frame}
\frametitle{Zeitplan}
\begin{center}
\includegraphics{zeitplan-1.pdf}
\end{center}
\[
\text{Wirkungsgrad}
=
\frac{1}{\text{Unterrichtszeit}}
\int_0^{100}\text{Aufnahmefähigkeit}(t)\,dt
\]
\end{frame}

\begin{frame}
\frametitle{Zeitplan}
\begin{center}
\includegraphics{zeitplan-2.pdf}
\end{center}
\[
\text{Wirkungsgrad}= 50\%
\]
\end{frame}

\begin{frame}
\frametitle{Zeitplan}
\begin{center}
\includegraphics{zeitplan-3.pdf}
\end{center}
\[
\text{Wirkungsgrad}= \color{red}66.6\%
\]
\end{frame}

