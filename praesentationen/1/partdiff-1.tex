\documentclass{beamer}
\usepackage{array}
\usepackage{graphicx}
\usepackage{german}
%\usepackage{txfonts}

\mode<beamer>{%
\usetheme[hideothersubsections,hidetitle]{Hannover}
}
\title[]{Partielle Differentialgleichungen}
\date[18.~Februar 2015]{18.~Februar 2015}
\author{Prof.~Dr.~Andreas M"uller}
\begin{document}

\begin{frame}
\titlepage
\end{frame}

\begin{frame}
\frametitle{Administratives}
\begin{enumerate}
\item Unterrichtsunterlagen $\to$ Moodle
\item "Ubungsbetrieb $\to$ 2h
\item Pr"ufung
\item Hilfsmittel $\to$ $2\times 10\times \text{A4}$, Taschenrechner
\item Pr"ufungsvorbereitung
\end{enumerate}
\end{frame}

\begin{frame}
\frametitle{Einf"uhrungsbeispiel}

{\bf Aufgabe:} Stelle die Bewegungsgleichung
einer schwingenden Saite der L"ange $l$ und der Massedichte $\mu$ auf.

\medskip
\pause
{\bf L"osungsansatz:}
Auslenkung $\psi(x,t)$ h"angt von Ort $x\in[0,l]$ und Zeit $t\in[0,\infty)$
ab:
\[
\psi\colon [0,l]\times [0,\infty)\to \mathbb R
\]
\pause
\medskip
\includegraphics{../../skript/images/saite-2.pdf}
\pause
\medskip

{\bf Randbedingung:}
\[
\psi(0,0)=\psi(l,0)=0
\]
\end{frame}

\begin{frame}
\frametitle{Bewegungsgleichung}
\includegraphics{../../skript/images/saite-3.pdf}
\vskip 6truecm

\end{frame}

\begin{frame}
\frametitle{Wellengleichung}
\includegraphics{../../skript/images/saite-1.pdf}
\begin{align*}
\text{Kraft}
=ma
&=
F\frac{\partial\psi}{\partial x}(x+\Delta x)-F\frac{\partial\psi}{\partial x}(x)
\\
\\
\mu \Delta x \frac{\partial^2 \psi}{\partial t^2}
&=
F\frac{\partial\psi}{\partial x}(x+\Delta x)-F\frac{\partial\psi}{\partial x}(x)
\\
\\
\frac{\partial^2 \psi}{\partial t^2}
&=
\frac{F}{\mu}
\frac{\partial^2\psi}{\partial x^2},
\qquad
c=\sqrt{\frac{F}{m}}
\end{align*}
Wellengleichung mit Ausbreitungsgeschwindigkeit $c$
\end{frame}

\begin{frame}
\frametitle{H"oheren Dimensionen}
In drei Dimensionen:
\[
\frac{\partial^2\psi}{\partial t^2}=c^2
\biggl(
\frac{\partial^2\psi}{\partial x^2}
+
\frac{\partial^2\psi}{\partial y^2}
+
\frac{\partial^2\psi}{\partial z^2}
\biggr)
\]
\pause
Mit Laplace-Operator
\[
\Delta
=
\frac{\partial^2}{\partial x^2}
+
\frac{\partial^2}{\partial y^2}
+
\frac{\partial^2}{\partial z^2}
\]
\pause
einfacher:
\[
\frac{\partial^2\psi}{\partial t^2}
=
c^2\Delta \psi.
\]
\end{frame}


\begin{frame}
\frametitle{Elektrisches Potential}
Potential 
$\varphi$ 
im Vakuum zwischen den Leitern,
\pause
erf"ullt die partielle Differentialgleichung
\[
\Delta \varphi =0
\]
\pause
{\bf Randbedingung:} Potentiale der Leiter
\end{frame}

\begin{frame}
\frametitle{W"armeleitungsgleichung}
Temperatur $T$ h"angt von Ort und Zeit ab:
\[
T(x,t)
\]
\pause
und erf"ullt die W"armeleitungsgleichung
\[
\frac{\partial T}{\partial t}=\frac{\partial^2T}{\partial x^2}
\]

\end{frame}


\end{document}
